%------------------------
% Bibliotecas para matemáticas de latex
%------------------------
\usepackage{amsthm}
\usepackage{amsmath}
\usepackage{tikz}
\usepackage{tikz-cd}
\usetikzlibrary{shapes,fit}
\usepackage{bussproofs}
\EnableBpAbbreviations{}
\usepackage{mathtools}
\usepackage{scalerel}
\usepackage{stmaryrd}


%------------------------
% Estilos para los teoremas
%------------------------
\theoremstyle{plain}
\newtheorem{theorem}{Teorema}
\newtheorem{proposition}{Proposición}
\newtheorem{lemma}{Lema}
\newtheorem{corollary}{Corolario}

\theoremstyle{definition}
\newtheorem{definition}{Definición}

% Change the proof style so it's in English and add \qed at the end.
\renewenvironment{proof}{{\bfseries Demostración.}}{\qed}

\theoremstyle{remark}
\newtheorem{remark}{Observación}
\newtheorem{exampleth}{Ejemplo}

% New style for postulates so they are tabulated
\makeatletter
\newtheoremstyle{indented}
	{3pt}% space before
	{3pt}% space after
	{\addtolength{\@totalleftmargin}{3.5em}
		\addtolength{\linewidth}{-3.5em}
		\parshape 1 3.5em \linewidth}% body font
	{}% indent
	{\bfseries}% header font
	{.}% punctuation
	{.5em}% after theorem header
	{}% header specification (empty for default)
\makeatother

% Apply the new style
\theoremstyle{indented}
\newtheorem{postulate}{Postulate}
\newtheorem*{postulate 3'}{Postulate 3'}
\newtheorem*{postulate 2'}{Projective Measurement}

%------------------------
% Macros
% ------------------------

\newcommand*{\B}{\mathbb{B}}
\newcommand*{\C}{\mathds{C}}
\newcommand*{\R}{\mathbb{R}}
\newcommand*{\ra}{\rangle}
\newcommand*{\la}{\langle}

% Para poner sonrisa sobre puntos suspensivos. Uso: \overplace{n}{\dotsc}
\newcommand{\overplace}[2]{%
	\overset{\substack{#1\\\smile}}{#2}%
}

% Para añadir un salto de línea al nuevo párrafo. Use: \newparagraph
\newcommand{\newparagraph}[1]{\paragraph{#1}\mbox{}\\}