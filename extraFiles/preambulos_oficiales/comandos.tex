% aquí van los comandos personalizados
% Comando para incluir imágenes
\newcommand{\incluirimagen}[3][]{%
\begin{figure}[H]
    \centering
    \includegraphics[width=\linewidth,#1]{#2}
    \caption{#3}
    \label{fig:#2}
\end{figure}
}

% comando para ejercicios con fondo
\newtheoremstyle{ejerciciostyle}
  {10pt}   % Espacio arriba
  {10pt}   % Espacio abajo
  {\itshape} % Fuente del cuerpo
  {}       % Sangría
  {\bfseries} % Fuente del encabezado
  {}      % Puntuación tras encabezado
  { }      % Espacio tras encabezado
  {\thmname{#1} \thmnumber{#2}. \thmnote{#3}}


% % comando formal para enunciado de ejercicios
% \theoremstyle{ejerciciostyle}
% \newtheorem{ejercicio}{Ejercicio}[chapter]

\theoremstyle{ejerciciostyle}
\newtheorem{ejercicio}{Ejercicio}[section]

\renewcommand{\theejercicio}{\thechapter.\arabic{section}.\arabic{ejercicio}}


% comando formal para soluciones
\theoremstyle{remark}
\newtheorem*{solucion}{Solución}

% Comando para dos imágenes en paralelo
\newcommand{\dosimagenes}[6]{%
    \begin{figure}[h!]
        \centering
        \begin{minipage}{0.48\linewidth}
            \centering
            \includegraphics[width=\linewidth]{#1}
            \caption{#2}
            \label{#5}
        \end{minipage}\hfill
        \begin{minipage}{0.48\linewidth}
            \centering
            \includegraphics[width=\linewidth]{#3}
            \caption{#4}
            \label{#6}
        \end{minipage}
    \end{figure}
}

% \dosimagenes{media/fondo.jpg}{Descripción 1}{media/fondo.jpg}{Descripción 2}{fig:descripcion1}{fig:descripcion2}

% \ref{fig:descripcion1} es la mejor
% \ref{fig:descripcion2} es la mejor

\newcommand{\portadaimg}{\VAR{portadaimg}}

% Comando para crear una nota estilo información
\newcommand{\nota}[2]{%
\begin{tcolorbox}[colframe=blue!75!black, colback=blue!5!white, title=\textbf{#1}]
    #2
\end{tcolorbox}
}
