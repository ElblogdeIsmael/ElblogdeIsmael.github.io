\documentclass[a4paper,12pt]{article}

% Paquetes necesarios
\usepackage[utf8]{inputenc}   % Codificación de caracteres
\usepackage[spanish]{babel}   % Idioma español
\usepackage[T1]{fontenc}      % Codificación de fuentes
\usepackage{amsmath, amssymb} % Símbolos matemáticos
\usepackage{graphicx}         % Inclusión de gráficos
\usepackage{cite}             % Gestión de citas
\usepackage{hyperref}         % Enlaces y referencias
\usepackage{geometry}         % Configuración de márgenes
\usepackage{fancyhdr}         % Encabezados y pies de página
\usepackage{titlesec}         % Formato de títulos
\usepackage{booktabs}         % Tablas profesionales
\usepackage{caption}          % Personalización de leyendas
\usepackage{enumitem}         % Personalización de listas
\usepackage{float}
\usepackage{tcolorbox}
\usepackage[table]{xcolor} % Paquete para colores en tablas
\usepackage{colortbl}       % Complemento para colorear celdas específicas
\usepackage{multirow}       % Combinar celdas en tablas
\usepackage{makecell}       % Combinar celdas en tablas
\usepackage{enumitem}
\usepackage{amsmath}
\usepackage{eurosym}
\usepackage{tikz}
\usepackage{listings}
\usepackage{color}
\usepackage{float}
\usepackage{pdfpages}

\lstset{
    language=Python,
    basicstyle=\ttfamily\itshape\footnotesize,
    keywordstyle=\color{blue}\bfseries,
    stringstyle=\color{red},
    commentstyle=\color{green}\itshape,
    numbers=left,
    numberstyle=\tiny\color{gray},
    stepnumber=1,
    numbersep=10pt,
    backgroundcolor=\color{white},
    showspaces=false,
    showstringspaces=false,
    showtabs=false,
    frame=single,
    rulecolor=\color{black},
    tabsize=4,
    captionpos=b,
    breaklines=true,
    breakatwhitespace=false,
    title=\lstname,
    escapeinside={\%*}{*)},
    morekeywords={*,...},
    literate=%
        {á}{{\'a}}1 {é}{{\'e}}1 {í}{{\'i}}1 {ó}{{\'o}}1 {ú}{{\'u}}1
        {Á}{{\'A}}1 {É}{{\'E}}1 {Í}{{\'I}}1 {Ó}{{\'O}}1 {Ú}{{\'U}}1
        {ñ}{{\~n}}1 {Ñ}{{\~N}}1
}

% Configuración de márgenes
\geometry{left=3cm, right=3cm, top=2.5cm, bottom=2.5cm}

% Configuración de encabezados y pies de página
% \setlength{\headheight}{14.49998pt}
\pagestyle{fancy}
\fancyhf{}
\fancyhead[L]{Universidad de Granada}
\fancyhead[L]{\nouppercase{Notas de la Práctica 1}}

% \fancyhead[C]{Escuela Técnica Superior de Ingenierías Informática}
\fancyhead[R]{Inteligencia Artificial}
\fancyfoot[L]{\rule[0pt]{\textwidth}{0.2pt}\\Ismael Sallami Moreno}
\fancyfoot[C]{\rule[0pt]{\textwidth}{0.2pt}\\\thepage}
\fancyfoot[R]{\rule[0pt]{\textwidth}{0.2pt}\\\today}
\renewcommand{\sectionmark}[1]{\markboth{#1}{}} % Configura \leftmark para que solo muestre la sección


% Formato de títulos
\titleformat{\section}{\large\bfseries}{\thesection.}{0.5em}{}
\titleformat{\subsection}{\normalsize\bfseries}{\thesubsection.}{0.5em}{}

% Datos del documento
\title{\textbf{Inteligencia Artificial: Práctica 1}}
\author{
    Ismael Sallami Moreno \\
    \texttt{ism350zsallami@correo.ugr.es}
}
\date{
    \vspace{1cm}
    \begin{tabular}{rl}
        \textbf{Asignatura:} & Inteligencia Artificial \\
        \textbf{Tema:} & Notas de Prácticas \\
        \textbf{Fecha:} & \today
    \end{tabular}
}

\begin{document}
\begin{titlepage}
    \centering
    \vspace*{5cm}
    {\Huge \textbf{Inteligencia Artificial: Práctica 1} \par}
    \vspace{2cm}
    {\Large Ismael Sallami Moreno \par}
    \vspace{0.5cm}
    {\large \texttt{ism350zsallami@correo.ugr.es} \par}
    \vfill
    \begin{tabular}{rl}
        \textbf{Asignatura:} & Inteligencia Artificial \\
        \textbf{Tema:} & Notas de Prácticas \\
        \textbf{Fecha:} & \today
    \end{tabular}
    \vspace{2cm}
    \vfill
    {\large Universidad de Granada \par}
\end{titlepage}


% Tabla de contenidos
% \tableofcontents
\newpage

Como enlace de interés adjunto el enlace de la conversación con ChatGPT: \url{https://chatgpt.com/share/67d47275-91dc-8012-9740-57ce1ed7abcb}

Además voy a añadir ambos códigos generados en la práctica 1 con Python con la IA, así como el archivo .json que he usado para entrenar el modelo.



\section{Archivo .json}

\lstinputlisting[language=Python, caption={Archivo .json para entrenar el modelo}, label={lst:json}]{pr1.json}

\section{Código 1}

\lstinputlisting[language=Python, caption={Código generado en la práctica 1}, label={lst:codigo1}]{solve_8puzzle.py}

\section{Código 2}

\lstinputlisting[language=Python, caption={Código generado en la práctica 1}, label={lst:codigo1}]{solve_8puzzle_v2.py}

Ambos códigos funcionan correctamente, ya que han sido comprobados en base a como se exponía en el guión. \textit{Para cualquier aclaración o duda, no dudéis en contactar conmigo.}

\newpage
% Referencias
\begin{thebibliography}{99}
\bibitem{Referencia1}
Ismael Sallami Moreno, \textbf{Estudiante del Doble Grado en Ingeniería Informática + ADE}, Universidad de Granada, 2025.
% \bibitem{Referencia2}
% Autor(es), \emph{Título del libro}, Editorial, año.

% \bibitem{Referencia3}
% Autor(es), \emph{Título del documento}, Nombre de la Conferencia, páginas, año.
\end{thebibliography}

\end{document}
