\chapter*{Reflexión final y aprendizajes}
\addcontentsline{toc}{chapter}{Reflexión final y aprendizajes}

Durante el desarrollo de esta práctica he aprendido de manera muy enriquecedora cómo funcionan los algoritmos de búsqueda en espacios de estados, especialmente la poda Alfa-Beta y las variantes que permiten optimizar el rendimiento de un agente en un juego de tablero como Parchís. Me ha sorprendido y gustado especialmente ver cómo se puede \textit{ordenar los movimientos para optimizar la poda} y cómo un simple cambio como la incorporación de un \textit{epsilon} en la poda probabilística puede cambiar drásticamente la eficiencia del algoritmo. He intentado hacer la memoria lo más corta posible, pero a la vez lo más completa. Además, quería reflejar el esfuerzo desempeñado en cuanto a la búsqueda de información e interés por los algoritmos de búsqueda y la asignatura en general.

He podido comprender mejor la importancia de las \textbf{heurísticas}\footnote{También me gustaría destacar la dificultad de desarrollar una verdaderamente eficiente, ya que para ello hay que conocer en profundidad el software donde se esta trabajando.}: cómo se construyen, qué impacto tienen en la calidad de las decisiones y cómo se adaptan a las necesidades de cada situación. En particular, la \textbf{heurística mejorada} que diseñé ha demostrado ser especialmente eficaz, utilizando pesos cuidadosamente ajustados para valorar la distancia a meta, las posiciones seguras, la formación de barreras y el progreso relativo frente al oponente.

La mejora que más me ha gustado proponer es la combinación de la poda Alfa-Beta con la \textbf{búsqueda de quietud}, junto con la poda probabilística. Esto ha permitido a la IA \textit{gestionar mejor las situaciones de riesgo táctico} como capturas y formaciones de barreras, haciendo que el comportamiento sea más realista y eficiente frente a oponentes fuertes como los ninjas.

No obstante, también quiero dejar constancia de que, debido a la carga académica y los exámenes finales, no he podido dedicarle todo el tiempo que me habría gustado a esta práctica. Hubiera querido profundizar todavía más en las comparaciones experimentales y en la optimización de las heurísticas. Estoy convencido de que con algo más de tiempo podría haber alcanzado resultados aún más sólidos y satisfactorios.

Finalmente, me gustaría expresar mi sincero agradecimiento a los profesores y al equipo docente de la asignatura, en particular al profesor \textbf{Juan Luis Suárez Díaz} por su constante apoyo, las dudas resueltas y la claridad de sus explicaciones. Gracias a su dedicación y orientación, he podido comprender mejor los algoritmos de búsqueda y disfrutar aún más de este fascinante campo de la inteligencia artificial.

Esta práctica me ha servido para reforzar mis conocimientos en inteligencia artificial, aprender a fondo sobre algoritmos de poda y valorar cómo la creatividad y la experimentación permiten encontrar soluciones más eficientes y adaptadas a cada juego. ¡Una experiencia muy positiva y gratificante!
