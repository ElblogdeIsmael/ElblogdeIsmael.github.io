En esta práctica se ha trabajado en el desarrollo de un agente deliberativo para el juego \textit{ParCheckers}, una adaptación determinista del Parchís en la que cada jugador controla dos colores y puede elegir el dado a usar en cada turno. El objetivo principal ha sido implementar y mejorar el algoritmo de poda Alfa-Beta y distintas heurísticas, buscando un equilibrio entre la eficiencia computacional (límite de nodos generados y evaluados, así como un límite de tiempo por movimiento) y la calidad de las decisiones tomadas por el agente.

Se han desarrollado varias versiones del algoritmo de poda Alfa-Beta, incluyendo variantes con profundidad dinámica, ordenación de movimientos, poda probabilística y búsqueda de quietud. Cada versión ha sido diseñada con el propósito de mejorar la exploración del árbol de juego y aumentar las probabilidades de victoria frente a los adversarios controlados por inteligencia artificial (los ``ninjas''). \textit{Cabe destacar que aunque el código parece estar correcto, el hecho de que se venza o no a los ninjas depende de la implementación de la heurística.}

Asimismo, se han creado distintas heurísticas que permiten valorar los estados del juego en función de criterios como la distancia a la meta, la seguridad de las fichas y las barreras, con el fin de dotar al agente de una estrategia más competitiva. Todas estas implementaciones han sido evaluadas y comparadas en un conjunto de partidas contra los jugadores automáticos del simulador, así como enfrentando heurísticas entre sí como se aconseja en el guión, con el objetivo de analizar su rendimiento y justificar las decisiones tomadas a lo largo del proceso.

\textit{Cabe destacar que me hubiese gustado poder dedicar más tiempo a esta práctica, ya que he aprendido mucho, pero al estar en época de exámenes y demás no he podido dedicarle el tiempo necesario.} Por otro lado, he priorizado el aprender los algoritmos de manera correcta\footnote{Se puede apreciar en el código numerosas implementaciones.}, para ello me he basado en libros y otros recursos.

Se añade la bibliografía de los libros y/o artículos que se han utilizado para la búsqueda de algoritmos, además de diversas fuentes de Internet que se han consultado para la implementación de las distintas heurísticas y el algoritmo de poda Alfa-Beta y demás herramientas (algún simulador de algoritmos de búsqueda, video de YouTube, etc.)\footnote{Estos últimos no se incluyen porque creo que son irrelevantes, solo se incluyen los libros.}.