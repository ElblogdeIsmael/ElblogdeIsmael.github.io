% ========================
% estilo.latex mínimo funcional
% ========================

\documentclass[12pt]{report} % report para capítulos

% ========================
% Paquetes y comandos extra
% ========================
% ===========================
% Paquetes básicos de idioma y codificación
% ===========================
\usepackage[utf8]{inputenc}   % Codificación UTF-8
\usepackage[T1]{fontenc}      % Acentos y caracteres correctos
\usepackage[spanish]{babel}   % Traducción al español (capítulos, índices, etc.)
\usepackage{csquotes}         % Citas tipográficas correctas

% ===========================
% Tipografía
% ===========================
\usepackage{lmodern}          % Fuente Latin Modern
\usepackage{microtype}        % Mejoras tipográficas (espaciado, justificación)

% ===========================
% Márgenes y geometría
% ===========================
\usepackage{geometry}         % Control de márgenes
\geometry{a4paper, top=3cm, bottom=3cm, left=3cm, right=3cm}

% ===========================
% Matemáticas
% ===========================
\usepackage{amsmath, amssymb, amsthm} % Paquetes AMS
\usepackage{mathtools}        % Extiende amsmath
\usepackage{physics}          % Notación física y matemática (derivadas, bra-ket, etc.)
\usepackage{siunitx}          % Unidades SI (e.g. \SI{3}{m/s})
% \sisetup{locale=ES}           % Configuración para español (coma decimal, etc.)
\AtBeginDocument{\RenewCommandCopy\qty\SI} % Resolve siunitx and physics conflict


% ===========================
% Gráficos, tablas y colores
% ===========================
\usepackage{graphicx}         % Insertar imágenes
\usepackage{xcolor}           % Colores personalizados
\usepackage{tikz}             % Dibujos vectoriales
\usetikzlibrary{calc,positioning,shapes,arrows} % Librerías útiles de TikZ
\usepackage{pgfplots}         % Gráficas de funciones
\pgfplotsset{compat=1.18}
\usepackage{float}            % Control de posición de figuras/tablas
\usepackage{booktabs}         % Tablas profesionales
\usepackage{multirow}         % Celdas que ocupan varias filas
\usepackage{array}            % Más control en tablas
\usepackage{colortbl}         % Tablas con colores
\usepackage{inconsolata}


% ===========================
% Listas y enumeraciones
% ===========================
\usepackage{enumitem}         % Control de listas enumeradas y viñetas

% ===========================
% Encabezados, pies y diseño
% ===========================
\usepackage{fancyhdr}         % Encabezados y pies de página
\usepackage{titlesec}         % Personalizar títulos de capítulos/secciones
\usepackage{setspace}         % Espaciado entre líneas
\usepackage{parskip}          % Control del espacio entre párrafos

% ===========================
% Referencias, hipervínculos y citas
% ===========================
\usepackage{hyperref}         % Hipervínculos en PDF
\hypersetup{
    colorlinks = true,
    linkcolor  = red!70,
    citecolor  = red!70,
    urlcolor   = red!70,
    pdfpagelayout = SinglePage, % Asegura que el contenido se ajuste a una sola página
    pdfstartview = Fit          % Ajusta el contenido al tamaño de la página
}
\usepackage{cleveref}         % Referencias inteligentes (\cref)

% ===========================
% Código fuente
% ===========================
\usepackage{listings}         % Mostrar código con estilo
\usepackage{minted}           % (mejor opción, requiere Python y pygments)

% ===========================
% Bibliografía
% ===========================
\usepackage[backend=biber,style=apa]{biblatex} % Ejemplo: estilo APA
\addbibresource{referencias.bib}              % Archivo .bib

% ===========================
% Otros útiles
% ===========================
\usepackage{pdfpages}         % Insertar PDFs externos
\usepackage{blindtext}        % Texto de prueba
\usepackage{caption}          % Personalizar pies de figura/tabla
\usepackage{subcaption}       % Subfiguras
\usepackage{tocloft} 
\usepackage{amsthm}
\usepackage{subcaption}
\usepackage{truncate} % permite truncar texto si no cabe
\usepackage{libertinus}  % reemplaza lmodern
\usepackage{booktabs}  % para \toprule, \midrule, \bottomrule
\usepackage{array}     % para definir columnas personalizadas
\usepackage{colortbl}  % colores en tablas
\usepackage{etoolbox}
\AtBeginEnvironment{tabular}{\rowcolors{2}{gray!10}{white}\renewcommand{\arraystretch}{1.2}}

% ===========================
% Opciones de fuentes sugeridas
% ===========================
% TeX Gyre Pagella (estilo Palatino)
% \usepackage{fontspec}
% \usepackage{unicode-math}
% \setmainfont{TeX Gyre Pagella}
% \setmathfont{TeX Gyre Pagella Math}

% TeX Gyre Termes (estilo Times)
% \setmainfont{TeX Gyre Termes}
% \setmathfont{TeX Gyre Termes Math}

% Libertinus (elegante y completa)
% \setmainfont{Libertinus Serif}
% \setmathfont{Libertinus Math}

% TeX Gyre Bonum (estilo Garamond)
% \setmainfont{TeX Gyre Bonum}
% \setmathfont{TeX Gyre Bonum Math}

% Latin Modern (moderno de Computer Modern)
% \setmainfont{Latin Modern Roman}
% \setmathfont{Latin Modern Math}


% \usepackage{helvet}
% \usepackage{libertine}
% \usepackage[sfdefault]{FiraSans}

\usepackage{tcolorbox} % para cajas de colores




  % si tienes paquetes personalizados
% aquí van los comandos personalizados
% Comando para incluir imágenes
\newcommand{\incluirimagen}[3][]{%
\begin{figure}[H]
    \centering
    \includegraphics[width=\linewidth,#1]{#2}
    \caption{#3}
    \label{fig:#2}
\end{figure}
}

% comando para ejercicios con fondo
\newtheoremstyle{ejerciciostyle}
  {10pt}   % Espacio arriba
  {10pt}   % Espacio abajo
  %{\itshape} % Fuente del cuerpo
  {}
  {}       % Sangría
  {\bfseries} % Fuente del encabezado
  {}      % Puntuación tras encabezado
  { }      % Espacio tras encabezado
  {\thmname{#1} \thmnumber{#2}. \thmnote{#3}}


% % comando formal para enunciado de ejercicios
% \theoremstyle{ejerciciostyle}
% \newtheorem{ejercicio}{Ejercicio}[chapter]

\theoremstyle{ejerciciostyle}
\newtheorem{ejercicio}{Ejercicio}[section]

\renewcommand{\theejercicio}{\thechapter.\arabic{section}.\arabic{ejercicio}}


% comando formal para soluciones
\theoremstyle{remark}
\newtheorem{solucion}{Solución}[ejercicio]

\renewcommand{\thesolucion}{\thechapter.\arabic{section}.\arabic{ejercicio}}

% Comando para dos imágenes en paralelo
\newcommand{\dosimagenes}[6]{%
    \begin{figure}[h!]
        \centering
        \begin{minipage}{0.48\linewidth}
            \centering
            \includegraphics[width=\linewidth]{#1}
            \caption{#2}
            \label{#5}
        \end{minipage}\hfill
        \begin{minipage}{0.48\linewidth}
            \centering
            \includegraphics[width=\linewidth]{#3}
            \caption{#4}
            \label{#6}
        \end{minipage}
    \end{figure}
}

% \dosimagenes{media/fondo.jpg}{Descripción 1}{media/fondo.jpg}{Descripción 2}{fig:descripcion1}{fig:descripcion2}

% \ref{fig:descripcion1} es la mejor
% \ref{fig:descripcion2} es la mejor

\newcommand{\portadaimg}{\VAR{portadaimg}}

% Comando para crear una nota estilo información
% \newcommand{\nota}[2]{%
% \begin{tcolorbox}[colframe=blue!75!black, colback=blue!5!white, title=\textbf{#1}]
%     #2
% \end{tcolorbox}
% }
\newtheorem{nota}{Nota}[chapter]


% Comando para poner dos códigos en paralelo
\newcommand{\doscodigos}[4]{%
  \noindent
  \begin{minipage}{0.48\linewidth}
    \lstset{language=#1}
    \lstinputlisting{#2}
  \end{minipage}\hfill
  \begin{minipage}{0.48\linewidth}
    \lstset{language=#3}
    \lstinputlisting{#4}
  \end{minipage}
}

% Comando para poner un solo código
\newcommand{\uncodigo}[2]{%
  \begin{lstlisting}[language=#1]
#2
  \end{lstlisting}
}


% % Listas de archivos (sin guiones en los nombres de macros)
% \newcommand{\listagdfilesSesion2Mallas2D}{cargatexturas.gd, envioinmediato.gd, malla2dcontexturas.gd, mallaconcoloresdevertices.gd, mallanoindentada.gd}
% \newcommand{\listagdfilesSesion2Mallas3D}{mallaindexada3d.gd, materialconcolordeplano.gd, materialconcoloresdevertices.gd, tablas.gd}

% % Macro que recorre una lista de archivos en un subdirectorio
% \newcommand{\includegdfiles}[2]{%
%   % #1 = subdirectorio
%   % #2 = nombre de la lista de archivos
%   \foreach \filename in #2 {%
%     \includecode[gdstyle]{code/#1/\filename}{\filename}
%   }%
% }



% Comando para ejercicio resuelto
\newtheoremstyle{ejercicioresueltostyle}
    {10pt}   % Espacio arriba
    {10pt}   % Espacio abajo
    {\itshape} % Fuente del cuerpo
    {}       % Sangría
    {\bfseries} % Fuente del encabezado
    {}      % Puntuación tras encabezado
    { }      % Espacio tras encabezado
    {\thmname{#1} \thmnumber{#2}. \thmnote{#3}}

\theoremstyle{ejercicioresueltostyle}
\newtheorem{ejercicioresuelto}{Ejercicio Resuelto}[section]

\renewcommand{\theejercicioresuelto}{\thechapter.\arabic{section}.\arabic{ejercicioresuelto}}


%======================================================================== 
% PRACTICAS
%========================================================================

% Comando para definir un tema
\newcommand{\tema}[1]{%
  \section{#1}
  \addcontentsline{toc}{section}{#1}
}
\usepackage{tikz}
\usepackage{graphicx} % necesario para \resizebox
\usepackage{etoolbox}

% ======== NODOS ========
\newcommand{\nodo}[4][]{\node[state, #1] (#2) at (#3) {$#4$};}
% Uso: \nodo[initial,accepting]{q0}{0,0}{q_0}

% ======== FLECHAS ========
\newcommand{\flecha}[4][]{\draw[->, #1] (#2) -- (#3) node[midway, above] {#4};}
% Uso: \flecha{q0}{q1}{0} o \flecha[bend left]{q1}{q2}{1}

\newcommand{\flechaabajo}[4][]{\draw[->, #1] (#2) -- (#3) node[midway, below, yshift=-6pt] {#4};}
% Igual que \flecha pero con etiqueta abajo
\newcommand{\flechaarriba}[4][]{\draw[->, #1] (#2) -- (#3) node[midway, above, yshift=6pt] {#4};}
% Igual que \flecha pero con etiqueta arriba
\newcommand{\flechaderecha}[4][]{\draw[->, #1] (#2) -- (#3) node[midway, right] {#4};}
% Igual que \flecha pero con etiqueta a la derecha
\newcommand{\flechaiquierda}[4][]{\draw[->, #1] (#2) -- (#3) node[midway, left] {#4};}
% Igual que \flecha pero con etiqueta a la izquierda

\newcommand{\curva}[5][]{\draw[->, bend #1] (#2) to node[midway, #5] {#4} (#3);}
% Uso: \curva[left]{q1}{q2}{1}{below}


\newcommand{\loopa}[3]{\draw[->] (#1) edge[loop above] node {#2} (#1);}
\newcommand{\loopb}[3]{\draw[->] (#1) edge[loop below] node {#2} (#1);}
\newcommand{\loopr}[3]{\draw[->] (#1) edge[loop right] node {#2} (#1);}
\newcommand{\loopl}[3]{\draw[->] (#1) edge[loop left] node {#2} (#1);}
% Uso: \loopa{q1}{0}

% ======== ESTILOS ESPECIALES ========
\tikzset{
    error/.style={state, fill=red!20, draw=red!80!black},
    final/.style={state, accepting, fill=green!15!white, draw=green!60!black}
}
% Uso: \nodo[error]{qe}{5,0}{q_e}  o \nodo[final]{qf}{7,0}{q_f}


\newcommand{\pa}{1}      % ejemplo de valor
\newcommand{\pUno}{2}
\newcommand{\pDos}{3}
  % comandos LaTeX propios
% ===========================
% Diseño general
% ===========================
\setstretch{1.15} % interlineado
\setlength{\parskip}{0.5em} % espacio entre párrafos
\setlength{\parindent}{0pt} % sin sangría

% ===========================
% Estilo de capítulos y secciones (titlesec)
% ===========================
\titleformat{\chapter}[display]
  {\bfseries\Huge}
  {\filleft\Large\scshape Capítulo \thechapter}
  {1ex}
  {\titlerule[1pt]\vspace{1ex}\filright}
  [\vspace{1ex}\titlerule]

\titlespacing*{\chapter}{0pt}{0pt}{2em}

\titleformat{\section}
  {\Large\bfseries}
  {\thesection}{1em}{}

\titleformat{\subsection}
  {\large\bfseries}
  {\thesubsection}{1em}{}

\titleformat{\subsubsection}
  {\normalsize\bfseries\itshape}
  {\thesubsubsection}{1em}{}

% ===========================
% Encabezados y pies de página (fancyhdr)
% ===========================
\pagestyle{fancy}
\fancyhf{} % limpia
\fancyhead[L]{\small\scshape\nouppercase{\leftmark}} % sección/capítulo en mayúsculas pequeñas
\fancyhead[R]{\small\thepage}                        % número de página
%\fancyfoot[C]{\scriptsize\itshape Apuntes de la carrera} % texto fijo abajo en cursiva
% Encabezados y pies de página personalizados
% \fancyfoot[L]{\scriptsize\itshape Nombre de la asignatura} % pie de página izquierdo en cursiva
\fancyfoot[R]{\normalsize Ismael Sallami Moreno}        % pie de página derecho con el nombre del autor

% Línea bajo el encabezado
\renewcommand{\headrulewidth}{0.5pt} % línea más gruesa en el encabezado
% Línea en el pie
\renewcommand{\footrulewidth}{0.4pt} % línea fina en el pie
\renewcommand{\sectionmark}[1]{%
  \markboth{\thesection\quad #1}{}%
}

% ===========================
% Numeración de elementos
% ===========================
\numberwithin{equation}{chapter} % ecuaciones numeradas por capítulo
\numberwithin{figure}{chapter}   % figuras numeradas por capítulo
\numberwithin{table}{chapter}    % tablas numeradas por capítulo

% ===========================
% Listas y enumeraciones
% ===========================
\setlist[itemize]{label=--, left=1.5em}
\setlist[enumerate]{label=\arabic*), left=1.5em}

% ===========================
% Estilo de citas y bibliografía
% ===========================
\DefineBibliographyStrings{spanish}{%
  references = {Bibliografía},
}

% ===========================
% Entornos personalizados
% ===========================
\newtheoremstyle{cajita} % nombre del estilo
  {1em}   % espacio arriba
  {1em}   % espacio abajo
  {}      % fuente del cuerpo
  {}      % indentación
  {\bfseries} % fuente del título
  {.}     % puntuación tras título
  {0.5em} % espacio tras título
  {\thmname{#1}\thmnumber{ #2} \thmnote{(#3)}} % formato


\theoremstyle{cajita}
\newtheorem{teorema}{Teorema}[chapter]
\newtheorem{definicion}{Definición}[chapter]
\newtheorem{ejemplo}{Ejemplo}[chapter]
\newtheorem{proposicion}{Proposición}[chapter]
\newtheorem{demostracion}{Demostración}[chapter]
\newtheorem{corolario}{Corolario}[chapter]
\newtheorem{propuesta}{Propuesta}[chapter]


\newtheoremstyle{anotacionstyle} % nombre del estilo
  {1em}   % espacio arriba
  {1em}   % espacio abajo
  {}      % fuente del cuerpo (sin cursiva)
  {}      % indentación
  {\itshape} % fuente del título (Nota en cursiva)
  {.}     % puntuación tras título
  {0.5em} % espacio tras título
  {\thmname{\itshape#1}\thmnumber{ #2} \thmnote{(#3)}} % formato (solo Nota en cursiva)

\theoremstyle{anotacionstyle}
\newtheorem{anotacion}{Nota}[chapter]

% ===========================
% Configuración de lstlisting
% ===========================

% ===============================================
% ESTILO 1: MODERNO Y MINIMALISTA
% ===============================================

% Definir colores personalizados
\definecolor{codegreen}{rgb}{0,0.6,0}
\definecolor{codegray}{rgb}{0.5,0.5,0.5}
\definecolor{codepurple}{rgb}{0.58,0,0.82}
\definecolor{backcolour}{rgb}{0.95,0.95,0.92}
\definecolor{framecolor}{rgb}{0.8,0.8,0.8}

\lstset{
  backgroundcolor=\color{backcolour},   
  commentstyle=\color{codegreen},
  keywordstyle=\color{magenta},
  numberstyle=\tiny\color{codegray},
  stringstyle=\color{codepurple},
  basicstyle=\ttfamily\footnotesize,
  breakatwhitespace=false,         
  breaklines=true,                 
  captionpos=b,                    
  keepspaces=true,                 
  numbers=left,                    
  numbersep=5pt,                  
  showspaces=false,                
  showstringspaces=false,
  showtabs=false,                  
  tabsize=2,
  frame=shadowbox,
  frameround=tttt,
  rulecolor=\color{framecolor},
  rulesepcolor=\color{framecolor},
  xleftmargin=20pt,
  xrightmargin=20pt,
  aboveskip=20pt,
  belowskip=20pt,
  inputencoding=utf8,
  extendedchars=true,
  literate=
    {←}{{$\leftarrow$}}1
    {→}{{$\rightarrow$}}1
    {↑}{{$\uparrow$}}1
    {↓}{{$\downarrow$}}1
    {↔}{{$\leftrightarrow$}}1
    {⇒}{{$\Rightarrow$}}1
    {⇐}{{$\Leftarrow$}}1
    {⇔}{{$\Leftrightarrow$}}1
    {α}{{$\alpha$}}1
    {β}{{$\beta$}}1
    {γ}{{$\gamma$}}1
    {δ}{{$\delta$}}1
    {ε}{{$\epsilon$}}1
    {θ}{{$\theta$}}1
    {λ}{{$\lambda$}}1
    {μ}{{$\mu$}}1
    {π}{{$\pi$}}1
    {σ}{{$\sigma$}}1
    {φ}{{$\phi$}}1
    {ψ}{{$\psi$}}1
    {ω}{{$\omega$}}1
    {Δ}{{$\Delta$}}1
    {Θ}{{$\Theta$}}1
    {Λ}{{$\Lambda$}}1
    {Π}{{$\Pi$}}1
    {Σ}{{$\Sigma$}}1
    {Φ}{{$\Phi$}}1
    {Ψ}{{$\Psi$}}1
    {Ω}{{$\Omega$}}1
    {á}{{\'a}}1
    {é}{{\'e}}1
    {í}{{\'i}}1
    {ó}{{\'o}}1
    {ú}{{\'u}}1
    {Á}{{\'A}}1
    {É}{{\'E}}1
    {Í}{{\'I}}1
    {Ó}{{\'O}}1
    {Ú}{{\'U}}1
    {ä}{{\"a}}1
    {ë}{{\"e}}1
    {ï}{{\"i}}1
    {ö}{{\"o}}1
    {ü}{{\"u}}1
    {Ä}{{\"A}}1
    {Ë}{{\"E}}1
    {Ï}{{\"I}}1
    {Ö}{{\"O}}1
    {Ü}{{\"U}}1
    {ñ}{{\~n}}1
    {Ñ}{{\~N}}1
    {ç}{{\c{c}}}1
    {Ç}{{\c{C}}}1
    {¿}{{?`}}1
    {¡}{{!`}}1
    {à}{{\`a}}1
    {è}{{\`e}}1
    {ì}{{\`i}}1
    {ò}{{\`o}}1
    {ù}{{\`u}}1
    {À}{{\`A}}1
    {È}{{\`E}}1
    {Ì}{{\`I}}1
    {Ò}{{\`O}}1
    {Ù}{{\`U}}1
    {-}{{-}}1
    {=}{{=\allowbreak}}1  % <--- ESTA LÍNEA ES EL TRUCO PARA CORTAR LOS '===='
    % {#}{{\#}}1 
}


% ===============================================
% ESTILO 2: ELEGANTE CON BORDES REDONDEADOS
% ===============================================

% Colores para estilo elegante
\definecolor{lightblue}{rgb}{0.93,0.95,1}
\definecolor{darkblue}{rgb}{0.1,0.2,0.5}
\definecolor{mediumblue}{rgb}{0.2,0.4,0.8}
\definecolor{darkgreen}{rgb}{0,0.5,0}
\definecolor{darkred}{rgb}{0.6,0,0}

\lstdefinestyle{elegant}{
    backgroundcolor=\color{lightblue},
    commentstyle=\color{darkgreen}\itshape,
    keywordstyle=\color{darkblue}\bfseries,
    numberstyle=\tiny\color{gray},
    stringstyle=\color{darkred},
    basicstyle=\ttfamily\small,
    breakatwhitespace=false,
    breaklines=true,
    captionpos=t,
    keepspaces=true,
    numbers=left,
    numbersep=8pt,
    showspaces=false,
    showstringspaces=false,
    showtabs=false,
    tabsize=4,
    frame=single,
    frameround=tttt,
    framesep=10pt,
    xleftmargin=15pt,
    xrightmargin=15pt,
    aboveskip=15pt,
    belowskip=15pt,
    columns=flexible
}

% ===============================================
% ESTILO 3: PROFESIONAL CORPORATIVO
% ===============================================

% Colores corporativos
\definecolor{corporatebg}{rgb}{0.98,0.98,0.98}
\definecolor{corporateblue}{rgb}{0.07,0.29,0.49}
\definecolor{corporategray}{rgb}{0.4,0.4,0.4}
\definecolor{corporategreen}{rgb}{0.13,0.55,0.13}
\definecolor{corporatered}{rgb}{0.8,0.2,0.2}

\lstdefinestyle{corporate}{
    backgroundcolor=\color{corporatebg},
    commentstyle=\color{corporategreen}\slshape,
    keywordstyle=\color{corporateblue}\bfseries,
    numberstyle=\scriptsize\color{corporategray},
    stringstyle=\color{corporatered},
    basicstyle=\ttfamily\footnotesize,
    breakatwhitespace=false,
    breaklines=true,
    captionpos=b,
    keepspaces=true,
    numbers=left,
    numbersep=12pt,
    showspaces=false,
    showstringspaces=false,
    showtabs=false,
    tabsize=3,
    frame=leftline,
    framerule=3pt,
    rulecolor=\color{corporateblue},
    xleftmargin=25pt,
    aboveskip=20pt,
    belowskip=20pt,
    lineskip=1pt
}

% ===============================================
% ESTILO 4: MODERNO CON SOMBRAS
% ===============================================

% Colores modernos
\definecolor{modernbg}{rgb}{0.97,0.97,0.97}
\definecolor{moderngray}{rgb}{0.3,0.3,0.3}
\definecolor{modernpurple}{rgb}{0.5,0.2,0.8}
\definecolor{modernteal}{rgb}{0,0.5,0.5}
\definecolor{modernorange}{rgb}{0.8,0.4,0}

\lstdefinestyle{modern}{
    backgroundcolor=\color{modernbg},
    commentstyle=\color{modernteal}\itshape,
    keywordstyle=\color{modernpurple}\bfseries,
    numberstyle=\tiny\color{moderngray},
    stringstyle=\color{modernorange},
    basicstyle=\ttfamily\small,
    breakatwhitespace=false,
    breaklines=true,
    captionpos=t,
    keepspaces=true,
    numbers=left,
    numbersep=10pt,
    showspaces=false,
    showstringspaces=false,
    showtabs=false,
    tabsize=4,
    frame=tb,
    framerule=2pt,
    rulecolor=\color{modernpurple},
    xleftmargin=20pt,
    xrightmargin=20pt,
    aboveskip=25pt,
    belowskip=25pt
}

% ===============================================
% CONFIGURACIÓN PARA DIFERENTES LENGUAJES
% ===============================================

% Python
\lstdefinestyle{python}{
    language=Python,
    style=elegant,
    morekeywords={True,False,None,self,cls,def,class,import,from,as,with,yield,async,await},
    morecomment=[l]{\#},
    morestring=[b]',
    morestring=[b]"
}

% Java
\lstdefinestyle{java}{
    language=Java,
    style=corporate,
    morekeywords={var,record,sealed,permits,non-sealed}
}

% C++
\lstdefinestyle{cpp}{
    language=C++,
    style=modern,
    morekeywords={constexpr,nullptr,auto,decltype,override,final}
}

% JavaScript
\lstdefinestyle{javascript}{
    language=Java,
    style=elegant,
    morekeywords={let,const,var,function,class,extends,import,export,default,async,await,yield},
    morecomment=[l]{//},
    morecomment=[s]{/*}{*/},
    morestring=[b]',
    morestring=[b]",
    morestring=[b]`
}

% ===============================================
% EJEMPLOS DE USO
% ===============================================

% Para usar el estilo por defecto:
% \begin{lstlisting}
% código aquí
% \end{lstlisting}

% Para usar un estilo específico:
% \begin{lstlisting}[style=elegant]
% código aquí
% \end{lstlisting}

% Para incluir un archivo con estilo específico:
% \lstinputlisting[style=python]{archivo.py}

% Para código inline:
% \lstinline[style=modern]{código inline}

% ===============================================
% CONFIGURACIÓN ADICIONAL PARA TÍTULOS Y CARACTERES
% ===============================================

% Personalizar el formato de los títulos de los listados
\renewcommand\lstlistingname{Código}
\renewcommand\lstlistlistingname{Lista de Códigos}

% Configurar el formato del título con soporte para tildes
\lstset{
    %title=\lstname,
    captionpos=t,
    abovecaptionskip=10pt,
    belowcaptionskip=5pt,
    % Configuración global para caracteres especiales
    inputencoding=utf8,
    extendedchars=true
}

% ===============================================
% COMANDOS PERSONALIZADOS ÚTILES
% ===============================================

% Comando para código inline con soporte automático de tildes
\newcommand{\codeinline}[2][modern]{\lstinline[style=#1,inputencoding=utf8,extendedchars=true]{#2}}

% Comando para bloques de código con título personalizado
\newcommand{\codeblock}[3][elegant]{%
    \begin{lstlisting}[style=#1,caption={#2},label={lst:#2},inputencoding=utf8,extendedchars=true]
    #3
    \end{lstlisting}
}

% Comando para incluir archivos con configuración automática
\newcommand{\includecode}[3][python]{%
    \lstinputlisting[style=#1,caption={#3},label={lst:#3},inputencoding=utf8,extendedchars=true]{#2}
}

% ===============================================
% CONFIGURACIONES ESPECIALES PARA IDIOMAS
% ===============================================

% Configuración específica para código en español
\lstdefinestyle{español}{
    style=elegant,
    inputencoding=utf8,
    extendedchars=true,
    % Palabras clave en español para pseudocódigo
    morekeywords={función,procedimiento,inicio,fin,si,entonces,sino,mientras,para,hasta,hacer,repetir,caso,segun,verdadero,falso,entero,real,caracter,cadena,booleano,leer,escribir,imprimir}
}

% Configuración para comentarios multilíngües
\lstset{
    morecomment=[l]{//\ },
    morecomment=[l]{\#\ },
    morecomment=[s]{/*}{*/},
    morecomment=[s]{}
}

% ===============================================
% CONFIGURACIÓN PARA DIFERENTES LENGUAJES
% ===============================================

% Python
\lstdefinestyle{style1}{
    language=Python,
    style=elegant,
    morekeywords={True,False,None,self,cls,def,class,import,from,as,with,yield,async,await},
    morecomment=[l]{\#},
    morestring=[b]',
    morestring=[b]",
    % Soporte para caracteres especiales
    inputencoding=utf8,
    extendedchars=true
}

% Java
\lstdefinestyle{style2}{
    language=Java,
    style=corporate,
    morekeywords={var,record,sealed,permits,non-sealed},
    % Soporte para caracteres especiales
    inputencoding=utf8,
    extendedchars=true
}

% C++
\lstdefinestyle{style3}{
    language=C++,
    style=modern,
    morekeywords={constexpr,nullptr,auto,decltype,override,final},
    % Soporte para caracteres especiales
    inputencoding=utf8,
    extendedchars=true
}

\lstdefinelanguage{GDScript}{
  keywords={func, var, extends, class_name, if, else, for, while, return, match, in, and, or, not, break, continue, pass},
  sensitive=true,
  morecomment=[l]{\#},
  morestring=[b]",
  morestring=[b]',
}

\lstdefinestyle{gdstyle}{
  language=GDScript,
  basicstyle=\ttfamily\small,
  keywordstyle=\color{blue}\bfseries,
  commentstyle=\color{gray},
  stringstyle=\color{red!60!black},
  numbers=left,
  numberstyle=\tiny\color{gray},
  breaklines=true,
  frame=single,
  tabsize=2,
}


% ===========================
% Estilo global de tablas
% ===========================

\usepackage{booktabs}   % reglas profesionales
\usepackage{colortbl}   % color en filas
\usepackage{xcolor}     % colores
\usepackage{float}      % [H]

% Color de filas alternadas
% \rowcolors{2}{gray!10}{white}

% % Espacio vertical entre filas
% \renewcommand{\arraystretch}{1.2}

% % Cambiar el tamaño de columna por defecto
% \setlength{\tabcolsep}{8pt}

% % Redefinir tabla para que todas las tablas tengan el estilo
% \let\oldtabular\tabular
% \let\endoldtabular\endtabular
% \renewenvironment{tabular}[1]{%
%   \oldtabular{#1}%
% }{%
%   \endoldtabular
% }

% \usepackage{longtable,booktabs,xcolor}
% \rowcolors{2}{gray!10}{white}   % filas alternadas
% \renewcommand{\arraystretch}{1.2} % espacio vertical entre filas

% % Mostrar siempre el número de la tabla
% \usepackage{caption}
% \captionsetup[table]{labelformat=default, labelsep=colon, textfont=bf}


% ===========================
% Estilos para tikz y figures
% ===========================

\usepackage{caption}
\captionsetup{
    font={it},       % fuente en cursiva
    labelfont={},  % etiqueta ("Figura 1") en negrita
    textfont={it},   % texto del caption en cursiva
    justification=centering,  % centra el texto (opcional)
    font={small},    % tamaño de fuente pequeño
}

\usepackage{tikz}
\usetikzlibrary{positioning}

\tikzset{
  state/.style={
    draw,
    circle,
    minimum size=1cm,
    thick,
    fill=yellow!20
  },
  block/.style={
    rectangle,
    draw,
    fill=blue!10,
    rounded corners,
    text centered,
    minimum height=1cm,
    minimum width=2cm,
    thick
  },
  none/.style={
    draw=none,
    fill=none,
    text centered
  },
  error/.style={
    draw,
    circle,
    minimum size=1cm,
    thick,
    fill=red!30
  },
  initial text={}
}   % estilos de secciones, etc.

% ========================
% Configuración índice y listas
% ========================
\setlength{\cftbeforesecskip}{5pt}
\setlength{\headheight}{14pt}  % un poco más que 13.6pt

\renewcommand{\normalsize}{\fontsize{10}{12}\selectfont}

% Fix para listas de Pandoc
\providecommand{\tightlist}{%
  \setlength{\itemsep}{0pt}\setlength{\parskip}{0pt}}


%=======================
% fancy with parameters
%=======================
%\fancyfoot[L]{\scriptsize\itshape Contabilidad de Gestión}
\fancyfoot[L]{\scriptsize\itshape Contabilidad de
Gestión} % pie de página izquierdo en cursiva

\setcounter{tocdepth}{1} % Muestra solo hasta subsecciones en el índice

% ========================
% Inicio del documento
% ========================
\begin{document}

%% portada.tex
\begin{titlepage}
    \newgeometry{top=2cm,bottom=2cm,left=2.5cm,right=2.5cm} % márgenes personalizados
    
    % Fondo con transparencia
    \begin{tikzpicture}[remember picture,overlay]
        \node[opacity=0.15,inner sep=0pt] at (current page.center)
            {\includegraphics[width=\paperwidth,height=\paperheight]{../../img/fondoPrueba.jpg}};
    \end{tikzpicture}

    % Contenido de la portada
    \begin{center}
        \vspace*{2cm}
        
        {\Huge \bfseries\scshape Título del Libro de Apuntes \par}
        \vspace{0.5cm}
        {\Large \itshape Subtítulo o Asignatura \par}
        \vspace{0.5cm}
        {\Large \itshape \href{https://ismael-sallami.github.io}{https://ismael-sallami.github.io} \par}


        \vfill
        
        {\LARGE Autor: \textbf{Tu Nombre Completo} \par}
        \vspace{0.3cm}
        % {\Large Universidad Ejemplo \par}
        
        \vspace{1cm}
        \includegraphics[width=0.25\textwidth]{../../img/ugr.png} % opcional: logo
        \vspace{1cm}
        
        {\large \today}
    \end{center}
    
    \restoregeometry
\end{titlepage}



%==========================
% PORTADA: ENTRADA MANUAL
%==========================

% portada.tex
\begin{titlepage}
    \newgeometry{top=2cm,bottom=2cm,left=2.5cm,right=2.5cm} % márgenes personalizados
    
    % Fondo con transparencia
    \begin{tikzpicture}[remember picture,overlay]
        % \node[opacity=0.15,inner sep=0pt] at (current page.center)
        \node[inner sep=0pt] at (current page.center)
            {\includegraphics[width=\paperwidth,height=\paperheight]{../../../extraFiles/img/fondo_ade.jpg}};
    \end{tikzpicture}

    % Contenido de la portada
    \begin{center}
        \vspace*{2cm}
        
        {\Huge \bfseries\scshape Teoría y Práctica \par}
        \vspace{0.5cm}
        {\Large \itshape Contabilidad de Gestión \par}
        \vspace{0.5cm}
        {\small \itshape \href{https://ismael-sallami.github.io}{https://ismael-sallami.github.io} \par}
        {\small \itshape \href{https://elblogdeismael.github.io}{https://elblogdeismael.github.io} \par}


        \vfill
        
        {\LARGE Autor: \textbf{Ismael Sallami Moreno} \par}
        \vspace{0.3cm}
        % {\Large Universidad de Granada \par}
        
        \vspace{1cm}
        \includegraphics[width=0.25\textwidth]{../../../extraFiles/img/ugr.png} % opcional: logo
        \vspace{1cm}
        
        {\large \today}
    \end{center}
    
    \restoregeometry
\end{titlepage}


% ===============================
% licencia.tex
% ===============================
\begin{tikzpicture}[remember picture,overlay]
\node[anchor=south west, xshift=1cm, yshift=1cm] at (current page.south west) {
\begin{minipage}{0.4\textwidth}
\begin{flushleft}
\section*{Licencia}

Este trabajo está bajo una 
\href{https://creativecommons.org/licenses/by-nc-nd/4.0/}{Licencia Creative Commons BY-NC-ND 4.0}.

\bigskip

Permisos: Se permite compartir, copiar y redistribuir el material en cualquier medio o formato.

\bigskip

Condiciones: Es necesario dar crédito adecuado, proporcionar un enlace a la licencia e indicar si se han realizado cambios. No se permite usar el material con fines comerciales ni distribuir material modificado.

\bigskip

\begin{center}
  \href{https://creativecommons.org/licenses/by-nc-nd/4.0/}{\includegraphics[width=0.35\textwidth]{../../../extraFiles/img/by-nc-nd.png}}
\end{center}
\end{flushleft}
\end{minipage}
};
\end{tikzpicture}
  % licencia
\thispagestyle{empty} % quitar número de página en la portada
\clearpage

% --- Índice ---
\tableofcontents
\listoffigures
\clearpage

\listoftables
\clearpage
\thispagestyle{empty} % quitar número de página en la portada
\clearpage

% Índice de código
\renewcommand{\lstlistlistingname}{Índice de Código}
\lstlistoflistings
\clearpage

% Índice de ecuaciones
\renewcommand{\listtheoremname}{Índice de Ecuaciones}
\listoftheorems[ignoreall,show={equation}]
\clearpage

% --- Contenido Markdown generado por Pandoc ---
\part{Teoría}

\hypertarget{la-contabilidad-de-gestiuxf3n}{%
\chapter{La Contabilidad de
Gestión}\label{la-contabilidad-de-gestiuxf3n}}

\hypertarget{modelo-buxe1sico-de-la-circulaciuxf3n-de-valores-de-la-empresa}{%
\section{Modelo básico de la circulación de valores de la
empresa}\label{modelo-buxe1sico-de-la-circulaciuxf3n-de-valores-de-la-empresa}}

Destacamos la interacción económica de la empresa con el mundo exterior
donde se compra los materiales que pasan por el proceso de producción
para su venta. Esto se conoce como corriente real (lo que entra y sale
de la empresa), y por otro lado está la corriente monetaria (lo que
pagamos). En esta asignatura nos interesan ambas corrientes para ver si
somos rentables y competitivos. Cada compra y venta genera un
equivalente monetario (gasto e ingreso, respectivamente).

Distinguimos:

\begin{itemize}
\tightlist
\item
  Corriente económico-financiera: compra de los recursos iniciales y
  demás para continuar con la empresa.
\item
  Corriente económico-técnica: transformación de las compras a los
  proveedores.
\end{itemize}

En cuanto a los subsistemas de una unidad económica de producción
encontramos:

\begin{itemize}
\tightlist
\item
  \textbf{Financiación}: Operaciones relativas a la obtención de
  recursos financieros.
\item
  \textbf{Inversión}: Operaciones relativas a la obtención de los
  factores productivos.
\item
  \textbf{Producción}: Operaciones relativas a la aplicación de los
  factores productivos al proceso productivo para la obtención de nuevos
  bienes o servicios.
\item
  \textbf{Desinversión}: Operaciones relativas a la colocación de los
  productos (bienes o servicios) en el mercado.
\end{itemize}

\incluirimagen[]{media/esq_circ.png}{Esquema de la circulación de valores de la empresa}

Siempre que hay una compra hay un gasto (equivalente monetario de la
compra), si compro materiales, voy a tener una factura donde se detalle
lo que he comprado.

No siempre que hay una compra hay un pago en ese mismo instante, si se
paga al momento se conoce como pago al contado, si es antes es pago
anticipado y si es después se conoce como compra a plazo.

Consumo: aplicación de los recursos para el proceso de producción.
Consumo y coste tienen lugar en el mismo instante, coste siendo el
equivalente monetario y el consumo el equivalente real.

Colocación: una vez que vendemos el producto, corresponde con la entrega
del producto/servicio.

Venta no es lo mismo que colocación. Venta hace referecia a que hay una
cesión de una propiedad (Aunque podemos ver que la venta coincide en
ciertos casos con la colocación). La venta puede producirse antes de la
colocación, pero al revés no.

El valor de la colocación está calculado en base a los costes, mientras
que el valor de venta es el ingreso como tal, por ende, la diferencia
nos da el margen que tenemos.

Distinguimos dos ámbitos:

\begin{itemize}
\item
  \textbf{Ámbito interno de la circulación}: Se refiere a las
  operaciones y procesos que ocurren dentro de la empresa, como la
  transformación de los recursos adquiridos en productos o servicios, y
  su preparación para la colocación en el mercado.
\item
  \textbf{Ámbito externo de la circulación}: Se refiere a las
  interacciones de la empresa con el entorno externo, como la
  adquisición de recursos de los proveedores y la colocación o venta de
  productos o servicios a los clientes.
\end{itemize}

Recdordamos que al final del ejercicio, si no vendemos, debemos de
calcular al variación de existencias, dando de bajo a las iniciales y de
alta a las finales, las cuales paso al inicio del año que viene como
iniciales. Para poder calcular el resultado en la contabilidad
financiera necesita de la de gestión las existencias finales. De manera
análoga, tenemos que la periodificación de gastos es necesario para la
contabilidad de gestión, cumpliendo con el principio de devengo.

\hypertarget{la-contabilidad-de-gestiuxf3n-delimitaciuxf3n-y-objetivos}{%
\section{La contabilidad de gestión: delimitación y
objetivos}\label{la-contabilidad-de-gestiuxf3n-delimitaciuxf3n-y-objetivos}}

\begin{table}[H]
\centering
\begin{tabular}{|l|p{5cm}|p{5cm}|}
\hline
\textbf{Aspecto} & \textbf{Contabilidad Financiera} & \textbf{Contabilidad de Gestión} \\ \hline
\textbf{Usuarios} & Externos e internos & Internos \\ \hline
\textbf{Restricciones} & Regulado: reglas emitidas por los principios contables generalmente aceptados y por el Estado & No regulado totalmente: sistemas e información determinados por la dirección para satisfacer sus necesidades estratégicas y operativas \\ \hline
\textbf{Naturaleza de la información} & Prima la objetividad y la fiabilidad. La información es precisa y auditable (verificable) & Prima la relevancia y la flexibilidad para la toma de decisiones. Información más subjetiva (estimaciones) \\ \hline
\textbf{Tipo de información} & Principalmente medidas financieras & Medidas financieras, operativas y físicas sobre procesos, tecnología, etc. \\ \hline
\textbf{Carácter} & Agregada y global: informa sobre el conjunto de la organización (Ámbito externo) & Desagregada y concreta: informa sobre decisiones y acciones de departamentos y segmentos de la organización (Ámbito interno) \\ \hline
\end{tabular}
\caption{Comparación entre Contabilidad Financiera y Contabilidad de Gestión}
\end{table}

Podemos distinguir dos magnitudes fundamentales del ámbito interno:

\begin{itemize}
\tightlist
\item
  Magnitudes flujo o corrientes: Consumos y costes de un período,
  producción y valor de la producción de un período, colocación y valor
  de la producciónn de un período.
\item
  Magnitudes fondo o stocks: producción en stock y valor de la
  producción en stock. Producción en curso de fabricación y valor de la
  producción en curso de fabricación.
\end{itemize}

\begin{ejemplo}
El cobro y el pago son ejemplos de magnitudes flujo, ya que representan movimientos de dinero en un período de tiempo determinado. Por otro lado, el saldo de tesorería es una magnitud fondo, ya que refleja el estado acumulado de los recursos financieros disponibles en un momento específico.
\end{ejemplo}

\incluirimagen[]{media/mag.png}{Magnitudes fundamentales del ámbito interno}

\begin{definicion}[Contabilidad de Gestión]
Rama de la Contabilidad aplicada que, con respecto a una microunidad económica, nos permite en todo momento el conocimiento cualitativo y cuantitativo de su realidad económico-técnica o interna, con el fin específico de permitir el control de la producción y los costes de dicha unidad y llevar a cabo la medida de la eficiencia técnico-productiva de la misma. Objetivos: representación de la problemática de la empresa, evaluación de las actividades productivas, resultado periódico con criterios económicos, descomposición del resultado interno, suministro de la información para el órgano de gestión. No se puede aplicar *LIFO* en la contabilidad financiera, pero *FIFO* sí.
\end{definicion}

\subsubsection*{Producción como efecto}

\underline{Clasificación} de la producción según su grado de
perfeccionamiento.

\begin{itemize}
\tightlist
\item
  Producción final: Productos acabados.
\item
  Producción intermedia:

  \begin{itemize}
  \item
    Productos en curso: factores productivos que se han proporcionado en
    la fase de producción y de la cual no ha salido nada, atendiendo al
    flujo de Consumo \(\rightarrow\) Producción \(\rightarrow\)
    Colocación.
  \item
    Producción semiterminada: ha salido de una fase y esta pendiente de
    entrar en otra fase. La diferencia respecto de la anterior es
    conceptual, cada una depende del contexto y la finalidad. Tendemos a
    asociar que semiterminada hace referencia a que esta más cerca de
    ser terminada, cosa que no siempre es así. No implica que tenga un
    determinado grado de avance. La primera esta en una fase del proceso
    productivo, mientras que la segunda está almacenada en la empresa
    esperando a entrar en otra parte del proceso productivo. El flujo
    sería:

    \nota{Flujo del producto}{
        MATERIA PRIMA $\rightarrow$ TALLER DE COSTE $\rightarrow$  PIEZAS CORTADAS $\rightarrow$ TALLER DE MONTAJE $\rightarrow$ MUEBLES MONTADAS $\rightarrow$ TALLER ACABADO $\rightarrow$ MUEBLES ACABADOS [Cada taller hace referencia a la producción en curso.]
    }
  \end{itemize}
\end{itemize}

Otra producción destaca en función de la utilidad que se le puede dar en
la empresa: - Subproducto: producto que aún se le puede dar provecho,
por otro lado, si la empresa lo crea para venderlo de manera
complementaria a otro producto, también lo es. - Desperdicios: aquellos
materiales o productos que le generan un coste, pérdidas.

En \textbf{contabilidad de gestión}, el término \textbf{producción}
puede tener varias acepciones según el enfoque que se utilice. De manera
general, se pueden señalar las siguientes definiciones:

\begin{enumerate}
\def\labelenumi{\arabic{enumi}.}
\item
  \textbf{Producción como actividad o proceso}\\
  Es el conjunto de operaciones mediante las cuales la empresa
  transforma materias primas, insumos y otros recursos en bienes o
  servicios destinados a la venta o al consumo interno. En este sentido,
  se entiende como el \textbf{flujo físico} de transformación.
\item
  \textbf{Producción como volumen o cantidad obtenida}\\
  Se refiere a la \textbf{medida cuantitativa} de los bienes o servicios
  producidos en un período determinado (unidades, toneladas, horas de
  servicio, etc.). Se usa como base para calcular costos unitarios,
  eficiencia y productividad.
\item
  \textbf{Producción como valor económico}\\
  Representa el valor monetario de los bienes y servicios generados, que
  se incorpora al cálculo del \textbf{costo de producción}. Incluye los
  costos de materiales, mano de obra y gastos generales de fabricación.
\item
  \textbf{Producción como función dentro de la empresa}\\
  Se entiende como el \textbf{área o departamento} responsable de
  planificar, coordinar y controlar los procesos productivos. Desde la
  contabilidad de gestión, se analizan sus costos, eficiencia y
  rentabilidad.
\end{enumerate}

\subsubsection*{Producción intermedia}

\incluirimagen{media/prodintermedia.png}{Flujo de la producción intermedia}

\subsubsection*{Producción como causa}

Un proceso productivo puede caracterizarse como una `'transformación'',
según una determinada técnica, de factores productivos en productos.
\footnote{Calafell, 1970.}

El flujo en este caso sería:

FACTORES PRODUCTIVOS \(\xrightarrow{\text{Consumo}}\)
\(\text{CENTROS DE TRABAJO (Actividad y \textit{Transformación productiva})}\)
\(\xrightarrow{\text{Producción}}\) \(\text{PRODUCTOS Y SERVICIOS}\)
Esto se conoce como \emph{vertiente técnica del proceso productivo}.

\underline{Factores}

Medios de producción o factores de producción: aquellos que hacen
posible el desarrollo del proceso productivo. Cada uno de los recursos
económicos o de los medios de producción naturales o previamente
elaborados, que son utilizados en la función de transformación
económica, sea esta industrial, comercial o financiera.

Si la empresa misma fabrica la pieza esta es un producto semiterminado,
si la compra en fuentes externas se trata de un factor productivo.

Clasificaciones: - En función de la participación en el proceso
productivo - Estructurales: permanecen en la empresa durante varios
periodos. - Perfeccionamiento: cuando se consumen, aún quedan en la
empresa. Ejemplo: trabajador. - Materia prima: aquella que resulta ser
la base para la transformación, sobre la que actúan los otros factores
productivos. - Medios colaboradores: los demás casos, como la energía,
\ldots{} - En función de la influencia en la caracterización del
producto final - Factores limitativos. Ejemplo: el sofá es de madera de
roble. - Factores sustitutivos: Mano de obra y máquina, pueden ser
sustitutivos.

\subsubsection*{Clases de centros de trabajo según diferentes niveles de agregación:}

\begin{definicion}[Unidad de trabajo]
Célula básica de la actividad, que me va a permitir desarrollar la actividad empresarial, integrada completamente por uno o más medios estructurañes y su correspondiente dotación de personal. Por su propia naturaleza y composición, la unidad de trabajo es indivisible.
\end{definicion}

\begin{definicion}[Lugar de trabajo]
Espacio físico dentro de la empresa donde se desarrollan las actividades productivas, equipado con los medios necesarios para llevar a cabo las tareas asignadas. Puede incluir maquinaria, herramientas y otros recursos necesarios para el proceso productivo.
\end{definicion}

\begin{definicion}[Sección de trabajo]
Conjunto de lugares de trabajo que comparten una función específica dentro del proceso productivo. Constituye una unidad organizativa dentro de la empresa, diseñada para optimizar la eficiencia y la coordinación de las actividades productivas.
\end{definicion}

\begin{ejemplo}
Una oficina es un ejemplo de unidad de trabajo, donde tenemos al trabajador y su equipo, escritorio y demás como Unidad de trabajo, como Lugar de trabajo, podemos mencionar la oficina, y com sección a que área en específico se centra.
\end{ejemplo}

\begin{itemize}
\tightlist
\item
  Producción simple: solo se produce un producto.
\item
  Producción múltiple o compuesta: se fabrican varios productos
  siguiendo diversas estrategias.

  \begin{itemize}
  \tightlist
  \item
    Producción pararela: se realizan varias líneas de producción de
    manera que no hay interferencia entre ellas, no hay dependencia
    entre ellas, cada una nececita factores productivos independientes.
  \item
    Producción alternativa: si se quiere producir más de un producto, se
    debe de reducir la de otro, hay una competencia interna por el uso
    de diversos recursos que influyen en el proceso productivo de ambos
    productos.
  \item
    Producción conjunta: se obtiene más de un producto, a diferencia de
    la alternativa, es inherente al proceso productivo, es decir, no hay
    decisión que afecte a ese proceso productivo para evitar la
    producción de uno de los productos, por ejemplo, es imposible evitar
    la generación de serrín en una empresa manufacturera de madera.
  \end{itemize}
\end{itemize}

\incluirimagen{media/esquema1.png}{Tipos de producción}

\dosimagenes{media/prodalternativa.png}{Producción alternativa}{media/prodsimple.png}{Producción Simple}{}{}

\dosimagenes{media/prodcomplejo.png}{Producción Compleja}{media/prodconjunta.png}{Producción Conjunta}{}{}

\dosimagenes{media/prodpararela.png}{Producción pararela}{media/prodintermedia.png}{Producción Intermedia}{}{}

\hypertarget{conceptos-buxe1sicos}{%
\chapter{Conceptos básicos}\label{conceptos-buxe1sicos}}

\hypertarget{la-nociuxf3n-de-coste-concepto-y-clases.}{%
\section{La noción de coste: concepto y
clases.}\label{la-nociuxf3n-de-coste-concepto-y-clases.}}

\begin{definicion}[Coste]
Se entiende por coste el consumo, valorado en dinero, de bienes y servicios para la producción que constituye el objetivo de la empresa. \footnote{Pedersen, 1959} Es una magnitud relativa debido a que la \textit{indeterminación} a menudo asociada a la medida y valoración de los consumos y por otro lado, la \textit{indeterminación} inherente a la asignación del coste incurrido a los centros de actividad y a los productos.
\end{definicion}

\hypertarget{tipologuxeda-de-costes}{%
\subsection{Tipología de costes}\label{tipologuxeda-de-costes}}

\hypertarget{costes-seguxfan-su-naturaleza}{%
\subsubsection{Costes según su
naturaleza}\label{costes-seguxfan-su-naturaleza}}

\begin{itemize}
\tightlist
\item
  Costes de materiales
\item
  Coste de personal
\item
  Coste de suministros
\item
  Coste de servicios exteriores
\item
  Coste de amortización
\item
  Coste financiero
\item
  Otros costes
\end{itemize}

\hypertarget{costes-seguxfan-su-relaciuxf3n-con-el-objeto-coste}{%
\subsubsection{Costes según su relación con el objeto
coste}\label{costes-seguxfan-su-relaciuxf3n-con-el-objeto-coste}}

\begin{definicion}[Costes directos]
Son aquellos que pueden ser asignados de forma inequívoca, económica y directa al objeto del coste. Ej: materia prima.
\end{definicion}

\begin{definicion}[Costes indirectos]
Son aquellos que no se conoce cómo llevar a cabo su asignación al objeto de coste y, por lo tanto, precisan de criterios de reparto para poder asignarlos. Ej: el coste de amortización del edificio en el que ubica la empresa.
\end{definicion}

\hypertarget{costes-seguxfan-su-relaciuxf3n-con-el-cuxe1lculo-del-resultado}{%
\subsubsection{Costes según su relación con el cálculo del
resultado}\label{costes-seguxfan-su-relaciuxf3n-con-el-cuxe1lculo-del-resultado}}

\begin{definicion}[Costes del producto]
Aquellos que se incorporan al resultado de la empresa en el periodo en el que se venden los productos, con independencia del periodo en el que se hayan producido el coste. Depende del modelo de asignación de costes.
\end{definicion}

\begin{definicion}[Costes periodo]
Son aquellos que se incorporan al resultado en el periodo en el que se ha producido el coste, con independencia de cuando se venda el producto. Depende del modelo de asignación de costes.
\end{definicion}

\hypertarget{costes-seguxfan-su-relaciuxf3n-con-el-momento-de-cuxe1lculo}{%
\subsubsection{Costes según su relación con el momento de
cálculo}\label{costes-seguxfan-su-relaciuxf3n-con-el-momento-de-cuxe1lculo}}

\begin{definicion}[Costes históricos]
Son aquellos cuyo cálculo se realiza una vez que se haya producido la actividad, es decir, su cálculo se realiza a posteriori.
\end{definicion}

\begin{definicion}[Costes predeterminados]
Son aquellos cuyo cálculo se realiza antes de que se produzca la actividad, es decir, su cálculo se realiza a priori.
\end{definicion}

\hypertarget{principales-funciones-de-coste-costes-fijos-y-costes-variables.}{%
\section{Principales funciones de coste: costes fijos y costes
variables.}\label{principales-funciones-de-coste-costes-fijos-y-costes-variables.}}

\hypertarget{costes-seguxfan-su-relaciuxf3n-con-el-volumen-de-actividad}{%
\subsubsection{Costes según su relación con el volumen de
actividad}\label{costes-seguxfan-su-relaciuxf3n-con-el-volumen-de-actividad}}

\begin{definicion}[Costes fijos]
Aquellos que no fluctúan frente a una variación en el volumen de producción, dentro de los límites definidos por la capacidad productiva disponible, a lo largo del periodo de estudio. Podemos distinguir:
\begin{itemize}
\item \textbf{Costes en estado parado o costes de estructura:} soportados por la empresa, aunque no se desarrolle ningún tipo de actividad, que son debidos a la estructura de la empresa y el mantenimiento de la misma.
\item \textbf{Costes de puesta en marcha o de preparación de la producción:} Se originan como consecuencia de la adecuación de la empresa para ponerse a fabricar, aunque no se produzca ninguna unidad de producto. 
\end{itemize} 
\end{definicion}

\begin{definicion}[Costes variables]
Se pueden dividir en:
\begin{itemize}
\item \textbf{Costes proporcionales:}  
  Son aquellos que varían de forma directa y proporcional a las variaciones que experimenta la variable independiente que tomamos como indicador de la actividad de la empresa. Por ejemplo, el coste de la materia prima.

\item \textbf{Costes progresivos: } 
  Son aquellos que varían de forma directa y más que proporcional a las variaciones que experimenta la variable independiente que tomamos como indicador de la actividad de la empresa. Por ejemplo, el coste de la mano de obra con horas extraordinarias.

\item \textbf{Costes degresivos: } 
  Son aquellos que varían de forma directa y menos que proporcional a las variaciones que experimenta la variable independiente que tomamos como indicador de la actividad de la empresa. Por ejemplo, cualquier elemento de coste cuyo mayor consumo vaya acompañado de tarifas degresivas (reprografía).

\item \textbf{Costes regresivos: } 
  Son aquellos que varían de forma inversa a las variaciones que experimenta la variable independiente que tomamos como indicador de la actividad de la empresa. Por ejemplo, el coste de calefacción en un teatro.

\item \textbf{Costes semifijos, variables a saltos o en escalera:}  
  Son aquellos que permanecen fijos para determinados intervalos del volumen de actividad, y luego varían, dan un salto cuantitativo, para comportarse, de nuevo, como fijos en el siguiente intervalo. Por ejemplo, el coste de los monitores en cursos de natación.  
  Estos costes semifijos, pueden ser \textit{reversibles o irreversibles}. En este último caso, se produce un fenómeno denominado histéresis de los costes.

\item \textbf{Costes semivariables: } 
  Son aquellos que tienen una parte fija y otra parte variable. A la hora de reclasificar los costes, la parte fija formará parte de los costes fijos y la parte variable se integrará dentro de los costes variables.


\end{itemize}
\end{definicion}

\hypertarget{costes-necesarios-versus-costes-no-necesarios-costes-de-la-actividad-y-costes-de-la-subactividad.}{%
\section{Costes necesarios versus costes no necesarios: costes de la
actividad y costes de la
subactividad.}\label{costes-necesarios-versus-costes-no-necesarios-costes-de-la-actividad-y-costes-de-la-subactividad.}}

Son aquellos asociados con la mano de obra y el equipo para el
desarrollo de la actividad. Sus causas pueden ser diversas.

\begin{thebibliography}{99}

  \bibitem{Referencia1}
  Ismael Sallami Moreno, \textbf{Estudiante del Doble Grado en Ingeniería Informática + ADE}, Universidad de Granada, 2025.
  
  \bibitem{DiapositivasAsignatura}
  Universidad de Granada, \emph{Diapositivas de la asignatura}, Curso 2025/2026.

  % \bibitem{Referencia2}
  % Autor Apellido, \emph{Título del libro o artículo}, Editorial o Revista, Año.
  
  % \bibitem{Referencia3}
  % Nombre Autor, \emph{Título del documento}, Conferencia/URL, Año.
  
  \end{thebibliography}
  

\end{document}
