\chapter{Formación y Asignación de Costes}

\section{Sistemas de Costes}

\subsection{Contabilidad Interna: Finalidad y Fundamentos}

La Contabilidad de Gestión, también conocida históricamente como contabilidad interna, analítica o de costes, constituye una rama de la Contabilidad aplicada centrada en la microunidad económica. Su propósito fundamental es proporcionar en todo momento el conocimiento cualitativo y cuantitativo de la realidad económico-técnica o interna de la empresa.

El sistema de información contable interno sintetiza la operativa de la empresa y suministra datos esenciales sobre la formación y estructura del coste de productos y servicios, secciones y de la empresa en general, facilitando la toma de decisiones de gestión.

\begin{definicion}[Contabilidad de Gestión]
La Contabilidad de Gestión es el proceso de identificación, medida, acumulación, análisis, preparación, interpretación y comunicación de la información acerca de los sucesos económicos que se originan en la empresa, siendo utilizado por los directivos para planear, evaluar y controlar la organización y para asegurar un uso y control responsable de los recursos.
\end{definicion}

Los principales \text{fines} de esta disciplina aplicada incluyen:
\begin{itemize}
    \item La captación, medida, valoración y representación de la problemática económico-técnica de la empresa.
    \item El análisis, evaluación y control de las actividades productivas.
    \item La determinación del resultado periódico interno mediante la aplicación de criterios económicos.
    \item El suministro de la información relevante necesaria para apoyar la toma de decisiones de los órganos de gestión.
\end{itemize}
A diferencia de la contabilidad financiera, que está regulada por normativas externas, la contabilidad de gestión no está sujeta a regulación externa, y su diseño se adapta a las características y necesidades específicas de cada organización.

\subsection{Modelos de Costes}

Los sistemas de costes son clasificados según varios criterios que definen su naturaleza y la forma en que el coste se calcula y se imputa.

\subsubsection{Según Formación: Orgánicos \textit{vs.} Inorgánicos}

Esta tipología se basa en el proceso de acumulación de costes.

\begin{itemize}
    \item \text{Modelos Orgánicos:} Acumulan los costes siguiendo la estructura técnico-organizativa de la empresa, creando \text{centros de localización de costes} o secciones. Son más complejos y típicos en Europa, e implican las fases de clasificación, localización e imputación.
    \item \text{Modelos Inorgánicos:} Ignoran la fase de localización de los costes en centros intermedios, llevándolos directamente a los productos o servicios. Se limitan a las fases de clasificación e imputación. Son valorados por su simplicidad y rapidez en el cálculo.
\end{itemize}

\subsubsection{Según Asignación: \textit{Full-Cost} \textit{vs.} \textit{Direct-Cost}}

Determinan el nivel de agregación o la proporción de costes que se incorpora al coste final del producto.

\begin{itemize}
    \item \text{Modelos de \textit{Full-Cost} (Costes Completos):} Son modelos de asignación completa que defienden que todas las cargas incorporables (tanto variables como fijas de producción) deben integrar el coste final del producto.
    \item \text{Modelos de \textit{Direct-Cost} (Costes Parciales):} Son modelos de asignación parcial. En su versión simple, solo los costes variables intervienen en el coste del producto, relegando los costes fijos a gastos del periodo.
\end{itemize}

\subsubsection{Según Cálculo: Históricos \textit{vs.} Predeterminados}

Esta clasificación se refiere al momento en que se calculan los costes, lo cual es vital para el control y la planificación.

\begin{itemize}
    \item \text{Costes Históricos (o Reales):} Se calculan \textit{a posteriori}, basándose en los consumos y valores que la empresa tuvo durante un período pasado.
    \item \text{Costes Predeterminados (o Estándares):} Se calculan \textit{a priori}, utilizando los consumos que la empresa espera tener en el futuro bajo condiciones normales. Su principal finalidad es mejorar la toma de decisiones y el control, al permitir la comparación con los costes reales y el análisis de desviaciones.
\end{itemize}

\section{Flujo de Costes}

\subsection{Proceso Productivo: Vertiente Técnica y Económica}

Un proceso productivo se define como la transformación de factores productivos (inputs) en productos y servicios (outputs) mediante una técnica determinada.

El conjunto de actividades en la empresa se concibe como una \text{cadena de valor} interrelacionada, donde cada etapa debe añadir valor a la siguiente. Los recursos consumidos durante estas actividades generan costes.

\begin{itemize}
    \item \text{Vertiente Técnica:} Se enfoca en la transformación productiva en los \text{Centros de Trabajo} o actividad.
    \item \text{Vertiente Económica:} Se enfoca en la transformación de valores (Coste y Valor de la producción), siendo medida en los \text{Centros de Coste}.
\end{itemize}
El sistema de costes busca calcular de forma secuencial y acumulada el coste a lo largo del proceso, de manera que el producto adquiere un coste mayor a medida que avanza hacia su terminación.

\subsection{Formación del Coste}

El proceso de formación del coste sigue una estructura lógica de tres pasos: Clasificación, Localización e Imputación.

\subsubsection{Clasificación: Naturaleza de Costes}

La clasificación es la medición y valoración del consumo de los factores productivos. Partiendo de los gastos de la contabilidad financiera, se seleccionan aquellos que son \text{cargas incorporables} (CI) al movimiento interno de valores, las cuales formarán los costes analíticos.

Las cargas incorporables se dividen según su relación con el objeto de coste:
\begin{itemize}
    \item \text{Costes Directos:} Asignados de forma inequívoca, económica y directa al objeto de coste (ej., materia prima).
    \item \text{Costes Indirectos:} No es posible su asignación directa y precisan de criterios de reparto (ej., amortización de un edificio).
\end{itemize}

\subsubsection{Localización: Centros de Coste}

La localización es la distribución del coste de los factores entre los centros de coste (Reparto Primario).

Un centro de coste es una unidad organizativa (tarea, actividad, función o departamento) para la cual se calculan costes, sirviendo como instrumento para la planificación y el control.

En el modelo orgánico, los centros se distinguen en:
\begin{itemize}
    \item \text{Centros Principales (Producción):} Aquellos que añaden valor observable al producto o servicio (ej., Ensamblaje).
    \item \text{Centros Auxiliares (Apoyo):} Aquellos que prestan servicios a otros centros (ej., Mantenimiento).
\end{itemize}
El coste de los centros auxiliares se traslada a los principales a través del \text{Reparto Secundario} (R2).

\subsubsection{Imputación: Portadores de Coste}

La imputación es la asignación de los costes de los centros a los \text{portadores de coste} (productos o servicios). Esta fase utiliza las \text{unidades de obra} o claves de reparto, que son las bases de medida de la actividad de la sección, expresadas en unidades físicas, temporales o monetarias (ej., horas/máquina o euros vendidos).

En el modelo orgánico, los costes finales del producto incluyen la afectación de los costes directos y la imputación de los costes indirectos acumulados en los centros principales.

\subsection{Modelos Contables}

\subsubsection{Modelo Orgánico: Redistribución Indirecta}

El modelo orgánico acumula los costes siguiendo la estructura técnica de la empresa. Se caracteriza por el uso de la \text{Localización} (Reparto Primario) y la \text{Redistribución Indirecta} (Reparto Secundario).

\begin{itemize}
    \item \text{Reparto Primario:} Distribución de los costes de los factores entre los centros.
    \item \text{Reparto Secundario (Redistribución Indirecta):} Redistribución de los costes de los centros auxiliares a los centros principales.
    \item \text{Imputación:} Afectación de costes directos e imputación de costes indirectos acumulados en los centros principales a los productos.
\end{itemize}

\subsubsection{Modelo Inorgánico: Imputación Directa}

El modelo inorgánico es más simple, ya que prescinde de la fase de localización.

\begin{itemize}
    \item \text{Clasificación:} Medición y valoración del consumo de factores productivos.
    \item \text{Imputación Directa:} La afectación del coste directo y la imputación propiamente dicha de los restantes costes indirectos se realiza directamente a los productos o servicios, sin pasar por los centros de coste.
\end{itemize}

\section{Asignación y Resultados}

\subsection{Costes y Resultados: Producto \textit{vs.} Periodo}

La clasificación de costes en costes del producto o del periodo determina el momento en que se comparan con los ingresos, afectando al resultado y a la valoración del activo (inventarios).

\begin{itemize}
    \item \text{Costes del Producto (Inventariables):} Son aquellos costes que se asignan a los bienes inventariables (ej., materias primas, mano de obra directa, costes indirectos de fabricación). Se incorporan al resultado solo en el periodo en que el producto se vende, independientemente de cuándo se incurrió en el coste.
    \item \text{Costes del Periodo:} Son aquellos que se incorporan al resultado en el mismo periodo en que se ha producido el coste. Incluyen, por ejemplo, los costes de administración, I+D, marketing y distribución.
\end{itemize}

\subsection{\textit{Full-Cost}}

El enfoque \textit{Full-Cost} (Costes Completos) se utiliza para la valoración de inventarios y para el análisis de resultados, ya que incluye todos los costes incorporables en el producto.

\subsubsection{Características y Críticas}

El \textit{Full-Cost} incorpora la totalidad de los costes fijos (CF) de fabricación en el coste unitario.

Las principales críticas al \textit{Full-Cost} provienen de los defensores del \textit{Direct-Costing}:
\begin{itemize}
    \item \text{Inventarización de CF:} Se critica que se inventarían costes que se tienen que soportar independientemente de la producción (costes ligados al tiempo), lo cual puede llevar a una sobrevaloración de las existencias.
    \item \text{Dependencia del Nivel de Actividad:} El coste unitario es sensible a las fluctuaciones del nivel de producción. Por ejemplo, una disminución en la producción, manteniendo los CF constantes, provoca un aumento artificial del coste unitario, sin que exista una mejora o empeoramiento real en la eficiencia del proceso productivo.
\end{itemize}

\subsubsection{Ajuste: Imputación Racional}

El \text{Método de Imputación Racional} (MIR) es una variante del \textit{Full-Cost} que busca mitigar la dependencia del coste unitario de las fluctuaciones de actividad.

El MIR considera que solo la parte de los costes fijos correspondiente a la \text{capacidad utilizada} (capacidad real/$X_r$) se incorpora al coste del producto, mientras que la parte debida a la \text{capacidad ociosa} (o coste de subactividad) se imputa al resultado del periodo.

Esto se logra multiplicando los costes fijos totales por el \text{coeficiente de imputación racional} ($\frac{X_r}{X_n}$, donde $X_n$ es la capacidad normal).

\begin{ejemplo}
El coste de subactividad se calcula mediante la fórmula: $C_f (1 - \frac{X_r}{X_n})$. Este coste es imputado al resultado del ejercicio.
\end{ejemplo}

\subsubsection{Ampliación: Función de Coste ($CF + CV \cdot X$)}

La función de Coste Total ($C_T$) es crucial para entender el comportamiento de los costes frente al volumen de actividad ($Q$).

$$
C_T = C_F + C_{V_u} \cdot Q
$$
Donde $C_F$ son los Costes Fijos totales, y $C_{V_u} \cdot Q$ son los Costes Variables totales, siendo $C_{V_u}$ el coste variable unitario. Los costes fijos son aquellos que no varían dentro de los límites de la capacidad productiva disponible, mientras que los costes variables cambian proporcionalmente al nivel de actividad. Esta distinción es la base para los modelos de costes parciales.

\subsection{\textit{Direct-Cost}}

El \textit{Direct-Costing} (Coste Variable) clasifica los costes en fijos y variables, e incluye \text{solo los costes variables} en el coste del producto, considerando los costes fijos como costes del periodo.

\subsubsection{Simple: Margen Bruto}

El \textit{Direct-Cost} Simple utiliza la clasificación básica de costes en fijos y variables y calcula el \text{Margen Bruto}.

\begin{definicion}[Margen Bruto]
El Margen Bruto (o de Contribución) se obtiene al restar a los ingresos por ventas los costes variables totales.
\end{definicion}

El Margen Bruto de un producto indica en qué medida está contribuyendo a cubrir los costes fijos (estructura) de la empresa. La estructura de resultados del \textit{Direct-Cost} Simple resta los costes variables de fabricación y comercialización de los ingresos, obteniendo el Margen Bruto, al que se le restan los costes fijos totales para llegar al Resultado Interno.

\subsubsection{Desarrollado: Margen Semibruto}

El \textit{Direct-Cost} Desarrollado (o evolucionado) refina la clasificación al distinguir entre costes fijos propios (directos) y costes fijos comunes (indirectos).

\begin{itemize}
    \item \text{Costes Fijos Propios ($KFP$):} Son directamente atribuibles a un objeto de coste específico (ej., producto o grupo de productos).
    \item \text{Costes Fijos Comunes ($KFC$):} No son atribuibles directamente y se cargan al resultado del periodo.
\end{itemize}
La estructura de resultados del modelo desarrollado calcula el \text{Margen Semibruto} ($MSB$) al restar los costes fijos propios del Margen Bruto. El resultado final se obtiene al restar los costes fijos comunes.

\subsubsection{Ampliación: Análisis CVB y Punto Muerto}

El análisis Coste-Volumen-Beneficio (CVB) se apoya en la función de coste ($C_T = C_F + C_{V_u} \cdot Q$) y en el Margen de Contribución (Margen Bruto).

El Margen Bruto, al ser la diferencia entre ingresos y costes variables, es la clave para determinar la capacidad de la empresa de cubrir sus costes fijos. La información sobre costes variables y fijos es esencial para predecir la conducta de los costes y la rentabilidad ante variaciones de la actividad.

\subsection{Comparativa: \textit{Full-Cost} \textit{vs.} \textit{Direct-Cost}}

La principal diferencia entre ambos modelos reside en el tratamiento contable de los costes fijos de fabricación, lo que genera diferencias en el resultado cuando la producción no es igual a las ventas (variación de inventarios).

\begin{itemize}
    \item \text{Impacto en Inventarios y Resultados:} En el \textit{Full-Cost}, los costes fijos son inventariados, por lo que una mayor producción que ventas resulta en un \text{Resultado Mayor} (los costes fijos se aplazan en el activo) que en el \textit{Direct-Cost}.
    \item \text{En el \textit{Direct-Cost}}, los costes fijos se cargan al periodo en que se incurren. Por lo tanto, si la producción es mayor que las ventas, el resultado del periodo es \text{Menor} que en el \textit{Full-Cost}, ya que todos los costes fijos de ese periodo han sido absorbidos por el resultado.
\end{itemize}

El \textit{Full-Cost} (y el MIR) son adecuados para la valoración de inventarios (a efectos de contabilidad financiera), mientras que el \textit{Direct-Cost} es preferido para la toma de decisiones, ya que se enfoca en la variabilidad de los costes y su relevancia a corto plazo.

Me complace actuar como su profesor experto. A continuación, presento la información solicitada en formato LaTeX, desarrollando una sección extensa de ejemplos y ampliaciones que profundizan en los conceptos de la Contabilidad de Gestión, tal como se extrae de las fuentes proporcionadas.



\section{Ampliaciones y Ejemplos Aplicados}

\subsection{Coste Basado en la Actividad (ABC)}

El sistema de Coste Basado en la Actividad (Activity Based Costing, ABC) constituye un refinamiento de los métodos de costes, diseñado para mejorar la exactitud en la asignación de cargas indirectas. Este método se basa en la premisa de que los costes indirectos incorporables deben asociarse a las \text{actividades que añaden valor} al producto. Posteriormente, el coste se traslada al producto o servicio en función de la cantidad de actividad que consume.

Desde la perspectiva eurocontinental, el ABC puede entenderse como una profundización de los métodos tradicionales, focalizada en la localización y análisis de los costes por cada una de las actividades que se desarrollan en la empresa.

\begin{ejemplo}[Ventaja de los Costes Basados en Causas (ABC)]
Consideremos una empresa, Aeronáuticos San Pablo, S. A., que fabrica un motor de avión con piezas simples y complejas. Si el coste del torno ($80.000 €$) se reparte únicamente por el número de piezas producidas (unidades producidas), cada una de las $20.000$ unidades (10.000 simples y 10.000 complejas) recibe un coste de $4 €$.

Sin embargo, si se tiene en cuenta el \text{factor causal} (el uso de la máquina): las piezas simples consumen $15$ minutos ($0,25$ h) y las complejas consumen $60$ minutos ($1$ h) de torno. El total de horas consumidas es de $12.500$ h ($2.500$ para simples y $10.000$ para complejas).

El coste de la hora de torno es de $6,4 €/h$ ($80.000 € / 12.500$ h).

\begin{itemize}
    \item \text{Coste unitario asignado a Pieza Simple:} $0,25$ h/pieza $\times 6,4 €/$h $= 1,6 €/$pieza.
    \item \text{Coste unitario asignado a Pieza Compleja:} $1$ h/pieza $\times 6,4 €/$h $= 6,4 €/$pieza.
\end{itemize}
La asignación basada en el consumo de tiempo de torno es más justa y lógica, ya que las piezas complejas, al usar más el recurso, absorben un coste cuatro veces superior al de las piezas simples. Esto demuestra cómo el uso de múltiples unidades de obra o bases de actividad (como el tiempo en el torno) puede mejorar significativamente la precisión del coste.
\end{ejemplo}

\subsection{Función de Coste y Análisis de Comportamiento}

La Contabilidad de Gestión requiere el entendimiento del comportamiento de los costes ante variaciones en el volumen de actividad para propósitos de planificación y control. Los costes se clasifican en fijos ($C_F$) y variables ($C_{V_u} \cdot Q$).

La \text{función de Coste Total} ($C_T$) se expresa como:
$$
C_T = C_F + C_{V_u} \cdot Q
$$

En muchas ocasiones, la distinción entre costos fijos y variables no es inmediata, ya que existen \text{costes mixtos o híbridos}. Los costes semivariables poseen una parte fija y otra variable (como el suministro de energía, con una cuota fija y un consumo variable). Para separar estos componentes, se pueden emplear métodos como la gráfica de dispersión o el método de los puntos extremos.

\begin{ejemplo}[Separación de Costes Mixtos: Método de los Puntos Extremos]
Consideremos los costes del lugar de coste de impregnación de la empresa TEJISA, registrados durante 12 meses.

\begin{center}
\begin{tabular}{lrr}
\toprule
\text{Mes} & \text{Costes Totales ($C_T$)} & \text{Actividad ($Q$)} \\
\midrule
Junio de 20X0 (Q Alta) & $990.000 €$ & $25.000 m^2$ \\
Enero de 20X1 (Q Baja) & $645.000 €$ & $10.000 m^2$ \\
\bottomrule
\end{tabular}
\end{center}

El método de los puntos extremos se basa en el principio de que la diferencia total de costes entre el punto de actividad máxima y mínima es explicada únicamente por la variación en el componente variable:

\begin{enumerate}
    \item \text{Cálculo del Coste Variable Unitario ($C_{V_u}$):}
    $$
    C_{V_u} = \frac{\text{Coste Alto} - \text{Coste Bajo}}{\text{Actividad Alta} - \text{Actividad Baja}} = \frac{990.000 € - 645.000 €}{25.000 m^2 - 10.000 m^2} = \frac{345.000 €}{15.000 m^2} = 23 €/m^2
    $$
    \item \text{Cálculo del Coste Fijo ($C_F$):} Se utiliza la función de coste total ($C_T = C_F + C_{V_u} \cdot Q$) en cualquiera de los puntos extremos. Usando el punto de actividad alta:
    $$
    C_F = C_T - (C_{V_u} \cdot Q) = 990.000 € - (23 €/m^2 \cdot 25.000 m^2) = 990.000 € - 575.000 € = 415.000 €
    $$
\end{enumerate}
Así, la función de coste estimada para el centro de impregnación es: $C_T = 415.000 € + 23 € \cdot Q$.
\end{ejemplo}

\begin{anotacion}
Una vez estimado el coste variable unitario de $23 €/m^2$, y sabiendo que cada prenda requiere, por término medio, $2 m^2$ de tejido, el coste variable de impregnación por prenda es de $46 €/prenda$ ($23 €/m^2 \times 2 m^2$).
\end{anotacion}

\subsection{Comparativa de Modelos de Resultados: Full-Cost \textit{vs.} Direct-Cost}

La principal diferencia entre los modelos de \textit{Full-Cost} (Costes Completos) y \textit{Direct-Cost} (Costes Variables) radica en el tratamiento de los Costes Fijos de Fabricación ($K_F$). El \textit{Full-Cost} los considera \text{costes inventariables} (costes del producto), mientras que el \textit{Direct-Cost} los trata como \text{costes del período} (cargas que se cargan al resultado en el ejercicio en que se incurren). Esta divergencia genera resultados distintos siempre que la producción ($A$) y la colocación o venta ($A_v$) no coinciden ($A \neq A_v$).

A continuación, se presentan tres escenarios consecutivos (P. X, P. X+1, P. X+2) basados en el mismo conjunto de costes: $K_T = 70.000 €$, $K_F = 40.000 €$, $K_V = 30.000 €$. Producción constante $A=1.000$ u.c., y Precio de Venta $p_v = 125 €/$u.c..

\subsubsection{Escenario 1: Producción = Ventas (P. X)}

No hay variación de existencias ($A = A_v = 1.000$ u.c.).

\begin{enumerate}
    \item \text{Coste Unitario de Producción:}
    *   $C_u (\text{Full-Cost}) = K_T / A = 70.000 € / 1.000 \text{ u.c.} = 70 €/\text{u.c.}$
    *   $C_u (\text{Direct-Cost}) = K_V / A = 30.000 € / 1.000 \text{ u.c.} = 30 €/\text{u.c.}$
\end{enumerate}

\begin{center}
\begin{tabular}{lcc}
\toprule
\text{Cálculo de Resultados} & \text{Full-Cost Radical} & \text{Direct-Cost Simple} \\
\midrule
Ingresos por Ventas ($1.000 \times 125 €$) & $125.000 €$ & $125.000 €$ \\
(-) Coste Colocación/Ventas (CF y CV) & $70.000 €$ & $30.000 €$ \\
\midrule
\text{Margen Bruto/Industrial} & $55.000 €$ & $95.000 €$ \\
(-) Costes Fijos del Periodo & $0 €$ & $40.000 €$ \\
\midrule
\text{Resultado Interno (RI)} & $\mathbf{55.000 €}$ & $\mathbf{55.000 €}$ \\
\bottomrule
\end{tabular}
\end{center}
\text{Conclusión:} Cuando las ventas igualan a la producción, el Resultado Interno (RI) es idéntico en ambos modelos.

\subsubsection{Escenario 2: Producción > Ventas (P. X+1)}

Se produce acumulación de inventarios: $A=1.000$ u.c. y $A_v=700$ u.c. ($A_f = 300$ u.c.).

\begin{center}
\begin{tabular}{lcc}
\toprule
\text{Cálculo de Resultados} & \text{Full-Cost Radical} & \text{Direct-Cost Simple} \\
\midrule
Ingresos por Ventas ($700 \times 125 €$) & $87.500 €$ & $87.500 €$ \\
(-) Coste Colocación/Ventas ($700 \times C_u$) & $49.000 €$ ($700 \times 70 €$) & $21.000 €$ ($700 \times 30 €$) \\
\midrule
\text{Margen Bruto/Contribución} & $38.500 €$ & $66.500 €$ \\
(-) Costes Fijos del Periodo (KF) & $0 €$ & $40.000 €$ \\
\midrule
\text{Resultado Interno (RI)} & $\mathbf{38.500 €}$ & $\mathbf{26.500 €}$ \\
\bottomrule
\end{tabular}
\end{center}
\text{Conclusión:} En el \textit{Full-Cost}, los costes fijos correspondientes a las $300$ u.c. no vendidas ($300 \times 40 €/\text{u.c.} = 12.000 €$) se aplazan y permanecen \text{inventariados} en el activo. En el \textit{Direct-Cost}, los $40.000 €$ de costes fijos se cargan inmediatamente al resultado. Por lo tanto, $\mathbf{RI (Full-Cost) > RI (Direct-Cost)}$ por la diferencia de costes fijos inventariados ($38.500 € - 26.500 € = 12.000 €$).

\subsubsection{Escenario 3: Producción < Ventas (P. X+2)}

Se utiliza el stock inicial ($A_p=300$ u.c. a coste del P. X+1) y la producción actual ($A=1.000$ u.c.) para vender $A_v=1.300$ u.c..

\begin{center}
\begin{tabular}{lcc}
\toprule
\text{Cálculo de Resultados} & \text{Full-Cost Radical} & \text{Direct-Cost Simple} \\
\midrule
Ingresos por Ventas ($1.300 \times 125 €$) & $162.500 €$ & $162.500 €$ \\
(-) Coste Colocación/Ventas ($1.300 \times C_u$) & $91.000 €$ ($1.300 \times 70 €$) & $39.000 €$ ($1.300 \times 30 €$) \\
\midrule
\text{Margen Bruto/Contribución} & $71.500 €$ & $123.500 €$ \\
(-) Costes Fijos del Periodo (KF) & $0 €$ & $40.000 €$ \\
\midrule
\text{Resultado Interno (RI)} & $\mathbf{71.500 €}$ & $\mathbf{83.500 €}$ \\
\bottomrule
\end{tabular}
\end{center}
\text{Conclusión:} En este periodo, el \textit{Full-Cost} está cargando al resultado los $12.000 €$ de costes fijos que se inventariaron en el P. X+1. Como el \textit{Direct-Cost} ya cargó esos costes en el periodo anterior, en el P. X+2 presenta un \text{Resultado Mayor}.

\subsection{Ajuste al \textit{Full-Cost}: Imputación Racional y Coste de Subactividad}

El método de \textit{Full-Cost} es criticado porque el Coste Unitario es sensible a las fluctuaciones del nivel de actividad: si la producción cae, el coste unitario aumenta artificialmente, ya que los Costes Fijos Totales ($C_F$) se reparten entre menos unidades.

El \text{Método de Imputación Racional} (MIR) soluciona esto incorporando al coste del producto solo la porción de los costes fijos asociada a la \text{capacidad utilizada} ($X_r$), transfiriendo el coste de la \text{capacidad ociosa} (o coste de subactividad) directamente al resultado del periodo.

\begin{ejemplo}[Cálculo de Coste de Subactividad - Adaptado del Cap. 5]
La empresa Box fabrica cajas. Sus datos son: Costes Fijos ($C_F$) de $40.000 €$ y Coste Variable Unitario ($C_{V_u}$) de $200 €$. La \text{Actividad Normal} ($X_n$) es de $8.000$ cajas.

Supongamos que en Enero la \text{Actividad Real} ($X_r$) fue de $4.000$ cajas.

\begin{enumerate}
    \item \text{Coste Total según Full-Cost Completo (MCC):}
    $$
    C_T = 40.000 € + (200 €/\text{caja} \cdot 4.000 \text{ cajas}) = 40.000 € + 800.000 € = 840.000 €
    $$
    $$
    C_u (\text{MCC}) = 840.000 € / 4.000 \text{ cajas} = 210 €/\text{caja}
    $$
    \item \text{Cálculo de Subactividad (Coste Fijo No Imputado) según MIR:}
    El \text{coeficiente de imputación racional} es: $C I R = \frac{X_r}{X_n} = \frac{4.000}{8.000} = 0,5$.
    $$
    \text{Coste Subactividad} = C_F \cdot (1 - C I R) = 40.000 € \cdot (1 - 0,5) = 20.000 €
    $$
    \item \text{Coste Total de Producción según MIR:}
    El coste imputado al producto incluye el Coste Variable Total y la porción de Coste Fijo utilizado ($C_F \cdot CIR$):
    $$
    C_T (\text{MIR}) = C_{V} + (C_F \cdot C I R) = 800.000 € + (40.000 € \cdot 0,5) = 800.000 € + 20.000 € = 820.000 €
    $$
    $$
    C_u (\text{MIR}) = 820.000 € / 4.000 \text{ cajas} = 205 €/\text{caja}
    $$
\end{enumerate}
\text{Conclusión:} El resultado es distinto ($840.000 €$ vs $820.000 €$ para la producción total). El MIR resulta en un coste unitario de $205 €$, que es más estable, ya que el coste de subactividad de $20.000 €$ es cargado directamente como gasto del periodo, poniendo de manifiesto la ineficiencia por no utilizar la capacidad disponible.
\end{ejemplo}

\subsection{Modelos de Asignación de Costes Auxiliares (Reparto Secundario)}

El \text{Reparto Secundario} (R2) es la fase del modelo orgánico donde se transfieren los costes de las secciones auxiliares (o de apoyo) a las secciones principales (o de producción), ya que solo las principales añaden valor observable al producto. Existen tres métodos principales para realizar esta asignación, diferenciándose en el tratamiento que dan a las prestaciones mutuas (interrelaciones) entre los centros auxiliares:

\begin{ejemplo}[Métodos de Reparto Secundario - Adaptado de Cap. 4]
Supongamos que la empresa tiene dos centros auxiliares: Energía ($250 €$ de Reparto Primario) y Mantenimiento ($160 €$ de Reparto Primario). Y dos centros principales: Molido ($100 €$) y Ensamblado ($60 €$). Energía presta servicios a Mantenimiento (200 Kw/h) y Mantenimiento presta servicios a Energía (1 h).

\subsubsection{Método Directo}
Asume que los centros auxiliares solo trabajan para los centros principales, ignorando las prestaciones entre sí.

\begin{itemize}
    \item \text{Coste Asignado a Molido:} $187,5 €$ (por Energía) $+ 80 €$ (por Mantenimiento) $= 267,5 €$.
    \item \text{Coste Asignado a Ensamblado:} $62,5 €$ (por Energía) $+ 80 €$ (por Mantenimiento) $= 142,5 €$.
    \item \text{Total Coste Molido:} $100 €$ (propio) $+ 267,5 €$ (asignado) $= 367,5 €$.
\end{itemize}
\text{Inconveniente:} Ignora la interrelación entre Energía y Mantenimiento, lo que provoca inexactitud. Sin embargo, su ventaja es la simplicidad.

\subsubsection{Método Secuencial}
Reconoce \text{algunas interrelaciones} entre auxiliares, siguiendo un orden preestablecido (normalmente por el mayor coste primario).

\begin{itemize}
    \item \text{Coste Asignado a Molido:} $150 €$ (por Energía) $+ 105 €$ (por Mantenimiento) $= 255 €$.
    \item \text{Coste Asignado a Ensamblado:} $50 €$ (por Energía) $+ 105 €$ (por Mantenimiento) $= 155 €$.
    \item \text{Total Coste Molido:} $100 €$ (propio) $+ 255 €$ (asignado) $= 355 €$.
\end{itemize}
\text{Ventaja:} Es más exacto que el directo, ya que tiene en cuenta que Energía sirve a Mantenimiento, pero aún ignora las prestaciones inversas (Mantenimiento a Energía).

\subsubsection{Método Recíproco}
Reconoce \text{todas las prestaciones mutuas}, incluyendo las autoprestaciones si existen, requiriendo la resolución de un sistema de ecuaciones. Es el método más exacto, aunque es el más complejo.

\begin{itemize}
    \item \text{Coste Asignado a Molido:} $162,9 €$ (por Energía) $+ 96,4 €$ (por Mantenimiento) $= 259,3 €$.
    \item \text{Coste Asignado a Ensamblado:} $54,3 €$ (por Energía) $+ 96,4 €$ (por Mantenimiento) $= 150,7 €$.
    \item \text{Total Coste Molido:} $100 €$ (propio) $+ 259,3 €$ (asignado) $= 359,3 €$.
\end{itemize}
\text{Observación:} Los resultados varían significativamente según el método, lo que puede causar desacuerdos entre los responsables de los centros (p. ej., el responsable de Molido prefiere el Secuencial, mientras que el de Ensamblado prefiere el Directo o Recíproco). La elección depende de ponderar la exactitud frente al coste de implementación y la comprensibilidad del sistema.
\end{ejemplo}
