% ================================================================
% CAPÍTULO 2: CONCEPTOS BÁSICOS
% ================================================================


\chapter{Conceptos Básicos}

\section{La noción de coste: concepto y clases}

El concepto de \textbf{coste} es fundamental en la contabilidad de gestión. Se define como el consumo de medios o recursos, expresado en términos monetarios, sacrificados para alcanzar un objetivo específico. Dicho de otro modo, es el consumo valorado en dinero de bienes y servicios destinados a la producción que constituye el objetivo de la empresa. Para que un consumo se considere coste, debe contribuir a la realización de la actividad productiva; de lo contrario, se tratará de un gasto que no se incorpora al coste de producción.

El coste es siempre relativo a un \textbf{objeto de coste}, que es cualquier elemento para el cual se desea una medición de su coste, como un producto, un servicio, un proyecto, un departamento o una actividad. Por ejemplo, el objeto de coste podría ser un bote de pintura, una línea de atención al cliente o un proyecto de I+D.

Es importante destacar la \textbf{relatividad de la magnitud coste}, que se debe a dos factores principales:
\begin{itemize}
    \item La indeterminación asociada a la medida y valoración de los consumos.
    \item La indeterminación inherente a la asignación de los costes a los centros de actividad y a los productos.
\end{itemize}

Dada la diversidad de decisiones que deben tomar los responsables en una empresa, los costes se clasifican de diferentes maneras según la necesidad del usuario de la información. A continuación, se presentan las tipologías de costes más relevantes:

\subsection{Clasificación según la naturaleza}
Esta clasificación agrupa los costes según el tipo de recurso consumido. Se distinguen, entre otros:
\begin{itemize}
    \item \textbf{Coste de materiales}: Valor de los consumos de materiales que forman parte de los productos.
    \item \textbf{Coste de personal}: Salarios y cargas sociales de los trabajadores.
    \item \textbf{Coste de suministros}: Como electricidad o agua.
    \item \textbf{Coste de servicios exteriores}: Servicios prestados por otras empresas.
    \item \textbf{Coste de amortización}: Depreciación del inmovilizado por su participación en el proceso productivo.
    \item \textbf{Coste financiero}: Derivado de la financiación de la empresa.
\end{itemize}

\subsection{Clasificación según la relación con el objeto de coste}
Este criterio diferencia los costes en función de su capacidad para identificarlos de forma inequívoca con un objeto de coste determinado.
\begin{itemize}
    \item \textbf{Costes directos}: Son aquellos que pueden identificarse de forma inequívoca y económicamente factible con un objeto de coste. Por ejemplo, la materia prima consumida en un producto. El proceso de aplicar estas cargas directas se denomina \textbf{afectación}.
    \item \textbf{Costes indirectos}: Son aquellos que no pueden identificarse de forma fácil e inequívoca con un objeto de coste concreto, por lo que precisan de un criterio de reparto para su asignación. Un ejemplo sería la amortización del edificio donde se ubica la empresa. El proceso de aplicar estas cargas indirectas se denomina \textbf{imputación}.
\end{itemize}

\subsection{Otras clasificaciones}
Existen otras tipologías de costes que son igualmente relevantes para la gestión empresarial:
\begin{itemize}
    \item \textbf{Según su relación con el cálculo del resultado}:
    \begin{itemize}
        \item \textbf{Costes del producto (o inventariables)}: Se incorporan al resultado en el período en que se venden los productos. Son aquellos asignados a los bienes inventariables, como el coste de producción.
        \item \textbf{Costes del período (o no inventariables)}: Se incorporan al resultado en el período en que se produce el coste, independientemente de la venta del producto. Son costes asociados a un período de tiempo, como los gastos de administración o comerciales.
    \end{itemize}
    \item \textbf{Según el momento de su cálculo}:
    \begin{itemize}
        \item \textbf{Costes históricos o reales}: Se calculan a posteriori, una vez que la actividad ha tenido lugar.
        \item \textbf{Costes predeterminados}: Se calculan a priori, antes de que se produzca la actividad. Incluyen los costes \textit{estándares} (lo que debería costar en condiciones normales) y los \textit{presupuestados} (lo que se espera que cueste).
    \end{itemize}
\end{itemize}

\section{Principales funciones de coste: costes fijos y costes variables}

Una de las clasificaciones más importantes en la contabilidad de gestión es la que atiende al comportamiento de los costes frente a variaciones en el \textbf{volumen de actividad}. Esta clasificación distingue principalmente entre costes fijos y costes variables.

No obstante, es crucial realizar algunas precisiones sobre esta clasificación:
\begin{itemize}
    \item Se debe precisar la \textbf{variable independiente} que sirve de base para establecer la relación funcional (p. ej., unidades producidas, horas-máquina).
    \item La clasificación depende del \textbf{plano de variación} o \textbf{escala relevante}, que es el intervalo de actividad para el cual se mantiene la relación definida. A largo plazo, todos los costes son variables.
    \item La fijeza o variabilidad puede cambiar según el nivel de análisis. Un coste fijo a nivel global de empresa puede ser variable para un centro de actividad particular.
    \item Las \textbf{decisiones empresariales} condicionan la fijeza o variabilidad de los costes en un período.
    \item La fijeza de un coste en un período no implica que se mantenga constante en períodos sucesivos.
\end{itemize}

\subsection{Costes fijos}
Son aquellos que no fluctúan ante variaciones en el volumen de producción, dentro de los límites definidos por la capacidad productiva disponible. También se les denomina \textbf{costes de estructura}, ya que están asociados a disponer de una cierta capacidad.

El coste fijo total permanece constante en la escala relevante, mientras que el \textbf{coste fijo unitario es decreciente} a medida que aumenta el nivel de actividad.

\subsection{Costes variables}
Son aquellos que varían de forma directamente proporcional al nivel de actividad. Se les conoce también como \textbf{costes operacionales}, ya que están ligados a la utilización de la capacidad. El coste variable total aumenta con la actividad, pero el \textbf{coste variable unitario permanece constante} en la escala relevante.

\subsection{Costes mixtos o híbridos}
Existen costes que no se comportan de forma puramente fija o variable.
\begin{itemize}
    \item \textbf{Costes semivariables}: Tienen un componente fijo y otro variable. Un ejemplo es la factura de teléfono, que suele incluir una cuota fija y un coste variable según el consumo.
    \item \textbf{Costes semifijos (o en escalera)}: Permanecen fijos para ciertos intervalos de actividad y dan un "salto" cuantitativo al cambiar de intervalo. Por ejemplo, el coste de los monitores en un curso de natación, donde se necesita un monitor adicional a partir de un cierto número de alumnos. Si al disminuir la actividad el coste no regresa a su nivel anterior, se produce el fenómeno de la \textbf{histéresis de los costes}.
\end{itemize}

\section{Costes necesarios versus costes no necesarios: costes de la actividad y costes de la subactividad}

Dentro de la estructura de costes de una empresa, es posible distinguir entre los costes que son estrictamente necesarios para la producción programada y aquellos que, aunque soportados por la empresa, no lo son. Esta distinción nos lleva al concepto de \textbf{costes de la subactividad} o \textbf{costes por exceso de capacidad}.

Se definen como los costes asociados a la mano de obra y al equipo productivo que, existiendo al inicio de un período, deben permanecer en la empresa durante el mismo, aunque \textbf{no sean necesarios para alcanzar el programa de producción}.

Las causas de la existencia de estos costes no necesarios pueden ser diversas:
\begin{itemize}
    \item \textbf{Decisiones adoptadas en el pasado} que limitan el campo de actuación presente del empresario.
    \item La \textbf{consideración del futuro}, que obliga a tomar medidas en el presente (como mantener cierta capacidad ociosa) que pueden parecer innecesarias desde una perspectiva momentánea.
\end{itemize}

Estos costes de subactividad, derivados de la infrautilización de la capacidad productiva normal, no deben formar parte del coste de producción, sino que deben ser imputados directamente al resultado del ejercicio. Su cálculo evita que los efectos de una disminución del nivel de actividad real afecten negativamente al coste unitario del producto.

\section{Las funciones de coste en Economía de la empresa: costes de estructura o en estado parado y costes de puesta en marcha}

Desde la perspectiva de la Economía de la Empresa, los costes fijos se pueden desagregar en dos categorías fundamentales que responden a diferentes estados de la actividad productiva. Esta distinción es especialmente útil para comprender la estructura de costes fijos totales.

\subsection{Costes de estructura o en estado parado}
Son aquellos costes que la empresa soporta \textbf{incluso si no se desarrolla ningún tipo de actividad productiva}. Estos costes son inherentes a la mera existencia de la estructura de la empresa y a su mantenimiento. Un ejemplo claro es la amortización del edificio donde se instala la empresa, que se genera independientemente del nivel de producción (incluso si es cero).

\subsection{Costes de puesta en marcha o de preparación de la producción}
Estos costes se originan como consecuencia de la \textbf{adecuación y preparación de la empresa para iniciar el proceso de fabricación}, aunque finalmente no se llegue a producir ninguna unidad. Son los costes necesarios para pasar del "estado parado" al estado "listo para producir". Por ejemplo, el coste del engrasado y la puesta a punto de una máquina para que pueda empezar a fabricar representa un coste de puesta en marcha.

En resumen, los \textbf{costes fijos totales} son la suma de los costes de estructura y los costes de puesta en marcha. Los costes de estructura existen con producción cero, mientras que los de puesta en marcha solo aparecen cuando se toma la decisión de producir, aunque sea una sola unidad.

\section{Costes variables: sus clases}
Los costes variables, definidos como aquellos que cambian en proporción directa al nivel de actividad, pueden mostrar diferentes patrones de comportamiento. Su análisis detallado permite una mejor comprensión de la estructura de costes de la empresa.

\subsection{Costes proporcionales}
Son aquellos que varían de forma \textbf{directa y estrictamente proporcional} a las variaciones de la actividad. Su representación gráfica es una línea recta que parte del origen. El ejemplo más característico es el coste de la materia prima: si para fabricar una unidad se necesita una cantidad X de material, para fabricar dos unidades se necesitará el doble. Su coste unitario es constante.

\subsection{Costes progresivos}
Varían de forma \textbf{directa y más que proporcional} a las variaciones de la actividad. Esto significa que, a medida que aumenta la producción, el coste aumenta a un ritmo creciente. Un ejemplo común es el coste de la mano de obra cuando se requieren horas extraordinarias, cuyo precio por hora es superior al de las horas normales. El coste unitario de un coste progresivo es creciente.

\subsection{Costes degresivos}
Varían de forma \textbf{directa pero menos que proporcional} a las variaciones de la actividad. El coste total aumenta a un ritmo decreciente. Esto puede ocurrir, por ejemplo, con costes que se benefician de tarifas degresivas o descuentos por volumen, como en el caso de la reprografía o ciertos suministros. El coste unitario de un coste degresivo es decreciente.

\subsection{Costes regresivos}
Son aquellos que varían de \textbf{forma inversa} a las variaciones de la actividad. Un ejemplo ilustrativo es el coste de calefacción en un teatro: a mayor número de espectadores (mayor actividad), menor será el coste de calefacción necesario debido al calor corporal generado por el público.



% EXTRA

\chapter{Análisis y estimación de las funciones de coste}

\section{Hipotesis de linealidad}

La \textbf{hipótesis de linealidad} es un postulado fundamental en la modelización contable de los costes de una empresa. Esta hipótesis establece una relación lineal entre el coste total y el volumen de actividad dentro de un marco específico, conocido como el \textbf{intervalo relevante} o \textbf{escala relevante}.

\subsection{El Intervalo Relevante}

El concepto de escala relevante representa el intervalo de actividad en el que, a pesar de que se incremente el volumen, se mantienen constantes las características de la definición de los costes fijos, es decir, su volumen permanece constante. Dentro de este intervalo, es posible realizar un análisis de los costes en función del volumen con información homogénea.

\subsection{Representación del Modelo Contable}

Gráficamente, la hipótesis de linealidad permite definir los costes totales (modelo contable) y los ingresos totales (modelo contable) como líneas rectas que interactúan con los costes fijos (modelo contable) y los costes fijos (modelo económico).

\begin{tcolorbox}[title=\textbf{Implicaciones del Modelo Lineal}]
El modelo lineal permite determinar puntos clave como el \textbf{Umbral de rentabilidad}. Aunque en la realidad económica el comportamiento del coste total puede ser complejo (modelo económico), a efectos contables, se simplifica a un comportamiento lineal dentro de dicho intervalo para facilitar el análisis y la toma de decisiones.
\end{tcolorbox}

\section{Métodos para analizar el comportamiento de los costes}

Para el análisis del comportamiento de los costes, particularmente para separar los costes mixtos o híbridos (aquellos que contienen componentes fijos y variables), se emplean diversas metodologías. Los métodos principales identificados son:

\begin{enumerate}
    \item \textbf{Método gráfico:} Consiste en representar gráficamente los datos estadísticos de volumen de producción y coste (nube de puntos). Posteriormente, se traza manualmente una recta que mejor se ajuste a estos puntos y que corte el eje de ordenadas "Y".
    \item \textbf{Método de los valores extremos o puntos extremos:} Utiliza únicamente los datos correspondientes a los niveles de actividad más alto y más bajo para estimar la función de coste.
    \item \textbf{Método de los mínimos cuadrados:} Es una técnica estadística que se utiliza para encontrar la línea de mejor ajuste a los datos históricos de costes y volumen de actividad.
\end{enumerate}

\section{Funciones a usar en Excel para la estimación}

El paquete de software \textit{Excel} provee funciones específicas para la aplicación de métodos estadísticos en la estimación de los componentes de las funciones de coste, particularmente para el método de los mínimos cuadrados (regresión lineal).

\subsection{Cálculo de Coeficientes de la Función Lineal}
Para calcular los valores estimados de la pendiente ($b$, coste variable unitario) y la ordenada en el origen ($a$, coste fijo) de la función de coste lineal ($\hat{y} = a + bX$), se utiliza la función matricial:
\begin{lstlisting}[style=elegant, basicstyle=\ttfamily\small]
{ESTIMACION.LINEAL(variable dependiente; variable independiente)}
\end{lstlisting}
Es imperativo recordar que, al ser una función matricial, su introducción en \textit{Excel} debe validarse presionando \textbf{Ctrl + Mayúscula + Enter}.

\subsection{Cálculo del Coeficiente de Determinación}
Para \textbf{enjuiciar la bondad del ajuste} de la ecuación obtenida, se calcula el coeficiente de determinación ($R^2$), que indica la proporción de la varianza del coste total que es explicada por la variación en el volumen de actividad. La función correspondiente es:
\begin{lstlisting}[style=elegant, basicstyle=\ttfamily\small]
COEFICIENTE.R2(variable dependiente; variable independiente)
\end{lstlisting}

\section{Enfoque general para estimar las funciones de coste}

El enfoque primordial en la estimación de funciones de coste radica en comprender el comportamiento de los costes respecto a las variaciones en el \textbf{volumen de actividad} de la empresa. Esto es esencial para la toma de decisiones basada en la información generada por la contabilidad de costes.

\subsection{Necesidad de la Estimación}
Mientras que los costes puramente fijos (que no fluctúan ante cambios en el volumen de producción dentro de límites definidos) y los costes puramente variables (aquellos que son proporcionales a las variaciones en la actividad) son fáciles de identificar en su comportamiento, los \textbf{costes mixtos o híbridos} presentan componentes fijos y variables. El objetivo de la estimación es, por lo tanto, segregar estos costes mixtos en sus elementos fijo ($F$) y variable ($V$) para formular la función del coste total ($CT$) en función del volumen de actividad ($X$):

$$CT = F + V \cdot X$$

\subsection{Consideraciones Clave}
Al establecer la relación funcional, es necesario precisar la \textbf{variable independiente} (el indicador de actividad) que servirá de base para establecer dicha relación. La estimación obtenida será válida únicamente dentro del \textbf{intervalo relevante} de actividad.

\section{Pasos en la estimación de una función de coste}

Aunque la metodología detallada varía según el método elegido (gráfico, valores extremos o mínimos cuadrados), el proceso general para estimar una función de coste lineal a partir de datos históricos implica pasos secuenciales:

\begin{enumerate}
    \item \textbf{Recolección y Organización de Datos Históricos:} Obtener una serie de datos de pares de observaciones que relacionen el \textbf{volumen de actividad} ($X$) con los \textbf{costes totales incurridos} ($Y$) durante periodos pasados.
    \item \textbf{Análisis Gráfico (Método de Dispersión):} Representar gráficamente los datos (pares $X, Y$) mediante puntos. Este paso inicial permite visualizar la \textbf{nube de puntos} y obtener una primera idea sobre el valor de los costes fijos (donde la recta corta el eje de ordenadas).
    \item \textbf{Selección y Aplicación del Método de Separación:} Utilizar un método analítico (como el de valores extremos o el de mínimos cuadrados) para determinar con precisión el coste fijo ($F$) y el coste variable unitario ($V$).
    \item \textbf{Formulación de la Función de Coste:} Definir la ecuación de la recta de costes totales ($CT$) con los parámetros estimados:
    $$CT = F + V \cdot X$$
\end{enumerate}

\section{Método de los valores extremos}

El \textbf{Método de los valores extremos} (o \textbf{Puntos Extremos}) es uno de los métodos utilizados para analizar el comportamiento de los costes. Este método es una alternativa al método de los mínimos cuadrados y al método gráfico para estimar los costes variables y fijos de una función.

\subsection{Criterio de Aplicación}

Este método se basa en el uso exclusivo de los datos de costes correspondientes a los \textbf{niveles de actividad más alto y más bajo}. Se ignora la dispersión del resto de los datos intermedios.

\subsection{Mecanismo de Cálculo (Ejemplo Implícito)}

Para determinar el coste variable unitario y el coste fijo, se identifican los dos puntos extremos de la actividad. El coste variable unitario se calcula en función de la diferencia de costes entre el punto de actividad máxima y el punto de actividad mínima, dividida por la diferencia entre los volúmenes de actividad correspondientes.

Por ejemplo, en la empresa TEJISA, se utilizó el método de los puntos extremos para estimar el coste variable de impregnación basándose en el volumen de tejido utilizado en diferentes meses. Asimismo, en el Caso Práctico 4, se solicita estimar los costes de un lugar de transformación para meses futuros usando este método. La ecuación de costes obtenida por este método puede ser comparada con la obtenida por el método de los mínimos cuadrados para enjuiciar las diferencias en la estimación.

\begin{tcolorbox}[title=\textbf{Nota Aclaratoria}]
A diferencia del método gráfico, donde la estimación depende de la habilidad de la persona que traza la línea, el método de los valores extremos es rápido y sencillo, ya que solo requiere la identificación de dos puntos de datos para la estimación de la función de coste.
\end{tcolorbox}

