\chapter{Relaciones de Ejercicios}

\section{Relación 2}

\begin{ejercicio}
TEMA 2 – CASO PRÁCTICO 1

NORASCA, S.A. es una empresa que manufactura una sola línea de productos consistente en gel de afeitar y loción para después del afeitado. La compañía desarrolla su Contabilidad interna sobre una base mensual, expresando su volumen de actividad en horas reales de mano de obra directa.

Durante los últimos doce meses, los costes indirectos de fabricación mensuales, junto con las horas de mano de obra directa trabajadas, fueron los siguientes:

\begin{table}[H]
    \centering
    \begin{tabular}{|c|c|c|}
        \hline
        MES & C.I.F. (€) & HORAS M.O.D. (h) \\
        \hline
        1 & 190,000 & 2,000 \\
        2 & 200,000 & 2,500 \\
        3 & 220,000 & 3,000 \\
        4 & 230,000 & 3,300 \\
        5 & 250,000 & 4,100 \\
        6 & 260,000 & 5,000 \\
        7 & 290,000 & 6,000 \\
        8 & 270,000 & 5,000 \\
        9 & 250,000 & 4,500 \\
        10 & 240,000 & 4,000 \\
        11 & 210,000 & 3,500 \\
        12 & 205,000 & 3,000 \\
        \hline
    \end{tabular}
\end{table}

SE PIDE:
\begin{enumerate}
    \item Determinar la ecuación que revele el comportamiento de los costes indirectos de fabricación, utilizando el método de los mínimos cuadrados.
    \item Enjuiciar la bondad del ajuste.
    \item Comparar la ecuación obtenida en el punto primero con los resultados que se obtendrían si se utilizara el método de los puntos extremos.
\end{enumerate}
\end{ejercicio}

\begin{solucion}
Los datos proporcionados se analizan utilizando el método de los mínimos cuadrados para ajustar una ecuación lineal de la forma \( \text{CIF} = b + m \cdot \text{HMOD} \), donde \( b \) es el término independiente y \( m \) es la pendiente.

Datos iniciales:
\begin{itemize}
    \item Número de observaciones (\( n \)): 12
    \item Sumas necesarias:
    \begin{align*}
        \sum x &= 45,900 \quad (\text{HMOD}) \\
        \sum y &= 2,815,000 \quad (\text{CIF}) \\
        \sum xy &= 11,149,000,000 \\
        \sum x^2 &= 190,450,000
    \end{align*}
    \item Promedios:
    \begin{align*}
        \bar{x} &= 3,825.00 \\
        \bar{y} &= 234,583.33
    \end{align*}
\end{itemize}

Cálculo de la pendiente (\( m \)) y el término independiente (\( b \)):
\[
m = \frac{n \sum xy - (\sum x)(\sum y)}{n \sum x^2 - (\sum x)^2}
\]
\[
m = \frac{12 \cdot 11,149,000,000 - 45,900 \cdot 2,815,000}{12 \cdot 190,450,000 - (45,900)^2} = 25.64
\]
\[
b = \frac{\sum y - m \sum x}{n}
\]
\[
b = \frac{2,815,000 - 25.64 \cdot 45,900}{12} = 136,500.64
\]

La ecuación ajustada es:
\[
\text{CIF} = 136,500.64 + 25.64 \cdot \text{HMOD}
\]

Bondad del ajuste:
\begin{itemize}
    \item Suma de residuos al cuadrado (\( SS_{\text{res}} \)): 587,084,943.17
    \item Suma total de cuadrados (\( SS_{\text{tot}} \)): 10,372,916,666.67
    \item Coeficiente de determinación (\( R^2 \)):
    \[
    R^2 = 1 - \frac{SS_{\text{res}}}{SS_{\text{tot}}} = 0.943
    \]
\end{itemize}

El valor de \( R^2 \) indica que el modelo explica el 94.3\% de la variabilidad en los costes indirectos de fabricación, lo que sugiere un buen ajuste.

Comparación con el método de los puntos extremos:
\begin{itemize}
    \item Seleccionando los puntos extremos:
    \begin{align*}
        \text{Punto 1: } & (2,000, 190,000) \\
        \text{Punto 2: } & (6,000, 290,000)
    \end{align*}
    \item Pendiente (\( m \)):
    \[
    m = \frac{290,000 - 190,000}{6,000 - 2,000} = 25.00
    \]
    \item Término independiente (\( b \)):
    \[
    b = 190,000 - 25.00 \cdot 2,000 = 140,000
    \]
    \item Ecuación obtenida:
    \[
    \text{CIF} = 140,000 + 25.00 \cdot \text{HMOD}
    \]
\end{itemize}

Conclusión:
El método de los mínimos cuadrados proporciona una ecuación más precisa (\( \text{CIF} = 136,500.64 + 25.64 \cdot \text{HMOD} \)) en comparación con el método de los puntos extremos (\( \text{CIF} = 140,000 + 25.00 \cdot \text{HMOD} \)), debido a que utiliza toda la información disponible y minimiza los errores cuadráticos.
\end{solucion}


























\section{Relación 3}


% =================================================================
% EJERCICIO 1
% =================================================================

\begin{ejercicio}
TEMA 3 – CASO PRÁCTICO 1

GRUNDASA es una empresa industrial que durante algunos años había fabricado y vendido cinco productos diferentes. En 20X1, las tareas contables se confiaron a un despacho profesional cuyos titulares habían ofrecido sus servicios a la empresa. A comienzos de 20X2, la gerencia recibió el siguiente Estado de Pérdidas y Ganancias:

AÑO 20X1

\begin{tabular}{lrrrrrr}
PRODUCTOS & 1 & 2 & 3 & 4 & 5 & TOTAL\\ \hline
+ Ingresos & 110.000 € & 90.000 € & 95.000 € & 105.000 € & 100.000 € & 500.000 €\\
- Costes & (74.000 €) & (76.000 €) & (83.000 €) & (97.000 €) & (150.000 €) & (480.000 €)*\\
= Resultado & 36.000 € & 14.000 € & 12.000 € & 8.000 € & (50.000 €) & 20.000 €
\end{tabular}

* Los costes fijos, por importe de 200.000 €, se asignan a los productos en función de los ingresos por ventas.

Por la naturaleza de los productos con los que operaba, la empresa no podía esperar un aumento de sus ventas a medio plazo. En un intento de mejorar los resultados y fortalecer la empresa, la gerencia decidió abandonar la fabricación del producto 5, que había arrojado una pérdida de 50.000 € en 20X1.

La gerencia se sintió satisfecha con su decisión cuando observó el Estado de Pérdidas y Ganancias de 20X2. Los beneficios de la empresa habían aumentado a 30.000 € después de abandonar el producto 5. No obstante, el producto 4 mostraba una pérdida de 2.500 €.

AÑO 20X2

\begin{tabular}{lrrrrr}
PRODUCTOS & 1 & 2 & 3 & 4 & TOTAL\\ \hline
+ Ingresos & 110.000 € & 90.000 € & 95.000 € & 105.000 € & 400.000 €\\
- Costes & (85.000 €) & (85.000 €) & (92.500 €) & (107.500 €) & (370.000 €)*\\
= Resultado & 25.000 € & 5.000 € & 2.500 € & (2.500 €) & 30.000 €
\end{tabular}

* Los costes fijos, por importe de 200.000 €, se asignan a los productos en función de los ingresos por ventas.

A comienzos de 20X3, la gerencia decide abandonar el producto 4. El Estado de Pérdidas y Ganancias de ese mismo año constituyó una desagradable sorpresa: el resultado global mostraba una pérdida de 20.000 €, a la vez que los productos 2 y 3 mostraban igualmente unos resultados parciales negativos. El ingeniero jefe de producción aseguró a la gerencia que no se habían producido cambios en el patrón de comportamiento de los costes.

AÑO 20X3

\begin{tabular}{lrrrr}
PRODUCTOS & 1 & 2 & 3 & TOTAL\\ \hline
+ Ingresos & 110.000 € & 90.000 € & 95.000 € & 295.000 €\\
- Costes & (104.576 €) & (101.017 €) & (109.407 €) & (315.000 €)*\\
= Resultado & 5.424 € & (11.017 €) & (14.407 €) & (20.000 €)
\end{tabular}
ffff
* Los costes fijos, por importe de 200.000 €, se asignan a los productos en función de los ingresos por ventas.

Se pide:
\begin{enumerate}
    \item Explique por qué los beneficios aumentaron cuando el producto 5 fue abandonado, y disminuyeron cuando se actuó de idéntico modo con el producto 4.
    \item Explique por qué los productos 2 y 3 muestran pérdidas en el ejercicio 20X3.
\end{enumerate}
\end{ejercicio}

\begin{solucion}
El problema central de este caso reside en el método de asignación de costes fijos que utiliza la empresa. La gerencia toma decisiones basándose en un resultado que incluye una asignación arbitraria de 200.000 € en costes fijos, repartidos en función de los ingresos por ventas. Estos 200.000 € son costes fijos comunes o estructurales, es decir, la empresa incurre en ellos independientemente de qué productos fabrique o elimine, como confirma el ingeniero jefe.

La decisión de eliminar o mantener un producto debe basarse en su margen de contribución (ingresos menos costes variables), no en el resultado artificial que muestra el informe.

\textbf{1. Explicación del aumento y disminución del beneficio}

Al eliminar el producto 5 en 20X2, aunque el informe de 20X1 mostraba una pérdida de 50.000 €, el beneficio global de la empresa aumentó de 20.000 € a 30.000 €. Esto se debe a que el producto 5 tenía un margen de contribución negativo: sus costes variables eran superiores a sus ingresos por ventas. Por tanto, cada unidad vendida del producto 5 generaba una pérdida operativa adicional. Al eliminarlo, la empresa dejó de incurrir en esa pérdida y el beneficio total aumentó.

Por el contrario, al eliminar el producto 4 en 20X3, aunque en 20X2 mostraba una "pérdida" de 2.500 €, esta pérdida era un dato engañoso, resultado de la asignación arbitraria de costes fijos. En realidad, el producto 4 tenía un margen de contribución positivo, es decir, sus ingresos por ventas eran superiores a sus costes variables y ayudaba a cubrir parte de los costes fijos comunes. Al eliminar el producto 4, la empresa perdió ese margen de contribución, pero los costes fijos de 200.000 € permanecieron. El margen total generado por los productos restantes ya no fue suficiente para cubrirlos, lo que llevó a la empresa a una pérdida global de 20.000 €.

\textbf{2. Explicación de las pérdidas de los productos 2 y 3 en 20X3}

Las pérdidas que muestran los productos 2 y 3 en 20X3 son un artificio contable causado por el mismo método erróneo de asignación de costes. Este fenómeno se conoce como la "espiral de la muerte" de los costes fijos: al reducirse la base de reparto (los ingresos totales), la tasa de asignación de costes fijos a cada producto aumenta. En 20X1, los 200.000 € de costes fijos se repartían entre los ingresos de cinco productos; en 20X2, entre cuatro; y en 20X3, entre tres. Así, la tasa de imputación de costes fijos por cada euro vendido es mucho más alta, y al asignar una porción tan grande de costes fijos a los productos 2 y 3, su resultado contable se vuelve negativo, aunque probablemente sigan teniendo márgenes de contribución positivos, dado que el patrón de costes no cambió.
\end{solucion}


% =================================================================
% EJERCICIO 1
% =================================================================

% =================================================================
% EJERCICIO 1
% =================================================================


% =================================================================
% EJERCICIO 4: FLITSA
% =================================================================
\begin{ejercicio}
FLITSA fabricó 700 u.c. de su producto A en 20X2. A nivel unitario los costes variables y fijos de producción fueron de $6 \text{ €}$ y de $2 \text{ €}$, respectivamente. El inventario físico del producto A al 31/12/20X2 arrojó una cantidad de 100 u.c..

Sabiendo que no había stock inicial alguno del producto A,

\textbf{SE PIDE:}
\begin{enumerate}
    \item ¿Cuál sería la diferencia entre el resultado periódico calculado según los enfoques de asignación de costes del \textit{direct cost simple} y del \textit{full cost atenuado}?.
    \item Suponiendo que el resultado del periodo fuese negativo, ¿bajo qué enfoque serían mayores las pérdidas?.
\end{enumerate}
\end{ejercicio}

\begin{solucion}

\textbf{Datos de Partida}:
\begin{itemize}
    \item Producción ($A$): 700 u.c.
    \item Existencias Iniciales ($A_i$): 0 u.c.
    \item Existencias Finales ($A_f$): 100 u.c.
    \item Coste Variable Unitario de Producción ($CV_{up}$): $6 \text{ €/u.c.}$
    \item Coste Fijo Unitario de Producción ($CF_{up}$): $2 \text{ €/u.c.}$
\end{itemize}

El coste fijo total de producción incurrido fue:
$$CF_{Total} = CF_{up} \times A = 2 \text{ €/u.c.} \times 700 \text{ u.c.} = 1.400 \text{ €}$$

La diferencia en el resultado entre el \textbf{Full Cost Atenuado (FCA)} y el \textbf{Direct Cost Simple (DCS)} se debe únicamente a la capitalización de los costes fijos de producción en el inventario.

\paragraph{1. Cálculo de la Diferencia de Resultados}
La diferencia es la variación neta de los costes fijos de producción que quedan inventariados:
$$\text{Diferencia Resultado} = \text{Costes Fijos en } A_f \text{ (FCA)} - \text{Costes Fijos en } A_i \text{ (FCA)}$$

\textbf{Costes Fijos Capitalizados en Existencias Iniciales ($CF_{A_i}$)}:
Dado que $A_i = 0 \text{ u.c.}$, los costes fijos capitalizados son:
$$CF_{A_i} = 0 \text{ u.c.} \times 2 \text{ €/u.c.} = 0 \text{ €}$$

\textbf{Costes Fijos Capitalizados en Existencias Finales ($CF_{A_f}$)}:
Bajo el enfoque FCA (o coste completo, como el \textit{full cost atenuado}), los costes fijos de producción se consideran costes del producto y se capitalizan.
$$CF_{A_f} = A_f \times CF_{up} = 100 \text{ u.c.} \times 2 \text{ €/u.c.} = \mathbf{200 \text{ €}}$$

\textbf{Cuantía de la Diferencia}:
$$\text{Diferencia (FCA - DCS)} = 200 \text{ €} - 0 \text{ €} = \mathbf{200 \text{ €}}$$

El resultado obtenido por el \textbf{Full Cost Atenuado (FCA) será $200 \text{ €}$ superior} al resultado del Direct Cost Simple (DCS).

\paragraph{2. Comparación de Pérdidas}
Si el resultado fuese negativo (pérdidas), el enfoque que capitaliza la mayor cantidad de costes fijos (el FCA) mostrará la pérdida menor.
\begin{itemize}
    \item El \textbf{Full Cost Atenuado (FCA)} difiere $200 \text{ €}$ de costes fijos al balance.
    \item El \textbf{Direct Cost Simple (DCS)} imputa estos $200 \text{ €}$ directamente a gastos del periodo.
\end{itemize}
Por lo tanto, el enfoque del \textbf{Direct Cost Simple (DCS) arrojaría las mayores pérdidas}.

\end{solucion}

\hrule
\vspace{1cm}

% =================================================================
% EJERCICIO 5: LAKERSA
% =================================================================
\begin{ejercicio}
LAKERSA utiliza el enfoque de \textit{full cost radical} (FCR) a efectos de valoración de la producción, mientras que en la información que elabora con destino a la toma de decisiones gerenciales el enfoque de asignación de costes empleado es el \textit{direct cost simple} (DCS).

Los siguientes datos se han tomado del ejercicio anterior:
\begin{table}[H]
    \centering
    \caption{Datos de LAKERSA}
    \begin{tabular}{l c c c}
        \toprule
        \textbf{Unidades} & \textbf{Cantidad} & \textbf{Coste Variable Unitario} & \textbf{Coste Completo Unitario} \\
        \midrule
        Existencias iniciales ($A_i$) & $6.000 \text{ u.c.}$ & $10 \text{ €/u.c.}$ & $14 \text{ €/u.c.}$ \\
        Existencias finales ($A_f$) & $9.000 \text{ u.c.}$ & $11 \text{ €/u.c.}$ & $16 \text{ €/u.c.}$ \\
        Producción ($A$) & $30.000 \text{ u.c.}$ & (Implícito) & (Implícito) \\
        \bottomrule
    \end{tabular}
\end{table}

Sabiendo que la valoración de las salidas de almacén se realiza siguiendo el criterio del \textbf{precio FIFO},

\textbf{SE PIDE:}
Determinar la cuantía en que diferiría el resultado del período de referencia, calculado de acuerdo a cada uno de los dos enfoques mencionados.
\end{ejercicio}

\begin{solucion}

La diferencia en el resultado se basa en la variación neta de los costes fijos de producción capitalizados:
$$\text{Diferencia Resultado (FCR - DCS)} = \text{Costes Fijos en } A_f - \text{Costes Fijos en } A_i$$

\paragraph{1. Determinación de los Costes Fijos Unitarios de Producción}
El coste fijo unitario ($CF_u$) es la diferencia entre el Coste Completo Unitario (FCR) y el Coste Variable Unitario (DCS):
\begin{itemize}
    \item $CF_{u, i}$ (Existencias Iniciales): $14 \text{ €/u.c.} - 10 \text{ €/u.c.} = \mathbf{4 \text{ €/u.c.}}$.
    \item $CF_{u, p}$ (Producción del Período): $16 \text{ €/u.c.} - 11 \text{ €/u.c.} = \mathbf{5 \text{ €/u.c.}}$.
\end{itemize}

\paragraph{2. Aplicación del Criterio FIFO}
El volumen de existencias colocadas (vendidas) es:
$$\text{Unidades Vendidas} (A_v) = A_i + A - A_f = 6.000 + 30.000 - 9.000 = 27.000 \text{ u.c.}$$

El criterio \textbf{FIFO} (First-In, First-Out) establece que las unidades que salen primero son las más antiguas. Por lo tanto, las existencias finales ($A_f = 9.000 \text{ u.c.}$) deben estar compuestas por unidades de la \textbf{última producción}, es decir, la producción del período ($CF_{u, p} = 5 \text{ €/u.c.}$).

\paragraph{3. Cálculo de los Costes Fijos Inventariados}

\textbf{Costes Fijos en Existencias Iniciales ($CF_{A_i}$)}:
$$CF_{A_i} = 6.000 \text{ u.c.} \times 4 \text{ €/u.c.} = \mathbf{24.000 \text{ €}}$$

\textbf{Costes Fijos en Existencias Finales ($CF_{A_f}$)}:
Puesto que las $9.000 \text{ u.c.}$ finales provienen de la producción del período, se valoran a $5 \text{ €/u.c.}$.
$$CF_{A_f} = 9.000 \text{ u.c.} \times 5 \text{ €/u.c.} = \mathbf{45.000 \text{ €}}$$

\paragraph{4. Determinación de la Diferencia de Resultados}

$$\text{Diferencia} = CF_{A_f} - CF_{A_i}$$
$$\text{Diferencia} = 45.000 \text{ €} - 24.000 \text{ €} = \mathbf{21.000 \text{ €}}$$

El resultado calculado bajo el enfoque de \textbf{Full Cost Radical (FCR) diferiría en $21.000 \text{ €}$} respecto al resultado del \textbf{Direct Cost Simple (DCS)}, siendo el resultado del FCR el mayor.
\end{solucion}

% =================================================================
% EJERCICIO 8
% =================================================================


\begin{ejercicio}
TEMA 3 – CASO PRÁCTICO 8

El gerente de ATLANTA S.A. ha solicitado que el siguiente Estado de Pérdidas y Ganancias, elaborado según el enfoque del direct cost simple, lo sea de acuerdo con el enfoque del full cost radical, de tal forma que el mismo pueda ser fácilmente comparado con el de las empresas pertenecientes al mismo sector de actividad:

\textbf{ESTADO DE PÉRDIDAS Y GANANCIAS DE ATLANTA S.A. DEL AÑO 20X0\\
ENFOQUE DIRECT COST SIMPLE}

\begin{tabular}{l r}
(+) Ingresos del periodo (10.000 u.c. a 10 €/u.c.) & 100.000 € \\
(-) Coste variable de la producción colocada & (39.996 €) \\
(+) Valor del stock inicial & 9.990 € \\
(+) Valor de la producción del período & 36.000 € \\
(-) Valor del stock final & (5.994 €) \\
(=) Margen bruto & 60.004 € \\
(-) Costes fijos & (45.000 €) \\
\quad Costes fijos de Compras y Transformación & 25.000 € \\
\quad Costes fijos de Ventas y Administración & 20.000 € \\
(=) Resultado del período & 15.004 € \\
\end{tabular}

La producción anual en los últimos ejercicios económicos viene ascendiendo a 10.000 u.c. Las salidas de almacén se valoran según el criterio del precio FIFO. Los costes fijos del año 20X0 fueron idénticos a las del año precedente, teniendo tal carácter la totalidad de los relativos a las secciones de ventas y de administración.

\textbf{SE PIDE:}

Formular el Estado de Pérdidas y Ganancias relativo al año 20X0, de acuerdo al enfoque de asignación de costes del full cost radical.

\end{ejercicio}

\begin{solucion}
\textbf{1. Cálculo de unidades en stock}

\begin{itemize}
    \item Producción 20X0: 10.000 u.c.
    \item Valor producción (CV): 36.000 € $\implies$ CV unitario 20X0: $36.000/10.000 = 3,60$ €/u.c.
    \item Valor stock final: 5.994 € $\implies$ Stock final: $5.994/3,60 = 1.665$ u.c.
    \item Valor stock inicial: 9.990 € $\implies$ Stock inicial: $9.990/6,00 = 1.665$ u.c. (CV unitario 20X-1: $9.990/1.665 = 6,00$ €/u.c.)
    \item Unidades vendidas: $1.665 + 10.000 - 1.665 = 10.000$ u.c.
\end{itemize}

\textbf{2. Cálculo de costes completos unitarios (FCR)}

\begin{itemize}
    \item Coste total 20X0: $36.000$ (CV) $+ 25.000$ (CF fabricación) $+ 20.000$ (CF ventas/adm.) $= 81.000$ €
    \item Coste completo unitario 20X0: $81.000/10.000 = 8,10$ €/u.c.
    \item Coste fijo unitario 20X0: $(25.000 + 20.000)/10.000 = 4,50$ €/u.c.
    \item Coste completo unitario 20X-1: $6,00$ (CV) $+ 4,50$ (CF) $= 10,50$ €/u.c.
\end{itemize}

\textbf{3. Cálculo del coste completo de la producción colocada (FIFO)}

\begin{itemize}
    \item Salida de stock inicial: $1.665$ u.c. $\times 10,50$ €/u.c. $= 17.482,50$ €
    \item Salida de producción 20X0: $8.335$ u.c. $\times 8,10$ €/u.c. $= 67.513,50$ €
    \item Total coste colocado: $17.482,50 + 67.513,50 = 84.996$ €
\end{itemize}

\textbf{4. Estado de Pérdidas y Ganancias (Full Cost Radical)}

\begin{tabular}{l r}
(+) Ingresos del periodo (10.000 u.c. a 10 €/u.c.) & 100.000 € \\
(-) Coste completo de la producción colocada & (84.996 €) \\
(=) Resultado del período & 15.004 € \\
\end{tabular}

\textbf{Conclusión:} El resultado del período según el enfoque del Full Cost Radical es de 15.004 €, idéntico al resultado del Direct Cost Simple, ya que no hay variación neta de existencias y los costes fijos unitarios no han cambiado.
\end{solucion}



% =================================================================
% EJERCICIO 10
% =================================================================



\begin{ejercicio}
TEMA 3 – CASO PRÁCTICO 10

La empresa GADESA, dedicada a la fabricación y comercialización del producto D15 en los mercados andaluz y extremeño, presenta la siguiente información del último período:

\begin{itemize}
    \item Coste de Materia prima: 5.000 €
    \item Coste total del Centro de Aprovisionamiento: 3.000 €
    \item Coste total del Centro de Producción: 12.000 €
    \item Coste total del Centro de Ventas: 6.000 €
    \item Coste total del Centro de Administración: 3.000 €
\end{itemize}

Adicionalmente:
\begin{itemize}
    \item Los costes del Centro de Aprovisionamiento son fijos en un 60\%.
    \item Los costes del Centro de Producción son variables en un 70\%.
    \item Los costes del Centro de Ventas son fijos en un 20\%. El 40\% de estos costes fijos corresponden al mercado extremeño y el 60\% al andaluz. Los costes variables de ventas se asignan según los ingresos por ventas.
    \item Los costes del Centro de Administración son fijos en su totalidad.
    \item Precios de venta: 28 €/u.c. (Andaluz), 30 €/u.c. (Extremeño).
    \item Producción: 1.000 u.c. (600 u.c. vendidas en Andalucía, 400 u.c. en Extremadura).
\end{itemize}

\textbf{Se pide:} Elaborar el Estado de Resultados por mercados siguiendo el enfoque del direct cost desarrollado.

\begin{solucion}

\textbf{1. Clasificación y reparto de costes}

\begin{itemize}
    \item \textbf{Costes Variables de Fabricación (CVfb):}
    \begin{itemize}
        \item Materia Prima: 5.000 € (100\% variable)
        \item Aprovisionamiento: 3.000 € × 40\% = 1.200 €
        \item Producción: 12.000 € × 70\% = 8.400 €
        \item \textbf{Total CVfb:} 5.000 + 1.200 + 8.400 = 14.600 €
    \end{itemize}
    \item \textbf{Costes Variables Comerciales (CVcm):}
    \begin{itemize}
        \item Ventas: 6.000 € × 80\% = 4.800 €
    \end{itemize}
    \item \textbf{Costes Fijos Propios (CFP):}
    \begin{itemize}
        \item Ventas: 6.000 € × 20\% = 1.200 €
        \item CFP Andaluz: 1.200 € × 60\% = 720 €
        \item CFP Extremeño: 1.200 € × 40\% = 480 €
    \end{itemize}
    \item \textbf{Costes Fijos Comunes (CFC):}
    \begin{itemize}
        \item Aprovisionamiento: 3.000 € × 60\% = 1.800 €
        \item Producción: 12.000 € × 30\% = 3.600 €
        \item Administración: 3.000 € (100\% fijo)
        \item \textbf{Total CFC:} 1.800 + 3.600 + 3.000 = 8.400 €
    \end{itemize}
\end{itemize}

\textbf{2. Asignación de costes variables y fijos propios por mercados}

\begin{itemize}
    \item \textbf{Ingresos:}
    \begin{itemize}
        \item Andaluz: 600 u.c. × 28 € = 16.800 €
        \item Extremeño: 400 u.c. × 30 € = 12.000 €
        \item Total: 28.800 €
    \end{itemize}
    \item \textbf{CVfb unitario:} 14.600 € / 1.000 u.c. = 14,60 €/u.c.
    \begin{itemize}
        \item Andaluz: 600 × 14,60 = 8.760 €
        \item Extremeño: 400 × 14,60 = 5.840 €
    \end{itemize}
    \item \textbf{CVcm:} Se reparte según ingresos.
    \begin{itemize}
        \item Andaluz: 16.800 / 28.800 = 58,33\% → 4.800 × 58,33\% = 2.800 €
        \item Extremeño: 12.000 / 28.800 = 41,67\% → 4.800 × 41,67\% = 2.000 €
    \end{itemize}
    \item \textbf{CFP:} Ya calculados arriba.
\end{itemize}

\textbf{3. Tablas resumen}

\begin{table}[H]
    \centering
    \caption{Clasificación y resumen de costes}
    \begin{tabular}{lccccc}
        \toprule
        \textbf{Concepto} & \textbf{Total (€)} & \textbf{CVfb} & \textbf{CVcm} & \textbf{CFP} & \textbf{CFC} \\
        \midrule
        Materia Prima & 5.000 & 5.000 & & & \\
        Aprovisionamiento & 3.000 & 1.200 & & & 1.800 \\
        Producción & 12.000 & 8.400 & & & 3.600 \\
        Ventas & 6.000 & & 4.800 & 1.200 & \\
        Administración & 3.000 & & & & 3.000 \\
        \midrule
        \textbf{TOTAL} & 29.000 & 14.600 & 4.800 & 1.200 & 8.400 \\
        \bottomrule
    \end{tabular}
\end{table}

\vspace{0.5cm}

\begin{table}[H]
    \centering
    \caption{Asignación de costes variables y fijos propios}
    \begin{tabular}{lccc}
        \toprule
        \textbf{Concepto} & \textbf{Andaluz} & \textbf{Extremeño} & \textbf{Total} \\
        \midrule
        CVfb & 8.760 & 5.840 & 14.600 \\
        CVcm & 2.800 & 2.000 & 4.800 \\
        CFP & 720 & 480 & 1.200 \\
        \bottomrule
    \end{tabular}
\end{table}

\vspace{0.5cm}

\begin{table}[H]
    \centering
    \caption{Estado de Resultados por Mercados (Direct Cost Desarrollado)}
    \begin{tabular}{lccc}
        \toprule
        \textbf{Concepto} & \textbf{Andaluz} & \textbf{Extremeño} & \textbf{TOTAL} \\
        \midrule
        (+) Ingresos por Ventas & 16.800 & 12.000 & 28.800 \\
        (-) CVfb & (8.760) & (5.840) & (14.600) \\
        (-) CVcm & (2.800) & (2.000) & (4.800) \\
        (=) Margen Bruto & 5.240 & 4.160 & 9.400 \\
        (-) CFP & (720) & (480) & (1.200) \\
        (=) Margen Semibruto & 4.520 & 3.680 & 8.200 \\
        (-) CFC & & & (8.400) \\
        (=) Resultado del Período & & & \textbf{(200)} \\
        \bottomrule
    \end{tabular}
\end{table}

\textbf{Conclusión:} El Estado de Resultados por mercados, bajo el enfoque del direct cost desarrollado, permite identificar la contribución de cada mercado y el impacto de los costes fijos comunes en el resultado global.

\end{solucion}
\end{ejercicio}











% =================================================================
% EJERCICIO 1
% =================================================================
% =================================================================
% EJERCICIO 1
% =================================================================
% =================================================================
% EJERCICIO 1
% =================================================================
% =================================================================
% EJERCICIO 1
% =================================================================
% =================================================================
% EJERCICIO 1
% =================================================================
% =================================================================
% EJERCICIO 1
% =================================================================
% =================================================================
% EJERCICIO 1
% =================================================================
% =================================================================
% EJERCICIO 1
% =================================================================
% =================================================================
% EJERCICIO 1
% =================================================================
% =================================================================
% EJERCICIO 1
% =================================================================
% =================================================================
% EJERCICIO 1
% =================================================================

