% ================================================================
% CAPÍTULO 1: INTRODUCCIÓN A LA CONTABILIDAD DE GESTIÓN
% ================================================================

\chapter{Naturaleza y contenido de la contabilidad de gestión}

\section{Modelo básico de la circulación de valores en la empresa}

La actividad empresarial puede entenderse como una continua \textbf{circulación de valores} que conecta a la empresa con su entorno y articula sus procesos internos. Este modelo se fundamenta en cuatro subsistemas interconectados que describen el ciclo económico de una unidad de producción:
\begin{enumerate}
    \item \textbf{Financiación}: Corresponde a las operaciones dedicadas a la obtención de los recursos financieros necesarios para la actividad. Constituye el punto de partida, donde se captan capitales (aportaciones, préstamos) que dotan a la empresa de liquidez (dinero).
    \item \textbf{Inversión}: Engloba las operaciones relativas a la adquisición de los factores productivos. La empresa utiliza el dinero obtenido para comprar los bienes y servicios (materiales, maquinaria, mano de obra) que necesita para producir. Esta fase transforma el dinero en factores de producción, y la magnitud que la representa es el \textbf{gasto}.
    \item \textbf{Producción}: Se refiere a las operaciones de aplicación de los factores productivos en el proceso de transformación para obtener nuevos bienes o servicios. Este es el núcleo del ámbito interno de la empresa, donde el consumo de los factores productivos da lugar al \textbf{coste} y se genera la producción valorada.
    \item \textbf{Desinversión}: Agrupa las operaciones relativas a la colocación de los productos (bienes o servicios) en el mercado. La venta de las existencias de mercancías acabadas a los clientes genera un \textbf{ingreso}, que idealmente retorna a la empresa en forma de dinero, cerrando así el ciclo.
\end{enumerate}

Este flujo se puede visualizar en dos ámbitos:
\begin{itemize}
    \item \textbf{Ámbito externo}: Comprende las transacciones de la empresa con el "mundo exterior", como las compras a proveedores y las ventas a clientes.
    \item \textbf{Ámbito interno}: Se centra en el proceso de transformación productiva, abarcando las fases de consumo, fabricación y almacenamiento.
\end{itemize}

\begin{figure}[H]
    \centering
    \includegraphics[width=0.8\textwidth]{media/placeholder.png} % Nota: Se asume que la imagen del esquema está disponible como placeholder.png
    \caption{Esquema de la circulación de valores en la empresa (Adaptado de Schneider, 1968).}
\end{figure}

\section{La Contabilidad de gestión: delimitación y objetivos}

\begin{definicion}
La \textbf{Contabilidad de Gestión} es una rama de la contabilidad que se enfoca en la realidad económico-técnica o interna de una microunidad económica. Su finalidad específica es permitir el control de la producción y los costes, así como medir la eficiencia técnico-productiva de la misma.
\end{definicion}

Este sistema de información está diseñado para ser utilizado por los directivos para planificar, evaluar y controlar la organización, asegurando un uso apropiado y responsable de los recursos. A diferencia de la contabilidad financiera, no está regulada externamente y se adapta a las necesidades estratégicas y operativas de cada empresa.

Los \textbf{objetivos} o fines principales de la Contabilidad de Gestión son:
\begin{itemize}
    \item \textbf{Planificación y control de gestión}: Ayuda a los directivos a cuantificar los efectos futuros de las decisiones (planificar), juzgar los resultados históricos frente a los planes (evaluar) y vigilar el rendimiento para tomar acciones correctivas (controlar).
    \item \textbf{Cálculo del coste de los productos}: Es fundamental para valorar los inventarios, controlar las operaciones y obtener el coste de los productos con el fin de tomar decisiones sobre precios, rentabilidad o fabricación.
    \item \textbf{Toma de decisiones}: Proporciona información relevante para la selección entre cursos de acción alternativos, tanto a corto como a largo plazo.
    \item \textbf{Análisis y evaluación de actividades}: Permite un conocimiento detallado de las actividades productivas para su control.
    \item \textbf{Determinación de resultados internos}: Calcula el resultado periódico con criterios económicos y lo descompone para conocer la contribución de cada área o producto a su generación.
\end{itemize}

Para cumplir estos objetivos, la Contabilidad de Gestión se centra en las \textbf{magnitudes fundamentales del ámbito interno}:
\begin{itemize}
    \item \textbf{Magnitudes flujo (corrientes)}: Consumos y costes de un período, producción y su valor, y colocación y su valor.
    \item \textbf{Magnitudes fondo (stocks)}: Producción en curso y su valor, y producción en stock y su valor.
\end{itemize}

\section{Contabilidad externa y Contabilidad interna}

La información contable de una organización se estructura en dos grandes áreas: la \textbf{Contabilidad Financiera} (externa) y la \textbf{Contabilidad de Gestión} (interna). Aunque ambas se nutren del mismo sistema de información, sus propósitos, usuarios y características son distintos.

\begin{table}[h!]
\centering
\caption{Principales diferencias entre Contabilidad Financiera y de Gestión.}
\begin{tabular}{p{0.25\linewidth} p{0.3\linewidth} p{0.3\linewidth}}
\hline
\textbf{Rasgo} & \textbf{Contabilidad Financiera (Externa)} & \textbf{Contabilidad de Gestión (Interna)} \\
\hline
\textbf{Usuarios} & Externos (accionistas, bancos, gobierno) e internos. & Exclusivamente internos (directivos, mandos intermedios, empleados). \\
\textbf{Regulación} & Regulada por principios contables generalmente aceptados (PCGA) y el Estado. & No regulada. Determinada por la dirección para satisfacer sus necesidades. \\
\textbf{Naturaleza de la información} & Prima la objetividad, fiabilidad y verificabilidad. Es precisa y auditable. & Prima la relevancia y flexibilidad para la toma de decisiones. Es más subjetiva (estimaciones). \\
\textbf{Tipo de información} & Principalmente medidas financieras. & Medidas financieras, operativas y físicas sobre procesos, clientes, etc.. \\
\textbf{Enfoque temporal} & Histórico, orientado al pasado. & Actual y orientado al futuro. \\
\textbf{Ámbito} & Agregada y global. Informa sobre el conjunto de la organización. & Desagregada y concreta. Informa sobre departamentos, segmentos o decisiones específicas. \\
\textbf{Obligatoriedad} & Obligatoria. & No obligatoria. \\
\hline
\end{tabular}
\end{table}

La Contabilidad Financiera se centra en registrar las operaciones de la empresa y presentar informes a terceros, proyectando una imagen global de su situación financiera. Sin embargo, esta información es insuficiente para la gestión diaria. Los directivos necesitan datos detallados para tomar decisiones rutinarias y no rutinarias, como analizar la rentabilidad de un producto o evaluar la eficiencia de una actividad. Es aquí donde la Contabilidad de Gestión (también denominada \textbf{Contabilidad interna}, \textbf{analítica} o \textbf{de costes}) cobra su relevancia, proporcionando la información desagregada que la gestión interna demanda.

\section{Producción: conceptos y clases}

El concepto de producción puede analizarse desde dos perspectivas: como efecto (el resultado del proceso) y como causa (el proceso de transformación en sí mismo).

\subsection{La producción como efecto}
Desde esta óptica, la producción se refiere a los bienes o servicios obtenidos. Se clasifica principalmente según su grado de perfeccionamiento:
\begin{itemize}
    \item \textbf{Producción final}: Corresponde a los productos acabados, listos para su venta o destino final.
    \item \textbf{Producción intermedia}: Incluye productos que aún no han completado el ciclo productivo. Se subdivide en:
        \begin{itemize}
            \item \textbf{Productos en curso}: Aquellos que se encuentran en una fase de elaboración dentro de un centro de trabajo.
            \item \textbf{Productos semiterminados}: Aquellos que han finalizado una fase del proceso y se encuentran almacenados a la espera de ser incorporados en una etapa posterior.
        \end{itemize}
    \item \textbf{Otra producción}:
        \begin{itemize}
            \item \textbf{Subproductos}: Productos de carácter secundario obtenidos simultáneamente con el producto principal (p. ej., el serrín en una fábrica de muebles).
            \item \textbf{Desperdicios}: Residuos generados en el proceso que pueden tener o no valor de venta.
        \end{itemize}
\end{itemize}

\subsection{La producción como causa}
Bajo esta perspectiva, la producción es el \textbf{proceso productivo} en sí, definido como una "transformación, según una determinada técnica, de factores productivos en productos". Este proceso tiene una vertiente técnica donde los factores son consumidos en centros de trabajo (actividad) para generar productos y servicios.

\subsubsection{Clasificación de la producción según el proceso}
Atendiendo a los tipos de productos y su forma de obtención, la producción puede ser:
\begin{itemize}
    \item \textbf{Producción simple}: Se obtiene un único tipo de producto, ya sea a través de un proceso lineal o complejo.
    \item \textbf{Producción múltiple o compuesta}: Se obtienen varios tipos de productos de forma simultánea o excluyente. Se divide en:
        \begin{itemize}
            \item \textbf{Producción paralela}: Se obtienen productos distintos en procesos independientes.
            \item \textbf{Producción alternativa}: La fabricación de un producto excluye la de otro, utilizando los mismos factores (p. ej., envasado de aceite de oliva o de girasol en la misma planta).
            \item \textbf{Producción conjunta (o acumulativa)}: Se obtienen simultáneamente varios productos a partir de un mismo proceso y materia prima. Es inevitable obtener todos los productos a la vez. Puede ser:
                \begin{itemize}
                    \item \textbf{Con coproductos}: Se obtienen varios productos principales (p. ej., carne y piel en la industria cárnica).
                    \item \textbf{Con subproductos}: Se obtiene un producto principal y otros secundarios.
                    \item \textbf{Acoplada}: Como en la destilación de la hulla, donde de una materia prima se obtienen múltiples productos (gas, coque, alquitrán, etc.) en proporciones fijas.
                \end{itemize}
        \end{itemize}
\end{itemize}

\section{Proceso productivo y medios de producción}
El proceso productivo requiere \textbf{factores de producción} o medios que hacen posible la transformación económica. Estos recursos se pueden clasificar de diversas formas:
\begin{itemize}
    \item \textbf{Según su participación en el proceso}:
        \begin{itemize}
            \item \textit{Factores estructurales}: Forman la capacidad productiva de la empresa (maquinaria, edificios).
            \item \textit{Factores para perfeccionamiento}: Se consumen o transforman en el proceso (materia prima, medios colaboradores como la energía).
        \end{itemize}
    \item \textbf{Según su influencia en el producto final}:
        \begin{itemize}
            \item \textit{Factores limitativos}: Deben utilizarse en proporciones fijas.
            \item \textit{Factores sustitutivos}: Pueden intercambiarse entre sí.
        \end{itemize}
\end{itemize}
Los centros de trabajo donde se aplican estos factores pueden agregarse en distintos niveles. La célula básica de actividad es la \textbf{unidad de trabajo}, que es una combinación indivisible de medios estructurales (p. ej., una máquina) y su correspondiente dotación de personal. Niveles superiores de agregación incluyen el lugar de trabajo y la sección de trabajo.

\section{Productividad y rendimiento: su medida}
La \textbf{productividad} es una medida de la eficiencia del proceso productivo que relaciona la producción obtenida (output) con la cantidad de factores o recursos utilizados (input). Se expresa como un cociente:
$$ \text{Productividad} = \frac{\text{Producción (Outputs)}}{\text{Factores empleados (Inputs)}} $$
Esta medida puede calcularse para un factor específico (productividad parcial, p.ej., productividad del trabajo) o para el conjunto de factores (productividad total o global).

Por otro lado, el \textbf{rendimiento} compara la producción real obtenida con la producción que se debería haber obtenido en condiciones estándar o normales. Se formula como:
$$ \text{Rendimiento} = \frac{\text{Producción Real}}{\text{Producción Estándar}} \quad \text{o} \quad \text{Rendimiento} = \frac{\text{Tiempo Estándar}}{\text{Tiempo Real}} $$
Un rendimiento mayor que 1 (o 100\%) indica una eficiencia superior a la estándar, mientras que un valor inferior a 1 señala una ineficiencia. Ambas magnitudes, productividad y rendimiento, son cruciales para el control de la gestión, pues permiten evaluar y mejorar la eficiencia con la que se utilizan los recursos de la empresa.
