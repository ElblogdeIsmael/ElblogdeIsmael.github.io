% =========================================================
% PAQUETES NECESARIOS
% =========================================================
\usepackage{xcolor}    % Para el color del recuadro
\usepackage{tikz}      % Para los diagramas
\usepackage{graphicx}  % Para imágenes
\usepackage{float}     % Para posicionamiento H
\usepackage{listings}  % Para código
\usepackage{etoolbox}  % Utilidades varias

% =========================================================
% LIMPIEZA PREVIA (Evita errores si ya existía el entorno)
% =========================================================
\let\ejercicio\relax
\let\endejercicio\relax
\let\ejerciciostyle\relax

% =========================================================
% COMANDOS DE IMÁGENES
% =========================================================

% Comando para incluir imágenes
\newcommand{\incluirimagen}[3][]{%
\begin{figure}[H]
    \centering
    \includegraphics[width=\linewidth,#1]{#2}
    \caption{#3}
    \label{fig:#2}
\end{figure}
}

% Comando para dos imágenes en paralelo
\newcommand{\dosimagenes}[6]{%
    \begin{figure}[h!]
        \centering
        \begin{minipage}{0.48\linewidth}
            \centering
            \includegraphics[width=\linewidth]{#1}
            \caption{#2}
            \label{#5}
        \end{minipage}\hfill
        \begin{minipage}{0.48\linewidth}
            \centering
            \includegraphics[width=\linewidth]{#3}
            \caption{#4}
            \label{#6}
        \end{minipage}
    \end{figure}
}


\usepackage{algorithm}
\usepackage{algpseudocode}

\newcommand{\portadaimg}{\VAR{portadaimg}}





% % =========================================================
% % NUEVA DEFINICIÓN DE EJERCICIO (CON CUADRO AZUL)
% % =========================================================

% % Definimos el estilo con fcolorbox en el encabezado
% \newtheoremstyle{ejerciciostyle}
%   {5pt}   % Espacio arriba
%   {5pt}   % Espacio abajo
%   {}       % Fuente del cuerpo (vacío = normal/recta, usa \itshape si quieres cursiva)
%   {}       % Sangría
%   {\bfseries} % Fuente del encabezado base
%   {}      % Puntuación tras encabezado (vacía, la gestionamos dentro)
%   {5pt}   % Espacio tras encabezado
%   {%
%     % Configuración de la caja
%     \setlength{\fboxsep}{3pt}% Espacio interno (padding)
%     \setlength{\fboxrule}{0.8pt}% Grosor del borde
%     % La caja: Borde Azul, Fondo Azul Claro
%     \fcolorbox{blue}{blue!10}{%
%         \color{black}\thmname{#1} \thmnumber{#2}%
%     }%
%     % Si hay nota (título), va fuera de la caja
%     \thmnote{ \textbf{(#3)}}%
%   }

% % Aplicamos el estilo y creamos el entorno
% \theoremstyle{ejerciciostyle}
% \newtheorem{ejercicio}{Ejercicio}[section]

% % Formato de numeración: Cap.Sec.Ejer
% \renewcommand{\theejercicio}{\thechapter.\arabic{section}.\arabic{ejercicio}}

% =========================================================
% NUEVA DEFINICIÓN DE EJERCICIO (CON CUADRO AZUL COMPLETO)
% =========================================================

% Contador para ejercicios
\newcounter{ejercicio}[section]
\renewcommand{\theejercicio}{\thechapter.\arabic{section}.\arabic{ejercicio}}

% Definimos el entorno con tcolorbox
\newenvironment{ejercicio}{%
    \refstepcounter{ejercicio}%
    \begin{tcolorbox}[
        colback=gray!5,         % Fondo gris muy claro (profesional)
        colframe=gray!60!black, % Borde gris oscuro
        boxrule=0.8pt,          % Grosor del borde
        arc=2pt,                % Redondeo de esquinas (opcional)
        left=5pt,               % Margen izquierdo interno   right=5pt,              % Margen derecho interno
        top=5pt,                % Margen superior interno
        bottom=5pt,             % Margen inferior interno
        before skip=10pt,       % Espacio antes del cuadro
        after skip=10pt,        % Espacio después del cuadro
        title={\textbf{Ejercicio \theejercicio}}
    ]%
}{%
    \end{tcolorbox}%
}





% =========================================================
% SOLUCIONES
% =========================================================
\theoremstyle{remark}
\newtheorem{solucion}{Solución}[ejercicio]
\renewcommand{\thesolucion}{\thechapter.\arabic{section}.\arabic{ejercicio}}


% =========================================================
% NOTAS
% =========================================================
\newtheorem{nota}{Nota}[chapter]


% =========================================================
% CÓDIGOS (LISTINGS)
% =========================================================

% Comando para poner dos códigos en paralelo
\newcommand{\doscodigos}[4]{%
  \noindent
  \begin{minipage}{0.48\linewidth}
    \lstset{language=#1}
    \lstinputlisting{#2}
  \end{minipage}\hfill
  \begin{minipage}{0.48\linewidth}
    \lstset{language=#3}
    \lstinputlisting{#4}
  \end{minipage}
}

% Comando para poner un solo código
\newcommand{\uncodigo}[2]{%
  \begin{lstlisting}[language=#1]
#2
  \end{lstlisting}
}

% \lstset{
%     basicstyle=\ttfamily\small, % O el estilo que uses
%     aboveskip=10pt,  % Espacio antes del código
%     belowskip=0pt    % Espacio después (0pt porque la sección siguiente ya pone espacio)
% }

\lstset{
    basicstyle=\ttfamily\footnotesize, % Un poco más pequeño que \small
  }

\raggedbottom % Evita espacios verticales grandes en páginas con poco contenido


% =========================================================
% EJERCICIOS RESUELTOS (Estilo antiguo sin caja)
% =========================================================
\newtheoremstyle{ejercicioresueltostyle}
    {10pt}   % Espacio arriba
    {10pt}   % Espacio abajo
    {\itshape} % Fuente del cuerpo
    {}       % Sangría
    {\bfseries} % Fuente del encabezado
    {}      % Puntuación tras encabezado
    { }      % Espacio tras encabezado
    {\thmname{#1} \thmnumber{#2}. \thmnote{#3}}

\theoremstyle{ejercicioresueltostyle}
\newtheorem{ejercicioresuelto}{Ejercicio Resuelto}[section]
\renewcommand{\theejercicioresuelto}{\thechapter.\arabic{section}.\arabic{ejercicioresuelto}}


% ========================================================= 
% PRÁCTICAS Y TIKZ (DIAGRAMAS)
% =========================================================

% Comando para definir un tema
\newcommand{\tema}[1]{%
  \section{#1}
  \addcontentsline{toc}{section}{#1}
}

% Estilos TikZ para Autómatas
\tikzset{
    error/.style={state, fill=red!20, draw=red!80!black},
    final/.style={state, accepting, fill=green!15!white, draw=green!60!black}
}

% Macros para nodos
\newcommand{\nodo}[4][]{\node[state, #1] (#2) at (#3) {$#4$};}

% Macros para flechas
\newcommand{\flecha}[4][]{\draw[->, #1] (#2) -- (#3) node[midway, above] {#4};}
\newcommand{\flechaabajo}[4][]{\draw[->, #1] (#2) -- (#3) node[midway, below, yshift=-6pt] {#4};}
\newcommand{\flechaarriba}[4][]{\draw[->, #1] (#2) -- (#3) node[midway, above, yshift=6pt] {#4};}
\newcommand{\flechaderecha}[4][]{\draw[->, #1] (#2) -- (#3) node[midway, right] {#4};}
\newcommand{\flechaiquierda}[4][]{\draw[->, #1] (#2) -- (#3) node[midway, left] {#4};}

% Macro para curvas
\newcommand{\curva}[5][]{\draw[->, bend #1] (#2) to node[midway, #5] {#4} (#3);}

% Macros para loops (bucles)
\newcommand{\loopa}[3]{\draw[->] (#1) edge[loop above] node {#2} (#1);}
\newcommand{\loopb}[3]{\draw[->] (#1) edge[loop below] node {#2} (#1);}
\newcommand{\loopr}[3]{\draw[->] (#1) edge[loop right] node {#2} (#1);}
\newcommand{\loopl}[3]{\draw[->] (#1) edge[loop left] node {#2} (#1);}

% Constantes de ejemplo
\newcommand{\pa}{1}
\newcommand{\pUno}{2}
\newcommand{\pDos}{3}