% ========================
% estilo.latex mínimo funcional
% ========================

\documentclass[12pt]{book} % report para capítulos

% ========================
% Paquetes y comandos extra
% ========================
% % Codificación y lengua
\usepackage[utf8]{inputenc}
\usepackage[spanish]{babel}
\usepackage[T1]{fontenc}

% Matemáticas y símbolos
\usepackage{amsmath, amssymb, nicefrac, eurosym, aligned-overset}

% Gráficos y figuras
\usepackage{graphicx, subcaption, tikz, pgfplots}

% Tablas
\usepackage{booktabs, colortbl, xcolor, multirow, makecell}

% Texto y estilos
\usepackage{titlesec, caption, enumitem, color, fancyhdr}

% Otros
\usepackage{hyperref, geometry, float, pdfpages, lastpage, listings, tcolorbox}
\hypersetup{
    colorlinks=true,
    linkcolor=blue,
    urlcolor=blue,
    citecolor=blue
}

\usetikzlibrary{arrows.meta}
\pgfplotsset{compat=1.18}

  % si tienes paquetes personalizados
% % aquí van los comandos personalizados
% Comando para incluir imágenes
\newcommand{\incluirimagen}[3][]{%
\begin{figure}[H]
    \centering
    \includegraphics[width=\linewidth,#1]{#2}
    \caption{#3}
    \label{fig:#2}
\end{figure}
}

% comando para ejercicios con fondo
\newtheoremstyle{ejerciciostyle}
  {10pt}   % Espacio arriba
  {10pt}   % Espacio abajo
  %{\itshape} % Fuente del cuerpo
  {}
  {}       % Sangría
  {\bfseries} % Fuente del encabezado
  {}      % Puntuación tras encabezado
  { }      % Espacio tras encabezado
  {\thmname{#1} \thmnumber{#2}. \thmnote{#3}}


% % comando formal para enunciado de ejercicios
% \theoremstyle{ejerciciostyle}
% \newtheorem{ejercicio}{Ejercicio}[chapter]

\theoremstyle{ejerciciostyle}
\newtheorem{ejercicio}{Ejercicio}[section]

\renewcommand{\theejercicio}{\thechapter.\arabic{section}.\arabic{ejercicio}}


% comando formal para soluciones
\theoremstyle{remark}
\newtheorem{solucion}{Solución}[ejercicio]

\renewcommand{\thesolucion}{\thechapter.\arabic{section}.\arabic{ejercicio}}

% Comando para dos imágenes en paralelo
\newcommand{\dosimagenes}[6]{%
    \begin{figure}[h!]
        \centering
        \begin{minipage}{0.48\linewidth}
            \centering
            \includegraphics[width=\linewidth]{#1}
            \caption{#2}
            \label{#5}
        \end{minipage}\hfill
        \begin{minipage}{0.48\linewidth}
            \centering
            \includegraphics[width=\linewidth]{#3}
            \caption{#4}
            \label{#6}
        \end{minipage}
    \end{figure}
}

% \dosimagenes{media/fondo.jpg}{Descripción 1}{media/fondo.jpg}{Descripción 2}{fig:descripcion1}{fig:descripcion2}

% \ref{fig:descripcion1} es la mejor
% \ref{fig:descripcion2} es la mejor

\newcommand{\portadaimg}{\VAR{portadaimg}}

% Comando para crear una nota estilo información
% \newcommand{\nota}[2]{%
% \begin{tcolorbox}[colframe=blue!75!black, colback=blue!5!white, title=\textbf{#1}]
%     #2
% \end{tcolorbox}
% }
\newtheorem{nota}{Nota}[chapter]


% Comando para poner dos códigos en paralelo
\newcommand{\doscodigos}[4]{%
  \noindent
  \begin{minipage}{0.48\linewidth}
    \lstset{language=#1}
    \lstinputlisting{#2}
  \end{minipage}\hfill
  \begin{minipage}{0.48\linewidth}
    \lstset{language=#3}
    \lstinputlisting{#4}
  \end{minipage}
}

% Comando para poner un solo código
\newcommand{\uncodigo}[2]{%
  \begin{lstlisting}[language=#1]
#2
  \end{lstlisting}
}


% % Listas de archivos (sin guiones en los nombres de macros)
% \newcommand{\listagdfilesSesion2Mallas2D}{cargatexturas.gd, envioinmediato.gd, malla2dcontexturas.gd, mallaconcoloresdevertices.gd, mallanoindentada.gd}
% \newcommand{\listagdfilesSesion2Mallas3D}{mallaindexada3d.gd, materialconcolordeplano.gd, materialconcoloresdevertices.gd, tablas.gd}

% % Macro que recorre una lista de archivos en un subdirectorio
% \newcommand{\includegdfiles}[2]{%
%   % #1 = subdirectorio
%   % #2 = nombre de la lista de archivos
%   \foreach \filename in #2 {%
%     \includecode[gdstyle]{code/#1/\filename}{\filename}
%   }%
% }



% Comando para ejercicio resuelto
\newtheoremstyle{ejercicioresueltostyle}
    {10pt}   % Espacio arriba
    {10pt}   % Espacio abajo
    {\itshape} % Fuente del cuerpo
    {}       % Sangría
    {\bfseries} % Fuente del encabezado
    {}      % Puntuación tras encabezado
    { }      % Espacio tras encabezado
    {\thmname{#1} \thmnumber{#2}. \thmnote{#3}}

\theoremstyle{ejercicioresueltostyle}
\newtheorem{ejercicioresuelto}{Ejercicio Resuelto}[section]

\renewcommand{\theejercicioresuelto}{\thechapter.\arabic{section}.\arabic{ejercicioresuelto}}


%======================================================================== 
% PRACTICAS
%========================================================================

% Comando para definir un tema
\newcommand{\tema}[1]{%
  \section{#1}
  \addcontentsline{toc}{section}{#1}
}
\usepackage{tikz}
\usepackage{graphicx} % necesario para \resizebox
\usepackage{etoolbox}

% ======== NODOS ========
\newcommand{\nodo}[4][]{\node[state, #1] (#2) at (#3) {$#4$};}
% Uso: \nodo[initial,accepting]{q0}{0,0}{q_0}

% ======== FLECHAS ========
\newcommand{\flecha}[4][]{\draw[->, #1] (#2) -- (#3) node[midway, above] {#4};}
% Uso: \flecha{q0}{q1}{0} o \flecha[bend left]{q1}{q2}{1}

\newcommand{\flechaabajo}[4][]{\draw[->, #1] (#2) -- (#3) node[midway, below, yshift=-6pt] {#4};}
% Igual que \flecha pero con etiqueta abajo
\newcommand{\flechaarriba}[4][]{\draw[->, #1] (#2) -- (#3) node[midway, above, yshift=6pt] {#4};}
% Igual que \flecha pero con etiqueta arriba
\newcommand{\flechaderecha}[4][]{\draw[->, #1] (#2) -- (#3) node[midway, right] {#4};}
% Igual que \flecha pero con etiqueta a la derecha
\newcommand{\flechaiquierda}[4][]{\draw[->, #1] (#2) -- (#3) node[midway, left] {#4};}
% Igual que \flecha pero con etiqueta a la izquierda

\newcommand{\curva}[5][]{\draw[->, bend #1] (#2) to node[midway, #5] {#4} (#3);}
% Uso: \curva[left]{q1}{q2}{1}{below}


\newcommand{\loopa}[3]{\draw[->] (#1) edge[loop above] node {#2} (#1);}
\newcommand{\loopb}[3]{\draw[->] (#1) edge[loop below] node {#2} (#1);}
\newcommand{\loopr}[3]{\draw[->] (#1) edge[loop right] node {#2} (#1);}
\newcommand{\loopl}[3]{\draw[->] (#1) edge[loop left] node {#2} (#1);}
% Uso: \loopa{q1}{0}

% ======== ESTILOS ESPECIALES ========
\tikzset{
    error/.style={state, fill=red!20, draw=red!80!black},
    final/.style={state, accepting, fill=green!15!white, draw=green!60!black}
}
% Uso: \nodo[error]{qe}{5,0}{q_e}  o \nodo[final]{qf}{7,0}{q_f}


\newcommand{\pa}{1}      % ejemplo de valor
\newcommand{\pUno}{2}
\newcommand{\pDos}{3}
  % comandos LaTeX propios
% % ===========================
% Diseño general
% ===========================
\setstretch{1.15} % interlineado
\setlength{\parskip}{0.5em} % espacio entre párrafos
\setlength{\parindent}{0pt} % sin sangría

% ===========================
% Estilo de capítulos y secciones (titlesec)
% ===========================
\titleformat{\chapter}[display]
  {\bfseries\Huge}
  {\filleft\Large\scshape Capítulo \thechapter}
  {1ex}
  {\titlerule[1pt]\vspace{1ex}\filright}
  [\vspace{1ex}\titlerule]

\titlespacing*{\chapter}{0pt}{0pt}{2em}

\titleformat{\section}
  {\Large\bfseries}
  {\thesection}{1em}{}

\titleformat{\subsection}
  {\large\bfseries}
  {\thesubsection}{1em}{}

\titleformat{\subsubsection}
  {\normalsize\bfseries\itshape}
  {\thesubsubsection}{1em}{}

% ===========================
% Encabezados y pies de página (fancyhdr)
% ===========================
\pagestyle{fancy}
\fancyhf{} % limpia
\fancyhead[L]{\small\scshape\nouppercase{\leftmark}} % sección/capítulo en mayúsculas pequeñas
\fancyhead[R]{\small\thepage}                        % número de página
%\fancyfoot[C]{\scriptsize\itshape Apuntes de la carrera} % texto fijo abajo en cursiva
% Encabezados y pies de página personalizados
% \fancyfoot[L]{\scriptsize\itshape Nombre de la asignatura} % pie de página izquierdo en cursiva
\fancyfoot[R]{\normalsize Ismael Sallami Moreno}        % pie de página derecho con el nombre del autor

% Línea bajo el encabezado
\renewcommand{\headrulewidth}{0.5pt} % línea más gruesa en el encabezado
% Línea en el pie
\renewcommand{\footrulewidth}{0.4pt} % línea fina en el pie
\renewcommand{\sectionmark}[1]{%
  \markboth{\thesection\quad #1}{}%
}

% ===========================
% Numeración de elementos
% ===========================
\numberwithin{equation}{chapter} % ecuaciones numeradas por capítulo
\numberwithin{figure}{chapter}   % figuras numeradas por capítulo
\numberwithin{table}{chapter}    % tablas numeradas por capítulo

% ===========================
% Listas y enumeraciones
% ===========================
\setlist[itemize]{label=--, left=1.5em}
\setlist[enumerate]{label=\arabic*), left=1.5em}

% ===========================
% Estilo de citas y bibliografía
% ===========================
\DefineBibliographyStrings{spanish}{%
  references = {Bibliografía},
}

% ===========================
% Entornos personalizados
% ===========================
\newtheoremstyle{cajita} % nombre del estilo
  {1em}   % espacio arriba
  {1em}   % espacio abajo
  {}      % fuente del cuerpo
  {}      % indentación
  {\bfseries} % fuente del título
  {.}     % puntuación tras título
  {0.5em} % espacio tras título
  {\thmname{#1}\thmnumber{ #2} \thmnote{(#3)}} % formato


\theoremstyle{cajita}
\newtheorem{teorema}{Teorema}[chapter]
\newtheorem{definicion}{Definición}[chapter]
\newtheorem{ejemplo}{Ejemplo}[chapter]
\newtheorem{proposicion}{Proposición}[chapter]
\newtheorem{demostracion}{Demostración}[chapter]
\newtheorem{corolario}{Corolario}[chapter]
\newtheorem{propuesta}{Propuesta}[chapter]


\newtheoremstyle{anotacionstyle} % nombre del estilo
  {1em}   % espacio arriba
  {1em}   % espacio abajo
  {}      % fuente del cuerpo (sin cursiva)
  {}      % indentación
  {\itshape} % fuente del título (Nota en cursiva)
  {.}     % puntuación tras título
  {0.5em} % espacio tras título
  {\thmname{\itshape#1}\thmnumber{ #2} \thmnote{(#3)}} % formato (solo Nota en cursiva)

\theoremstyle{anotacionstyle}
\newtheorem{anotacion}{Nota}[chapter]

% ===========================
% Configuración de lstlisting
% ===========================

% ===============================================
% ESTILO 1: MODERNO Y MINIMALISTA
% ===============================================

% Definir colores personalizados
\definecolor{codegreen}{rgb}{0,0.6,0}
\definecolor{codegray}{rgb}{0.5,0.5,0.5}
\definecolor{codepurple}{rgb}{0.58,0,0.82}
\definecolor{backcolour}{rgb}{0.95,0.95,0.92}
\definecolor{framecolor}{rgb}{0.8,0.8,0.8}

\lstset{
  backgroundcolor=\color{backcolour},   
  commentstyle=\color{codegreen},
  keywordstyle=\color{magenta},
  numberstyle=\tiny\color{codegray},
  stringstyle=\color{codepurple},
  basicstyle=\ttfamily\footnotesize,
  breakatwhitespace=false,         
  breaklines=true,                 
  captionpos=b,                    
  keepspaces=true,                 
  numbers=left,                    
  numbersep=5pt,                  
  showspaces=false,                
  showstringspaces=false,
  showtabs=false,                  
  tabsize=2,
  frame=shadowbox,
  frameround=tttt,
  rulecolor=\color{framecolor},
  rulesepcolor=\color{framecolor},
  xleftmargin=20pt,
  xrightmargin=20pt,
  aboveskip=20pt,
  belowskip=20pt,
  inputencoding=utf8,
  extendedchars=true,
  literate=
    {←}{{$\leftarrow$}}1
    {→}{{$\rightarrow$}}1
    {↑}{{$\uparrow$}}1
    {↓}{{$\downarrow$}}1
    {↔}{{$\leftrightarrow$}}1
    {⇒}{{$\Rightarrow$}}1
    {⇐}{{$\Leftarrow$}}1
    {⇔}{{$\Leftrightarrow$}}1
    {α}{{$\alpha$}}1
    {β}{{$\beta$}}1
    {γ}{{$\gamma$}}1
    {δ}{{$\delta$}}1
    {ε}{{$\epsilon$}}1
    {θ}{{$\theta$}}1
    {λ}{{$\lambda$}}1
    {μ}{{$\mu$}}1
    {π}{{$\pi$}}1
    {σ}{{$\sigma$}}1
    {φ}{{$\phi$}}1
    {ψ}{{$\psi$}}1
    {ω}{{$\omega$}}1
    {Δ}{{$\Delta$}}1
    {Θ}{{$\Theta$}}1
    {Λ}{{$\Lambda$}}1
    {Π}{{$\Pi$}}1
    {Σ}{{$\Sigma$}}1
    {Φ}{{$\Phi$}}1
    {Ψ}{{$\Psi$}}1
    {Ω}{{$\Omega$}}1
    {á}{{\'a}}1
    {é}{{\'e}}1
    {í}{{\'i}}1
    {ó}{{\'o}}1
    {ú}{{\'u}}1
    {Á}{{\'A}}1
    {É}{{\'E}}1
    {Í}{{\'I}}1
    {Ó}{{\'O}}1
    {Ú}{{\'U}}1
    {ä}{{\"a}}1
    {ë}{{\"e}}1
    {ï}{{\"i}}1
    {ö}{{\"o}}1
    {ü}{{\"u}}1
    {Ä}{{\"A}}1
    {Ë}{{\"E}}1
    {Ï}{{\"I}}1
    {Ö}{{\"O}}1
    {Ü}{{\"U}}1
    {ñ}{{\~n}}1
    {Ñ}{{\~N}}1
    {ç}{{\c{c}}}1
    {Ç}{{\c{C}}}1
    {¿}{{?`}}1
    {¡}{{!`}}1
    {à}{{\`a}}1
    {è}{{\`e}}1
    {ì}{{\`i}}1
    {ò}{{\`o}}1
    {ù}{{\`u}}1
    {À}{{\`A}}1
    {È}{{\`E}}1
    {Ì}{{\`I}}1
    {Ò}{{\`O}}1
    {Ù}{{\`U}}1
    {-}{{-}}1
    {=}{{=\allowbreak}}1  % <--- ESTA LÍNEA ES EL TRUCO PARA CORTAR LOS '===='
    % {#}{{\#}}1 
}


% ===============================================
% ESTILO 2: ELEGANTE CON BORDES REDONDEADOS
% ===============================================

% Colores para estilo elegante
\definecolor{lightblue}{rgb}{0.93,0.95,1}
\definecolor{darkblue}{rgb}{0.1,0.2,0.5}
\definecolor{mediumblue}{rgb}{0.2,0.4,0.8}
\definecolor{darkgreen}{rgb}{0,0.5,0}
\definecolor{darkred}{rgb}{0.6,0,0}

\lstdefinestyle{elegant}{
    backgroundcolor=\color{lightblue},
    commentstyle=\color{darkgreen}\itshape,
    keywordstyle=\color{darkblue}\bfseries,
    numberstyle=\tiny\color{gray},
    stringstyle=\color{darkred},
    basicstyle=\ttfamily\small,
    breakatwhitespace=false,
    breaklines=true,
    captionpos=t,
    keepspaces=true,
    numbers=left,
    numbersep=8pt,
    showspaces=false,
    showstringspaces=false,
    showtabs=false,
    tabsize=4,
    frame=single,
    frameround=tttt,
    framesep=10pt,
    xleftmargin=15pt,
    xrightmargin=15pt,
    aboveskip=15pt,
    belowskip=15pt,
    columns=flexible
}

% ===============================================
% ESTILO 3: PROFESIONAL CORPORATIVO
% ===============================================

% Colores corporativos
\definecolor{corporatebg}{rgb}{0.98,0.98,0.98}
\definecolor{corporateblue}{rgb}{0.07,0.29,0.49}
\definecolor{corporategray}{rgb}{0.4,0.4,0.4}
\definecolor{corporategreen}{rgb}{0.13,0.55,0.13}
\definecolor{corporatered}{rgb}{0.8,0.2,0.2}

\lstdefinestyle{corporate}{
    backgroundcolor=\color{corporatebg},
    commentstyle=\color{corporategreen}\slshape,
    keywordstyle=\color{corporateblue}\bfseries,
    numberstyle=\scriptsize\color{corporategray},
    stringstyle=\color{corporatered},
    basicstyle=\ttfamily\footnotesize,
    breakatwhitespace=false,
    breaklines=true,
    captionpos=b,
    keepspaces=true,
    numbers=left,
    numbersep=12pt,
    showspaces=false,
    showstringspaces=false,
    showtabs=false,
    tabsize=3,
    frame=leftline,
    framerule=3pt,
    rulecolor=\color{corporateblue},
    xleftmargin=25pt,
    aboveskip=20pt,
    belowskip=20pt,
    lineskip=1pt
}

% ===============================================
% ESTILO 4: MODERNO CON SOMBRAS
% ===============================================

% Colores modernos
\definecolor{modernbg}{rgb}{0.97,0.97,0.97}
\definecolor{moderngray}{rgb}{0.3,0.3,0.3}
\definecolor{modernpurple}{rgb}{0.5,0.2,0.8}
\definecolor{modernteal}{rgb}{0,0.5,0.5}
\definecolor{modernorange}{rgb}{0.8,0.4,0}

\lstdefinestyle{modern}{
    backgroundcolor=\color{modernbg},
    commentstyle=\color{modernteal}\itshape,
    keywordstyle=\color{modernpurple}\bfseries,
    numberstyle=\tiny\color{moderngray},
    stringstyle=\color{modernorange},
    basicstyle=\ttfamily\small,
    breakatwhitespace=false,
    breaklines=true,
    captionpos=t,
    keepspaces=true,
    numbers=left,
    numbersep=10pt,
    showspaces=false,
    showstringspaces=false,
    showtabs=false,
    tabsize=4,
    frame=tb,
    framerule=2pt,
    rulecolor=\color{modernpurple},
    xleftmargin=20pt,
    xrightmargin=20pt,
    aboveskip=25pt,
    belowskip=25pt
}

% ===============================================
% CONFIGURACIÓN PARA DIFERENTES LENGUAJES
% ===============================================

% Python
\lstdefinestyle{python}{
    language=Python,
    style=elegant,
    morekeywords={True,False,None,self,cls,def,class,import,from,as,with,yield,async,await},
    morecomment=[l]{\#},
    morestring=[b]',
    morestring=[b]"
}

% Java
\lstdefinestyle{java}{
    language=Java,
    style=corporate,
    morekeywords={var,record,sealed,permits,non-sealed}
}

% C++
\lstdefinestyle{cpp}{
    language=C++,
    style=modern,
    morekeywords={constexpr,nullptr,auto,decltype,override,final}
}

% JavaScript
\lstdefinestyle{javascript}{
    language=Java,
    style=elegant,
    morekeywords={let,const,var,function,class,extends,import,export,default,async,await,yield},
    morecomment=[l]{//},
    morecomment=[s]{/*}{*/},
    morestring=[b]',
    morestring=[b]",
    morestring=[b]`
}

% ===============================================
% EJEMPLOS DE USO
% ===============================================

% Para usar el estilo por defecto:
% \begin{lstlisting}
% código aquí
% \end{lstlisting}

% Para usar un estilo específico:
% \begin{lstlisting}[style=elegant]
% código aquí
% \end{lstlisting}

% Para incluir un archivo con estilo específico:
% \lstinputlisting[style=python]{archivo.py}

% Para código inline:
% \lstinline[style=modern]{código inline}

% ===============================================
% CONFIGURACIÓN ADICIONAL PARA TÍTULOS Y CARACTERES
% ===============================================

% Personalizar el formato de los títulos de los listados
\renewcommand\lstlistingname{Código}
\renewcommand\lstlistlistingname{Lista de Códigos}

% Configurar el formato del título con soporte para tildes
\lstset{
    %title=\lstname,
    captionpos=t,
    abovecaptionskip=10pt,
    belowcaptionskip=5pt,
    % Configuración global para caracteres especiales
    inputencoding=utf8,
    extendedchars=true
}

% ===============================================
% COMANDOS PERSONALIZADOS ÚTILES
% ===============================================

% Comando para código inline con soporte automático de tildes
\newcommand{\codeinline}[2][modern]{\lstinline[style=#1,inputencoding=utf8,extendedchars=true]{#2}}

% Comando para bloques de código con título personalizado
\newcommand{\codeblock}[3][elegant]{%
    \begin{lstlisting}[style=#1,caption={#2},label={lst:#2},inputencoding=utf8,extendedchars=true]
    #3
    \end{lstlisting}
}

% Comando para incluir archivos con configuración automática
\newcommand{\includecode}[3][python]{%
    \lstinputlisting[style=#1,caption={#3},label={lst:#3},inputencoding=utf8,extendedchars=true]{#2}
}

% ===============================================
% CONFIGURACIONES ESPECIALES PARA IDIOMAS
% ===============================================

% Configuración específica para código en español
\lstdefinestyle{español}{
    style=elegant,
    inputencoding=utf8,
    extendedchars=true,
    % Palabras clave en español para pseudocódigo
    morekeywords={función,procedimiento,inicio,fin,si,entonces,sino,mientras,para,hasta,hacer,repetir,caso,segun,verdadero,falso,entero,real,caracter,cadena,booleano,leer,escribir,imprimir}
}

% Configuración para comentarios multilíngües
\lstset{
    morecomment=[l]{//\ },
    morecomment=[l]{\#\ },
    morecomment=[s]{/*}{*/},
    morecomment=[s]{}
}

% ===============================================
% CONFIGURACIÓN PARA DIFERENTES LENGUAJES
% ===============================================

% Python
\lstdefinestyle{style1}{
    language=Python,
    style=elegant,
    morekeywords={True,False,None,self,cls,def,class,import,from,as,with,yield,async,await},
    morecomment=[l]{\#},
    morestring=[b]',
    morestring=[b]",
    % Soporte para caracteres especiales
    inputencoding=utf8,
    extendedchars=true
}

% Java
\lstdefinestyle{style2}{
    language=Java,
    style=corporate,
    morekeywords={var,record,sealed,permits,non-sealed},
    % Soporte para caracteres especiales
    inputencoding=utf8,
    extendedchars=true
}

% C++
\lstdefinestyle{style3}{
    language=C++,
    style=modern,
    morekeywords={constexpr,nullptr,auto,decltype,override,final},
    % Soporte para caracteres especiales
    inputencoding=utf8,
    extendedchars=true
}

\lstdefinelanguage{GDScript}{
  keywords={func, var, extends, class_name, if, else, for, while, return, match, in, and, or, not, break, continue, pass},
  sensitive=true,
  morecomment=[l]{\#},
  morestring=[b]",
  morestring=[b]',
}

\lstdefinestyle{gdstyle}{
  language=GDScript,
  basicstyle=\ttfamily\small,
  keywordstyle=\color{blue}\bfseries,
  commentstyle=\color{gray},
  stringstyle=\color{red!60!black},
  numbers=left,
  numberstyle=\tiny\color{gray},
  breaklines=true,
  frame=single,
  tabsize=2,
}


% ===========================
% Estilo global de tablas
% ===========================

\usepackage{booktabs}   % reglas profesionales
\usepackage{colortbl}   % color en filas
\usepackage{xcolor}     % colores
\usepackage{float}      % [H]

% Color de filas alternadas
% \rowcolors{2}{gray!10}{white}

% % Espacio vertical entre filas
% \renewcommand{\arraystretch}{1.2}

% % Cambiar el tamaño de columna por defecto
% \setlength{\tabcolsep}{8pt}

% % Redefinir tabla para que todas las tablas tengan el estilo
% \let\oldtabular\tabular
% \let\endoldtabular\endtabular
% \renewenvironment{tabular}[1]{%
%   \oldtabular{#1}%
% }{%
%   \endoldtabular
% }

% \usepackage{longtable,booktabs,xcolor}
% \rowcolors{2}{gray!10}{white}   % filas alternadas
% \renewcommand{\arraystretch}{1.2} % espacio vertical entre filas

% % Mostrar siempre el número de la tabla
% \usepackage{caption}
% \captionsetup[table]{labelformat=default, labelsep=colon, textfont=bf}


% ===========================
% Estilos para tikz y figures
% ===========================

\usepackage{caption}
\captionsetup{
    font={it},       % fuente en cursiva
    labelfont={},  % etiqueta ("Figura 1") en negrita
    textfont={it},   % texto del caption en cursiva
    justification=centering,  % centra el texto (opcional)
    font={small},    % tamaño de fuente pequeño
}

\usepackage{tikz}
\usetikzlibrary{positioning}

\tikzset{
  state/.style={
    draw,
    circle,
    minimum size=1cm,
    thick,
    fill=yellow!20
  },
  block/.style={
    rectangle,
    draw,
    fill=blue!10,
    rounded corners,
    text centered,
    minimum height=1cm,
    minimum width=2cm,
    thick
  },
  none/.style={
    draw=none,
    fill=none,
    text centered
  },
  error/.style={
    draw,
    circle,
    minimum size=1cm,
    thick,
    fill=red!30
  },
  initial text={}
}   % estilos de secciones, etc.

% Codificación y lengua
\usepackage[utf8]{inputenc}
\usepackage[spanish]{babel}
\usepackage[T1]{fontenc}

% Matemáticas y símbolos
\usepackage{amsmath, amssymb, nicefrac, eurosym, aligned-overset}

% Gráficos y figuras
\usepackage{graphicx, subcaption, tikz, pgfplots}

% Tablas
\usepackage{booktabs, colortbl, xcolor, multirow, makecell}

% Texto y estilos
\usepackage{titlesec, caption, enumitem, color, fancyhdr}

% Otros
\usepackage{hyperref, geometry, float, pdfpages, lastpage, listings, tcolorbox}
\hypersetup{
    colorlinks=true,
    linkcolor=blue,
    urlcolor=blue,
    citecolor=blue
}

\usetikzlibrary{arrows.meta}
\pgfplotsset{compat=1.18}

  % si tienes paquetes personalizados
% ===========================
% Diseño general
% ===========================
\setstretch{1.15} % interlineado
\setlength{\parskip}{0.5em} % espacio entre párrafos
\setlength{\parindent}{0pt} % sin sangría

% ===========================
% Estilo de capítulos y secciones (titlesec)
% ===========================
\titleformat{\chapter}[display]
  {\bfseries\Huge}
  {\filleft\Large\scshape Capítulo \thechapter}
  {1ex}
  {\titlerule[1pt]\vspace{1ex}\filright}
  [\vspace{1ex}\titlerule]

\titlespacing*{\chapter}{0pt}{0pt}{2em}

\titleformat{\section}
  {\Large\bfseries}
  {\thesection}{1em}{}

\titleformat{\subsection}
  {\large\bfseries}
  {\thesubsection}{1em}{}

\titleformat{\subsubsection}
  {\normalsize\bfseries\itshape}
  {\thesubsubsection}{1em}{}

% ===========================
% Encabezados y pies de página (fancyhdr)
% ===========================
\pagestyle{fancy}
\fancyhf{} % limpia
\fancyhead[L]{\small\scshape\nouppercase{\leftmark}} % sección/capítulo en mayúsculas pequeñas
\fancyhead[R]{\small\thepage}                        % número de página
%\fancyfoot[C]{\scriptsize\itshape Apuntes de la carrera} % texto fijo abajo en cursiva
% Encabezados y pies de página personalizados
% \fancyfoot[L]{\scriptsize\itshape Nombre de la asignatura} % pie de página izquierdo en cursiva
\fancyfoot[R]{\normalsize Ismael Sallami Moreno}        % pie de página derecho con el nombre del autor

% Línea bajo el encabezado
\renewcommand{\headrulewidth}{0.5pt} % línea más gruesa en el encabezado
% Línea en el pie
\renewcommand{\footrulewidth}{0.4pt} % línea fina en el pie
\renewcommand{\sectionmark}[1]{%
  \markboth{\thesection\quad #1}{}%
}

% ===========================
% Numeración de elementos
% ===========================
\numberwithin{equation}{chapter} % ecuaciones numeradas por capítulo
\numberwithin{figure}{chapter}   % figuras numeradas por capítulo
\numberwithin{table}{chapter}    % tablas numeradas por capítulo

% ===========================
% Listas y enumeraciones
% ===========================
\setlist[itemize]{label=--, left=1.5em}
\setlist[enumerate]{label=\arabic*), left=1.5em}

% ===========================
% Estilo de citas y bibliografía
% ===========================
\DefineBibliographyStrings{spanish}{%
  references = {Bibliografía},
}

% ===========================
% Entornos personalizados
% ===========================
\newtheoremstyle{cajita} % nombre del estilo
  {1em}   % espacio arriba
  {1em}   % espacio abajo
  {}      % fuente del cuerpo
  {}      % indentación
  {\bfseries} % fuente del título
  {.}     % puntuación tras título
  {0.5em} % espacio tras título
  {\thmname{#1}\thmnumber{ #2} \thmnote{(#3)}} % formato


\theoremstyle{cajita}
\newtheorem{teorema}{Teorema}[chapter]
\newtheorem{definicion}{Definición}[chapter]
\newtheorem{ejemplo}{Ejemplo}[chapter]
\newtheorem{proposicion}{Proposición}[chapter]
\newtheorem{demostracion}{Demostración}[chapter]
\newtheorem{corolario}{Corolario}[chapter]
\newtheorem{propuesta}{Propuesta}[chapter]


\newtheoremstyle{anotacionstyle} % nombre del estilo
  {1em}   % espacio arriba
  {1em}   % espacio abajo
  {}      % fuente del cuerpo (sin cursiva)
  {}      % indentación
  {\itshape} % fuente del título (Nota en cursiva)
  {.}     % puntuación tras título
  {0.5em} % espacio tras título
  {\thmname{\itshape#1}\thmnumber{ #2} \thmnote{(#3)}} % formato (solo Nota en cursiva)

\theoremstyle{anotacionstyle}
\newtheorem{anotacion}{Nota}[chapter]

% ===========================
% Configuración de lstlisting
% ===========================

% ===============================================
% ESTILO 1: MODERNO Y MINIMALISTA
% ===============================================

% Definir colores personalizados
\definecolor{codegreen}{rgb}{0,0.6,0}
\definecolor{codegray}{rgb}{0.5,0.5,0.5}
\definecolor{codepurple}{rgb}{0.58,0,0.82}
\definecolor{backcolour}{rgb}{0.95,0.95,0.92}
\definecolor{framecolor}{rgb}{0.8,0.8,0.8}

\lstset{
  backgroundcolor=\color{backcolour},   
  commentstyle=\color{codegreen},
  keywordstyle=\color{magenta},
  numberstyle=\tiny\color{codegray},
  stringstyle=\color{codepurple},
  basicstyle=\ttfamily\footnotesize,
  breakatwhitespace=false,         
  breaklines=true,                 
  captionpos=b,                    
  keepspaces=true,                 
  numbers=left,                    
  numbersep=5pt,                  
  showspaces=false,                
  showstringspaces=false,
  showtabs=false,                  
  tabsize=2,
  frame=shadowbox,
  frameround=tttt,
  rulecolor=\color{framecolor},
  rulesepcolor=\color{framecolor},
  xleftmargin=20pt,
  xrightmargin=20pt,
  aboveskip=20pt,
  belowskip=20pt,
  inputencoding=utf8,
  extendedchars=true,
  literate=
    {←}{{$\leftarrow$}}1
    {→}{{$\rightarrow$}}1
    {↑}{{$\uparrow$}}1
    {↓}{{$\downarrow$}}1
    {↔}{{$\leftrightarrow$}}1
    {⇒}{{$\Rightarrow$}}1
    {⇐}{{$\Leftarrow$}}1
    {⇔}{{$\Leftrightarrow$}}1
    {α}{{$\alpha$}}1
    {β}{{$\beta$}}1
    {γ}{{$\gamma$}}1
    {δ}{{$\delta$}}1
    {ε}{{$\epsilon$}}1
    {θ}{{$\theta$}}1
    {λ}{{$\lambda$}}1
    {μ}{{$\mu$}}1
    {π}{{$\pi$}}1
    {σ}{{$\sigma$}}1
    {φ}{{$\phi$}}1
    {ψ}{{$\psi$}}1
    {ω}{{$\omega$}}1
    {Δ}{{$\Delta$}}1
    {Θ}{{$\Theta$}}1
    {Λ}{{$\Lambda$}}1
    {Π}{{$\Pi$}}1
    {Σ}{{$\Sigma$}}1
    {Φ}{{$\Phi$}}1
    {Ψ}{{$\Psi$}}1
    {Ω}{{$\Omega$}}1
    {á}{{\'a}}1
    {é}{{\'e}}1
    {í}{{\'i}}1
    {ó}{{\'o}}1
    {ú}{{\'u}}1
    {Á}{{\'A}}1
    {É}{{\'E}}1
    {Í}{{\'I}}1
    {Ó}{{\'O}}1
    {Ú}{{\'U}}1
    {ä}{{\"a}}1
    {ë}{{\"e}}1
    {ï}{{\"i}}1
    {ö}{{\"o}}1
    {ü}{{\"u}}1
    {Ä}{{\"A}}1
    {Ë}{{\"E}}1
    {Ï}{{\"I}}1
    {Ö}{{\"O}}1
    {Ü}{{\"U}}1
    {ñ}{{\~n}}1
    {Ñ}{{\~N}}1
    {ç}{{\c{c}}}1
    {Ç}{{\c{C}}}1
    {¿}{{?`}}1
    {¡}{{!`}}1
    {à}{{\`a}}1
    {è}{{\`e}}1
    {ì}{{\`i}}1
    {ò}{{\`o}}1
    {ù}{{\`u}}1
    {À}{{\`A}}1
    {È}{{\`E}}1
    {Ì}{{\`I}}1
    {Ò}{{\`O}}1
    {Ù}{{\`U}}1
    {-}{{-}}1
    {=}{{=\allowbreak}}1  % <--- ESTA LÍNEA ES EL TRUCO PARA CORTAR LOS '===='
    % {#}{{\#}}1 
}


% ===============================================
% ESTILO 2: ELEGANTE CON BORDES REDONDEADOS
% ===============================================

% Colores para estilo elegante
\definecolor{lightblue}{rgb}{0.93,0.95,1}
\definecolor{darkblue}{rgb}{0.1,0.2,0.5}
\definecolor{mediumblue}{rgb}{0.2,0.4,0.8}
\definecolor{darkgreen}{rgb}{0,0.5,0}
\definecolor{darkred}{rgb}{0.6,0,0}

\lstdefinestyle{elegant}{
    backgroundcolor=\color{lightblue},
    commentstyle=\color{darkgreen}\itshape,
    keywordstyle=\color{darkblue}\bfseries,
    numberstyle=\tiny\color{gray},
    stringstyle=\color{darkred},
    basicstyle=\ttfamily\small,
    breakatwhitespace=false,
    breaklines=true,
    captionpos=t,
    keepspaces=true,
    numbers=left,
    numbersep=8pt,
    showspaces=false,
    showstringspaces=false,
    showtabs=false,
    tabsize=4,
    frame=single,
    frameround=tttt,
    framesep=10pt,
    xleftmargin=15pt,
    xrightmargin=15pt,
    aboveskip=15pt,
    belowskip=15pt,
    columns=flexible
}

% ===============================================
% ESTILO 3: PROFESIONAL CORPORATIVO
% ===============================================

% Colores corporativos
\definecolor{corporatebg}{rgb}{0.98,0.98,0.98}
\definecolor{corporateblue}{rgb}{0.07,0.29,0.49}
\definecolor{corporategray}{rgb}{0.4,0.4,0.4}
\definecolor{corporategreen}{rgb}{0.13,0.55,0.13}
\definecolor{corporatered}{rgb}{0.8,0.2,0.2}

\lstdefinestyle{corporate}{
    backgroundcolor=\color{corporatebg},
    commentstyle=\color{corporategreen}\slshape,
    keywordstyle=\color{corporateblue}\bfseries,
    numberstyle=\scriptsize\color{corporategray},
    stringstyle=\color{corporatered},
    basicstyle=\ttfamily\footnotesize,
    breakatwhitespace=false,
    breaklines=true,
    captionpos=b,
    keepspaces=true,
    numbers=left,
    numbersep=12pt,
    showspaces=false,
    showstringspaces=false,
    showtabs=false,
    tabsize=3,
    frame=leftline,
    framerule=3pt,
    rulecolor=\color{corporateblue},
    xleftmargin=25pt,
    aboveskip=20pt,
    belowskip=20pt,
    lineskip=1pt
}

% ===============================================
% ESTILO 4: MODERNO CON SOMBRAS
% ===============================================

% Colores modernos
\definecolor{modernbg}{rgb}{0.97,0.97,0.97}
\definecolor{moderngray}{rgb}{0.3,0.3,0.3}
\definecolor{modernpurple}{rgb}{0.5,0.2,0.8}
\definecolor{modernteal}{rgb}{0,0.5,0.5}
\definecolor{modernorange}{rgb}{0.8,0.4,0}

\lstdefinestyle{modern}{
    backgroundcolor=\color{modernbg},
    commentstyle=\color{modernteal}\itshape,
    keywordstyle=\color{modernpurple}\bfseries,
    numberstyle=\tiny\color{moderngray},
    stringstyle=\color{modernorange},
    basicstyle=\ttfamily\small,
    breakatwhitespace=false,
    breaklines=true,
    captionpos=t,
    keepspaces=true,
    numbers=left,
    numbersep=10pt,
    showspaces=false,
    showstringspaces=false,
    showtabs=false,
    tabsize=4,
    frame=tb,
    framerule=2pt,
    rulecolor=\color{modernpurple},
    xleftmargin=20pt,
    xrightmargin=20pt,
    aboveskip=25pt,
    belowskip=25pt
}

% ===============================================
% CONFIGURACIÓN PARA DIFERENTES LENGUAJES
% ===============================================

% Python
\lstdefinestyle{python}{
    language=Python,
    style=elegant,
    morekeywords={True,False,None,self,cls,def,class,import,from,as,with,yield,async,await},
    morecomment=[l]{\#},
    morestring=[b]',
    morestring=[b]"
}

% Java
\lstdefinestyle{java}{
    language=Java,
    style=corporate,
    morekeywords={var,record,sealed,permits,non-sealed}
}

% C++
\lstdefinestyle{cpp}{
    language=C++,
    style=modern,
    morekeywords={constexpr,nullptr,auto,decltype,override,final}
}

% JavaScript
\lstdefinestyle{javascript}{
    language=Java,
    style=elegant,
    morekeywords={let,const,var,function,class,extends,import,export,default,async,await,yield},
    morecomment=[l]{//},
    morecomment=[s]{/*}{*/},
    morestring=[b]',
    morestring=[b]",
    morestring=[b]`
}

% ===============================================
% EJEMPLOS DE USO
% ===============================================

% Para usar el estilo por defecto:
% \begin{lstlisting}
% código aquí
% \end{lstlisting}

% Para usar un estilo específico:
% \begin{lstlisting}[style=elegant]
% código aquí
% \end{lstlisting}

% Para incluir un archivo con estilo específico:
% \lstinputlisting[style=python]{archivo.py}

% Para código inline:
% \lstinline[style=modern]{código inline}

% ===============================================
% CONFIGURACIÓN ADICIONAL PARA TÍTULOS Y CARACTERES
% ===============================================

% Personalizar el formato de los títulos de los listados
\renewcommand\lstlistingname{Código}
\renewcommand\lstlistlistingname{Lista de Códigos}

% Configurar el formato del título con soporte para tildes
\lstset{
    %title=\lstname,
    captionpos=t,
    abovecaptionskip=10pt,
    belowcaptionskip=5pt,
    % Configuración global para caracteres especiales
    inputencoding=utf8,
    extendedchars=true
}

% ===============================================
% COMANDOS PERSONALIZADOS ÚTILES
% ===============================================

% Comando para código inline con soporte automático de tildes
\newcommand{\codeinline}[2][modern]{\lstinline[style=#1,inputencoding=utf8,extendedchars=true]{#2}}

% Comando para bloques de código con título personalizado
\newcommand{\codeblock}[3][elegant]{%
    \begin{lstlisting}[style=#1,caption={#2},label={lst:#2},inputencoding=utf8,extendedchars=true]
    #3
    \end{lstlisting}
}

% Comando para incluir archivos con configuración automática
\newcommand{\includecode}[3][python]{%
    \lstinputlisting[style=#1,caption={#3},label={lst:#3},inputencoding=utf8,extendedchars=true]{#2}
}

% ===============================================
% CONFIGURACIONES ESPECIALES PARA IDIOMAS
% ===============================================

% Configuración específica para código en español
\lstdefinestyle{español}{
    style=elegant,
    inputencoding=utf8,
    extendedchars=true,
    % Palabras clave en español para pseudocódigo
    morekeywords={función,procedimiento,inicio,fin,si,entonces,sino,mientras,para,hasta,hacer,repetir,caso,segun,verdadero,falso,entero,real,caracter,cadena,booleano,leer,escribir,imprimir}
}

% Configuración para comentarios multilíngües
\lstset{
    morecomment=[l]{//\ },
    morecomment=[l]{\#\ },
    morecomment=[s]{/*}{*/},
    morecomment=[s]{}
}

% ===============================================
% CONFIGURACIÓN PARA DIFERENTES LENGUAJES
% ===============================================

% Python
\lstdefinestyle{style1}{
    language=Python,
    style=elegant,
    morekeywords={True,False,None,self,cls,def,class,import,from,as,with,yield,async,await},
    morecomment=[l]{\#},
    morestring=[b]',
    morestring=[b]",
    % Soporte para caracteres especiales
    inputencoding=utf8,
    extendedchars=true
}

% Java
\lstdefinestyle{style2}{
    language=Java,
    style=corporate,
    morekeywords={var,record,sealed,permits,non-sealed},
    % Soporte para caracteres especiales
    inputencoding=utf8,
    extendedchars=true
}

% C++
\lstdefinestyle{style3}{
    language=C++,
    style=modern,
    morekeywords={constexpr,nullptr,auto,decltype,override,final},
    % Soporte para caracteres especiales
    inputencoding=utf8,
    extendedchars=true
}

\lstdefinelanguage{GDScript}{
  keywords={func, var, extends, class_name, if, else, for, while, return, match, in, and, or, not, break, continue, pass},
  sensitive=true,
  morecomment=[l]{\#},
  morestring=[b]",
  morestring=[b]',
}

\lstdefinestyle{gdstyle}{
  language=GDScript,
  basicstyle=\ttfamily\small,
  keywordstyle=\color{blue}\bfseries,
  commentstyle=\color{gray},
  stringstyle=\color{red!60!black},
  numbers=left,
  numberstyle=\tiny\color{gray},
  breaklines=true,
  frame=single,
  tabsize=2,
}


% ===========================
% Estilo global de tablas
% ===========================

\usepackage{booktabs}   % reglas profesionales
\usepackage{colortbl}   % color en filas
\usepackage{xcolor}     % colores
\usepackage{float}      % [H]

% Color de filas alternadas
% \rowcolors{2}{gray!10}{white}

% % Espacio vertical entre filas
% \renewcommand{\arraystretch}{1.2}

% % Cambiar el tamaño de columna por defecto
% \setlength{\tabcolsep}{8pt}

% % Redefinir tabla para que todas las tablas tengan el estilo
% \let\oldtabular\tabular
% \let\endoldtabular\endtabular
% \renewenvironment{tabular}[1]{%
%   \oldtabular{#1}%
% }{%
%   \endoldtabular
% }

% \usepackage{longtable,booktabs,xcolor}
% \rowcolors{2}{gray!10}{white}   % filas alternadas
% \renewcommand{\arraystretch}{1.2} % espacio vertical entre filas

% % Mostrar siempre el número de la tabla
% \usepackage{caption}
% \captionsetup[table]{labelformat=default, labelsep=colon, textfont=bf}


% ===========================
% Estilos para tikz y figures
% ===========================

\usepackage{caption}
\captionsetup{
    font={it},       % fuente en cursiva
    labelfont={},  % etiqueta ("Figura 1") en negrita
    textfont={it},   % texto del caption en cursiva
    justification=centering,  % centra el texto (opcional)
    font={small},    % tamaño de fuente pequeño
}

\usepackage{tikz}
\usetikzlibrary{positioning}

\tikzset{
  state/.style={
    draw,
    circle,
    minimum size=1cm,
    thick,
    fill=yellow!20
  },
  block/.style={
    rectangle,
    draw,
    fill=blue!10,
    rounded corners,
    text centered,
    minimum height=1cm,
    minimum width=2cm,
    thick
  },
  none/.style={
    draw=none,
    fill=none,
    text centered
  },
  error/.style={
    draw,
    circle,
    minimum size=1cm,
    thick,
    fill=red!30
  },
  initial text={}
}   % estilos de secciones, etc.
% % aquí van los comandos personalizados
% Comando para incluir imágenes
\newcommand{\incluirimagen}[3][]{%
\begin{figure}[H]
    \centering
    \includegraphics[width=\linewidth,#1]{#2}
    \caption{#3}
    \label{fig:#2}
\end{figure}
}

% comando para ejercicios con fondo
\newtheoremstyle{ejerciciostyle}
  {10pt}   % Espacio arriba
  {10pt}   % Espacio abajo
  %{\itshape} % Fuente del cuerpo
  {}
  {}       % Sangría
  {\bfseries} % Fuente del encabezado
  {}      % Puntuación tras encabezado
  { }      % Espacio tras encabezado
  {\thmname{#1} \thmnumber{#2}. \thmnote{#3}}


% % comando formal para enunciado de ejercicios
% \theoremstyle{ejerciciostyle}
% \newtheorem{ejercicio}{Ejercicio}[chapter]

\theoremstyle{ejerciciostyle}
\newtheorem{ejercicio}{Ejercicio}[section]

\renewcommand{\theejercicio}{\thechapter.\arabic{section}.\arabic{ejercicio}}


% comando formal para soluciones
\theoremstyle{remark}
\newtheorem{solucion}{Solución}[ejercicio]

\renewcommand{\thesolucion}{\thechapter.\arabic{section}.\arabic{ejercicio}}

% Comando para dos imágenes en paralelo
\newcommand{\dosimagenes}[6]{%
    \begin{figure}[h!]
        \centering
        \begin{minipage}{0.48\linewidth}
            \centering
            \includegraphics[width=\linewidth]{#1}
            \caption{#2}
            \label{#5}
        \end{minipage}\hfill
        \begin{minipage}{0.48\linewidth}
            \centering
            \includegraphics[width=\linewidth]{#3}
            \caption{#4}
            \label{#6}
        \end{minipage}
    \end{figure}
}

% \dosimagenes{media/fondo.jpg}{Descripción 1}{media/fondo.jpg}{Descripción 2}{fig:descripcion1}{fig:descripcion2}

% \ref{fig:descripcion1} es la mejor
% \ref{fig:descripcion2} es la mejor

\newcommand{\portadaimg}{\VAR{portadaimg}}

% Comando para crear una nota estilo información
% \newcommand{\nota}[2]{%
% \begin{tcolorbox}[colframe=blue!75!black, colback=blue!5!white, title=\textbf{#1}]
%     #2
% \end{tcolorbox}
% }
\newtheorem{nota}{Nota}[chapter]


% Comando para poner dos códigos en paralelo
\newcommand{\doscodigos}[4]{%
  \noindent
  \begin{minipage}{0.48\linewidth}
    \lstset{language=#1}
    \lstinputlisting{#2}
  \end{minipage}\hfill
  \begin{minipage}{0.48\linewidth}
    \lstset{language=#3}
    \lstinputlisting{#4}
  \end{minipage}
}

% Comando para poner un solo código
\newcommand{\uncodigo}[2]{%
  \begin{lstlisting}[language=#1]
#2
  \end{lstlisting}
}


% % Listas de archivos (sin guiones en los nombres de macros)
% \newcommand{\listagdfilesSesion2Mallas2D}{cargatexturas.gd, envioinmediato.gd, malla2dcontexturas.gd, mallaconcoloresdevertices.gd, mallanoindentada.gd}
% \newcommand{\listagdfilesSesion2Mallas3D}{mallaindexada3d.gd, materialconcolordeplano.gd, materialconcoloresdevertices.gd, tablas.gd}

% % Macro que recorre una lista de archivos en un subdirectorio
% \newcommand{\includegdfiles}[2]{%
%   % #1 = subdirectorio
%   % #2 = nombre de la lista de archivos
%   \foreach \filename in #2 {%
%     \includecode[gdstyle]{code/#1/\filename}{\filename}
%   }%
% }



% Comando para ejercicio resuelto
\newtheoremstyle{ejercicioresueltostyle}
    {10pt}   % Espacio arriba
    {10pt}   % Espacio abajo
    {\itshape} % Fuente del cuerpo
    {}       % Sangría
    {\bfseries} % Fuente del encabezado
    {}      % Puntuación tras encabezado
    { }      % Espacio tras encabezado
    {\thmname{#1} \thmnumber{#2}. \thmnote{#3}}

\theoremstyle{ejercicioresueltostyle}
\newtheorem{ejercicioresuelto}{Ejercicio Resuelto}[section]

\renewcommand{\theejercicioresuelto}{\thechapter.\arabic{section}.\arabic{ejercicioresuelto}}


%======================================================================== 
% PRACTICAS
%========================================================================

% Comando para definir un tema
\newcommand{\tema}[1]{%
  \section{#1}
  \addcontentsline{toc}{section}{#1}
}
\usepackage{tikz}
\usepackage{graphicx} % necesario para \resizebox
\usepackage{etoolbox}

% ======== NODOS ========
\newcommand{\nodo}[4][]{\node[state, #1] (#2) at (#3) {$#4$};}
% Uso: \nodo[initial,accepting]{q0}{0,0}{q_0}

% ======== FLECHAS ========
\newcommand{\flecha}[4][]{\draw[->, #1] (#2) -- (#3) node[midway, above] {#4};}
% Uso: \flecha{q0}{q1}{0} o \flecha[bend left]{q1}{q2}{1}

\newcommand{\flechaabajo}[4][]{\draw[->, #1] (#2) -- (#3) node[midway, below, yshift=-6pt] {#4};}
% Igual que \flecha pero con etiqueta abajo
\newcommand{\flechaarriba}[4][]{\draw[->, #1] (#2) -- (#3) node[midway, above, yshift=6pt] {#4};}
% Igual que \flecha pero con etiqueta arriba
\newcommand{\flechaderecha}[4][]{\draw[->, #1] (#2) -- (#3) node[midway, right] {#4};}
% Igual que \flecha pero con etiqueta a la derecha
\newcommand{\flechaiquierda}[4][]{\draw[->, #1] (#2) -- (#3) node[midway, left] {#4};}
% Igual que \flecha pero con etiqueta a la izquierda

\newcommand{\curva}[5][]{\draw[->, bend #1] (#2) to node[midway, #5] {#4} (#3);}
% Uso: \curva[left]{q1}{q2}{1}{below}


\newcommand{\loopa}[3]{\draw[->] (#1) edge[loop above] node {#2} (#1);}
\newcommand{\loopb}[3]{\draw[->] (#1) edge[loop below] node {#2} (#1);}
\newcommand{\loopr}[3]{\draw[->] (#1) edge[loop right] node {#2} (#1);}
\newcommand{\loopl}[3]{\draw[->] (#1) edge[loop left] node {#2} (#1);}
% Uso: \loopa{q1}{0}

% ======== ESTILOS ESPECIALES ========
\tikzset{
    error/.style={state, fill=red!20, draw=red!80!black},
    final/.style={state, accepting, fill=green!15!white, draw=green!60!black}
}
% Uso: \nodo[error]{qe}{5,0}{q_e}  o \nodo[final]{qf}{7,0}{q_f}


\newcommand{\pa}{1}      % ejemplo de valor
\newcommand{\pUno}{2}
\newcommand{\pDos}{3}
  % comandos LaTeX propios
% aquí van los comandos personalizados
% Comando para incluir imágenes
\newcommand{\incluirimagen}[3][]{%
\begin{figure}[H]
    \centering
    \includegraphics[width=\linewidth,#1]{#2}
    \caption{#3}
    \label{fig:#2}
\end{figure}
}

% comando para ejercicios con fondo
\newtheoremstyle{ejerciciostyle}
  {10pt}   % Espacio arriba
  {10pt}   % Espacio abajo
  %{\itshape} % Fuente del cuerpo
  {}
  {}       % Sangría
  {\bfseries} % Fuente del encabezado
  {}      % Puntuación tras encabezado
  { }      % Espacio tras encabezado
  {\thmname{#1} \thmnumber{#2}. \thmnote{#3}}


% % comando formal para enunciado de ejercicios
% \theoremstyle{ejerciciostyle}
% \newtheorem{ejercicio}{Ejercicio}[chapter]

\theoremstyle{ejerciciostyle}
\newtheorem{ejercicio}{Ejercicio}[section]

\renewcommand{\theejercicio}{\thechapter.\arabic{section}.\arabic{ejercicio}}


% comando formal para soluciones
\theoremstyle{remark}
\newtheorem{solucion}{Solución}[ejercicio]

\renewcommand{\thesolucion}{\thechapter.\arabic{section}.\arabic{ejercicio}}

% Comando para dos imágenes en paralelo
\newcommand{\dosimagenes}[6]{%
    \begin{figure}[h!]
        \centering
        \begin{minipage}{0.48\linewidth}
            \centering
            \includegraphics[width=\linewidth]{#1}
            \caption{#2}
            \label{#5}
        \end{minipage}\hfill
        \begin{minipage}{0.48\linewidth}
            \centering
            \includegraphics[width=\linewidth]{#3}
            \caption{#4}
            \label{#6}
        \end{minipage}
    \end{figure}
}

% \dosimagenes{media/fondo.jpg}{Descripción 1}{media/fondo.jpg}{Descripción 2}{fig:descripcion1}{fig:descripcion2}

% \ref{fig:descripcion1} es la mejor
% \ref{fig:descripcion2} es la mejor

\newcommand{\portadaimg}{\VAR{portadaimg}}

% Comando para crear una nota estilo información
% \newcommand{\nota}[2]{%
% \begin{tcolorbox}[colframe=blue!75!black, colback=blue!5!white, title=\textbf{#1}]
%     #2
% \end{tcolorbox}
% }
\newtheorem{nota}{Nota}[chapter]


% Comando para poner dos códigos en paralelo
\newcommand{\doscodigos}[4]{%
  \noindent
  \begin{minipage}{0.48\linewidth}
    \lstset{language=#1}
    \lstinputlisting{#2}
  \end{minipage}\hfill
  \begin{minipage}{0.48\linewidth}
    \lstset{language=#3}
    \lstinputlisting{#4}
  \end{minipage}
}

% Comando para poner un solo código
\newcommand{\uncodigo}[2]{%
  \begin{lstlisting}[language=#1]
#2
  \end{lstlisting}
}


% % Listas de archivos (sin guiones en los nombres de macros)
% \newcommand{\listagdfilesSesion2Mallas2D}{cargatexturas.gd, envioinmediato.gd, malla2dcontexturas.gd, mallaconcoloresdevertices.gd, mallanoindentada.gd}
% \newcommand{\listagdfilesSesion2Mallas3D}{mallaindexada3d.gd, materialconcolordeplano.gd, materialconcoloresdevertices.gd, tablas.gd}

% % Macro que recorre una lista de archivos en un subdirectorio
% \newcommand{\includegdfiles}[2]{%
%   % #1 = subdirectorio
%   % #2 = nombre de la lista de archivos
%   \foreach \filename in #2 {%
%     \includecode[gdstyle]{code/#1/\filename}{\filename}
%   }%
% }



% Comando para ejercicio resuelto
\newtheoremstyle{ejercicioresueltostyle}
    {10pt}   % Espacio arriba
    {10pt}   % Espacio abajo
    {\itshape} % Fuente del cuerpo
    {}       % Sangría
    {\bfseries} % Fuente del encabezado
    {}      % Puntuación tras encabezado
    { }      % Espacio tras encabezado
    {\thmname{#1} \thmnumber{#2}. \thmnote{#3}}

\theoremstyle{ejercicioresueltostyle}
\newtheorem{ejercicioresuelto}{Ejercicio Resuelto}[section]

\renewcommand{\theejercicioresuelto}{\thechapter.\arabic{section}.\arabic{ejercicioresuelto}}


%======================================================================== 
% PRACTICAS
%========================================================================

% Comando para definir un tema
\newcommand{\tema}[1]{%
  \section{#1}
  \addcontentsline{toc}{section}{#1}
}
\usepackage{tikz}
\usepackage{graphicx} % necesario para \resizebox
\usepackage{etoolbox}

% ======== NODOS ========
\newcommand{\nodo}[4][]{\node[state, #1] (#2) at (#3) {$#4$};}
% Uso: \nodo[initial,accepting]{q0}{0,0}{q_0}

% ======== FLECHAS ========
\newcommand{\flecha}[4][]{\draw[->, #1] (#2) -- (#3) node[midway, above] {#4};}
% Uso: \flecha{q0}{q1}{0} o \flecha[bend left]{q1}{q2}{1}

\newcommand{\flechaabajo}[4][]{\draw[->, #1] (#2) -- (#3) node[midway, below, yshift=-6pt] {#4};}
% Igual que \flecha pero con etiqueta abajo
\newcommand{\flechaarriba}[4][]{\draw[->, #1] (#2) -- (#3) node[midway, above, yshift=6pt] {#4};}
% Igual que \flecha pero con etiqueta arriba
\newcommand{\flechaderecha}[4][]{\draw[->, #1] (#2) -- (#3) node[midway, right] {#4};}
% Igual que \flecha pero con etiqueta a la derecha
\newcommand{\flechaiquierda}[4][]{\draw[->, #1] (#2) -- (#3) node[midway, left] {#4};}
% Igual que \flecha pero con etiqueta a la izquierda

\newcommand{\curva}[5][]{\draw[->, bend #1] (#2) to node[midway, #5] {#4} (#3);}
% Uso: \curva[left]{q1}{q2}{1}{below}


\newcommand{\loopa}[3]{\draw[->] (#1) edge[loop above] node {#2} (#1);}
\newcommand{\loopb}[3]{\draw[->] (#1) edge[loop below] node {#2} (#1);}
\newcommand{\loopr}[3]{\draw[->] (#1) edge[loop right] node {#2} (#1);}
\newcommand{\loopl}[3]{\draw[->] (#1) edge[loop left] node {#2} (#1);}
% Uso: \loopa{q1}{0}

% ======== ESTILOS ESPECIALES ========
\tikzset{
    error/.style={state, fill=red!20, draw=red!80!black},
    final/.style={state, accepting, fill=green!15!white, draw=green!60!black}
}
% Uso: \nodo[error]{qe}{5,0}{q_e}  o \nodo[final]{qf}{7,0}{q_f}


\newcommand{\pa}{1}      % ejemplo de valor
\newcommand{\pUno}{2}
\newcommand{\pDos}{3}





\usepackage{etoolbox}
\usetikzlibrary{shapes, arrows, positioning, calc}

\AtBeginEnvironment{center}{\vspace{1em}}
\AtEndEnvironment{center}{\vspace{1em}}

% ========================
% Configuración índice y listas
% ========================
\setlength{\cftbeforesecskip}{5pt}
\setlength{\headheight}{14pt}  % un poco más que 13.6pt

\renewcommand{\normalsize}{\fontsize{10}{12}\selectfont}

% Fix para listas de Pandoc
\providecommand{\tightlist}{%
  \setlength{\itemsep}{0pt}\setlength{\parskip}{0pt}}



%===============
% ESPACIOS
%===============

% --- Compactar secciones ---
\titlespacing*{\section}{0pt}{1.2ex plus 0.5ex minus 0.2ex}{0.8ex}
\titlespacing*{\subsection}{0pt}{1ex plus 0.3ex minus 0.2ex}{0.5ex}

% --- Compactar flotantes (figuras/tablas) ---
\setlength{\textfloatsep}{8pt}
\setlength{\intextsep}{6pt}
\setlength{\floatsep}{6pt}

% --- Compactar listas ---
\setlist{nosep}

% --- Espacio entre párrafos ---
\setlength{\parskip}{4pt}



%=======================
% fancy with parameters
%=======================
%\fancyfoot[L]{\scriptsize\itshape Informática Gráfica}
\fancyfoot[L]{\normalsize Informática
Gráfica} % pie de página izquierdo con tamaño normal

\setcounter{tocdepth}{1} % Muestra solo hasta subsecciones en el índice

%=======================
% reset chapter 0 with one more \part
%=======================

\makeatletter
\@addtoreset{chapter}{part}
\makeatother


% ========================
% Inicio del documento
% ========================
\begin{document}

% Cambiar puntos suspensivos en el índice
\renewcommand{\cftsecleader}{\cftdotfill{\cftdotsep}}

% Ajustar formato de secciones y subsecciones en el índice
\renewcommand{\cftsecfont}{\bfseries} % Secciones en negrita
\renewcommand{\cftsecpagefont}{\bfseries} % Números de página en negrita para secciones
\renewcommand{\cftsubsecfont}{\normalfont} % Subsecciones en formato normal
\renewcommand{\cftsubsecpagefont}{\normalfont} % Números de página en formato normal para subsecciones

% Espaciado entre entradas del índice
\setlength{\cftbeforesecskip}{8pt} % Espaciado antes de secciones
\setlength{\cftbeforesubsecskip}{4pt} % Espaciado antes de subsecciones



%% portada.tex
\begin{titlepage}
    \newgeometry{top=2cm,bottom=2cm,left=2.5cm,right=2.5cm} % márgenes personalizados
    
    % Fondo con transparencia
    \begin{tikzpicture}[remember picture,overlay]
        \node[opacity=0.15,inner sep=0pt] at (current page.center)
            {\includegraphics[width=\paperwidth,height=\paperheight]{../../img/fondoPrueba.jpg}};
    \end{tikzpicture}

    % Contenido de la portada
    \begin{center}
        \vspace*{2cm}
        
        {\Huge \bfseries\scshape Título del Libro de Apuntes \par}
        \vspace{0.5cm}
        {\Large \itshape Subtítulo o Asignatura \par}
        \vspace{0.5cm}
        {\Large \itshape \href{https://ismael-sallami.github.io}{https://ismael-sallami.github.io} \par}


        \vfill
        
        {\LARGE Autor: \textbf{Tu Nombre Completo} \par}
        \vspace{0.3cm}
        % {\Large Universidad Ejemplo \par}
        
        \vspace{1cm}
        \includegraphics[width=0.25\textwidth]{../../img/ugr.png} % opcional: logo
        \vspace{1cm}
        
        {\large \today}
    \end{center}
    
    \restoregeometry
\end{titlepage}



%==========================
% PORTADA: ENTRADA MANUAL
%==========================

% portada.tex
\begin{titlepage}
    \newgeometry{top=2cm,bottom=2cm,left=2.5cm,right=2.5cm} % márgenes personalizados
    
    % Fondo con transparencia
    \begin{tikzpicture}[remember picture,overlay]
        % \node[opacity=0.15,inner sep=0pt] at (current page.center)
        \node[inner sep=0pt] at (current page.center)
            {\includegraphics[width=\paperwidth,height=\paperheight]{../../../ElblogdeIsmael.github.io/extraFiles/img/fondo_info.jpg}};
    \end{tikzpicture}

    % Contenido de la portada
    \begin{center}
        \vspace*{2cm}
        
        \vspace{5cm} % Añadir más espacio antes del contenido
        {\Huge \bfseries\scshape\textcolor{white}{Informática
Gráfica} \par}
        \vspace{0.5cm}
        {\Large \itshape\textcolor{white}{Ejercicio Resueltos} \par}
        \vspace{0.5cm}
        % {\small \itshape \href{https://ismael-sallami.github.io}{https://ismael-sallami.github.io} \par}
        % {\small \itshape \href{https://elblogdeismael.github.io}{https://elblogdeismael.github.io} \par}


        \vfill
        
        % {\LARGE Ismael Sallami Moreno \par}

        \begin{flushright}
            {Ismael Sallami Moreno \par}
            {\small \itshape \href{https://elblogdeismael.github.io}{Recursos Ingeniería Informática y Ade} \par}
        \end{flushright}
        \vspace{0.3cm}
        % {\Large Universidad de Granada \par}
        
        % \vspace{1cm}
        % \includegraphics[width=0.25\textwidth]{../../../ElblogdeIsmael.github.io/extraFiles/img/ugr.png} % opcional: logo
        % \vspace{1cm}
        
        % {\large \today}
    \end{center}
    
    \restoregeometry
\end{titlepage}


%==========================
% LICENCIA
%==========================

\begin{tikzpicture}[remember picture,overlay]
\node[anchor=south west, xshift=1cm, yshift=1cm] at (current page.south west) {
\begin{minipage}{0.4\textwidth}
\begin{flushleft}
\section*{Licencia}

Este trabajo está bajo una 
\href{https://creativecommons.org/licenses/by-nc-nd/4.0/}{Licencia Creative Commons BY-NC-ND 4.0}.

\bigskip

Permisos: Se permite compartir, copiar y redistribuir el material en cualquier medio o formato.

\bigskip

Condiciones: Es necesario dar crédito adecuado, proporcionar un enlace a la licencia e indicar si se han realizado cambios. No se permite usar el material con fines comerciales ni distribuir material modificado.

\bigskip

\begin{center}
  \href{https://creativecommons.org/licenses/by-nc-nd/4.0/}{\includegraphics[width=0.35\textwidth]{../../../ElblogdeIsmael.github.io/extraFiles/img/by-nc-nd.png}}
\end{center}
\end{flushleft}
\end{minipage}
};
\end{tikzpicture}

\thispagestyle{empty}
\clearpage

%==========================
% AUTOR
%==========================

% Página del autor
\begin{center}
    \vspace*{3cm} % Añadir más espacio en la parte superior
    {\Huge Informática
Gráfica}\\[2cm] % Incrementar el espacio entre líneas
    {\Large Ismael Sallami Moreno}\\[1cm] % Incrementar el espacio entre líneas
    % \includegraphics[width=0.3\textwidth]{autor.jpg} % opcional foto
    \vfill % Añadir espacio flexible para centrar verticalmente
\end{center}
\thispagestyle{empty}

% Página en blanco sin estilo
\newpage
\thispagestyle{empty}
\mbox{}

% Otra página en blanco sin estilo
\newpage
\thispagestyle{empty}
\mbox{}



%==========================
% BIOGRAFÍA
%==========================

% Breve descripción del autor
% \chapter*{Biografía}
% \addcontentsline{toc}{chapter}{Biografía} % aparece en índice
% Aquí escribes una breve descripción sobre ti, tu formación, experiencia, etc.
% \cleardoublepage


% % ===============================
% licencia.tex
% ===============================
\begin{tikzpicture}[remember picture,overlay]
\node[anchor=south west, xshift=1cm, yshift=1cm] at (current page.south west) {
\begin{minipage}{0.4\textwidth}
\begin{flushleft}
\section*{Licencia}

Este trabajo está bajo una 
\href{https://creativecommons.org/licenses/by-nc-nd/4.0/}{Licencia Creative Commons BY-NC-ND 4.0}.

\bigskip

Permisos: Se permite compartir, copiar y redistribuir el material en cualquier medio o formato.

\bigskip

Condiciones: Es necesario dar crédito adecuado, proporcionar un enlace a la licencia e indicar si se han realizado cambios. No se permite usar el material con fines comerciales ni distribuir material modificado.

\bigskip

\begin{center}
  \href{https://creativecommons.org/licenses/by-nc-nd/4.0/}{\includegraphics[width=0.35\textwidth]{../../../extraFiles/img/by-nc-nd.png}}
\end{center}
\end{flushleft}
\end{minipage}
};
\end{tikzpicture}
  % licencia
% \thispagestyle{empty} % quitar número de página en la portada
% \clearpage

% --- Índice ---
\tableofcontents
% \listoffigures
\clearpage

%\listoftables
%\clearpage
%\thispagestyle{empty} % quitar número de página en la portada
%\clearpage
%
% Índice de código
%\renewcommand{\lstlistlistingname}{Índice de Código}
%\lstlistoflistings
%\clearpage
%
% Índice de ecuaciones
%\renewcommand{\listtheoremname}{Índice de Ecuaciones}
%\listoftheorems[ignoreall,show={equation}]
%\clearpage

% --- Contenido Markdown generado por Pandoc ---
\part{Teoría}
% % ================================================================
% CAPÍTULO 1: INTRODUCCIÓN A LA CONTABILIDAD DE GESTIÓN
% ================================================================

\chapter{Naturaleza y contenido de la contabilidad de gestión}

\section{Modelo básico de la circulación de valores en la empresa}

La actividad empresarial puede entenderse como una continua \textbf{circulación de valores} que conecta a la empresa con su entorno y articula sus procesos internos. Este modelo se fundamenta en cuatro subsistemas interconectados que describen el ciclo económico de una unidad de producción:
\begin{enumerate}
    \item \textbf{Financiación}: Corresponde a las operaciones dedicadas a la obtención de los recursos financieros necesarios para la actividad. Constituye el punto de partida, donde se captan capitales (aportaciones, préstamos) que dotan a la empresa de liquidez (dinero).
    \item \textbf{Inversión}: Engloba las operaciones relativas a la adquisición de los factores productivos. La empresa utiliza el dinero obtenido para comprar los bienes y servicios (materiales, maquinaria, mano de obra) que necesita para producir. Esta fase transforma el dinero en factores de producción, y la magnitud que la representa es el \textbf{gasto}.
    \item \textbf{Producción}: Se refiere a las operaciones de aplicación de los factores productivos en el proceso de transformación para obtener nuevos bienes o servicios. Este es el núcleo del ámbito interno de la empresa, donde el consumo de los factores productivos da lugar al \textbf{coste} y se genera la producción valorada.
    \item \textbf{Desinversión}: Agrupa las operaciones relativas a la colocación de los productos (bienes o servicios) en el mercado. La venta de las existencias de mercancías acabadas a los clientes genera un \textbf{ingreso}, que idealmente retorna a la empresa en forma de dinero, cerrando así el ciclo.
\end{enumerate}

Este flujo se puede visualizar en dos ámbitos:
\begin{itemize}
    \item \textbf{Ámbito externo}: Comprende las transacciones de la empresa con el "mundo exterior", como las compras a proveedores y las ventas a clientes.
    \item \textbf{Ámbito interno}: Se centra en el proceso de transformación productiva, abarcando las fases de consumo, fabricación y almacenamiento.
\end{itemize}

\begin{figure}[H]
    \centering
    \includegraphics[width=0.8\textwidth]{media/placeholder.png} % Nota: Se asume que la imagen del esquema está disponible como placeholder.png
    \caption{Esquema de la circulación de valores en la empresa (Adaptado de Schneider, 1968).}
\end{figure}

\section{La Contabilidad de gestión: delimitación y objetivos}

\begin{definicion}
La \textbf{Contabilidad de Gestión} es una rama de la contabilidad que se enfoca en la realidad económico-técnica o interna de una microunidad económica. Su finalidad específica es permitir el control de la producción y los costes, así como medir la eficiencia técnico-productiva de la misma.
\end{definicion}

Este sistema de información está diseñado para ser utilizado por los directivos para planificar, evaluar y controlar la organización, asegurando un uso apropiado y responsable de los recursos. A diferencia de la contabilidad financiera, no está regulada externamente y se adapta a las necesidades estratégicas y operativas de cada empresa.

Los \textbf{objetivos} o fines principales de la Contabilidad de Gestión son:
\begin{itemize}
    \item \textbf{Planificación y control de gestión}: Ayuda a los directivos a cuantificar los efectos futuros de las decisiones (planificar), juzgar los resultados históricos frente a los planes (evaluar) y vigilar el rendimiento para tomar acciones correctivas (controlar).
    \item \textbf{Cálculo del coste de los productos}: Es fundamental para valorar los inventarios, controlar las operaciones y obtener el coste de los productos con el fin de tomar decisiones sobre precios, rentabilidad o fabricación.
    \item \textbf{Toma de decisiones}: Proporciona información relevante para la selección entre cursos de acción alternativos, tanto a corto como a largo plazo.
    \item \textbf{Análisis y evaluación de actividades}: Permite un conocimiento detallado de las actividades productivas para su control.
    \item \textbf{Determinación de resultados internos}: Calcula el resultado periódico con criterios económicos y lo descompone para conocer la contribución de cada área o producto a su generación.
\end{itemize}

Para cumplir estos objetivos, la Contabilidad de Gestión se centra en las \textbf{magnitudes fundamentales del ámbito interno}:
\begin{itemize}
    \item \textbf{Magnitudes flujo (corrientes)}: Consumos y costes de un período, producción y su valor, y colocación y su valor.
    \item \textbf{Magnitudes fondo (stocks)}: Producción en curso y su valor, y producción en stock y su valor.
\end{itemize}

\section{Contabilidad externa y Contabilidad interna}

La información contable de una organización se estructura en dos grandes áreas: la \textbf{Contabilidad Financiera} (externa) y la \textbf{Contabilidad de Gestión} (interna). Aunque ambas se nutren del mismo sistema de información, sus propósitos, usuarios y características son distintos.

\begin{table}[h!]
\centering
\caption{Principales diferencias entre Contabilidad Financiera y de Gestión.}
\begin{tabular}{p{0.25\linewidth} p{0.3\linewidth} p{0.3\linewidth}}
\hline
\textbf{Rasgo} & \textbf{Contabilidad Financiera (Externa)} & \textbf{Contabilidad de Gestión (Interna)} \\
\hline
\textbf{Usuarios} & Externos (accionistas, bancos, gobierno) e internos. & Exclusivamente internos (directivos, mandos intermedios, empleados). \\
\textbf{Regulación} & Regulada por principios contables generalmente aceptados (PCGA) y el Estado. & No regulada. Determinada por la dirección para satisfacer sus necesidades. \\
\textbf{Naturaleza de la información} & Prima la objetividad, fiabilidad y verificabilidad. Es precisa y auditable. & Prima la relevancia y flexibilidad para la toma de decisiones. Es más subjetiva (estimaciones). \\
\textbf{Tipo de información} & Principalmente medidas financieras. & Medidas financieras, operativas y físicas sobre procesos, clientes, etc.. \\
\textbf{Enfoque temporal} & Histórico, orientado al pasado. & Actual y orientado al futuro. \\
\textbf{Ámbito} & Agregada y global. Informa sobre el conjunto de la organización. & Desagregada y concreta. Informa sobre departamentos, segmentos o decisiones específicas. \\
\textbf{Obligatoriedad} & Obligatoria. & No obligatoria. \\
\hline
\end{tabular}
\end{table}

La Contabilidad Financiera se centra en registrar las operaciones de la empresa y presentar informes a terceros, proyectando una imagen global de su situación financiera. Sin embargo, esta información es insuficiente para la gestión diaria. Los directivos necesitan datos detallados para tomar decisiones rutinarias y no rutinarias, como analizar la rentabilidad de un producto o evaluar la eficiencia de una actividad. Es aquí donde la Contabilidad de Gestión (también denominada \textbf{Contabilidad interna}, \textbf{analítica} o \textbf{de costes}) cobra su relevancia, proporcionando la información desagregada que la gestión interna demanda.

\section{Producción: conceptos y clases}

El concepto de producción puede analizarse desde dos perspectivas: como efecto (el resultado del proceso) y como causa (el proceso de transformación en sí mismo).

\subsection{La producción como efecto}
Desde esta óptica, la producción se refiere a los bienes o servicios obtenidos. Se clasifica principalmente según su grado de perfeccionamiento:
\begin{itemize}
    \item \textbf{Producción final}: Corresponde a los productos acabados, listos para su venta o destino final.
    \item \textbf{Producción intermedia}: Incluye productos que aún no han completado el ciclo productivo. Se subdivide en:
        \begin{itemize}
            \item \textbf{Productos en curso}: Aquellos que se encuentran en una fase de elaboración dentro de un centro de trabajo.
            \item \textbf{Productos semiterminados}: Aquellos que han finalizado una fase del proceso y se encuentran almacenados a la espera de ser incorporados en una etapa posterior.
        \end{itemize}
    \item \textbf{Otra producción}:
        \begin{itemize}
            \item \textbf{Subproductos}: Productos de carácter secundario obtenidos simultáneamente con el producto principal (p. ej., el serrín en una fábrica de muebles).
            \item \textbf{Desperdicios}: Residuos generados en el proceso que pueden tener o no valor de venta.
        \end{itemize}
\end{itemize}

\subsection{La producción como causa}
Bajo esta perspectiva, la producción es el \textbf{proceso productivo} en sí, definido como una "transformación, según una determinada técnica, de factores productivos en productos". Este proceso tiene una vertiente técnica donde los factores son consumidos en centros de trabajo (actividad) para generar productos y servicios.

\subsubsection{Clasificación de la producción según el proceso}
Atendiendo a los tipos de productos y su forma de obtención, la producción puede ser:
\begin{itemize}
    \item \textbf{Producción simple}: Se obtiene un único tipo de producto, ya sea a través de un proceso lineal o complejo.
    \item \textbf{Producción múltiple o compuesta}: Se obtienen varios tipos de productos de forma simultánea o excluyente. Se divide en:
        \begin{itemize}
            \item \textbf{Producción paralela}: Se obtienen productos distintos en procesos independientes.
            \item \textbf{Producción alternativa}: La fabricación de un producto excluye la de otro, utilizando los mismos factores (p. ej., envasado de aceite de oliva o de girasol en la misma planta).
            \item \textbf{Producción conjunta (o acumulativa)}: Se obtienen simultáneamente varios productos a partir de un mismo proceso y materia prima. Es inevitable obtener todos los productos a la vez. Puede ser:
                \begin{itemize}
                    \item \textbf{Con coproductos}: Se obtienen varios productos principales (p. ej., carne y piel en la industria cárnica).
                    \item \textbf{Con subproductos}: Se obtiene un producto principal y otros secundarios.
                    \item \textbf{Acoplada}: Como en la destilación de la hulla, donde de una materia prima se obtienen múltiples productos (gas, coque, alquitrán, etc.) en proporciones fijas.
                \end{itemize}
        \end{itemize}
\end{itemize}

\section{Proceso productivo y medios de producción}
El proceso productivo requiere \textbf{factores de producción} o medios que hacen posible la transformación económica. Estos recursos se pueden clasificar de diversas formas:
\begin{itemize}
    \item \textbf{Según su participación en el proceso}:
        \begin{itemize}
            \item \textit{Factores estructurales}: Forman la capacidad productiva de la empresa (maquinaria, edificios).
            \item \textit{Factores para perfeccionamiento}: Se consumen o transforman en el proceso (materia prima, medios colaboradores como la energía).
        \end{itemize}
    \item \textbf{Según su influencia en el producto final}:
        \begin{itemize}
            \item \textit{Factores limitativos}: Deben utilizarse en proporciones fijas.
            \item \textit{Factores sustitutivos}: Pueden intercambiarse entre sí.
        \end{itemize}
\end{itemize}
Los centros de trabajo donde se aplican estos factores pueden agregarse en distintos niveles. La célula básica de actividad es la \textbf{unidad de trabajo}, que es una combinación indivisible de medios estructurales (p. ej., una máquina) y su correspondiente dotación de personal. Niveles superiores de agregación incluyen el lugar de trabajo y la sección de trabajo.

\section{Productividad y rendimiento: su medida}
La \textbf{productividad} es una medida de la eficiencia del proceso productivo que relaciona la producción obtenida (output) con la cantidad de factores o recursos utilizados (input). Se expresa como un cociente:
$$ \text{Productividad} = \frac{\text{Producción (Outputs)}}{\text{Factores empleados (Inputs)}} $$
Esta medida puede calcularse para un factor específico (productividad parcial, p.ej., productividad del trabajo) o para el conjunto de factores (productividad total o global).

Por otro lado, el \textbf{rendimiento} compara la producción real obtenida con la producción que se debería haber obtenido en condiciones estándar o normales. Se formula como:
$$ \text{Rendimiento} = \frac{\text{Producción Real}}{\text{Producción Estándar}} \quad \text{o} \quad \text{Rendimiento} = \frac{\text{Tiempo Estándar}}{\text{Tiempo Real}} $$
Un rendimiento mayor que 1 (o 100\%) indica una eficiencia superior a la estándar, mientras que un valor inferior a 1 señala una ineficiencia. Ambas magnitudes, productividad y rendimiento, son cruciales para el control de la gestión, pues permiten evaluar y mejorar la eficiencia con la que se utilizan los recursos de la empresa.

% \chapter{Gestión de la Cadena de Suministro}

\section{Importancia Estratégica}
La \textbf{Gestión de la Cadena de Suministro (GCS)}, o \textit{Supply Chain Management (SCM)}, se ha consolidado como una herramienta estratégica fundamental en el modelo de negocio empresarial, trascendiendo su concepción inicial como una mera operación logística. Su objetivo principal es coordinar todas las actividades dentro de la cadena, desde los proveedores iniciales hasta el consumidor final, para maximizar su ventaja competitiva y los beneficios percibidos por el cliente.

El concepto ha evolucionado desde una gestión centrada en los flujos internos de la empresa hacia un enfoque de integración con proveedores y clientes. Este cambio de paradigma implica que la competencia ya no se produce entre empresas individuales, sino a nivel de cadenas de suministro. Las compras representan un porcentaje significativo de los costes de una empresa, por lo que la gestión eficiente de las relaciones con los proveedores, considerándolos "socios" estratégicos, es clave para obtener una ventaja competitiva.

\subsection{Concepto de Cadena de Suministro}
A lo largo del tiempo, diversos autores han definido la GCS con un alcance progresivamente más amplio.
\begin{itemize}
    \item \textbf{Jones y Riley (1985):} La gestión del flujo total de materiales e información, desde los proveedores de materias primas hasta la entrega al consumidor final.
    \item \textbf{Christopher (1998):} El conjunto de empresas interrelacionadas en los procesos y actividades que generan valor en forma de productos y servicios para el cliente final.
    \item \textbf{Ballou (2004):} Una red de organizaciones y personas involucradas en el flujo de materia prima, productos, información y dinero, desde los proveedores hasta el consumidor.
    \item \textbf{Arias y Minguela (2018):} La coordinación sistemática y estratégica de las funciones de negocio, tanto dentro de una empresa como entre las empresas de la cadena, con el fin de mejorar el rendimiento a largo plazo de cada parte y de la cadena en su conjunto.
\end{itemize}

\subsection{Diferencia entre GCS y Logística}
Es importante distinguir la GCS de la logística, aunque a menudo se usen como sinónimos. La \textbf{logística} es la parte del proceso de la cadena de suministro que planifica, implementa y controla el flujo y almacenamiento de bienes, servicios e información desde el origen hasta el consumo para satisfacer los requerimientos del cliente. Su origen se remonta al ámbito militar y tradicionalmente se asocia al transporte y almacenamiento.

La GCS tiene un alcance mucho más amplio, ya que integra procesos clave que van más allá del movimiento de bienes, abarcando la gestión de la oferta y la demanda dentro y entre empresas. La logística empresarial se puede segmentar en:
\begin{itemize}
    \item \textbf{Logística de entrada (\textit{inbound logistics}):} Corresponde al proceso de aprovisionamiento.
    \item \textbf{Logística interna:} Vinculada a los movimientos de materiales dentro del proceso de producción.
    \item \textbf{Logística de salida (\textit{outbound logistics}):} Relacionada con el proceso de distribución del producto final.
\end{itemize}

\subsection{Impacto de la Estrategia Corporativa en la GCS}
La estrategia de la cadena de suministro se enmarca dentro de la jerarquía estratégica de la empresa (corporativa, competitiva y funcional). Las decisiones de la GCS deben estar alineadas con la estrategia corporativa, que puede ser de bajo coste, de respuesta rápida o de diferenciación. La Tabla siguiente detalla este impacto:

\begin{table}[H]
\centering
\caption{Impacto de la Estrategia Corporativa en las decisiones de la Cadena de Suministro.}
\begin{tabular}{p{0.2\textwidth} p{0.2\textwidth} p{0.2\textwidth} p{0.2\textwidth}}
\toprule
 & \textbf{Estrategia de bajo coste} & \textbf{Estrategia de respuesta rápida} & \textbf{Estrategia de diferenciación} \\
\midrule
\textbf{Selección de Proveedores} & Basada en el coste. & Basada en capacidad, velocidad y flexibilidad. & Basada en habilidades para el desarrollo de productos. \\
\textbf{Inventario} & Minimizar para reducir costes. & Utilizar stocks de reserva para asegurar rapidez. & Minimizar para evitar obsolescencia. \\
\textbf{Distribución} & Transporte económico, venta a través de distribuidores de descuento. & Transporte rápido, servicio al cliente excelente. & Recopilar y comunicar datos de mercado. \\
\textbf{Diseño del Producto} & Maximizar rendimiento y minimizar costes. & Diseño que permita bajos tiempos de preparación y rápido incremento de producción. & Diseño modular que facilite la diferenciación. \\
\bottomrule
\end{tabular}
\end{table}

\section{Elementos y Procesos}

\subsection{Elementos Clave de la GCS}
Toda cadena de suministro está compuesta, en general, por tres elementos o eslabones fundamentales:
\begin{itemize}
    \item \textbf{Proveedores:} Se organizan en distintos niveles. El proveedor de primer nivel suministra directamente al fabricante, el de segundo nivel al de primer nivel, y así sucesivamente.
    \item \textbf{Fabricantes:} Transforman los materiales y componentes (\textit{inputs}) en productos acabados. Pueden operar en una o varias fábricas, lo que afecta a la complejidad y coordinación de la cadena.
    \item \textbf{Distribuidores:} Incluyen mayoristas (venden a otras empresas) y minoristas (venden al cliente final).
\end{itemize}
Es importante destacar que no todos los eslabones deben estar formados por empresas diferentes; existen modelos de negocio con alta integración vertical. Además, una misma empresa, ya sea proveedora o distribuidora, puede formar parte de múltiples cadenas de suministro. Estos conceptos son aplicables tanto a empresas de productos como de servicios.

\subsection{Canales de Distribución}
El canal de distribución es el camino que sigue un producto desde el fabricante hasta el consumidor. Puede ser:
\begin{itemize}
    \item \textbf{Canal Directo:} El fabricante vende directamente al consumidor, sin intermediarios. El uso de Internet como canal directo ha crecido significativamente.
    \item \textbf{Canales Indirectos:} Involucran a uno o más intermediarios (mayoristas, minoristas). Un canal con mayoristas necesariamente debe incluir también a un minorista para llegar al consumidor final.
\end{itemize}

\subsection{Procesos Clave en la GCS}
Para que una cadena de suministro sea competitiva, es crucial la integración de sus procesos de negocio clave. Según el marco de Cooper et al. (1997), estos procesos son:
\begin{enumerate}
    \item \textbf{Gestión de las relaciones con clientes:} Define cómo se desarrollarán y mantendrán las relaciones con los clientes, segmentándolos según sus necesidades para ofrecerles los productos adecuados y mantener su satisfacción al menor coste posible.
    \item \textbf{Gestión del servicio al cliente:} Establece los puntos de contacto con el cliente y gestiona incidencias, buscando resolverlas antes de que afecten al usuario final.
    \item \textbf{Gestión de la demanda:} Busca equilibrar las necesidades del cliente con la capacidad productiva de la cadena para asegurar un flujo ininterrumpido.
    \item \textbf{Gestión del flujo de producción:} Abarca todas las actividades de fabricación, incorporando la flexibilidad necesaria para servir a los clientes.
    \item \textbf{Cumplimiento de los pedidos:} Incluye las actividades necesarias para crear una red que cumpla con las solicitudes de los clientes en plazo y cantidad, minimizando los costes de envío.
    \item \textbf{Gestión de las relaciones con los proveedores:} Proceso análogo a la gestión de clientes, pero enfocado en seleccionar un grupo de proveedores clave para establecer relaciones a largo plazo, buscando un beneficio mutuo (\textit{win-win situation}).
    \item \textbf{Desarrollo y comercialización de nuevos productos:} Integra aportaciones de clientes y proveedores para reducir el tiempo de introducción de un nuevo producto en el mercado.
    \item \textbf{Devoluciones (Logística inversa):} Gestiona todas las actividades relacionadas con el retorno de productos por parte de los clientes.
\end{enumerate}
La integración de estos procesos, tanto a nivel intraorganizacional como interorganizacional, es determinante para generar y mantener ventajas competitivas.

\section{Estrategias de Gestión de la Cadena de Suministro}
No existe una estrategia única de GCS, ya que esta debe adaptarse a la naturaleza del producto y a la predictibilidad de su demanda. Centrándose en estos factores, se pueden identificar dos enfoques principales.
\begin{itemize}
    \item \textbf{Productos funcionales:} Satisfacen necesidades básicas, con demanda estable y predecible, márgenes reducidos y baja variedad.
    \item \textbf{Productos innovadores:} Tienen un ciclo de vida corto, gran variedad, márgenes altos y una demanda difícil de predecir.
\end{itemize}

\subsection{Estrategias Lean y Ágil}
Basándose en la naturaleza de la demanda y en el objetivo de la cadena, surgen dos estrategias fundamentales:
\begin{itemize}
    \item \textbf{GCS Lean (Eficiencia):} Adecuada para productos funcionales, se enfoca en la eficiencia, la productividad y la eliminación de despilfarros para lograr bajos costes logísticos y de inventario. Utiliza sistemas de fabricación de empuje (\textit{push}).
    \item \textbf{GCS Ágil (Respuesta rápida):} Orientada a productos innovadores, prioriza la flexibilidad y la capacidad de respuesta, con una alta velocidad de distribución y selección de proveedores basada en su rapidez. Emplea sistemas de fabricación de arrastre (\textit{pull}).
\end{itemize}
Muchas empresas se ven presionadas a combinar eficiencia y rapidez, lo que ha dado lugar a estrategias híbridas como la \textit{ejecución diferida (postponement)} o el \textit{reaprovisionamiento continuo}.

\subsection{Seis Estrategias de Suministro}
Además de los enfoques lean y ágil, existen seis estrategias de suministro que una empresa puede adoptar para configurar sus relaciones externas:
\begin{enumerate}
    \item \textbf{Muchos proveedores:} Estrategia basada en la competencia agresiva entre proveedores, común para productos estándar (\textit{commodity}). Se selecciona la oferta más barata para cada petición.
    \item \textbf{Pocos proveedores:} Busca establecer relaciones a largo plazo con un número reducido de proveedores, lo que les permite alcanzar economías de escala. El coste de cambiar de proveedor es alto.
    \item \textbf{Integración vertical:} Consiste en producir internamente bienes que antes se compraban o adquirir un proveedor (integración hacia atrás) o un distribuidor (integración hacia adelante).
    \item \textbf{Joint Ventures (empresas conjuntas):} Colaboración formal en la que varias empresas establecen una propiedad común para desarrollar nuevos productos o mercados.
    \item \textbf{Redes Keiretsu:} Coalición de empresas en la que los proveedores se integran profundamente, combinando colaboración, compra a pocos proveedores e integración vertical. Se basa en relaciones a largo plazo y apoyo mutuo.
    \item \textbf{Empresas virtuales:} Organizaciones que dependen de una red de proveedores externos para proporcionar servicios bajo demanda. La cadena de suministro es, en esencia, la propia empresa.
\end{enumerate}

\section{Riesgos en la Cadena de Suministro}
La gestión de la cadena de suministro conlleva riesgos, definidos como la posibilidad y el efecto de un desajuste entre la oferta y la demanda. La creciente dependencia de la cadena (comprar más, fabricar menos), la especialización con pocos proveedores y los bajos inventarios incrementan el riesgo. Estos riesgos pueden ser de origen diverso: naturales, políticos, financieros o sistémicos, como los derivados de pandemias o desastres naturales con impacto global.

La gestión de estos riesgos se ha convertido en un reto estratégico, ya que trabajar con muchos proveedores aumenta la complejidad logística, mientras que hacerlo con pocos aumenta la dependencia. Para mitigar estos riesgos, las empresas pueden aplicar diversas tácticas:

\begin{itemize}
    \item \textbf{Riesgos de proveedores (fallos en envío o calidad):} Mitigados con el uso de múltiples proveedores, contratos con penalizaciones, una cuidadosa selección y supervisión, y la disponibilidad de subcontratas de reserva.
    \item \textbf{Riesgos logísticos (retrasos o daños):} Se gestionan mediante la diversificación de modos de transporte y almacenes, embalajes seguros y contratos eficaces.
    \item \textbf{Pérdida de información:} Se previene con bases de datos redundantes, sistemas de TI seguros y formación de los socios de la cadena.
    \item \textbf{Riesgos políticos y económicos:} Se mitigan con seguros, diversificación internacional, franquicias y coberturas para el riesgo del tipo de cambio.
    \item \textbf{Catástrofes, robos o terrorismo:} Se afrontan con seguros, fuentes de suministro alternativas, diversificación y medidas de seguridad (por ejemplo, GPS).
\end{itemize}
El proceso para mitigar los riesgos se desarrolla en cuatro etapas: 1) interpretación y visualización de riesgos, 2) medición y priorización, 3) toma de acciones, y 4) seguimiento y revisión continua.

\section{Ética y Sostenibilidad}
La gestión de la cadena de suministro debe regirse por principios éticos y de sostenibilidad. La ética personal y la ética dentro de la cadena exigen que las empresas establezcan normas para sus proveedores, similares a las que aplican para sí mismas, ya que la sociedad demanda un comportamiento ético a lo largo de toda la cadena.

\subsection{Sostenibilidad en la GCS}
La sostenibilidad en la GCS implica gestionar los flujos de productos, información y finanzas con un enfoque en las preocupaciones sociales y medioambientales. Los tres pilares de la sostenibilidad son:
\begin{itemize}
    \item \textbf{Pilar medioambiental:} Centrado en la criticidad de los recursos y la promoción de la economía circular.
    \item \textbf{Inclusión social y equidad distributiva:} Busca garantizar que quienes fabrican los productos compartan equitativamente los beneficios y que no se produzcan situaciones de exclusión o "esclavitud moderna". Leyes como la Ley de Transparencia en las Cadenas de Suministro de California buscan frenar estas prácticas.
\end{itemize}
Existe una creciente conciencia en el sector empresarial sobre la importancia de la sostenibilidad. Muchas empresas invierten en cadenas más sostenibles, no solo porque es lo correcto, sino porque también genera retornos financieros. Un aspecto clave es la distinción entre:
\begin{itemize}
    \item \textbf{Logística directa:} Flujo de productos hacia el cliente.
    \item \textbf{Logística inversa:} Flujo de productos desde el cliente de vuelta a la empresa, relacionado con devoluciones, reciclaje o reacondicionamiento.
\end{itemize}
Una \textbf{cadena de suministro de bucle cerrado} se refiere al diseño proactivo de una cadena que optimiza tanto los flujos hacia adelante como los flujos inversos, integrando la sostenibilidad desde el inicio.

% \chapter{Diseño de Bienes y Servicios}

% \textbf{Fuentes principales: CAPÍTULO 3 DE ARIAS Y MINGUELA, 2024; CAPÍTULO 5 DE HEIZER Y RENDER, 2015}

\section{Definición de desarrollo de nuevos productos}

El \textbf{concepto de producto} se define como \textit{“algo que se ofrece a un mercado con la finalidad de que se le preste atención, sea adquirido, usado o consumido, con el objeto de satisfacer un deseo o necesidad”}. Un producto engloba un conjunto de atributos, tanto tangibles como intangibles (envase, precio, marca, servicios, etc.), que los compradores perciben como capaces de satisfacer sus necesidades.

Las actividades de producción de bienes y servicios se denominan \textbf{operaciones}, y su gestión, \textbf{Dirección de Operaciones}. Esta disciplina busca crear valor transformando recursos (inputs) en productos (outputs). Las diferencias entre bienes (manufacturas) y servicios son clave para entender su diseño y gestión:
\begin{itemize}
    \item \textbf{Bienes}: Son productos físicos y duraderos, que pueden ser inventariados. Generalmente, implican poco contacto con el cliente, tiempos de respuesta largos y su calidad es más fácil de medir objetivamente.
    \item \textbf{Servicios}: Son intangibles y perecederos, no se pueden inventariar. Requieren un alto contacto con el cliente, tiempos de respuesta cortos y su calidad es más subjetiva y, por tanto, más difícil de medir.
\end{itemize}

\begin{table}[H]
\centering
\caption{Las diferencias entre bienes y servicios influyen en cómo se aplican las 10 decisiones de operaciones}
\begin{tabular}{|p{3cm}|p{4cm}|p{4cm}|}
\hline
\textbf{Decisiones de operaciones} & \textbf{Bienes} & \textbf{Servicios} \\ \hline
Diseño de bienes y servicios & Normalmente el producto es tangible. & El producto no es tangible. Una nueva gama de atributos del producto: una sonrisa. \\ \hline
Gestión de la calidad & Muchas normas de calidad objetivas. & Muchas normas de calidad subjetivas: un color bonito. \\ \hline
Diseño del proceso y de la capacidad & El cliente no está implicado en la mayor parte del proceso. & El cliente puede estar implicado directamente en el proceso: un corte de pelo. La capacidad debe adecuarse a la demanda para evitar pérdida de ventas: los clientes normalmente evitan esperar. \\ \hline
Selección de localización & Puede ser necesario estar cerca de las materias primas o de la mano de obra. & Puede ser necesario estar cerca del cliente: alquiler de coches. \\ \hline
Diseño del layout & El layout puede mejorar la eficiencia. & Puede mejorar el producto y la producción. Ej. layout de un restaurante elegante. \\ \hline
Recursos humanos y diseño del puesto de trabajo & Mano de obra centrada en habilidades técnicas. Los estándares de trabajo pueden ser constantes. Posible sistema salarial basado en la producción. & La mano de obra directa necesita normalmente poder relacionarse con el cliente: cajero de un banco. Los estándares de trabajo varían según las exigencias del cliente: procesos legales. \\ \hline
Gestión de la cadena de suministros & Las relaciones en la cadena de suministros son vitales para el producto final. & Las relaciones de la cadena de suministros son importantes pero no son vitales. \\ \hline
Inventario & Las materias primas, los productos semiacabados y los acabados pueden almacenarse. & La mayor parte de los servicios no puede almacenarse, por lo que hay que encontrar otras formas de acomodarse a los cambios de la demanda. \\ \hline
Programación & La posibilidad de almacenar puede permitir nivelar la tasa de producción. & Ocupada en satisfacer los plazos inmediatos del cliente utilizando los recursos humanos. \\ \hline
Mantenimiento & El mantenimiento es habitualmente preventivo, y se da en el lugar de producción. & El mantenimiento es normalmente una “reparación”, que se realiza en el lugar donde está el cliente. \\ \hline
\end{tabular}
\end{table}

El \textbf{Desarrollo de Nuevos Productos (DNP)} es la secuencia de decisiones que conduce a la creación de un nuevo bien o servicio. Este proceso se puede clasificar según:
\begin{itemize}
    \item \textbf{El tipo de innovación}: Puede ser \textbf{radical}, si crea algo completamente nuevo, o \textbf{incremental}, si consiste en mejoras sobre productos ya existentes.
    \item \textbf{La complejidad}: Depende de la cantidad de variables y de la sofisticación del conocimiento requerido.
\end{itemize}

El DNP es una \textbf{decisión transversal} que requiere la coordinación de múltiples áreas funcionales como Marketing, I+D, Producción, Compras, Finanzas y Dirección General. La colaboración entre departamentos se puede gestionar mediante dos enfoques:
\begin{itemize}
    \item \textbf{Enfoque secuencial}: Cada departamento completa su fase antes de pasarla al siguiente. Es un proceso más lento y menos flexible.
    \item \textbf{Enfoque concurrente o simultáneo}: Los departamentos trabajan de forma paralela y coordinada, lo cual acelera el desarrollo del producto.
\end{itemize}

\begin{table}[H]
\centering
\caption{Comparativa de ventajas e inconvenientes de los enfoques secuencial y concurrente}
\begin{tabular}{|p{4cm}|p{5cm}|p{5cm}|}
\hline
\textbf{Aspecto} & \textbf{Enfoque Secuencial} & \textbf{Enfoque Concurrente} \\ \hline
\textbf{Ventajas} & 
\begin{itemize}
    \item Proceso estructurado y fácil de gestionar.
    \item Menor riesgo de conflictos entre departamentos.
    \item Claridad en la asignación de responsabilidades.
\end{itemize} & 
\begin{itemize}
    \item Reducción del tiempo total de desarrollo.
    \item Mayor flexibilidad y capacidad de respuesta.
    \item Fomenta la colaboración y la innovación.
\end{itemize} \\ \hline
\textbf{Inconvenientes} & 
\begin{itemize}
    \item Proceso más lento debido a la naturaleza secuencial.
    \item Menor flexibilidad ante cambios en el entorno.
    \item Posible falta de integración entre departamentos.
\end{itemize} & 
\begin{itemize}
    \item Mayor complejidad en la gestión del proyecto.
    \item Riesgo de conflictos entre departamentos.
    \item Requiere una comunicación y coordinación más intensiva.
\end{itemize} \\ \hline
\end{tabular}
\end{table}

Ventajas: mayor información, información disponible con antelación y previsión de posibles problemas.
Desventajas: mayor complejidad organizativa, mayores tiempos de toma de decisiones y mayor dificultad en la toma de decisiones. [COMPLETAR]



\section{Ciclo de vida de los productos y servicios}

El \textbf{Ciclo de Vida del Producto (CVP)} describe las distintas etapas por las que pasa un producto desde su lanzamiento al mercado hasta su desaparición. Este modelo pone en relación el tiempo (eje X) y el volumen de ventas (eje Y). Cada etapa presenta desafíos y oportunidades que exigen reajustar las estrategias de operaciones, marketing y finanzas. La duración del CVP varía según la naturaleza del producto, pero no debe confundirse con la vida útil del mismo para el consumidor.

Las cuatro fases del ciclo de vida son:
\begin{enumerate}
    \item \textbf{Introducción}: Las ventas crecen lentamente mientras el producto se da a conocer. Se caracteriza por:
    \begin{itemize}
        \item Fuertes desembolsos en I+D y modificaciones de procesos, generando pérdidas y un flujo de caja (cash-flow) negativo.
        \item Métodos de producción flexibles y poco eficientes, baja gama de producto y poca competencia.
        \item Los esfuerzos de diseño y desarrollo del producto son críticos.
        \item En esta fase se encuentran los ''early adopters'', clientes que buscan innovación y están dispuestos a pagar un precio premium.
        \item Ejemplos: coches voladores, coches autónomos, realidad virtual.
    \end{itemize}

    \item \textbf{Crecimiento}: La demanda aumenta rápidamente. En esta fase:
    \begin{itemize}
        \item El diseño del producto comienza a estabilizarse y se realizan inversiones para aumentar la capacidad productiva.
        \item El beneficio y el flujo de caja se vuelven positivos.
        \item La previsión de la demanda se vuelve crítica para la gestión de la capacidad.
    \end{itemize}

    \item \textbf{Madurez}: El mercado se satura y el volumen de ventas se estabiliza. Es la fase de mayor rentabilidad. Se caracteriza por:
    \begin{itemize}
        \item Estandarización del producto y del proceso, buscando economías de escala para reducir costes.
        \item La competencia es intensa, a menudo basada en costes.
        \item Es el momento ideal para iniciar el desarrollo de nuevos productos que sustituyan a los actuales.
    \end{itemize}

    \item \textbf{Declive}: Las ventas y los beneficios disminuyen paulatinamente. La empresa debe decidir si:
    \begin{itemize}
        \item Abandona el producto, liquidando existencias.
        \item Reubica la inversión en productos con mayor potencial.
        \item Intenta "reinventar" el producto.
        \item Desde operaciones, se debe reducir la capacidad y minimizar costes, eliminando productos sin margen aceptable.
    \end{itemize}
\end{enumerate}

\section{Etapas en el desarrollo de nuevos productos}

El proceso de DNP se puede estructurar en una secuencia de decisiones clave, que abarca desde la concepción de la idea hasta su llegada al mercado. Las etapas fundamentales son:
\begin{enumerate}
    \item \textbf{Identificación de la oportunidad de negocio (Generación y selección de ideas)}: Las ideas para nuevos productos pueden surgir de diversas fuentes:
    \begin{itemize}
        \item \textit{Consumidores (Tirón de la Demanda)}: Comprender las necesidades y deseos del cliente es un punto de partida fundamental.
        \item \textit{I+D (Empuje Tecnológico)}: Los avances tecnológicos hacen posibles nuevos productos.
        \item \textit{Competidores}: A través del benchmarking se pueden obtener ideas para mejorar o diferenciar la oferta.
        \item \textit{Empleados e Innovación Abierta}.
    \end{itemize}
    Una vez generadas, las ideas se someten a un filtro de viabilidad comercial (marketing), técnica (operaciones) y financiera (finanzas).

    \item \textbf{Diseño (Preliminar y Final)}: Esta fase transforma la idea en un concepto tangible. Las decisiones de diseño abarcan la función, costes, calidad, impacto medioambiental y métodos de producción. Para ello se utilizan diferentes \textbf{elementos y herramientas de diseño}:
    \begin{itemize}
        \item \textbf{Diseño Robusto}: Busca que pequeñas variaciones en la producción no afecten negativamente al producto final.
        \item \textbf{Diseño Modular}: Subdivide el producto en módulos intercambiables, lo que facilita la variedad y la reparación.
        \item \textbf{Diseño Asistido por Ordenador (CAD)} y \textbf{Fabricación Asistida por Ordenador (CAM)}: Programas informáticos que agilizan el diseño, la preparación de la documentación de ingeniería y el control de los equipos de producción, reduciendo costes y tiempos.
        \item \textbf{Despliegue de la Función de Calidad (QFD)}: Es una herramienta, cuya representación gráfica es la "casa de la calidad", que permite traducir los deseos del cliente en características técnicas del producto.
        \item \textbf{Ingeniería de Valor y Análisis de Valor}: Se centran en la mejora del diseño y las especificaciones para reducir costes sin sacrificar funcionalidad, antes de la producción (ingeniería) o durante ella (análisis).
    \end{itemize}

    \item \textbf{Construcción y evaluación de prototipos}: Se crean modelos o versiones iniciales del producto (maquetas, plantas piloto) para realizar evaluaciones técnicas y de mercado (lanzamiento en zonas piloto, paneles de consumidores). Esto permite probar el producto antes de comprometer recursos a gran escala.

    \item \textbf{Producción}: En esta fase, se toman las decisiones relativas al diseño del proceso y la planificación de la capacidad.

    \item \textbf{Comercialización}: Es la introducción del producto en el mercado, una función gestionada principalmente por el área de marketing.
\end{enumerate}


\section{Estrategias en el desarrollo de nuevos productos}

La creciente sofisticación tecnológica y la reducción de los ciclos de vida de los productos obligan a las empresas a \textbf{acelerar su proceso de desarrollo}. La competencia basada en el tiempo, que busca rapidez en el diseño, producción y entrega, se ha convertido en una ventaja competitiva clave.

Las estrategias de DNP se pueden clasificar en un continuo que va desde el desarrollo interno hasta el externo:

\begin{itemize}
    \item \textbf{Estrategias de desarrollo interno}:
    \begin{itemize}
        \item \textit{Mejoras de productos existentes}: Cambios incrementales en productos actuales.
        \item \textit{Migraciones de productos existentes}: Se aprovechan las plataformas de productos actuales para crear nuevas versiones, acelerando el desarrollo y reduciendo costes y riesgos.
        \item \textit{Nuevos productos desarrollados internamente}: Es la opción más lenta y arriesgada, pero ofrece un control total.
    \end{itemize}

    \item \textbf{Estrategias de desarrollo externo}: Buscan adquirir tecnología o experiencia fuera de la empresa para acelerar el proceso.
    \begin{itemize}
        \item \textit{Adquisición de tecnología}: Comprar una empresa que ya ha desarrollado la tecnología deseada.
        \item \textit{Empresas conjuntas (Joint Ventures)}: Dos o más empresas establecen una propiedad común para lanzar un nuevo producto. El riesgo y el coste se comparten.
        \item \textit{Alianzas}: Acuerdos de cooperación donde las empresas permanecen independientes pero persiguen objetivos comunes. Son adecuadas cuando las tecnologías son incipientes y los riesgos elevados.
    \end{itemize}
\end{itemize}


\section{Técnicas de resolución de ejercicios para la toma de decisiones sobre diseño de bienes y servicios}

Las decisiones sobre el diseño de productos a menudo se toman en condiciones de \textbf{riesgo} o \textbf{incertidumbre}. Para abordar estas situaciones, se utilizan técnicas cuantitativas que ayudan a estructurar el problema y a evaluar las alternativas.

\begin{itemize}
    \item \textbf{Matriz de Decisión}: Es una herramienta que permite analizar un problema con una decisión única. Se estructura con:
    \begin{itemize}
        \item \textbf{Estrategias o alternativas}: Las diferentes opciones que el decisor puede elegir.
        \item \textbf{Estados de la naturaleza}: Sucesos futuros que no están bajo el control del decisor y para los cuales se pueden conocer (o no) sus probabilidades de ocurrencia.
        \item \textbf{Resultados o desenlaces}: Las consecuencias (beneficios, costes) de cada combinación de estrategia y estado de la naturaleza.
    \end{itemize}
    En condiciones de riesgo, se conoce la probabilidad de cada estado de la naturaleza. El criterio de decisión más común es el \textbf{Valor Monetario Esperado (VME)}, que se calcula para cada alternativa sumando los resultados ponderados por sus probabilidades. Se elige la alternativa con el mayor VME.

    \item \textbf{Árboles de Decisión}: Se utilizan cuando el decisor se enfrenta a una \textbf{secuencia de decisiones} dependientes entre sí. Un árbol de decisión es un esquema gráfico que representa:
    \begin{itemize}
        \item \textbf{Nudos decisionales} (cuadrados): Puntos donde se elige entre varias alternativas (ramas decisionales).
        \item \textbf{Nudos aleatorios} (círculos): Puntos donde ocurren los estados de la naturaleza (ramas aleatorias), cada uno con su probabilidad asociada.
        \item \textbf{Resultados esperados}: Se sitúan al final de cada secuencia de ramas.
    \end{itemize}
    La resolución del árbol se realiza "hacia atrás", desde la derecha hacia la izquierda, calculando el VME en cada nudo aleatorio y eligiendo la rama con el mejor resultado en cada nudo decisional.
\end{itemize}
Ambas técnicas, aunque son herramientas estratégicas y tácticas generales, son de aplicación directa en las decisiones de diseño de productos y servicios.

% \chapter{Modelos de Objetos y Representación mediante Mallas Indexadas}

\section{Introducción a los Modelos Geométricos}

En el ámbito de la Informática Gráfica, la representación digital de entes espaciales constituye el pilar fundamental sobre el que se sustentan los procesos de visualización y simulación. Un \textbf{modelo geométrico} se define formalmente como una abstracción matemática diseñada para representar un objeto que reside en un espacio afín, típicamente de dos () o tres dimensiones ().

La condición \textit{sine qua non} para cualquier modelo computacional es que debe permitir la visualización del objeto representado mediante algoritmos. Históricamente y en la práctica actual, distinguimos varias categorías principales de representación:

\begin{itemize}
\item \textbf{Modelos de Fronteras (B-Rep):} Representan la superficie que delimita al objeto, separando el interior del exterior. La implementación más ubicua de este paradigma son las mallas de polígonos, específicamente las mallas de triángulos.
\item \textbf{Enumeración Espacial (Vóxeles):} Aproximación volumétrica donde el espacio se discretiza en celdas regulares (vóxeles), clasificando cada una como interior o exterior al objeto. Es análogo al píxel en  pero extendido a .
\item \textbf{Modelos Implícitos y Procedurales:} Se basan en definiciones matemáticas continuas, como las Funciones de Distancia con Signo (\textit{Signed Distance Functions} o SDF).
\end{itemize}

\subsection{Formalización Matemática}
Desde una perspectiva teórica rigurosa, un objeto se modela como un subconjunto de puntos  en un espacio euclídeo . Por ejemplo, una esfera de radio  y centro  se define como:
\begin{equation}
S = { \mathbf{p} \in E \mid | \mathbf{p} - \mathbf{c} | \le r }
\end{equation}
Estos conjuntos deben ser cerrados (contienen a su frontera ), acotados (extensión finita) y poseer una superficie diferenciable. Dado que la memoria de un computador es finita y discreta, es imposible representar el conjunto infinito de puntos de  de manera exacta en todos los casos, lo que obliga a recurrir a aproximaciones computacionales (mallas o vóxeles) o representaciones algorítmicas.

\subsection{Modelos Algorítmicos: SDFs}
Los modelos procedurales no almacenan geometría explícita, sino que codifican el objeto mediante una función evaluable en cualquier punto del espacio . Distinguimos dos variantes:
\begin{enumerate}
\item \textbf{Función de Pertenencia:} Devuelve un valor booleano indicando si  está dentro del objeto.
\item \textbf{Función de Distancia con Signo (SDF):} Devuelve la distancia euclídea mínima desde  hasta la superficie del objeto. El signo indica si el punto es interior (negativo) o exterior (positivo).
\end{enumerate}
Las SDFs son cruciales en la visualización moderna (especialmente en \textit{Ray Marching} y redes neuronales profundas para geometría) debido a su capacidad para representar topologías complejas y fractales con precisión arbitraria.

\section{Modelos de Fronteras: Mallas de Polígonos}

Una malla de polígonos (\textit{Polygon Mesh}) es una colección de vértices, aristas y caras que define la forma de un objeto poliédrico. Este modelo aproxima superficies curvas mediante un conjunto finito de facetas planas.

\subsection{Elementos Topológicos y Geométricos}
Es imperativo distinguir entre la \textbf{geometría} (la posición espacial de los puntos) y la \textbf{topología} (la conectividad entre dichos puntos).

\begin{itemize}
\item \textbf{Vértice:} Entidad fundamental compuesta por una posición en el espacio afín y, crucialmente, un índice identificador único entero ( a ). El índice permite abstraer la conectividad de la posición geométrica.
\item \textbf{Cara:} Superficie plana delimitada por un polígono. Se define topológicamente como una secuencia ordenada de índices de vértices.
\item \textbf{Arista:} Segmento de recta que conecta dos vértices. Se define por un par de índices de vértices.
\end{itemize}

\subsection{Propiedades de las 2-Variedades (2-Manifolds)}
Para garantizar la consistencia en algoritmos de renderizado y simulación, las mallas suelen restringirse a ser \textbf{2-variedades}. Esto implica que la vecindad local de cualquier punto de la superficie es homeomorfa a un disco abierto en . Las condiciones discretas para que una malla sea una 2-variedad incluyen:
\begin{enumerate}
\item Cada arista es compartida por, como máximo, dos caras.
\item Las caras incidentes a un vértice forman un "abanico" continuo (o un ciclo completo si es un punto interior).
\item No existen vértices aislados ni singularidades topológicas (como dos conos unidos únicamente por el ápice).
\end{enumerate}

Las mallas pueden ser \textbf{cerradas} (sin fronteras, encierran un volumen) o \textbf{abiertas} (poseen bordes). Una malla es cerrada si y solo si todas sus aristas son adyacentes a exactamente dos caras.

\subsection{Orientación y Cribado (Culling)}
La orientación de una cara se determina por el orden de recorrido de sus vértices. Esto define un vector normal a la superficie. En visualización, es fundamental la coherencia en la orientación (típicamente antihoraria o \textit{Counter-Clockwise} para la cara frontal). El \textbf{cribado de caras traseras} (\textit{back-face culling}) es una técnica de optimización que descarta el renderizado de caras cuya normal apunta en dirección opuesta al observador, asumiendo que son ocultadas por la parte delantera del objeto cerrado.

\section{Atributos de la Malla}

Más allá de la posición espacial, los vértices y caras portan información adicional necesaria para el modelo de iluminación y texturizado:

\begin{itemize}
\item \textbf{Normales:} Vectores unitarios perpendiculares a la superficie.
\begin{itemize}
\item \textit{Normal de la cara:} Calculada mediante el producto vectorial de dos aristas no colineales del polígono. Es constante para toda la cara (sombreado plano).
\item \textit{Normal del vértice:} Promedio ponderado de las normales de las caras adyacentes. Fundamental para el sombreado de Gouraud o Phong, permitiendo simular curvatura en geometría facetada.
\end{itemize}
\item \textbf{Coordenadas de Textura:} Mapean puntos de la superficie 3D a un espacio 2D de imagen (espacio UV).
\item \textbf{Colores:} Valores RGB asignados a vértices para interpolación.
\end{itemize}

Es importante notar que las discontinuidades en la superficie (bordes duros) requieren la duplicación de vértices en la misma posición espacial pero con diferentes normales, rompiendo la continuidad topológica para preservar la discontinuidad geométrica visual.

\section{Estructuras de Datos y Representación en Memoria}

La eficiencia del procesamiento gráfico depende críticamente de cómo se organizan estos datos en memoria. Analizamos las estructuras principales:

\subsection{Triángulos Aislados (Triangle List)}
Es la representación más simple. Se almacena una lista lineal de vértices, donde cada terna consecutiva define un triángulo.
\begin{equation}
V = {v_0, v_1, v_2, v_3, v_4, v_5, \dots }
\end{equation}
\textbf{Desventaja:} Alta redundancia. Un vértice compartido por  triángulos se almacena y procesa  veces. No codifica topología explícita.

\subsection{Tiras de Triángulos (Triangle Strips)}
Estructura optimizada donde cada nuevo vértice, junto con los dos anteriores, define un nuevo triángulo.
\begin{equation}
\text{Triángulo } i = {v_i, v_{i+1}, v_{i+2}}
\end{equation}
Reduce la redundancia geométrica, pero impone restricciones en la construcción de la malla y no elimina completamente la duplicación.

\subsection{Mallas Indexadas (Indexed Meshes)}
Es el estándar \textit{de facto} en la industria. Se compone de dos estructuras separadas:
\begin{enumerate}
\item \textbf{Tabla de Vértices:} Array conteniendo las coordenadas y atributos de cada vértice único.
\item \textbf{Tabla de Índices (o Caras):} Array de enteros que referencian a la tabla de vértices.
\end{enumerate}
Esta separación desacopla la topología de la geometría.
\begin{itemize}
\item \textbf{Eficiencia:} Los vértices se almacenan una sola vez (ahorro de memoria).
\item \textbf{Cómputo:} La GPU puede cachear los vértices transformados, evitando recálculos.
\item \textbf{Topología:} La conectividad es explícita a través de los índices compartidos.
\end{itemize}

\subsection{Aristas Aladas (Winged-Edge)}
Para operaciones que requieren un recorrido eficiente de la topología (como subdivisión o simplificación de mallas), las mallas indexadas son insuficientes (consultas de adyacencia costosas). La estructura de aristas aladas almacena explícitamente relaciones de vecindad en una tabla de aristas.
Cada entrada de arista contiene:
\begin{itemize}
\item Índices de los vértices extremos (inicial y final).
\item Índices de las caras adyacentes (izquierda y derecha).
\item Índices de las aristas predecesoras y sucesoras en el ciclo de cada cara.
\end{itemize}
Esto permite consultas de adyacencia en tiempo constante , a costa de un mayor consumo de memoria y complejidad de mantenimiento.

\section{Formatos de Archivo}

La persistencia de estos modelos se realiza mediante formatos estandarizados:
\begin{itemize}
\item \textbf{PLY (Polygon File Format):} Flexible, puede ser ASCII o binario. Estructura simple basada en listas de elementos (vértices, caras).
\item \textbf{OBJ (Wavefront):} Formato de texto ampliamente soportado. Permite índices independientes para posición, normales y texturas (), lo cual optimiza el almacenamiento pero requiere procesamiento para convertirlo a una estructura de malla indexada estricta (donde un índice apunta a un conjunto único de atributos).
\item \textbf{glTF/GLB:} Estándar moderno de Khronos Group. Diseñado para la transmisión eficiente (JSON para jerarquía + Binario para datos), soportando materiales PBR y escenas complejas.
\end{itemize}

\section{Análisis de Complejidad y Eficiencia}

El análisis asintótico del almacenamiento revela diferencias significativas. Para una malla cerrada triangular con topología de esfera, si  es el número de vértices, el número de caras es  y el de aristas  (consecuencia de la característica de Euler).

\begin{itemize}
\item \textbf{Mallas Indexadas:} El coste de memoria es proporcional a . Dado que los índices son enteros (menor tamaño que vectores flotantes), esta es una compresión significativa respecto a los triángulos aislados.
\item \textbf{Triángulos Aislados:} El coste es . Al ser  grande, la redundancia de datos geométricos es masiva (factor de  comparado con vértices únicos).
\end{itemize}

En conclusión, la elección de la estructura de datos (Malla Indexada vs. Tiras vs. Aristas Aladas) representa un compromiso clásico en ingeniería entre eficiencia espacial, velocidad de renderizado y capacidad de manipulación topológica.
% \chapter{Modelos Jerárquicos y Grafos de Escena}

\section{Introducción a los Modelos Jerárquicos}

En el ámbito de la Informática Gráfica, la gestión de la complejidad geométrica y espacial es un desafío fundamental. Los modelos jerárquicos constituyen una solución estructural a este problema, definiéndose como estructuras de datos organizadas en forma de grafo que representan las relaciones espaciales y lógicas entre los diversos componentes de una aplicación interactiva o una simulación visual.

Esta aproximación permite descomponer objetos complejos en componentes más simples y reutilizables. Un componente se define como un objeto geométrico básico (2D o 3D), tal como una malla poligonal, o una agrupación lógica de otros componentes. Las ventajas inherentes a este enfoque incluyen:

\begin{itemize}
    \item \textbf{Abstracción y Modularidad:} Facilita el diseño de sistemas complejos mediante la composición de entidades primitivas.
    \item \textbf{Reutilización:} Permite instanciar múltiples veces una misma definición geométrica en diferentes contextos espaciales.
    \item \textbf{Colaboración:} Habilita el desarrollo concurrente, donde distintos ingenieros pueden trabajar en componentes aislados sin interferencias destructivas.
\end{itemize}

\section{Teoría de Grafos de Escena}

El \textit{Scene Graph} o Grafo de Escena es la implementación concreta del modelo jerárquico. Matemáticamente, se modela como un Grafo Dirigido Acíclico (DAG, por sus siglas en inglés), aunque frecuentemente se simplifica a una estructura de árbol.

\subsection{Tipología de Nodos}
Los elementos constitutivos del grafo se clasifican en dos categorías principales:
\begin{enumerate}
    \item \textbf{Nodos Terminales (Hojas):} Representan los objetos geométricos fundamentales que no se subdividen más, típicamente asociados a mallas poligonales con vértices definidos en un espacio local.
    \item \textbf{Nodos No Terminales (Intermedios):} Actúan como agrupadores lógicos o contenedores de transformaciones que afectan a sus subgrafos descendientes (nodos hijos).
\end{enumerate}

Existe una relación jerárquica estricta donde cada nodo, a excepción del nodo raíz, posee al menos un nodo padre. Las aristas del grafo, que conectan padres con hijos, tienen asociadas transformaciones geométricas afines.

\subsection{Instanciación y Transformaciones Afines}
La renderización de la escena implica el recorrido del grafo. Un objeto instanciado es, en esencia, una réplica de la geometría asociada a un nodo, modificada por una cadena de transformaciones.

Cada nodo $N$ define un \textbf{Marco de Referencia Local} ($\mathcal{N}$). Las coordenadas de los vértices almacenados en dicho nodo son relativas a este marco. Para situar estos vértices en el contexto global de la escena, se debe aplicar una transformación afín $T$. Si denotamos el marco del nodo padre como $\mathcal{P}$, la relación entre ambos marcos se define mediante una matriz de transformación $M_N$ tal que:

\begin{equation}
    \mathcal{P} M_N = \mathcal{N}
\end{equation}

La posición final de un vértice en el \textbf{Marco de la Escena} (o Espacio de Mundo) se obtiene mediante la composición de las transformaciones acumuladas a lo largo del camino desde la raíz hasta el nodo en cuestión. Si un nodo $S_k$ es descendiente de una cadena de nodos con transformaciones $T_1, T_2, \dots, T_k$, la transformación global $T_{global}$ aplicada a la geometría de $S_k$ es:

\begin{equation}
    T_{global} = T_1 \circ T_2 \circ \dots \circ T_k
\end{equation}

Esto implica que las operaciones se aplican de izquierda a derecha en términos de composición de funciones, o equivalentemente, multiplicando las matrices de transformación en el orden correspondiente descendente.

\section{Arquitectura de Grafos de Escena en Godot Engine}

El motor gráfico Godot implementa estos conceptos teóricos a través de una arquitectura basada en Nodos, Escenas y Árboles de Escena.

\subsection{Estructura Fundamental}
\begin{itemize}
    \item \textbf{Proyecto:} Unidad contenedora de todos los recursos y configuraciones.
    \item \textbf{Escena:} Un árbol de nodos que puede ser guardado en disco (archivos \texttt{.tscn}). Una escena posee siempre un nodo raíz.
    \item \textbf{Nodo:} La unidad atómica de construcción. Todo nodo debe pertenecer a una escena y poseer un único padre (salvo la raíz).
    \item \textbf{Árbol de Escena (SceneTree):} La estructura en tiempo de ejecución que contiene la jerarquía completa de nodos activos.
\end{itemize}

\subsection{Tipos de Nodos y Gestión de Memoria}
Godot clasifica los nodos según su funcionalidad espacial:
\begin{itemize}
    \item \textbf{Node2D (CanvasItem):} Base para objetos bidimensionales.
    \item \textbf{Node3D:} Base para objetos tridimensionales (anteriormente conocido como Spatial).
    \item \textbf{Instancia de Escena:} Un mecanismo de composición que permite incluir una escena completa dentro de otra como si fuera un único nodo. Esto es crucial para la instanciación múltiple de objetos complejos.
\end{itemize}

Desde la perspectiva de la ingeniería de software y la eficiencia computacional, es imperativo evitar la duplicación de datos geométricos pesados. Godot utiliza el patrón de diseño \textit{Flyweight} mediante la clase \texttt{MeshInstance} (2D o 3D). Estos nodos no contienen la geometría en sí, sino una referencia a un recurso de tipo \texttt{Mesh}. Dado que \texttt{Mesh} hereda de \texttt{RefCounted}, el motor gestiona automáticamente la memoria, liberando el recurso solo cuando el contador de referencias desciende a cero.

\section{Matemática de las Transformaciones en Godot}

Cada nodo espacial ($N$) mantiene una propiedad \texttt{transform} que codifica la matriz $M_N$ responsable de mapear las coordenadas locales al espacio del padre.

\subsection{Transformaciones en el Espacio 2D}
En 2D, la transformación se representa mediante la clase \texttt{Transform2D}, una matriz de $3 \times 2$ (asumiendo coordenadas homogéneas implícitas). Los atributos que componen esta matriz son:
\begin{itemize}
    \item \textbf{Position:} Vector de traslación $(t_x, t_y)$.
    \item \textbf{Rotation:} Escalar $\theta$ (radianes).
    \item \textbf{Scale:} Vector de escalado $(s_x, s_y)$.
    \item \textbf{Skew:} Ángulo de distorsión o cizalladura.
\end{itemize}

El orden de composición de estas operaciones es crítico para la consistencia matemática. En Godot, la matriz resultante equivale a aplicar las operaciones en el siguiente orden (de derecha a izquierda en notación matricial sobre vectores columna):
\begin{equation}
    M_{2D} = T_{raslacion} \cdot R_{otacion} \cdot Skew \cdot S_{cale}
\end{equation}

\subsection{Transformaciones en el Espacio 3D}
En 3D, se utiliza la clase \texttt{Transform3D} (matriz $4 \times 3$ o $4 \times 4$ conceptualmente). A diferencia del caso 2D, no se incluye el \textit{skew} por defecto, y la rotación es vectorial (ángulos de Euler o Cuaterniones). El orden de composición estándar es:
\begin{equation}
    M_{3D} = T_{raslacion} \cdot R_{otacion} \cdot S_{cale}
\end{equation}
Las rotaciones se aplican intrínsecamente en el orden Y-X-Z (Euler), aunque esto es configurable.

\subsection{Manipulación Programática de Transformaciones}
La actualización de la matriz de transformación puede realizarse mediante asignación directa de propiedades o mediante métodos de composición. Es vital distinguir entre:
\begin{itemize}
    \item \textbf{Composición por la Izquierda (Global/Padre):} Métodos como \texttt{rotate()} o \texttt{translate()} aplican la transformación respecto al marco de referencia del padre. Matemáticamente: $M'_{N} = M_{operacion} \cdot M_{N}$.
    \item \textbf{Composición por la Derecha (Local):} Métodos como \texttt{rotate\_object\_local()} aplican la transformación respecto al marco local del objeto. Matemáticamente: $M'_{N} = M_{N} \cdot M_{operacion}$.
\end{itemize}

\section{Gestión Dinámica del Árbol de Escena}

La ingeniería de aplicaciones interactivas requiere la manipulación del grafo en tiempo de ejecución (runtime). Godot proporciona una API robusta para este fin mediante \textit{GDScript}.

\subsection{Ciclo de Vida de los Nodos}
\begin{enumerate}
    \item \textbf{Instanciación:} Se utiliza el método \texttt{.new()} de la clase correspondiente. El nodo se crea en un estado "huérfano".
    \item \textbf{Vinculación:} Se inserta en el grafo mediante \texttt{add\_child()}. Solo entonces se hace activo y visible en la escena.
    \item \textbf{Desvinculación:} El método \texttt{remove\_child()} desconecta el nodo del árbol, devolviéndolo al estado huérfano pero manteniéndolo en memoria.
    \item \textbf{Destrucción:} Para liberar la memoria, se invoca \texttt{queue\_free()}. Este método no elimina el nodo inmediatamente, sino que agenda su destrucción segura al finalizar el procesamiento del \textit{frame} actual, evitando errores de referencia colgante durante la ejecución de la lógica.
\end{enumerate}

\subsection{Búsqueda y Acceso}
El acceso a nodos dentro de la jerarquía se realiza mediante rutas relativas o absolutas utilizando \texttt{get\_node("ruta/al/nodo")}. La nomenclatura de los nodos (\texttt{name}) actúa como identificador único entre hermanos dentro del mismo padre.

\section{Diseño de Grafos Parametrizados}

Un concepto avanzado en el modelado jerárquico es la \textbf{parametrización}. Esto implica diseñar el grafo de tal manera que las transformaciones de sus nodos dependan de variables externas (parámetros o grados de libertad), en lugar de valores estáticos.

\subsection{Aplicaciones}
\begin{itemize}
    \item \textbf{Animación Procedural:} Vincular parámetros de transformación (rotación, escala) al tiempo ($t$). Por ejemplo, una rotación continua se define como $\theta(t) = \theta_0 + \omega \cdot t$.
    \item \textbf{Variación de Diseño:} Generar múltiples variantes de un objeto (e.g., un edificio) alterando parámetros como altura o anchura, los cuales propagan cambios de escala a través de la jerarquía.
    \item \textbf{Interacción:} Permitir que las entradas del usuario modifiquen directamente los parámetros de la matriz de transformación.
\end{itemize}

\subsection{Implementación Lógica}
En la práctica, esto se traduce en scripts que sobrescriben el método \texttt{\_process(delta)}. En cada ciclo de renderizado, se recalculan las matrices de transformación basándose en el estado actual de los parámetros y el tiempo delta transcurrido, asegurando una animación suave e independiente de la tasa de fotogramas (frame-rate independent).
% \section*{¡Excelente comienzo!}

El texto es claro y completo, contiene información fundamental sobre la teoría monetaria clásica. No está cortado, así que podemos trabajar perfectamente con él.

Aquí tienes el análisis estructurado para asegurar ese 10 en el examen.

\subsection*{1. EXPLICACIÓN DETALLADA}

Este fragmento cubre dos grandes bloques: la definición general del \textbf{Sistema Monetario Internacional (SMI)} y el funcionamiento específico del \textbf{Patrón Oro}.

\textbf{Sobre el SMI (Conceptos generales):}

El texto define el SMI como el conjunto de reglas e instituciones que gestionan los pagos entre países. Imagínalo como el "reglamento de tráfico" del dinero mundial.

\begin{itemize}
    \item \textbf{Elementos clave:} Para que funcione, necesita una "moneda base" (como fue el oro o es el dólar), acuerdos sobre cómo comportarse (normas bancarias) y un \textbf{mecanismo de ajuste}.
    \item \textbf{Mecanismo de ajuste:} Esto es crucial. Si un país gasta más de lo que ingresa (déficit), ¿cómo se soluciona?
    \begin{itemize}
        \item \textit{Tipo de cambio flexible:} La moneda baja de precio (se deprecia) sola.
        \item \textit{Tipo de cambio fijo:} El gobierno interviene para mantener el precio.
    \end{itemize}
\end{itemize}

\textbf{Sobre el Patrón Oro (Finales s. XIX - I Guerra Mundial):}

Es un sistema de \textbf{tipos de cambio fijos}.

\begin{itemize}
    \item \textbf{La teoría (David Hume):} Se basa en el \textit{mecanismo de flujo de especie-precio}. Es un sistema automático. Si España compra mucho a Inglaterra (déficit español), España tiene que pagar en oro. El oro sale de España hacia Inglaterra.
    \item \textbf{La consecuencia:} Como en el Patrón Oro la cantidad de billetes depende del oro que tengas en la caja fuerte del banco central, si sale oro, hay menos dinero circulando.
    \item \textbf{El ajuste:} Menos dinero $\rightarrow$ la gente gasta menos $\rightarrow$ los precios bajan (deflación) $\rightarrow$ los productos españoles se vuelven baratos $\rightarrow$ España exporta más $\rightarrow$ el oro vuelve a entrar. ¡El sistema se equilibra solo!
    \item \textbf{Puntos del oro:} Aunque el cambio es fijo, había un pequeño margen de maniobra. Mover oro físico en barco cuesta dinero (seguro, transporte). Si la diferencia de valor de la moneda era muy pequeña, no merecía la pena enviar el barco con oro. Solo si la moneda caía por debajo del coste de enviar el oro (el "punto de salida"), se producía el envío físico.
    \item \textbf{El problema (Sesgo deflacionista):} Para corregir un déficit, el país tenía que sufrir una bajada de precios (deflación). Esto arruinaba a los deudores (como agricultores con hipotecas), porque sus ingresos bajaban al vender sus cosechas baratas, pero la deuda con el banco seguía siendo la misma cantidad nominal.
\end{itemize}

\subsection*{2. PUNTOS CLAVE (BULLET POINTS)}

Memoriza estos datos, son "carne de cañón" para el test:

\begin{itemize}
    \item \textbf{Objetivos del SMI:} Ajuste (corregir desequilibrios), Liquidez (reservas suficientes) y Confianza.
    \item \textbf{Cronología del Patrón Oro:} Último cuarto del siglo XIX hasta la I Guerra Mundial (aprox. 1870-1914).
    \item \textbf{Líder del sistema:} Gran Bretaña.
    \item \textbf{Teórico principal:} David Hume (siglo XVIII) con el "Modelo de los flujos de oro y los precios".
    \item \textbf{Determinación del tipo de cambio:} Indirecta. Se comparaba la paridad de cada moneda con el oro (ej. cuántos dólares vale una onza vs. cuántas libras vale una onza).
    \item \textbf{Puntos del oro:} Bandas de fluctuación del tipo de cambio determinadas por los \textbf{costes de transporte y seguro} del oro.
    \item \textbf{Ajuste automático (Precio-Flujo en especie):}
    \begin{itemize}
        \item Déficit comercial $\rightarrow$ Salida de oro.
        \item Salida de oro $\rightarrow$ Reducción de oferta monetaria $\rightarrow$ Bajada de precios (ganancia de competitividad).
    \end{itemize}
    \item \textbf{Principal fallo:} Sesgo deflacionista (perjudica a los deudores y a la actividad económica interna para salvar el equilibrio externo).
\end{itemize}

\subsection*{3. POSIBLES PREGUNTAS DE EXAMEN}

Aquí tienes tres preguntas diseñadas para poner a prueba tu comprensión, con el formato "trampa" que suelen usar:

\begin{enumerate}
    \item \textbf{Pregunta 1:} Según el modelo de David Hume sobre el Patrón Oro, ¿cuál es la consecuencia inmediata de que un país tenga un déficit continuado en su balanza de pagos?\\
    A) Una entrada masiva de oro para financiar el déficit.\\
    B) Una apreciación de su moneda local debido a la demanda de importaciones.\\
    C) Una salida de oro que provoca una disminución de la oferta monetaria y una caída de los precios internos.\\
    D) Ninguna de las anteriores es correcta.
    \item \textbf{Pregunta 2:} ¿Qué función cumplían los llamados "puntos del oro" en el sistema de tipos de cambio fijos del Patrón Oro?\\
    A) Establecían el precio máximo al que el Banco Central podía vender oro.\\
    B) Determinaban las bandas de fluctuación del tipo de cambio basadas en los costes de transporte y seguro del metal.\\
    C) Señalaban el momento exacto en el que el gobierno debía devaluar la moneda oficialmente.\\
    D) Todas las anteriores son correctas.
    \item \textbf{Pregunta 3:} Para el correcto funcionamiento de un Sistema Monetario Internacional, ¿qué condiciones básicas deben cumplirse según el texto?\\
    A) Debe existir un mecanismo de ajuste eficiente con el menor coste posible.\\
    B) Debe proporcionar suficiente liquidez para las transacciones internacionales.\\
    C) Debe generar confianza en los agentes participantes.\\
    D) Todas las anteriores son correctas.
\end{enumerate}

\textbf{Soluciones:}
\begin{itemize}
    \item \textbf{P1:} La respuesta correcta es la \textbf{C}. Es la esencia del mecanismo de ajuste de Hume: Déficit = Salida de Oro = Bajada de precios.
    \item \textbf{P2:} La respuesta correcta es la \textbf{B}. Los puntos del oro son los límites (coste de flete/seguro) que determinan cuándo es rentable mover físicamente el oro; actúan como unas "mini bandas" de fluctuación.
    \item \textbf{P3:} La respuesta correcta es la \textbf{D (Todas son correctas)}. El texto enumera explícitamente estas tres condiciones: Ajuste, Liquidez y Confianza.
\end{itemize}

\vspace{0.5cm}
¿Cómo ves este primer bloque? ¿Te ha quedado claro el concepto de los "Puntos del oro"? ¡Pásame el siguiente fragmento cuando estés listo!

\section*{¡Magnífico! Este bloque es \textbf{crucial}. Aquí es donde suelen ir a "pillarte" con las fechas y las diferencias sutiles entre sistemas. Tenemos tres fases históricas muy marcadas: el caos de entreguerras, el orden de Bretton Woods y su colapso.}

Vamos a desglosarlo para asegurar esa nota alta.

\hrulefill

\subsection*{1. EXPLICACIÓN DETALLADA}

\textbf{A. El caos de Entreguerras (1918-1939):}\\
Durante la I Guerra Mundial, los países dejaron de cambiar billetes por oro para guardar el metal para la guerra. Al acabar, intentaron volver a la normalidad, pero cometieron errores:

\begin{itemize}
    \item \textbf{El error de Gran Bretaña:} Intentó volver al Patrón Oro al \textbf{mismo precio} (paridad) que tenía antes de la guerra. Pero su economía estaba peor (inflación, deuda). Al fijar la libra tan cara (sobrevaluada), sus productos eran carísimos para el extranjero $\rightarrow$ cayeron las exportaciones $\rightarrow$ paro masivo. Tuvieron que rendirse y salir del patrón oro en 1931.
    \item \textbf{Devaluaciones competitivas:} Cuando el sistema colapsó, los países empezaron a jugar sucio. "Si yo bajo el valor de mi moneda, mis productos son más baratos y vendo más que mi vecino". Esto provocó proteccionismo y empeoró la Gran Depresión.
\end{itemize}

\textbf{B. El Sistema de Bretton Woods (1944-1971):}\\
Para evitar el caos anterior, EE. UU. (la nueva gran potencia) diseñó un sistema nuevo.

\begin{itemize}
    \item \textbf{Patrón Cambios-Oro (Gold-Exchange Standard):} Ya no es "todos contra el oro". Ahora es una pirámide:
    \begin{enumerate}
        \item \textbf{Dólar:} Es la única moneda convertible en oro (a \textbf{35\$ la onza}). EE. UU. garantiza cambiar dólares por oro a los bancos centrales.
        \item \textbf{Resto de monedas:} Fijan su precio respecto al \textbf{Dólar}.
    \end{enumerate}
    \item \textbf{Tipo de cambio AJUSTABLE:} Es un sistema de tipo fijo, pero "con truco". Si un país tenía una crisis muy grave (\textit{desequilibrio fundamental}), el FMI le daba permiso para cambiar su paridad (devaluar o revaluar).
    \item \textbf{Bandas:} Las monedas podían moverse un poquito ($\pm$1\%) arriba o abajo.
    \item \textbf{El problema de la especulación:} Como las bandas eran estrechas, los especuladores apostaban a lo seguro. Si veían una moneda débil, sabían que solo podía bajar (devaluarse). Apostaban en contra y ganaban mucho dinero con riesgo casi nulo.
\end{itemize}

\textbf{C. El Dilema de Triffin (\textit{¡Concepto Clave!}):}\\
Es la paradoja que mató a Bretton Woods. Para que el mundo creciera, hacían falta dólares (liquidez). Para que hubiera dólares, EE. UU. tenía que gastar más de lo que ingresaba (déficit).

\begin{itemize}
    \item Si EE. UU. tiene mucho déficit $\rightarrow$ hay muchos dólares por el mundo $\rightarrow$ la gente desconfía de que haya oro suficiente para respaldarlos (\textbf{Falta de Confianza}).
    \item Si EE. UU. corrige el déficit $\rightarrow$ no hay dólares para el comercio mundial (\textbf{Falta de Liquidez}).
    \item \textbf{Solución fallida:} Crearon los \textbf{DEG (Derechos Especiales de Giro)} en 1969 como una moneda "artificial" del FMI para no depender tanto del dólar, pero llegaron tarde.
\end{itemize}

\textbf{D. La Caída (1971-1973):}\\
EE. UU. gastó demasiado (Guerra de Vietnam, gasto social) y Alemania y Japón exportaban mucho. Había demasiados dólares y poco oro.

\begin{itemize}
    \item \textbf{Nixon Shock (1971):} Nixon dijo "se acabó" y rompió la convertibilidad del dólar en oro. El dólar ya no valía una cantidad fija de oro.
    \item \textbf{Acuerdo Smithsoniano:} Un intento desesperado de salvar el sistema. Devaluaron el dólar (a 38\$ la onza) y ampliaron las bandas al \textbf{2,25\%}. Fracasó en 1973.
\end{itemize}

\hrulefill

\subsection*{2. PUNTOS CLAVE (BULLET POINTS)}

\begin{itemize}
    \item \textbf{Entreguerras:} Caracterizado por la inestabilidad, el fin de la libre circulación de capitales y las \textbf{devaluaciones competitivas}.
    \item \textbf{Error británico (1925):} Volver a la paridad de preguerra provocó sobrevaluación de la libra y desempleo. Abandonó el patrón en 1931.
    \item \textbf{Bretton Woods (1944):}
    \begin{itemize}
        \item Creación del \textbf{FMI} (para dar préstamos y estabilidad).
        \item Sistema de \textbf{tipo de cambio fijo pero ajustable} (ante desequilibrios fundamentales).
        \item \textbf{Ancla:} El Dólar (convertible en oro a 35\$/onza). El resto se fija al dólar.
        \item \textbf{Bandas:} $\pm$1\%.
    \end{itemize}
    \item \textbf{Dilema de Triffin:} Conflicto entre proveer \textbf{liquidez} (necesita déficit de EE. UU.) y mantener la \textbf{confianza} (necesita superávit/estabilidad de EE. UU.).
    \item \textbf{DEG (1969):} Activo de reserva creado para complementar al dólar y al oro.
    \item \textbf{Crisis final:}
    \begin{itemize}
        \item \textbf{Agosto 1971:} Nixon suspende la convertibilidad dólar-oro.
        \item \textbf{Acuerdo Smithsoniano:} Intento de rescate. Dólar a 38\$/onza y bandas al $\pm$2,25\%.
        \item \textbf{1973:} Colapso definitivo y paso a la flotación controlada.
    \end{itemize}
\end{itemize}

\hrulefill

\subsection*{3. POSIBLES PREGUNTAS DE EXAMEN}

\textbf{Pregunta 1: En el contexto del sistema de Bretton Woods, ¿en qué consistía el conocido como "Dilema de Triffin"?}\\
A) En la imposibilidad de mantener tipos de cambio fijos y libre circulación de capitales simultáneamente.\\
B) En la contradicción de que EE. UU. debía incurrir en déficit para proveer liquidez internacional, lo cual minaba la confianza en la convertibilidad del dólar en oro.\\
C) En el conflicto entre los objetivos de inflación del FMI y las políticas de empleo de los países miembros.\\
D) Ninguna de las anteriores es correcta.

\vspace{0.2cm}

\textbf{Pregunta 2: ¿Cuál de las siguientes afirmaciones sobre el Acuerdo Smithsoniano es correcta?}\\
A) Fue el tratado que dio origen al Fondo Monetario Internacional en 1944.\\
B) Estableció la vuelta al Patrón Oro clásico tras la Segunda Guerra Mundial.\\
C) Fue un intento de salvar el sistema de Bretton Woods ampliando las bandas de fluctuación al 2,25\% y devaluando el dólar frente al oro.\\
D) Todas las anteriores son correctas.

\vspace{0.2cm}

\textbf{Pregunta 3: Durante el periodo de entreguerras, ¿cuál fue una de las principales causas de la inestabilidad económica internacional descrita en el texto?}\\
A) La adopción generalizada y exitosa del sistema de tipos de cambio flexibles desde 1918.\\
B) La política de "devaluaciones competitivas" y el retorno al proteccionismo tras el colapso del sistema.\\
C) La decisión de Gran Bretaña de devaluar su moneda inmediatamente después de la guerra para ganar competitividad.\\
D) Todas las anteriores son correctas.

\hrulefill

\subsection*{Soluciones:}

\begin{itemize}
    \item \textbf{P1:} La respuesta correcta es la \textbf{B}. Es la definición de libro del dilema: Liquidez vs. Confianza.
    \item \textbf{P2:} La respuesta correcta es la \textbf{C}. Es el "último suspiro" del sistema. Fíjate que cambia el precio del oro (\$38 en vez de \$35) y amplía la banda.
    \item \textbf{P3:} La respuesta correcta es la \textbf{B}. La A es falsa (intentaron volver al oro) y la C es falsa (Gran Bretaña hizo lo contrario, revaluó al volver a la paridad antigua, lo que la arruinó).
\end{itemize}

\vspace{0.5cm}
¡Vas muy bien! Estos temas históricos suelen ser densos, pero si entiendes la lógica de \textit{por qué} falló cada uno (Rigidez en el S.XIX $\rightarrow$ Caos en Entreguerras $\rightarrow$ Rigidez del Dólar en BW), tienes el aprobado asegurado.

¿Listo para el siguiente?

\section*{¡Perfecto! Hemos llegado a la parte moderna y a la integración europea.}

Este fragmento es denso en conceptos específicos ("Euromercado" no es lo que parece intuitivamente) y en la cronología de la UE.

Aquí tienes el análisis para dominar este bloque:

\hrulefill

\subsection*{1. EXPLICACIÓN DETALLADA}

\textbf{A. Los Euromercados (¡Cuidado con el nombre!):}\\
No pienses en "Europa". Un "Euromercado" es cualquier mercado donde se opera con una moneda \textbf{fuera de su país de origen}.

\begin{itemize}
    \item \textit{Ejemplo:} Si un banco en Londres (Reino Unido) presta dólares (EE. UU.), eso es el euromercado de dólares ("eurodólares"). Si un banco en Singapur presta yenes japoneses, eso también es un euromercado. El prefijo "euro" aquí significa "externo" u "offshore" (pero legal y regulado, no confundir con paraísos fiscales opacos).
    \item \textbf{Origen:} Nació en la Guerra Fría. La URSS tenía dólares pero le daba miedo guardarlos en bancos de EE. UU. (por si se los congelaban), así que los guardó en bancos de Londres.
    \item \textbf{Diferencia clave:}
    \begin{itemize}
        \item \textit{Euromercado:} Operas en moneda extranjera (ej. Dólares en España).
        \item \textit{Mercado doméstico:} Operas en moneda local (ej. Euros en España).
        \item \textit{Mercado Off-shore puro:} Intermedias entre dos extranjeros (ej. un banco en Panamá gestiona dinero de un alemán para prestárselo a un brasileño).
    \end{itemize}
\end{itemize}

\textbf{B. El Sistema Monetario Europeo (SME):}\\
Es el "abuelo" del Euro. Europa quería estabilidad entre sus monedas para poder comerciar sin miedo a que el precio cambiara de repente.

\begin{enumerate}
    \item \textbf{La Serpiente Monetaria (1972):} Primer intento fallido. Querían mantener las monedas europeas fluctuando muy poco entre ellas (como una serpiente moviéndose dentro de un túnel estrecho), mientras todas juntas flotaban respecto al dólar. Fracasó porque no había coordinación y Alemania era demasiado fuerte; acabó siendo una "zona marco".
    \item \textbf{El SME (1979):} El intento serio.
    \begin{itemize}
        \item \textbf{Objetivos:} Estabilidad externa (tipo de cambio) e interna (precios/inflación).
        \item \textbf{El ECU:} Una moneda "cesta" (ficticia, no existía en billetes) que era la media de todas las monedas. Servía de referencia.
        \item \textbf{Mecanismo de ajuste:} Si el Franco francés bajaba mucho respecto al Marco alemán, \textit{ambos} bancos centrales debían intervenir. En teoría era simétrico, pero en la práctica Alemania (el país fuerte) no quería imprimir marcos para ayudar a los demás porque temía la inflación. Así que el peso del ajuste caía siempre sobre el débil (Francia tenía que gastar reservas o subir tipos de interés).
    \end{itemize}
\end{enumerate}

\textbf{C. La Unión Económica y Monetaria (UEM - El Euro):}\\
Para evitar las crisis del SME, decidieron crear una moneda única.

\begin{itemize}
    \item \textbf{Tratado de Maastricht (1991):} Puso las reglas del juego. No cualquiera podía entrar al club del Euro. Había que cumplir los \textbf{Criterios de Convergencia}:
    \begin{enumerate}
        \item \textbf{Inflación:} Baja (cerca de los mejores).
        \item \textbf{Tipos de interés:} Bajos.
        \item \textbf{Déficit público:} Máximo 3\% del PIB.
        \item \textbf{Deuda pública:} Máximo 60\% del PIB (o bajando rápido).
        \item \textbf{Estabilidad cambiaria:} No haber devaluado en 2 años.
    \end{enumerate}
    \item \textbf{Cronología del Euro:}
    \begin{itemize}
        \item 1999: Nace el Euro como moneda contable (bancos/bolsa). Se fijan los tipos de cambio irrevocables.
        \item 2002: Llegan los billetes y monedas físicos a nuestros bolsillos.
    \end{itemize}
\end{itemize}

\hrulefill

\subsection*{2. PUNTOS CLAVE (BULLET POINTS)}

\begin{itemize}
    \item \textbf{Definición de Euromercado:} Mercado bancario en una moneda distinta a la del país donde se realiza la operación (ej. Dólares en Londres). Origen ligado a la URSS y los "petrodólares".
    \item \textbf{Serpiente Monetaria (1972):} Primer intento de estabilidad cambiaria europea. Fracasó y se convirtió en una "zona marco".
    \item \textbf{SME (1979):} Creado por Giscard D'Estaing y Helmut Schmidt.
    \item \textbf{ECU:} Unidad de cuenta (cesta de monedas).
    \item \textbf{Parrilla de paridades:} Bandas de fluctuación ($\pm$2,25\% general, luego $\pm$15\% tras crisis 92-93).
    \item \textbf{Asimetría:} Aunque teóricamente el ajuste era compartido, en la práctica recaía sobre la moneda débil.
    \item \textbf{Tratado de Maastricht (1991):} Establece el camino al Euro (enfoque gradual, no de choque).
    \item \textbf{Criterios de Convergencia (Maastricht):} Inflación controlada, Déficit $<$ 3\%, Deuda $<$ 60\%, Tipos de interés bajos, Estabilidad cambiaria.
    \item \textbf{Fechas Euro:}
    \begin{itemize}
        \item \textbf{1999:} Inicio de la 3ª fase (fijación de tipos, nace el BCE).
        \item \textbf{2002:} Circulación física de billetes y monedas.
    \end{itemize}
\end{itemize}

\hrulefill

\subsection*{3. POSIBLES PREGUNTAS DE EXAMEN}

\textbf{Pregunta 1: En el contexto de los mercados financieros internacionales, ¿qué se entiende por una operación en el "euromercado"?}\\
A) Cualquier operación realizada en euros entre países de la Unión Europea.\\
B) Una operación bancaria realizada en una moneda distinta a la del país donde se localiza la institución bancaria (ej. un depósito en dólares en un banco de París).\\
C) La compraventa de bonos emitidos exclusivamente por el Banco Central Europeo.\\
D) Ninguna de las anteriores es correcta.

\vspace{0.2cm}

\textbf{Pregunta 2: ¿Cuál fue una de las principales críticas al funcionamiento práctico del Sistema Monetario Europeo (SME) antes de la creación del Euro?}\\
A) La inexistencia de una unidad de cuenta común como el ECU.\\
B) La asimetría en el ajuste, donde el peso de la intervención recaía principalmente en los países con monedas débiles, a pesar de que la teoría estipulaba cooperación bilateral.\\
C) La prohibición absoluta de modificar los tipos de cambio centrales, lo que impedía cualquier tipo de ajuste ante crisis.\\
D) Todas las anteriores son correctas.

\vspace{0.2cm}

\textbf{Pregunta 3: Según los criterios de convergencia de Maastricht para acceder a la Unión Económica y Monetaria, ¿cuáles eran los límites establecidos para el déficit público y la deuda pública respectivamente?}\\
A) Déficit público no superior al 3\% del PIB y Deuda pública no superior al 60\% del PIB.\\
B) Déficit público no superior al 1\% del PIB y Deuda pública no superior al 100\% del PIB.\\
C) Déficit público 0\% (equilibrio presupuestario) y Deuda pública no superior al 50\% del PIB.\\
D) Ninguna de las anteriores, ya que Maastricht solo fijó criterios de inflación y tipos de cambio.

\hrulefill

\subsection*{Soluciones:}

\begin{itemize}
    \item \textbf{P1:} La respuesta correcta es la \textbf{B}. Recuerda: "Euro" aquí significa "externo", no la moneda euro.
    \item \textbf{P2:} La respuesta correcta es la \textbf{B}. Alemania no quería "importar inflación", así que no ayudaba tanto como la teoría decía. Los países débiles sufrían el ajuste.
    \item \textbf{P3:} La respuesta correcta es la \textbf{A}. Estos números (3\% y 60\%) son fundamentales y muy preguntables.
\end{itemize}

\vspace{0.5cm}
¡Con esto tienes cubierto todo el bloque de historia y sistemas monetarios! ¿Te queda algún fragmento más o quieres que repasemos algún concepto específico? ¡Vamos a por ese 10!


\section*{¡Vamos allá! Este es el último bloque teórico y es \textbf{denso} en conceptos modernos y propuestas de reforma. Aquí es donde los profesores suelen poner preguntas para diferenciar el notable del sobresaliente, porque incluye nombres propios (McKinnon, Williamson, Tobin) y definiciones técnicas muy precisas.}

Aquí tienes el análisis para rematar el tema:

\hrulefill

\subsection*{1. EXPLICACIÓN DETALLADA}

\textbf{A. La coordinación internacional post-Bretton Woods:}\\
Tras el colapso de 1973, el mundo pasó a tipos de cambio flexibles, pero no fue un caos total. Hubo intentos de "poner orden":

\begin{itemize}
    \item \textbf{Acuerdo del Plaza (1985):} El \textbf{G-5} (EE. UU., Reino Unido, Alemania, Japón y Francia) se reunió en el Hotel Plaza. El problema era que el \textbf{Dólar estaba demasiado fuerte} (apreciado). Acordaron intervenir para bajarlo.
    \item \textbf{Acuerdo del Louvre (1987):} Dos años después, ya con el \textbf{G-7} (se unieron Italia y Canadá), decidieron que el dólar ya había bajado suficiente y estaba en su nivel correcto. Acordaron intervenir solo para \textbf{estabilizarlo}.
    \item \textbf{Régimen actual:} Se llama \textbf{"Flotación sucia" o dirigida}. Las monedas flotan libremente (el mercado decide el precio), pero los Bancos Centrales intervienen puntualmente si la cosa se desmadra.
\end{itemize}

\textbf{B. Regímenes Cambiarios "Exóticos" (Emergentes):}\\
No todo es fijo o flexible. Hay híbridos muy usados en países en desarrollo:

\begin{itemize}
    \item \textbf{Crawling Peg (Tipos de cambio móviles):} Es un tipo de cambio fijo que \textbf{cambia poquito a poco}. Imagina que tu moneda pierde valor un 1\% cada mes de forma programada porque tienes mucha inflación. Esto da certeza a los inversores pero ajusta la realidad económica.
    \item \textbf{Caja de Conversión (Currency Board):} Es el sistema \textbf{más rígido} (usado por Argentina en los 90). Por ley, el país promete cambiar su moneda por una extranjera (ej. dólar) a un precio fijo.
    \begin{itemize}
        \item \textit{El truco:} El Banco Central \textbf{NO puede imprimir billetes} si no tiene dólares en la caja fuerte que los respalden.
        \item \textit{Consecuencia:} Ganas mucha credibilidad (se acaba la inflación), pero el gobierno pierde totalmente el control de la política monetaria (si hay crisis, no pueden "imprimir dinero" para ayudar).
    \end{itemize}
\end{itemize}

\textbf{C. Propuestas de Reforma (El "No-Sistema"):}\\
Como el sistema actual es un poco anárquico, hay tres grandes propuestas teóricas para arreglarlo:

\begin{enumerate}
    \item \textbf{Propuesta de McKinnon:} Quiere volver a tipos fijos estrictos. Dice que el problema es la \textbf{sustitución de monedas} (la gente cambia dólares por euros y crea volatilidad). Propone controlar la \textbf{oferta monetaria mundial} en conjunto.
    \item \textbf{Propuesta de Williamson (Zonas Objetivo):} Ni fijo ni flexible. Propone calcular un tipo de cambio "justo" a largo plazo (\textbf{FEER} - Tipo de Cambio de Equilibrio Fundamental) y dejar que la moneda se mueva en unas \textbf{bandas amplias} ($\pm$10\%). Si se sale de ahí, se interviene suavemente.
    \item \textbf{La Tasa Tobin:} Su idea es poner \textbf{arena en las ruedas} de la especulación. Un impuesto pequeñito (0,1\% - 0,25\%) sobre \textit{cada} cambio de moneda.
    \begin{itemize}
        \item \textit{Efecto:} A quien cambia dinero por comercio real (turismo, importación) no le afecta ese 0,1\%. Pero al especulador que compra y vende mil veces al día para ganar céntimos, ese impuesto le arruina el negocio.
        \item \textit{Problema:} Si no lo aplican TODOS los países a la vez, el dinero se va a paraísos fiscales.
    \end{itemize}
\end{enumerate}

\hrulefill

\subsection*{2. PUNTOS CLAVE (BULLET POINTS)}

\begin{itemize}
    \item \textbf{Acuerdo del Plaza (1985):} G-5. Objetivo: \textbf{Depreciar} el dólar.
    \item \textbf{Acuerdo del Louvre (1987):} G-7 (G-5 + Italia y Canadá). Objetivo: \textbf{Estabilizar} el dólar (ya estaba en equilibrio).
    \item \textbf{Régimen actual:} Predominio de la \textbf{flotación sucia} o dirigida (intervención puntual).
    \item \textbf{Crawling Peg (Tipo de cambio móvil):} Ajuste periódico y preanunciado de la paridad para compensar diferencias de inflación.
    \item \textbf{Caja de Conversión (Currency Board):}
    \begin{itemize}
        \item Respaldo total de la emisión monetaria con reservas extranjeras.
        \item Renuncia a la política monetaria nacional.
        \item Ejemplos históricos: Argentina (hasta 2001), Estonia, Lituania.
    \end{itemize}
    \item \textbf{Propuesta McKinnon:} Tipos de cambio fijos (bandas $\pm$5\%) controlando la oferta monetaria global.
    \item \textbf{Propuesta Williamson:} \textbf{Zonas Objetivo} con bandas amplias ($\pm$10\%) alrededor del \textbf{FEER} (Tipo de Cambio de Equilibrio Fundamental).
    \item \textbf{Tasa Tobin:} Impuesto sobre transacciones cambiarias para frenar la especulación a corto plazo ("echar arena en los engranajes").
\end{itemize}

\hrulefill

\subsection*{3. POSIBLES PREGUNTAS DE EXAMEN}

\textbf{Pregunta 1: ¿Cuál es la diferencia fundamental entre el Acuerdo del Plaza (1985) y el Acuerdo del Louvre (1987)?}\\
A) El Acuerdo del Plaza buscaba la apreciación del dólar, mientras que el Louvre buscaba su depreciación.\\
B) El Acuerdo del Plaza, firmado por el G-5, buscaba corregir la excesiva apreciación del dólar, mientras que el Louvre (G-7) anunció que el dólar ya había alcanzado un valor consistente y buscaba estabilizarlo.\\
C) El Acuerdo del Louvre supuso la creación del Euro, mientras que el Plaza estableció el fin del patrón oro.\\
D) Ninguna de las anteriores es correcta.

\vspace{0.2cm}

\textbf{Pregunta 2: En un régimen de Caja de Conversión (Currency Board), como el que tuvo Argentina en los 90:}\\
A) El Banco Central mantiene total libertad para emitir dinero sin respaldo para financiar el déficit público.\\
B) Existe un compromiso legislativo de cambiar la moneda nacional por una extranjera a tipo fijo, y la emisión monetaria debe estar respaldada por reservas de dicha divisa extranjera.\\
C) El tipo de cambio fluctúa libremente según la oferta y demanda del mercado sin intervención estatal.\\
D) Todas las anteriores son correctas.

\vspace{0.2cm}

\textbf{Pregunta 3: ¿En qué consiste la propuesta de reforma del Sistema Monetario Internacional conocida como "Zonas Objetivo" de Williamson?}\\
A) En establecer una moneda mundial única para eliminar los tipos de cambio.\\
B) En gravar con un impuesto las transacciones financieras para evitar la especulación (Tasa Tobin).\\
C) En definir un Tipo de Cambio de Equilibrio Fundamental (FEER) y establecer bandas de fluctuación amplias (ej. $\pm$10\%) alrededor del mismo.\\
D) Todas las anteriores son correctas.

\hrulefill

\subsection*{Soluciones:}

\begin{itemize}
    \item \textbf{P1:} La respuesta correcta es la \textbf{B}. Es vital recordar la secuencia: Plaza (bajar dólar) $\rightarrow$ Louvre (mantener dólar).
    \item \textbf{P2:} La respuesta correcta es la \textbf{B}. La clave es "compromiso legislativo" y "respaldo de reservas". Es lo opuesto a la libertad monetaria (opción A).
    \item \textbf{P3:} La respuesta correcta es la \textbf{C}. La opción B es la definición de la Tasa Tobin, no la de Williamson.
\end{itemize}

\vspace{0.5cm}
¡Hemos terminado el análisis de los textos! Con estos esquemas tienes el tema del Sistema Monetario Internacional \textbf{dominado}.

¿Quieres que repasemos algún concepto global, que hagamos un simulacro rápido de 5 preguntas mezcladas, o tienes algún otro material? ¡Estás listo para sacar nota!

\section*{Casos de Estudio: Resumen de \texttt{casosT6.pdf}}

¡Claro que sí! He analizado el documento completo \texttt{casosT6.pdf}. Este archivo es una recopilación de diapositivas que cubren varios "Casos de Estudio" fundamentales para la asignatura de Economía Internacional (probablemente del manual de Feenstra \& Taylor y Krugman).

A continuación, te explico cada uno de los casos estructurados por temas para que puedas estudiarlos fácilmente.

\subsection*{1. La Crisis de la Eurozona: Diseño vs. Realidad}

Este caso explica cómo la crisis financiera rompió las reglas originales del euro.

\begin{itemize}
    \item \textbf{El diseño original:} La Eurozona nació bajo el principio de "no rescate" (no-bailout). Se suponía que cada país era responsable exclusivo de sus propios problemas fiscales, sin reparto de cargas entre naciones.
    \item \textbf{El dilema:} Cuando estalló la crisis en la periferia (Grecia, España, etc.), la UE tuvo que elegir entre mantener las reglas (y arriesgar un colapso sistémico) o romperlas para rescatar a los países.
    \item \textbf{La solución (El Rescate):} Se eligió el rescate, no solo por solidaridad, sino para proteger a los bancos del "núcleo" (Alemania, Francia) que estaban expuestos a la deuda periférica y para evitar el pánico financiero global.
    \item \textbf{El precio político:} A cambio de los rescates, la soberanía económica se desplazó de las capitales nacionales a la "Troika" (Comisión Europea, BCE, FMI), imponiendo austeridad y reformas estructurales.
    \item \textbf{El papel de Mario Draghi:} En 2012, ante el riesgo de ruptura del euro, Draghi prometió hacer "lo que fuera necesario" (\textit{whatever it takes}). El BCE pasó de ser un árbitro técnico contra la inflación a un actor político que garantizaba la supervivencia del euro, aceptando activos de menor calidad para inyectar liquidez.
\end{itemize}

\subsection*{2. ¿Es la Eurozona un Área Monetaria Óptima (AMO)? (Caso 6.6)}

Este es uno de los casos más importantes. Compara la Eurozona con Estados Unidos usando la teoría de las Áreas Monetarias Óptimas. El diagnóstico es que \textbf{Europa NO es un área monetaria óptima}. Se basa en tres criterios:

\begin{enumerate}
    \item \textbf{Integración Comercial:}
    \begin{itemize}
        \item EE. UU.: El comercio interno es muy alto (cerca del 66\% del PIB).
        \item Eurozona: Aunque el euro impulsó el comercio, la integración es menor (alrededor del 17-18\% del PIB) y la convergencia de precios es incompleta (hay diferencias grandes en sectores como automóviles).
    \end{itemize}
    \item \textbf{Movilidad Laboral (El talón de Aquiles):}
    \begin{itemize}
        \item EE. UU.: Es muy alta; la gente se muda si no hay trabajo. Más del 40\% vive en un estado distinto al de nacimiento.
        \item Eurozona: Es bajísima debido a barreras lingüísticas y culturales. Solo el 14\% vive en otro país. Esto impide que el desempleo se ajuste moviendo trabajadores de países en crisis a países en auge.
    \end{itemize}
    \item \textbf{Federalismo Fiscal:}
    \begin{itemize}
        \item EE. UU.: Tienen un sistema de transferencias automáticas. Si un estado entra en crisis, recibe fondos federales que compensan gran parte de la caída.
        \item Eurozona: No existe un mecanismo fiscal centralizado potente para rescatar regiones automáticamente.
    \end{itemize}
\end{enumerate}

\subsection*{3. El Fin de Bretton Woods y la Gran Inflación (Caso 6.3)}

Explica por qué colapsó el sistema de tipos de cambio fijos del siglo XX.

\begin{itemize}
    \item \textbf{El problema:} Bretton Woods dependía de que EE. UU. mantuviera una inflación baja, ya que el dólar era el ancla.
    \item \textbf{La causa:} Para financiar la Guerra de Vietnam y programas sociales ("Great Society"), EE. UU. imprimió mucho dinero, aumentando su inflación.
    \item \textbf{Transmisión:} Como los demás países tenían el tipo de cambio fijo con el dólar, se vieron obligados a "importar" esa inflación comprando dólares para mantener la paridad.
    \item \textbf{El colapso (Nixon Shock):} En 1971, Nixon suspendió la convertibilidad del dólar en oro, rompiendo el sistema. En 1973, el mundo pasó a tipos de cambio flexibles.
\end{itemize}

\subsection*{4. La Recesión de "Doble Caída" y el Brexit (Caso 6.11)}

Describe la resaca de la crisis de 2008 en Europa.

\begin{itemize}
    \item \textbf{Doble Recesión:} Mientras otros se recuperaban, la Eurozona volvió a caer en recesión en 2011 debido a las políticas de austeridad prematuras.
    \item \textbf{Sin válvula de escape:} Los países del sur (como España o Grecia) tenían mucha deuda y, al estar en el euro, no podían devaluar su moneda para recuperar competitividad.
    \item \textbf{Consecuencia política:} El estancamiento económico y la imposibilidad de ajustar el tipo de cambio alimentaron el euroescepticismo, lo que culminó en el Brexit en 2016.
\end{itemize}

\subsection*{5. Economía Política del Tipo de Cambio: EE. UU. 1890s (Caso 6.2)}

Este caso histórico demuestra que elegir un sistema monetario es una decisión política, no solo técnica.

\begin{itemize}
    \item \textbf{El conflicto:} A finales del siglo XIX en EE. UU., había una lucha entre mantener el \textbf{Patrón Oro} (defendido por bancos y acreedores, generaba deflación y estabilidad) o pasar al \textbf{Patrón Plata} (defendido por agricultores y deudores, generaba inflación y aliviaba deudas).
    \item \textbf{Conclusión:} Ganó el oro, pero ilustra cómo el tipo de cambio afecta la distribución de la riqueza entre grupos sociales (acreedores vs. deudores).
\end{itemize}

\subsection*{6. El Costo de las Crisis Cambiarias (Titular 6.1)}

\begin{itemize}
    \item \textbf{Datos:} Una crisis cambiaria (depreciación brutal de la moneda) provoca una caída del PIB de entre el 3\% y el 6\% y la recuperación es muy lenta.
    \item \textbf{Mecanismo de daño:} Aumenta el costo de la deuda externa (porque ahora tu moneda vale menos), debilita los bancos y provoca fuga de capitales.
\end{itemize}

\subsection*{7. La Economía Política de las Crisis (Caso 6.1)}

¿Por qué caen los gobiernos tras una devaluación?

\begin{itemize}
    \item \textbf{Países Emergentes (ej. Argentina):} Los líderes caen por el costo económico directo (pobreza, inflación).
    \item \textbf{Países Avanzados (ej. Reino Unido):} Los líderes caen por \textbf{costo de reputación}. Incluso si la economía mejora (como en UK tras 1992), el gobierno pierde credibilidad por parecer incompetente.
\end{itemize}

\subsection*{8. El Trilema y las Crisis (Timeline)}

Se explica mediante el concepto del "Trilema" o Trinidad Imposible.

\begin{itemize}
    \item \textbf{Teoría:} Un país no puede tener al mismo tiempo: 1) Movilidad de capitales, 2) Tipo de cambio fijo, y 3) Política monetaria autónoma.
    \item \textbf{La Crisis:} Si intentas mantener el tipo fijo con capitales libres, pierdes el control de tu política monetaria. Si los inversores desconfían, atacan la moneda, agotan tus reservas y fuerzan una devaluación.
\end{itemize}

\subsection*{9. Impacto del Euro en el Comercio}

\begin{itemize}
    \item \textbf{Mito:} Al principio se pensaba que el euro triplicaría el comercio (+235\%).
    \item \textbf{Realidad:} El aumento real fue modesto, entre el 5\% y el 15\%.
    \item \textbf{Dato Clave:} Los países que \textit{no} adoptaron el euro (como Suecia o Reino Unido) también aumentaron su comercio con la Eurozona, lo que demuestra que la moneda única no es estrictamente necesaria para la integración comercial.
\end{itemize}

\vspace{0.5cm}
Espero que este resumen estructurado te sea de gran utilidad para tu examen. ¡Mucho ánimo con el estudio!
% \chapter{Fundamentos Avanzados de Iluminación Computacional y Renderizado}

\section{Introducción a la Radiometría y Percepción Visual}

El estudio de la síntesis de imágenes realistas requiere una comprensión profunda de la física de la luz y su interacción con la materia, así como de la psicofísica de la percepción humana.

\subsection{Naturaleza de la Radiación Electromagnética}
La luz visible comprende una franja estrecha del espectro electromagnético, específicamente longitudes de onda ($\lambda$) entre aproximadamente 390 nm y 750 nm. Aunque la física cuántica describe la luz mediante una dualidad onda-partícula, en la informática gráfica, y específicamente en la óptica geométrica, adoptamos principalmente el \textbf{modelo de partículas}.

Bajo este modelo, la luz se conceptualiza como un flujo de fotones que viajan en trayectorias rectilíneas. La magnitud fundamental para cuantificar este flujo es la \textbf{Radiancia} ($L$), definida como la densidad de flujo de energía radiante por unidad de tiempo, por unidad de área proyectada y por unidad de ángulo sólido. Matemáticamente, la radiancia en un punto $p$ en la dirección $\vec{v}$ se denota como $L(\lambda, p, \vec{v})$.

\subsection{El Sistema Visual Humano y la Teoría del Color}
La percepción del color es el resultado de la respuesta espectral de los fotorreceptores en la retina. El sistema visual humano reduce la distribución espectral de potencia continua de la luz entrante a tres valores discretos (tristímulos), correspondientes a la sensibilidad de los conos S, M y L.

Esta reducción dimensional permite modelar el color mediante un espacio vectorial tridimensional. En computación, utilizamos el modelo \textbf{RGB}, donde un color se representa mediante la mezcla aditiva de tres primarios: Rojo, Verde y Azul. Matemáticamente, la percepción de una distribución de radiancia $L$ se aproxima mediante una función lineal $f(L) \approx (r, g, b)$. Es crucial notar que el espacio RGB es dependiente del dispositivo; una misma tupla $(r, g, b)$ puede resultar en estímulos psicofísicos diferentes dependiendo de las características del hardware de visualización (monitor o proyector).

\section{La Ecuación de Renderizado}

El comportamiento global de la luz en una escena se describe formalmente mediante la \textbf{Ecuación de Renderizado} (Kajiya, 1986). Esta ecuación integral expresa la conservación de la energía radiante en equilibrio. La radiancia saliente $L_o$ desde un punto $p$ en la dirección $\vec{v}$ es la suma de la radiancia emitida por el propio punto (si es una fuente de luz) y la radiancia reflejada proveniente de todas las direcciones incidentes sobre el hemisferio $\Omega$:

\begin{equation}
L_o(p, \vec{v}) = L_{em}(p, \vec{v}) + \int_{\Omega} f_r(p, \vec{v}, \vec{\omega}_{in}) L_{in}(p, \vec{\omega}_{in}) (\vec{n} \cdot \vec{\omega}_{in}) d\vec{\omega}_{in}
\end{equation}

Donde:
\begin{itemize}
    \item $L_{em}$ es la radiancia emitida.
    \item $L_{in}$ es la radiancia incidente desde la dirección $\vec{\omega}_{in}$.
    \item $f_r$ es la Función de Distribución de Reflectancia Bidireccional (BRDF).
    \item $(\vec{n} \cdot \vec{\omega}_{in})$ es el factor de atenuación geométrica (ley del coseno de Lambert).
\end{itemize}

Esta ecuación integral de Fredholm de segunda especie no tiene solución analítica general, lo que obliga a utilizar métodos numéricos (como Monte Carlo) o modelos simplificados para su resolución.

\section{El Modelo de Iluminación Local de Phong}

Debido a la complejidad computacional de la ecuación de renderizado completa, históricamente se han adoptado modelos empíricos simplificados. El modelo de Phong es el estándar clásico en la rasterización por hardware. Este modelo descompone la interacción de la luz en tres componentes independientes: ambiental, difusa y especular.

\subsection{Suposiciones y Simplificaciones}
El modelo asume fuentes de luz puntuales, ignora las inter-reflexiones complejas (iluminación global) y trata los materiales como opacos. La radiancia total se calcula como:

\begin{equation}
L(p, \vec{v}) = \sum_{i=1}^{N} S_i [f_{am} + f_{dl} + f_{sp}]
\end{equation}

\subsection{Componente Ambiental}
Aproxima la iluminación indirecta global mediante un término constante, evitando que las zonas en sombra sean completamente negras:
\begin{equation}
f_{am} = k_a \cdot C(p)
\end{equation}
Donde $k_a$ es el coeficiente ambiental y $C(p)$ el color base del objeto.

\subsection{Componente Difusa (Lambertiana)}
Modela la reflexión en superficies mate ideales, donde la luz se dispersa uniformemente en todas direcciones. Depende exclusivamente de la posición de la luz respecto a la normal de la superficie $\vec{n}$, no de la posición del observador:
\begin{equation}
f_{dl} = k_d \cdot C(p) \cdot \max(0, \vec{n} \cdot \vec{l})
\end{equation}
Donde $\vec{l}$ es el vector unitario hacia la fuente de luz.

\subsection{Componente Especular (Phong y Blinn-Phong)}
Simula los reflejos brillantes característicos de materiales pulidos.
\begin{itemize}
    \item \textbf{Modelo original de Phong:} Se basa en el ángulo entre el vector de reflexión perfecta $\vec{r}$ y el vector hacia el observador $\vec{v}$.
    \begin{equation}
    f_{ph} = k_s \cdot (\vec{r} \cdot \vec{v})^e
    \end{equation}
    Donde $\vec{r} = 2(\vec{n} \cdot \vec{l})\vec{n} - \vec{l}$.
    
    \item \textbf{Modelo de Blinn-Phong:} Una optimización computacional y visualmente más robusta que utiliza el vector medio ("halfway vector") $\vec{h}$, definido como la bisectriz entre $\vec{l}$ y $\vec{v}$:
    \begin{equation}
    \vec{h} = \frac{\vec{l} + \vec{v}}{||\vec{l} + \vec{v}||}, \quad f_{bp} = k_s \cdot (\vec{n} \cdot \vec{h})^e
    \end{equation}
\end{itemize}
El exponente $e$ controla la "nitidez" del brillo especular (la suavidad de la superficie).

\section{Modelado Realista y la Función BRDF}

Para alcanzar el fotorrealismo, es necesario sustituir las aproximaciones empíricas por modelos basados en la física (Physically Based Rendering - PBR). La pieza central es la BRDF ($f_r$), que cuantifica la relación entre irradiancia incidente y radiancia reflejada.

\subsection{Leyes de Fresnel}
En la interfaz entre dos medios con diferentes índices de refracción, la luz se divide en un componente reflejado y otro refractado (transmitido). Las ecuaciones de Fresnel dictan esta proporción. Para materiales dieléctricos, la reflectancia especular aumenta drásticamente a medida que el ángulo de incidencia se aproxima a la rasante (90 grados).
En computación gráfica, a menudo se utiliza la aproximación de Schlick para el término de Fresnel ($F$):
\begin{equation}
F(\theta) \approx F_0 + (1 - F_0)(1 - \cos\theta)^5
\end{equation}
Donde $F_0$ es la reflectancia en incidencia normal.

\subsection{Teoría de Microfacetas}
Los modelos modernos (como Cook-Torrance o GGX) asumen que, a nivel microscópico, las superficies rugosas están compuestas por pequeños espejos perfectos (microfacetas) orientados aleatoriamente. La BRDF especular se modela como:

\begin{equation}
f_{micro}(\vec{l}, \vec{v}) = \frac{D(\vec{h}) G(\vec{l}, \vec{v}, \vec{h}) F(\vec{l}, \vec{h})}{4 (\vec{n} \cdot \vec{l}) (\vec{n} \cdot \vec{v})}
\end{equation}

Donde:
\begin{itemize}
    \item \textbf{D (Distribución de Normales):} Describe la probabilidad estadística de que una microfaceta esté orientada hacia el vector medio $\vec{h}$. La distribución GGX (Trowbridge-Reitz) es el estándar actual por su capacidad para modelar colas especulares largas.
    \item \textbf{G (Geometría/Enmascaramiento-Sombreado):} Modela la auto-oclusión de las microfacetas (shadowing) y el bloqueo de la luz reflejada (masking).
    \item \textbf{F (Fresnel):} La reflectancia física de las microfacetas individuales.
\end{itemize}

\subsection{Rugosidad y Anisotropía}
El parámetro de rugosidad ($\alpha$) controla la dispersión de la distribución $D$. Si $\alpha_x \neq \alpha_y$, el material es anisotrópico (como el metal cepillado), variando su apariencia al rotar la superficie alrededor de su normal.

\section{Modelos de Fuentes de Luz}

La iluminación de una escena depende críticamente de la tipología de las fuentes emisoras.

\subsection{Fuentes Puntuales y Direccionales}
\begin{itemize}
    \item \textbf{Puntual (Point Light):} Emite desde una posición $q$ en todas direcciones. La intensidad decae cuadráticamente con la distancia ($1/r^2$), aunque en motores gráficos se suelen usar funciones de atenuación modificadas para control artístico.
    \item \textbf{Direccional (Distant Light):} Simula una fuente en el infinito (como el sol). Los rayos son paralelos y no hay atenuación por distancia.
\end{itemize}

\subsection{Fuentes Avanzadas}
\begin{itemize}
    \item \textbf{Spot Light (Foco):} Una fuente puntual restringida a un cono. La intensidad se atenúa angularmente desde el eje central del cono hacia los bordes, a menudo modelado mediante funciones coseno elevadas a una potencia (similar a Phong).
    \item \textbf{Luces de Área:} Emiten luz desde una superficie geométrica (disco, rectángulo). Son esenciales para generar sombras suaves realistas, aunque su coste computacional es elevado.
    \item \textbf{Luces Fotométricas:} Utilizan perfiles de intensidad tabulados (como archivos IES) medidos de luminarias reales, permitiendo patrones de emisión complejos y físicamente exactos.
\end{itemize}

% \chapter{Texturas, Sombreado y Materiales}

\section{Introducción Teórica a las Texturas}

En el ámbito de la Informática Gráfica, la representación de superficies realistas requiere ir más allá de la geometría poligonal pura. Los objetos reales presentan variaciones de color, rugosidad y reflectividad a una escala microscópica que sería computacionalmente prohibitivo modelar mediante polígonos individuales. Para resolver esto, introducimos el concepto de \textbf{textura}.

Una textura se define formalmente como una función $T$ que mapea un dominio bidimensional $D$, usualmente normalizado en el intervalo $[0,1] \times [0,1]$, a un espacio de atributos del Modelo de Iluminación Local (MIL).
\begin{equation}
    T: D \subseteq \mathbb{R}^2 \rightarrow \mathcal{A}
\end{equation}
Donde $\mathcal{A}$ representa el conjunto de parámetros modificables, siendo el color difuso ($C(p)$) el más común, aunque también se aplica a coeficientes de especularidad, vectores normales y transparencia.

Existen dos modalidades fundamentales para representar esta función $T$:
\begin{itemize}
    \item \textbf{Texturas de Imagen (Image Textures):} La función se discretiza en una matriz de elementos denominados \textit{texels} (texture elements).
    \item \textbf{Texturas Procedurales:} La función $T(s)$ se define algorítmicamente mediante un subprograma, permitiendo resolución infinita y patrones matemáticos complejos sin consumo de memoria de almacenamiento de imagen.
\end{itemize}

\subsection{Coordenadas de Textura y Mapeo}

Para aplicar una función de textura sobre una superficie tridimensional arbitraria, es imperativo establecer una correspondencia biyectiva entre los puntos de la superficie $S \subset \mathbb{R}^3$ y el dominio de la textura $D$. Sea $p = (x,y,z)$ un punto en la superficie, debe existir una función de proyección $f$:
\begin{equation}
    (u, v) = f(x, y, z)
\end{equation}
Donde el par $(u, v)$ constituye las coordenadas de textura. Esta función suele descomponerse en componentes escalares $u = f_u(p)$ y $v = f_v(p)$.

\subsubsection{Estrategias de Asignación}
La implementación de la función $f$ se realiza mediante dos estrategias principales:
\begin{enumerate}
    \item \textbf{Asignación Explícita:} Las coordenadas $(u, v)$ se almacenan como atributos directos de los vértices en la malla poligonal. Durante la etapa de rasterización, estas coordenadas se interpolan linealmente a través de la superficie del polígono. Esta técnica es fundamental en el modelado CAD y requiere atención especial en la continuidad de las aristas (topología de la malla) para evitar artefactos visuales o discontinuidades en el mapeo.
    \item \textbf{Asignación Procedural:} Las coordenadas se calculan dinámicamente mediante una función matemática basada en la posición espacial del punto. Esto puede realizarse a nivel de vértice (interpolando posteriormente) o a nivel de fragmento (per-pixel) para mayor precisión en superficies no lineales.
\end{enumerate}

\subsubsection{Funciones de Proyección Procedural}
Se definen diversas topologías de proyección para envolver objetos geométricos:
\begin{itemize}
    \item \textbf{Proyección Planar (Lineal):} Se proyecta el punto $p$ sobre un plano definido por un punto de anclaje $q$ y dos vectores ortonormales base $e_u, e_v$.
    \begin{equation}
        u = (p - q) \cdot e_u, \quad v = (p - q) \cdot e_v
    \end{equation}
    \item \textbf{Coordenadas Esféricas:} Se utiliza una conversión a coordenadas polares, ideal para objetos con topología esférica. Dado un punto $(x,y,z)$, los ángulos $\alpha$ (azimut) y $\beta$ (elevación) determinan $(u, v)$:
    \begin{equation}
        u = \frac{1}{2} + \frac{\text{atan2}(z, x)}{2\pi}, \quad v = \frac{1}{2} + \frac{\text{atan2}(y, \sqrt{x^2+z^2})}{\pi}
    \end{equation}
    \item \textbf{Coordenadas Cilíndricas:} Se proyecta radialmente sobre un cilindro, utilizando el ángulo para $u$ y la altura $y$ normalizada para $v$.
\end{itemize}

\subsection{Filtrado y Consulta de Texels}
Dado que la proyección de un píxel de pantalla sobre el espacio de textura raramente coincide exactamente con un texel, se requieren algoritmos de muestreo:
\begin{itemize}
    \item \textbf{Vecino más cercano (Nearest Neighbor):} Selecciona el texel cuyo centroide está más próximo a $(u, v)$. Computacionalmente económico pero propenso a aliasing (pixelación).
    \item \textbf{Interpolación Bilineal:} Calcula el promedio ponderado de los cuatro texels adyacentes a $(u, v)$, suavizando las transiciones y mejorando la calidad visual en primeros planos.
\end{itemize}

\section{Teoría del Sombreado (Shading)}

El sombreado se refiere al proceso de interpolación de la iluminación sobre las superficies poligonales. En el pipeline gráfico, la evaluación del Modelo de Iluminación Local (MIL) puede ocurrir en diferentes etapas, determinando el costo computacional y la calidad visual.

\subsection{Sombreado Plano (Flat Shading)}
El MIL se evalúa una única vez por polígono, generalmente en su centroide o primer vértice. Se utiliza una única normal $n_p$ para toda la cara.
\begin{itemize}
    \item \textbf{Ventaja:} Alta eficiencia computacional.
    \item \textbf{Desventaja:} Facetado visible y discontinuidades de iluminación en las aristas.
    \item \textbf{Fenómeno Psico-visual:} Exacerba el efecto de las \textit{Bandas de Mach}, una ilusión óptica donde el contraste lateral de la retina exagera los límites entre zonas de intensidad constante.
\end{itemize}

\subsection{Sombreado de Gouraud (Vertex Shading)}
El MIL se evalúa en cada vértice de la malla, utilizando normales promediadas de las caras adyacentes para simular curvatura. Los colores resultantes $C_{vert}$ se interpolan linealmente en el interior del polígono durante la rasterización.
\begin{itemize}
    \item \textbf{Limitación Crítica:} La interpolación lineal de colores pierde componentes de alta frecuencia, como los brillos especulares (highlights), si estos caen en el interior de un polígono grande y no coinciden con un vértice.
\end{itemize}

\subsection{Sombreado de Phong (Pixel Shading)}
El MIL se evalúa para cada fragmento (píxel) generado. En lugar de interpolar colores, se interpolan los vectores normales desde los vértices.
\begin{itemize}
    \item \textbf{Ventaja:} Captura brillos especulares precisos y produce gradientes suaves, eliminando casi totalmente las bandas de Mach geométricas.
    \item \textbf{Costo:} Requiere evaluar la ecuación de iluminación millones de veces por frame, aunque es el estándar actual en hardware gráfico moderno.
\end{itemize}

\section{Implementación en Motores Gráficos: Godot}

El motor Godot implementa estos conceptos teóricos mediante una arquitectura de nodos y un modelo de materiales basado en física (PBR - Physically Based Rendering).

\subsection{Fuentes de Iluminación}
Las luces se modelan como nodos en el árbol de escena (\texttt{Light3D}), interactuando con las superficies mediante sus normales y propiedades materiales.
\begin{itemize}
    \item \textbf{DirectionalLight3D:} Simula fuentes en el infinito (como el sol). Los rayos son paralelos y la intensidad no se atenúa con la distancia. Su orientación se define por un vector local, típicamente alineado con el eje Z negativo.
    \item \textbf{OmniLight3D:} Fuente puntual isotrópica. La atenuación sigue la ley del inverso del cuadrado de la distancia:
    \begin{equation}
        I(r) \propto \frac{1}{r^e}
    \end{equation}
    Donde $e=2$ corresponde a la realidad física, aunque se permite su modificación artística.
    \item \textbf{SpotLight3D:} Fuente cónica restringida por un ángulo de apertura $\theta_{max}$. La intensidad decae tanto por distancia como por desviación angular respecto al eje principal del foco.
\end{itemize}

Adicionalmente, se soporta Iluminación Basada en Imágenes (IBL) mediante \texttt{WorldEnvironment} y \texttt{PanoramaSkyMaterial}, permitiendo que texturas esféricas de alto rango dinámico iluminen la escena de manera global y difusa.

\subsection{El Modelo de Material Estándar}
La clase \texttt{StandardMaterial3D} en Godot encapsula los parámetros del modelo PBR. La ecuación fundamental que determina el color final $I$ de un fragmento es una combinación lineal ponderada por la "metalicidad" del material:

\begin{equation}
    I = e + a + (1 - m)(b \cdot d + s \cdot p_{\alpha}) + m \cdot b \cdot p_{\alpha}
\end{equation}

Donde:
\begin{itemize}
    \item $e$: Emisión (luz propia del material).
    \item $a$: Luz ambiental reflejada (proveniente del mapa de entorno).
    \item $m$: Factor \textit{metallic} (0.0 a 1.0). Distingue entre dieléctricos y conductores.
    \item $b$: Color base (\textit{Albedo}), modulado por texturas.
    \item $d$: Reflectividad difusa (Lambertiana o Burley).
    \item $s$: Factor especular para no metales (\textit{metallic\_specular}).
    \item $p_{\alpha}$: Lóbulo especular (BRDF), dependiente de la rugosidad $\alpha$ (\textit{roughness}).
\end{itemize}

\subsubsection{Análisis de los extremos del modelo}
\begin{itemize}
    \item \textbf{Dieléctricos ($m=0$):} El término predominante es $(b \cdot d + s \cdot p_{\alpha})$. El color base afecta a la componente difusa, pero el brillo especular es blanco (o del color de la luz), gobernado por el índice de fresnel implícito en $s$.
    \item \textbf{Metales ($m=1$):} La ecuación se simplifica a $b \cdot p_{\alpha}$. No existe componente difusa. Toda la luz reflejada es especular y está tintada por el color base del material (el albedo define el color del reflejo metálico).
\end{itemize}

\subsection{Transparencia y Renderizado}
Godot gestiona la transparencia mediante la propiedad \texttt{transparency} del material o el canal Alfa del color base. A diferencia de la óptica física real, en el modo de renderizado estándar (rasterización), los objetos transparentes no suelen generar refracción compleja ni proyectar sombras parciales, a menos que se utilicen técnicas avanzadas de ray-tracing o shaders específicos de refracción en espacio de pantalla.


% \chapter{Fundamentos de la Interacción en Sistemas Gráficos}

\section{Introducción a los Sistemas Gráficos Interactivos}

Un Sistema Gráfico Interactivo (SGI) se define como una arquitectura de software diseñada para mantener un ciclo continuo de retroalimentación con el usuario. A diferencia de los sistemas de procesamiento por lotes, un SGI debe responder a las acciones del operador —típicamente en un intervalo de tiempo imperceptible, del orden de décimas de segundo— y presentar dicha respuesta mediante una visualización gráfica bidimensional o tridimensional.

La arquitectura subyacente de estos sistemas opera mediante un bucle infinito que gestiona una estructura de datos residente en memoria (el modelo). En cada iteración de este ciclo, el sistema espera o detecta una acción externa, captura los datos característicos de dicha acción, modifica el estado interno del modelo en consecuencia y, finalmente, renderiza una nueva imagen que refleja el cambio de estado.

\subsection{Interactividad frente a Tiempo Real}
Es crucial establecer una distinción taxonómica entre sistemas interactivos y sistemas de tiempo real, aunque a menudo convergen:
\begin{itemize}
    \item \textbf{Sistemas Interactivos:} El requisito primordial es una latencia lo suficientemente baja para mantener la percepción de causalidad entre la acción del usuario y la respuesta del sistema.
    \item \textbf{Sistemas de Tiempo Real:} Se rigen por restricciones deterministas donde la latencia debe ser estrictamente menor o igual a un umbral predefinido. Superar este límite constituye un fallo del sistema. Ejemplos críticos incluyen simuladores de vuelo y aplicaciones de realidad virtual (VR) donde la latencia induce cinetosis.
\end{itemize}

\subsection{Taxonomía de Dispositivos y Modos de Entrada}
La interacción se facilita mediante dispositivos físicos (hardware como teclados, ratones, digitalizadores) y dispositivos lógicos (abstracciones software como un puntero en pantalla o un reconocedor de gestos). La gestión de estos dispositivos se clasifica en tres modos operativos fundamentales:
\begin{enumerate}
    \item \textbf{Modo de Muestreo (Sampling/Polling):} La aplicación interroga activamente el estado del dispositivo en instantes arbitrarios. Es eficiente en memoria pero puede omitir cambios de estado breves si la frecuencia de muestreo es insuficiente.
    \item \textbf{Modo de Petición (Request):} El sistema detiene su ejecución esperando explícitamente a que ocurra un evento específico. Garantiza la captura del evento pero bloquea el flujo de ejecución.
    \item \textbf{Modo de Cola de Eventos (Event Queue):} El sistema operativo o el controlador del dispositivo acumula los cambios de estado en una cola FIFO (First In, First Out). La aplicación procesa esta cola de manera asíncrona, garantizando que no se pierdan eventos sin necesidad de bloqueo ni muestreo constante.
\end{enumerate}

\section{Arquitectura de Eventos en Godot Engine}

El motor Godot implementa un sistema de gestión de entrada basado primordialmente en eventos, encapsulados en la clase base \texttt{InputEvent}. Esta arquitectura permite desacoplar la lógica del juego de los detalles del hardware.

\subsection{Jerarquía y Tipología de Eventos}
Los eventos son objetos que transportan información sobre cambios de estado. La jerarquía de clases incluye:
\begin{itemize}
    \item \texttt{InputEventFromWindow}: Eventos originados en una ventana o viewport.
    \item \texttt{InputEventWithModifiers}: Subclase para entradas que pueden alterarse mediante teclas modificadoras (Ctrl, Alt, Shift), abarcando \texttt{InputEventKey} (teclado), \texttt{InputEventMouse} (ratón) e \texttt{InputEventGesture} (pantallas táctiles).
    \item \texttt{InputEventAction}: Eventos abstractos definidos semánticamente en el mapa de entradas.
\end{itemize}

\subsection{Flujo de Propagación de Eventos}
El procesamiento de eventos en el árbol de escena sigue un orden de prioridad estricto. Cuando se instancia un evento, el motor invoca secuencialmente los siguientes métodos virtuales en los nodos activos:
\begin{enumerate}
    \item \texttt{\_input()}: Primer método en ejecutarse. Procesa eventos generales.
    \item \texttt{\_shortcut\_input()}: Específico para atajos de teclado y eventos de control.
    \item \texttt{\_unhandled\_key\_input()}: Captura eventos de teclado no consumidos previamente.
    \item \texttt{\_unhandled\_input()}: El último recurso para eventos no procesados, ideal para la lógica del mundo del juego (p.ej., movimiento de cámara) que no debe interferir con la interfaz de usuario (GUI).
\end{enumerate}
Un concepto fundamental es el consumo del evento. Un nodo puede marcar un evento como "manejado" mediante \texttt{set\_input\_as\_handled()}, deteniendo su propagación hacia otros nodos en la jerarquía.

\subsection{Abstracción mediante el Mapa de Entradas (Input Map)}
El \texttt{InputMap} actúa como una capa de indirección que asocia entradas físicas (teclas específicas, botones de joystick) con acciones semánticas (p.ej., "saltar", "mover\_adelante"). Esto permite a los desarrolladores programar lógica basada en acciones (\texttt{InputEventAction}) en lugar de hardware específico, facilitando la reconfiguración de controles por parte del usuario final.

\section{Interfaz de Usuario (GUI) y el Patrón Observador}

La interfaz gráfica de usuario en motores gráficos se distingue de la proyección de la escena 3D/2D. En Godot, se construye mediante nodos derivados de la clase \texttt{Control}, los cuales poseen propiedades de posicionamiento, dimensionamiento y estilo (temas).

\subsection{Elementos de Control y Contenedores}
La construcción de interfaces complejas se basa en la composición jerárquica de controles.
\begin{itemize}
    \item \textbf{Controles Básicos:} Incluyen \texttt{Label} (texto), \texttt{Button} (interacción binaria), \texttt{TextureRect} (imágenes) y rangos numéricos como \texttt{HSlider} o \texttt{SpinBox}.
    \item \textbf{Contenedores:} Clases derivadas de \texttt{Container} (como \texttt{VBoxContainer}, \texttt{GridContainer}, \texttt{MarginContainer}) que administran automáticamente la disposición espacial de sus hijos, adaptándose a resoluciones dinámicas.
\end{itemize}

\subsection{Sistema de Señales}
Godot implementa el patrón de diseño Observador a través del sistema de \textit{Signals}. Una señal es un mecanismo de comunicación desacoplado donde un objeto emisor notifica un cambio de estado sin conocer a sus receptores.
\begin{itemize}
    \item \textbf{Definición y Emisión:} Los nodos pueden definir señales personalizadas (keyword \texttt{signal}) y emitirlas ante eventos específicos.
    \item \textbf{Conexión:} Los objetos interesados se suscriben (conectan) a dichas señales, vinculándolas a funciones de respuesta (\textit{callbacks}). Esto elimina el acoplamiento fuerte entre componentes lógicos y de interfaz, permitiendo, por ejemplo, que una barra de vida se actualice cuando el jugador recibe daño sin que el jugador tenga referencia directa a la barra de vida.
\end{itemize}

\section{Selección y Picking en Entornos Tridimensionales}

La selección (\textit{picking}) es el proceso de traducir una interacción 2D (clic del ratón en coordenadas de pantalla) en la identificación de un objeto 3D dentro de la escena.

\subsection{Técnicas de Implementación}
Existen dos paradigmas principales para resolver este problema:
\begin{enumerate}
    \item \textbf{Buffer de Selección (Color Picking):} Renderización de la escena en un framebuffer oculto donde cada objeto se dibuja con un color único que codifica su ID. Al hacer clic, se lee el color del píxel correspondiente. Es preciso a nivel de píxel pero requiere una pasada de renderizado adicional.
    \item \textbf{Lanzamiento de Rayos (Ray Casting):} Cálculo analítico de la intersección entre un rayo proyectado desde la cámara y la geometría de la escena. Es el método estándar en motores modernos debido a su flexibilidad.
\end{enumerate}

\subsection{Implementación en Godot mediante Ray Casting}
El proceso de selección en Godot utiliza el sistema de física para realizar consultas espaciales:
\begin{itemize}
    \item \textbf{Generación del Rayo:} Se utilizan las funciones de la cámara (\texttt{project\_ray\_origin} y \texttt{project\_ray\_normal}) para desproyectar la posición 2D del ratón hacia un rayo en el espacio 3D.
    \item \textbf{Colisionadores:} La intersección no se calcula contra la malla visual compleja, sino contra versiones simplificadas invisibles denominadas \textit{colliders} (\texttt{StaticBody3D} con \texttt{CollisionShape3D}). Esto optimiza el coste computacional.
    \item \textbf{Consulta al Espacio de Estado:} Se emplea el objeto \texttt{PhysicsRayQueryParameters3D} para configurar la consulta y el método \texttt{intersect\_ray()} del \texttt{PhysicsDirectSpaceState} para obtener el primer objeto interceptado.
\end{itemize}

\subsection{Fundamento Matemático: Intersección Rayo-Triángulo}
En el nivel más bajo, la selección por rayos implica resolver si una semirrecta $R(t) = O + tD$ intersecta un triángulo definido por los vértices $V_0, V_1, V_2$. Esto requiere verificar dos condiciones:
\begin{enumerate}
    \item El rayo debe intersectar el plano que contiene al triángulo.
    \item El punto de intersección debe encontrarse dentro de los límites del triángulo, lo cual se determina habitualmente mediante el cálculo de coordenadas baricéntricas $(u, v, w)$ tal que $u \geq 0, v \geq 0, u+v \leq 1$.
\end{enumerate}
% \input{../src/tex/t10}
% \input{../src/tex/t11}

% --- INICIO DEL ÍNDICE ---
\textit{Nota:} Se adjunta un índice para buscar más fácil el contenido.
\section*{\centering \textsc{I. Matemáticas y Demostraciones Vectoriales}}
\vspace{0.2cm}
\textit{Fundamentos teóricos sobre operaciones con vectores, matrices y transformaciones.}
\vspace{0.3cm}
\hrule
\vspace{0.5cm}

\begin{itemize}
    \setlength{\itemsep}{12pt} % Espacio extra entre ejercicios

    \item \textbf{1.2.1} \textbf{Producto Escalar (Dot Product)} \\
    Demostración del cálculo mediante suma de componentes en base ortonormal.

    \item \textbf{1.2.2} \textbf{Producto Vectorial (Cross Product)} \\
    Demostración del cálculo utilizando coordenadas cartesianas.

    \item \textbf{1.2.3} \textbf{Ortogonalidad} \\
    Demostración de que el producto vectorial es perpendicular a los vectores originales.

    \item \textbf{1.2.4} \textbf{Invariancia Rotacional 2D} \\
    Prueba de que el producto escalar se mantiene constante tras aplicar una rotación.

    \item \textbf{1.2.5} \textbf{Isometría (Conservación de Norma)} \\
    Demostración de que la rotación no altera la longitud del vector.

    \item \textbf{1.2.6} \textbf{Rotación de 90 Grados} \\
    Demostración de perpendicularidad: $\vec{v} \cdot R(\vec{v}) = 0$.

    \item \textbf{1.2.7} \textbf{Matrices Ortonormales} \\
    Análisis de la matriz de rotación 2D: filas y columnas unitarias y ortogonales.

    \item \textbf{1.2.8} \textbf{No Conmutatividad (Escalado)} \\
    Prueba de que $R \cdot S \neq S \cdot R$ si el escalado no es uniforme.

    \item \textbf{1.2.9} \textbf{No Conmutatividad (Traslación)} \\
    Prueba de que el orden importa entre rotación y traslación.

    \item \textbf{1.2.10} \textbf{Invariancia en 3D} \\
    El producto escalar es invariante bajo rotaciones en ejes cartesianos.

    \item \textbf{1.2.11} \textbf{Rotación del Producto Cruz} \\
    Demostración de la propiedad distributiva de la rotación sobre el producto vectorial.
\end{itemize}

\vspace{1cm}

\section*{\centering \textsc{II. Implementación en GDScript (Godot)}}
\vspace{0.2cm}
\textit{Scripts para generación de geometría, jerarquías de escena y lógica de control.}
\vspace{0.3cm}
\hrule
\vspace{0.5cm}

\subsection*{\textit{A. Geometría Procedural y Mallas}}
\begin{itemize}
    \setlength{\itemsep}{10pt}
    
    \item \textbf{1.1.1} \textbf{Polígono Regular Relleno} \\
    Creación de una malla de $N$ lados mediante \texttt{MeshInstance2D}.
    
    \item \textbf{1.1.2} \textbf{Gradientes de Color} \\
    Uso de \textit{Vertex Colors} para interpolación de colores en la malla.
    
    \item \textbf{1.1.4} \textbf{Visualización de Aristas (Wireframe)} \\
    Diferencias de implementación entre mallas indexadas y no indexadas.
    
    \item \textbf{1.1.5} \textbf{Debug de Normales} \\
    Script global (autoload) para generar líneas que visualicen las normales de una malla.
    
    \item \textbf{1.2.12} \textbf{Función Gancho} \\
    Generación de una polilínea simple mediante código.
    
    \item \textbf{1.4.1} \textbf{Figuras Compuestas} \\
    Script para generar un cuadrado azul con un triángulo inscrito y bordes diferenciados.
    
    \item \textbf{1.4.3} \textbf{Triangulación Manual (Tronco)} \\
    Generación de un polígono cóncavo mediante descomposición en triángulos.
    
    \item \textbf{1.4.5} \textbf{Modelado por Código (Logo Android)} \\
    Construcción 3D usando primitivas cilíndricas y semiesféricas.
\end{itemize}

\subsection*{\textit{B. Escena, Jerarquías y Animación}}
\begin{itemize}
    \setlength{\itemsep}{10pt}

    \item \textbf{1.2.13} \textbf{Instanciación y Pivotes} \\
    Rotaciones complejas alrededor de un punto de pivote desplazado.
    
    \item \textbf{1.4.2} \textbf{Transformaciones Jerárquicas} \\
    Uso de nodos padre/hijo con escalado negativo (efecto espejo).
    
    \item \textbf{1.4.4} \textbf{Árbol Fractal Recursivo} \\
    Script recursivo para generar ramas transformadas geométricamente.
    
    \item \textbf{1.8.1} \textbf{Gestión de Input} \\
    Lógica para detectar la duración exacta de la pulsación de una tecla.
    
    \item \textbf{1.10.1} \textbf{Curvas de Hermite} \\
    Interpolación suave de movimiento pasando por puntos de control con tangentes.
    
    \item \textbf{1.10.2} \textbf{Oscilación Controlada} \\
    Movimiento periódico con velocidad constante y rebote exacto en extremos.
    
    \item \textbf{1.10.3} \textbf{Reloj Analógico} \\
    Rotación de agujas sincronizada con el tiempo del sistema (\texttt{Time}).
    
    \item \textbf{1.10.4} \textbf{Simulación de Péndulo} \\
    Animación basada en funciones armónicas ($sin/cos$) para oscilación física.
    
    \item \textbf{1.10.5} \textbf{Tiro Parabólico} \\
    Animación física basada en la ecuación $p = p_0 + v_0t + 0.5at^2$.
\end{itemize}

\vspace{1cm}

\section*{\centering \textsc{III. Algoritmos y Pseudocódigo (Ray Tracing)}}
\vspace{0.2cm}
\textit{Diseño lógico para cálculo de intersecciones y selección (Picking).}
\vspace{0.3cm}
\hrule
\vspace{0.5cm}

\begin{itemize}
    \setlength{\itemsep}{12pt}

    \item \textbf{1.8.2} \textbf{Intersección Rayo-Triángulo} \\
    Algoritmo completo: Intersección con plano + Coordenadas Baricéntricas.
    
    \item \textbf{1.8.3} \textbf{Picking (Unproject)} \\
    Cálculo del rayo 3D en coordenadas de mundo a partir de un click en pantalla 2D.
    
    \item \textbf{1.9.1} \textbf{Intersección Rayo-Disco} \\
    Lógica de intersección plano-rayo y verificación de distancia al centro (radio).
    
    \item \textbf{1.9.2} \textbf{Intersección Rayo-Esfera} \\
    Resolución mediante ecuación cuadrática para esferas unitarias y genéricas.
    
    \item \textbf{1.9.3} \textbf{Intersección Rayo-Cilindro/Cono} \\
    Algoritmos para cuádricas infinitas con \textit{clipping} por altura finita.
    
    \item \textbf{1.3.5} \textbf{Extracción de Aristas} \\
    Algoritmo para generar una tabla de aristas únicas desde una lista de triángulos.
    
    \item \textbf{1.3.6} \textbf{Cálculo de Área} \\
    Algoritmo para sumar las áreas de los triángulos de una malla (producto cruz).
\end{itemize}

\vspace{1cm}

\section*{\centering \textsc{IV. Teoría de Mallas y Texturas}}
\vspace{0.2cm}
\textit{Eficiencia espacial, topología y mapeo de coordenadas UV.}
\vspace{0.3cm}
\hrule
\vspace{0.5cm}

\begin{itemize}
    \setlength{\itemsep}{12pt}

    \item \textbf{1.3.1} \textbf{Eficiencia de Memoria} \\
    Comparativa: Enumeración Espacial (Vóxeles $O(k^3)$) vs Malla Indexada ($O(k^2)$).
    
    \item \textbf{1.3.2} \textbf{Rejilla Rectangular} \\
    Cálculo de memoria requerida para una topología de rejilla $M \times N$.
    
    \item \textbf{1.3.3} \textbf{Triangle Strips} \\
    Análisis coste-beneficio: Ahorro de memoria vs coste de Vertex Shader.
    
    \item \textbf{1.3.4} \textbf{Topología (Euler-Poincaré)} \\
    Demostración de relaciones en mallas cerradas: $N_A = 3(N_V-2)$.
    
    \item \textbf{1.7.1} \textbf{Mapeo UV (Dado)} \\
    Diseño de tabla de vértices mínima (14 vértices) para textura continua.
    
    \item \textbf{1.7.2} \textbf{Normales y Costuras (Hard Edges)} \\
    Justificación de duplicado de vértices (24) para iluminación en cubo.
    
    \item \textbf{1.7.3} \textbf{Textura Repetida (Tiling)} \\
    Tabla de coordenadas UV para repetir una imagen en todas las caras.
\end{itemize}

\vspace{1cm}

\section*{\centering \textsc{V. Cámara, Proyección e Iluminación}}
\vspace{0.2cm}
\textit{Matemáticas de la cámara virtual y modelos de reflexión de luz.}
\vspace{0.3cm}
\hrule
\vspace{0.5cm}

\subsection*{\textit{A. Configuración de Cámara}}
\begin{itemize}
    \setlength{\itemsep}{10pt}

    \item \textbf{1.5.1} \textbf{Cámara de Seguimiento} \\
    Script para posicionar la cámara detrás y arriba de un objetivo móvil.
    
    \item \textbf{1.5.2} \textbf{LookAt (Ejes Alineados)} \\
    Cálculo de vectores $a, u, n$ para una configuración ortogonal específica.
    
    \item \textbf{1.5.3} \textbf{LookAt (Con Rotación)} \\
    Cálculo de vectores de cámara incluyendo rotación sobre el eje de vista (\textit{Roll}).
    
    \item \textbf{1.5.4} \textbf{Base de la Cámara} \\
    Código para derivar la base ortonormal ($u, v, n$) desde parámetros de vista.
    
    \item \textbf{1.5.5} \textbf{Matriz de Vista} \\
    Construcción manual de la \texttt{Transform3D} (inversa de la cámara).
    
    \item \textbf{1.5.6} \textbf{Control de Aspect Ratio} \\
    Script para mantener el FOV fijo (75º) independientemente del tamaño de ventana.
\end{itemize}

\subsection*{\textit{B. Proyección y Frustum}}
\begin{itemize}
    \setlength{\itemsep}{10pt}

    \item \textbf{1.5.7} \textbf{Frustum Ajustado (Cubo)} \\
    Cálculo de planos ($n, f, l, r, t, b$) para encuadrar perfectamente un cubo.
    
    \item \textbf{1.5.8} \textbf{Frustum Ajustado (Esfera)} \\
    Ajuste de planos de proyección para encuadrar una esfera tangente.
    
    \item \textbf{1.5.9} \textbf{Frustum No Cuadrado} \\
    Adaptación de la proyección para relaciones de aspecto \textit{Landscape} y \textit{Portrait}.
    
    \item \textbf{1.5.10} \textbf{Posicionamiento por FOV} \\
    Cálculo de la distancia de la cámara dado un ángulo de apertura $\beta$.
\end{itemize}

\subsection*{\textit{C. Modelos de Iluminación}}
\begin{itemize}
    \setlength{\itemsep}{10pt}

    \item \textbf{1.6.1} \textbf{Especularidad} \\
    Implementación de las fórmulas de Phong y Blinn-Phong en GDScript.
    
    \item \textbf{1.6.2} \textbf{Puntos de Brillo Máximo} \\
    Cálculo teórico de la posición del brillo en una esfera (Lambert/Phong).
    
    \item \textbf{1.6.3} \textbf{BRDF GGX (Microfacetas)} \\
    Implementación completa: Fresnel Schlick + Geometría + Distribución Normal.
\end{itemize}

% --- FIN DEL ÍNDICE ---
\chapter{Ejercicios Teóricos}

\textit{Observación.} Estos ejercicios usan una numeración distinta a la de las diapositivas, aunque están en el mismo orden. Hay veces que cuandos se hace referencia a un ejercicio se usa la enumeración de las diapositivas para encontrarlo más fácilmente. De la misma manera se recomienda revisar el orden de defición de vértices y demás para triángulos por si es el correcto.

\section{Sesión 2}

Para la resolución de los siguientes ejercicios se ha usado varios scripts como autoload:
\lstinputlisting{../code/EjerciciosTeoria/raiz_problemas_2D.gd}
\lstinputlisting{../code/EjerciciosTeoria/Global.gd}
\lstinputlisting{../code/EjerciciosTeoria/funciones_auxiliares_t5.gd}

Puede ser que se usasen de otros ficheros, pero estos son los principales. De todas formas, si no se incluye la implementación de algún  método se sobreentiende.

\begin{ejercicio}
    \textbf{Polígono regular relleno de color plano}

    Implementa un nodo de tipo \texttt{MeshInstance2D} con una malla (no indexada) para un polígono regular de $n$ lados relleno de color naranja plano (RGB(1.0, 0.7, 0.0)), con radio $r$ y centro en el origen.

    El polígono estará formado por $n$ triángulos, cada uno con un vértice en el centro y los otros dos en el contorno.

    Los valores de $n$ y $r$ se declaran como dos constantes de GDScript (\texttt{const}), como se indica a continuación:
    \begin{verbatim}
    const n: int = 8
    const r: float = 0.8
    \end{verbatim}
    Los valores de estas constantes se podrán cambiar sin tocar nada del resto del script.

    \begin{center}
    \begin{tikzpicture}
        % Polígono regular de 8 lados (n=8, r=1.5 visual)
        \coordinate (O) at (0,0);
        \foreach \i in {1,...,8} {
            \coordinate (P\i) at ({1.5*cos((\i-1)*360/8)}, {1.5*sin((\i-1)*360/8)});
        }
        % Relleno naranja
        \fill[orange] (O) -- (P1) -- (P2) -- cycle;
        \fill[orange] (O) -- (P2) -- (P3) -- cycle;
        \fill[orange] (O) -- (P3) -- (P4) -- cycle;
        \fill[orange] (O) -- (P4) -- (P5) -- cycle;
        \fill[orange] (O) -- (P5) -- (P6) -- cycle;
        \fill[orange] (O) -- (P6) -- (P7) -- cycle;
        \fill[orange] (O) -- (P7) -- (P8) -- cycle;
        \fill[orange] (O) -- (P8) -- (P1) -- cycle;
        
        % Aristas blancas para distinguir los triángulos
        \draw[white, thick] (O) -- (P1);
        \draw[white, thick] (O) -- (P2);
        \draw[white, thick] (O) -- (P3);
        \draw[white, thick] (O) -- (P4);
        \draw[white, thick] (O) -- (P5);
        \draw[white, thick] (O) -- (P6);
        \draw[white, thick] (O) -- (P7);
        \draw[white, thick] (O) -- (P8);
        \draw[white, thick] (P1) -- (P2) -- (P3) -- (P4) -- (P5) -- (P6) -- (P7) -- (P8) -- cycle;
    \end{tikzpicture}
    \end{center}
\end{ejercicio}

\begin{solucion} Solución al problema 2.1:
    \lstinputlisting{../code/EjerciciosTeoria/problema_2_1.gd}
\end{solucion}

\begin{ejercicio}
    \textbf{Polígono regular relleno con gradaciones}

    Crea otro \texttt{Node2D}, y asígnale un script para visualizar el mismo polígono regular que antes (también con una malla no indexada), solo que ahora debes asignar colores a los vértices para que los triángulos aparezcan con una graduación en tonos de gris como en la figura.

    Cada triángulo que forma el polígono regular será blanco en el vértice del centro, gris claro en otro vértice del borde y gris oscuro en el tercero.

    Responde razonadamente a esta cuestión: ¿cuántos vértices debe tener la tabla de vértices?

    \begin{center}
    \begin{tikzpicture}
        \coordinate (O) at (0,0);
        % Definición de colores aproximados
        \definecolor{grisclaro}{gray}{0.8}
        \definecolor{grisoscuro}{gray}{0.4}
        
        % Dibujamos cada triángulo interpolando colores (simulado con shade)
        % Nota: TikZ básico no interpola 3 colores por vértice fácilmente en un mesh simple sin librerías extra,
        % aquí usamos un shading radial/axial aproximado para ilustrar el enunciado.
        \foreach \i in {1,...,8} {
            \pgfmathsetmacro{\angleA}{(\i-1)*360/8}
            \pgfmathsetmacro{\angleB}{\i*360/8}
            \shadedraw[shading=axis, bottom color=grisoscuro, top color=grisclaro, shading angle=\angleA+45] 
                (O) -- ({1.5*cos(\angleA)}, {1.5*sin(\angleA)}) -- ({1.5*cos(\angleB)}, {1.5*sin(\angleB)}) -- cycle;
            % Simulamos el centro blanco con un círculo difuso o superposición
            \fill[white, opacity=0.3] (0,0) circle (0.5); 
        }
        % Etiquetas de color para clarificar el requisito
        \node[font=\tiny, fill=white, inner sep=1pt] at (0,0) {Blanco};
        \node[font=\tiny, fill=white, inner sep=1pt] at ({1.6*cos(0)}, {1.6*sin(0)}) {Gris Claro};
        \node[font=\tiny, fill=white, inner sep=1pt] at ({1.6*cos(360/8)}, {1.6*sin(360/8)}) {Gris Oscuro};
    \end{tikzpicture}
    \end{center}
\end{ejercicio}

\begin{solucion} Solución al problema 2.2:
    \lstinputlisting{../code/EjerciciosTeoria/problema_2_2.gd}
    Para ver cuantos vértices tiene la tabla de vértices, hay que tener en cuenta que cada triángulo tiene 3 vértices, y como hay $n$ triángulos, la tabla de vértices debe tener $3n$ vértices. Por lo tanto, la respuesta es $3n$. Siendon $n=8$, la tabla de vértices tiene $24$ vértices.
\end{solucion}

\begin{ejercicio}
    % \textbf{Polígono regular hecho de líneas}

    Repite los dos problemas anteriores (2.1 y 2.2), con los mismos requerimientos, pero ahora usando \textbf{mallas indexadas}, de forma que el número de vértices e índices sea mínimo.

    Responde razonadamente a estas cuestiones:
    \begin{itemize}
        \item ¿Cuántos vértices debe tener ahora la tabla de vértices en cada caso?
        \item ¿Y cuántos índices debe haber?
    \end{itemize}
\end{ejercicio}

\begin{solucion} Solución al problema 2.3:
    \lstinputlisting{../code/EjerciciosTeoria/problema_2_3.gd}
\end{solucion}

\begin{ejercicio}
    \textbf{Aristas del polígono regular}

    Crea un nuevo nodo \texttt{MeshInstance2D} de forma que ahora veamos simplemente las aristas del contorno del polígono regular descrito en los anteriores problemas. En la figura se ve el resultado para $n=16$ y el mismo radio.

    Considera dos casos:
    \begin{enumerate}
        \item Usando una malla \textbf{no indexada}.
        \item Usando una malla \textbf{indexada}.
    \end{enumerate}

    \begin{center}
    \begin{tikzpicture}
        % Polígono regular de 16 lados (solo contorno)
        \coordinate (center) at (0,0);
        \draw[thin, color=gray!60] (0:1.5)
            \foreach \i in {1,...,15} { -- (\i*360/16:1.5) } -- cycle;
        % Radios
        \foreach \i in {0,...,15} {
            \draw[thin, color=gray!60] (center) -- (\i*360/16:1.5);
        }
        % Ejes de coordenadas
        \draw[->, thick, color=red] (center) -- (0:2) node[right] {};
        \draw[->, thick, color=green!80!black] (center) -- (90:2) node[above] {};
        \end{tikzpicture}
    \end{center}
\end{ejercicio}

\begin{solucion} Solución al problema 2.4:
    \lstinputlisting{../code/EjerciciosTeoria/problema_2_4.gd}
\end{solucion}

\begin{ejercicio}
    \textbf{Generación de malla con segmentos de normales}

    Crea un script global (autoload) con una función que genere un objeto de tipo \texttt{MeshInstance3D} con una malla no indexada que contenga los segmentos representando las normales de una malla dada. La función tendrá la siguiente declaración:

    \begin{verbatim}
func genSegNormales(
    verts: PackedVector3Array, 
    norms: PackedVector3Array,
    lon: float, 
    color: Color 
) -> MeshInstance3D:
    \end{verbatim}

    Donde \texttt{verts} es la tabla de vértices de la malla original, \texttt{norms} la tabla de normales, \texttt{lon} la longitud de los segmentos y \texttt{color} el color de los segmentos. Usa el tipo de primitiva líneas (\texttt{PRIMITIVE\_LINES}), y asegúrate de que a los segmentos no les afecta la iluminación.

    \textbf{Continuación (Uso):}
    Una vez tengas la función disponible, úsala en la función \texttt{\_ready} de alguna malla (por ejemplo, el Donut o los cubos de la práctica), para añadir al objeto un nodo hijo con la malla de segmentos creada por la función.

    Puedes capturar el evento de pulsación de la tecla \textbf{N} del objeto para activar y desactivar la visualización de las normales en ese objeto. Para ello, usa un valor lógico y el atributo de visibilidad de la malla de segmentos.

    \begin{center}
    \begin{tikzpicture}
        % Representación esquemática de una superficie con normales
        \draw[thick, fill=blue!10] plot [smooth cycle] coordinates {(0,0) (1,0.5) (2,0) (1.5,-1) (0.5,-0.8)};
        % Normales
        \draw[->, red, thick] (0,0) -- (0, 0.8);
        \draw[->, red, thick] (1,0.5) -- (1, 1.3);
        \draw[->, red, thick] (2,0) -- (2.2, 0.7);
        \draw[->, red, thick] (1.5,-1) -- (1.6, -0.2);
        \draw[->, red, thick] (0.5,-0.8) -- (0.4, 0);
        \node[font=\footnotesize] at (1,-1.5) {Esquema conceptual: Superficie y Normales};
    \end{tikzpicture}
    \end{center}
\end{ejercicio}

\begin{solucion} Solución al problema 2.5:
    Se encuentra en el fichero \texttt{Global.gd} del autoload, cargando anteriormente.
    De todas formas, se añade aquí la función para que se vea más fácilmente:
    \begin{lstlisting}[language=GDScript]
func genSegNormales( verts, norms : PackedVector3Array, lon : float, color : Color ) -> MeshInstance3D:
    var line_verts = PackedVector3Array()
    var line_colors = PackedColorArray()

    for i in range(verts.size()):
        var origen = verts[i]
        var destino = origen + norms[i] * lon

        line_verts.push_back(origen)
        line_verts.push_back(destino)

        line_colors.push_back(color)
        line_colors.push_back(color)

    var arrays = []
    arrays.resize(Mesh.ARRAY_MAX)
    arrays[Mesh.ARRAY_VERTEX] = line_verts
    arrays[Mesh.ARRAY_COLOR] = line_colors

    var arr_mesh = ArrayMesh.new()
    arr_mesh.add_surface_from_arrays(Mesh.PRIMITIVE_LINES, arrays)

    var material = StandardMaterial3D.new()
    material.shading_mode = BaseMaterial3D.SHADING_MODE_UNSHADED
    material.vertex_color_use_as_albedo = true

    var mi = MeshInstance3D.new()
    mi.mesh = arr_mesh
    mi.material_override = material

	return mi
    \end{lstlisting}
\end{solucion}



\section{Sesión 3}

% -------------
% Ejercicios s03
% -------------

% Problema 3.1
\begin{ejercicio}
%\textbf{Problema 3.1: Demostración del cálculo del Producto Escalar}

Demuestra que efectivamente el producto escalar de dos vectores se puede
calcular (usando sus coordenadas en cualquier marco cartesiano) como la
suma del producto componente a componente. Usa las propiedades que
definen dicho producto escalar.
\end{ejercicio}

\begin{solucion}

\textbf{Hipótesis y Datos de Partida:}
\begin{enumerate}
    \item Definimos dos vectores $\vec{u}$ y $\vec{v}$ en un espacio vectorial $V$.
    \item Trabajamos en un \textbf{marco cartesiano}, lo que implica una base ortonormal $\{\hat{e}_i\}$.
    \item Las propiedades de esta base especial son:
    \begin{itemize}
        \item $\hat{e}_i \cdot \hat{e}_i = 1$ (son unitarios).
        \item $\hat{e}_i \cdot \hat{e}_j = 0$ si $i \neq j$ (son perpendiculares).
    \end{itemize}
    \item Expresamos los vectores mediante sus coordenadas en esta base:
    \begin{align*}
        \vec{u} &= \sum_{i=1}^n a_i \hat{e}_i \\
        \vec{v} &= \sum_{j=1}^n b_j \hat{e}_j
    \end{align*}
\end{enumerate}

\textbf{Demostración Paso a Paso:}
\begin{enumerate}
    \item \textbf{Planteamiento del producto:}
    \[
        \vec{u} \cdot \vec{v} = \left( \sum_{i=1}^n a_i \hat{e}_i \right) \cdot \left( \sum_{j=1}^n b_j \hat{e}_j \right)
    \]
    \item \textbf{Aplicación de la Propiedad Distributiva:}
    \[
        = \sum_{i=1}^n \sum_{j=1}^n a_i b_j (\hat{e}_i \cdot \hat{e}_j)
    \]
    \item \textbf{Aplicación de la Propiedad Asociativa (Escalares):}
    Los coeficientes $a_i$ y $b_j$ son reales, así que pueden factorizarse fuera del producto escalar.
    \item \textbf{Aplicación de las Propiedades de la Base Ortonormal:}
    \begin{itemize}
        \item Cuando $i \neq j$, el término es $0$.
        \item Cuando $i = j$, el término es $1$.
    \end{itemize}
    Así, sólo sobreviven los términos con $i = j$:
    \[
        \vec{u} \cdot \vec{v} = \sum_{i=1}^n a_i b_i
    \]
\end{enumerate}

\textbf{Conclusión:} Queda demostrado que, en un marco cartesiano, el producto escalar es la suma de los productos de las componentes homólogas.
\end{solucion}

% Problema 3.2
\begin{ejercicio}
%\textbf{Problema 3.2: Demostración del cálculo del Producto Vectorial}

Demuestra que el producto vectorial de dos vectores se puede calcular usando sus coordenadas en cualquier marco cartesiano según se ha indicado.
\end{ejercicio}

\begin{solucion}
\textbf{Hipótesis y Datos de Partida:}
\begin{enumerate}
    \item Trabajamos en 3D con la base especial $\{\hat{x}, \hat{y}, \hat{z}\}$.
    \item Propiedades definitorias del producto vectorial en esta base:
    \begin{itemize}
        \item $\hat{x} \times \hat{y} = \hat{z}$, $\hat{y} \times \hat{z} = \hat{x}$, $\hat{z} \times \hat{x} = \hat{y}$ (ciclo dextrógiro).
        \item Por la propiedad anticonmutativa, el orden inverso invierte el signo: $\hat{y} \times \hat{x} = -\hat{z}$, etc.
        \item El producto de un vector por sí mismo es nulo: $\hat{x} \times \hat{x} = \vec{0}$, etc.
    \end{itemize}
    \item Vectores definidos por coordenadas:
    \begin{align*}
        \vec{u} &= x_0\hat{x} + y_0\hat{y} + z_0\hat{z} \\
        \vec{v} &= x_1\hat{x} + y_1\hat{y} + z_1\hat{z}
    \end{align*}
\end{enumerate}

\textbf{Demostración Paso a Paso:}
\begin{enumerate}
    \item \textbf{Planteamiento:}
    \[
        \vec{u} \times \vec{v} = (x_0\hat{x} + y_0\hat{y} + z_0\hat{z}) \times (x_1\hat{x} + y_1\hat{y} + z_1\hat{z})
    \]
    \item \textbf{Expansión (Distributiva):}
    \begin{align*}
        &= x_0x_1(\hat{x}\times\hat{x}) + x_0y_1(\hat{x}\times\hat{y}) + x_0z_1(\hat{x}\times\hat{z}) \\
        &\quad + y_0x_1(\hat{y}\times\hat{x}) + y_0y_1(\hat{y}\times\hat{y}) + y_0z_1(\hat{y}\times\hat{z}) \\
        &\quad + z_0x_1(\hat{z}\times\hat{x}) + z_0y_1(\hat{z}\times\hat{y}) + z_0z_1(\hat{z}\times\hat{z})
    \end{align*}
    \item \textbf{Simplificación con Propiedades de la Base:}
    \begin{align*}
        &= 0 + x_0y_1\hat{z} + x_0z_1(-\hat{y}) \\
        &\quad + y_0x_1(-\hat{z}) + 0 + y_0z_1\hat{x} \\
        &\quad + z_0x_1\hat{y} + z_0y_1(-\hat{x}) + 0
    \end{align*}
    \item \textbf{Agrupación por componentes:}
    \begin{align*}
        \text{Componente } \hat{x}: &\quad y_0z_1 - z_0y_1 \\
        \text{Componente } \hat{y}: &\quad z_0x_1 - x_0z_1 \\
        \text{Componente } \hat{z}: &\quad x_0y_1 - y_0x_1
    \end{align*}
    \[
        \vec{u} \times \vec{v} = (y_0z_1 - z_0y_1)\hat{x} + (z_0x_1 - x_0z_1)\hat{y} + (x_0y_1 - y_0x_1)\hat{z}
    \]
\end{enumerate}

\textbf{Conclusión:} El vector resultante en coordenadas es:
\[
    \vec{u} \times \vec{v} = \begin{pmatrix}
        y_0z_1 - z_0y_1 \\
        z_0x_1 - x_0z_1 \\
        x_0y_1 - y_0x_1
    \end{pmatrix}
\]
Esto coincide exactamente con la definición matricial dada en el documento.
\end{solucion}

% Problema 3.3
\begin{ejercicio}
%\textbf{Problema 3.3: Demostración de Ortogonalidad}

Demuestra que el producto vectorial de dos vectores es perpendicular a cada
uno de esos dos vectores.
\end{ejercicio}

\begin{solucion}
\textbf{Datos de Partida:}
\begin{enumerate}
    \item Condición de perpendicularidad: Dos vectores son perpendiculares si su producto escalar es $0$.
    \item Usaremos los resultados demostrados en los Problemas 3.1 y 3.2.
\end{enumerate}

\textbf{Demostración (para $\vec{u}$):}

Queremos probar que $\vec{u} \cdot \vec{w} = 0$.

Sea $\vec{w} = \vec{u} \times \vec{v}$. Sus componentes son (del Prob 3.2):
\begin{align*}
    w_x &= y_0z_1 - z_0y_1 \\
    w_y &= z_0x_1 - x_0z_1 \\
    w_z &= x_0y_1 - y_0x_1
\end{align*}

Calculamos el producto escalar $\vec{u} \cdot \vec{w}$ usando la fórmula de componentes (del Prob 3.1):
\[
    \vec{u} \cdot \vec{w} = x_0w_x + y_0w_y + z_0w_z
\]
Sustituyendo los componentes de $\vec{w}$:
\begin{align*}
    &= x_0(y_0z_1 - z_0y_1) + y_0(z_0x_1 - x_0z_1) + z_0(x_0y_1 - y_0x_1) \\
    &= x_0y_0z_1 - x_0z_0y_1 + y_0z_0x_1 - y_0x_0z_1 + z_0x_0y_1 - z_0y_0x_1
\end{align*}

Reordenamos los términos para ver las cancelaciones:
\begin{itemize}
    \item $x_0y_0z_1$ se cancela con $-y_0x_0z_1$ (son idénticos, el orden de factores reales no altera el producto).
    \item $-x_0z_0y_1$ se cancela con $z_0x_0y_1$.
    \item $y_0z_0x_1$ se cancela con $-z_0y_0x_1$.
\end{itemize}

\textbf{Resultado:}
\[
    \vec{u} \cdot \vec{w} = 0
\]

\textit{(Nota: La demostración para $\vec{v}$ es análoga, sustituyendo las coordenadas de $\vec{v}$ en el producto escalar, y también resultará en $0$).}

\textbf{Conclusión:}
Hemos demostrado algebraicamente la propiedad geométrica fundamental mencionada en la página 30: el producto vectorial genera una dirección perpendicular al plano formado por los dos vectores originales.
\end{solucion}

\begin{ejercicio}
Demuestra que el producto escalar de vectores en 2D es invariante por rotación. Es decir, que para cualquier ángulo $\theta$ y vectores $\vec{u}$ y $\vec{v}$ se cumple:
$$ \vec{u} \cdot \vec{v} = R_\theta(\vec{u}) \cdot R_\theta(\vec{v}) $$
Se requiere realizar la demostración utilizando las coordenadas de los vectores en un marco cartesiano arbitrario.
\end{ejercicio}

\begin{solucion}
Para demostrar la invariancia del producto escalar bajo una transformación de rotación en el espacio euclídeo bidimensional ($\mathbb{R}^2$), procederemos algebraicamente definiendo los componentes de los vectores y la matriz de transformación correspondiente.

Sean $\vec{u}$ y $\vec{v}$ dos vectores libres en $\mathbb{R}^2$ definidos por sus componentes en un marco cartesiano:
$$ \vec{u} = \begin{pmatrix} u_x \\ u_y \end{pmatrix}, \quad \vec{v} = \begin{pmatrix} v_x \\ v_y \end{pmatrix} $$

La definición estándar del producto escalar (producto punto) en coordenadas cartesianas viene dada por:
\begin{equation}
\vec{u} \cdot \vec{v} = u_x v_x + u_y v_y
\end{equation}

Sea $R_\theta$ la transformación de rotación por un ángulo $\theta$ alrededor del origen. La matriz asociada a esta transformación en 2D, $M_R$, se define como:
$$ M_R = \begin{pmatrix} \cos\theta & -\sin\theta \\ \sin\theta & \cos\theta \end{pmatrix} $$

Aplicamos la transformación lineal a los vectores $\vec{u}$ y $\vec{v}$ mediante la multiplicación matricial:

1. Para el vector $\vec{u}' = R_\theta(\vec{u})$:
$$ \vec{u}' = \begin{pmatrix} \cos\theta & -\sin\theta \\ \sin\theta & \cos\theta \end{pmatrix} \begin{pmatrix} u_x \\ u_y \end{pmatrix} = \begin{pmatrix} u_x \cos\theta - u_y \sin\theta \\ u_x \sin\theta + u_y \cos\theta \end{pmatrix} $$
Denotamos las componentes transformadas como $u'_x = u_x \cos\theta - u_y \sin\theta$ y $u'_y = u_x \sin\theta + u_y \cos\theta$.

2. Para el vector $\vec{v}' = R_\theta(\vec{v})$:
$$ \vec{v}' = \begin{pmatrix} \cos\theta & -\sin\theta \\ \sin\theta & \cos\theta \end{pmatrix} \begin{pmatrix} v_x \\ v_y \end{pmatrix} = \begin{pmatrix} v_x \cos\theta - v_y \sin\theta \\ v_x \sin\theta + v_y \cos\theta \end{pmatrix} $$
Denotamos las componentes transformadas como $v'_x = v_x \cos\theta - v_y \sin\theta$ y $v'_y = v_x \sin\theta + v_y \cos\theta$.

Procedemos ahora a calcular el producto escalar de los vectores transformados, $R_\theta(\vec{u}) \cdot R_\theta(\vec{v}) = u'_x v'_x + u'_y v'_y$:

\begin{align*}
R_\theta(\vec{u}) \cdot R_\theta(\vec{v}) &= (u_x \cos\theta - u_y \sin\theta)(v_x \cos\theta - v_y \sin\theta) \\
&\quad + (u_x \sin\theta + u_y \cos\theta)(v_x \sin\theta + v_y \cos\theta)
\end{align*}

Expandimos los términos algebraicos:

\begin{align*}
R_\theta(\vec{u}) \cdot R_\theta(\vec{v}) &= (u_x v_x \cos^2\theta - u_x v_y \cos\theta \sin\theta - u_y v_x \sin\theta \cos\theta + u_y v_y \sin^2\theta) \\
&\quad + (u_x v_x \sin^2\theta + u_x v_y \sin\theta \cos\theta + u_y v_x \cos\theta \sin\theta + u_y v_y \cos^2\theta)
\end{align*}

Agrupamos los términos comunes en función de los coeficientes de los vectores originales:

\begin{align*}
R_\theta(\vec{u}) \cdot R_\theta(\vec{v}) &= u_x v_x (\cos^2\theta + \sin^2\theta) \\
&\quad + u_y v_y (\sin^2\theta + \cos^2\theta) \\
&\quad + u_x v_y (-\cos\theta \sin\theta + \sin\theta \cos\theta) \\
&\quad + u_y v_x (-\sin\theta \cos\theta + \cos\theta \sin\theta)
\end{align*}

Aplicamos la identidad trigonométrica fundamental $\cos^2\theta + \sin^2\theta = 1$ y observamos que los términos cruzados se cancelan:

\begin{align*}
R_\theta(\vec{u}) \cdot R_\theta(\vec{v}) &= u_x v_x (1) + u_y v_y (1) + u_x v_y (0) + u_y v_x (0) \\
&= u_x v_x + u_y v_y
\end{align*}

Comparando este resultado con la definición inicial en la Ecuación (1), concluimos que:
$$ R_\theta(\vec{u}) \cdot R_\theta(\vec{v}) = \vec{u} \cdot \vec{v} $$
Q.E.D.
\end{solucion}

\begin{ejercicio}
Demuestra que en 2D las rotaciones no modifican la longitud de un vector (isometría). Es decir, que para cualquier ángulo $\theta$ y vector $\vec{v}$, se cumple:
$$ \| R_\theta(\vec{v}) \| = \| \vec{v} \| $$
\end{ejercicio}

\begin{solucion}
Para demostrar que la rotación es una transformación isométrica que preserva la norma (longitud) de los vectores, utilizaremos la relación fundamental entre la norma euclídea y el producto escalar.

La definición de la norma de un vector $\vec{v}$ en función del producto escalar es:
$$ \| \vec{v} \| = \sqrt{\vec{v} \cdot \vec{v}} $$
Elevando al cuadrado ambos lados, tenemos:
\begin{equation}
\| \vec{v} \|^2 = \vec{v} \cdot \vec{v}
\end{equation}

Consideremos ahora la norma al cuadrado del vector transformado $R_\theta(\vec{v})$:
$$ \| R_\theta(\vec{v}) \|^2 = R_\theta(\vec{v}) \cdot R_\theta(\vec{v}) $$

Basándonos en la propiedad demostrada en el Ejercicio 3.4 (invariancia del producto escalar bajo rotación), sabemos que para cualesquiera vectores $\vec{a}$ y $\vec{b}$, se cumple $\vec{a} \cdot \vec{b} = R_\theta(\vec{a}) \cdot R_\theta(\vec{b})$.

En este caso particular, hacemos $\vec{a} = \vec{v}$ y $\vec{b} = \vec{v}$. Aplicando la propiedad de invariancia:
$$ R_\theta(\vec{v}) \cdot R_\theta(\vec{v}) = \vec{v} \cdot \vec{v} $$

Sustituyendo esto en la expresión de la norma transformada:
$$ \| R_\theta(\vec{v}) \|^2 = \vec{v} \cdot \vec{v} $$

Dado que $\vec{v} \cdot \vec{v} = \| \vec{v} \|^2$ según la Ecuación (2), obtenemos:
$$ \| R_\theta(\vec{v}) \|^2 = \| \vec{v} \|^2 $$

Tomando la raíz cuadrada positiva en ambos lados (dado que la norma es una magnitud no negativa):
$$ \| R_\theta(\vec{v}) \| = \| \vec{v} \| $$

Por lo tanto, queda demostrado que la aplicación de una matriz de rotación $R_\theta$ no altera la longitud del vector, confirmando que las rotaciones son isometrías.
\end{solucion}

\begin{ejercicio}
Demuestra que si rotamos en 2D un vector +90 grados ($\pi/2$) o -90 grados ($-\pi/2$), obtenemos otro vector perpendicular al original. Es decir, si $\|\theta\| = \pi/2$, entonces:
$$ \vec{v} \cdot R_\theta(\vec{v}) = 0 $$
\end{ejercicio}

\begin{solucion}
Para demostrar la perpendicularidad entre un vector original $\vec{v}$ y su versión rotada $\pm 90^\circ$, utilizaremos la definición algebraica del producto escalar y la matriz de rotación específica para estos ángulos.

Sea $\vec{v}$ un vector arbitrario en $\mathbb{R}^2$:
$$ \vec{v} = \begin{pmatrix} v_x \\ v_y \end{pmatrix} $$

La matriz de rotación general $R_\theta$ es:
$$ R_\theta = \begin{pmatrix} \cos\theta & -\sin\theta \\ \sin\theta & \cos\theta \end{pmatrix} $$

Analizaremos los dos casos solicitados: $\theta = \pi/2$ y $\theta = -\pi/2$.

\textbf{Caso 1: Rotación de $+\pi/2$ ($90^\circ$)}
Sustituimos $\theta = \pi/2$ en la matriz de rotación, sabiendo que $\cos(\pi/2) = 0$ y $\sin(\pi/2) = 1$:
$$ R_{\pi/2} = \begin{pmatrix} 0 & -1 \\ 1 & 0 \end{pmatrix} $$

Calculamos el vector transformado $\vec{v}' = R_{\pi/2}(\vec{v})$:
$$ \vec{v}' = \begin{pmatrix} 0 & -1 \\ 1 & 0 \end{pmatrix} \begin{pmatrix} v_x \\ v_y \end{pmatrix} = \begin{pmatrix} -v_y \\ v_x \end{pmatrix} $$

Ahora calculamos el producto escalar entre el vector original y el transformado:
$$ \vec{v} \cdot \vec{v}' = v_x(-v_y) + v_y(v_x) = -v_x v_y + v_x v_y = 0 $$

\textbf{Caso 2: Rotación de $-\pi/2$ ($-90^\circ$)}
Sustituimos $\theta = -\pi/2$ en la matriz, sabiendo que $\cos(-\pi/2) = 0$ y $\sin(-\pi/2) = -1$:
$$ R_{-\pi/2} = \begin{pmatrix} 0 & 1 \\ -1 & 0 \end{pmatrix} $$

Calculamos el vector transformado $\vec{v}'' = R_{-\pi/2}(\vec{v})$:
$$ \vec{v}'' = \begin{pmatrix} 0 & 1 \\ -1 & 0 \end{pmatrix} \begin{pmatrix} v_x \\ v_y \end{pmatrix} = \begin{pmatrix} v_y \\ -v_x \end{pmatrix} $$

Calculamos el producto escalar:
$$ \vec{v} \cdot \vec{v}'' = v_x(v_y) + v_y(-v_x) = v_x v_y - v_x v_y = 0 $$

\textbf{Conclusión:}
En ambos casos, el producto escalar es nulo. Dado que $\vec{a} \cdot \vec{b} = 0 \iff \vec{a} \perp \vec{b}$ (para vectores no nulos), queda demostrado que el vector rotado $\pm 90^\circ$ es perpendicular al original.
\end{solucion}

\begin{ejercicio}
Demuestra que una matriz de rotación en 2D es siempre ortonormal, independientemente del ángulo. Esto implica demostrar que:
1. Sus filas son ortogonales entre sí (perpendiculares).
2. Sus columnas son ortogonales entre sí.
3. Cada fila y cada columna tiene norma (longitud) igual a 1.
\end{ejercicio}

\begin{solucion}
Una matriz $M$ es ortonormal (u ortogonal) si cumple que $M^T M = I$, lo cual equivale a que sus filas y columnas formen una base ortonormal. Analizaremos las propiedades de filas y columnas de la matriz de rotación general.

Sea $R_\theta$ la matriz de rotación:
$$ R_\theta = \begin{pmatrix} \cos\theta & -\sin\theta \\ \sin\theta & \cos\theta \end{pmatrix} $$

Denotamos las filas como vectores $\vec{r}_1, \vec{r}_2$ y las columnas como $\vec{c}_1, \vec{c}_2$:
$$ \vec{r}_1 = (\cos\theta, -\sin\theta), \quad \vec{r}_2 = (\sin\theta, \cos\theta) $$
$$ \vec{c}_1 = \begin{pmatrix} \cos\theta \\ \sin\theta \end{pmatrix}, \quad \vec{c}_2 = \begin{pmatrix} -\sin\theta \\ \cos\theta \end{pmatrix} $$

\textbf{1. Ortogonalidad de las filas:}
Calculamos el producto escalar $\vec{r}_1 \cdot \vec{r}_2$:
$$ \vec{r}_1 \cdot \vec{r}_2 = (\cos\theta)(\sin\theta) + (-\sin\theta)(\cos\theta) = \sin\theta\cos\theta - \sin\theta\cos\theta = 0 $$
Las filas son perpendiculares.

\textbf{2. Normalidad de las filas (Longitud unitaria):}
Calculamos la norma al cuadrado de cada fila usando la identidad $\cos^2\theta + \sin^2\theta = 1$:
$$ \|\vec{r}_1\|^2 = (\cos\theta)^2 + (-\sin\theta)^2 = \cos^2\theta + \sin^2\theta = 1 \implies \|\vec{r}_1\| = 1 $$
$$ \|\vec{r}_2\|^2 = (\sin\theta)^2 + (\cos\theta)^2 = \sin^2\theta + \cos^2\theta = 1 \implies \|\vec{r}_2\| = 1 $$

\textbf{3. Ortogonalidad de las columnas:}
Calculamos el producto escalar $\vec{c}_1 \cdot \vec{c}_2$:
$$ \vec{c}_1 \cdot \vec{c}_2 = (\cos\theta)(-\sin\theta) + (\sin\theta)(\cos\theta) = -\sin\theta\cos\theta + \sin\theta\cos\theta = 0 $$
Las columnas son perpendiculares.

\textbf{4. Normalidad de las columnas:}
$$ \|\vec{c}_1\|^2 = \cos^2\theta + \sin^2\theta = 1 \implies \|\vec{c}_1\| = 1 $$
$$ \|\vec{c}_2\|^2 = (-\sin\theta)^2 + \cos^2\theta = \sin^2\theta + \cos^2\theta = 1 \implies \|\vec{c}_2\| = 1 $$

\textbf{Conclusión:}
Dado que tanto las filas como las columnas son vectores unitarios y ortogonales entre sí para cualquier valor de $\theta$, la matriz de rotación $R_\theta$ es siempre una matriz ortonormal.
\end{solucion}

\begin{ejercicio}
Demuestra que, en 2D, el producto de una matriz de rotación y una de escalado no es conmutativo en general, excepto si el escalado es uniforme.
\end{ejercicio}

\begin{solucion}
Para analizar la conmutatividad entre la rotación y el escalado, definiremos las matrices correspondientes en el espacio bidimensional. Consideraremos las matrices de $2 \times 2$, dado que ambas son transformaciones lineales y no requieren necesariamente coordenadas homogéneas para demostrar esta propiedad (aunque el resultado es idéntico en $3 \times 3$ con la última fila/columna canónica).

Sean las matrices de rotación $R_\theta$ y de escalado $S(s_x, s_y)$:
$$ R_\theta = \begin{pmatrix} \cos\theta & -\sin\theta \\ \sin\theta & \cos\theta \end{pmatrix}, \quad S = \begin{pmatrix} s_x & 0 \\ 0 & s_y \end{pmatrix} $$

Calculamos el producto $R_\theta \cdot S$ (aplicar escalado y luego rotación):
$$ R_\theta \cdot S = \begin{pmatrix} \cos\theta & -\sin\theta \\ \sin\theta & \cos\theta \end{pmatrix} \begin{pmatrix} s_x & 0 \\ 0 & s_y \end{pmatrix} = \begin{pmatrix} s_x \cos\theta & -s_y \sin\theta \\ s_x \sin\theta & s_y \cos\theta \end{pmatrix} $$

Calculamos el producto $S \cdot R_\theta$ (aplicar rotación y luego escalado):
$$ S \cdot R_\theta = \begin{pmatrix} s_x & 0 \\ 0 & s_y \end{pmatrix} \begin{pmatrix} \cos\theta & -\sin\theta \\ \sin\theta & \cos\theta \end{pmatrix} = \begin{pmatrix} s_x \cos\theta & -s_x \sin\theta \\ s_y \sin\theta & s_y \cos\theta \end{pmatrix} $$

Para que las matrices conmuten, es decir, $R_\theta \cdot S = S \cdot R_\theta$, sus componentes deben ser idénticos término a término. Comparamos los términos fuera de la diagonal principal:

\begin{enumerate}
    \item Elemento $(1,2)$: $-s_y \sin\theta = -s_x \sin\theta \implies (s_x - s_y)\sin\theta = 0$
    \item Elemento $(2,1)$: $s_x \sin\theta = s_y \sin\theta \implies (s_x - s_y)\sin\theta = 0$
\end{enumerate}

Para que la igualdad se cumpla para un ángulo de rotación general ($\sin\theta \neq 0$), es condición necesaria y suficiente que:
$$ s_x = s_y $$

\textbf{Conclusión:}
Si $s_x \neq s_y$ (escalado no uniforme), los productos matriciales son distintos, demostrando que la operación no es conmutativa en general.
Si $s_x = s_y = s$ (escalado uniforme), la matriz de escalado se convierte en $sI$ (donde $I$ es la identidad), la cual conmuta con cualquier matriz cuadrada.
\end{solucion}

\begin{ejercicio}
Demuestra que en 2D, el producto de una matriz de rotación y otra de traslación (por un vector no nulo) no es conmutativo.
\end{ejercicio}

\begin{solucion}
Dado que la traslación es una transformación afín y no lineal, es imprescindible utilizar \textbf{coordenadas homogéneas} para representarla como una multiplicación matricial. Trabajaremos con matrices de $3 \times 3$.

Sea $R_\theta$ la matriz de rotación y $T_{\vec{t}}$ la matriz de traslación por un vector $\vec{t} = (t_x, t_y)$:
$$ R_\theta = \begin{pmatrix} \cos\theta & -\sin\theta & 0 \\ \sin\theta & \cos\theta & 0 \\ 0 & 0 & 1 \end{pmatrix}, \quad T_{\vec{t}} = \begin{pmatrix} 1 & 0 & t_x \\ 0 & 1 & t_y \\ 0 & 0 & 1 \end{pmatrix} $$

Calculamos el producto $R_\theta \cdot T_{\vec{t}}$ (primero se traslada, luego se rota):
$$ R_\theta \cdot T_{\vec{t}} = \begin{pmatrix} \cos\theta & -\sin\theta & 0 \\ \sin\theta & \cos\theta & 0 \\ 0 & 0 & 1 \end{pmatrix} \begin{pmatrix} 1 & 0 & t_x \\ 0 & 1 & t_y \\ 0 & 0 & 1 \end{pmatrix} = \begin{pmatrix} \cos\theta & -\sin\theta & t_x \cos\theta - t_y \sin\theta \\ \sin\theta & \cos\theta & t_x \sin\theta + t_y \cos\theta \\ 0 & 0 & 1 \end{pmatrix} $$
Geométricamente, esto rota el punto y también rota el vector de traslación aplicado.

Calculamos el producto $T_{\vec{t}} \cdot R_\theta$ (primero se rota, luego se traslada):
$$ T_{\vec{t}} \cdot R_\theta = \begin{pmatrix} 1 & 0 & t_x \\ 0 & 1 & t_y \\ 0 & 0 & 1 \end{pmatrix} \begin{pmatrix} \cos\theta & -\sin\theta & 0 \\ \sin\theta & \cos\theta & 0 \\ 0 & 0 & 1 \end{pmatrix} = \begin{pmatrix} \cos\theta & -\sin\theta & t_x \\ \sin\theta & \cos\theta & t_y \\ 0 & 0 & 1 \end{pmatrix} $$
Geométricamente, esto rota el punto alrededor del origen y luego aplica la traslación original sin modificar.

\textbf{Comparación:}
Observamos la tercera columna (la componente de traslación resultante) de ambas matrices resultantes:
$$ \begin{pmatrix} t_x \cos\theta - t_y \sin\theta \\ t_x \sin\theta + t_y \cos\theta \\ 1 \end{pmatrix} \neq \begin{pmatrix} t_x \\ t_y \\ 1 \end{pmatrix} $$

Para que fuesen iguales en un caso general ($\theta \neq 0$), se requeriría que $t_x = 0$ y $t_y = 0$. Dado que el enunciado especifica un vector de traslación no nulo, concluimos que las matrices son distintas.

\textbf{Conclusión:}
El orden de las operaciones altera el resultado final: rotar y luego trasladar lleva a una posición diferente que trasladar y luego rotar (donde el desplazamiento también sufre la rotación). Por tanto, no son conmutativas.
\end{solucion}

\begin{ejercicio}
Demuestra que el producto escalar de vectores en 3D es invariante por rotaciones entorno a los ejes cartesianos, y que estas tampoco modifican la longitud de un vector.
\end{ejercicio}

\begin{solucion}
Para demostrar la invariancia del producto escalar en $\mathbb{R}^3$ bajo rotaciones cartesianas, tomaremos sin pérdida de generalidad el caso de una rotación alrededor del eje $Z$ por un ángulo $\theta$. El procedimiento es análogo para los ejes $X$ e $Y$ debido a la simetría cíclica de las coordenadas.

La matriz de rotación $R_{z,\theta}$ se define como:
$$ R_{z,\theta} = \begin{pmatrix} \cos\theta & -\sin\theta & 0 \\ \sin\theta & \cos\theta & 0 \\ 0 & 0 & 1 \end{pmatrix} $$

Sean $\vec{u} = (u_x, u_y, u_z)$ y $\vec{v} = (v_x, v_y, v_z)$ dos vectores arbitrarios. El producto escalar original es:
\begin{equation}
\vec{u} \cdot \vec{v} = u_x v_x + u_y v_y + u_z v_z
\end{equation}

Calculamos los vectores transformados $\vec{u}' = R_{z,\theta}(\vec{u})$ y $\vec{v}' = R_{z,\theta}(\vec{v})$:
$$ \vec{u}' = \begin{pmatrix} u_x \cos\theta - u_y \sin\theta \\ u_x \sin\theta + u_y \cos\theta \\ u_z \end{pmatrix}, \quad \vec{v}' = \begin{pmatrix} v_x \cos\theta - v_y \sin\theta \\ v_x \sin\theta + v_y \cos\theta \\ v_z \end{pmatrix} $$

Ahora calculamos el producto escalar de los vectores transformados:
\begin{align*}
\vec{u}' \cdot \vec{v}' &= (u_x \cos\theta - u_y \sin\theta)(v_x \cos\theta - v_y \sin\theta) \\
&\quad + (u_x \sin\theta + u_y \cos\theta)(v_x \sin\theta + v_y \cos\theta) \\
&\quad + u_z v_z
\end{align*}

Expandiendo los términos correspondientes a las componentes $x$ e $y$ (idéntico al caso 2D):
\begin{align*}
&= (u_x v_x \cos^2\theta - u_x v_y \cos\theta \sin\theta - u_y v_x \sin\theta \cos\theta + u_y v_y \sin^2\theta) \\
&\quad + (u_x v_x \sin^2\theta + u_x v_y \sin\theta \cos\theta + u_y v_x \cos\theta \sin\theta + u_y v_y \cos^2\theta) \\
&\quad + u_z v_z
\end{align*}

Agrupando y simplificando usando $\sin^2\theta + \cos^2\theta = 1$:
\begin{align*}
\vec{u}' \cdot \vec{v}' &= u_x v_x(\cos^2\theta + \sin^2\theta) + u_y v_y(\sin^2\theta + \cos^2\theta) + u_z v_z \\
&= u_x v_x + u_y v_y + u_z v_z \\
&= \vec{u} \cdot \vec{v}
\end{align*}

\textbf{Invariancia de la longitud:}
Utilizando la relación $\| \vec{v} \|^2 = \vec{v} \cdot \vec{v}$ y la propiedad recién demostrada:
$$ \| R_{z,\theta}(\vec{v}) \|^2 = R_{z,\theta}(\vec{v}) \cdot R_{z,\theta}(\vec{v}) = \vec{v} \cdot \vec{v} = \| \vec{v} \|^2 $$
Tomando la raíz cuadrada, concluimos que $\| R_{z,\theta}(\vec{v}) \| = \| \vec{v} \|$.
\end{solucion}

\begin{ejercicio}
Demuestra que el producto vectorial de dos vectores rota igual que lo hacen esos dos vectores. Es decir, para cualesquiera vectores $\vec{u}, \vec{v}$ y un ángulo $\theta$ con eje $\hat{e}$, se cumple:
$$ R_{\theta,\hat{e}}(\vec{u} \times \vec{v}) = R_{\theta,\hat{e}}(\vec{u}) \times R_{\theta,\hat{e}}(\vec{v}) $$
\end{ejercicio}

\begin{solucion}
Para esta demostración, consideraremos la rotación alrededor del eje $Z$ ($R_{z,\theta}$), ya que la lógica es extensible a cualquier eje cartesiano por permutación de índices.

Definimos $\vec{w} = \vec{u} \times \vec{v}$. Sus componentes son:
$$ \vec{w} = \begin{pmatrix} w_x \\ w_y \\ w_z \end{pmatrix} = \begin{pmatrix} u_y v_z - u_z v_y \\ u_z v_x - u_x v_z \\ u_x v_y - u_y v_x \end{pmatrix} $$

\textbf{Parte 1: Rotación del producto vectorial original ($R_{z,\theta}(\vec{w})$)}
Aplicamos la matriz de rotación al vector $\vec{w}$:
$$ R_{z,\theta}(\vec{w}) = \begin{pmatrix} w_x \cos\theta - w_y \sin\theta \\ w_x \sin\theta + w_y \cos\theta \\ w_z \end{pmatrix} $$
Sustituyendo los componentes de $\vec{w}$:
\begin{equation}
R_{z,\theta}(\vec{w})_x = (u_y v_z - u_z v_y)\cos\theta - (u_z v_x - u_x v_z)\sin\theta
\end{equation}
\begin{equation}
R_{z,\theta}(\vec{w})_y = (u_y v_z - u_z v_y)\sin\theta + (u_z v_x - u_x v_z)\cos\theta
\end{equation}
\begin{equation}
R_{z,\theta}(\vec{w})_z = u_x v_y - u_y v_x
\end{equation}

\textbf{Parte 2: Producto vectorial de los vectores rotados ($\vec{u}' \times \vec{v}'$)}
Sean $\vec{u}' = R_{z,\theta}(\vec{u})$ y $\vec{v}' = R_{z,\theta}(\vec{v})$.
Sus componentes son:
$$ \vec{u}' = (u_x c - u_y s, \ u_x s + u_y c, \ u_z) $$
$$ \vec{v}' = (v_x c - v_y s, \ v_x s + v_y c, \ v_z) $$
(donde $c=\cos\theta, s=\sin\theta$).

Calculamos la componente X de $\vec{u}' \times \vec{v}'$:
\begin{align*}
(\vec{u}' \times \vec{v}')_x &= u'_y v'_z - u'_z v'_y \\
&= (u_x s + u_y c)v_z - u_z(v_x s + v_y c) \\
&= u_x v_z s + u_y v_z c - u_z v_x s - u_z v_y c \\
&= c(u_y v_z - u_z v_y) - s(u_z v_x - u_x v_z)
\end{align*}
Este resultado coincide exactamente con la Ecuación (1).

Calculamos la componente Y de $\vec{u}' \times \vec{v}'$:
\begin{align*}
(\vec{u}' \times \vec{v}')_y &= u'_z v'_x - u'_x v'_z \\
&= u_z(v_x c - v_y s) - (u_x c - u_y s)v_z \\
&= u_z v_x c - u_z v_y s - u_x v_z c + u_y v_z s \\
&= s(u_y v_z - u_z v_y) + c(u_z v_x - u_x v_z)
\end{align*}
Este resultado coincide exactamente con la Ecuación (2).

Calculamos la componente Z de $\vec{u}' \times \vec{v}'$:
\begin{align*}
(\vec{u}' \times \vec{v}')_z &= u'_x v'_y - u'_y v'_x \\
&= (u_x c - u_y s)(v_x s + v_y c) - (u_x s + u_y c)(v_x c - v_y s)
\end{align*}
Desarrollando y simplificando:
\begin{align*}
&= (u_x v_x cs + u_x v_y c^2 - u_y v_x s^2 - u_y v_y sc) - (u_x v_x sc - u_x v_y s^2 + u_y v_x c^2 - u_y v_y cs) \\
&= u_x v_y(c^2 + s^2) - u_y v_x(s^2 + c^2) \\
&= u_x v_y - u_y v_x
\end{align*}
Este resultado coincide con la Ecuación (3).

\textbf{Conclusión:}
Dado que todas las componentes coinciden, queda demostrado que:
$$ R_{z,\theta}(\vec{u} \times \vec{v}) = R_{z,\theta}(\vec{u}) \times R_{z,\theta}(\vec{v}) $$
\end{solucion}

\begin{ejercicio}
    % \textbf{Problema 3.12:}

    Crea un script global (autoload) con una función llamada \texttt{gancho} (sin parámetros) que crea y devuelve un objeto de la clase \texttt{Mesh} con una polilínea azul como la de la figura (los ejes se han dibujado por claridad).

    Crea en tu proyecto un nodo 2D de tipo \texttt{MeshInstance2D} y en \texttt{\_ready} asígnale como malla (propiedad \texttt{mesh}) el objeto resultado de llamar a \texttt{gancho()}, ponle un color azul (propiedad \texttt{modulate}) y verifica que el gancho aparece en pantalla al ejecutar el proyecto.

    \begin{center}
    \begin{tikzpicture}[scale=1.5]
        % Rejilla y ejes
        \draw[step=1cm, gray!20, very thin] (-0.5,-0.5) grid (1.5, 2.5);
        \draw[->, black!60] (0,0) -- (1.5,0) node[right] {X+};
        \draw[->, black!60] (0,0) -- (0,2.5) node[above] {Y+};
        \foreach \x in {0,1} \draw (\x,1pt) -- (\x,-1pt) node[anchor=north] {\tiny \x};
        \foreach \y in {0,1,2} \draw (1pt,\y) -- (-1pt,\y) node[anchor=east] {\tiny \y};

        % El gancho
        \draw[blue, ultra thick] (0,0) -- (1,0) -- (1,1) -- (0,1) -- (0,2);
        
        % Puntos destacados (opcional para claridad)
        \fill[blue] (0,0) circle (1.5pt);
        \fill[blue] (1,0) circle (1.5pt);
        \fill[blue] (1,1) circle (1.5pt);
        \fill[blue] (0,1) circle (1.5pt);
        \fill[blue] (0,2) circle (1.5pt);
    \end{tikzpicture}
    \end{center}
\end{ejercicio}

\begin{solucion} Solución al problema 3.12:
    \lstinputlisting{../code/EjerciciosTeoria/problema_3_12.gd}
    Además, añadimos el codigo de gancho que se encuentra en el script global:
    \begin{lstlisting}[language=gdscript]
    func gancho() -> ArrayMesh:
        var vertices = PackedVector2Array([
            Vector2(0,0),
            Vector2(1,0),
            Vector2(1,1),
            Vector2(0,1),
            Vector2(0,2)
        ])

        var arrays = []
        arrays.resize(Mesh.ARRAY_MAX)
        arrays[Mesh.ARRAY_VERTEX] = vertices

        var mesh = ArrayMesh.new()
        mesh.add_surface_from_arrays(Mesh.PRIMITIVE_LINE_STRIP, arrays)
        return mesh
    \end{lstlisting}
\end{solucion}

\begin{ejercicio}
    %\textbf{Problema 3.13:}

    Crea un nodo 2D de tipo \texttt{Node2D} y llámalo \texttt{Gancho\_x4}. En \texttt{\_ready}, añádele cuatro nodos hijos de tipo \texttt{MeshInstance2D}, cada uno de ellos con un malla creada con la función \texttt{gancho} del problema anterior, pero con su \texttt{transform} modificada para que el objeto \texttt{Gancho\_x4} se vea como en la figura (la rejilla y los ejes en rojo se han dibujado por claridad).

    \begin{center}
    \begin{tikzpicture}[scale=1]
    % 1. Rejilla de fondo (ajustada a los límites de la imagen)
    \draw[step=0.5cm, gray!30, very thin] (-3.5,-1.5) grid (1.5, 3.5);
    \draw[step=1cm, black!40, thin] (-3.5,-1.5) grid (1.5, 3.5);

    % 2. Ejes rojos
    \draw[red, thick] (-3.5,0) -- (1.5,0);
    \draw[red, thick] (0,-1.5) -- (0,3.5);

    % 3. Etiquetas de los ejes
    \foreach \x in {-3,-2,-1,0,1} \node[anchor=north] at (\x,0) {\scriptsize \x};
    \foreach \y in {-1,0,1,2,3} \node[anchor=east] at (0,\y) {\scriptsize \y};

    % 4. El polígono azul continuo
    % He trazado las coordenadas siguiendo tu imagen, empezando en (0,0)
    \draw[blue, line width=2pt] 
        (0,0) -- (1,0) -- (1,1) -- (0,1) --   % Brazo derecho
        (0,3) -- (-1,3) -- (-1,2) --          % Brazo superior (llega a Y=3)
        (-3,2) -- (-3,1) -- (-2,1) --         % Brazo izquierdo (llega a X=-3)
        (-2,-1) -- (-1,-1) -- (-1,0) --       % Brazo inferior (llega a Y=-1)
        cycle;                                % Cierra la figura volviendo a (0,0)

    \end{tikzpicture}
    \end{center}
\end{ejercicio}

\begin{solucion} Solución al problema 3.13:
    \lstinputlisting{../code/EjerciciosTeoria/problema_3_13.gd}
\end{solucion}

\section{Sesión 4}

\begin{ejercicio}
Supongamos que queremos codificar una esfera de radio $1/2$ y centro en el origen de dos formas:

\begin{enumerate}
\item Por enumeración espacial, dividiendo el cubo que engloba a la esfera en celdas, de forma que haya $k$ celdas por lado del cubo, todas ellas son cubos de $1/k$ de ancho. Cada celda ocupa un bit de memoria (si su centro está en la esfera, se guarda un 1, en otro caso un 0).
\item Usando un modelo de fronteras (una malla indexada de triángulos), en el cual se usa una rejilla de triángulos y aristas que siguen los meridianos y paralelos, habiendo en cada meridiano y en cada paralelo un total de $k$ vértices (se guarda únicamente la tabla de vértices y la de triángulos).
\end{enumerate}

Asumiendo que un \texttt{float} y un \texttt{int} ocupan 4 bytes cada uno, contesta a estas cuestiones:

\begin{enumerate}
\item Expresa el tamaño de ambas representaciones en bytes como una función de k.
\item Suponiendo que $k=16$ calcula cuántos KB de memoria ocupa cada estructura.
\item Haz lo mismo asumiendo ahora que $k=1024$ (expresa los resultados en MB).
\item Compara los tamaños de ambas representaciones en ambos casos ($k=16$ y $k=1024$).
\end{enumerate}
\end{ejercicio}

\begin{solucion}
Para resolver este ejercicio, analizaremos detalladamente los requisitos de memoria de cada uno de los modelos propuestos, basándonos en la teoría de representación de modelos geométricos, específicamente la diferencia entre modelos de volúmenes (enumeración espacial) y modelos de fronteras (mallas de polígonos).

\begin{enumerate}
\item \emph{Expresión del tamaño en memoria como función de k.}


Analicemos primero el modelo por \emph{enumeración espacial}.

El espacio que engloba a la esfera de radio $r=1/2$ es un cubo de lado $L = 2r = 1$. Este cubo se discretiza en una rejilla tridimensional de $k$ celdas por lado. Por lo tanto, el número total de celdas (vóxeles) en el volumen es:
\[ N_{celdas} = k \times k \times k = k^3 \]

El enunciado especifica que cada celda ocupa exactamente 1 bit. Para obtener el tamaño en bytes, debemos dividir el número total de bits por 8 (dado que 1 byte = 8 bits).

\[ Mem_{enum}(k) = \frac{k^3}{8} \text{ bytes} \]

Analicemos ahora el modelo de fronteras mediante \emph{malla indexada de triángulos}.

Una malla indexada consta de dos estructuras de datos principales: la tabla de vértices y la tabla de triángulos (índices).

El enunciado indica que la malla se forma siguiendo meridianos y paralelos con $k$ vértices en cada uno. Esto sugiere una topología de rejilla rectangular de dimensiones $k \times k$ mapeada sobre la esfera. En consecuencia, el número de vértices $n_V$ es:
\[ n_V = k^2 \]

Para una malla cerrada y conexa que representa una esfera, topológicamente equivalente a una rejilla envolvente, el número de caras (triángulos) $n_T$ se aproxima al doble del número de vértices (según la característica de Euler para mallas triangulares cerradas donde $n_T \approx 2n_V$). Si consideramos una rejilla de $(k-1) \times (k-1)$ cuadriláteros, y cada cuadrilátero se divide en 2 triángulos, tendríamos $2(k-1)^2$ triángulos. Para valores grandes de $k$, podemos aproximar el número de triángulos como:
\[ n_T \approx 2k^2 \]

Calculamos ahora el uso de memoria para cada tabla:
\begin{enumerate}
    \item \emph{Tabla de vértices:} Cada vértice almacena 3 coordenadas $(x, y, z)$ de tipo \texttt{float}. Si cada \texttt{float} ocupa 4 bytes, el tamaño de un vértice es $3 \times 4 = 12$ bytes.
    \[ Mem_{vert} = 12 \times n_V = 12k^2 \text{ bytes} \]
    
    \item \emph{Tabla de triángulos:} Cada triángulo almacena 3 índices de tipo \texttt{int}. Si cada \texttt{int} ocupa 4 bytes, el tamaño de un triángulo es $3 \times 4 = 12$ bytes.
    \[ Mem_{tri} = 12 \times n_T \approx 12 \times (2k^2) = 24k^2 \text{ bytes} \]
\end{enumerate}

El tamaño total de la malla indexada es la suma de ambas tablas:
\[ Mem_{malla}(k) = 12k^2 + 24k^2 = 36k^2 \text{ bytes} \]

\item \emph{Cálculo de memoria para $k=16$ (en KB).}

Sustituimos $k=16$ en las funciones obtenidas:

Para la \emph{enumeración espacial}:
\[ Mem_{enum}(16) = \frac{16^3}{8} = \frac{4096}{8} = 512 \text{ bytes} \]
Convirtiendo a Kilobytes ($1 \text{ KB} = 1024 \text{ bytes}$):
\[ Mem_{enum}(16) = \frac{512}{1024} = 0.5 \text{ KB} \]

Para la \emph{malla indexada}:
\[ Mem_{malla}(16) = 36 \times 16^2 = 36 \times 256 = 9216 \text{ bytes} \]
Convirtiendo a Kilobytes:
\[ Mem_{malla}(16) = \frac{9216}{1024} = 9 \text{ KB} \]

\item \emph{Cálculo de memoria para $k=1024$ (en MB).}

Sustituimos $k=1024$ en las funciones. Nótese que $1024 = 2^{10}$.

Para la \emph{enumeración espacial}:
\[ Mem_{enum}(1024) = \frac{(2^{10})^3}{2^3} = \frac{2^{30}}{2^3} = 2^{27} \text{ bytes} \]
Sabemos que $1 \text{ MB} = 1024^2 \text{ bytes} = 2^{20} \text{ bytes}$.
\[ Mem_{enum}(1024) = \frac{2^{27}}{2^{20}} = 2^7 = 128 \text{ MB} \]

Para la \emph{malla indexada}:
\[ Mem_{malla}(1024) = 36 \times (1024)^2 = 36 \times 2^{20} \text{ bytes} \]
Convirtiendo a Megabytes:
\[ Mem_{malla}(1024) = 36 \text{ MB} \]

\item \emph{Comparación de tamaños.}

Los resultados obtenidos ilustran la diferencia fundamental en la complejidad espacial entre los modelos volumétricos y los de frontera.

\begin{enumerate}
    \item \emph{Caso $k=16$ (Baja resolución):}
    La enumeración espacial ocupa \emph{menos} memoria ($0.5$ KB) que la malla indexada ($9$ KB). Esto se debe a que, para resoluciones muy bajas, el coste de almacenar coordenadas e índices explícitos (36 bytes por elemento efectivo) supera el coste de almacenar simplemente 1 bit por celda, dado que el volumen total ($k^3$) aún no ha crecido lo suficiente para dominar la expresión.
    
    \item \emph{Caso $k=1024$ (Alta resolución):}
    La enumeración espacial ocupa significativamente \emph{más} memoria ($128$ MB) que la malla indexada ($36$ MB). Aquí se observa la naturaleza cúbica $O(k^3)$ de la enumeración espacial frente a la naturaleza cuadrática $O(k^2)$ del modelo de fronteras. Al aumentar la resolución, el número de celdas interiores (volumen) crece mucho más rápido que el número de vértices necesarios para representar la superficie (área).
\end{enumerate}

\emph{Conclusión:} La enumeración espacial es extremadamente ineficiente en memoria para altas resoluciones, mientras que los modelos de frontera (mallas) son mucho más eficientes para representar objetos sólidos mediante su superficie, especialmente a medida que aumenta la precisión requerida ($k$).



\end{enumerate}
\end{solucion}

\begin{ejercicio}
Considera una malla indexada (tabla de vértices y tabla de caras, esta última con índices de vértices) con una topología de rejilla rectangular. La rejilla está compuesta por $n$ columnas de pares de triángulos y $m$ filas. Esto implica que la estructura tiene $n+1$ columnas de vértices y $m+1$ filas de vértices, con $n, m > 0$.

La figura siguiente ilustra un esquema simplificado de dicha topología (donde los puntos azules representan los vértices y las líneas las aristas que forman los triángulos):

\begin{center}
\begin{tikzpicture}[scale=0.8]
% Definición de parámetros para el dibujo
\def\cols{8}
\def\rows{4}
% Dibujar la malla de triángulos
\foreach \y in {0,...,\rows} {
    \foreach \x in {0,...,\cols} {
        % Dibujar vértices
        \fill[blue!70!black] (\x,\y) circle (2pt);
        
        % Dibujar aristas horizontales y verticales (si no estamos en el borde final)
        \ifnum\x<\cols
            \draw[thin, gray] (\x,\y) -- (\x+1,\y);
            % Diagonales y cierres de triángulos
            \ifnum\y<\rows
                \draw[thin, gray] (\x,\y) -- (\x,\y+1);
                \draw[thin, gray] (\x,\y) -- (\x+1,\y+1); % Diagonal
            \fi
        \fi
    }
    % Cerrar el borde derecho vertical
    \ifnum\y<\rows
        \draw[thin, gray] (\cols,\y) -- (\cols,\y+1);
    \fi
}

% Etiquetas
\node[anchor=north] at (\cols/2, -0.5) {Ancho ($n$ columnas de quads)};
\node[anchor=east, rotate=90] at (-0.5, \rows/2) {Alto ($m$ filas)};
\end{tikzpicture}
\end{center}

En relación a este tipo de mallas, responde a las siguientes cuestiones:
\begin{enumerate}[label=(\alph*)]
\item Supongamos que un \texttt{float} ocupa 4 bytes y un \texttt{int} ocupa también 4 bytes. ¿Qué tamaño en memoria ocupa la malla completa en bytes? Ten en cuenta únicamente el tamaño de la tabla de vértices y la tabla de triángulos. Expresa el tamaño como una función de $m$ y $n$.
\item Calcula el tamaño exacto en KiloBytes (KB) suponiendo que $m = n = 128$.
\item Supongamos que $m$ y $n$ son ambos grandes (es decir, asumimos que términos como $1/n$ y $1/m$ son despreciables frente a 1). Deduce qué relación aproximada existe entre el número de caras ($n_C$) y el número de vértices ($n_V$) en este tipo de mallas.
\end{enumerate}
\end{ejercicio}

\begin{solucion}
Para resolver este problema, analizaremos por separado el consumo de memoria de la geometría (tabla de vértices) y de la topología (tabla de triángulos).

\begin{enumerate}[label=(\alph*)]
\item \textbf{Cálculo de la función de memoria $Mem(m, n)$ en bytes}

Primero determinamos la cantidad de elementos:
\begin{itemize}
    \item \textbf{Número de vértices ($n_V$):} 
    La rejilla tiene $m$ filas de celdas y $n$ columnas de celdas. Los vértices se sitúan en las intersecciones.
    \[ \text{Filas de vértices} = m + 1 \]
    \[ \text{Columnas de vértices} = n + 1 \]
    \[ n_V = (m+1)(n+1) \]
    
    \item \textbf{Número de caras/triángulos ($n_C$):}
    Cada celda de la rejilla (formada por la intersección de una fila y una columna) es un cuadrilátero dividido en 2 triángulos.
    \[ \text{Número de celdas} = m \times n \]
    \[ n_C = 2 \times (m \times n) = 2mn \]
\end{itemize}

Ahora calculamos el uso de memoria sabiendo que 1 \texttt{float} = 4 bytes y 1 \texttt{int} = 4 bytes:

\begin{itemize}
    \item \textbf{Memoria de la Tabla de Vértices ($M_V$):}
    Cada vértice almacena 3 coordenadas ($x, y, z$) de tipo \texttt{float}.
    \[ M_V = n_V \times 3 \times 4 \text{ bytes} = 12(m+1)(n+1) \text{ bytes} \]
    
    \item \textbf{Memoria de la Tabla de Triángulos ($M_T$):}
    Cada triángulo almacena 3 índices de vértices ($i, j, k$) de tipo \texttt{int}.
    \[ M_T = n_C \times 3 \times 4 \text{ bytes} = 12 \times (2mn) \text{ bytes} = 24mn \text{ bytes} \]
\end{itemize}

\textbf{Memoria Total ($Mem$):}
\[ Mem(m, n) = M_V + M_T \]
\[ Mem(m, n) = 12(mn + m + n + 1) + 24mn \]
Agrupando términos semejantes:
\[ Mem(m, n) = 12mn + 12m + 12n + 12 + 24mn \]
\[ \boxed{Mem(m, n) = 36mn + 12m + 12n + 12 \text{ bytes}} \]

\item \textbf{Cálculo para $m = n = 128$}

Sustituimos $m$ y $n$ por 128 en la fórmula obtenida:
\[ Mem(128, 128) = 36(128 \times 128) + 12(128) + 12(128) + 12 \]
\[ Mem(128, 128) = 36(16384) + 1536 + 1536 + 12 \]
\[ Mem(128, 128) = 589824 + 3084 \]
\[ Mem(128, 128) = 592908 \text{ bytes} \]

Para convertir a KiloBytes (KB), dividimos por 1024:
\[ \text{Memoria en KB} = \frac{592908}{1024} \approx 579.01 \text{ KB} \]

\textbf{Resultado:} Aproximadamente \textbf{579 KB}.

\item \textbf{Relación asintótica entre $n_C$ y $n_V$}

Partimos de las expresiones deducidas en el apartado (a):
\[ n_V = (m+1)(n+1) = mn + m + n + 1 \]
\[ n_C = 2mn \]

Si asumimos que $m$ y $n$ son grandes, los términos lineales ($m, n$) y el término constante ($1$) son despreciables frente al término cuadrático ($mn$). Matemáticamente:
\[ \lim_{m,n \to \infty} \frac{n_V}{mn} = \lim_{m,n \to \infty} \frac{mn + m + n + 1}{mn} = 1 \]
Por tanto, para valores grandes, podemos aproximar:
\[ n_V \approx mn \]

\textit{Nota: se divide por el término de mayor grado porque de esta manera, en matemáticas, vemos como se comporta en el infinito, otra opción es el mismo límite de nc/nv.}

Calculamos la relación (ratio) entre el número de caras y el número de vértices:
\[ \frac{n_C}{n_V} \approx \frac{2mn}{mn} = 2 \]

\textbf{Conclusión:} En mallas cerradas o mallas de rejilla densas (donde los efectos de borde son insignificantes), \textbf{el número de caras (triángulos) es aproximadamente el doble que el número de vértices}:
\[ n_C \approx 2 n_V \]
\end{enumerate}
\end{solucion}

\begin{ejercicio}

Imagina de nuevo una malla con topología de rejilla, en la cual hay $n$ columnas de pares de triángulos y $m$ filas. Supongamos que usamos una representación como \textbf{tiras de triángulos} (Triangle Strips), de forma que cada fila de triángulos (con $2n$ triángulos) se almacena en una tira independiente, habiendo un total de $m$ tiras.

La estructura de datos consta de una tabla de punteros a tiras (que tiene un entero para el número de tiras y $m$ punteros, donde cada puntero ocupa 8 bytes) y los arrays de coordenadas de las tiras. Asume que las coordenadas son de tipo \texttt{float} (4 bytes) y que no se usan índices (las coordenadas se almacenan explícitamente en el orden de la tira).

Responde a las siguientes cuestiones:
\begin{enumerate}[label=(\alph*)]
    \item Indica qué cantidad de memoria ocupa esta representación en dos casos:
    \begin{enumerate}[label=(\arabic*)]
        \item Como función de $n$ y $m$, en bytes.
        \item Suponiendo $m=n=128$, en KB.
    \end{enumerate}
    \item Para $m$ y $n$ grandes (asumiendo que los términos lineales son despreciables frente a los cuadráticos), describe qué relación hay entre el tamaño en memoria de la malla indexada (Problema 4.2) y el tamaño de la malla almacenada como tiras de triángulos.
    \item Si suponemos que la transformación de cada vértice se hace en un tiempo constante igual a la unidad, describe qué relación hay entre los tiempos de procesamiento de vértices para esta malla cuando se representa como una malla indexada y como tiras de triángulos.
\end{enumerate}
\end{ejercicio}

\begin{solucion}
% \section*{Análisis de la Estructura de Tiras}

Para resolver este ejercicio, primero debemos determinar cuántos vértices se almacenan explícitamente en una tira de triángulos que representa una fila de la rejilla.

\begin{itemize}
    \item Una tira de triángulos que contiene $k$ triángulos requiere $k+2$ vértices. Básicamente, sabemos que cada nuevo triángulo en la tira comparte dos vértices con el triángulo anterior, si para 2 triangulos necesitamos 4 vértices, por inducción (3 vértices $\times$ (k-1) restantes $\times$ 1 vértice que añadimos) se llega a la fórmula $k + 2$.
    \item En la rejilla descrita, cada fila contiene $n$ celdas cuadradas (pares de triángulos). Por lo tanto, el número de triángulos por fila (por tira) es $k = 2n$.
    \item El número de vértices almacenados por cada tira es:
    $$
    V_{tira} = (2n) + 2 = 2n + 2
    $$
    \item Cada vértice consta de 3 coordenadas ($x, y, z$) de tipo \texttt{float} (4 bytes cada uno). El tamaño de un vértice es:
    $$
    B_{vertice} = 3 \times 4 \text{ bytes} = 12 \text{ bytes}
    $$
\end{itemize}

\begin{center}
\begin{tikzpicture}[scale=1.5]
% Draw grid points and strip path
\foreach \x in {0,1,2,3} {\foreach \y in {0,1} {\filldraw (\x,\y) circle (1.5pt);}}
% Draw strip lines (zig-zag)
\draw[blue, thick, ->] (0,0) -- (0,1) -- (1,0) -- (1,1) -- (2,0) -- (2,1) -- (3,0) -- (3,1);
\node[below] at (1.5, -0.2) {Ejemplo de una tira ($n=3$ quads, $2n=6$ triángulos, $2n+2=8$ vértices)};
\end{tikzpicture}
\end{center}

\underline{(a) Cálculo de Memoria}

\textbf{(1) Función de $n$ y $m$ en bytes}

El tamaño total $M_{total}$ se compone del tamaño de los datos de las tiras y la sobrecarga de la estructura de punteros.

\begin{enumerate}
    \item \textbf{Memoria de los vértices:} Hay $m$ tiras.
    $$
    M_{geom} = m \times (2n + 2) \text{ vértices} \times 12 \text{ bytes/vértice}
    $$
    $$
    M_{geom} = 12m(2n + 2) = 24nm + 24m \text{ bytes}
    $$
    \item \textbf{Memoria de la tabla de punteros:} Contiene 1 entero (4 bytes) y $m$ punteros (8 bytes c/u\footnote{cada uno}).
    $$
    M_{estructura} = 4 + 8m \text{ bytes}
    $$
    \item \textbf{Memoria Total:}
    $$
    M_{total}(n, m) = (24nm + 24m) + (8m + 4)
    $$
    $$
    \boxed{M_{total}(n, m) = 24nm + 32m + 4 \text{ bytes}}
    $$
\end{enumerate}

\textbf{(2) Cálculo para $m=n=128$}

Sustituimos los valores en la fórmula obtenida:
$$
M_{total}(128, 128) = 24(128 \times 128) + 32(128) + 4
$$
$$
M_{total} = 24(16384) + 4096 + 4
$$
$$
M_{total} = 393216 + 4096 + 4 = 397316 \text{ bytes}
$$

Para convertir a Kilobytes (asumiendo $1 \text{ KB} = 1024 \text{ bytes}$):
$$
M_{KB} = \frac{397316}{1024} \approx \boxed{388.00 \text{ KB}}
$$



\underline{(b) Relación de tamaño con Malla Indexada}

Para $n, m$ grandes, solo consideramos los términos de mayor orden ($nm$).

\textbf{1. Tamaño Malla Indexada (del Problema 4.2):}
\begin{itemize}
    \item Vértices únicos: $\approx nm$. Tamaño: $nm \times 12$ bytes.
    \item Triángulos: $\approx 2nm$. Índices: $2nm \times 3 \text{ índices} \times 4 \text{ bytes} = 24nm$ bytes. El cálculo de los índices ha sido número de triángulos por 3 índices por triángulo por 4 bytes por índice.
    \item Total Indexada: $12nm + 24nm = \mathbf{36nm}$ bytes.
\end{itemize}

\textbf{2. Tamaño Tiras de Triángulos (obtenido en a):}
\begin{itemize}
    \item Total Tiras: $\mathbf{24nm}$ bytes.
\end{itemize}

\textbf{Comparación:}

Calculamos la relación (ratio) entre ambas representaciones:
$$
\frac{\text{Memoria Tiras}}{\text{Memoria Indexada}} \approx \frac{24nm}{36nm} = \frac{2}{3}
$$

\textbf{Conclusión:}

La representación mediante tiras de triángulos ocupa aproximadamente \textbf{el 66.6\% (dos tercios)} de la memoria que ocupa la malla indexada para esta topología de rejilla. Esto se debe a que, aunque las tiras duplican los vértices compartidos entre filas adyacentes, evitan el coste de almacenar 3 enteros por cada triángulo, que es más costoso que almacenar coordenadas repetidas en este escenario específico.



\underline{(c) Comparación de tiempos de procesamiento}

El tiempo de procesamiento de vértices ($T_{proc}$) en la GPU depende del número de veces que se debe ejecutar el \textit{Vertex Shader}.

\textbf{1. Malla Indexada:}

Gracias al \textit{Post-Transform Cache} de la GPU, los vértices indexados suelen procesarse una sola vez por cada vértice único (idealmente).
$$
V_{unicos} \approx nm \implies T_{index} \propto nm
$$

\textbf{2. Tiras de Triángulos (No Indexadas):}

En la implementación descrita (arrays de arrays), los vértices se envían explícitamente por cada tira. Los vértices que se encuentran en la frontera entre la fila $i$ y la fila $i+1$ están duplicados en memoria (aparecen en la tira $i$ y en la tira $i+1$). La GPU no sabe que son el mismo vértice geométrico y debe procesarlos dos veces.
$$
V_{tiras} = m(2n+2) \approx 2nm \implies T_{tiras} \propto 2nm
$$

\textbf{Conclusión:}
$$
\frac{T_{tiras}}{T_{index}} \approx \frac{2nm}{nm} = 2
$$

El tiempo de procesamiento usando tiras de triángulos independientes (no indexadas) es aproximadamente \textbf{el doble} que usando una malla indexada. Aunque las tiras ahorran memoria de almacenamiento en disco/RAM en este caso, son menos eficientes computacionalmente porque obligan a la GPU a transformar los mismos vértices frontera múltiples veces.
\end{solucion}

\begin{ejercicio}

Supongamos una malla cerrada, simplemente conexa (topológicamente equivalente a una esfera), cuyas caras son triángulos y cuyas aristas son todas adyacentes a exactamente dos caras (la malla es un poliedro simplemente conexo de caras triangulares).

Considera el número de vértices $n_V$, el número de aristas $n_A$ y el número de caras $n_C$ en este tipo de mallas.

Demuestra que cualquiera de esos números determina a los otros dos, en concreto, demuestra que se cumplen estas dos igualdades:
$$n_A = 3(n_V - 2)$$
$$n_C = 2(n_V - 2)$$
\end{ejercicio}

\begin{solucion}
Para demostrar las igualdades propuestas, utilizaremos dos propiedades fundamentales de la topología de superficies cerradas y de las mallas triangulares. Procederemos paso a paso estableciendo un sistema de ecuaciones.

\begin{enumerate}
\item \textbf{Aplicación de la Fórmula de Euler-Poincaré:}

Dado que el enunciado especifica que la malla es cerrada y topológicamente equivalente a una esfera (género $g=0$), se cumple la característica de Euler para poliedros convexos:
\begin{equation}
n_V - n_A + n_C = 2 \label{eq:euler}
\end{equation}
Donde:
\begin{itemize}
\item $n_V$: Número de vértices.
\item $n_A$: Número de aristas.
\item $n_C$: Número de caras.
\end{itemize}

\item \textbf{Relación de adyacencia Caras-Aristas:}

En una malla compuesta exclusivamente por triángulos, cada cara tiene exactamente 3 aristas. Además, al ser una variedad cerrada (manifold), cada arista es compartida exactamente por 2 caras.

Podemos contar el número total de ''lados'' de los triángulos de dos formas:
\begin{itemize}
    \item Multiplicando el número de caras por 3: $3 \cdot n_C$.
    \item Multiplicando el número de aristas por 2 (ya que cada arista cuenta para dos caras): $2 \cdot n_A$.
\end{itemize}
Igualando ambas cantidades obtenemos la segunda ecuación fundamental:
\begin{equation}
    3n_C = 2n_A \implies n_C = \frac{2}{3}n_A \quad \text{o bien} \quad n_A = \frac{3}{2}n_C \label{eq:adyacencia}
\end{equation}

\item \textbf{Demostración de $n_C = 2(n_V - 2)$:}

Sustituimos $n_A$ en la Ecuación (\ref{eq:euler}) utilizando la relación obtenida en (\ref{eq:adyacencia}) ($n_A = \frac{3}{2}n_C$):
$$n_V - \left(\frac{3}{2}n_C\right) + n_C = 2$$
Multiplicamos toda la ecuación por 2 para eliminar la fracción:
$$2n_V - 3n_C + 2n_C = 4$$
Simplificamos los términos de $n_C$:
$$2n_V - n_C = 4$$
Despejamos $n_C$:
$$n_C = 2n_V - 4$$
Factorizamos el 2:
$$\boxed{n_C = 2(n_V - 2)}$$
\textit{Q.E.D. (Queda demostrado que el número de caras es aproximadamente el doble que el de vértices).}

\item \textbf{Demostración de $n_A = 3(n_V - 2)$:}

Partimos de nuevo de la Ecuación (\ref{eq:euler}), pero esta vez sustituimos $n_C$ despejándolo de (\ref{eq:adyacencia}) como $n_C = \frac{2}{3}n_A$:
$$n_V - n_A + \left(\frac{2}{3}n_A\right) = 2$$
Multiplicamos toda la ecuación por 3 para eliminar la fracción:
$$3n_V - 3n_A + 2n_A = 6$$
Simplificamos los términos de $n_A$:
$$3n_V - n_A = 6$$
Despejamos $n_A$:
$$n_A = 3n_V - 6$$
Factorizamos el 3:
$$\boxed{n_A = 3(n_V - 2)}$$
\end{enumerate}

\begin{center}
\begin{tikzpicture}[scale=2, line join=round, line cap=round]
% Visualización de un Tetraedro (Malla triangular cerrada más simple)
% nV=4, nC=4, nA=6.
% Comprobación: nC = 2(4-2) = 4. nA = 3(4-2) = 6.
\coordinate (A) at (0,1);
\coordinate (B) at (-0.86,-0.5);
\coordinate (C) at (0.86,-0.5);
\coordinate (D) at (0, -0.2); % Vértice ''oculto/central'' proyectado

% Caras
\fill[opacity=0.1, blue] (A) -- (B) -- (C) -- cycle;
\fill[opacity=0.2, blue] (A) -- (B) -- (D) -- cycle;
\fill[opacity=0.2, blue] (B) -- (C) -- (D) -- cycle;

% Aristas
\draw[thick] (A) -- (B);
\draw[thick] (B) -- (C);
\draw[thick] (C) -- (A);
\draw[dashed] (A) -- (D);
\draw[dashed] (B) -- (D);
\draw[dashed] (C) -- (D);

% Etiquetas
\foreach \p in {A,B,C,D} \filldraw [black] (\p) circle (1pt);
\node[above] at (A) {$v_0$};
\node[left] at (B) {$v_1$};
\node[right] at (C) {$v_2$};
\node[below] at (D) {$v_3$};

\node[below=1cm] at (0,-0.5) {Ejemplo: Tetraedro ($n_V=4, n_A=6, n_C=4$)};
\end{tikzpicture}
\end{center}
\end{solucion}


\begin{ejercicio}

En una malla indexada, queremos añadir a la estructura de datos una tabla de aristas. Será un vector \texttt{ari}, que en cada entrada tendrá una tupla de tipo \texttt{Vector2i} (contiene dos \texttt{int}) con los índices en la tabla de vértices de los dos vértices en los extremos de la arista. El orden en el que aparecen los vértices en una arista es indiferente, pero cada arista debe aparecer una sola vez.

Escribe el código de una función GDScript para crear y calcular la tabla de aristas a partir de la tabla de triángulos. Intenta encontrar una solución con la mínima complejidad en tiempo y memoria posible. Suponer que el número de vértices adyacentes a uno cualquiera de ellos es como mucho un valor constante $k > 0$, valor que no depende del número total de vértices, que llamamos $n$.

Considerar dos casos:
\begin{enumerate}[label=(\alph*)]
\item Los triángulos se dan con orientación no coherente: esto quiere decir que si un triángulo está formado por los vértices $i, j, k$, estos tres índices pueden aparecer en cualquier orden en la correspondiente entrada de la tabla de triángulos. Además, no sabemos si la malla es cerrada o no.
\item Los triángulos se dan con orientación coherente: esto quiere decir que si dos triángulos comparten una arista entre los vértices $i$ y $j$, entonces en uno de los triángulos la arista aparece como $(i, j)$ y en el otro aparece como $(j, i)$. Además, asumimos que la malla es cerrada, es decir, que cada arista es compartida por exactamente dos triángulos.
\end{enumerate}
\end{ejercicio}

\begin{solucion}
Para resolver este problema, debemos iterar sobre la tabla de triángulos y extraer las aristas potenciales. La diferencia fundamental entre los dos casos radica en cómo garantizamos la unicidad de las aristas (evitar duplicados) de manera eficiente.

\underline{\textbf{Caso (a): Orientación no coherente y malla general}}

En este escenario, no podemos predecir el orden de los índices ni cuántas veces aparece una arista (podría ser 1 si es frontera, o 2 si es interna, o más si la malla no es ''manifold'').

\textbf{Estrategia:}
\begin{enumerate}
\item Recorremos cada triángulo y extraemos sus 3 aristas: $(v_0, v_1)$, $(v_1, v_2)$ y $(v_2, v_0)$.
\item Para identificar una arista de forma única sin importar el orden (es decir, que la arista $i-j$ sea igual a $j-i$), ordenamos los índices de cada par: guardamos siempre $(\min(i,j), \max(i,j))$.
\item Usamos una estructura de datos tipo \textit{Set} (Conjunto) o un Diccionario para almacenar las aristas encontradas. Esto elimina duplicados automáticamente con una complejidad promedio de $O(1)$ por inserción.
\end{enumerate}

\textbf{Código GDScript:}
\begin{lstlisting}
func calcular_aristas_caso_a(triangulos: Array[Vector3i]) -> Array[Vector2i]:
    var aristas_unicas = {} # Usamos un diccionario como Set
    for t in triangulos:
        # Extraemos los 3 pares de vértices
        var pares = [
            Vector2i(t[0], t[1]),
            Vector2i(t[1], t[2]),
            Vector2i(t[2], t[0])
        ]
        for par in pares:
            # Normalizamos la arista: (menor, mayor)
            var a = par.x
            var b = par.y
            var key: Vector2i
            if a < b:
                key = Vector2i(a, b)
            else:
                key = Vector2i(b, a)
            # Insertamos en el diccionario (la clave evita duplicados)
            aristas_unicas[key] = true
    # Convertimos las claves del diccionario a un Array
    var ari: Array[Vector2i] = []
    for key in aristas_unicas.keys():
        ari.append(key)
    return ari
\end{lstlisting}

\textbf{Complejidad:}
\begin{itemize}
\item Tiempo: $O(N_t)$, donde $N_t$ es el número de triángulos (asumiendo inserción en hash map constante).
\item Memoria: $O(N_a)$, donde $N_a$ es el número de aristas únicas, necesario para el diccionario auxiliar.
\end{itemize}

\underline{\textbf{Caso (b): Orientación coherente y malla cerrada}}

En este escenario, tenemos una propiedad topológica fuerte: cada arista interna es compartida por exactamente dos triángulos. Debido a la orientación coherente, si la arista conecta los vértices A y B:
\begin{itemize}
\item En el Triángulo 1 aparecerá como secuencia $\dots \rightarrow A \rightarrow B \rightarrow \dots$
\item En el Triángulo 2 aparecerá como secuencia $\dots \rightarrow B \rightarrow A \rightarrow \dots$
\end{itemize}

\textbf{Estrategia:}
Para evitar duplicados sin usar memoria extra (diccionarios), podemos aplicar una regla de selección simple: \textbf{Solo añadimos la arista si el índice de origen es menor que el índice de destino ($i < j$)}.
\begin{itemize}
\item Cuando procesemos el par $(i, j)$ donde $i < j$, lo guardamos.
\item Cuando procesemos el par $(j, i)$ (que existirá obligatoriamente en el triángulo vecino), como $j > i$, lo ignoramos.
\end{itemize}
Esto garantiza que cada arista se añade exactamente una vez.

\textbf{Código GDScript:}
\begin{lstlisting}
func calcular_aristas_caso_b(triangulos: Array[Vector3i]) -> Array[Vector2i]:
    var ari: Array[Vector2i] = []
    for t in triangulos:
        # Definimos los 3 pares tal cual aparecen en el orden del triángulo
        # Arista 0-1
        if t[0] < t[1]:
            ari.append(Vector2i(t[0], t[1]))
        # Arista 1-2
        if t[1] < t[2]:
            ari.append(Vector2i(t[1], t[2]))
        # Arista 2-0
        if t[2] < t[0]:
            ari.append(Vector2i(t[2], t[0]))
    return ari
\end{lstlisting}

\textbf{Complejidad:}
\begin{itemize}
\item Tiempo: $O(N_t)$. Es extremadamente rápido porque solo implica comparaciones de enteros.
\item Memoria: $O(1)$ de memoria auxiliar (no necesitamos estructuras intermedias como diccionarios, escribimos directamente en el resultado).
\end{itemize}
\end{solucion}

\begin{ejercicio}

Escribe el pseudo-código de la función para calcular el área total de una malla indexada de triángulos, a partir de la tabla de vértices y de la tabla de triángulos.

Será una función GDScript que acepta ambas tablas:
\begin{itemize}
\item \texttt{vertices}: un array de tipo \texttt{Vector3} que contiene las posiciones espaciales.
\item \texttt{triangulos}: un array de tipo \texttt{Vector3i}, donde cada elemento contiene los tres índices enteros que forman una cara.
\end{itemize}
La función debe devolver el área total como un valor de punto flotante (\texttt{float}).
\end{ejercicio}

\begin{solucion}
Para resolver este problema, debemos basarnos en la geometría vectorial. El área de cualquier polígono complejo en 3D (la malla) es la suma de las áreas de sus primitivas individuales (los triángulos).

\subsubsection*{Fundamento Matemático}
El área de un triángulo en el espacio 3D definido por tres puntos $P_0, P_1, P_2$ se puede calcular utilizando el \textbf{producto vectorial} (o producto cruz).

\begin{enumerate}
\item Definimos dos vectores que representen dos lados del triángulo partiendo de un vértice común, por ejemplo $P_0$:
$$\vec{u} = P_1 - P_0$$
$$\vec{v} = P_2 - P_0$$
\item El producto vectorial $\vec{w} = \vec{u} \times \vec{v}$ genera un vector perpendicular al plano del triángulo.
\item La magnitud (o longitud) de este vector resultante, $||\vec{w}||$, es igual al área del \textbf{paralelogramo} formado por los vectores $\vec{u}$ y $\vec{v}$.
\item Dado que un triángulo es la mitad de un paralelogramo, el área del triángulo es la mitad de dicha magnitud:
$$\text{Área}_{tri} = \frac{1}{2} ||\vec{u} \times \vec{v}||$$
\end{enumerate}

\begin{center}
\begin{tikzpicture}[scale=2]
% Coordenadas
\coordinate (P0) at (0,0);
\coordinate (P1) at (2,0.5);
\coordinate (P2) at (0.5, 1.5);
% Relleno triángulo
\fill[blue!10] (P0) -- (P1) -- (P2) -- cycle;

% Vectores
\draw[->, thick, blue] (P0) -- (P1) node[midway, below right] {$\vec{u} = P_1 - P_0$};
\draw[->, thick, red] (P0) -- (P2) node[midway, left] {$\vec{v} = P_2 - P_0$};

% Vértices
\filldraw (P0) circle (1pt) node[below left] {$P_0$};
\filldraw (P1) circle (1pt) node[right] {$P_1$};
\filldraw (P2) circle (1pt) node[above] {$P_2$};

% Paralelogramo fantasma
\coordinate (P3) at (2.5, 2.0);
\draw[dashed, gray] (P1) -- (P3) -- (P2);

\node at (2.5, 1) {\small $||\vec{u} \times \vec{v}|| = \text{Área Paralelogramo}$};
\end{tikzpicture}
\end{center}

\subsubsection*{Implementación en GDScript}
El algoritmo consiste en iterar sobre la tabla de triángulos, recuperar las coordenadas de los vértices usando los índices, calcular el área de cada triángulo individual y acumularla en una variable total.

\begin{lstlisting}
func calcular_area_malla(vertices: Array[Vector3], triangulos: Array[Vector3i]) -> float:
    var area_total: float = 0.0
    for t in triangulos:
        var p0: Vector3 = vertices[t[0]]
        var p1: Vector3 = vertices[t[1]]
        var p2: Vector3 = vertices[t[2]]
        var u: Vector3 = p1 - p0
        var v: Vector3 = p2 - p0
        var vector_area: Vector3 = u.cross(v)
        var area_triangulo: float = vector_area.length() * 0.5
        area_total += area_triangulo
    return area_total
\end{lstlisting}

\textbf{Análisis de complejidad:} Si $N_t$ es el número de triángulos (longitud del array \texttt{triangulos}), la complejidad temporal es $O(N_t)$, ya que realizamos un número constante de operaciones matemáticas (restas y producto vectorial) por cada cara de la malla.
\end{solucion}


\section{Sesión 5}
\begin{ejercicio}
    Implementa un proyecto cuya escena principal tenga un nodo de tipo \texttt{Node2D} con varios nodos hijos, que formen la figura con un cuadrado de lado 2, centrado en el origen, y con un triángulo inscrito.

    El cuadrado debe estar relleno de azul claro, el triángulo de blanco, y las aristas deben verse de color azul oscuro.

    \begin{center}
    \begin{tikzpicture}
        % Cuadrado de lado 2 centrado en (0,0) -> de -1,-1 a 1,1
        \fill[blue!20] (-1,-1) rectangle (1,1);
        \draw[blue!80!black, ultra thick] (-1,-1) rectangle (1,1);
        
        % Triángulo inscrito (aproximado según la figura del pdf)
        % Base en la parte inferior, pico arriba pero no centrado completamente o quizás sí.
        % En la figura original el triángulo es isosceles estrecho desplazado a la izquierda.
        % Simulamos la apariencia visual del problema 5.1 (figura página 58)
        \coordinate (A) at (-0.5, -0.8);
        \coordinate (B) at (0.0, 0.2);
        \coordinate (C) at (0.5, -0.8);
        % Ajustando para que parezca el de la diapositiva (triángulo blanco borde azul)
        \fill[white] (-0.6, -0.8) -- (-0.2, 0.5) -- (0.2, -0.8) -- cycle;
        \draw[blue!80!black, ultra thick] (-0.6, -0.8) -- (-0.2, 0.5) -- (0.2, -0.8) -- cycle;
    \end{tikzpicture}
    \end{center}
\end{ejercicio}

\begin{solucion} Solución al problema 5.1:
    \lstinputlisting{../code/EjerciciosTeoria/problema_5_1.gd}
\end{solucion}

\begin{ejercicio}
    % \textbf{Proyecto con dos escenas}

    Crea un proyecto Godot con una escena principal con un nodo raíz compuesto. Ese nodo tendrá tres hijos, cada uno es una instancia de la escena del problema anterior, pero con una transformación distinta.

    \begin{center}
    % \begin{tikzpicture}[line width=1.5pt, blue, fill=blue!20]
    %     % Definición de coordenadas base
    %     % Cuadrado de la izquierda (1x1)
    %     \draw [fill] (0,0) rectangle (1.5,1.5);
    %     \draw [white, fill=white] (0.15,0.2) -- (0.45,0.2) -- (0.35,1) -- cycle;

    %     % Rombo central (Cuadrado rotado 45 grados)
    %     % El vértice izquierdo del rombo toca el vértice superior derecho del primer cuadrado (1.5, 1.5)
    %     % El lado del rombo es aproximadamente 2.12 para que sus extremos conecten bien
    %     \begin{scope}[shift={(3, 1.5)}, rotate=45]
    %         \draw [fill] (-1.06,-1.06) rectangle (1.06,1.06);
    %         % Triángulo apuntando hacia arriba dentro del rombo
    %         \draw [white, fill=white] (0,0.5) -- (-0.4,-0.2) -- (0.4,-0.2) -- cycle;
    %     \end{scope}

    %     % Rectángulo de la derecha
    %     % Empieza donde termina el rombo en el eje x (4.5)
    %     \draw [fill] (4.5,0) rectangle (7.5,1.5);
    %     \draw [white, fill=white] (4.8, 1.3) -- (5.8, 1.3) -- (5.3, 0.5) -- cycle;
    % \end{tikzpicture}
    \begin{center}
        \includegraphics[width=0.8\textwidth]{../media/ej5-2.png}
    \end{center}
    \end{center}
\end{ejercicio}

\begin{solucion} Solución al problema 5.2:
    \lstinputlisting{../code/EjerciciosTeoria/problema_5_2.gd}
\end{solucion}

\begin{ejercicio}
    % \textbf{Escena simple: Tronco}

    Implementa un proyecto Godot con una función \texttt{Tronco} que crea y devuelve un \texttt{Node2D} con dos nodos hijos que forman la figura de aquí abajo (uno para el relleno y otro para las aristas).

    Tabla de coordenadas:
    \begin{center}
    \begin{tabular}{c c}
        0 & $(+0.0, +0.0)$ \\
        1 & $(+1.0, +0.0)$ \\
        2 & $(+1.0, +1.0)$ \\
        3 & $(+2.0, +2.0)$ \\
        4 & $(+1.5, +2.5)$ \\
        5 & $(+0.5, +1.5)$ \\
        6 & $(+0.0, +3.0)$ \\
        7 & $(-0.5, +3.0)$ \\
        8 & $(+0.0, +1.5)$ \\
    \end{tabular}
    \end{center}

    \begin{center}
    \begin{tikzpicture}[scale=1.5]
        \coordinate (P0) at (0.0, 0.0);
        \coordinate (P1) at (1.0, 0.0);
        \coordinate (P2) at (1.0, 1.0);
        \coordinate (P3) at (2.0, 2.0);
        \coordinate (P4) at (1.5, 2.5);
        \coordinate (P5) at (0.5, 1.5);
        \coordinate (P6) at (0.0, 3.0);
        \coordinate (P7) at (-0.5, 3.0);
        \coordinate (P8) at (0.0, 1.5);

        % Relleno
        \fill[blue!20] (P0) -- (P1) -- (P2) -- (P3) -- (P4) -- (P5) -- (P6) -- (P7) -- (P8) -- cycle;
        % Aristas
        \draw[blue, ultra thick] (P0) -- (P1) -- (P2) -- (P3) -- (P4) -- (P5) -- (P6) -- (P7) -- (P8) -- cycle;

        % Etiquetas de vértices
        \node[left] at (P0) {0};
        \node[right] at (P1) {1};
        \node[right] at (P2) {2};
        \node[right] at (P3) {3};
        \node[right] at (P4) {4};
        \node[below] at (P5) {5};
        \node[right] at (P6) {6};
        \node[left] at (P7) {7};
        \node[left] at (P8) {8};
    \end{tikzpicture}
    \end{center}
\end{ejercicio}

\begin{solucion} Solución al problema 5.3:
    \lstinputlisting{../code/EjerciciosTeoria/problema_5_3.gd}
\end{solucion}

\begin{ejercicio}
    % \textbf{Figura recursiva}

    Implementa otro proyecto Godot que use la función del problema anterior para otra función, \texttt{Arbol(n)}, que genera un árbol de escena con la figura de aquí abajo, que incluye múltiples instancias de \texttt{Tronco}, situadas recursivamente unas adyacentes a otras, hasta un nivel de recursividad dado por $n$.

\end{ejercicio}

\begin{solucion} Solución al problema 5.4:
    \lstinputlisting{../code/EjerciciosTeoria/problema_5_4.gd}
\end{solucion}

\begin{ejercicio}
    % \textbf{Árbol de escena 3D}

    En un proyecto Godot 3D (puedes usar la práctica 2), crea una figura como el logo de Android, usando únicamente dos objetos \texttt{ArrayMesh}, uno con un cilindro y otro con una semiesfera.

    \begin{center}
    \begin{tikzpicture}
        % Representación esquemática del logo de Android
        \definecolor{androidgreen}{RGB}{164, 198, 57}
        
        % Cuerpo (Cilindro simplificado como rectángulo con base curva)
        \fill[androidgreen] (-1.2, 0) -- (1.2, 0) -- (1.2, -2.5) arc(0:-180:0.2 and 0.1) -- (-1.2, -2.5) -- cycle;
        \fill[androidgreen] (-1.2, -2.5) arc(180:360:1.2 and 0.2); % Base redondeada
        
        % Cabeza (Semiesfera)
        \fill[androidgreen] (0, 0.2) circle [x radius=1.25, y radius=1.0, start angle=0, end angle=180];
        \fill[white] (-1.3, 0.2) rectangle (1.3, -0.1); % Cortar la parte de abajo para hacerla plana (gap)
        \fill[androidgreen] (1.25, 0.2) arc(0:180:1.25);
        
        % Ojos
        \fill[white] (-0.5, 0.8) circle (0.12);
        \fill[white] (0.5, 0.8) circle (0.12);
        
        % Antenas
        \draw[androidgreen, line width=3pt, line cap=round] (-0.6, 1.2) -- (-0.9, 1.7);
        \draw[androidgreen, line width=3pt, line cap=round] (0.6, 1.2) -- (0.9, 1.7);
        
        % Brazos (Cilindros con terminaciones esféricas)
        \fill[androidgreen, rounded corners=5pt] (-1.8, 0.1) rectangle (-1.4, -1.8);
        \fill[androidgreen, rounded corners=5pt] (1.4, 0.1) rectangle (1.8, -1.8);
        
        % Piernas (Cilindros)
        \fill[androidgreen, rounded corners=5pt] (-0.8, -2.2) rectangle (-0.4, -3.2);
        \fill[androidgreen, rounded corners=5pt] (0.4, -2.2) rectangle (0.8, -3.2);

    \end{tikzpicture}
    \end{center}
    \begin{center}
        \includegraphics[width=0.5\textwidth]{../media/android-logo.png}
    \end{center}
\end{ejercicio}

\begin{solucion} Solución al problema 5.5:
    \lstinputlisting{../code/EjerciciosTeoria/problema_5_5.gd}
\end{solucion}



\section{Sesión 6}

\begin{ejercicio}

Escribe el código GDScript para adjuntar a un nodo de tipo Camera3D, de forma que en cada frame la cámara apunte a un objeto móvil objetivo (por ejemplo un coche), con estos requerimientos:
\begin{itemize}
    \item La posición y el vector de velocidad del objetivo (en coordenadas de mundo) se pueden obtener con dos funciones globales, llamadas \texttt{objetivo.posicion()} y \texttt{objetivo.velocidad()}, ambas devuelven un objeto de tipo \texttt{Vector3}.
    \item La cámara debe situarse detrás del objetivo, de forma que el punto devuelto por \texttt{objetivo.posicion()} se proyecte en el centro del viewport, y además la cámara esté situada 3 unidades en horizontal por detrás del objetivo, y 2 unidades por encima (en el eje Y).
\end{itemize}
\end{ejercicio}

\begin{solucion} Nuestro objetivo móvil va a ser un coche. La resolución detallada es la siguiente:\\

\subsubsection*{Requerimientos Geométricos:}

\begin{enumerate}
    \item \textbf{Punto de Atención (Look At):} La cámara debe apuntar al objetivo. Esto significa que el eje $-Z$ de la cámara (en Godot, la cámara ''mira'' hacia $-Z$ local) debe alinearse con el vector que va desde la cámara hasta el objetivo. El punto $\vec{p}_{obj}$ se proyectará en el centro del \emph{viewport}.
    \item \textbf{Posición Relativa:}
    \begin{itemize}
        \item \textbf{''Detrás'' (Horizontal):} 3 unidades por detrás. ''Detrás'' se define en relación con el movimiento. Si el coche se mueve hacia adelante, ''detrás'' es la dirección opuesta a la velocidad. Debemos considerar solo la componente horizontal para evitar que la cámara se incline hacia el suelo si el coche sube una pendiente.
        \item \textbf{''Arriba'' (Vertical):} 2 unidades por encima del objetivo (eje Y global).
    \end{itemize}
\end{enumerate}

\subsubsection*{Fundamentación Teórica}

Para resolver esto, utilizamos conceptos de \textbf{Espacios Afines} y \textbf{Operaciones con Vectores} (tratados en el pdf \texttt{ig-s03.pdf}):

\begin{enumerate}
    \item \textbf{Definición de ''Atrás'':}
    El vector velocidad $\vec{v}_{obj}$ nos da la dirección del movimiento. Para situarnos ''detrás'' horizontalmente:
    \begin{itemize}
        \item Tomamos $\vec{v}_{obj}$ y anulamos su componente Y (para que sea puramente horizontal): $\vec{d}_{hz} = (v_x, 0, v_z)$.
        \item Normalizamos este vector para obtener una dirección unitaria: $\hat{d}_{hz} = \vec{d}_{hz} / |\vec{d}_{hz}|$.
        \item El vector ''hacia atrás'' es $-\hat{d}_{hz}$.
        \item El desplazamiento horizontal deseado es $-3 \cdot \hat{d}_{hz}$.
    \end{itemize}
    \item \textbf{Composición de la Posición de la Cámara ($\vec{p}_{cam}$):}
    \[
        \vec{p}_{cam} = \vec{p}_{obj} + (0,2,0) - 3 \cdot \hat{d}_{hz}
    \]
    Donde:
    \begin{itemize}
        \item $(0,2,0)$: 2 unidades arriba.
        \item $-3 \cdot \hat{d}_{hz}$: 3 unidades atrás.
    \end{itemize}
    \item \textbf{Transformación de Vista (LookAt):}
    Una vez tenemos $\vec{p}_{cam}$, necesitamos construir la matriz de vista. En Godot, la clase \texttt{Node3D} (de la cual hereda \texttt{Camera3D}) tiene métodos auxiliares para esto. El método \texttt{look\_at(target, up)} ajusta la transformación del nodo para que mire a \texttt{target} manteniendo el vector \texttt{up} orientado hacia arriba tanto como sea posible.
\end{enumerate}

\subsubsection*{Solución: Código GDScript}

\begin{lstlisting}[language=gdscript]
extends Camera3D

# Asumimos que 'objetivo' es un singleton (AutoLoad) o una clase global accesible.
# Si no fuera global, habría que obtener la referencia al nodo (ej. get_node(''../Coche''))

func _process(delta: float):
    # 1. Obtener datos del objetivo (en coordenadas de mundo)
    # Según el enunciado, existen estas funciones globales.
    var p_obj: Vector3 = objetivo.posicion()
    var v_obj: Vector3 = objetivo.velocidad()
    
    # 2. Calcular la dirección horizontal del movimiento
    # Creamos un vector con la velocidad pero ignorando la componente Y
    var direccion_hz: Vector3 = Vector3(v_obj.x, 0.0, v_obj.z)
    
    # IMPORTANTE: Si el coche está parado (velocidad casi 0), no podemos normalizar 
    # (división por cero). En un caso real, mantendríamos la última dirección válida.
    # Para el ejercicio, asumimos movimiento o usamos una dirección por defecto (ej. eje Z).
    if direccion_hz.length_squared() > 0.001:
        direccion_hz = direccion_hz.normalized()
    else:
        # Fallback: si está quieto, asumimos que ''detrás'' es el eje Z positivo (por ejemplo)
        # O idealmente, usaríamos la orientación del nodo objetivo (basis.z)
        direccion_hz = Vector3(0, 0, 1) 

    # 3. Calcular la posición deseada de la cámara
    # - Situada en la posición del objetivo
    # - Desplazada 2 unidades hacia ARRIBA (Eje Y global)
    # - Desplazada 3 unidades hacia ATRÁS (opuesto a la dirección horizontal)
    var nueva_posicion: Vector3 = p_obj + Vector3(0, 2, 0) - (direccion_hz * 3.0)
    
    # 4. Aplicar la posición a la cámara
    # Usamos global_position para asegurar que estamos en coords de mundo
    global_position = nueva_posicion
    
    # 5. Orientar la cámara (Transformación de Vista)
    # Hacemos que la cámara mire al punto objetivo.
    # El vector ''Arriba'' (Up) suele ser el eje Y global (Vector3.UP)
    look_at(p_obj, Vector3.UP)
\end{lstlisting}

\subsubsection*{Explicación detallada de la implementación}

\begin{enumerate}
    \item \textbf{\texttt{extends Camera3D}}: El script hereda de la clase base de cámaras en Godot, permitiendo controlar la proyección y vista.
    \item \textbf{\texttt{\_process(delta)}}: Usamos este método del bucle principal (\texttt{MainLoop}) porque el enunciado pide que la cámara se actualice ''en cada frame''.
    \item \textbf{Cálculo del vector dirección}:
    \begin{itemize}
        \item El enunciado especifica ''3 unidades en horizontal''. Esto es crucial. Si usáramos el vector velocidad completo (incluyendo Y) para calcular el ''atrás'', y el coche subiera una rampa muy empinada, la cámara se metería bajo tierra. Por eso proyectamos sobre el plano XZ haciendo $v_{obj}.y = 0$ y luego normalizamos $v.normalized()$.
    \end{itemize}
    \item \textbf{Posicionamiento (\texttt{global\_position})}:
    \begin{itemize}
        \item Calculamos la posición final sumando vectores. Matemáticamente: $\vec{p}_{cam} = \vec{p}_{obj} + (0,2,0) - 3 \cdot \hat{d}_{hz}$.
    \end{itemize}
    \item \textbf{Orientación (\texttt{look\_at})}:
    \begin{itemize}
        \item Este método es fundamental en la \textbf{Transformación de Vista}. Recalcula la matriz de transformación del nodo (\texttt{transform}) para que su eje $-Z$ (visión) apunte a $\vec{p}_{obj}$ y su eje $Y$ se alinee con \texttt{Vector3.UP}. Esto resuelve la parte compleja de crear la matriz de rotación ortonormal manualmente.
    \end{itemize}
\end{enumerate}

\end{solucion}

\begin{ejercicio}
Supongamos una escena que contiene una representación visible del marco de coordenadas del mundo como tres flechas (roja X, verde Y y azul Z), como ocurre en las prácticas. Queremos visualizar esa escena en pantalla, de forma que:
\begin{enumerate}
    \item El eje Y aparezca vertical, hacia arriba, el eje X horizontal, hacia la derecha, el eje Z horizontal, hacia la izquierda (los ejes X y Z se visualizan con la misma longitud aparente).
    \item El punto de coordenadas $(0, 0.5, 0)$ (aparece como un disco de color morado en la figura) debe aparecer en el centro del viewport.
    \item El observador (foco de la proyección) estará a 3 unidades de distancia del punto $(0, 0.5, 0)$.
\end{enumerate}

Escribe unos valores que podríamos usar para $\mathbf{a}$, $\mathbf{u}$ y $\mathbf{n}$ de forma que se cumplan estos requisitos. En la figura se observa una vista esquemática de cómo quedaría la figura en un viewport cuadrado.

\begin{center}
\begin{tikzpicture}
    % Viewport frame
    \draw[thick] (-3,-3) rectangle (3,3);
    
    % Origin point (approximate projection below center)
    \coordinate (O) at (0, -1);
    
    % Center point (a)
    \coordinate (A) at (0, 0); 
    \fill[violet] (A) circle (0.15);
    \node[right, violet] at (0.2, 0) {$(0, 0.5, 0)$};
    
    % Axes
    % Y axis (Green, Up)
    \draw[->, ultra thick, green!70!black] (O) -- (0, 2) node[above] {Y};
    
    % X axis (Red, Right)
    \draw[->, ultra thick, red] (O) -- (2.5, -1) node[right] {X};
    
    % Z axis (Blue, Left)
    \draw[->, ultra thick, blue] (O) -- (-2.5, -1) node[left] {Z};
    
    % Origin dot
    \fill[black] (O) circle (0.05);
    \node[below] at (O) {Origen $(0,0,0)$};
    
\end{tikzpicture}
\hspace{1cm}
\includegraphics[height=6cm]{../media/p6-2.png}
\end{center}
\end{ejercicio}

\begin{solucion}
Para determinar los parámetros de la matriz de vista ($\mathbf{a}$, $\mathbf{u}$, $\mathbf{n}$), analizamos cada requerimiento paso a paso:

\begin{enumerate}
    \item \textbf{Determinación del punto de atención ($\mathbf{a}$):}
    El enunciado establece que el punto de coordenadas $(0, 0.5, 0)$ debe aparecer en el centro del viewport. Por definición, el punto de atención $\mathbf{a}$ (Look-At point) es el punto hacia el que apunta la cámara y que se proyecta en el centro del plano de imagen.
    
    Por lo tanto:
    \[ \mathbf{a} = (0, 0.5, 0) \]

    \item \textbf{Determinación del vector hacia arriba ($\mathbf{u}$):}
    Se requiere que el eje Y del mundo aparezca vertical y hacia arriba en la imagen. Dado que el eje Y del mundo es $(0, 1, 0)$, la forma más directa de conseguir que se proyecte verticalmente es alineando el vector \textit{view-up} ($\mathbf{u}$) con el eje Y del mundo (siempre que la dirección de vista no sea paralela a este eje, lo cual verificaremos en el siguiente paso).
    
    Por lo tanto:
    \[ \mathbf{u} = (0, 1, 0) \]

    \item \textbf{Determinación del vector normal de vista ($\mathbf{n}$):}
    El vector $\mathbf{n}$ define la dirección desde el punto de atención hacia el observador (es decir, la inversa de la dirección de la vista). También determina la posición del observador $\mathbf{o}_{ec}$ mediante la relación $\mathbf{o}_{ec} = \mathbf{a} + \mathbf{n}$.
    
    Analizamos las condiciones para $\mathbf{n} = (n_x, n_y, n_z)$:
    \begin{itemize}
        \item \textbf{Longitud:} El observador debe estar a 3 unidades de distancia de $\mathbf{a}$. Como $\mathbf{n}$ es el vector que une $\mathbf{a}$ con el observador, su norma debe ser 3:
        \[ ||\mathbf{n}|| = 3 \]
        
        \item \textbf{Orientación Horizontal:} Para que el eje Y se vea perfectamente vertical y centrado, la cámara debe estar a la misma altura o el vector de visión debe estar contenido en un plano vertical que contenga al eje Y. Sin embargo, la condición crítica proviene de los ejes X y Z.
        
        \item \textbf{Orientación de X y Z:}
        \begin{itemize}
            \item El eje X debe verse horizontal hacia la derecha.
            \item El eje Z debe verse horizontal hacia la izquierda.
            \item Ambos deben tener la misma longitud aparente.
        \end{itemize}
        
        Esto implica que el observador debe situarse en una posición simétrica respecto a los ejes X e Z positivos (primer cuadrante del plano XZ respecto a $\mathbf{a}$), de forma que la línea de visión biseque el ángulo de 90 grados entre X y Z.
        
        Si nos situamos en la bisectriz del primer cuadrante del plano XZ, el vector de dirección tendrá componentes X y Z iguales y positivas. El eje X (derecha) y el eje Z (adelante) formarán ambos un ángulo de $45^\circ$ con el plano de proyección, proyectándose hacia lados opuestos (derecha e izquierda) con la misma deformación (longitud aparente).
        
        Por tanto, la dirección de $\mathbf{n}$ debe ser $(1, 0, 1)$.
    \end{itemize}
    
    Calculamos $\mathbf{n}$:
    \begin{enumerate}
        \item Tomamos el vector director base: $\vec{d} = (1, 0, 1)$.
        \item Calculamos su norma: $||\vec{d}|| = \sqrt{1^2 + 0^2 + 1^2} = \sqrt{2}$.
        \item Normalizamos y escalamos por la distancia requerida (3 unidades):
        \[ \mathbf{n} = 3 \cdot \frac{\vec{d}}{||\vec{d}||} = 3 \cdot \frac{(1, 0, 1)}{\sqrt{2}} = \left( \frac{3}{\sqrt{2}}, 0, \frac{3}{\sqrt{2}} \right) \]
    \end{enumerate}
    
    Aproximando los valores:
    \[ \frac{3}{\sqrt{2}} = \frac{3 \sqrt{2}}{2} \approx 2.1213 \]
    
    Así, $\mathbf{n} \approx (2.12, 0, 2.12)$.
\end{enumerate}

\textbf{Resultado Final:}
Los valores que cumplen los requisitos son:
\[ \mathbf{a} = (0, 0.5, 0) \]
\[ \mathbf{u} = (0, 1, 0) \]
\[ \mathbf{n} = \left( \frac{3}{\sqrt{2}}, 0, \frac{3}{\sqrt{2}} \right) \]

\end{solucion}

\begin{ejercicio}
Repite el problema anterior 6.2, pero ahora para esta vista (ver figura). Usa una rotación del marco de vista entorno a uno de sus propios ejes.

\begin{center}
\begin{tikzpicture}
    % Viewport frame
    \draw[thick] (-3,-3) rectangle (3,3);
    
    % Origin point (black dot)
    \coordinate (O) at (0, 0);
    
    % Center point a (purple dot on Y axis)
    \coordinate (A) at (-0.5, 1); 
    
    % Axes visual representation based on Image 6.3
    % Y axis (Green, Up-Left)
    \draw[->, ultra thick, green!70!black] (O) -- (-1.5, 3) node[above left] {Y};
    \fill[violet] (A) circle (0.15); % Point a
    
    % Z axis (Blue, Up-Right)
    \draw[->, ultra thick, blue] (O) -- (2.5, 1.5) node[right] {Z};
    
    % X axis (Red, Down-Left)
    \draw[->, ultra thick, red] (O) -- (-2, -2) node[below left] {X};
    
    % Origin dot
    \fill[black] (O) circle (0.05);
    \node[right] at (0.1, -0.2) {$\dot{o}_{wc}$};
    
\end{tikzpicture}
\hspace{1cm}
\includegraphics[height=6cm]{../media/p6-3.png}
\end{center}

Escribe los valores para $\mathbf{a}$, $\mathbf{u}$ y $\mathbf{n}$.
\end{ejercicio}

\begin{solucion}
Para obtener la configuración visual mostrada en la figura, partimos de la solución del ejercicio 6.2 y aplicamos las transformaciones necesarias.

\begin{enumerate}
    \item \textbf{Punto de atención ($\mathbf{a}$):}
    Al igual que en el ejercicio anterior, el punto $(0, 0.5, 0)$ (disco morado) debe aparecer en el centro del viewport. Por tanto:
    \[ \mathbf{a} = (0, 0.5, 0) \]

    \item \textbf{Vector normal de vista ($\mathbf{n}$):}
    Observamos la orientación de los ejes X y Z:
    \begin{itemize}
        \item El eje X (rojo) apunta hacia la izquierda y abajo.
        \item El eje Z (azul) apunta hacia la derecha y arriba.
    \end{itemize}
    En el ejercicio 6.2, mirábamos desde el primer cuadrante $(+X, +Z)$, viendo el eje X a la derecha y Z a la izquierda. Aquí la situación horizontal se ha invertido (X a la izquierda, Z a la derecha), lo que implica que el observador se ha movido a la posición opuesta (''detrás'' de la escena), mirando desde el cuadrante $(-X, -Z)$.
    
    El vector de dirección base sería $(-1, 0, -1)$. Normalizando y aplicando la distancia de 3 unidades:
    \[ \mathbf{n} = 3 \cdot \frac{(-1, 0, -1)}{\sqrt{(-1)^2 + 0^2 + (-1)^2}} = 3 \cdot \left( \frac{-1}{\sqrt{2}}, 0, \frac{-1}{\sqrt{2}} \right) \]
    
    Aproximando:
    \[ \mathbf{n} \approx (-2.12, 0, -2.12) \]

    \item \textbf{Vector hacia arriba ($\mathbf{u}$):}
    Observamos el eje Y (verde). En lugar de apuntar verticalmente hacia arriba (como haría con $\mathbf{u}=(0,1,0)$), apunta hacia arriba a la izquierda. Esto indica una rotación de la cámara (Roll) alrededor del eje de visión $\mathbf{n}$.
    
    Si usáramos $\mathbf{u}_{base} = (0,1,0)$ desde la posición trasera, veríamos el eje Y vertical. Para que el eje Y se incline hacia la izquierda en la pantalla, la cámara debe rotar en sentido horario (CW). Una rotación de 45 grados en sentido horario del vector $\mathbf{u}_{base}$ alrededor del eje de visión nos da el vector necesario.
    
    Calculamos $\mathbf{u}$ como una combinación lineal que se incline hacia el eje Z negativo y X negativo (para mantener la ortogonalidad con $\mathbf{n}$):
    \[ \mathbf{u} = (-1, 1, 1) \]
    (Nota: Se puede normalizar a $(-1/\sqrt{3}, 1/\sqrt{3}, 1/\sqrt{3})$).
    
    Verificación rápida: $\mathbf{n} \cdot \mathbf{u} = (-1)(-1) + (0)(1) + (-1)(1) = 1 + 0 - 1 = 0$. Son perpendiculares.

    \vspace{1em}
    \textbf{¿Cómo se calcula $\mathbf{u}$ exactamente?}

    El vector $\mathbf{u}$ (View-Up) indica la dirección de ''arriba'' para la cámara. El procedimiento ordenado para deducir $\mathbf{u} = (-1, 1, 1)$ es:

    \begin{enumerate}
        \item \textbf{Definir la base sin rotar:}
        \begin{itemize}
            \item Nos situamos ''detrás'' de la escena (lado opuesto al ejercicio 6.2), ya que el eje X va a la izquierda y el Z a la derecha.
            \item Vector de vista ideal: $\mathbf{n} = (-1, 0, -1)$.
            \item Vector arriba estándar: $\mathbf{u}_{base} = (0, 1, 0)$.
        \end{itemize}
        \item \textbf{Calcular el vector ''Derecha'':}
        \[
            \text{Derecha} = \mathbf{u}_{base} \times \mathbf{n} = (0, 1, 0) \times (-1, 0, -1) = (-1, 0, 1)
        \]
        Sabemos que $\text{Arriba} \times \text{Atrás} = \text{Derecha}$. Para el caso de la izquierda sería el opuesto. (n es atrás y u es arriba).
        \item \textbf{Aplicar la rotación (mezclar arriba y derecha):}
        \begin{itemize}
            \item Para rotar la cámara hacia la derecha (sentido horario), sumamos el vector arriba original y el vector derecha:
            \[
                \mathbf{u} = \mathbf{u}_{base} + \text{Derecha} = (0, 1, 0) + (-1, 0, 1) = (-1, 1, 1)
            \]
            \item Este vector tiene componente en Y (arriba), pero también en X y Z, inclinando el ''arriba'' de la cámara hacia la derecha de la pantalla, logrando el efecto de rotación deseado.
        \end{itemize}
    \end{enumerate}


\end{enumerate}

\textbf{Valores Finales:}
\[ \mathbf{a} = (0, 0.5, 0) \]
\[ \mathbf{u} = (-1, 1, 1) \quad (\text{o normalizado } \approx (-0.577, 0.577, 0.577)) \]
\[ \mathbf{n} = \left( \frac{-3}{\sqrt{2}}, 0, \frac{-3}{\sqrt{2}} \right) \approx (-2.12, 0, -2.12) \]

\end{solucion}

\begin{ejercicio}
Escribe el código GDScript para calcular los vectores de coordenadas $o_{ec}$, $x_{ec}$, $y_{ec}$ y $z_{ec}$ que definen el marco de vista a partir de los vectores de coordenadas $a$, $u$ y $n$ (todos estos vectores de coordenadas de mundo, en objetos de tipo Vector3).
\end{ejercicio}

\begin{solucion}
Para construir el marco de referencia de vista (view reference frame) a partir de los vectores dados, seguimos el procedimiento estándar de la transformación de cámara en gráficos 3D:

\begin{enumerate}
    \item \textbf{Cálculo del origen del marco ($o_{ec}$):}
    El origen del marco de cámara (posición del observador) se obtiene sumando el punto de atención $a$ y el vector normal $n$:
    \[
        o_{ec} = a + n
    \]

    \item \textbf{Cálculo del eje $z_{ec}$:}
    El eje $z_{ec}$ es la dirección de la vista (normalizada) y se obtiene normalizando el vector $n$:
    \[
        z_{ec} = \frac{n}{\|n\|}
    \]

    \item \textbf{Cálculo del eje $x_{ec}$:}
    El eje $x_{ec}$ (derecha de la cámara) se obtiene como el producto vectorial entre el vector hacia arriba $u$ y el vector normal $n$, normalizado:
    \[
        x_{ec} = \frac{u \times n}{\|u \times n\|}
    \]

    \item \textbf{Cálculo del eje $y_{ec}$:}
    El eje $y_{ec}$ (arriba de la cámara) se obtiene como el producto vectorial entre $z_{ec}$ y $x_{ec}$:
    \[
        y_{ec} = z_{ec} \times x_{ec}
    \]
\end{enumerate}

El siguiente código GDScript implementa estos pasos, suponiendo que $a$, $u$ y $n$ son objetos de tipo \texttt{Vector3}:

\begin{lstlisting}
# a, u, n: Vector3 (coordenadas de mundo)

# 1. Origen del marco de cámara
var o_ec : Vector3 = a + n

# 2. Eje Z (dirección de la vista, normalizado)
var z_ec : Vector3 = n.normalized()

# 3. Eje X (derecha, ortogonal a u y n, normalizado)
var x_ec : Vector3 = u.cross(n).normalized()

# 4. Eje Y (arriba, ortogonal a z_ec y x_ec)
var y_ec : Vector3 = z_ec.cross(x_ec)
\end{lstlisting}

Este procedimiento garantiza que los vectores $x_{ec}$, $y_{ec}$ y $z_{ec}$ forman una base ortonormal adecuada para definir el sistema de referencia de la cámara.
\end{solucion}

\begin{ejercicio}
Partiendo de los vectores de coordenadas $o_{ec}$, $x_{ec}$, $y_{ec}$ y $z_{ec}$ que se calculan en el problema anterior, escribe el código que calcula explícitamente la matriz de vista, es una variable de tipo \texttt{Transform3D}.
\end{ejercicio}

\begin{solucion}
Para construir la matriz de vista (\textit{View Matrix}) a partir del marco de cámara definido por $o_{ec}$ (origen), $x_{ec}$, $y_{ec}$ y $z_{ec}$ (vectores ortonormales), seguimos el procedimiento estándar de gráficos 3D:

\begin{enumerate}
    \item \textbf{Definición:} La matriz de vista transforma coordenadas del mundo al sistema de la cámara. Se compone de una rotación (alineando los ejes del mundo con los de la cámara) y una traslación (llevando el origen de la cámara al origen del sistema).
    \item \textbf{Expresión matricial:}
    \[
    V = \begin{pmatrix}
    x_{ec}.x & x_{ec}.y & x_{ec}.z & -(x_{ec} \cdot o_{ec}) \\
    y_{ec}.x & y_{ec}.y & y_{ec}.z & -(y_{ec} \cdot o_{ec}) \\
    z_{ec}.x & z_{ec}.y & z_{ec}.z & -(z_{ec} \cdot o_{ec}) \\
    0 & 0 & 0 & 1
    \end{pmatrix}
    \]
    \item \textbf{Implementación en Godot (\texttt{Transform3D}):}
    En Godot, la clase \texttt{Transform3D} almacena la base (rotación) y el origen (traslación). La base se define por columnas, por lo que debemos transponer la matriz formada por $x_{ec}$, $y_{ec}$ y $z_{ec}$ como filas.
\end{enumerate}

\hspace{1cm}

\textbf{Código GDScript:}
\begin{lstlisting}
# Suponemos disponibles: x_ec, y_ec, z_ec, o_ec (Vector3)

# 1. Construir la base (rotación): columnas de la base son los ejes de cámara
var R := Basis(x_ec, y_ec, z_ec)
var vista_basis := R.transposed()

# 2. Calcular la traslación (origen) según la fórmula de la matriz de vista
var d_x = -x_ec.dot(o_ec)
var d_y = -y_ec.dot(o_ec)
var d_z = -z_ec.dot(o_ec)
var vista_origin = Vector3(d_x, d_y, d_z)

# 3. Construir la matriz de vista final
var matriz_vista = Transform3D(vista_basis, vista_origin)

# La función de Transform3D lo que es empaqueta todo en un solo objeto, en este caso, lo que hace es crear una matriz de 4x4 a partir de una matriz de 3x3 (Basis) y un vector de traslación (origin).
\end{lstlisting}

\textbf{Explicación:} La matriz de vista es la inversa de la transformación de la cámara en el mundo. La base ortonormal se transpone para invertir la rotación, y la traslación se obtiene proyectando el origen del marco de cámara sobre cada eje y cambiando el signo, lo que equivale a trasladar el mundo al sistema de la cámara. Usamos cross para construir el marco de referencia, y dot para situar puntos dentro de ese marco, en este caso como lo que se busca es \underline{proyectar usamos dot}.
\end{solucion}

\begin{ejercicio}

En una copia independiente del código de prácticas, modifica el nodo de la cámara orbital simple para conseguir que el fov mínimo (vertical u horizontal) sea siempre de $75^{\circ}$. Esto servirá, por ejemplo, para ver el cubo de las prácticas siempre completo independientemente del ancho y alto de la ventana.

Para ello:
\begin{enumerate}
    \item Añadir al script del nodo de cámara una función que se ejecute siempre que se redimensione la ventana (y al inicio).
    \item En esa función, obtener el tamaño (alto y ancho) del viewport.
    \item Calcular la relación de aspecto ($ancho/alto$).
    \item Usar ajuste de la proyección en vertical si el viewport es más ancho que alto, y ajuste en horizontal en caso contrario.
\end{enumerate}

\begin{center}
\begin{tikzpicture}[scale=0.8]
    % Escenario 1: Landscape (Ancho > Alto)
    \draw[thick, fill=blue!10] (0,0) rectangle (4, 2.5);
    \node at (2, 2.8) {\textbf{Caso A: Ancho $>$ Alto}};
    \node at (2, 1.25) {Mantener Altura};
    \draw[<->, red, thick] (4.2, 0) -- (4.2, 2.5) node[midway, right] {Fijo ($75^\circ$)};
    
    % Espacio extra entre los dos escenarios
    \begin{scope}[shift={(6.5,0)}]
        % Escenario 2: Portrait (Alto > Ancho)
        \draw[thick, fill=green!10] (0, -1) rectangle (2.5, 3.5);
        \node at (1.25, 3.8) {\textbf{Caso B: Alto $>$ Ancho}};
        \node[align=center] at (1.25, 1.25) {Mantener\\Anchura};
        \draw[<->, red, thick] (0, -1.2) -- (2.5, -1.2) node[midway, below] {Fijo ($75^\circ$)};
    \end{scope}
\end{tikzpicture}
\end{center}
\end{ejercicio}

\begin{solucion}
Para resolver este problema, debemos manipular la propiedad \texttt{keep\_aspect} de la clase \texttt{Camera3D} en Godot. Esta propiedad determina qué eje (horizontal o vertical) mantiene el ángulo de visión (\texttt{fov}) fijo cuando cambia la relación de aspecto de la ventana.

El objetivo es asegurar que el objeto siempre sea visible. Si la ventana se estrecha horizontalmente, debemos fijar el FOV horizontal. Si se estrecha verticalmente, debemos fijar el FOV vertical.

\begin{enumerate}
    \item \textbf{Lógica del algoritmo:}
    \begin{itemize}
        \item Obtenemos el tamaño del viewport: $w$ (ancho) y $h$ (alto).
        \item Calculamos la relación de aspecto $r = w / h$.
        \item Si $r \geq 1$ (formato apaisado o cuadrado): El ancho es suficiente para contener la escena si fijamos la altura. Usamos \texttt{KEEP\_HEIGHT}.
        \item Si $r < 1$ (formato vertical o ''retrato''): El ancho es el factor limitante. Para evitar que se recorte la escena lateralmente, debemos fijar el ángulo horizontal. Usamos \texttt{KEEP\_WIDTH}.
    \end{itemize}

    \item \textbf{Implementación en GDScript:}
    Añadimos la función \texttt{\_actualiza\_proyeccion} y la conectamos a la señal \texttt{size\_changed} del viewport raíz en la función \texttt{\_ready}.

\begin{lstlisting}
extends Camera3D

# -------------
# constantes y variables de instancia 

const at   := 2.5   # angulo de rot. con teclas
const ar   := 0.5   # angulo de rot. con raton
var   bdrp := false # boton derecho del raton presionado si/no
var   dz   := 3.0   # distancia en Z de la camara al origen
var   dxy  := Vector2( 0.0, 0.0 ) # angulos hor. y vert.

# -------------
# actualiza la variable 'transform' de este nodo camara

func _actualiza_transf_vista(  ) -> void : 
    var ahr  := ((45.0+float(dxy.x))*2.0*PI)/360.0 
    var avr  := ((30.0+float(dxy.y))*2.0*PI)/360.0 
    var tras := Transform3D().translated( Vector3( 0.0, 0.0, dz))   
    var rotx := Transform3D().rotated( Vector3.RIGHT, -avr )
    var roty := Transform3D().rotated( Vector3.UP, ahr ) 
    transform = roty*rotx*tras   

# -------------
# NUEVA FUNCION: Ajuste dinamico de la proyeccion (Problema 6.6)
func _actualiza_proyeccion() -> void:
    # 1. Obtener tamano del viewport
    var vp_size := get_viewport().size
    
    # Evitamos division por cero si la ventana se minimiza completamente
    if vp_size.y == 0: return 

    # 2 y 3. Calcular relacion de aspecto (ancho / alto)
    var aspect_ratio := float(vp_size.x) / float(vp_size.y)
    
    # 4. Ajuste segun la forma de la ventana
    if aspect_ratio < 1.0:
        # Si es mas alto que ancho (Portrait), fijamos el ancho
        keep_aspect = Camera3D.KEEP_WIDTH
    else:
        # Si es mas ancho que alto (Landscape), fijamos el alto (por defecto)
        keep_aspect = Camera3D.KEEP_HEIGHT
        
    # Aseguramos que el FOV base sea siempre 75 grados
    fov = 75.0

# -------------
func _ready() -> void :  
    _actualiza_transf_vista() 
    
    # Conectamos la senal de redimensionado a nuestra nueva funcion
    get_tree().root.size_changed.connect(_actualiza_proyeccion)
    
    # Llamamos a la funcion una vez al inicio para configurar el estado inicial
    _actualiza_proyeccion()
    
# -------------
# procesa evento de entrada (sin cambios respecto al original)

func _input( event : InputEvent ): 
    var av : bool = true 
    
    if event is InputEventKey and event.pressed: 
        match event.keycode:
            KEY_UP:    dxy += Vector2( 0, -at )
            KEY_DOWN:  dxy += Vector2( 0, +at )
            KEY_RIGHT: dxy += Vector2( -at, 0 )
            KEY_LEFT:  dxy += Vector2( at, 0 )
            KEY_MINUS, KEY_PAGEDOWN, KEY_KP_SUBTRACT: dz *= 1.05 
            KEY_PLUS, KEY_PAGEUP, KEY_KP_ADD: dz = max( dz/1.05, 0.1 )
            _: av = false
                
    elif event is InputEventMouseButton: 
        match event.button_index:
            MOUSE_BUTTON_RIGHT: 
                bdrp = event.pressed 
                av = false 
            MOUSE_BUTTON_WHEEL_DOWN: dz *= 1.05
            MOUSE_BUTTON_WHEEL_UP:   dz = max( dz/1.05, 0.1 )
            _: av = false
                
    elif event is InputEventMouseMotion and bdrp: 
        dxy += ar * Vector2( -event.relative.x, event.relative.y ) 
        
    else: 
        av = false 

    if av:
        _actualiza_transf_vista( )
\end{lstlisting}

Para simplificar, lo que se hace es añadir esta función y usarla en el \texttt{\_ready} y cada vez que se redimensiona la ventana:
\begin{lstlisting}[language=GDScript]
# NUEVA FUNCION: Ajuste dinamico de la proyeccion (Problema 6.6)
func _actualiza_proyeccion() -> void:
    # 1. Obtener tamano del viewport
    var vp_size := get_viewport().size
    
    # Evitamos division por cero si la ventana se minimiza completamente
    if vp_size.y == 0: return 

    # 2 y 3. Calcular relacion de aspecto (ancho / alto)
    var aspect_ratio := float(vp_size.x) / float(vp_size.y)
    
    # 4. Ajuste segun la forma de la ventana
    if aspect_ratio < 1.0:
        # Si es mas alto que ancho (Portrait), fijamos el ancho
        keep_aspect = Camera3D.KEEP_WIDTH
    else:
        # Si es mas ancho que alto (Landscape), fijamos el alto (por defecto)
        keep_aspect = Camera3D.KEEP_HEIGHT
        
    # Aseguramos que el FOV base sea siempre 75 grados
    fov = 75.0
\end{lstlisting}
\end{enumerate}
\end{solucion}


% \begin{ejercicio}
% Queremos visualizar una escena con mallas indexadas de la cual sabemos que tiene todos los vértices dentro de un cubo de lado $s$ unidades cuyo centro es el punto de coordenadas del mundo $c=(c_{x},c_{y},c_{z})$. Para construir la matriz de vista, se sitúa el observador en el punto $o_{ec}=(c_{x},c_{y},c_{z}+s+2)$, el punto de atención $a$ se hace igual a $c$ (el centro del cubo se ve en el centro de la imagen), y el vector $u$ es $(0, 1, 0)$. Se visualizará en un viewport cuadrado.

% Queremos construir la matriz de proyección perspectiva $Q$ de forma que se cumplan estos requerimientos:
% \begin{enumerate}
%     \item No se recorta ningún triángulo.
%     \item El tamaño aparente de los objetos (proyectados en pantalla) es el mayor posible.
%     \item El valor del parámetro $n$ es el mayor posible.
%     \item El valor del parámetro $f$ es el menor posible.
%     \item Los objetos no aparecen deformados.
% \end{enumerate}

% Con estos requerimientos, indica cómo calcular los valores $l, r, t, b, n$ y $f$ (para obtener la matriz $Q$ de proyección), en función de $s$ y $c=(c_{x},c_{y},c_{z})$.
% \end{ejercicio}

% \begin{solucion}
% Para resolver este problema, debemos determinar los parámetros del volumen de vista (view frustum) definidos por los planos de recorte: $n$ (near), $f$ (far), $l$ (left), $r$ (right), $b$ (bottom) y $t$ (top). Procederemos analizando la posición de la cámara y la geometría del objeto en el espacio de coordenadas de la cámara (Eye Coordinates, EC).

% \begin{enumerate}
%     \item \textbf{Análisis del Marco de Coordenadas de Vista (EC):}
    
%     Primero determinamos la posición relativa del cubo respecto a la cámara.
%     \begin{itemize}
%         \item \textbf{Posición de la cámara ($E$):} $o_{ec} = (c_x, c_y, c_z + s + 2)$.
%         \item \textbf{Punto de atención ($A$):} $c = (c_x, c_y, c_z)$.
%         \item \textbf{Vector de visión ($\vec{v}$):} $\vec{v} = A - E = (0, 0, -(s+2))$. La cámara mira hacia el eje $Z$ negativo del mundo.
%         \item \textbf{Vector hacia arriba ($u$):} $(0, 1, 0)$.
%     \end{itemize}
    
%     Dado que el vector de visión es paralelo al eje $Z$ negativo y el vector $u$ es el eje $Y$, el marco de coordenadas de la cámara está alineado con el del mundo, pero trasladado. El origen de la cámara está en $z = c_z + s + 2$.
    
%     La transformación de un punto $P_{wc}$ a coordenadas de cámara $P_{ec}$ es una traslación:
%     \[ P_{ec} = P_{wc} - E \]
    
%     Calculamos la posición del centro del cubo $c$ en coordenadas de cámara:
%     \[ c_{ec} = (c_x, c_y, c_z) - (c_x, c_y, c_z + s + 2) = (0, 0, -(s+2)) \]
    
%     El cubo tiene lado $s$, por lo que se extiende $\pm s/2$ desde su centro en cada eje.
    
%     \item \textbf{Cálculo de los planos de recorte en Z ($n$ y $f$):}
    
%     Los valores $n$ y $f$ representan las distancias positivas desde la cámara hacia los planos de recorte delantero (near) y trasero (far). Como la cámara mira hacia $-Z$, los puntos visibles tienen coordenada $z_{ec}$ negativa, y la distancia es $-z_{ec}$.
    
%     El cubo se extiende en el eje Z desde $z_{center} + s/2$ hasta $z_{center} - s/2$:
%     \begin{itemize}
%         \item Cara delantera (más cercana): $z_{front} = -(s+2) + s/2 = -s - 2 + 0.5s = -(0.5s + 2)$.
%         \item Cara trasera (más lejana): $z_{back} = -(s+2) - s/2 = -(1.5s + 2)$.
%     \end{itemize}
    
%     Para cumplir los requisitos:
%     \begin{itemize}
%         \item \textbf{$n$ mayor posible (Req. 3) y sin recortar (Req. 1):} El plano $near$ debe estar justo en la cara delantera del cubo.
%         \[ n = -z_{front} = \frac{s}{2} + 2 \]
%         \item \textbf{$f$ menor posible (Req. 4) y sin recortar (Req. 1):} El plano $far$ debe estar justo en la cara trasera del cubo.
%         \[ f = -z_{back} = \frac{3s}{2} + 2 \]
%     \end{itemize}

%     \item \textbf{Cálculo de los planos de recorte en X e Y ($l, r, b, t$):}
    
%     Los parámetros $l, r, b, t$ definen la ventana de proyección en el plano $near$ ($z = -n$).
    
%     \begin{itemize}
%         \item \textbf{Simetría:} Como el centro del cubo está en $(0,0)$ en coordenadas de cámara (ejes X e Y), el frustum debe ser simétrico para no deformar y centrar el objeto.
%         \[ r = -l \quad \text{y} \quad t = -b \]
%         \item \textbf{Viewport cuadrado:} El viewport es cuadrado, por lo que la relación de aspecto es 1.
%         \[ r - l = t - b \implies 2r = 2t \implies r = t \]
%         \item \textbf{Tamaño aparente máximo (Req. 2):} Para que el objeto se vea lo más grande posible sin recortarse, el frustum debe ''tocar'' los bordes del objeto en el plano de proyección más restrictivo. En la proyección perspectiva, los objetos más cercanos se proyectan más grandes. Debemos asegurar que la cara delantera del cubo (la más ancha visualmente) entre en el frustum.
        
%         La cara delantera del cubo en $z = -n$ se extiende en X desde $-s/2$ hasta $s/2$ y en Y desde $-s/2$ hasta $s/2$.
        
%         Para incluir esta cara completamente en el plano $near$:
%         \[ r = \frac{s}{2} \]
%         \[ t = \frac{s}{2} \]
        
%         Por simetría:
%         \[ l = -\frac{s}{2}, \quad b = -\frac{s}{2} \]
        
%         \textit{Nota:} Al ajustar el frustum a la cara delantera, la cara trasera (que tiene el mismo tamaño físico $s$) también estará dentro del frustum porque, en perspectiva, la región visible se expande con la distancia (el frustum es una pirámide), mientras que el cubo mantiene su tamaño constante.
%     \end{itemize}

%     \item \textbf{Resumen de resultados:}
    
%     Los valores calculados en función de $s$ son:
%     \begin{itemize}
%         \item $n = \frac{s}{2} + 2$
%         \item $f = \frac{3s}{2} + 2$
%         \item $l = -\frac{s}{2}$
%         \item $r = \frac{s}{2}$
%         \item $b = -\frac{s}{2}$
%         \item $t = \frac{s}{2}$
%     \end{itemize}

%     \item \textbf{Representación Gráfica:}
    
%     A continuación se muestra un esquema de la situación en el plano YZ (perfil), mostrando el frustum (en rojo) ajustado al cubo (en azul).

%     \begin{center}
%     \begin{tikzpicture}[scale=1.5]
%         % Ejes
%         \draw[->] (-0.5,0) -- (4,0) node[right] {$-Z_{ec}$ (Distancia)};
%         \draw[->] (0,-1.5) -- (0,1.5) node[above] {$Y_{ec}$};
        
%         % Definición de variables para el dibujo
%         \def\s{1.0} % lado del cubo
%         \def\dist{2.0} % distancia extra
%         \def\n{\s/2 + \dist} % n = 2.5
%         \def\f{3*\s/2 + \dist} % f = 3.5
%         \def\h{\s/2} % altura t = 0.5
        
%         % Cubo (visto de lado)
%         % Z va desde n hasta f en distancia (positivo hacia la derecha en el dibujo)
%         \draw[blue, thick, fill=blue!10] (\n, -\s/2) rectangle (\f, \s/2);
%         \node[blue] at (\n + \s/2, 0) {Cubo};
%         \draw[<->, blue] (\f+0.2, -\s/2) -- (\f+0.2, \s/2) node[midway, right] {$s$};
        
%         % Frustum (lineas de proyección)
%         % Origen de la cámara en (0,0)
%         \draw[red, thick, dashed] (0,0) -- (\f+0.5, {(\f+0.5)*\h/\n});
%         \draw[red, thick, dashed] (0,0) -- (\f+0.5, {-(\f+0.5)*\h/\n});
        
%         % Planos near y far
%         \draw[red, thick] (\n, -\h) -- (\n, \h) node[above] {Plano Near ($n$)};
%         \draw[red, thick] (\f, {-\f*\h/\n}) -- (\f, {\f*\h/\n}) node[below right] {Plano Far ($f$)};
        
%         % Cotas
%         \draw[<->] (0, -1.2) -- (\n, -1.2) node[midway, below] {$n = s/2 + 2$};
%         \draw[<->] (0, -1.5) -- (\f, -1.5) node[midway, below] {$f = 3s/2 + 2$};
        
%         % Parametro t
%         \draw[<->] (\n - 0.2, 0) -- (\n - 0.2, \h) node[midway, left] {$t=s/2$};
        
%         % Cámara
%         \filldraw (0,0) circle (2pt) node[left] {Ojo ($o_{ec}$)};
        
%     \end{tikzpicture}
%     \end{center}
% \end{enumerate}
% \end{solucion}

\begin{ejercicio}
% \textbf{Problema anterior hecho de una manera diferente.} Visualización de un cubo.

Se desea calcular los parámetros de la matriz de proyección perspectiva ($l, r, b, t, n, f$) para visualizar una escena compuesta por un cubo de lado $s$.

\textbf{Datos conocidos:}
\begin{itemize}
    \item El cubo tiene lado $s$.
    \item El centro del cubo está en coordenadas del mundo $c = (c_x, c_y, c_z)$.
    \item La cámara (observador) se sitúa en $o_{ec} = (c_x, c_y, c_z + s + 2)$.
    \item La cámara mira hacia el centro del cubo ($a = c$) y el vector arriba es $(0,1,0)$.
\end{itemize}

\textbf{Requerimientos:}
\begin{itemize}
    \item Ajustar la vista para que el objeto se vea lo más grande posible sin recortarse (zoom máximo).
    \item Ajustar los planos de recorte $near$ y $far$ lo más ceñidos posible al objeto.
    \item Mantener la proporción (sin deformación) en un viewport cuadrado.
\end{itemize}
\end{ejercicio}

\begin{solucion}
Para resolver esto, imaginemos que trasladamos todo el sistema para que la cámara sea el centro del universo $(0,0,0)$. Analizaremos distancias relativas desde la cámara hasta el objeto.

\begin{enumerate}
    \item \textbf{Paso 1: Entender la posición relativa (Distancia D).}
    
    La cámara y el cubo están alineados en los ejes X e Y (tienen las mismas coordenadas $c_x, c_y$). La única diferencia es la profundidad (eje Z).
    
    Calculamos la distancia $D$ desde el ojo hasta el \textbf{centro} del cubo:
    \[ D = Z_{ojo} - Z_{cubo} = (c_z + s + 2) - c_z = s + 2 \]
    
    La cámara mira hacia el eje $-Z$, por lo que el cubo está flotando delante de nosotros a una distancia de $s+2$ unidades.

    \item \textbf{Paso 2: Calcular los planos de profundidad ($n$ y $f$).}
    
    Los parámetros $n$ (near/cerca) y $f$ (far/lejos) definen qué ''rebanada'' del mundo ve la cámara. Queremos que esta rebanada empiece justo en la cara frontal del cubo y termine justo en la cara trasera.
    
    Sabemos que el cubo mide $s$ de profundidad. Por tanto, desde su centro, se extiende $s/2$ hacia adelante (hacia la cámara) y $s/2$ hacia atrás.
    
    \begin{itemize}
        \item \textbf{Plano Near ($n$):} Es la distancia desde el ojo hasta la cara más cercana del cubo.
        \[ n = \text{Distancia al centro} - \text{Mitad del cubo} \]
        \[ n = (s + 2) - \frac{s}{2} = \frac{s}{2} + 2 \]
        
        \item \textbf{Plano Far ($f$):} Es la distancia desde el ojo hasta la cara más lejana del cubo.
        \[ f = \text{Distancia al centro} + \text{Mitad del cubo} \]
        \[ f = (s + 2) + \frac{s}{2} = \frac{3s}{2} + 2 \]
    \end{itemize}

    \item \textbf{Paso 3: Calcular el marco de la ventana ($l, r, b, t$).}
    
    Estos parámetros definen el tamaño del ''marco de la ventana'' a través del cual miramos, situado en la distancia $n$. 
    
    Queremos que el objeto ocupe toda la pantalla. En perspectiva, si la cara delantera del cubo entra justa en la ventana, la cara trasera (que está más lejos) se verá más pequeña y entrará seguro. Por tanto, ajustamos la ventana al tamaño de la cara delantera.
    
    La cara delantera del cubo es un cuadrado de lado $s$. Como la cámara está centrada:
    \begin{itemize}
        \item La mitad del cubo va hacia la derecha y la mitad hacia la izquierda.
        \item La mitad va hacia arriba y la mitad hacia abajo.
    \end{itemize}
    
    Por tanto, en el plano de proyección (que hemos situado pegado a la cara delantera, en $n$):
    \[ r = \text{mitad del ancho} = \frac{s}{2} \]
    \[ l = -\text{mitad del ancho} = -\frac{s}{2} \]
    \[ t = \text{mitad de la altura} = \frac{s}{2} \]
    \[ b = -\text{mitad de la altura} = -\frac{s}{2} \]

    \item \textbf{Esquema Gráfico de la Solución:}
    
    El siguiente diagrama muestra la vista lateral (perfil). El ojo está en el origen. El cubo (azul) está delimitado por los planos $n$ y $f$ (rojo). Las líneas discontinuas muestran el campo de visión.

    \begin{center}
    \begin{tikzpicture}[scale=1.2]
        % Definir parámetros visuales
        \def\eye{0}
        \def\s{1.5} % Tamaño visual del lado s
        \def\gap{2} % El ''+2'' del enunciado
        \def\dist{\s + \gap} % Distancia al centro = s + 2
        \def\near{\dist - \s/2} % n
        \def\far{\dist + \s/2}  % f
        \def\halfS{\s/2}
        
        % Eje Z negativo (hacia donde miramos)
        \draw[->] (0,0) -- (6,0) node[right] {Distancia ($-Z_{ec}$)};
        \draw[->] (0,-2) -- (0,2) node[above] {$Y_{ec}$};
        
        % El Ojo
        \filldraw (0,0) circle (2pt) node[left=5pt] {Ojo};
        
        % El Cubo
        \draw[blue, thick, fill=blue!10] (\near, -\halfS) rectangle (\far, \halfS);
        \node[blue] at (\dist, 0) {Cubo};
        \draw[<->, blue] (\far+0.2, -\halfS) -- (\far+0.2, \halfS) node[midway, right] {Lado $s$};
        
        % Planos Near y Far
        \draw[red, thick] (\near, -2) -- (\near, 2) node[above] {$n$};
        \draw[red, thick] (\far, -2) -- (\far, 2) node[above] {$f$};
        
        % Líneas de visión (Frustum)
        % Conectan el ojo con los bordes de la cara delantera (que define el recorte)
        \draw[dashed, thick, red] (0,0) -- (\far+0.5, {(\far+0.5)*\halfS/\near});
        \draw[dashed, thick, red] (0,0) -- (\far+0.5, {-(\far+0.5)*\halfS/\near - 0.7});
        
        % Cotas explicativas
        \draw[<->] (0, -2.2) -- (\near, -2.2) node[midway, fill=white] {$n = s/2 + 2$};
        \draw[<->] (0, -2.6) -- (\far, -2.6) node[midway, fill=white] {$f = 3s/2 + 2$};
        
        % Top y Bottom
        \draw[<->] (\near-0.2, 0) -- (\near-0.2, \halfS) node[midway, left] {$t=s/2$};
        \draw[<->] (\near-0.2, 0) -- (\near-0.2, -\halfS) node[midway, left] {$b=-s/2$};
        
    \end{tikzpicture}
    \end{center}

    \item \textbf{Resultado Final:}
    Los valores calculados únicamente en función de $s$ son:
    \[ n = \frac{s}{2} + 2, \quad f = \frac{3s}{2} + 2 \]
    \[ r = \frac{s}{2}, \quad l = -\frac{s}{2}, \quad t = \frac{s}{2}, \quad b = -\frac{s}{2} \]
\end{enumerate}
\end{solucion}

\begin{ejercicio}
Repetimos el problema 6.7 con los mismos requerimientos y suposiciones, pero ahora la escena está contenida en una esfera de radio $r$ con centro en $c=(c_x, c_y, c_z)$, en lugar de un cubo.

\textbf{Datos y Adaptación del Enunciado:}
\begin{itemize}
    \item Objeto: Esfera de radio $r$.
    \item Centro: $c = (c_x, c_y, c_z)$.
    \item Cámara: Para mantener la equivalencia con el ejercicio anterior (donde la distancia dependía del tamaño del objeto $s$), sustituimos el lado del cubo $s$ por el diámetro de la esfera $2r$.
    \item Posición de la cámara: $o_{ec} = (c_x, c_y, c_z + 2r + 2)$.
    \item Orientación: Mira hacia $c$, vector arriba $(0,1,0)$.
\end{itemize}

\textbf{Requerimientos:}
\begin{itemize}
    \item $n$ y $f$ ajustados al máximo al objeto.
    \item Tamaño aparente máximo sin recortar (la esfera debe entrar completa en la imagen).
    \item Viewport cuadrado (aspect ratio 1).
\end{itemize}
\end{ejercicio}

\begin{solucion}
Procederemos de forma análoga al caso del cubo, utilizando la \textbf{caja englobante} (bounding box) de la esfera para asegurar que esta quede completamente dentro del volumen de vista. Una esfera de radio $r$ cabe perfectamente dentro de un cubo de lado $s = 2r$.

\begin{enumerate}
    \item \textbf{Paso 1: Análisis de Distancias en el Eje Z.}
    
    Transformamos el centro de la esfera a coordenadas de cámara (poniendo la cámara en el origen).
    La distancia $D$ desde el ojo hasta el centro $c$ es la diferencia en la coordenada $Z$:
    \[ D = Z_{ojo} - Z_{centro} = (c_z + 2r + 2) - c_z = 2r + 2 \]
    
    La esfera se extiende una distancia $r$ (el radio) hacia adelante y hacia atrás desde su centro.
    
    \item \textbf{Paso 2: Cálculo de los planos de recorte ($n$ y $f$).}
    
    \begin{itemize}
        \item \textbf{Plano Near ($n$):} Debe situarse justo delante del punto más cercano de la esfera.
        \[ n = D - \text{radio} = (2r + 2) - r = r + 2 \]
        
        \item \textbf{Plano Far ($f$):} Debe situarse justo detrás del punto más lejano de la esfera.
        \[ f = D + \text{radio} = (2r + 2) + r = 3r + 2 \]
    \end{itemize}

    \item \textbf{Paso 3: Cálculo de la ventana de proyección ($l, r, b, t$).}
    
    Para asegurar que la esfera se vea completa y lo más grande posible, ajustaremos el frustum para que englobe el cuadrado frontal de la ''caja imaginaria'' que contiene a la esfera.
    
    Si el plano de proyección está en $n$, la sección de la caja englobante en ese plano tiene una altura y anchura igual al diámetro de la esfera ($2r$). Sin embargo, debido a la perspectiva, si ajustamos la ventana para cubrir el tamaño del objeto en el plano $near$, garantizamos que cualquier parte del objeto detrás de ese plano también será visible (ya que el frustum se ensancha).
    
    La ''cara delantera'' de nuestra caja imaginaria en $z=-n$ tendría un tamaño de $2r \times 2r$. Como la cámara apunta al centro:
    
    \begin{itemize}
        \item Ancho total = $2r \implies$ Del centro a la derecha = $r$.
        \item Alto total = $2r \implies$ Del centro hacia arriba = $r$.
    \end{itemize}
    
    Por tanto:
    \[ r = r \quad (\text{coincide con el radio}) \]
    \[ t = r \]
    \[ l = -r \]
    \[ b = -r \]
    
    \textit{Nota: Al usar $t=r$ en el plano $n$, estamos definiendo un frustum que pasa exactamente por los bordes de la esfera en su punto más cercano. Como la esfera se curva ''hacia adentro'', esto garantiza holgura y que la esfera completa sea visible.}

    \item \textbf{Representación Gráfica:}
    
    El esquema muestra la esfera (azul) y cómo los planos $n$ y $f$ la encierran (rojo).

    \begin{center}
    \begin{tikzpicture}[scale=1.2]
        % Definir parámetros
        \def\rad{1.0} % radio r
        \def\gap{2.0} % el +2
        \def\D{2*\rad + \gap} % Distancia D = 2r + 2 = 4
        \def\near{\rad + \gap} % n = r + 2 = 3
        \def\far{3*\rad + \gap} % f = 3r + 2 = 5
        
        % Ejes
        \draw[->] (0,0) -- (6,0) node[right] {$-Z_{ec}$};
        \draw[->] (0,-2) -- (0,2) node[above] {$Y_{ec}$};
        \filldraw (0,0) circle (2pt) node[left] {Ojo};
        
        % Esfera
        \draw[blue, thick, fill=blue!10] (\D, 0) circle (\rad);
        \node[blue] at (\D, 0) {$c$};
        \draw[blue, ->] (\D, 0) -- (\D, \rad) node[midway, right] {$r$};
        
        % Planos Near y Far
        \draw[red, thick] (\near, -1.5) -- (\near, 1.5) node[above] {$n$};
        \draw[red, thick] (\far, -1.5) -- (\far, 1.5) node[above] {$f$};
        
        % Frustum (Líneas de visión pasando por r en n)
        % Nota: t=r en distancia n. Pendiente = r/n.
        \draw[dashed, thick, red] (0,0) -- (\far+0.5, {(\far+0.5)*0.3*\rad/\near});
        \draw[dashed, thick, red] (0,0) -- (\far+0.5, {-(\far+0.5)*0.9*\rad/\near});      
        % Cotas
        \draw[<->] (0, -1.8) -- (\near, -1.8) node[midway, fill=white] {$n = r+2$};
        \draw[<->] (0, -2.2) -- (\far, -2.2) node[midway, fill=white] {$f = 3r+2$};
        
        % Top (t)
        \draw[<->] (\near-0.1, 0) -- (\near-0.1, \rad) node[midway, left] {$t=r$};
        
    \end{tikzpicture}
    \end{center}

    \item \textbf{Resumen de resultados:}
    Los parámetros en función de $r$ son:
    \[ n = r + 2, \quad f = 3r + 2 \]
    \[ r_{param} = r, \quad l = -r, \quad t = r, \quad b = -r \]
    (Donde $r_{param}$ es el parámetro \textit{right} del frustum y $r$ es el radio de la esfera).
\end{enumerate}
\end{solucion}

\begin{ejercicio}
Repetimos el problema 6.7 (visualización de un cubo de lado $s$), con los mismos requerimientos de optimización (tamaño máximo, sin recortes, $n$ y $f$ ajustados), pero con una diferencia importante:
El viewport (la ventana donde se dibuja la imagen) ya no es necesariamente cuadrado. Tiene dimensiones de $w$ píxeles de ancho y $h$ píxeles de alto.

\textbf{Datos conocidos:}
\begin{itemize}
    \item Objeto: Cubo de lado $s$, centrado en $c$.
    \item Cámara: Posición $o_{ec} = (c_x, c_y, c_z + s + 2)$, mirando a $c$.
    \item Viewport: Resolución $w \times h$. Relación de aspecto $aspect = w/h$.
\end{itemize}

\textbf{Objetivo:} Calcular $n, f, l, r, b, t$ para que el cubo llene la pantalla lo máximo posible sin perder la proporción (sin deformarse) y sin recortarse.
\end{ejercicio}

\begin{solucion}
Este problema introduce el concepto de \textbf{Relación de Aspecto (Aspect Ratio)}. Si la ventana de nuestro programa es rectangular, el volumen de vista (frustum) también debe ser rectangular con la misma proporción, o de lo contrario el cubo se verá estirado o aplastado.

\begin{enumerate}
    \item \textbf{Paso 1: Planos de profundidad ($n$ y $f$).}
    
    La forma del viewport (rectangular o cuadrada) no afecta a la profundidad. La distancia de la cámara al objeto sigue siendo la misma que en el problema 6.7.
    
    Distancia al centro: $D = s + 2$.
    
    Los planos $n$ y $f$ dependen solo de la coordenada Z del cubo:
    \[ n = \frac{s}{2} + 2 \]
    \[ f = \frac{3s}{2} + 2 \]
    
    (Estos valores son idénticos al problema 6.7).

    \item \textbf{Paso 2: Relación de Aspecto.}
    
    Definimos la relación de aspecto del viewport como:
    \[ a = \frac{\text{ancho}}{\text{alto}} = \frac{w}{h} \]
    
    Para evitar deformaciones, las dimensiones físicas de la ventana de proyección ($r-l$ y $t-b$) deben mantener esta misma proporción:
    \[ \frac{r - l}{t - b} = \frac{2r}{2t} = \frac{r}{t} = a \implies r = t \cdot a \]
    (Asumiendo simetría $r = -l$ y $t = -b$).

    \item \textbf{Paso 3: Cálculo de la ventana ($l, r, b, t$).}
    
    La cara del cubo que debemos encuadrar es un \textbf{cuadrado de lado $s$}.
    Tenemos que meter ese cuadrado de tamaño $s \times s$ dentro de un rectángulo de proporción $w \times h$.
    
    Debemos distinguir dos casos posibles para garantizar que el cubo entre entero (''tamaño aparente mayor posible'' significa ajustar a la dimensión más restrictiva).

    \textbf{CASO A: Viewport Apaisado o ''Landscape'' ($w \ge h$)}
    \begin{itemize}
        \item La ventana es más ancha que alta.
        \item Si ajustamos el ancho de la ventana al ancho del cubo ($2r = s$), la altura de la ventana ($2t$) sería proporcionalmente menor a $s$, y cortaríamos el cubo por arriba y abajo.
        \item \textbf{Solución:} El factor limitante es la \textbf{altura}. Debemos igualar la altura de la ventana a la altura del cubo.
        \[ t = \frac{s}{2}, \quad b = -\frac{s}{2} \]
        \item El ancho se ajusta automáticamente para mantener la proporción (será mayor que $s$, dejando espacio libre a los lados):
        \[ r = t \cdot \frac{w}{h} = \frac{s}{2} \cdot \frac{w}{h} \]
        \[ l = -r = -\frac{s}{2} \cdot \frac{w}{h} \]
    \end{itemize}

    \textbf{CASO B: Viewport Vertical o ''Portrait'' ($w < h$)}
    \begin{itemize}
        \item La ventana es más alta que ancha.
        \item Si ajustamos la altura de la ventana a la altura del cubo ($2t = s$), el ancho ($2r$) sería menor que $s$, y cortaríamos el cubo por los lados.
        \item \textbf{Solución:} El factor limitante es el \textbf{ancho}. Debemos igualar el ancho de la ventana al ancho del cubo.
        \[ r = \frac{s}{2}, \quad l = -\frac{s}{2} \]
        \item La altura se ajusta automáticamente (será mayor que $s$, dejando espacio libre arriba y abajo):
        \[ t = \frac{r}{a} = r \cdot \frac{h}{w} = \frac{s}{2} \cdot \frac{h}{w} \]
        \[ b = -t = -\frac{s}{2} \cdot \frac{h}{w} \]
    \end{itemize}

    \item \textbf{Resumen Gráfico de los Casos:}

    \begin{center}
    \begin{tikzpicture}[scale=0.8]
        % Caso A: Landscape
        \node at (2, 3.5) {\textbf{Caso A: } $w > h$ (Apaisado)};
        % Viewport (Rectángulo rojo)
        \draw[red, thick] (0,0) rectangle (4, 2.5);
        \node[red, above] at (2, 2.5) {Ventana};
        % Cubo (Cuadrado azul)
        % Ajustado en altura
        \draw[blue, thick, fill=blue!20] (0.75, 0) rectangle (3.25, 2.5);
        \node[blue] at (2, 1.25) {Cubo $s \times s$};
        \draw[<->] (4.2, 0) -- (4.2, 2.5) node[midway, right] {$2t = s$};

        % Más separación entre los dos casos
        \begin{scope}[xshift=9cm] % antes era 6cm, ahora 9cm para más separación
            \node at (1.5, 5) {\textbf{Caso B: } $w < h$ (Vertical)};
            % Viewport (Rectángulo rojo)
            \draw[red, thick] (0,0) rectangle (3, 4);
            \node[red, above] at (1.5, 4) {Ventana};
            % Cubo (Cuadrado azul)
            % Ajustado en anchura
            \draw[blue, thick, fill=blue!20] (0, 0.5) rectangle (3, 3.5);
            \node[blue] at (1.5, 2) {Cubo};
            \draw[<->] (0, -0.2) -- (3, -0.2) node[midway, below] {$2r = s$};
        \end{scope}
    \end{tikzpicture}
    \end{center}

    \item \textbf{Resultado General Unificado:}
    Podemos expresar la solución usando la función máximo para cubrir ambos casos:
    \[ n = \frac{s}{2} + 2, \quad f = \frac{3s}{2} + 2 \]
    \[ r = \frac{s}{2} \cdot \max\left(1, \frac{w}{h}\right), \quad t = \frac{s}{2} \cdot \max\left(1, \frac{h}{w}\right) \]
    \[ l = -r, \quad b = -t \]
\end{enumerate}
\end{solucion}

\begin{ejercicio} \textbf{Alta complejidad.}
Posicionamiento de cámara dado un FOV ($\beta$).

Repetimos el problema 6.7 (cubo de lado $s$ centrado en $c$), manteniendo los requerimientos de optimización (viewport cuadrado, sin recortes, $n$ máximo, $f$ mínimo).

\textbf{Nueva condición:}
En lugar de darnos la posición de la cámara, se nos da el ángulo de apertura vertical (Field of View) $\beta$.
Debemos calcular:
\begin{enumerate}
    \item La coordenada Z de la posición del observador ($o_{ec}$), sabiendo que $o_x = c_x$ y $o_y = c_y$.
    \item Los parámetros de la proyección $l, r, t, b, n, f$ en función de $\beta, s$ y $c$.
\end{enumerate}
\end{ejercicio}

\begin{solucion}
Este problema es ''inverso'' al anterior en cierto sentido. Antes fijábamos la distancia y calculábamos qué apertura necesitábamos (implícitamente). Ahora, fijamos la apertura (el ángulo de la lente) y tenemos que calcular a qué distancia ponernos para que el cubo llene la pantalla perfectamente.

\begin{enumerate}
    \item \textbf{Paso 1: Entender la geometría del FOV ($\beta$).}
    
    El ángulo $\beta$ es la apertura total vertical. La mitad de ese ángulo es $\beta/2$.
    En un triángulo rectángulo formado por la línea de visión, el plano de proyección y el borde superior del frustum:
    \[ \tan(\beta/2) = \frac{\text{altura del marco}}{\text{distancia al marco}} = \frac{t}{n} \]
    
    Queremos que el cubo llene la pantalla. Esto ocurre cuando el ''marco'' de visión en el plano más cercano ($n$) coincide exactamente con la cara delantera del cubo.
    
    La cara delantera del cubo tiene altura $s$. Por tanto, desde el centro hacia arriba mide $s/2$.
    Esto fija nuestro valor de $t$:
    \[ t = \frac{s}{2} \]

    \item \textbf{Paso 2: Calcular la distancia al plano Near ($n$).}
    
    Sustituimos $t$ en la ecuación del FOV y despejamos $n$:
    \[ \tan(\beta/2) = \frac{s/2}{n} \]
    \[ n = \frac{s/2}{\tan(\beta/2)} = \frac{s}{2} \cdot \cot(\beta/2) \]
    
    Ahora ya sabemos cuánto espacio debe haber entre el ojo y la cara delantera del cubo ($n$).

    \item \textbf{Paso 3: Calcular la posición de la cámara ($o_z$).}
    
    Sabemos dónde está el cubo en el mundo (en $c_z$).
    \begin{itemize}
        \item El centro del cubo está en $c_z$.
        \item La cara delantera está en $c_z + s/2$ (hacia nosotros).
        \item El ojo está una distancia $n$ más allá de la cara delantera.
    \end{itemize}
    
    \[ o_z = \text{Posición cara delantera} + n \]
    \[ o_z = (c_z + \frac{s}{2}) + n \]
    
    Sustituyendo el valor de $n$ calculado antes:
    \[ o_z = c_z + \frac{s}{2} + \frac{s}{2}\cot(\beta/2) = c_z + \frac{s}{2} \left( 1 + \cot(\frac{\beta}{2}) \right) \]
    
    Por tanto, la posición del observador es:
    \[ o_{ec} = \left( c_x, c_y, c_z + \frac{s}{2} \left( 1 + \cot(\frac{\beta}{2}) \right) \right) \]

    \item \textbf{Paso 4: Calcular el resto de parámetros ($f, l, r, b$).}
    
    \begin{itemize}
        \item \textbf{$f$ (Far):} Es la distancia desde el ojo hasta la cara trasera. La cara trasera está a una distancia $s$ (la profundidad del cubo) más lejos que la cara delantera ($n$).
        \[ f = n + s \]
        \[ f = \frac{s}{2}\cot(\beta/2) + s \]
        
        \item \textbf{$t, b, l, r$:} Como el viewport es cuadrado (según enunciado 6.7) y queremos ajustar a la cara delantera ($s \times s$):
        \[ t = \frac{s}{2} \]
        \[ b = -\frac{s}{2} \]
        \[ r = \frac{s}{2} \]
        \[ l = -\frac{s}{2} \]
    \end{itemize}

    \item \textbf{Esquema Gráfico:}
    
    El diagrama muestra cómo el ángulo $\beta$ determina la distancia $n$ para que el frustum coincida con la altura $s/2$.

    \begin{center}
    \begin{tikzpicture}[scale=1.5]
        % Parámetros
        \def\angle{25} % beta/2 aprox
        \def\h{0.75}   % s/2
        \def\n{1.61}   % n calculado manualmente para evitar problemas con tan()
        \def\s{1.5}    % s = 2*h
        \def\front{\n}
        \def\back{\n + \s}
        \def\centerZ{\n + 0.5*\s}
        
        % Eje Z
        \draw[->] (-0.5,0) -- (5,0) node[right] {$-Z_{ec}$ (Distancia)};
        \draw[->] (0,-1.5) -- (0,1.5) node[above] {$Y_{ec}$};
        \filldraw (0,0) circle (2pt) node[left] {Ojo ($o_{ec}$)};
        
        % Cono de visión (FOV)
        \draw[red, thick, dashed] (0,0) -- (4.5, {4.5*tan(\angle)});
        \draw[red, thick, dashed] (0,0) -- (4.5, {-4.5*tan(\angle)});
        
        % Arco beta
        \draw[red] (0.8, -0.3) arc (-20:20:0.9);
        \node[red] at (1.1, 0) {$\beta$};
        
        % Cubo
        \draw[blue, thick, fill=blue!10] (\front, -\h) rectangle (\back, \h);
        \node[blue] at (\centerZ, 0) {Cubo};
        
        % Planos n y f
        \draw[thick] (\front, -1.2) -- (\front, 1.2) node[above] {$n$};
        \draw[thick] (\back, -1.2) -- (\back, 1.2) node[above] {$f$};
        
        % Cotas y relaciones
        \draw[<->] (\front, 0.1) -- (\front, \h) node[midway, left] {$t=s/2$};
        \draw[<->] (0, -1.2) -- (\front, -1.2) node[midway, fill=white] {$n$};
        
        % Triángulo explicativo
        \draw[gray, dotted] (0,0) -- (\front, 0);
        \node[gray, scale=0.6] at ({0.5*\front}, 0.2) {Cateto adyacente ($n$)};
        \node[gray, scale=0.6, rotate=90] at ({\front+0.2}, {0.5*\h}) {Opuesto ($s/2$)};
    \end{tikzpicture}
    \end{center}

    \item \textbf{Resumen de Fórmulas:}
    \[ o_z = c_z + \frac{s}{2} \left( 1 + \cot\frac{\beta}{2} \right) \]
    \[ n = \frac{s}{2} \cot\frac{\beta}{2} \]
    \[ f = n + s \]
    \[ r = t = \frac{s}{2}, \quad l = b = -\frac{s}{2} \]
\end{enumerate}
\end{solucion}   
\section{Sesión 7}

\begin{ejercicio}
    \textbf{Implementación de Componentes Especulares (Phong y Blinn-Phong).}
    
    Escribe el código en GDScript para dos funciones que calculen la reflectividad debida a la componente pseudo-especular de los modelos de iluminación local:
    
    \begin{enumerate}
        \item \textbf{Modelo de Phong:} Evaluar la expresión $f_{ph}$ (Ecuación 6).
        \item \textbf{Modelo de Blinn-Phong:} Evaluar la expresión $f_{bp}$ (Ecuación 7).
    \end{enumerate}
    
    Ambas funciones recibirán como parámetros:
    \begin{itemize}
        \item Los vectores unitarios: Normal en el punto ($\mathbf{n}_p$), vector hacia el observador ($\mathbf{v}$) y vector hacia la fuente de luz ($\mathbf{l}_i$).
        \item El exponente de brillo $e$ (shininess).
        \item El coeficiente especular $k_{s}$ (o $k_{ph}/k_{bp}$).
    \end{itemize}
    
    La función debe devolver un valor de tipo \texttt{float} que represente la intensidad de la luz reflejada especularmente.

    \vspace{0.5cm}
    
    % Diagrama vectorial de los modelos
    \begin{center}
    \begin{tikzpicture}[scale=1.5, >=stealth]
        % Superficie
        \draw[thick] (-2,0) -- (2,0) node[right] {Superficie};
        \fill[gray!20] (-2,0) rectangle (2,-0.2);
        \fill (0,0) circle (1.5pt) node[below] {$p$};
        
        % Vectores base
        \draw[->, thick, blue] (0,0) -- (0,1.5) node[above] {$\mathbf{n}_p$};
        \draw[->, thick, orange] (0,0) -- (-1.2, 1.2) node[above left] {$\mathbf{l}_i$};
        \draw[->, thick, purple] (0,0) -- (1.5, 0.8) node[above right] {$\mathbf{v}$};
        
        % Vector Reflejado (Phong)
        \draw[->, thick, red, dashed] (0,0) -- (1.2, 1.2) node[above right] {$\mathbf{r}_i$};
        \draw (-0.3, 0.4) arc (135:90:0.4);
        \draw (0.3, 0.4) arc (45:90:0.4);
        \node at (0, 1.8) [font=\small, color=red] {Phong: $\alpha = \angle(\mathbf{r}_i, \mathbf{v})$};
        
        % Vector Halfway (Blinn-Phong)
        % L = (-1.2, 1.2), V = (1.5, 0.8). Sum approx (0.3, 2). Normalized approx up.
        \draw[->, thick, teal, dashed] (0,0) -- (0.2, 1.4) node[above] {$\mathbf{h}_i$};
        \node at (0.2, 0.5) [font=\small, color=teal, right] {Blinn: $\gamma = \angle(\mathbf{n}_p, \mathbf{h}_i)$};
        
    \end{tikzpicture}
    \captionof{figure}{Esquema de vectores para Phong ($\mathbf{r}_i$) y Blinn-Phong ($\mathbf{h}_i$).}
    \end{center}
\end{ejercicio}

\begin{solucion}
    A continuación se detalla el procedimiento geométrico y la implementación en código GDScript para ambos modelos.
    
    \begin{enumerate}
        \item \textbf{Modelo de Sombreado de Phong ($f_{ph}$)}
        
        El modelo de Phong calcula el brillo especular basándose en el ángulo entre el vector de visión $\mathbf{v}$ y el vector de reflexión perfecta de la luz $\mathbf{r}_i$.
        
        \textit{Fórmulas requeridas:}
        \begin{itemize}
            \item Vector de reflexión: $\mathbf{r}_i = 2(\mathbf{n}_p \cdot \mathbf{l}_i)\mathbf{n}_p - \mathbf{l}_i$.
            \item Condición de luz incidente: $d_i = 1$ si $\mathbf{n}_p \cdot \mathbf{l}_i > 0$, de lo contrario $0$.
            \item Intensidad: $I = k_{ph} \cdot (\max(0, \mathbf{r}_i \cdot \mathbf{v}))^e$.
        \end{itemize}

        \textbf{Código GDScript:}
\begin{lstlisting}
func calcular_phong_especular(n: Vector3, v: Vector3, l: Vector3, e: float, k_ph: float) -> float:
    # 1. Calcular el producto punto entre la normal y la luz (Lambert)
    var n_dot_l : float = n.dot(l)
    
    # 2. Si la luz está detrás de la superficie, no hay especularidad
    if n_dot_l <= 0.0:
        return 0.0
        
    # 3. Calcular el vector reflejado r
    # Fórmula: r = 2 * (n . l) * n - l
    # En GDScript se puede usar reflect(), pero ojo: reflect devuelve 
    # el vector reflejado dada la dirección incidente y la normal. 
    # La fórmula manual es más explícita para teoría.
    var r : Vector3 = (2.0 * n_dot_l * n - l).normalized()
    
    # 4. Calcular el factor especular (r . v)^e
    var r_dot_v : float = max(0.0, r.dot(v))
    var specular : float = pow(r_dot_v, e)
    
    # 5. Devolver intensidad final ponderada por k_ph
    return k_ph * specular
\end{lstlisting}
        
        \vspace{0.5cm}
        
        \item \textbf{Modelo de Blinn-Phong ($f_{bp}$)}
        
        El modelo de Blinn-Phong optimiza el cálculo y suaviza el resultado utilizando el vector intermedio o \textit{halfway vector} $\mathbf{h}_i$, que es la bisectriz entre la luz $\mathbf{l}_i$ y la visión $\mathbf{v}$.
        
        \textit{Fórmulas requeridas:}
        \begin{itemize}
            \item Vector Halfway: $\mathbf{h}_i = \frac{\mathbf{l}_i + \mathbf{v}}{||\mathbf{l}_i + \mathbf{v}||}$.
            \item Intensidad: $I = k_{bp} \cdot (\mathbf{n}_p \cdot \mathbf{h}_i)^e$.
        \end{itemize}

        \textbf{Código GDScript:}
\begin{lstlisting}
func calcular_blinn_phong_especular(n: Vector3, v: Vector3, l: Vector3, e: float, k_bp: float) -> float:
    # 1. Calcular el producto punto N.L para descartar luz trasera
    var n_dot_l : float = n.dot(l)
    
    if n_dot_l <= 0.0:
        return 0.0
        
    # 2. Calcular el vector halfway (bisectriz) h
    # Es la suma de L y V, normalizada
    var h : Vector3 = (l + v).normalized()
    
    # 3. Calcular el producto punto entre la normal y el halfway vector
    var n_dot_h : float = max(0.0, n.dot(h))
    
    # 4. Elevar a la potencia (exponente de brillo)
    var specular : float = pow(n_dot_h, e)
    
    # 5. Devolver resultado ponderado
    return k_bp * specular
\end{lstlisting}
    \end{enumerate}
    
    \textit{Nota técnica:} En GDScript, la clase \texttt{Vector3} asume que los vectores ya están normalizados si el enunciado dice ''vectores unitarios''. Si no se garantiza, se debería llamar a \texttt{.normalized()} sobre los parámetros de entrada antes de operar.
\end{solucion}

\begin{ejercicio}
    \textbf{Cálculo de máximos de intensidad y visibilidad en una esfera.}

    Supongamos una esfera de radio unidad centrada en el origen.
    \begin{itemize}
        \item Se ilumina con una fuente de luz puntual en $\mathbf{p} = (0, 2, 0)$.
        \item El observador está situado en $\mathbf{o} = (2, 0, 0)$.
    \end{itemize}
    
    Determinar razonadamente el punto de la superficie donde el brillo será máximo y si dicho punto es visible para el observador para los siguientes casos:
    \begin{enumerate}
        \item Componente difusa (Lambertiana).
        \item Componente pseudo-especular de Phong.
        \item Componente pseudo-especular de Blinn-Phong.
    \end{enumerate}

    \begin{center}
    \begin{tikzpicture}[scale=1.5]
        % Ejes
        \draw[->] (-1.5,0) -- (2.5,0) node[right] {$X$};
        \draw[->] (0,-1.5) -- (0,2.5) node[above] {$Y$};
        
        % Esfera (círculo en 2D pues Z=0 para los puntos de interés)
        \draw[thick] (0,0) circle (1);
        \node at (-0.2,-0.2) {$C(0,0)$};
        
        % Luz
        \fill[yellow] (0,2) circle (0.1) node[right] {Luz $(0,2)$};
        \draw[dashed, yellow!80!black] (0,2) -- (0,1);
        
        % Observador
        \fill[blue] (2,0) circle (0.05) node[below] {Obs $(2,0)$};
        \draw[dashed, blue] (2,0) -- (1,0);
        \draw[dashed, blue] (2,0) -- (0.5, 0.866); % Tangente aprox
        \draw[dashed, blue] (2,0) -- (0.5, -0.866);
        
        % Punto Difuso
        \fill[red] (0,1) circle (0.05) node[below left] {$P_{dif}$};
        
        % Punto Especular
        \fill[green!60!black] (0.707,0.707) circle (0.05) node[above right] {$P_{esp}$};
        
        % Vector Normal en P_esp
        \draw[->, thick] (0.707,0.707) -- (1.0, 1.0) node[right] {$\mathbf{n}$};
        
    \end{tikzpicture}
    \captionof{figure}{Diagrama de la escena en el plano $XY$ ($z=0$).}
    \end{center}
\end{ejercicio}

\begin{solucion}
    Analizaremos cada caso paso a paso. Dado que tanto la luz como el observador están en el plano $XY$ ($z=0$) y la esfera está centrada en el origen, los puntos de máximo brillo estarán necesariamente en el círculo máximo del plano $XY$.
    
    Datos geométricos generales para un punto $P(x,y,z)$ en la superficie de la esfera unitaria:
    \begin{itemize}
        \item Radio $R=1$, Centro $C=(0,0,0)$.
        \item La normal en la superficie es $\mathbf{n}_p = P - C = (x,y,z)$.
        \item Vector hacia la luz: $\mathbf{l} = \text{normalizar}(\mathbf{p} - P)$.
        \item Vector hacia el observador: $\mathbf{v} = \text{normalizar}(\mathbf{o} - P)$.
    \end{itemize}

    \textbf{Condición de Visibilidad:} 
    Un punto $P$ es visible si el ángulo entre la normal $\mathbf{n}_p$ y el vector de visión $\mathbf{v}$ es menor de 90 grados, es decir, $\mathbf{n}_p \cdot \mathbf{v} > 0$.
    
    Analicemos el horizonte de visibilidad para el observador en $(2,0,0)$:
    $$ \mathbf{v}_{aprox} \approx (2,0,0) - (x,y,z) = (2-x, -y, -z) $$
    $$ \mathbf{n}_p \cdot \mathbf{v}_{aprox} \propto (x,y,z) \cdot (2-x, -y, -z) = 2x - (x^2+y^2+z^2) = 2x - 1 $$
    La condición $\mathbf{n}_p \cdot \mathbf{v} > 0 \implies 2x - 1 > 0 \implies x > 0.5$.
    \textit{Cualquier punto con coordenada $x \le 0.5$ está oculto por el horizonte de la esfera.}

    \begin{enumerate}
        \item \textbf{Componente Difusa (Lambertiana)}
        
        La intensidad difusa es proporcional a $\mathbf{n}_p \cdot \mathbf{l}$. El brillo es máximo cuando la normal apunta directamente a la luz ($\mathbf{n}_p \parallel \mathbf{l}$).
        
        \begin{itemize}
            \item Dirección desde el centro a la luz: $(0,2,0) - (0,0,0) = (0,2,0)$.
            \item El punto de la superficie en esa dirección es $P_{dif} = (0, 1, 0)$.
            \item \textbf{Visibilidad:} La coordenada $x$ de $P_{dif}$ es $0$.
            \item Como $0 \le 0.5$, el punto \textbf{NO es visible}. Está en la parte superior de la esfera, pero el observador, situado a la derecha, solo ve hasta $x > 0.5$.
        \end{itemize}

        \item \textbf{Componente Pseudo-especular (Phong)}
        
        La intensidad es proporcional a $(\mathbf{r} \cdot \mathbf{v})^e$, donde $\mathbf{r}$ es el reflejo de la luz sobre la normal. El máximo ocurre cuando $\mathbf{r} = \mathbf{v}$ (reflexión perfecta). Esto implica que la normal $\mathbf{n}_p$ debe ser la bisectriz del ángulo formado por el vector luz $\mathbf{l}$ y el vector visión $\mathbf{v}$.
        
        Debido a la simetría del problema (Luz en eje Y, Observador en eje X, distancias iguales al origen), el punto debe estar en la bisectriz del primer cuadrante ($x=y$).
        
        \begin{itemize}
            \item Punto en la esfera a 45 grados: $P_{esp} = (\cos(45^\circ), \sin(45^\circ), 0) = \left(\frac{\sqrt{2}}{2}, \frac{\sqrt{2}}{2}, 0\right) \approx (0.707, 0.707, 0)$.
            \item Comprobación geométrica: La normal en este punto apunta a $(1,1)$. La luz está en $(0,2)$ y el ojo en $(2,0)$. El vector normal divide simétricamente el ángulo entre la luz y el ojo.
            \item \textbf{Visibilidad:} La coordenada $x$ de $P_{esp}$ es $0.707$.
            \item Como $0.707 > 0.5$, el punto \textbf{SÍ es visible}. El brillo especular aparecerá en el ''hombro'' de la esfera mirando hacia el observador.
        \end{itemize}

        \item \textbf{Modelo de Blinn-Phong}
        
        La intensidad es proporcional a $(\mathbf{n}_p \cdot \mathbf{h})^e$, donde $\mathbf{h}$ (halfway vector) es la bisectriz entre $\mathbf{l}$ y $\mathbf{v}$. El máximo ocurre cuando la normal $\mathbf{n}_p$ coincide con $\mathbf{h}$.
        
        \begin{itemize}
            \item Geométricamente, la condición ''la normal coincide con la bisectriz de L y V'' es idéntica a la condición de reflexión perfecta del modelo de Phong descrita arriba.
            \item Por tanto, el punto de máximo brillo es el mismo: $P_{bp} = \left(\frac{\sqrt{2}}{2}, \frac{\sqrt{2}}{2}, 0\right)$.
            \item \textbf{Visibilidad:} Al ser el mismo punto, \textbf{SÍ es visible}.
        \end{itemize}
    \end{enumerate}
\end{solucion}

\begin{ejercicio}
    \textbf{Evaluación de la BRDF de Microfacetas (GGX).}

    Escribe el código en GDScript de una función para calcular la reflectividad debida a la BRDF de microfacetas GGX, evaluando la expresión de $f_{ggx}$ (Ecuación 10).
    
    La función recibirá los siguientes parámetros:
    \begin{itemize}
        \item Vectores unitarios: Dirección de iluminación ($\mathbf{w}_i$), dirección de visión ($\mathbf{w}_o$), tangente X ($\mathbf{t}_x$), tangente Y ($\mathbf{t}_y$) y normal de la macrosuperficie ($\mathbf{n}_x$).
        \item Valores de rugosidad: $\alpha_x$ y $\alpha_y$ (tipo float).
    \end{itemize}
    
    La función debe devolver un valor de tipo \texttt{float}.

    \vspace{0.5cm}
    
    % Diagrama de Microfacetas GGX
    \begin{center}
    \begin{tikzpicture}[scale=2, >=stealth]
        % Macrosuperficie
        \draw[thick] (-2,0) -- (2,0) node[right] {Macrosuperficie};
        
        % Microfaceta (curva representativa)
        \draw[thick, red!70!black, smooth] plot coordinates {(-1.5,0) (-1, 0.3) (-0.5, -0.1) (0, 0) (0.5, 0.4) (1, 0.1) (1.5, 0)};
        
        % Punto de evaluación
        \fill (0,0) circle (1pt);
        
        % Vectores principales
        \draw[->, thick, blue] (0,0) -- (0,1.2) node[left] {$\mathbf{n}_x$};
        \draw[->, thick, orange] (0,0) -- (-0.8, 1.0) node[above] {$\mathbf{w}_i$};
        \draw[->, thick, purple] (0,0) -- (1.2, 0.8) node[above] {$\mathbf{w}_o$};
        
        % Vector Halfway (Normal de la microfaceta activa)
        \draw[->, thick, teal, dashed] (0,0) -- (0.2, 1.3) node[right] {$\mathbf{m} = \mathbf{h}$};
        
        % Anotación
        \node at (1.5, -0.3) [font=\small, color=gray] {$\mathbf{h} = \frac{\mathbf{w}_i + \mathbf{w}_o}{||\mathbf{w}_i + \mathbf{w}_o||}$};
    \end{tikzpicture}
    \captionof{figure}{Geometría de microfacetas: El vector $\mathbf{h}$ actúa como la normal de la microfaceta ($\mathbf{m}$) que refleja $\mathbf{w}_i$ hacia $\mathbf{w}_o$.}
    \end{center}
\end{ejercicio}

\begin{solucion}
    Para implementar la BRDF GGX completa, debemos desglosar la Ecuación 10 en sus tres componentes principales: la Distribución de Normales ($D$), el Enmascaramiento-Sombreado ($G$) y el término de Fresnel ($F$).
    
    \begin{enumerate}
        \item \textbf{Cálculo del Vector Halfway ($\mathbf{h}$):}
        Es la bisectriz entre el vector de luz y el de visión. Representa la orientación que debe tener una microfaceta para reflejar la luz perfectamente hacia el observador.
        $$ \mathbf{h} = \frac{\mathbf{w}_i + \mathbf{w}_o}{||\mathbf{w}_i + \mathbf{w}_o||} $$
        
        \item \textbf{Distribución de Normales Anisotrópica ($D$):}
        Evaluamos la probabilidad de que una microfaceta esté alineada con $\mathbf{h}$. Usamos la fórmula GGX anisotrópica (Ecuación de transparencia 75):
        $$ D(\mathbf{h}) = \frac{1}{\pi \alpha_x \alpha_y \left( (\frac{\mathbf{h} \cdot \mathbf{t}_x}{\alpha_x})^2 + (\frac{\mathbf{h} \cdot \mathbf{t}_y}{\alpha_y})^2 + (\mathbf{h} \cdot \mathbf{n}_x)^2 \right)^2} $$
        
        \item \textbf{Enmascaramiento y Sombreado ($G_2$):}
        Usamos la aproximación \textit{Height Correlated Masking and Shadowing} (Ecuación de transparencia 77). Se define mediante una función auxiliar $\Lambda(w)$:
        $$ \Lambda(\mathbf{w}) = \frac{1}{2} \left( -1 + \sqrt{1 + \frac{\alpha_x^2 x^2 + \alpha_y^2 y^2}{z^2}} \right) $$
        Donde $x, y, z$ son las proyecciones del vector $\mathbf{w}$ sobre $\mathbf{t}_x, \mathbf{t}_y, \mathbf{n}_x$.
        $$ G_2 = \frac{1}{1 + \Lambda(\mathbf{w}_i) + \Lambda(\mathbf{w}_o)} $$
        
        \item \textbf{Término de Fresnel ($F$):}
        Usamos la aproximación de Schlick (Ecuación de transparencia 78). Aunque el enunciado no proporciona el índice de refracción ($f_0$), es necesario para la ecuación. Asumiremos un valor estándar de $0.04$ (dieléctrico común) para completar el cálculo.
        $$ F \approx f_0 + (1 - f_0)(1 - (\mathbf{w}_i \cdot \mathbf{h}))^5 $$
        
        \item \textbf{Combinación Final ($f_{ggx}$):}
        $$ f_{ggx} = \frac{F \cdot D \cdot G_2}{4 (\mathbf{w}_i \cdot \mathbf{n}_x) (\mathbf{w}_o \cdot \mathbf{n}_x)} $$
    \end{enumerate}

    \textbf{Implementación en GDScript:}

\begin{lstlisting}
func calcular_brdf_ggx(wi: Vector3, wo: Vector3, tx: Vector3, ty: Vector3, nx: Vector3, ax: float, ay: float) -> float:
    # 1. Calcular el vector Halfway (h)
    var h: Vector3 = (wi + wo).normalized()
    
    # Pre-cálculo de productos punto necesarios
    var n_dot_wi = max(0.0001, nx.dot(wi)) # Evitar división por cero
    var n_dot_wo = max(0.0001, nx.dot(wo))
    var n_dot_h  = max(0.0, nx.dot(h))
    var h_dot_wi = max(0.0, h.dot(wi))
    
    # Proyecciones para anisotropía
    var h_dot_tx = h.dot(tx)
    var h_dot_ty = h.dot(ty)
    
    # 2. Calcular Distribución D (GGX Anisotrópica)
    var term_x = pow(h_dot_tx / ax, 2)
    var term_y = pow(h_dot_ty / ay, 2)
    var term_z = pow(n_dot_h, 2)
    
    var denom_d = PI * ax * ay * pow(term_x + term_y + term_z, 2)
    var D = 1.0 / max(0.0001, denom_d)
    
    # 3. Calcular Geometría G2 (Height Correlated)
    # Función Lambda auxiliar inline para wi
    var wi_x = wi.dot(tx) * ax
    var wi_y = wi.dot(ty) * ay
    var wi_z = n_dot_wi
    var lambda_wi = 0.5 * (-1.0 + sqrt(1.0 + (pow(wi_x, 2) + pow(wi_y, 2)) / pow(wi_z, 2)))
    
    # Función Lambda auxiliar inline para wo
    var wo_x = wo.dot(tx) * ax
    var wo_y = wo.dot(ty) * ay
    var wo_z = n_dot_wo
    var lambda_wo = 0.5 * (-1.0 + sqrt(1.0 + (pow(wo_x, 2) + pow(wo_y, 2)) / pow(wo_z, 2)))
                    
    var G2 = 1.0 / (1.0 + lambda_wi + lambda_wo)
    
    # 4. Calcular Fresnel F (Aproximación de Schlick)
    var f0 = 0.04 # Valor asumido para dieléctricos si no se provee
    var F = f0 + (1.0 - f0) * pow(1.0 - h_dot_wi, 5)
    
    # 5. Resultado final combinado
    var numerador = F * D * G2
    var denominador = 4.0 * n_dot_wi * n_dot_wo
    
    return numerador / max(0.0001, denominador)
\end{lstlisting}
\end{solucion}

\section{Sesión 8}

\begin{ejercicio}
%\textbf{Problema 8.1: Coordenadas de textura.}
Supongamos que se desea crear una malla indexada para un cubo, de forma que deseamos aplicar una textura que incluya las caras de un dado. Para ello disponemos de una imagen de textura que tiene una relación de aspecto 4:3.
\begin{center}
% Dibujo de la cruz de un dado (desplegable de cubo) usando TikZ
\begin{tikzpicture}[scale=1.5, dot/.style={circle, fill=black, minimum size=4pt, inner sep=0pt}]
    % Definición de las caras (cuadrados)
    % Centro (1), Arriba (5), Abajo (2), Izquierda (3), Derecha (4), Extrema Izquierda (6)
    \foreach \x/\y in {0/0, 0/1, 0/-1, -1/0, 1/0, -2/0} {
        \draw[thick] (\x-0.5,\y-0.5) rectangle (\x+0.5,\y+0.5);
    }

    % Cara Central (1 punto)
    \node[dot] at (0,0) {};

    % Cara Superior (5 puntos)
    \node[dot] at (0,1) {};
    \node[dot] at (-0.25,1.25) {};
    \node[dot] at (0.25,1.25) {};
    \node[dot] at (-0.25,0.75) {};
    \node[dot] at (0.25,0.75) {};

    % Cara Inferior (2 puntos - diagonal invertida según imagen)
    \node[dot] at (-0.25,-0.75) {};
    \node[dot] at (0.25,-1.25) {};

    % Cara Derecha (4 puntos)
    \node[dot] at (0.75,0.25) {};
    \node[dot] at (1.25,0.25) {};
    \node[dot] at (0.75,-0.25) {};
    \node[dot] at (1.25,-0.25) {};

    % Cara Izquierda (3 puntos - diagonal)
    \node[dot] at (-1.25,0.25) {};
    \node[dot] at (-1,0) {};
    \node[dot] at (-0.75,-0.25) {};

    % Cara Extrema Izquierda (6 puntos)
    \node[dot] at (-2.25,0.25) {};
    \node[dot] at (-2,0.25) {};
    \node[dot] at (-1.75,0.25) {};
    \node[dot] at (-2.25,-0.25) {};
    \node[dot] at (-2,-0.25) {};
    \node[dot] at (-1.75,-0.25) {};

\end{tikzpicture}
\end{center}
\begin{enumerate}
    \item Describe razonadamente cuántos vértices (como mínimo) tendrá el modelo.
    \item Escribe la tabla de coordenadas de vértices, la tabla de coordenadas de textura y la tabla de triángulos. Ten en cuenta que el cubo tiene lado unidad y su centro está en $(0.5, 0.5, 0.5)$.
    \item Dibuja un esquema de la textura en la cual cada vértice del modelo aparezca etiquetado con su número de vértice más sus coordenadas de textura.
\end{enumerate}
\end{ejercicio}

\begin{solucion}
\begin{enumerate} 

La resolución del ejercicio es la siguiente:
\item Número de Vértices del Modelo

Aunque un cubo geométrico estándar tiene 8 vértices espaciales (esquinas), en informática gráfica, un vértice en una malla indexada se define como una tupla única de atributos: $(x, y, z, u, v, \dots)$. Si un mismo punto geométrico (esquina del cubo) necesita tener dos coordenadas de textura distintas (por ejemplo, en una costura donde la textura se corta), el vértice debe duplicarse.

Observando la distribución de la textura en cruz proporcionada en las diapositivas, la imagen tiene un aspect ratio 4:3, lo que implica una rejilla de $4 \times 3$ caras. La disposición es:
\begin{itemize}
    \item Fila superior: Cara 5.
    \item Fila media: Caras 6, 3, 1, 4.
    \item Fila inferior: Cara 2.
\end{itemize}

Para calcular el número mínimo de vértices, analizamos la conectividad en el espacio UV:
\begin{itemize}
    \item Si tratamos cada cara como un cuadrado independiente, tendríamos $6 \times 4 = 24$ vértices.
    \item Restamos los vértices que se comparten en las aristas continuas en la textura (donde no hay corte UV). Las conexiones visuales son: 6-3, 3-1, 1-4, 5-1 y 1-2.
    \item Hay 5 aristas compartidas. Cada arista fusiona 2 pares de vértices.
    \item Total vértices = $24 - (5 \text{ aristas} \times 2 \text{ vértices}) = 14$.
\end{itemize}

Por tanto, el modelo necesita \textbf{14 vértices} únicos.
\vspace{1cm}
\item  Tablas de Definición del Modelo

Asumimos el sistema de referencia donde la cara 1 es el Frontal ($z=1$), la cara 5 es Arriba ($y=1$), la cara 2 es Abajo ($y=0$), la cara 3 es Izquierda ($x=0$), la cara 4 es Derecha ($x=1$) y la cara 6 es Atrás ($z=0$).
El cubo va de $(0,0,0)$ a $(1,1,1)$.

Dividimos el dominio de textura $u \in [0,1], v \in [0,1]$ según la rejilla 4x3\footnote{Es en base al enunciado.}:
\begin{itemize}
    \item Paso en $u$: $1/4 = 0.25$. Columnas: $0, 0.25, 0.5, 0.75, 1.0$.
    \item Paso en $v$: $1/3 \approx 0.333$. Filas: $0, 0.33, 0.66, 1.0$.
\end{itemize}

\textit{Nota:} Las divisiones que se hacen de u y v corresponden a cada vértice, de manera que tan solo tenemos que imaginar que la textura es como una tabla, si vemos en la cara 5 (arriba) esta entre u=0.5 a u=0.75 y v=0.66 a v=1.0. En la tabla se hace referencia a top-esquina, lo que se conoce como top-left en inglés, por ende, debemos debemos de tener en cuenta que la coordenada v=1.0 es la parte superior de la textura y v=0.0 es la parte inferior. Se le debe de atribuir u=0.5 y v=1.0 a la esquina superior izquierda de la cara 5 (arriba).


\vspace{1cm}
\underline{Tabla de Vértices (Geometría + Textura)} \\\\
Ordenamos los vértices recorriendo la textura de arriba a abajo y de izquierda a derecha.

\begin{center}
\begin{tabular}{|c|c|c|c|}
\hline
\textbf{Índice ($i$)} & \textbf{Posición $(x, y, z)$} & \textbf{Coord. Textura $(u, v)$} & \textbf{Descripción (UV)} \\ \hline
0 & $(0, 1, 0)$ & $(0.50, 1.00)$ & Top-Esq Cara 5 \\ \hline
1 & $(1, 1, 0)$ & $(0.75, 1.00)$ & Top-Der Cara 5 \\ \hline
2 & $(1, 1, 0)$ & $(0.00, 0.66)$ & Top-Esq Cara 6 \\ \hline
3 & $(0, 1, 0)$ & $(0.25, 0.66)$ & Top-Der 6 / Top-Esq 3 \\ \hline
4 & $(0, 1, 1)$ & $(0.50, 0.66)$ & Top-Der 3 / Top-Esq 1 / Bot-Esq 5 \\ \hline
5 & $(1, 1, 1)$ & $(0.75, 0.66)$ & Top-Der 1 / Top-Esq 4 / Bot-Der 5 \\ \hline
6 & $(1, 1, 0)$ & $(1.00, 0.66)$ & Top-Der 4 \\ \hline
7 & $(1, 0, 0)$ & $(0.00, 0.33)$ & Bot-Esq Cara 6 \\ \hline
8 & $(0, 0, 0)$ & $(0.25, 0.33)$ & Bot-Der 6 / Bot-Esq 3 \\ \hline
9 & $(0, 0, 1)$ & $(0.50, 0.33)$ & Bot-Der 3 / Bot-Esq 1 / Top-Esq 2 \\ \hline
10 & $(1, 0, 1)$ & $(0.75, 0.33)$ & Bot-Der 1 / Bot-Esq 4 / Top-Der 2 \\ \hline
11 & $(1, 0, 0)$ & $(1.00, 0.33)$ & Bot-Der 4 \\ \hline
12 & $(0, 0, 0)$ & $(0.50, 0.00)$ & Bot-Esq Cara 2 \\ \hline
13 & $(1, 0, 0)$ & $(0.75, 0.00)$ & Bot-Der Cara 2 \\ \hline
\end{tabular}
\end{center}
\vspace{1cm}
\underline{Tabla de Triángulos} \\\\
Definimos dos triángulos por cara (sentido antihorario visto desde fuera).

\begin{center}
\begin{tabular}{|c|c|c|}
\hline
\textbf{Cara (Dado)} & \textbf{Triángulo 1 $(v_a, v_b, v_c)$} & \textbf{Triángulo 2 $(v_a, v_c, v_d)$} \\ \hline
5 (Arriba) & $(0, 1, 4)$ & $(1, 5, 4)$ \\ \hline
6 (Atrás) & $(2, 3, 7)$ & $(3, 8, 7)$ \\ \hline
3 (Izq) & $(3, 4, 8)$ & $(4, 9, 8)$ \\ \hline
1 (Frente) & $(4, 5, 9)$ & $(5, 10, 9)$ \\ \hline
4 (Der) & $(5, 6, 10)$ & $(6, 11, 10)$ \\ \hline
2 (Abajo) & $(9, 10, 12)$ & $(10, 13, 12)$ \\ \hline
\end{tabular}
\end{center}

\textit{Nota:} Usamos orden horario.

\vspace{1cm}
\item Esquema de la Textura

A continuación se muestra el espacio de coordenadas de textura $(u,v)$ con los vértices etiquetados según la tabla anterior.

\begin{center}
\begin{tikzpicture}[scale=3]
    % Draw grid
    \draw[step=0.25cm,gray,very thin] (0,0) grid (1,1);
    
    % Draw axes
    \draw[->] (0,0) -- (1.1,0) node[right] {$u$};
    \draw[->] (0,-0.05) -- (0,1.1) node[above] {$v$};
    
    % Ticks
    \foreach \x in {0, 0.25, 0.5, 0.75, 1} \draw (\x,1pt) -- (\x,-1pt) node[anchor=north, font=\tiny] {\x};
    \foreach \y/\yl in {0/0, 0.33/0.33, 0.66/0.66, 1/1} \draw (1pt,\y) -- (-1pt,\y) node[anchor=east, font=\tiny] {\yl};

    % Draw Texture Map Outline (The Cross)
    \draw[thick, blue] (0, 0.33) -- (1, 0.33) -- (1, 0.66) -- (0.75, 0.66) -- (0.75, 1) -- (0.5, 1) -- (0.5, 0.66) -- (0, 0.66) -- cycle;
    \draw[thick, blue] (0.5, 0.33) -- (0.5, 0) -- (0.75, 0) -- (0.75, 0.33);
    
    % Fill Faces labels
    \node at (0.125, 0.5) {6};
    \node at (0.375, 0.5) {3};
    \node at (0.625, 0.5) {1};
    \node at (0.875, 0.5) {4};
    \node at (0.625, 0.83) {5};
    \node at (0.625, 0.16) {2};

    % Plot Vertices
    % Row v=1
    \filldraw [red] (0.5, 1) circle (0.5pt) node[above right, font=\tiny] {$v_0$};
    \filldraw [red] (0.75, 1) circle (0.5pt) node[above left, font=\tiny] {$v_1$};
    
    % Row v=0.66
    \filldraw [red] (0, 0.66) circle (0.5pt) node[above right, font=\tiny] {$v_2$};
    \filldraw [red] (0.25, 0.66) circle (0.5pt) node[above right, font=\tiny] {$v_3$};
    \filldraw [red] (0.5, 0.66) circle (0.5pt) node[above left, font=\tiny] {$v_4$};
    \filldraw [red] (0.75, 0.66) circle (0.5pt) node[above right, font=\tiny] {$v_5$};
    \filldraw [red] (1, 0.66) circle (0.5pt) node[above left, font=\tiny] {$v_6$};

    % Row v=0.33
    \filldraw [red] (0, 0.33) circle (0.5pt) node[below right, font=\tiny] {$v_7$};
    \filldraw [red] (0.25, 0.33) circle (0.5pt) node[below right, font=\tiny] {$v_8$};
    \filldraw [red] (0.5, 0.33) circle (0.5pt) node[below right, font=\tiny, xshift=-2pt] {$v_9$};
    % Separation between v9 and v10
    \draw[dashed, gray!60!black, thick] (0.5,0.33) -- (0.75,0.33);
    \filldraw [red] (0.75, 0.33) circle (0.5pt) node[below left, font=\tiny, xshift=2pt] {$v_{10}$};
    \filldraw [red] (1, 0.33) circle (0.5pt) node[below left, font=\tiny] {$v_{11}$};

    % Row v=0
    \filldraw [red] (0.5, 0) circle (0.5pt) node[below right, font=\tiny, xshift=-4pt] {$v_{12}$};
    % Separation between v12 and v13
    \draw[dashed, gray!60!black, thick] (0.5,0) -- (0.75,0);
    \filldraw [red] (0.75, 0) circle (0.5pt) node[below left, font=\tiny, xshift=4pt] {$v_{13}$};

\end{tikzpicture}
\end{center}
\end{enumerate}

El código GDScript para definir las tablas de vértices, coordenadas de textura y triángulos es el siguiente:
\begin{lstlisting}[language=gdscript]
# Definición de los vértices: posición y coordenadas de textura
var vertices = [
    Vector3(0, 1, 0),   # v0
    Vector3(1, 1, 0),   # v1
    Vector3(1, 1, 0),   # v2
    Vector3(0, 1, 0),   # v3
    Vector3(0, 1, 1),   # v4
    Vector3(1, 1, 1),   # v5
    Vector3(1, 1, 0),   # v6
    Vector3(1, 0, 0),   # v7
    Vector3(0, 0, 0),   # v8
    Vector3(0, 0, 1),   # v9
    Vector3(1, 0, 1),   # v10
    Vector3(1, 0, 0),   # v11
    Vector3(0, 0, 0),   # v12
    Vector3(1, 0, 0),   # v13
]

var uvs = [
    Vector2(0.50, 1.00),   # v0
    Vector2(0.75, 1.00),   # v1
    Vector2(0.00, 0.66),   # v2
    Vector2(0.25, 0.66),   # v3
    Vector2(0.50, 0.66),   # v4
    Vector2(0.75, 0.66),   # v5
    Vector2(1.00, 0.66),   # v6
    Vector2(0.00, 0.33),   # v7
    Vector2(0.25, 0.33),   # v8
    Vector2(0.50, 0.33),   # v9
    Vector2(0.75, 0.33),   # v10
    Vector2(1.00, 0.33),   # v11
    Vector2(0.50, 0.00),   # v12
    Vector2(0.75, 0.00),   # v13
]

# Definición de los triángulos (índices de vértices) en orden horario (sentido antihorario visto desde fuera)
var triangles = [
    # Cara 5 (Arriba)
    0, 1, 4,
    1, 5, 4,
    # Cara 6 (Atrás)
    2, 3, 7,
    3, 8, 7,
    # Cara 3 (Izquierda)
    3, 4, 8,
    4, 9, 8,
    # Cara 1 (Frente)
    4, 5, 9,
    5, 10, 9,
    # Cara 4 (Derecha)
    5, 6, 10,
    6, 11, 10,
    # Cara 2 (Abajo)
    9, 10, 12,
    10, 13, 12,
]
\end{lstlisting}
\end{solucion}

\begin{ejercicio}
%\textbf{Problema 8.2:}
Considera de nuevo el cubo y la textura del problema anterior (un cubo de lado unidad con centro en $(0.5, 0.5, 0.5)$ y una textura de imagen con relación de aspecto 4:3 que despliega las caras de un dado). Supón que ahora queremos visualizar el cubo iluminado, para lo cual debemos asignar normales a los vértices.

Responde a estas cuestiones:
\begin{enumerate}
    \item Describe razonadamente si sería posible usar la misma tabla de vértices y la misma tabla de coordenadas de textura que en el problema anterior (donde se buscaba el número mínimo de vértices), o si es necesario usar tablas distintas.
    \item Si has respondido que no es posible usar las mismas tablas, escribe la nueva tabla de vértices, la nueva tabla de coordenadas de textura y la tabla de normales.
\end{enumerate}
\end{ejercicio}

\begin{solucion} La resolución del ejercicio es la siguiente:

\textbf{1. Análisis de la reutilización de la tabla de vértices}

Para responder a esta cuestión, debemos entender qué define un \textit{vértice} en el contexto del cauce gráfico (pipeline) cuando aplicamos iluminación.

En el problema anterior (8.1), buscábamos minimizar el espacio geométrico. Un cubo tiene geométricamente 8 esquinas. Si solo nos importara la posición $(x, y, z)$, podríamos definir solo 8 vértices y reutilizarlos mediante índices.

Sin embargo, para la iluminación (sombreado), necesitamos asociar un \textbf{vector normal} ($\vec{n}$) a cada vértice. El vector normal indica hacia dónde ''mira'' la superficie en ese punto para calcular cómo rebota la luz.

\begin{itemize}
    \item \textbf{El problema de la continuidad:} En una esfera suave, la normal en un vértice es el promedio de las caras adyacentes, permitiendo un sombreado suave (Gouraud).
    \item \textbf{El caso del cubo (aristas vivas):} Un cubo tiene aristas afiladas (no suaves). Consideremos una esquina del cubo, por ejemplo, la superior-derecha-frontal $(1, 1, 1)$.
    \begin{itemize}
        \item Para la cara \textbf{Frontal}, la normal debe apuntar hacia adelante: $\vec{n} = (0, 0, 1)$.
        \item Para la cara \textbf{Superior}, la normal debe apuntar hacia arriba: $\vec{n} = (0, 1, 0)$.
        \item Para la cara \textbf{Derecha}, la normal debe apuntar a la derecha: $\vec{n} = (1, 0, 0)$.
    \end{itemize}
\end{itemize}

Como un vértice en la memoria de la GPU es una estructura de datos única que contiene $\{Posici\acute{o}n, Normal, UV\}$, no podemos tener un solo vértice con tres normales distintas simultáneamente.

\vspace{0.3cm}
\begin{center}
\begin{tikzpicture}[scale=2]
    % Coordenadas del cubo
    \coordinate (O) at (0,0,0);
    \coordinate (A) at (1,0,0);
    \coordinate (B) at (1,1,0);
    \coordinate (C) at (0,1,0);
    \coordinate (D) at (0,0,1);
    \coordinate (E) at (1,0,1);
    \coordinate (F) at (1,1,1);
    \coordinate (G) at (0,1,1);

    % Caras
    \draw[fill=gray!20] (F) -- (G) -- (C) -- (B) -- cycle; % Top
    \draw[fill=gray!10] (F) -- (E) -- (A) -- (B) -- cycle; % Right
    \draw[fill=gray!30] (F) -- (G) -- (D) -- (E) -- cycle; % Front

    % Aristas
    \draw (F) -- (G);
    \draw (F) -- (E);
    \draw (F) -- (B);

    % Vector Normales en el vertice F
    \draw[->, ultra thick, blue] (F) -- +(0, 0.5, 0) node[above] {$\vec{n}_{top} (0,1,0)$};
    \draw[->, ultra thick, red] (F) -- +(0.5, 0, 0) node[right] {$\vec{n}_{right} (1,0,0)$};
    \draw[->, ultra thick, green!50!black] (F) -- +(0, 0, 1) node[below left] {$\vec{n}_{front} (0,0,1)$};

    \node at (F) [circle, fill, inner sep=1.5pt] {};
    \node[below right] at (F) {Vértice compartido geométricamente};
\end{tikzpicture}
\end{center}
\vspace{0.3cm}

\textbf{Conclusión:} \textbf{No es posible} usar la misma tabla reducida de 8 vértices. Es necesario duplicar los vértices en las costuras de las aristas. Necesitaremos vértices independientes para cada cara del cubo.
\\
Total de vértices necesarios: $6 \text{ caras} \times 4 \text{ vértices/cara} = \textbf{24 vértices}$.

\vspace{0.5cm}

\textbf{2. Definición de las nuevas tablas}

Para construir las tablas, asumiremos la disposición de textura ''en cruz'' típica para una relación de aspecto 4:3, tal como sugiere el enunciado del Problema 8.1.

\textbf{Esquema de la Textura (Relación 4:3):}
Dividimos la textura en una cuadrícula de $4 \times 3$.
\begin{itemize}
    \item Ancho de celda ($u$): $1/4 = 0.25$
    \item Alto de celda ($v$): $1/3 \approx 0.333$
\end{itemize}

\begin{center}
\begin{tikzpicture}[x=2cm, y=2cm]
    \draw[step=1cm, gray, very thin] (0,0) grid (4,3);
    
    % Dibujar la cruz del dado
    \draw[fill=orange!20] (0,1) rectangle (1,2) node[pos=.5] {Izq (L)};
    \draw[fill=orange!20] (1,1) rectangle (2,2) node[pos=.5] {Frente (F)};
    \draw[fill=orange!20] (2,1) rectangle (3,2) node[pos=.5] {Der (R)};
    \draw[fill=orange!20] (3,1) rectangle (4,2) node[pos=.5] {Tras (B)};
    \draw[fill=orange!20] (1,2) rectangle (2,3) node[pos=.5] {Arriba (T)};
    \draw[fill=orange!20] (1,0) rectangle (2,1) node[pos=.5] {Abajo (D)};
    
    % Ejes UV
    \draw[->] (0,0) -- (4.2,0) node[right] {$u$};
    \draw[->] (0,0) -- (0,3.2) node[above] {$v$};
    
    % Etiquetas coordenadas
    \foreach \x/\label in {0/0, 1/0.25, 2/0.5, 3/0.75, 4/1.0}
        \draw (\x, 1pt) -- (\x, -1pt) node[below, font=\tiny] {\label};
    \foreach \y/\label in {0/0, 1/0.33, 2/0.66, 3/1.0}
        \draw (1pt, \y) -- (-1pt, \y) node[left, font=\tiny] {\label};
\end{tikzpicture}
\end{center}

A continuación, definimos las tablas. Dado que el cubo tiene lado 1 y centro en $(0.5, 0.5, 0.5)$, las coordenadas van de $0.0$ a $1.0$ en los ejes X, Y, Z.

\textbf{Nota de notación:}
\begin{itemize}
    \item \textbf{Posición:} $(x, y, z)$
    \item \textbf{Normal:} $(nx, ny, nz)$
    \item \textbf{Textura:} $(u, v)$
\end{itemize}

Desglosaremos la tabla cara por cara (cada cara genera 4 vértices únicos).

\vspace{0.3cm}

\textbf{Tabla Completa de Vértices (Datos combinados)}

\begin{enumerate}[itemsep=2em]

    \item \textbf{Cara Frontal (Z = 1)}: Corresponde a la celda $(u \in [0.25, 0.5], v \in [0.33, 0.66])$.
    Normal $\vec{n} = (0, 0, 1)$.
    \begin{center}
    \begin{tabular}{|c|c|c|c|}
    \hline
    \textbf{Índice} & \textbf{Posición $(x,y,z)$} & \textbf{Normal $(nx,ny,nz)$} & \textbf{Textura $(u,v)$} \\ \hline
    0 & $(0, 0, 1)$ & $(0, 0, 1)$ & $(0.25, 0.33)$ \\ \hline
    1 & $(1, 0, 1)$ & $(0, 0, 1)$ & $(0.50, 0.33)$ \\ \hline
    2 & $(1, 1, 1)$ & $(0, 0, 1)$ & $(0.50, 0.66)$ \\ \hline
    3 & $(0, 1, 1)$ & $(0, 0, 1)$ & $(0.25, 0.66)$ \\ \hline
    \end{tabular}
    \end{center}

    \item \textbf{Cara Derecha (X = 1)}: Corresponde a la celda $(u \in [0.5, 0.75], v \in [0.33, 0.66])$.
    Normal $\vec{n} = (1, 0, 0)$.
    \begin{center}
    \begin{tabular}{|c|c|c|c|}
    \hline
    \textbf{Índice} & \textbf{Posición $(x,y,z)$} & \textbf{Normal $(nx,ny,nz)$} & \textbf{Textura $(u,v)$} \\ \hline
    4 & $(1, 0, 1)$ & $(1, 0, 0)$ & $(0.50, 0.33)$ \\ \hline
    5 & $(1, 0, 0)$ & $(1, 0, 0)$ & $(0.75, 0.33)$ \\ \hline
    6 & $(1, 1, 0)$ & $(1, 0, 0)$ & $(0.75, 0.66)$ \\ \hline
    7 & $(1, 1, 1)$ & $(1, 0, 0)$ & $(0.50, 0.66)$ \\ \hline
    \end{tabular}
    \end{center}

    \item \textbf{Cara Trasera (Z = 0)}: Corresponde a la celda $(u \in [0.75, 1.0], v \in [0.33, 0.66])$.
    Normal $\vec{n} = (0, 0, -1)$.
    \begin{center}
    \begin{tabular}{|c|c|c|c|}
    \hline
    \textbf{Índice} & \textbf{Posición $(x,y,z)$} & \textbf{Normal $(nx,ny,nz)$} & \textbf{Textura $(u,v)$} \\ \hline
    8 & $(1, 0, 0)$ & $(0, 0, -1)$ & $(0.75, 0.33)$ \\ \hline
    9 & $(0, 0, 0)$ & $(0, 0, -1)$ & $(1.00, 0.33)$ \\ \hline
    10 & $(0, 1, 0)$ & $(0, 0, -1)$ & $(1.00, 0.66)$ \\ \hline
    11 & $(1, 1, 0)$ & $(0, 0, -1)$ & $(0.75, 0.66)$ \\ \hline
    \end{tabular}
    \end{center}

    \item \textbf{Cara Izquierda (X = 0)}: Corresponde a la celda $(u \in [0.0, 0.25], v \in [0.33, 0.66])$.
    Normal $\vec{n} = (-1, 0, 0)$.
    \begin{center}
    \begin{tabular}{|c|c|c|c|}
    \hline
    \textbf{Índice} & \textbf{Posición $(x,y,z)$} & \textbf{Normal $(nx,ny,nz)$} & \textbf{Textura $(u,v)$} \\ \hline
    12 & $(0, 0, 0)$ & $(-1, 0, 0)$ & $(0.00, 0.33)$ \\ \hline
    13 & $(0, 0, 1)$ & $(-1, 0, 0)$ & $(0.25, 0.33)$ \\ \hline
    14 & $(0, 1, 1)$ & $(-1, 0, 0)$ & $(0.25, 0.66)$ \\ \hline
    15 & $(0, 1, 0)$ & $(-1, 0, 0)$ & $(0.00, 0.66)$ \\ \hline
    \end{tabular}
    \end{center}

    \item \textbf{Cara Superior (Y = 1)}: Corresponde a la celda superior central $(u \in [0.25, 0.5], v \in [0.66, 1.0])$.
    Normal $\vec{n} = (0, 1, 0)$.
    \begin{center}
    \begin{tabular}{|c|c|c|c|}
    \hline
    \textbf{Índice} & \textbf{Posición $(x,y,z)$} & \textbf{Normal $(nx,ny,nz)$} & \textbf{Textura $(u,v)$} \\ \hline
    16 & $(0, 1, 1)$ & $(0, 1, 0)$ & $(0.25, 0.66)$ \\ \hline
    17 & $(1, 1, 1)$ & $(0, 1, 0)$ & $(0.50, 0.66)$ \\ \hline
    18 & $(1, 1, 0)$ & $(0, 1, 0)$ & $(0.50, 1.00)$ \\ \hline
    19 & $(0, 1, 0)$ & $(0, 1, 0)$ & $(0.25, 1.00)$ \\ \hline
    \end{tabular}
    \end{center}

    \item \textbf{Cara Inferior (Y = 0)}: Corresponde a la celda inferior central $(u \in [0.25, 0.5], v \in [0.0, 0.33])$.
    Normal $\vec{n} = (0, -1, 0)$.
    \begin{center}
    \begin{tabular}{|c|c|c|c|}
    \hline
    \textbf{Índice} & \textbf{Posición $(x,y,z)$} & \textbf{Normal $(nx,ny,nz)$} & \textbf{Textura $(u,v)$} \\ \hline
    20 & $(0, 0, 0)$ & $(0, -1, 0)$ & $(0.25, 0.00)$ \\ \hline
    21 & $(1, 0, 0)$ & $(0, -1, 0)$ & $(0.50, 0.00)$ \\ \hline
    22 & $(1, 0, 1)$ & $(0, -1, 0)$ & $(0.50, 0.33)$ \\ \hline
    23 & $(0, 0, 1)$ & $(0, -1, 0)$ & $(0.25, 0.33)$ \\ \hline
    \end{tabular}
    \end{center}
\end{enumerate}

El código GDScript para definir las nuevas tablas de vértices, normales y coordenadas de textura es el siguiente:
\begin{lstlisting}[language=gdscript]
# Tabla de posiciones (24 vértices: 6 caras x 4 vértices)
var vertices = [
    # Cara Frontal (Z=1)
    Vector3(0,0,1), Vector3(1,0,1), Vector3(1,1,1), Vector3(0,1,1),
    # Cara Derecha (X=1)
    Vector3(1,0,1), Vector3(1,0,0), Vector3(1,1,0), Vector3(1,1,1),
    # Cara Trasera (Z=0)
    Vector3(1,0,0), Vector3(0,0,0), Vector3(0,1,0), Vector3(1,1,0),
    # Cara Izquierda (X=0)
    Vector3(0,0,0), Vector3(0,0,1), Vector3(0,1,1), Vector3(0,1,0),
    # Cara Superior (Y=1)
    Vector3(0,1,1), Vector3(1,1,1), Vector3(1,1,0), Vector3(0,1,0),
    # Cara Inferior (Y=0)
    Vector3(0,0,0), Vector3(1,0,0), Vector3(1,0,1), Vector3(0,0,1),
]

# Tabla de normales (una por vértice, constante por cara)
var normals = [
    # Frontal
    Vector3(0,0,1), Vector3(0,0,1), Vector3(0,0,1), Vector3(0,0,1),
    # Derecha
    Vector3(1,0,0), Vector3(1,0,0), Vector3(1,0,0), Vector3(1,0,0),
    # Trasera
    Vector3(0,0,-1), Vector3(0,0,-1), Vector3(0,0,-1), Vector3(0,0,-1),
    # Izquierda
    Vector3(-1,0,0), Vector3(-1,0,0), Vector3(-1,0,0), Vector3(-1,0,0),
    # Superior
    Vector3(0,1,0), Vector3(0,1,0), Vector3(0,1,0), Vector3(0,1,0),
    # Inferior
    Vector3(0,-1,0), Vector3(0,-1,0), Vector3(0,-1,0), Vector3(0,-1,0),
]

# Tabla de coordenadas de textura (UV)
var uvs = [
    # Frontal (u: 0.25-0.5, v: 0.33-0.66)
    Vector2(0.25,0.33), Vector2(0.50,0.33), Vector2(0.50,0.66), Vector2(0.25,0.66),
    # Derecha (u: 0.5-0.75, v: 0.33-0.66)
    Vector2(0.50,0.33), Vector2(0.75,0.33), Vector2(0.75,0.66), Vector2(0.50,0.66),
    # Trasera (u: 0.75-1.0, v: 0.33-0.66)
    Vector2(0.75,0.33), Vector2(1.00,0.33), Vector2(1.00,0.66), Vector2(0.75,0.66),
    # Izquierda (u: 0.0-0.25, v: 0.33-0.66)
    Vector2(0.00,0.33), Vector2(0.25,0.33), Vector2(0.25,0.66), Vector2(0.00,0.66),
    # Superior (u: 0.25-0.5, v: 0.66-1.0)
    Vector2(0.25,0.66), Vector2(0.50,0.66), Vector2(0.50,1.00), Vector2(0.25,1.00),
    # Inferior (u: 0.25-0.5, v: 0.00-0.33)
    Vector2(0.25,0.00), Vector2(0.50,0.00), Vector2(0.50,0.33), Vector2(0.25,0.33),
]

# Tabla de triángulos 
var triangles = [
    # Frontal
    0,1,2, 0,2,3,
    # Derecha
    4,5,6, 4,6,7,
    # Trasera
    8,9,10, 8,10,11,
    # Izquierda
    12,13,14, 12,14,15,
    # Superior
    16,17,18, 16,18,19,
    # Inferior
    20,21,22, 20,22,23,
]
\end{lstlisting}


\end{solucion}


\begin{ejercicio}
%\textbf{Problema 8.3:}
Considera un cubo de lado unidad y con centro en $(\frac{1}{2}, \frac{1}{2}, \frac{1}{2})$. Se quiere visualizar con una textura a partir de una única imagen (cuadrada) que se replicará en las 6 caras de dicho cubo. Asume que no se va a usar iluminación (no es necesario calcular la tabla de normales).

Escribe ahora la tabla de coordenadas de vértices y la tabla de coordenadas de textura necesarias para renderizar este objeto correctamente.
\end{ejercicio}

\begin{solucion}
Para resolver este problema, debemos entender primero cómo funciona el mapeado de texturas en un motor gráfico (como OpenGL o el usado en Godot).

\begin{enumerate}
    \item \textbf{Análisis de la Geometría:}
    El cubo tiene lado $L=1$ y su centro es $C=(0.5, 0.5, 0.5)$. Esto implica que las coordenadas espaciales de los vértices varían desde:
    \[ x_{min} = 0.5 - 0.5 = 0, \quad x_{max} = 0.5 + 0.5 = 1 \]
    Lo mismo aplica para $y$ y $z$. Por tanto, el cubo ocupa el volumen $[0,1]^3$.

    \item \textbf{El Problema de la Continuidad (Por qué necesitamos 24 vértices):}
    Un cubo geométrico tiene solo 8 esquinas (vértices físicos). Sin embargo, nos piden replicar la imagen completa en \textit{cada una} de las 6 caras.
    
    Imaginemos la esquina superior derecha de la cara frontal. Sus coordenadas espaciales son $(1,1,1)$.
    \begin{itemize}
        \item Para la \textbf{Cara Frontal}, esta esquina corresponde a la coordenada de textura $(u,v) = (1,1)$ (arriba-derecha de la imagen).
        \item Para la \textbf{Cara Derecha}, esa misma esquina espacial $(1,1,1)$ corresponde a $(u,v) = (0,1)$ (arriba-izquierda de la imagen).
        \item Para la \textbf{Cara Superior}, esa esquina corresponde a $(u,v) = (1,0)$ (abajo-derecha de la imagen, dependiendo de la orientación).
    \end{itemize}
    
    En informática gráfica, un \textbf{vértice} se define por la tupla única de sus atributos: $(Posicion, UV)$. Como una misma posición espacial requiere distintos UVs según la cara que estemos dibujando, debemos \textbf{duplicar} los vértices. No podemos usar solo 8 vértices compartidos (mesh indexada simple); necesitamos definir 4 vértices únicos por cada una de las 6 caras.
    
    \[ \text{Total de vértices} = 6 \text{ caras} \times 4 \text{ vértices/cara} = 24 \text{ vértices}. \]

    \item \textbf{Esquema Visual del Mapeado:}
    A continuación, representamos cómo se asignan las coordenadas $(u,v)$ a una cara genérica para que la imagen se vea derecha (no rotada ni espejada).
    
    \begin{center}
    \begin{tikzpicture}[scale=3]
        % Dibujo de una cara cuadrada representando la textura
        \draw[thick, fill=gray!10] (0,0) rectangle (1,1);
        
        % Ejes UV
        \draw[->, thick, blue] (0,0) -- (1.2,0) node[right] {$u$};
        \draw[->, thick, red] (0,0) -- (0,1.2) node[above] {$v$};
        
        % Puntos
        \filldraw (0,0) circle (0.5pt) node[below left] {$(0,0)$};
        \filldraw (1,0) circle (0.5pt) node[below right] {$(1,0)$};
        \filldraw (1,1) circle (0.5pt) node[above right] {$(1,1)$};
        \filldraw (0,1) circle (0.5pt) node[above left] {$(0,1)$};
        
        % Texto explicativo
        \node at (0.5, 0.5) {Imagen de Textura};
    \end{tikzpicture}
    \end{center}

    \item \textbf{Tablas de Definición del Modelo:}
    Definiremos los vértices cara por cara. Asumiremos el orden de vértices estándar para formar dos triángulos (por ejemplo: 0-1-2 y 0-2-3 para un quad) en sentido antihorario (CCW).
\end{enumerate}



\begin{table}[H]
\centering
\small
\begin{tabular}{|c|c|c|c|}
\hline
\textbf{Cara} & \textbf{Índice ($i$)} & \textbf{Posición $(x, y, z)$} & \textbf{Coord. Textura $(u, v)$} \\
\hline
\hline
\multirow{4}{*}{\textbf{Frontal} ($z=1$)} 
& 0 & $(0, 0, 1)$ & $(0, 0)$ \\ 
& 1 & $(1, 0, 1)$ & $(1, 0)$ \\ 
& 2 & $(1, 1, 1)$ & $(1, 1)$ \\ 
& 3 & $(0, 1, 1)$ & $(0, 1)$ \\ 
\hline
\multirow{4}{*}{\textbf{Trasera} ($z=0$)} 
& 4 & $(1, 0, 0)$ & $(0, 0)$ \\ 
& 5 & $(0, 0, 0)$ & $(1, 0)$ \\ 
& 6 & $(0, 1, 0)$ & $(1, 1)$ \\ 
& 7 & $(1, 1, 0)$ & $(0, 1)$ \\ 
\hline
\multirow{4}{*}{\textbf{Derecha} ($x=1$)} 
& 8 & $(1, 0, 1)$ & $(0, 0)$ \\ 
& 9 & $(1, 0, 0)$ & $(1, 0)$ \\ 
& 10 & $(1, 1, 0)$ & $(1, 1)$ \\ 
& 11 & $(1, 1, 1)$ & $(0, 1)$ \\ 
\hline
\multirow{4}{*}{\textbf{Izquierda} ($x=0$)} 
& 12 & $(0, 0, 0)$ & $(0, 0)$ \\ 
& 13 & $(0, 0, 1)$ & $(1, 0)$ \\ 
& 14 & $(0, 1, 1)$ & $(1, 1)$ \\ 
& 15 & $(0, 1, 0)$ & $(0, 1)$ \\ 
\hline
\multirow{4}{*}{\textbf{Superior} ($y=1$)} 
& 16 & $(0, 1, 1)$ & $(0, 0)$ \\ 
& 17 & $(1, 1, 1)$ & $(1, 0)$ \\ 
& 18 & $(1, 1, 0)$ & $(1, 1)$ \\ 
& 19 & $(0, 1, 0)$ & $(0, 1)$ \\ 
\hline
\multirow{4}{*}{\textbf{Inferior} ($y=0$)} 
& 20 & $(0, 0, 0)$ & $(0, 0)$ \\ 
& 21 & $(1, 0, 0)$ & $(1, 0)$ \\ 
& 22 & $(1, 0, 1)$ & $(1, 1)$ \\ 
& 23 & $(0, 0, 1)$ & $(0, 1)$ \\ 
\hline
\end{tabular}
\caption{Tabla Combinada de Vértices y Coordenadas de Textura}
\end{table}

\vspace{0.3cm}
\textit{Nota sobre la orientación:} En la cara trasera y las laterales, el orden de los vértices y la asignación de $(u,v)$ se ha elegido para mantener la coherencia visual (que la imagen no se vea ''espejada'') y el orden de los vértices (winding order) sea consistente para el ''culling'' de caras traseras.

El código GDScript para definir las tablas de vértices y coordenadas de textura es el siguiente:
\begin{lstlisting}[language=gdscript]
# Tabla de posiciones (24 vértices: 6 caras x 4 vértices)
var vertices = [
    # Frontal (z=1)
    Vector3(0,0,1), Vector3(1,0,1), Vector3(1,1,1), Vector3(0,1,1),
    # Trasera (z=0)
    Vector3(1,0,0), Vector3(0,0,0), Vector3(0,1,0), Vector3(1,1,0),
    # Derecha (x=1)
    Vector3(1,0,1), Vector3(1,0,0), Vector3(1,1,0), Vector3(1,1,1),
    # Izquierda (x=0)
    Vector3(0,0,0), Vector3(0,0,1), Vector3(0,1,1), Vector3(0,1,0),
    # Superior (y=1)
    Vector3(0,1,1), Vector3(1,1,1), Vector3(1,1,0), Vector3(0,1,0),
    # Inferior (y=0)
    Vector3(0,0,0), Vector3(1,0,0), Vector3(1,0,1), Vector3(0,0,1),
]

# Tabla de coordenadas de textura (UV)
var uvs = [
    # Frontal
    Vector2(0,0), Vector2(1,0), Vector2(1,1), Vector2(0,1),
    # Trasera
    Vector2(0,0), Vector2(1,0), Vector2(1,1), Vector2(0,1),
    # Derecha
    Vector2(0,0), Vector2(1,0), Vector2(1,1), Vector2(0,1),
    # Izquierda
    Vector2(0,0), Vector2(1,0), Vector2(1,1), Vector2(0,1),
    # Superior
    Vector2(0,0), Vector2(1,0), Vector2(1,1), Vector2(0,1),
    # Inferior
    Vector2(0,0), Vector2(1,0), Vector2(1,1), Vector2(0,1),
]

# Tabla de triángulos 
var triangles = [
    # Frontal
    0,1,2, 0,2,3,
    # Trasera
    4,5,6, 4,6,7,
    # Derecha
    8,9,10, 8,10,11,
    # Izquierda
    12,13,14, 12,14,15,
    # Superior
    16,17,18, 16,18,19,
    # Inferior
    20,21,22, 20,22,23,
]
\end{lstlisting}

\end{solucion}

\section{Sesión 9}

\begin{ejercicio}
En una aplicación Godot cualquiera, añade código al nodo raíz de forma que cada vez que se pulse y luego se levante una tecla (por ejemplo la tecla P), se imprima en pantalla un mensaje con el tiempo total en segundos que dicha tecla ha estado pulsada, en los casos en los que ha permanecido pulsada al menos el tiempo de un frame.
\end{ejercicio}

\begin{solucion}

Para resolver este problema, debemos comprender cómo funciona el ciclo de vida de un videojuego o aplicación gráfica interactiva en tiempo real. No basta con saber que una tecla ha sido pulsada; necesitamos cuantificar la duración temporal de ese estado.

En Godot, la función \texttt{\_process(delta)} se ejecuta en cada fotograma (frame). El parámetro \texttt{delta} representa el tiempo transcurrido (en segundos) desde el fotograma anterior. Por lo tanto, la estrategia consiste en acumular este valor \texttt{delta} mientras la tecla esté presionada y, en el momento exacto en que se libera, mostrar el total acumulado.

A continuación, se presenta el diagrama de flujo lógico que seguiremos para implementar el algoritmo:

\begin{center}
\begin{tikzpicture}[node distance=2cm, every node/.style={font=\small}]

% --- ESTILOS CORREGIDOS (Se añade align=center para permitir saltos de línea \\) ---
\tikzstyle{startstop} = [rectangle, rounded corners, minimum width=2.5cm, minimum height=1cm, align=center, draw=black, fill=red!30]
\tikzstyle{io} = [trapezium, trapezium left angle=70, trapezium right angle=110, minimum width=2.5cm, minimum height=1cm, align=center, draw=black, fill=blue!30]
\tikzstyle{process} = [rectangle, minimum width=2.5cm, minimum height=1cm, align=center, draw=black, fill=orange!30]
\tikzstyle{decision} = [diamond, aspect=2, minimum width=2.5cm, minimum height=1cm, align=center, draw=black, fill=green!30, inner sep=0pt]
\tikzstyle{arrow} = [thick,->,>=stealth]

% --- NODOS ---
\node (start) [startstop] {\_process(delta)};

\node (dec1) [decision, below=of start] {¿Tecla 'P'\\pulsada?};

% Movemos proc1 a la derecha
\node (proc1) [process, right=3.5cm of dec1] {tiempo += delta};

\node (dec2) [decision, below=of dec1] {¿Estaba pulsada\\antes?};

\node (io1) [io, below=of dec2] {Imprimir tiempo};

\node (reset) [process, below=of io1] {Resetear tiempo};

% --- FLECHAS Y CONEXIONES ---

% 1. Flujo inicial
\draw [arrow] (start) -- (dec1);

% 2. Decisión 1: SI (Derecha y vuelta arriba)
\draw [arrow] (dec1) -- node[anchor=south] {Sí} (proc1); 
\draw [arrow] (proc1) |- (start); 

% 3. Decisión 1: NO (Abajo)
\draw [arrow] (dec1) -- node[anchor=east] {No} (dec2);

% 4. Decisión 2: NO (Izquierda y vuelta arriba - Bucle ''Idle'')
\draw [arrow] (dec2.west) -- ++(-2.5,0) |- (start.west);
\node[anchor=south east] at ($(dec2.west) + (-0.2,0)$) {No};

% 5. Decisión 2: SI (Abajo - Soltada)
\draw [arrow] (dec2) -- node[anchor=east] {Sí (Soltada)} (io1);
\draw [arrow] (io1) -- (reset);

% 6. Retorno final (Desde Reset hasta Start por la izquierda exterior)
\draw [arrow] (reset.west) -- ++(-3.5,0) |- (start.west);

\end{tikzpicture}
\end{center}

\vspace{0.5cm}

\textbf{Implementación paso a paso:}

\begin{enumerate}
\item \textbf{Definición de variables de estado:}
Necesitamos una variable para acumular el tiempo (\texttt{tiempo\_pulsado}) y una variable booleana (\texttt{tecla\_activa}) para saber si estamos en medio de una acción de pulsación. Esto es necesario para detectar el evento ''just released'' (acaba de ser soltada) manualmente o mediante la lógica de estados.

\item \textbf{Uso del bucle de procesamiento:}
Utilizaremos la función virtual \texttt{\_process(delta)}, que Godot invoca continuamente.

\item \textbf{Lógica de entrada (Input):}
Usaremos la clase \texttt{Input} para sondear (polling) el estado físico de la tecla 'P' (código \texttt{KEY\_P}).

\item \textbf{Acumulación y Reporte:}
\begin{itemize}
    \item Si la tecla está pulsada: Sumamos \texttt{delta} a nuestra variable acumuladora.
    \item Si la tecla NO está pulsada pero \texttt{tecla\_activa} es verdadera: Significa que el usuario acaba de soltar la tecla. En ese momento imprimimos el valor y reiniciamos las variables.
\end{itemize}

\end{enumerate}

\vspace{0.5cm}

\textbf{Código GDScript Solución:}

\begin{lstlisting}
extends Node

# Variable para almacenar el tiempo acumulado en segundos

var tiempo_acumulado: float = 0.0

# Bandera para controlar el estado de la tecla (si se está manteniendo pulsada)

var tecla_esta_pulsada: bool = false

func _process(delta: float) -> void:
# Verificamos si la tecla P está siendo presionada en este frame
if Input.is_key_pressed(KEY_P):
# Marcamos que la tecla está activa
tecla_esta_pulsada = true

    # Acumulamos el tiempo transcurrido desde el último frame
    tiempo_acumulado += delta
    
else:
    # Si la tecla NO está pulsada, verificamos si lo estaba en el frame anterior
    # Esto indica el evento ''Just Released'' (Acaba de soltarse)
    if tecla_esta_pulsada:
        
        # Verificamos la condición del enunciado: 
        # ''permanecido pulsada al menos el tiempo de un frame''
        # Si tiempo_acumulado > 0, significa que al menos un frame sumó delta.
        if tiempo_acumulado > 0.0:
            print(''La tecla P se mantuvo pulsada durante: '', 
                  tiempo_acumulado, '' segundos.'')
        
        # Reiniciamos el estado para la próxima pulsación
        tiempo_acumulado = 0.0
        tecla_esta_pulsada = false

\end{lstlisting}

\end{solucion}

\begin{ejercicio}
Una posibilidad para hacer selección en mallas de triángulos es usar cálculo
de intersecciones entre un rayo (una semirrecta que pasa por el centro de un
píxel) y cada uno de los triángulos de la malla. 

Diseña un algoritmo en pseudo-código para el cálculo de intersecciones entre un rayo y un triángulo:

\begin{itemize}
    \item El rayo tiene como origen o extremo el punto cuyas coordenadas del mundo es la tupla $\mathbf{o}$, y como vector de dirección la tupla $\mathbf{d}$ (la suponemos normalizada).
    \item Las coordenadas del mundo de los vértices del triángulo son $\mathbf{v}_0$, $\mathbf{v}_1$ y $\mathbf{v}_2$.
    \item El algoritmo debe indicar si hay intersección o no, y, en caso de que la haya, calcular las coordenadas del mundo del punto de intersección.
\end{itemize}

%Diseña un algoritmo en pseudo-código para el cálculo de intersecciones entre un rayo y un triángulo. 

Ten en cuenta que habrá intersección si y solo si se cumplen cada una de estas
dos condiciones:

\begin{itemize}
    \item El rayo intersecta con el plano del triángulo si y solo si existe $t > 0$ tal que el punto $p_t = o + t d$ está en el plano. Esto equivale a que el vector $p_t - v_0$ es perpendicular a la normal del plano $n$ (es decir, su producto escalar es nulo).
    \item El punto $p_t$ está dentro del triángulo si existen dos valores reales no negativos $a$ y $b$ (con $0 \leq a + b \leq 1$) tales que el vector $p_t - v_0 = a (v_1 - v_0) + b (v_2 - v_0)$. A los tres valores $a$, $b$ y $c \equiv 1 - a - b$ se les llama coordenadas baricéntricas de $p_t$ en el triángulo.
\end{itemize}

\begin{center}
\begin{tikzpicture}[scale=1.5, >=stealth]
% Coordenadas
\coordinate (O) at (0, 1, 3); % Origen rayo
\coordinate (V0) at (0, 0, 0);
\coordinate (V1) at (3, 0.5, 0.5);
\coordinate (V2) at (1, 2.5, 0);
\coordinate (Pt) at (1.2, 0.9, 0.16); % Punto intersección aproximado


% Triángulo
\filldraw[fill=cyan!10, draw=blue, thick] (V0) -- (V1) -- (V2) -- cycle;

% Vectores aristas
\draw[->, blue!80, thick] (V0) -- (V1) node[midway, below] {$v_1 - v_0$};
\draw[->, blue!80, thick] (V0) -- (V2) node[midway, left] {$v_2 - v_0$};

% Rayo
\draw[->, red, very thick] (O) -- (Pt) node[midway, above right] {$d$};
\draw[red, dashed] (Pt) -- (1.5, 0.875, -0.55);

% Puntos y etiquetas
\node[below left] at (V0) {$v_0$};
\node[right] at (V1) {$v_1$};
\node[above] at (V2) {$v_2$};
\node[left] at (O) {$o$};
\fill[red] (Pt) circle (1.5pt) node[below right] {$p_t$};

% Vector w
\draw[->, purple, dashed, thick] (V0) -- (Pt) node[midway, right] {$p_t - v_0$};



\end{tikzpicture}
\end{center}
\end{ejercicio}

\begin{solucion}
Para resolver el problema siguiendo estrictamente las condiciones dadas, el algoritmo se estructura en dos fases secuenciales: encontrar el punto en el plano (Condición 1) y validar si dicho punto está contenido en la región triangular (Condición 2).

\textbf{Procedimiento detallado:}

\begin{enumerate}
\item \textbf{Cálculo de la Normal del Plano:}
Primero, definimos los vectores directores del plano del triángulo basándonos en sus aristas:
$$
e_1 = v_1 - v_0
$$
$$
e_2 = v_2 - v_0
$$

La normal $n$ se obtiene mediante el producto vectorial:
$$
n = e_1 \times e_2
$$


\item \textbf{Condición 1: Intersección con el Plano:}
Buscamos un $t$ tal que el vector desde $v_0$ hasta el punto de impacto $p_t$ sea ortogonal a la normal.
La ecuación del plano es $(p - v_0) \cdot n = 0$.
Sustituyendo la ecuación del rayo $p = o + t \cdot d$:
$$((o + t \cdot d) - v_0) \cdot n = 0$$
$$(o - v_0) \cdot n + t(d \cdot n) = 0$$
Despejando $t$:
$$t = \frac{(v_0 - o) \cdot n}{d \cdot n}$$
Si $d \cdot n \approx 0$, el rayo es paralelo al plano (no hay intersección). Si $t \leq 0$, el triángulo está detrás del origen. 

\item \textbf{Condición 2: Inclusión en el Triángulo (Coordenadas Baricéntricas):}
Una vez tenemos $p_t = o + t \cdot d$, definimos el vector $w = p_t - v_0$. Según el enunciado, debemos encontrar $a$ y $b$ tales que:
$$w = a \cdot e_1 + b \cdot e_2$$
Esto es un sistema de ecuaciones lineales. Para resolverlo eficientemente usando productos escalares, multiplicamos la ecuación por $e_1$ y por $e_2$:
\begin{enumerate}
    \item $(w \cdot e_1) = a(e_1 \cdot e_1) + b(e_2 \cdot e_1)$
    \item $(w \cdot e_2) = a(e_1 \cdot e_2) + b(e_2 \cdot e_2)$
\end{enumerate}
Aplicando la regla de Cramer para despejar $a$ y $b$:
$$a = \frac{(w \cdot e_1)(e_2 \cdot e_2) - (w \cdot e_2)(e_1 \cdot e_2)}{(e_1 \cdot e_1)(e_2 \cdot e_2) - (e_1 \cdot e_2)^2}$$
$$b = \frac{(e_1 \cdot e_1)(w \cdot e_2) - (e_1 \cdot e_2)(w \cdot e_1)}{(e_1 \cdot e_1)(e_2 \cdot e_2) - (e_1 \cdot e_2)^2}$$
Finalmente, verificamos si $a \geq 0$, $b \geq 0$ y $a + b \leq 1$.



\end{enumerate}

\vspace{0.5cm}

\textbf{Algoritmo en Pseudo-código:}

\begin{lstlisting}[language=C++,  frame=single, numbers=left, breaklines=true, keywordstyle=\color{blue}, commentstyle=\color{green!60!black}, stringstyle=\color{purple}]
Funcion IntersectarRayoTriangulo(o, d, v0, v1, v2):
// --- Pre-computo de vectores del triangulo ---
Vector3 e1 = v1 - v0
Vector3 e2 = v2 - v0
Vector3 n  = ProductoCruz(e1, e2) // Normal del plano


// --- Condicion 1: Interseccion Rayo-Plano ---

// Calculamos el denominador (d . n)
float det = ProductoPunto(d, n)

// Si es cercano a 0, el rayo es paralelo al triangulo
Si valor_absoluto(det) < EPSILON:
    Retornar {Falso, Nulo}

// Calculamos t usando la formula derivada: t = ((v0 - o) . n) / det
Vector3 origen_a_v0 = v0 - o
float t = ProductoPunto(origen_a_v0, n) / det

// Verificamos que la interseccion esta delante de la camara (t > 0)
Si t < EPSILON:
    Retornar {Falso, Nulo}

// Calculamos el punto de interseccion en el plano
Vector3 pt = o + (d * t)

// --- Condicion 2: Punto dentro del triangulo ---
// Debemos resolver: pt - v0 = a*e1 + b*e2

Vector3 w = pt - v0 

// Calculo de productos punto para el sistema de Cramer
float uu = ProductoPunto(e1, e1)
float uv = ProductoPunto(e1, e2)
float vv = ProductoPunto(e2, e2)
float wu = ProductoPunto(w, e1)
float wv = ProductoPunto(w, e2)

// Denominador del sistema (determinante)
float denominador = (uu * vv) - (uv * uv)

// Si denominador es 0, el triangulo es degenerado (linea o punto)
Si valor_absoluto(denominador) < EPSILON:
    Retornar {Falso, Nulo}

// Calculo de coordenadas baricentricas a y b
float a = ((wu * vv) - (wv * uv)) / denominador
float b = ((uu * wv) - (wu * uv)) / denominador

// Verificacion final de limites baricentricos
// 0 <= a, 0 <= b, a + b <= 1
Si (a >= 0.0) Y (b >= 0.0) Y (a + b <= 1.0):
    Retornar {Verdadero, pt}
Sino:
    Retornar {Falso, Nulo}



\end{lstlisting}
\end{solucion}

\begin{solucion} Otra resolución alternativa y más detallada es la que se proporciona a continuación.
Para resolver este problema, debemos traducir la geometría 3D a una serie de pasos lógicos. No basta con aplicar fórmulas; hay que entender qué significan. El proceso se divide en tres fases: definir la pared (plano), buscar el choque y verificar si el choque está dentro del triángulo.

\subsubsection*{1. Definición del Plano (La Pared)}
Un triángulo es plano. Para saber si un rayo choca con él, primero necesitamos saber la orientación de la pared invisible donde está pegado el triángulo.
\begin{itemize}
    \item Calculamos dos vectores que bordean el triángulo desde $\mathbf{v}_0$:
    $$ \mathbf{e}_1 = \mathbf{v}_1 - \mathbf{v}_0, \quad \mathbf{e}_2 = \mathbf{v}_2 - \mathbf{v}_0 $$
    \item La orientación (el vector normal $\mathbf{n}$) es perpendicular a ambos bordes:
    $$ \mathbf{n} = \mathbf{e}_1 \times \mathbf{e}_2 $$
\end{itemize}

\subsubsection*{2. El Choque (Cálculo de $t$)}
El rayo es una línea que empieza en $\mathbf{o}$ y avanza en dirección $\mathbf{d}$. La fórmula del impacto en el plano es:
$$ t = \frac{(\mathbf{v}_0 - \mathbf{o}) \cdot \mathbf{n}}{\mathbf{d} \cdot \mathbf{n}} $$

\textbf{¿Por qué $t < 0$ significa ``detrás''?}
Imagina que el rayo son tus pasos.
\begin{itemize}
    \item $\mathbf{t=0}$ es donde estás parado (el origen).
    \item $\mathbf{t>0}$ son pasos hacia adelante (lo que ves).
    \item $\mathbf{t<0}$ son pasos hacia atrás (a tu espalda).
\end{itemize}
La fórmula matemática asume una recta infinita (hacia adelante y atrás). Si el cálculo da $t = -5$, significa que el plano está 5 pasos a tu espalda. Como una cámara solo "ve" hacia adelante, descartamos cualquier $t < 0$.

\subsubsection*{3. ¿Dentro o Fuera? (Coordenadas Baricéntricas)}
Si $t > 0$, el rayo golpea la pared en el punto $\mathbf{p}$. Ahora usamos coordenadas baricéntricas ($a, b$) para ver si ese punto cae dentro del dibujo del triángulo. Es como preguntar: \textit{"¿Puedo llegar al punto $\mathbf{p}$ dando pasos solo a lo largo de los bordes $\mathbf{e}_1$ y $\mathbf{e}_2$ sin salirme?"}.

Se resuelve el sistema $\mathbf{p} - \mathbf{v}_0 = a \mathbf{e}_1 + b \mathbf{e}_2$. Si $a \geq 0$, $b \geq 0$ y $a+b \leq 1$, estamos dentro.

\vspace{0.5cm}

\begin{algorithm}[H]
\caption{Intersección Rayo-Triángulo}
\begin{algorithmic}[1]
\State \textbf{Entrada:} Rayo ($\mathbf{o}, \mathbf{d}$), Triángulo ($\mathbf{v}_0, \mathbf{v}_1, \mathbf{v}_2$)
\State \textbf{Salida:} Bool (¿Impacto?), Punto ($\mathbf{p}$)

\Statex \Comment{--- Fase 1: Preparar vectores ---}
\State $\mathbf{e}_1 \gets \mathbf{v}_1 - \mathbf{v}_0$
\State $\mathbf{e}_2 \gets \mathbf{v}_2 - \mathbf{v}_0$
\State $\mathbf{n} \gets \mathbf{e}_1 \times \mathbf{e}_2$ \Comment{Producto Vectorial (Normal)}

\Statex \Comment{--- Fase 2: Intersección con el plano ---}
\State $det \gets \mathbf{d} \cdot \mathbf{n}$
\If{$|det| < \epsilon$} \Comment{¿Es el rayo paralelo al plano?}
    \State \Return \textbf{Falso}, Nulo
\EndIf

\State $\mathbf{vec\_origen} \gets \mathbf{v}_0 - \mathbf{o}$
\State $t \gets (\mathbf{vec\_origen} \cdot \mathbf{n}) / det$

\If{$t < 0$} \Comment{Si t es negativo, el triángulo está detrás}
    \State \Return \textbf{Falso}, Nulo
\EndIf

\Statex \Comment{--- Fase 3: Test de inclusión (Baricéntricas) ---}
\State $\mathbf{p} \gets \mathbf{o} + (t \cdot \mathbf{d})$ \Comment{Punto de impacto en el plano}
\State $\mathbf{w} \gets \mathbf{p} - \mathbf{v}_0$

\Statex \Comment{Resolvemos sistema lineal usando prod. escalares (Cramer)}
\State $uu \gets \mathbf{e}_1 \cdot \mathbf{e}_1; \quad uv \gets \mathbf{e}_1 \cdot \mathbf{e}_2; \quad vv \gets \mathbf{e}_2 \cdot \mathbf{e}_2$
\State $wu \gets \mathbf{w} \cdot \mathbf{e}_1; \quad wv \gets \mathbf{w} \cdot \mathbf{e}_2$
\State $D \gets (uu \cdot vv) - (uv \cdot uv)$

\State $a \gets ((wu \cdot vv) - (wv \cdot uv)) / D$
\State $b \gets ((uu \cdot wv) - (uv \cdot wu)) / D$

\If{$a \geq 0 \land b \geq 0 \land (a + b \leq 1)$}
    \State \Return \textbf{Verdadero}, $\mathbf{p}$ \Comment{¡Impacto confirmado!}
\Else
    \State \Return \textbf{Falso}, Nulo \Comment{Fuera del triángulo}
\EndIf
\end{algorithmic}
\end{algorithm}
\hspace{1cm}

\end{solucion}

\begin{ejercicio}
Para implementar la selección usando intersecciones es necesario calcular el rayo que tiene como origen el observador y pasa por el centro del pixel donde se ha hecho click.

Escribe el pseudo-código del algoritmo que calcula el rayo a partir de las coordenadas del pixel donde se ha hecho click:
\begin{itemize}
\item Tenemos una vista perspectiva, y conocemos los 6 valores  usados para construir la matriz de proyección (left, right, top, bottom, near, far).
\item También conocemos el marco de coordenadas de vista, es decir, las tuplas  y  con los versores y la tupla  con el punto origen (todos en coordenadas del mundo).
\item El viewport tiene  columnas y  filas de pixels.
\item Se ha hecho click en el pixel de coordenadas enteras  e .
\end{itemize}
El algoritmo debe producir como salida las tuplas  y  (normalizado) que definen el rayo.

\begin{center}
\begin{tikzpicture}[scale=1.2, >=stealth]
% Sistema de coordenadas de la cámara (Eye Coordinates)
\coordinate (Oec) at (0, 0, 4);


% Plano de imagen (Near Plane)
\coordinate (TL) at (-1.5, 1.5, 2); % Top-Left
\coordinate (TR) at (1.5, 1.5, 2);  % Top-Right
\coordinate (BL) at (-1.5, -1.5, 2); % Bottom-Left
\coordinate (BR) at (1.5, -1.5, 2);  % Bottom-Right

% Pixel clickado (representado en el plano near)
\coordinate (Pixel) at (0.5, 0.5, 2);

% Ejes de la cámara
\draw[->, thick, blue] (Oec) -- (1, 0, 4) node[right] {$x_{ec}$};
\draw[->, thick, green!60!black] (Oec) -- (0, 1, 4) node[above] {$y_{ec}$};
\draw[->, thick, orange] (Oec) -- (0, 0, 3) node[below left] {$-z_{ec}$};

% Dibujar pirámide de visión (Frustum truncado visualmente)
\draw[dashed, gray] (Oec) -- (TL);
\draw[dashed, gray] (Oec) -- (TR);
\draw[dashed, gray] (Oec) -- (BL);
\draw[dashed, gray] (Oec) -- (BR);

% Plano Near
\draw[thick, black] (TL) -- (TR) -- (BR) -- (BL) -- cycle;
\node at (-1.6, 1.6, 2) {Plano Near ($n$)};

% Rayo
\draw[->, red, very thick] (Oec) -- (Pixel) -- (0.75, 0.75, 1) node[right] {$d$};
\fill[red] (Pixel) circle (1.5pt) node[below right] {Centro Pixel ($u, v$)};
\fill[black] (Oec) circle (1.5pt) node[above left] {$o_{ec}$ (Origen)};



\end{tikzpicture}
\end{center}
\end{ejercicio}

\begin{solucion}

El objetivo de este ejercicio es realizar el proceso inverso a la rasterización: en lugar de proyectar un punto 3D a un pixel 2D, queremos proyectar un pixel 2D hacia el espacio 3D (''Un-project'').

Para ello, debemos transformar las coordenadas del pixel  desde el espacio de pantalla al espacio de la cámara (View Space), y finalmente rotar ese vector al espacio del mundo (World Space) usando la base de la cámara dada.

\textbf{Procedimiento paso a paso:}

\begin{enumerate}
\item \textbf{Mapeo de Pixel a Plano de Imagen (View Plane):}
El plano de proyección se encuentra a una distancia  (near) de la cámara. Los límites de este plano son  en horizontal y  en vertical.


Los pixels se indexan generalmente desde la esquina superior izquierda $(0,0)$ hasta $(w, filas)$. Sin embargo, el sistema de coordenadas de la cámara suele tener el eje $Y$ apuntando hacia arriba. Debemos tener cuidado con esta inversión.

Calculamos las coordenadas físicas $(u, v)$ en el plano near correspondientes al centro del pixel:
\begin{itemize}
    \item Sumamos $0.5$ a $x_p$ y $y_p$ para apuntar al \textit{centro} del pixel, no a su esquina.
    \item Interpolamos linealmente:
    $$u = l + (r - l) \cdot \frac{x_p + 0.5}{w}$$
    $$v = t - (t - b) \cdot \frac{y_p + 0.5}{filas}$$
    \textit{Nota: Asumimos que $y_p=0$ es la parte superior (top) y $y_p=filas$ es la inferior (bottom), por eso restamos en $v$.}
\end{itemize}

\item \textbf{Construcción del Vector en Espacio de Cámara:}
En el sistema de referencia local de la cámara:
\begin{itemize}
    \item El origen del rayo es $(0,0,0)$.
    \item El rayo atraviesa el plano near en $(u, v, -n)$. (Recordemos que en OpenGL/Godot la cámara mira hacia $-Z$).
    \item El vector dirección local es $\vec{d}_{local} = (u, v, -n)$.
\end{itemize}

\item \textbf{Transformación al Espacio del Mundo:}
Ahora usamos la base de la cámara dada ($x_{ec}, y_{ec}, z_{ec}$) para orientar este vector en el mundo. 
$$\vec{d}_{mundo} = u \cdot \vec{x}_{ec} + v \cdot \vec{y}_{ec} + (-n) \cdot \vec{z}_{ec}$$

El origen del rayo $o$ es simplemente la posición de la cámara $o_{ec}$.

\item \textbf{Normalización:}
Finalmente, normalizamos el vector dirección resultante.



\end{enumerate}

\vspace{0.5cm}

\textbf{Algoritmo en Pseudo-código:}

\begin{lstlisting}[language=C++,  frame=single, numbers=left, breaklines=true, keywordstyle=\color{blue}, commentstyle=\color{green!60!black}, stringstyle=\color{purple}]
Funcion CalcularRayoDesdePixel(xp, yp, w, filas, l, r, b, t, n, o_ec, x_ec, y_ec, z_ec):

// 1. Calcular coordenadas normalizadas del centro del pixel (0.0 a 1.0)
// Sumamos 0.5 para tomar el centro exacto del pixel
float ratio_x = (xp + 0.5) / w
float ratio_y = (yp + 0.5) / filas

// 2. Mapear al tamaño fisico del plano near (View Plane)
// Coordenada u (horizontal): interpolar entre left (l) y right (r)
float u = l + ((r - l) * ratio_x)

// Coordenada v (vertical): interpolar entre top (t) y bottom (b)
// IMPORTANTE: Asumimos que yp=0 es arriba (top) y yp=filas es abajo (bottom)
// Por tanto, a mayor yp, nos acercamos mas a 'b' y nos alejamos de 't'
float v = t - ((t - b) * ratio_y) 

// 3. Construir el vector de direccion en coordenadas del mundo
// El vector en espacio camara es (u, v, -n)
// Lo transformamos multiplicando por los versores de la base de la camara
// d = u*Right + v*Up + (-n)*Back

Vector3 direccion_no_norm = (x_ec * u) + (y_ec * v) - (z_ec * n)

// 4. Normalizar la direccion
Vector3 d = Normalizar(direccion_no_norm)

// 5. El origen del rayo es la posicion de la camara (proyeccion perspectiva)
Vector3 o = o_ec

Retornar {o, d}



\end{lstlisting}
\end{solucion}


\section{Sesión 10}

\begin{ejercicio}

Supongamos que un rayo (una semirrecta en 3D) tiene como origen o extremo el punto cuyas coordenadas del mundo es la tupla $\mathbf{o}$, y como vector de dirección la tupla $\mathbf{d}$ (la suponemos normalizada). Además, sabemos que un disco de radio $r$ tiene como centro el punto de coordenadas de mundo $\mathbf{c}$ y está en el plano perpendicular al vector $\mathbf{n}$.

Con estos datos de entrada, diseña un algoritmo para calcular si hay intersección entre el rayo y el disco.

Ten en cuenta que habrá intersección si y solo si se cumplen cada
una de estas dos condiciones:
\begin{enumerate}
    \item El rayo interseca con el plano que contiene al disco, es
    decir, existe $t > 0$ tal que el punto $p_t \equiv o + t d$ está en dicho
    plano. Equivale a decir que el vector $p_t - c$ es perpendicular a
    la normal al plano $n$.
    \item El punto $p_t$ citado arriba está dentro del disco, es decir, su
    distancia a $c$ es inferior al radio.
\end{enumerate}
% \vspace{0.5cm}
% \centering
% \begin{tikzpicture}[scale=1.2]
% % Definición de coordenadas para simular 3D
% \coordinate (O) at (-2, 2); % Origen del rayo
% \coordinate (C) at (2, 0);  % Centro del disco
% \coordinate (P) at (1, 0.5); % Punto de intersección aproximado


% % Dibujo del plano (perspectiva)
% \draw[fill=gray!10, dashed] (-1, -1.5) -- (4, -1.5) -- (5, 2.5) -- (0, 2.5) -- cycle;
% \node at (4.5, 2.2) {Plano $\pi$};

% % Dibujo del disco
% \draw[fill=blue!20, opacity=0.8] (C) ellipse (1.5 and 0.8);
% \node[below right] at (C) {$c$ (centro)};
% \fill (C) circle (1.5pt);

% % Vector Normal
% \draw[->, thick, red] (C) -- ++(0, 1.5) node[above] {$n$};

% % Rayo (parte visible antes del plano)
% \draw[->, thick, black] (O) -- (P);
% \fill (O) circle (1.5pt) node[left] {$o$ (origen)};
% \node[above] at (-0.5, 1.25) {$d$};

% % Intersección
% \fill[red] (P) circle (1.5pt) node[right, yshift=-0.2cm] {$p_t$ (intersección)};

% % Rayo (proyección imaginaria)
% \draw[dashed] (P) -- ++(1, -0.33);

% % Radio
% \draw[<->, thin] (C) -- ++(1.5, 0) node[midway, below] {$r$};

% % Vector p - c
% \draw[->, blue, thick] (C) -- (P) node[midway, above left, scale=0.7] {$p_t - c$};



% \end{tikzpicture}
\end{ejercicio}

\begin{solucion}
Para resolver este problema geométrico fundamental en el trazado de rayos (\textit{ray-tracing}), se debe descomponer la situación en dos etapas lógicas secuenciales. Primero, se determina el punto donde el rayo infinito cruza el plano matemático que contiene al disco. Segundo, se verifica si dicho punto de cruce se encuentra dentro de los límites finitos del disco (es decir, dentro de su radio).

\begin{enumerate}
\item \textbf{Definición Algebraica del Rayo y el Plano:}


Un rayo se define paramétricamente como una línea que parte de un origen $o$ y avanza en la dirección $d$. Cualquier punto $p(t)$ sobre el rayo se puede expresar como:
$$p(t) = o + t \cdot d$$
Donde $t$ es un escalar real ($t \ge 0$) que representa la distancia desde el origen a lo largo del vector dirección.

Por otro lado, un plano en el espacio 3D queda definido por un punto conocido (en este caso, el centro del disco $c$) y un vector normal $n$ perpendicular a la superficie. La condición para que un punto genérico $p$ pertenezca al plano es que el vector formado entre el centro y ese punto sea perpendicular a la normal. Matemáticamente, esto implica que su producto escalar (\textit{dot product}) es cero:
$$(p - c) \cdot n = 0$$

\item \textbf{Cálculo del parámetro de intersección $t$:}

Para encontrar la intersección, se sustituye la ecuación del rayo en la ecuación del plano:
$$((o + t \cdot d) - c) \cdot n = 0$$

Aplicando la propiedad distributiva del producto escalar:
$$(o - c) \cdot n + (t \cdot d) \cdot n = 0$$
$$(o \cdot n) - (c \cdot n) + t(d \cdot n) = 0$$

Despejando $t$:
$$t(d \cdot n) = (c \cdot n) - (o \cdot n)$$
$$t(d \cdot n) = (c - o) \cdot n$$
$$t = \frac{(c - o) \cdot n}{d \cdot n}$$

Aquí surgen consideraciones críticas de implementación:
\begin{itemize}
    \item Si el denominador $d \cdot n$ es igual a 0, significa que el rayo es perpendicular a la normal del plano (es decir, el rayo es paralelo al plano), por lo que no hay intersección (o el rayo está contenido en el plano).
    \item Si $t < 0$, la intersección ocurre ''detrás'' del origen del rayo, por lo que no es visible y debe descartarse.
\end{itemize}

\item \textbf{Cálculo del punto de intersección $p_t$:}

Una vez obtenido un $t$ válido ($t > 0$), se calcula la coordenada exacta del punto en el espacio:
$$p_t = o + t \cdot d$$

\item \textbf{Verificación de pertenencia al disco:}

El hecho de que $p_t$ esté en el plano no garantiza que golpee el disco. Para que haya colisión, la distancia entre el punto de intersección $p_t$ y el centro del disco $c$ debe ser menor o igual al radio $r$.
$$\| p_t - c \| \le r$$

Computacionalmente, calcular la raíz cuadrada para el módulo de un vector es costoso. Es preferible comparar los cuadrados de las distancias:
$$v = p_t - c$$
$$v \cdot v \le r^2$$
$$(v_x^2 + v_y^2 + v_z^2) \le r^2$$



\end{enumerate}

A continuación, se presenta el algoritmo formal en pseudocódigo:

\begin{lstlisting}[language=C++, frame=single,  keywordstyle=\color{blue}, commentstyle=\color{green!60!black}, captionpos=b, caption={Algoritmo de Intersección Rayo-Disco}]
// Estructuras de datos:
// Vec3: tupla (x, y, z) con operaciones de suma, resta y producto punto
// Rayo: origen (Vec3), direccion (Vec3)
// Disco: centro (Vec3), normal (Vec3), radio (float)

bool IntersectaDisco(Rayo ray, Disco disco, float &t_salida) {
    // 1. Calcular el denominador (producto punto entre normal y direccion)
    float denom = dot(disco.normal, ray.direccion);


    // Si denom es cercano a 0, el rayo es paralelo al plano
    if (abs(denom) < 1e-6) {
        return false; 
    }

    // 2. Calcular el vector desde el origen del rayo al centro del disco
    Vec3 vector_origen_centro = disco.centro - ray.origen;

    // 3. Calcular t
    float t = dot(vector_origen_centro, disco.normal) / denom;

    // Verificar si la interseccion esta detras de la camara
    if (t < 0) {
        return false;
    }

    // 4. Calcular el punto exacto de interseccion en el plano
    Vec3 p = ray.origen + (ray.direccion * t);

    // 5. Verificar si el punto esta dentro del radio del disco
    Vec3 v = p - disco.centro;
    float dist_cuadrada = dot(v, v); // |v|^2

    if (dist_cuadrada <= (disco.radio * disco.radio)) {
        t_salida = t; // Guardamos la distancia a la colision
        return true;  // Hay interseccion valida
    }

    return false; // Intersecta el plano, pero fuera del disco



}
\end{lstlisting}

Otro formato del algoritmo en pseudocódigo es el siguiente:

\begin{algorithm}[H]
\caption{Intersección Rayo-Disco}
\begin{algorithmic}[1]
\Function{InterseccionRayoDisco}{o, d, c, n, r}
    \State denom $\gets$ $d \cdot n$
    \If{$|\text{denom}| < \epsilon$}
        \State \Return (\textbf{FALSO}, NULO)
    \EndIf
    \State $t \gets \frac{(c - o) \cdot n}{\text{denom}}$
    \If{$t < 0$}
        \State \Return (\textbf{FALSO}, NULO)
    \EndIf
    \State $p \gets o + t \cdot d$
    \If{$(p - c) \cdot (p - c) \leq r^2$}
        \State \Return (\textbf{VERDADERO}, $p$)
    \Else
        \State \Return (\textbf{FALSO}, NULO)
    \EndIf
\EndFunction
\end{algorithmic}
\end{algorithm}
\hspace{1cm}

\textit{Observación.} Sabemos que $\varepsilon$ es un valor muy pequeño (por ejemplo, $10^{-6}$) para evitar divisiones por cero numéricas.

\end{solucion}


\begin{ejercicio}
% \textbf{Problema 10.2: Intersección Rayo-Esfera}

Diseña un algoritmo para calcular la primera intersección entre un
rayo (con origen en $o$ y vector $d$, normalizado) y una esfera de radio
unidad y centro en el origen, si hay alguna.

Ten en cuenta que un punto cualquiera $p$ está en la esfera si y solo si el
módulo de $p$ es la unidad, es decir, si y solo si $F(p) = 0$, donde $F$ es
el campo escalar definido así:
\[
F(p) \equiv p \cdot p - 1
\]

Describe cómo podría usarse ese mismo algoritmo para calcular la
intersección con una esfera con centro y radio arbitrarios (este
problema puede reducirse al anterior si el rayo se traslada a un
espacio de coordenadas donde la esfera tiene centro en el origen y
radio unidad).

\vspace{0.5cm}
\centering
\begin{tikzpicture}[scale=1.5]
% Esfera (círculo en 2D)
\shade[ball color=blue!20, opacity=0.6] (0,0) circle (1cm);
\draw (0,0) circle (1cm);
\draw[->, thin] (0,0) -- (1,0) node[midway, below] {};
\fill (0,0) circle (1.5pt) node[below left] {Origen};

% Rayo
\coordinate (O) at (-2, 1.5);
\coordinate (D) at (0.8, -0.6); % Dirección aproximada
\draw[->, thick, black] (O) -- (0.5, -0.375); % Rayo visual
\draw[dashed] (0.5, -0.375) -- (1.5, -1.125); % Proyección

% Elementos del rayo
\fill (O) circle (1.5pt) node[above] {$o$};
\node at (-1.5, 1.2) {$d$};

% Puntos de intersección
% Intersección 1 (entrada)
\coordinate (P1) at (-0.57, 0.42); % Punto aproximado en la esfera
\fill[red] (P1) circle (1.5pt) node[left, xshift=-0.2cm] {$p$ (intersección)};

% Vector p
\draw[->, red, dashed] (0,0) -- (P1);

\end{tikzpicture}
\end{ejercicio}

\begin{solucion}
Para resolver el problema de la intersección entre un rayo y una esfera unitaria centrada en el origen, se procede algebraicamente sustituyendo la ecuación paramétrica del rayo en la ecuación implícita de la esfera. El objetivo es hallar el valor del parámetro $t$ (distancia desde el origen del rayo) donde ocurre el contacto.

\begin{enumerate}
\item \textbf{Planteamiento de las ecuaciones:}

La ecuación del rayo es:
$$p(t) = o + t \cdot d$$
donde $t \geq 0$.

La ecuación implícita de la esfera unitaria centrada en el origen es:
$$p \cdot p - 1 = 0 \quad (\text{o bien } \|p\|^2 = 1)$$

\item \textbf{Sustitución:}

Se sustituye $p(t)$ en la ecuación de la esfera:
$$(o + t \cdot d) \cdot (o + t \cdot d) - 1 = 0$$

Expandiendo el producto escalar (propiedad distributiva):
$$(o \cdot o) + 2t(o \cdot d) + t^2(d \cdot d) - 1 = 0$$

\item \textbf{Simplificación:}

Dado que el vector de dirección $d$ está normalizado, sabemos que $d \cdot d = 1$. La ecuación se convierte en una ecuación cuadrática de la forma $At^2 + Bt + C = 0$:
$$t^2 + 2(o \cdot d)t + (o \cdot o - 1) = 0$$

Identificamos los coeficientes:
\begin{itemize}
    \item $A = 1$
    \item $B = 2(o \cdot d)$
    \item $C = o \cdot o - 1$
\end{itemize}

\item \textbf{Resolución de la ecuación cuadrática:}
\hspace{1cm}
\textit{Observación.} Aunque la resolución sea trivial, se detalla

Usamos la fórmula general para hallar $t$:
$$t = \frac{-B \pm \sqrt{B^2 - 4AC}}{2A}$$

Sustituyendo $A=1$, $B=2(o \cdot d)$, $C=o \cdot o - 1$:
$$t = \frac{-2(o \cdot d) \pm \sqrt{4(o \cdot d)^2 - 4(o \cdot o - 1)}}{2}$$
$$t = -(o \cdot d) \pm \sqrt{(o \cdot d)^2 - (o \cdot o - 1)}$$

\item \textbf{Interpretación del discriminante ($\Delta$):}

El término dentro de la raíz es el discriminante: $\Delta = (o \cdot d)^2 - (o \cdot o - 1)$.
\begin{itemize}
    \item Si $\Delta < 0$: El rayo no toca la esfera (pasa de largo). No hay solución real.
    \item Si $\Delta = 0$: El rayo es tangente a la esfera (un punto de contacto).
    \item Si $\Delta > 0$: El rayo atraviesa la esfera (dos puntos de contacto, entrada y salida).
\end{itemize}
Se busca la \textbf{primera intersección}, que corresponde al menor valor positivo de $t$. Si ambos $t$ son negativos, la esfera está detrás del origen del rayo.

\item \textbf{Generalización para Esfera Arbitraria (Centro $C$, Radio $R$):}

Para reutilizar el algoritmo de la esfera unitaria, se aplica una transformación al rayo para llevarlo al ''espacio de la esfera unitaria''.

La ecuación de una esfera genérica es $\|p - C\|^2 = R^2$, que se puede reescribir como:
$$\left\| \frac{p - C}{R} \right\|^2 = 1$$

Si definimos un nuevo origen de rayo transformado $o'$:
$$o' = \frac{o - C}{R}$$

El problema se reduce a encontrar la intersección de un rayo que parte de $o'$ con dirección $d$ contra la esfera unitaria en el origen. Si el algoritmo base devuelve un parámetro de intersección $t_{unit}$, la distancia real $t_{real}$ en el mundo original será:
$$t_{real} = t_{unit} \times R$$

Esto se debe a que hemos escalado el espacio dividiendo por $R$, por lo que las distancias calculadas están ''comprimidas'' y deben restaurarse multiplicando por $R$.

\end{enumerate}

A continuación, se presenta el pseudocódigo que implementa esta lógica:

\begin{lstlisting}[language=C++, frame=single,  keywordstyle=\color{blue}, commentstyle=\color{green!60!black}, captionpos=b, caption={Algoritmo de Intersección Rayo-Esfera}]
// Estructuras auxiliares
struct Rayo { Vec3 origen; Vec3 direccion; }; // direccion normalizada
struct Esfera { Vec3 centro; float radio; };

// Algoritmo Base: Interseccion con Esfera Unitaria en (0,0,0)
// Retorna true si hay colision, y guarda la distancia en t_out
bool IntersectaEsferaUnidad(Vec3 o, Vec3 d, float &t_out) {
    // Coeficientes de la ecuacion t^2 + Bt + C = 0
    // A es 1 porque d esta normalizado
    float B = 2.0f * dot(o, d);
    float C = dot(o, o) - 1.0f;

    float discriminante = (B * B) - (4.0f * C);

    if (discriminante < 0.0f) return false; // No hay interseccion

    float raiz = sqrt(discriminante);

    // Soluciones de la ecuacion
    float t0 = (-B - raiz) / 2.0f; // Entrada (mas cercana)
    float t1 = (-B + raiz) / 2.0f; // Salida (mas lejana)

    // Verificar orden y positividad para encontrar la primera valida
    if (t0 > 0.001f) { 
        t_out = t0; 
        return true; 
    }
    if (t1 > 0.001f) { 
        t_out = t1; 
        return true; // El origen esta dentro de la esfera
    }

    return false; // Ambas intersecciones estan detras del rayo
}

// Algoritmo General: Reduccion al caso unitario
bool IntersectaEsferaGenerica(Rayo ray, Esfera esf, float &t_real) {
    // 1. Transformar el origen del rayo al espacio de la esfera unitaria
    // Se traslada el mundo para que el centro sea (0,0,0) y se escala por 1/R
    Vec3 o_prima = (ray.origen - esf.centro) / esf.radio;

    // La direccion d no se escala para mantener la coherencia geometrica
    // del rayo, pero esto implica que el 't' resultante estara escalado.

    float t_unit;
    if (IntersectaEsferaUnidad(o_prima, ray.direccion, t_unit)) {
        // 2. Escalar la distancia resultante para volver al mundo real
        t_real = t_unit * esf.radio;
        return true;
    }

    return false;
}
\end{lstlisting}

\end{solucion}

\begin{ejercicio} Se pide:\\

\textbf{Parte 1: Cilindro.} 
Describa cómo se puede definir el campo escalar cuyos ceros corresponden a los puntos de un cilindro de altura unidad y radio unidad (sin considerar las tapas). Utilizando esa definición, diseñe un algoritmo para calcular la intersección rayo-cilindro.

\textbf{Parte 2: Cono.} 
Describa asimismo el campo escalar y el algoritmo correspondientes a un cono de altura unidad y radio de la base unidad (sin considerar el disco de la base).

\vspace{0.5cm}
\centering
\begin{tikzpicture}[scale=1.5]
    % Cilindro
    \draw (0,0) ellipse (0.5 and 0.1);
    \draw (-0.5,0) -- (-0.5,1);
    \draw (0.5,0) -- (0.5,1);
    \draw (0,1) ellipse (0.5 and 0.1);
    \node at (0,-0.3) {Cilindro ($r=1, h=1$)};
    % Ejes locales cilindro
    \draw[->, gray, thin] (0,0) -- (0,1.5) node[right] {$y$};
    \draw[->, gray, thin] (0,0) -- (0.8,0) node[right] {$x$};

    % Cono
    \begin{scope}[xshift=2.5cm]
        \draw (0,0) ellipse (0.5 and 0.1);
        \draw (-0.5,0) -- (0,1);
        \draw (0.5,0) -- (0,1);
        \node at (0,-0.3) {Cono ($r=1, h=1$)};
        % Ejes locales cono
        \draw[->, gray, thin] (0,0) -- (0,1.5) node[right] {$y$};
    \end{scope}

    % Rayo genérico
    \draw[->, thick, blue] (-1, 1.2) -- (0.2, 0.5) node[midway, above] {Rayo};
\end{tikzpicture}
\end{ejercicio}

\begin{solucion}
Este problema aborda la intersección con superficies cuádricas canónicas (cilindros y conos) acotadas espacialmente. A diferencia de la esfera, estas superficies son infinitas por definición algebraica, por lo que el algoritmo debe incorporar un paso adicional de ''recorte'' (\textit{clipping}) para respetar la altura finita. Asumiremos, por convención estándar en gráficos, que ambos objetos están alineados con el eje $Y$.

\begin{enumerate}
    \item \textbf{Definición del Campo Escalar para el Cilindro:}

    Un cilindro infinito de radio $r=1$ alineado con el eje $Y$ cumple que, para cualquier punto $p=(x,y,z)$ en su superficie, la distancia horizontal al eje $Y$ es 1.
    \[
    x^2 + z^2 = 1
    \]
    Por lo tanto, el campo escalar $F_{cyl}(p)$ se define como:
    \[
    F_{cyl}(p) \equiv x^2 + z^2 - 1
    \]
    Los ceros de este campo ($F_{cyl}(p)=0$) definen la superficie del cilindro infinito. Para obtener el cilindro de altura unidad, se añade la condición de restricción:
    \[
    0 \le y \le 1
    \]

    \item \textbf{Algoritmo de Intersección Rayo-Cilindro:}

    Sea el rayo $p(t) = o + t \cdot d$, donde $o=(o_x, o_y, o_z)$ y $d=(d_x, d_y, d_z)$. Sustituimos las coordenadas del rayo en la ecuación implícita $x^2 + z^2 - 1 = 0$:
    \[
    (o_x + t d_x)^2 + (o_z + t d_z)^2 - 1 = 0
    \]
    Expandiendo y agrupando términos por potencias de $t$, obtenemos una ecuación cuadrática $At^2 + Bt + C = 0$:
    \begin{itemize}
        \item $A = d_x^2 + d_z^2$
        \item $B = 2(o_x d_x + o_z d_z)$
        \item $C = o_x^2 + o_z^2 - 1$
    \end{itemize}
    Se resuelve para $t$. Si existen soluciones reales $t_0, t_1$, se calcula el punto de impacto $p_{hit} = o + t \cdot d$. Finalmente, se descarta la intersección si la componente $y$ de $p_{hit}$ no cumple $0 \le p_y \le 1$.

    \item \textbf{Definición del Campo Escalar para el Cono:}

    Un cono infinito alineado con el eje $Y$, con vértice en el origen y que pasa por $(1,1,0)$, tiene una pendiente de 1 ($radio/altura = 1/1$). La relación es que el radio horizontal $\sqrt{x^2 + z^2}$ es igual a la altura $y$.
    \[
    x^2 + z^2 = y^2
    \]
    El campo escalar $F_{cone}(p)$ es:
    \[
    F_{cone}(p) \equiv x^2 + z^2 - y^2
    \]
    Con la restricción de altura $0 \le y \le 1$ (y asumiendo la hoja superior del cono, $y \ge 0$).

    \item \textbf{Algoritmo de Intersección Rayo-Cono:}

    Sustituyendo el rayo en $x^2 + z^2 - y^2 = 0$:
    \[
    (o_x + t d_x)^2 + (o_z + t d_z)^2 - (o_y + t d_y)^2 = 0
    \]
    Esto genera nuevamente coeficientes para la ecuación cuadrática:
    \begin{itemize}
        \item $A = d_x^2 + d_z^2 - d_y^2$
        \item $B = 2(o_x d_x + o_z d_z - o_y d_y)$
        \item $C = o_x^2 + o_z^2 - o_y^2$
    \end{itemize}
    Se resuelve para $t$, se obtiene $p_{hit}$ y se verifica que $0 \le p_y \le 1$.
\end{enumerate}

A continuación, el pseudocódigo unificado para ambas estructuras:

\begin{lstlisting}[language=C++, frame=single,  keywordstyle=\color{blue}, commentstyle=\color{green!60!black}, captionpos=b, caption={Algoritmo Genérico para Cuádricas Acotadas}]
// TipoObjeto: CILINDRO o CONO
bool IntersectaCuadrica(Rayo ray, TipoObjeto tipo, float &t_out) {
    float A, B, C;
    float ox = ray.origen.x, oz = ray.origen.z, oy = ray.origen.y;
    float dx = ray.direccion.x, dz = ray.direccion.z, dy = ray.direccion.y;
    if (tipo == CILINDRO) {
        // x^2 + z^2 - 1 = 0
        A = dx*dx + dz*dz;
        B = 2*(ox*dx + oz*dz);
        C = ox*ox + oz*oz - 1;
    } else { // CONO
        // x^2 + z^2 - y^2 = 0
        A = dx*dx + dz*dz - dy*dy;
        B = 2*(ox*dx + oz*dz - oy*dy);
        C = ox*ox + oz*oz - oy*oy;
    }

    float discrim = B*B - 4*A*C;
    if (discrim < 0) return false; // No hay interseccion con la superficie infinita

    float raiz = sqrt(discrim);
    float t0 = (-B - raiz) / (2*A);
    float t1 = (-B + raiz) / (2*A);

    // Buscar la interseccion mas cercana que este dentro de la altura
    float t_candidata = t0;
    if (t0 < 0.001) t_candidata = t1;
    if (t_candidata < 0.001) return false;

    // Calcular la altura del punto de impacto
    float y_impacto = oy + t_candidata * dy;

    // VALIDACION DE ALTURA (Clipping)
    // El cilindro y el cono tienen altura 1 (de y=0 a y=1)
    if (y_impacto >= 0.0 && y_impacto <= 1.0) {
        t_out = t_candidata;
        return true;
    }

    // Si t0 falla, probamos con t1 (podria ser que entramos por arriba/abajo)
    // Nota: Esto es necesario si estamos dentro del objeto o para el ''lado lejano''
    y_impacto = oy + t1 * dy;
    if (t1 > 0.001 && y_impacto >= 0.0 && y_impacto <= 1.0) {
         t_out = t1;
         return true;
    }

    return false;
}
\end{lstlisting}
\end{solucion}

\section{Sesión 11}

\begin{ejercicio}
% \textbf{Problema 11.1: Curva Hermite para una trayectoria}

Implementar un proyecto en Godot en el cual el nodo raíz tiene un script que define dos arrays con: una serie de $n$ puntos $p_{0}, p_{1}, \dots, p_{n-1}$ del plano $y=0$, y una serie de instantes de tiempo $t_{0}, t_{1}, \dots, t_{n-1}$ (en segundos) con $t_{0}=0$.

\begin{enumerate}
    \item Sitúa en cada uno de esos puntos un disco pequeño visible, a modo de marcador.
    \item Incluye una función para calcular la posición y velocidad de la curva de Hermite que pasa por los puntos en los instantes dados, a partir de un $t$ en $[0, t_{n-1}]$.
    \item En cada punto $p_{i}$ el vector de velocidad $v_{i}$ se calcula a partir de los puntos anterior y siguiente.
    \item En el método \texttt{\_process(delta)} del script, calcula la posición y velocidad de la curva en el tiempo transcurrido desde el inicio, y mueve un objeto (un coche, por ejemplo) a esa posición, alineado con la dirección de la curva.
\end{enumerate}
\end{ejercicio}

\begin{solucion}
Se presenta a continuación la resolución detallada del problema, fundamentada en la teoría de curvas paramétricas y Splines Cúbicos de Hermite expuesta en las diapositivas del curso (páginas 63-91).

\subsubsection*{1. Fundamentos Teóricos: Interpolación de Hermite a Trozos}

Para definir una trayectoria suave que pase por una secuencia de puntos $p_0, \dots, p_{n-1}$ en tiempos específicos, se utiliza una curva definida a trozos. Para un instante de tiempo $t$ que se encuentra en el intervalo $[t_i, t_{i+1}]$, la posición se obtiene interpolando entre $p_i$ y $p_{i+1}$ considerando las velocidades (tangentes) $v_i$ y $v_{i+1}$ en dichos puntos.

Se definen las siguientes variables auxiliares para el i-ésimo intervalo:
\begin{itemize}
    \item La duración del intervalo: $s_i = t_{i+1} - t_i$.
    \item El parámetro normalizado de tiempo: $u = \frac{t - t_i}{s_i}$, donde $0 \le u \le 1$.
\end{itemize}

La ecuación vectorial para la posición $P(t)$ en este intervalo viene dada por la combinación lineal de las bases de Hermite:
\begin{equation}
P(t) = p_i h_{00}(u) + p_{i+1} h_{01}(u) + s_i v_i h_{10}(u) + s_i v_{i+1} h_{11}(u)
\end{equation}

Es crucial notar que las velocidades $v$ se multiplican por la duración del intervalo $s_i$ para ajustar la magnitud de la tangente al dominio normalizado $[0,1]$. Las funciones base son:
\begin{align*}
h_{00}(u) &= 2u^3 - 3u^2 + 1 \\
h_{01}(u) &= -2u^3 + 3u^2 \\
h_{10}(u) &= u^3 - 2u^2 + u \\
h_{11}(u) &= u^3 - u^2
\end{align*}

\subsubsection*{2. Cálculo Automático de Velocidades (Tangentes)}

Dado que el enunciado no proporciona las velocidades explícitas, estas se calculan numéricamente para asegurar que la curva sea suave (continuidad $C^1$) en los puntos de unión. Se utiliza el método de diferencias finitas centradas (Catmull-Rom):

\begin{equation}
v_i = \frac{p_{i+1} - p_{i-1}}{t_{i+1} - t_{i-1}}, \quad \text{para } 0 < i < n-1
\end{equation}

Para los puntos extremos ($i=0$ e $i=n-1$), se asume velocidad nula ($v=0$) o se puede usar una diferencia simple, pero el enunciado sugiere seguir el ejemplo de suavizado estándar.

\subsubsection*{3. Representación Visual de la Trayectoria}

La siguiente figura ilustra la geometría del problema: los puntos de control (rojos) definen el paso obligado, mientras que los vectores de velocidad calculados (azules) definen la curvatura en dichos puntos.

\begin{center}
\begin{tikzpicture}[scale=1.1, >=stealth]

    % Puntos de paso (plano y = 0)
    \coordinate (P0) at (0,0);
    \coordinate (P1) at (3,1.8);
    \coordinate (P2) at (6,0);
    \coordinate (P3) at (9,1);

    % Curva Hermite (aproximada con Bézier cúbica por tramo)
    \draw[thick, red]
        (P0)
        .. controls +(1.5,0.6) and +(-1.5,0.6) .. (P1)
        .. controls +(1.5,-0.6) and +(-1.5,-0.6) .. (P2)
        .. controls +(1.5,0.4) and +(-1.5,0.4) .. (P3);

    % Puntos de control
    \foreach \p in {P0,P1,P2,P3}
        \fill[black] (\p) circle (2pt);

    % Etiquetas de los puntos
    \node[below left] at (P0) {$p_0,t_0$};
    \node[above]      at (P1) {$p_1,t_1$};
    \node[below]      at (P2) {$p_2,t_2$};
    \node[above right]at (P3) {$p_3,t_3$};

    % Tangentes (velocidades) corregidas
    \draw[->, blue, thick] (P1) -- (P2) node[midway, above right] {$v_1$};
    \draw[->, blue, thick] (P2) -- (P3) node[midway, above right] {$v_2$};

    % Cuerda usada para calcular v1
    \draw[dashed, gray] (P0) -- (P2);
    \node[gray, font=\footnotesize] at (3,-0.6)
        {$v_1 \parallel (p_2 - p_0)$};

\end{tikzpicture}
\end{center}


\subsubsection*{4. Implementación en GDScript}

El siguiente código implementa la lógica completa. Se asume que este script se adjunta al nodo raíz de la escena y que existe un nodo hijo llamado ''Coche'' (MeshInstance3D o similar).

\begin{lstlisting}[language=Python,  frame=single, breaklines=true, numbers=left, numberstyle=\tiny, caption={Script de Interpolación Hermite}]
extends Node3D

# Datos de entrada: Puntos de paso y sus instantes de tiempo
var puntos = [
    Vector3(0, 0, 0),
    Vector3(4, 0, 4),
    Vector3(8, 0, -2),
    Vector3(12, 0, 5)
]
var tiempos = [0.0, 2.0, 5.0, 8.0] # t0 debe ser 0.0

# Almacen de velocidades calculadas
var velocidades = []

# Referencia al objeto visual (el coche)
onready var objeto_movil = $Coche
var tiempo_actual = 0.0

func _ready():
    # 1. Calcular tangentes automaticamente
    calcular_velocidades_hermite()
    
    # 2. Visualizar marcadores (discos)
    crear_marcadores_visuales()

func calcular_velocidades_hermite():
    var n = puntos.size()
    velocidades.resize(n)
    
    # Velocidad 0 en extremos (arranque y parada suave)
    velocidades[0] = Vector3.ZERO
    velocidades[n-1] = Vector3.ZERO
    
    # Calculo para puntos intermedios: v_i = (p_next - p_prev) / (t_next - t_prev)
    for i in range(1, n - 1):
        var dist_vector = puntos[i+1] - puntos[i-1]
        var intervalo_t = tiempos[i+1] - tiempos[i-1]
        velocidades[i] = dist_vector / intervalo_t

func crear_marcadores_visuales():
    for p in puntos:
        var marcador = CSGCylinder3D.new()
        marcador.radius = 0.3
        marcador.height = 0.1
        marcador.material = StandardMaterial3D.new()
        marcador.material.albedo_color = Color(1, 0, 0) # Rojo
        add_child(marcador)
        marcador.global_position = p

# Funcion principal de interpolacion
func obtener_posicion_velocidad(t):
    var n = puntos.size()
    
    # Caso limite: si t supera el tiempo final
    if t >= tiempos[n-1]:
        return {''pos'': puntos[n-1], ''dir'': Vector3.FORWARD}
    
    # Buscar el intervalo [i, i+1] correspondiente al tiempo t
    var i = 0
    while i < n - 1 and t > tiempos[i+1]:
        i += 1
        
    # Datos del tramo actual
    var p0 = puntos[i]
    var p1 = puntos[i+1]
    var v0 = velocidades[i]
    var v1 = velocidades[i+1]
    var t0 = tiempos[i]
    var t1 = tiempos[i+1]
    
    # Parametro u normalizado (0 a 1)
    var s = t1 - t0 # Duracion del intervalo
    var u = (t - t0) / s
    
    # Pre-calculo de potencias de u
    var u2 = u * u
    var u3 = u2 * u
    
    # Funciones base de Hermite (h00, h10, h01, h11)
    var h00 = 2*u3 - 3*u2 + 1
    var h10 = u3 - 2*u2 + u
    var h01 = -2*u3 + 3*u2
    var h11 = u3 - u2
    
    # Interpolacion de la Posicion (notese v * s para escalar la tangente)
    var pos = h00*p0 + h10*s*v0 + h01*p1 + h11*s*v1
    
    # Calculo de la velocidad instantanea (Derivada de P respecto a t)
    # Derivadas de las bases respecto a u:
    var dh00 = 6*u2 - 6*u
    var dh10 = 3*u2 - 4*u + 1
    var dh01 = -6*u2 + 6*u
    var dh11 = 3*u2 - 2*u
    
    # v(t) = P'(u) * (du/dt) = P'(u) * (1/s)
    var vel = (dh00*p0 + dh10*s*v0 + dh01*p1 + dh11*s*v1) / s
    
    return {''pos'': pos, ''dir'': vel}

func _process(delta):
    tiempo_actual += delta
    
    # Reiniciar bucle si termina
    if tiempo_actual > tiempos.back():
        tiempo_actual = 0.0
        
    # Calcular estado fisico
    var estado = obtener_posicion_velocidad(tiempo_actual)
    
    # Aplicar transformaciones
    if objeto_movil:
        objeto_movil.global_position = estado[''pos'']
        
        # Orientar el objeto segun el vector de velocidad (tangente)
        # Se evita el error si la velocidad es muy cercana a cero
        if estado[''dir''].length_squared() > 0.001:
            var objetivo_mirar = estado[''pos''] + estado[''dir'']
            objeto_movil.look_at(objetivo_mirar, Vector3.UP)
\end{lstlisting}

\subsubsection*{Explicación Paso a Paso del Código}

\begin{enumerate}
\item \textbf{Inicialización (\texttt{\_ready}):} Se calculan las velocidades (tangentes) en cada punto de control usando diferencias centradas, y se crean los marcadores visuales en la escena para cada punto.

\item \textbf{Cálculo de Velocidades:} Para los puntos intermedios, la velocidad se obtiene como el cociente entre la diferencia de posiciones y la diferencia de tiempos de los puntos anterior y siguiente. En los extremos, se asigna velocidad cero.

\item \textbf{Interpolación Hermite:} La función principal busca el intervalo de tiempo correspondiente y normaliza el parámetro temporal ($u$) al rango $[0,1]$. Se aplican las bases polinómicas de Hermite para calcular la posición y la velocidad instantánea en ese tramo.

\item \textbf{Actualización por Frame (\texttt{\_process}):} En cada fotograma, se incrementa el tiempo, se calcula la posición y dirección de la curva en ese instante, y se mueve el objeto (por ejemplo, un coche) a esa posición, orientándolo según la dirección de la curva usando \texttt{look\_at}.
\end{enumerate}

\end{solucion}

\begin{ejercicio}
% \textbf{Problema 11.2: Posición oscilante}

Crea un proyecto Godot con una animación de una esfera cuya posición en $X$ oscile periódicamente, con estas condiciones:
\begin{enumerate}
    \item El centro de la esfera tiene coordenada $Z$ igual a 0, su coordenada $Y$ es igual al radio, y su coordenada $X$ varía entre $-s$ y $+s$, donde $s > 0$ es una constante declarada en el script.
    \item El período (tiempo en volver al mismo punto viajando en la misma dirección) es una constante $T > 0$ declarada en el script (con unidades de segundos).
    \item La esfera se mueve siempre a velocidad constante en magnitud (es siempre $s/T$), y el signo depende de la dirección.
    \item Tu animación debe producir esa velocidad constante, incluso teniendo en cuenta que los sucesivos valores de delta pueden cambiar entre frames.
    \item Especialmente, la magnitud de la velocidad debe ser constante aunque entre dos frames haya ocurrido un cambio de dirección en un extremo.
\end{enumerate}
\end{ejercicio}

\begin{solucion}
Se aborda la resolución de este problema mediante la programación de un script en GDScript, gestionando manualmente la actualización de la posición en cada fotograma para garantizar una velocidad constante y un rebote preciso en los extremos.

\subsubsection*{1. Análisis del Movimiento y Velocidad}

El movimiento solicitado describe una onda triangular. La esfera oscila entre $-s$ y $+s$. Un ciclo completo (Período $T$) consiste en el recorrido:
\[ 0 \to +s \to 0 \to -s \to 0 \]
La distancia total recorrida en un ciclo es $D = s + s + s + s = 4s$.

Para que este ciclo se complete exactamente en $T$ segundos con velocidad uniforme, la magnitud de la velocidad ($v$) debe ser:
\begin{equation}
v = \frac{\text{Distancia Total}}{\text{Tiempo}} = \frac{4s}{T}
\end{equation}

\textit{Nota técnica: El enunciado indica entre paréntesis que la velocidad es $s/T$. Sin embargo, matemáticamente, si la velocidad fuera $s/T$, el objeto tardaría $4T$ en completar el ciclo en lugar de $T$. En esta solución se prioriza el cumplimiento del Período $T$, por lo que se utilizará $v = 4s/T$.}

\subsubsection*{2. Algoritmo de Actualización y Corrección de ''Overshoot''}

El reto principal en sistemas de tiempo real (como el método \texttt{\_process} de Godot) es que el tiempo entre frames (\texttt{delta}) es variable. Si el objeto está cerca de un extremo (por ejemplo, $x=4.9$ y $s=5.0$) y el siguiente paso es grande (0.2), la posición teórica sería $5.1$, excediendo el límite.

Para mantener la velocidad constante y la precisión:
\begin{enumerate}
    \item Se calcula el desplazamiento propuesto: $\Delta x = v \cdot \delta$.
    \item Se suma a la posición actual.
    \item Si la nueva posición excede los límites ($s$ o $-s$), se calcula el exceso (\textit{overshoot}).
    \item Se ''refleja'' el exceso hacia adentro del intervalo y se invierte la dirección. Esto simula que el rebote ocurrió en el instante exacto entre los frames.
\end{enumerate}

\subsubsection*{3. Implementación en GDScript}

El siguiente código se debe adjuntar a un nodo en la escena (por ejemplo, un \texttt{Node3D}) que contenga un hijo llamado ''Esfera'' (visualización).

\begin{lstlisting}[language=Python,  frame=single, numbers=left, caption={Script de Oscilación Triangular Controlada}]
extends Node3D

# Variables de configuracion (exportadas para editar en el inspector)
export var s: float = 5.0      # Amplitud maxima (metros)
export var T: float = 2.0      # Periodo completo (segundos)
export var radio: float = 0.5  # Radio visual de la esfera

# Variables de estado
var x_actual: float = 0.0
var direccion: int = 1         # 1: Derecha, -1: Izquierda
var velocidad: float = 0.0     # Magnitud de la velocidad

# Referencia al nodo visual
onready var esfera = $Esfera

func _ready():
    # Calculo de la velocidad necesaria para cumplir el periodo T
    # Distancia total por ciclo = 4 * s
    if T > 0:
        velocidad = (4.0 * s) / T
    else:
        velocidad = 0.0
        
    # Ajuste visual inicial
    if esfera:
        # Si es un CSGSphere3D, ajustamos el radio propiedad
        if ''radius'' in esfera:
            esfera.radius = radio
        # Posicion inicial
        esfera.position = Vector3(0, radio, 0)

func _process(delta):
    # 1. Calcular el paso teorico en este frame
    var distancia_paso = velocidad * delta
    
    # 2. Aplicar movimiento
    x_actual += distancia_paso * direccion
    
    # 3. Verificacion de limites y correccion de rebote
    
    # Limite derecho (+s)
    if x_actual > s:
        var exceso = x_actual - s
        x_actual = s - exceso   # Reflejar el exceso hacia atras
        direccion = -1          # Invertir direccion
        
    # Limite izquierdo (-s)
    elif x_actual < -s:
        var exceso = -s - x_actual # Cuanto nos pasamos por la izquierda
        x_actual = -s + exceso     # Reflejar el exceso hacia delante
        direccion = 1              # Invertir direccion
        
    # 4. Actualizar la posicion del nodo visual
    if esfera:
        esfera.position.x = x_actual
        esfera.position.y = radio
        esfera.position.z = 0.0
\end{lstlisting}

Este algoritmo asegura que la magnitud de la velocidad se mantenga constante en todo momento, respetando la física del rebote perfecto sin perder tiempo ni energía en los extremos.
\end{solucion}


\begin{ejercicio}
%\textbf{Problema 11.3: Animación de un reloj}

Desarrolla un proyecto Godot para el ejemplo de animación de un reloj con tres agujas. Las condiciones especificadas en la teoría son:
\begin{enumerate}
    \item Se desea visualizar un reloj con tres agujas: horas, minutos y segundos.
    \item Cada aguja se modela como una malla de polígonos en posición vertical (paralelo al eje Y), con el origen en el punto del eje del reloj.
    \item Se usan matrices de rotación en torno al eje Z.
    \item Los ángulos de rotación dependen linealmente del tiempo $t$ (segundos transcurridos desde el comienzo del día).
\end{enumerate}
\end{ejercicio}

\begin{solucion}
Se procede a la implementación de un sistema de animación jerárquica para simular un reloj analógico funcional en tiempo real. La solución se basa en la aplicación directa de las transformaciones de rotación descritas en las diapositivas 32 a 35 del material de curso.

\subsubsection*{1. Modelo Matemático: Ángulos y Tiempo}

Según la teoría, el estado de las agujas está determinado por tres ángulos $\theta_h, \theta_m, \theta_s$ que son funciones lineales del tiempo $t$. El tiempo $t$ representa los segundos totales transcurridos en el ciclo actual (el ciclo de 12 horas para la aguja horaria).

Las relaciones angulares (en radianes) son:
\begin{itemize}
    \item \textbf{Segundero ($\theta_s$):} Da una vuelta completa ($2\pi$) cada 60 segundos.
    \begin{equation}
    \theta_s(t) = \frac{2\pi}{60} \cdot t
    \end{equation}
    \item \textbf{Minutero ($\theta_m$):} Da una vuelta completa cada hora ($60^2 = 3600$ segundos).
    \begin{equation}
    \theta_m(t) = \frac{2\pi}{3600} \cdot t
    \end{equation}
    \item \textbf{Horario ($\theta_h$):} Da una vuelta completa cada 12 horas ($12 \cdot 60^2 = 43200$ segundos).
    \begin{equation}
    \theta_h(t) = \frac{2\pi}{43200} \cdot t
    \end{equation}
\end{itemize}

\textit{Nota de implementación:} En la convención estándar matemática y de Godot, una rotación positiva en el eje Z es antihoraria. Dado que los relojes giran en sentido horario, aplicaremos el signo negativo a estos ángulos en el código ($\text{rotación} = -\theta$).

\subsubsection*{2. Estructura del Grafo de Escena}

Para cumplir con el requisito de que las agujas tengan su origen en el eje de rotación pero se extiendan a lo largo del eje Y positivo, utilizaremos una jerarquía de nodos:
\begin{enumerate}
    \item \textbf{Nodo Raíz (Reloj):} Contenedor principal.
    \item \textbf{Pivotes (Node3D):} Tres nodos hijos situados en $(0,0,0)$. Estos nodos serán los que roten.
    \item \textbf{Mallas (MeshInstance3D):} Hijos de los pivotes. Se desplazarán verticalmente (offset) para que su base coincida con el pivote, logrando el efecto de girar desde el extremo.
\end{enumerate}

\subsubsection*{3. Implementación en GDScript}

El siguiente script crea la geometría procedimentalmente (para facilitar la prueba sin modelos externos) y aplica la lógica de rotación basada en la hora del sistema.

\begin{lstlisting}[language=Python,  frame=single, numbers=left, caption={Script del Reloj Analógico}]
extends Node3D

# Referencias a los nodos de las agujas (Pivotes)
var pivote_segundos: Node3D
var pivote_minutos: Node3D
var pivote_horas: Node3D

func _ready():
    # 1. Construccion procedimental de la escena
    crear_geometria_reloj()

func crear_geometria_reloj():
    # Creamos una esfera central como base
    var esfera = CSGSphere3D.new()
    esfera.radius = 0.5
    add_child(esfera)
    
    # Creamos las tres agujas. 
    # Usamos una funcion auxiliar para configurar: (Nombre, Largo, Ancho, Color)
    pivote_horas = crear_aguja(''Horas'', 2.0, 0.2, Color.black)
    pivote_minutos = crear_aguja(''Minutos'', 3.0, 0.15, Color.darkgray)
    pivote_segundos = crear_aguja(''Segundos'', 3.5, 0.05, Color.red)

func crear_aguja(nombre, largo, ancho, color) -> Node3D:
    # 1. El Pivote: Este nodo estara en (0,0,0) y es el que rotamos
    var pivote = Node3D.new()
    pivote.name = ''Pivote'' + nombre
    add_child(pivote)
    
    # 2. La Malla Visual: Hija del pivote
    var mesh = CSGBox3D.new()
    mesh.size = Vector3(ancho, largo, 0.1)
    
    # IMPORTANTE: Desplazamos la malla hacia arriba (Y+) la mitad de su largo.
    # Asi, el centro de rotacion (el pivote) queda en la base de la aguja.
    mesh.position = Vector3(0, largo / 2.0, 0)
    
    # Material
    var material = StandardMaterial3D.new()
    material.albedo_color = color
    mesh.material = material
    
    pivote.add_child(mesh)
    return pivote

func _process(delta):
    # 1. Obtener el tiempo actual del sistema
    var tiempo = Time.get_time_dict_from_system()
    var horas = tiempo[''hour'']
    var minutos = tiempo[''minute'']
    var segundos = tiempo[''second'']
    
    # 2. Calcular t (segundos totales desde las 12:00)
    # Ajustamos horas a formato 12h para la formula
    horas = horas % 12
    
    # Calculo de alta precision para movimiento suave (incluyendo milisegundos si se quisiera)
    # t para segundos (ciclo 60s)
    var t_sec = segundos 
    # t para minutos (ciclo 3600s). Sumamos segundos para movimiento continuo
    var t_min = (minutos * 60.0) + segundos
    # t para horas (ciclo 43200s). Sumamos minutos y segundos
    var t_hour = (horas * 3600.0) + (minutos * 60.0) + segundos
    
    # 3. Calcular angulos (Theta) usando las formulas de la teoria
    # Theta = (2 * PI / Periodo) * t
    # Usamos negativo para rotacion en sentido horario (Clockwise)
    
    var theta_s = -(2.0 * PI / 60.0) * t_sec
    var theta_m = -(2.0 * PI / 3600.0) * t_min
    var theta_h = -(2.0 * PI / 43200.0) * t_hour
    
    # 4. Aplicar rotacion en el eje Z
    if pivote_segundos:
        pivote_segundos.rotation.z = theta_s
    if pivote_minutos:
        pivote_minutos.rotation.z = theta_m
    if pivote_horas:
        pivote_horas.rotation.z = theta_h
\end{lstlisting}

\subsubsection*{4. Análisis del Código}

\begin{enumerate}
    \item \textbf{Generación de Geometría:} Se sigue la especificación de la diapositiva 35: un nodo raíz (la esfera central) y un nodo para cada aguja. Dentro de cada aguja, se separa la transformación (el Pivote) de la geometría (la Malla). El desplazamiento \texttt{mesh.position.y = largo / 2.0} es crítico para que la rotación ocurra en el extremo de la aguja y no en su centro geométrico.
    \item \textbf{Cálculo del Tiempo ($t$):} En lugar de un acumulador simple \texttt{delta}, utilizamos \texttt{Time.get\_time\_dict\_from\_system()}. Esto sincroniza la animación con la hora real. Para las agujas de minutos y horas, se suman las fracciones correspondientes de las unidades menores (por ejemplo, a los minutos se le suman los segundos convertidos) para lograr un movimiento fluido y realista, en lugar de saltos discretos.
    \item \textbf{Aplicación de la Rotación:} Se asignan los ángulos calculados a la propiedad \texttt{rotation.z}. El signo negativo asegura que el giro sea en el sentido de las agujas del reloj, corrigiendo la convención matemática estándar (antihoraria) del sistema de coordenadas de Godot.
\end{enumerate}

\end{solucion}

\begin{ejercicio}
%\textbf{Problema 11.4: Animación de un péndulo}

Desarrolla un proyecto Godot para el ejemplo de animación de un péndulo.
Las condiciones teóricas especificadas son:
\begin{enumerate}
    \item El péndulo consiste en una masa colgando de un punto fijo por una cuerda de longitud $l$.
    \item El ángulo $\theta$ entre la cuerda y la vertical varía con el tiempo $t$.
    \item La oscilación es periódica con un período $T > 0$ (tiempo en segundos para completar un ciclo).
    \item El ángulo oscila entre $-\theta_m$ y $\theta_m$.
    \item La función que describe el ángulo es $\theta(t) = \theta_m \cdot \sin(\frac{2\pi t}{T})$ (o una variante cosinusoidal equivalente).
\end{enumerate}
\end{ejercicio}

\begin{solucion}
Se detalla a continuación la implementación de un péndulo físico simple utilizando animación procedimental en Godot. La solución aplica las fórmulas de oscilación armónica descritas en las diapositivas 36 a 38 del material de referencia.

\subsubsection*{1. Modelo Matemático del Movimiento}

El movimiento del péndulo se modela mediante una función sinusoidal que define el ángulo de rotación $\theta(t)$ en el eje Z.

Según la teoría proporcionada:
\begin{itemize}
    \item Se define una función base oscilante $f(t) = \sin(\pi t)$ que tiene un período de 2 unidades.
    \item Para adaptar esta función a un período arbitrario $T$, se escala el tiempo: $\theta(t) = \theta_{max} \cdot f(\frac{2t}{T})$.
\end{itemize}

Sustituyendo la función base, obtenemos la fórmula final para la implementación:
\begin{equation}
\theta(t) = \theta_{max} \cdot \sin\left(\pi \cdot \frac{2t}{T}\right) = \theta_{max} \cdot \sin\left(\frac{2\pi t}{T}\right)
\end{equation}

Donde:
\begin{itemize}
    \item $\theta_{max}$ es la amplitud máxima (en radianes).
    \item $T$ es el período de oscilación (en segundos).
    \item $t$ es el tiempo acumulado.
\end{itemize}

\subsubsection*{2. Estructura del Grafo de Escena}

Para simular correctamente el péndulo, es fundamental establecer la jerarquía de nodos adecuada, ya que la rotación debe ocurrir en el punto de anclaje (extremo superior) y no en el centro de masa del péndulo.

\begin{enumerate}
    \item \textbf{Nodo Raíz (Soporte):} Punto fijo en el espacio.
    \item \textbf{Pivote (Node3D):} Hijo del soporte. Este nodo se ubica en $(0,0,0)$ relativo al soporte y es el que recibirá la rotación $\theta(t)$.
    \item \textbf{Varilla (MeshInstance3D):} Hija del Pivote. Se desplaza verticalmente hacia abajo (eje Y negativo) una distancia $L/2$ y se escala para tener longitud $L$.
    \item \textbf{Masa/Bob (MeshInstance3D):} Hija del Pivote (o de la varilla). Se desplaza verticalmente hacia abajo una distancia $L$.
\end{enumerate}

\subsubsection*{3. Implementación en GDScript}

El siguiente script se debe adjuntar al nodo \textbf{Pivote}. Este script genera la geometría visual procedimentalmente para facilitar la prueba y aplica la fórmula de oscilación.

\begin{lstlisting}[language=Python,  frame=single, numbers=left, caption={Script del Péndulo Oscilante}]
extends Node3D

# Parametros fisicos configurables
export var theta_max_degrees: float = 45.0 # Amplitud maxima en grados
export var periodo: float = 2.0            # Periodo T en segundos
export var longitud_cuerda: float = 3.0    # Longitud L

# Variables internas
var tiempo_acumulado: float = 0.0
var theta_max_rad: float = 0.0

# Referencias a los nodos visuales (se crearan por codigo si no existen)
var varilla: CSGBox3D
var masa: CSGSphere3D

func _ready():
    # Convertir grados a radianes para las funciones trigonometricas
    theta_max_rad = deg_to_rad(theta_max_degrees)
    
    # Construccion procedimental del pendulo visual
    construir_geometria()

func construir_geometria():
    # 1. Crear la varilla (Cuerda)
    varilla = CSGBox3D.new()
    varilla.size = Vector3(0.1, longitud_cuerda, 0.1) # Grosor y largo
    
    # IMPORTANTE: Desplazar la varilla hacia abajo la mitad de su longitud.
    # Asi, el extremo superior coincide con el origen del Pivote (0,0,0).
    varilla.position = Vector3(0, -longitud_cuerda / 2.0, 0)
    
    # Material visual para la varilla
    var mat_varilla = StandardMaterial3D.new()
    mat_varilla.albedo_color = Color.gray
    varilla.material = mat_varilla
    
    add_child(varilla)
    
    # 2. Crear la masa (Esfera en el extremo)
    masa = CSGSphere3D.new()
    masa.radius = 0.4
    
    # La masa se coloca al final de la cuerda
    masa.position = Vector3(0, -longitud_cuerda, 0)
    
    # Material visual para la masa
    var mat_masa = StandardMaterial3D.new()
    mat_masa.albedo_color = Color.red
    masa.material = mat_masa
    
    add_child(masa)

func _process(delta):
    # 1. Acumular el tiempo
    tiempo_acumulado += delta
    
    # Opcional: Evitar desbordamiento de float reseteando cada periodo
    if tiempo_acumulado > periodo:
        tiempo_acumulado -= periodo
        
    # 2. Calcular el angulo actual usando la formula armonica
    # theta(t) = theta_max * sin(2 * PI * t / T)
    var theta = theta_max_rad * sin((2.0 * PI * tiempo_acumulado) / periodo)
    
    # 3. Aplicar la rotacion al Pivote
    # Se rota en el eje Z para oscilar izquierda-derecha
    rotation.z = theta
\end{lstlisting}

\subsubsection*{4. Explicación del Código}

\begin{itemize}
    \item \textbf{Setup (\texttt{\_ready}):} Se convierten los grados a radianes, ya que las funciones matemáticas de Godot y la propiedad \texttt{rotation} trabajan en radianes. Se invoca la construcción de la malla.
    \item \textbf{Geometría (\texttt{construir\_geometria}):} Se crean primitivas CSG. El punto clave es \texttt{varilla.position.y = -longitud\_cuerda / 2.0}. Esto asegura que, aunque el centro geométrico del cubo está en su mitad, visualmente la varilla ''cuelga'' del nodo padre (el Pivote en 0,0,0). La masa se coloca en \texttt{-longitud\_cuerda}.
    \item \textbf{Animación (\texttt{\_process}):}
    \begin{enumerate}
        \item Se actualiza el tiempo $t$.
        \item Se calcula el valor de la función seno, que oscilará suavemente entre $-1$ y $+1$.
        \item Se multiplica por $\theta_{max}$ para escalar la oscilación a la amplitud deseada.
        \item Se asigna directamente a \texttt{rotation.z}. Al ser este nodo el padre de la varilla y la masa, ambos rotarán rígidamente alrededor del punto de anclaje, simulando la física del péndulo.
    \end{enumerate}
\end{itemize}

\end{solucion}

\begin{ejercicio}
% \textbf{Problema 11.5: Animación de una bala de cañón}

Desarrolla un proyecto Godot para el ejemplo de animación de una bala de cañón.
Las condiciones y supuestos teóricos son:
\begin{enumerate}
    \item La bola sale del cañón en una posición inicial $p(0)$ y con una velocidad inicial conocida $v_0$.
    \item La bola está sujeta a la gravedad ($g = 9.8 \text{ m/s}^2$).
    \item No se consideran efectos de fricción con el aire.
    \item La animación simula la trayectoria hasta que la bola vuelve a la altura inicial.
    \item Se utiliza la ecuación de la curva paramétrica: $p(t) = p(0) + v_0 t + \frac{1}{2} a t^2$, donde $a = (0, -g, 0)$.
\end{enumerate}
\end{ejercicio}

\begin{solucion}
Se presenta la implementación de la trayectoria parabólica de un proyectil. A diferencia de las simulaciones físicas que integran la velocidad frame a frame (Euler), este ejercicio pide implementar la solución analítica exacta (curva paramétrica) dependiente del tiempo acumulado $t$.

\subsubsection*{1. Modelo Físico-Matemático}

La posición $p(t)$ en el instante $t$ se calcula mediante la fórmula vectorial del movimiento uniformemente acelerado:
\begin{equation}
p(t) = p_0 + v_0 \cdot t + \frac{1}{2} \cdot \vec{a} \cdot t^2
\end{equation}

Desglosando los componentes:
\begin{itemize}
    \item \textbf{Vector aceleración ($\vec{a}$):} Si la gravedad actúa hacia abajo en el eje Y, entonces $\vec{a} = (0, -9.8, 0)$.
    \item \textbf{Vector velocidad inicial ($v_0$):} $(v_x, v_y, v_z)$. Es crucial que $v_y > 0$ para que haya un arco parabólico.
    \item \textbf{Duración del vuelo:} El proyectil vuelve a la altura $y=0$ (suponiendo $p_0=0$) en el instante $t_{fin} = \frac{2 v_{0y}}{g}$.
\end{itemize}

\subsubsection*{2. Configuración de la Escena}

\begin{enumerate}
    \item \textbf{Nodo Raíz (Node3D):} Controlador de la escena.
    \item \textbf{Suelo (CSGBox3D):} Referencia visual estática.
    \item \textbf{Bala (MeshInstance3D o CSGSphere3D):} El objeto móvil. Inicialmente en $(0, 0, 0)$.
\end{enumerate}

\subsubsection*{3. Implementación en GDScript}

El siguiente script controla la posición absoluta de la bala basándose en el tiempo transcurrido desde el disparo.

\begin{lstlisting}[language=Python,  frame=single, numbers=left, caption={Script de Trayectoria Balística Paramétrica}]
extends Node3D

# Parametros de lanzamiento (Vector3)
# v_y debe ser positiva para que suba.
# v_z o v_x dan el desplazamiento horizontal.
export var velocidad_inicial: Vector3 = Vector3(0, 15, 10) 
export var gravedad: float = 9.8

# Variables de estado
var tiempo_vuelo: float = 0.0
var posicion_inicial: Vector3
var vector_gravedad: Vector3

# Referencia al objeto visual
onready var bala = $Bala

func _ready():
    # Guardamos la posicion original para reiniciar el ciclo
    if bala:
        posicion_inicial = bala.global_position
    else:
        posicion_inicial = Vector3.ZERO
        
    # Pre-calculamos el vector de aceleracion
    vector_gravedad = Vector3(0, -gravedad, 0)
    
    # Configuracion visual opcional (crear bala si no existe)
    if not bala:
        crear_bala_procedimental()

func crear_bala_procedimental():
    var mesh = CSGSphere3D.new()
    mesh.radius = 0.5
    mesh.name = ''Bala''
    add_child(mesh)
    bala = mesh
    mesh.global_position = posicion_inicial
    
    # Material rojo para visibilidad
    var mat = StandardMaterial3D.new()
    mat.albedo_color = Color(1, 0, 0)
    mesh.material = mat

func _process(delta):
    # 1. Acumular el tiempo real transcurrido
    tiempo_vuelo += delta
    
    # 2. Calcular la posicion usando la formula parametrica exacta:
    # p(t) = p0 + v0*t + 0.5 * a * t^2
    var desplazamiento_vel = velocidad_inicial * tiempo_vuelo
    var desplazamiento_acel = 0.5 * vector_gravedad * pow(tiempo_vuelo, 2)
    
    var nueva_posicion = posicion_inicial + desplazamiento_vel + desplazamiento_acel
    
    # 3. Aplicar al objeto
    if bala:
        bala.global_position = nueva_posicion
        
    # 4. Logica de reinicio (Loop)
    # Si la bala cae por debajo de la altura inicial y ha pasado algo de tiempo
    if nueva_posicion.y < posicion_inicial.y and tiempo_vuelo > 0.1:
        reiniciar_animacion()

func reiniciar_animacion():
    tiempo_vuelo = 0.0
    if bala:
        bala.global_position = posicion_inicial
        
    # Opcional: Imprimir duracion teorica vs real
    # T_teorico = 2 * Vy / g
    # var t_teorico = (2.0 * velocidad_inicial.y) / gravedad
    # print(''Ciclo completado. T esperado: '', t_teorico)
\end{lstlisting}

\subsubsection*{4. Análisis del Código}

\begin{itemize}
    \item \textbf{Cálculo Vectorial:} Se aprovecha la capacidad de Godot para operar con vectores completos (\texttt{Vector3}). La línea \texttt{velocidad\_inicial * tiempo\_vuelo} escala todas las componentes simultáneamente.
    \item \textbf{Gravedad:} Se aplica como un vector constante hacia abajo $(0, -9.8, 0)$. El término cuadrático ($t^2$) es lo que genera la forma parabólica característica: el movimiento horizontal es lineal (velocidad constante), mientras que el vertical se frena y luego acelera hacia abajo.
    \item \textbf{Reinicio:} La condición \texttt{nueva\_posicion.y < posicion\_inicial.y} detecta cuándo el proyectil ha completado el arco y cruza el plano del suelo, momento en el que se resetea el tiempo $t=0$ para repetir la animación en bucle.
\end{itemize}

\end{solucion}


% \part{Práctica}
% \chapter{Práctica 1. Escena básica y modos de visualización}

\section{Objetivos}
El propósito fundamental de esta práctica es iniciar al estudiante en el entorno de desarrollo integrado (IDE) que ofrece el motor de juegos Godot. Se busca una familiarización con su sistema de nodos y los principios elementales para la construcción de escenas tridimensionales simples. Al concluir esta sesión, el alumno deberá ser competente en la creación de una escena 3D que contenga geometría básica, aplicar materiales para definir el aspecto visual de los objetos, e implementar mecanismos de interacción básicos para el control de la cámara y los modos de visualización mediante entradas de teclado y ratón.

\section{Requisitos previos}
Para la correcta ejecución de esta práctica, es imperativo disponer de una instalación funcional de Godot Engine en su versión 4.0 o superior. Aunque no se requiere experiencia previa en el ámbito de los gráficos por computador, es fundamental poseer conocimientos básicos de programación orientada a objetos. Asimismo, se deberán descargar los ficheros de script proporcionados, con extensión \texttt{.gd}, que facilitarán la implementación de funcionalidades específicas como la visualización de ejes coordinados, la generación procedural de una pirámide y el control de una cámara orbital.

\section{Actividades}
\subsection{Crear un nuevo proyecto Godot}
El primer paso consiste en la configuración de un nuevo proyecto en Godot Engine. Este proceso se inicia desde el gestor de proyectos, donde se debe seleccionar la opción "Nuevo proyecto". Es necesario asignar un nombre único al proyecto y especificar un directorio en el sistema de ficheros donde se almacenarán todos sus recursos y configuraciones. Para esta práctica, se utilizará el renderizador por defecto, \textbf{Forward+}.

Una vez creado el proyecto, se abre el editor de Godot, que presenta una interfaz para el diseño de escenas, programación de scripts y gestión de recursos. Se procederá a crear una nueva escena 3D, cuyo nodo raíz será de tipo \texttt{Node3D}, renombrado a \texttt{EscenaPrincipal} para una mejor organización jerárquica. En Godot, una escena es una estructura de datos jerárquica (un árbol) compuesta por nodos que representan los distintos elementos de la aplicación.

\subsection{Crear un cubo en la escena}
La inserción de geometría básica es un procedimiento fundamental en el modelado 3D. Se añadirá un objeto tridimensional con forma de cubo.
\begin{enumerate}
    \item \textbf{Añadir un nodo de malla:} Como hijo del nodo raíz \texttt{EscenaPrincipal}, se debe instanciar un nodo de tipo \texttt{MeshInstance3D}. Este tipo de nodo se utiliza para visualizar una malla geométrica (\texttt{Mesh}) en una escena 3D, asignándole una transformación espacial (posición, rotación y escala).
    \item \textbf{Asignar la geometría:} En el panel \textit{Inspector}, se debe asignar un recurso de tipo \texttt{CubeMesh} a la propiedad \textit{Mesh} del nodo. \texttt{CubeMesh} es una clase derivada de \texttt{PrimitiveMesh}, que proporciona geometrías predefinidas por el motor.
    \item \textbf{Posicionar el objeto:} Se renombra el nodo a \texttt{Cubo} y se modifica su posición a las coordenadas (0, 0.5, 0) para elevarlo ligeramente sobre el plano base de la escena.
\end{enumerate}

\subsection{Añadir ejes de coordenadas}
Para una mejor comprensión espacial de la escena, es útil visualizar un sistema de ejes de coordenadas. Se utilizará un script proporcionado (\texttt{ejes3D.gd}) que genera proceduralmente la geometría de los ejes.
\begin{enumerate}
    \item Se crea un nuevo nodo de tipo \texttt{Node3D}, renombrado a \texttt{Ejes3D}.
    \item A este nodo se le asocia un nuevo script, cargando el fichero existente \texttt{ejes3D.gd}. Este script generará mallas que no se ven afectadas por la iluminación de la escena, sirviendo como una referencia visual constante.
\end{enumerate}

\subsection{Añadir cámara}
La visualización de una escena 3D requiere la presencia de una \textbf{cámara virtual}. Este elemento define la posición, orientación y ángulo de visión desde los cuales se renderiza la imagen final.
\begin{enumerate}
    \item Se instancia un nodo \texttt{Camera3D} como hijo del nodo raíz. Este tipo de nodo define un punto de vista para el renderizado.
    \item Se posiciona la cámara en \texttt{(1.5, 1.5, 2.0)} y se orienta para que apunte hacia el origen \texttt{(0, 0, 0)}. Esto se puede lograr mediante un script simple que, en su función \texttt{\_ready()}, establece la posición y utiliza el método \texttt{look\_at}.
\end{enumerate}

\begin{lstlisting}[language=GDScript, caption=Script para posicionamiento inicial de la cámara.]
extends Camera3D

func _ready():
    position = Vector3(1.5, 1.5, 2.0)
    look_at(Vector3(0.0, 0.0, 0.0), Vector3.UP)
\end{lstlisting}

\subsection{Asignar materiales}
El aspecto visual de un objeto se define mediante \textbf{materiales}. Un material describe cómo la superficie de un objeto interactúa con la luz, determinando propiedades como el color, el brillo o la textura.
\begin{enumerate}
    \item Con el nodo \texttt{Cubo} seleccionado, en el panel \textit{Inspector}, se crea un nuevo recurso \texttt{StandardMaterial3D} en la propiedad \textit{Surface Material Override}. \texttt{StandardMaterial3D} es una clase que implementa un modelo de materiales complejo con múltiples parámetros.
    \item Se modifica la propiedad \textbf{Albedo} del material, asignándole un color amarillo. El albedo representa el color base de la superficie.
\end{enumerate}
Al ejecutar la escena, se observará el cubo amarillo, pero su color será plano y sin sombreado, ya que aún no hay fuentes de luz que interactúen con el material.

\subsection{Añadir luz}
La iluminación es un componente crucial para el realismo en la visualización 3D. Las fuentes de luz emiten fotones que, al interactuar con los materiales de los objetos, producen el sombreado y los reflejos que percibimos.
\begin{enumerate}
    \item Se añade un nodo de tipo \texttt{DirectionalLight3D} a la escena. Este tipo de luz simula una fuente infinitamente lejana, como el sol, donde todos los rayos de luz son paralelos.
    \item Se posiciona y rota la luz para que ilumine las caras visibles del cubo con distinta intensidad, generando así un sombreado que revela su volumen tridimensional. Un modelo de iluminación simple como el de Lambert calcula la intensidad reflejada en función del coseno del ángulo entre la normal de la superficie y la dirección de la luz, lo que provoca que las caras no orientadas directamente hacia la luz aparezcan más oscuras.
\end{enumerate}
Tras añadir la luz, la ejecución mostrará el cubo con un sombreado que distingue sus diferentes caras.

\subsection{Crear una pirámide (generación por código)}
Godot permite la \textbf{generación procedural de geometría}, que consiste en crear mallas (\texttt{Mesh}) mediante código en tiempo de ejecución. Esto es útil para formas complejas o para geometrías que deben variar dinámicamente. Utilizaremos la clase \texttt{SurfaceTool} para construir la malla de una pirámide triángulo a triángulo.
\begin{enumerate}
    \item Se crea un nuevo nodo \texttt{Node3D} llamado \texttt{Piramide} y se le asocia el script \texttt{piramide.gd}.
    \item El script define la función \texttt{crear\_piramide}, que utiliza un objeto \texttt{SurfaceTool} para definir la geometría.
    \item Se inicializa el \texttt{SurfaceTool} para construir primitivas de tipo triángulo (\texttt{Mesh.PRIMITIVE\_TRIANGLES}).
    \item Se definen los vértices de la base y el ápice. Las caras laterales y la base se construyen añadiendo los vértices de cada triángulo con \texttt{add\_vertex}.
    \item Para cada triángulo, se calcula y asigna su vector normal con \texttt{set\_normal}. La normal es un vector unitario perpendicular al plano del triángulo, esencial para los cálculos de iluminación.
    \item Finalmente, \texttt{st.commit()} genera y devuelve un recurso de tipo \texttt{ArrayMesh} con la geometría definida. Este recurso se asigna a un nuevo nodo \texttt{MeshInstance3D} que se añade a la escena como hijo del nodo \texttt{Piramide}.
    \item La pirámide se posiciona sobre el cubo en (0, 1, 0).
\end{enumerate}

\begin{lstlisting}[language=GDScript, caption={Fragmento del script \texttt{piramide.gd} para la generación procedural.}]
func crear_piramide(h: float) -> ArrayMesh:
    var st = SurfaceTool.new()
    st.begin(Mesh.PRIMITIVE_TRIANGLES)
    
    # Coordenadas de la base (cuadrado centrado en el origen, lado 1)
    var p1 = Vector3(-0.5, 0, -0.5)
    var p2 = Vector3( 0.5, 0, -0.5)
    var p3 = Vector3( 0.5, 0,  0.5)
    var p4 = Vector3(-0.5, 0,  0.5)
    var apex = Vector3(0, h, 0)
    
    # Caras laterales (triangulos)
    _add_triangulo(st, p1, p2, apex)
    _add_triangulo(st, p2, p3, apex)
    _add_triangulo(st, p3, p4, apex)
    _add_triangulo(st, p4, p1, apex)
    
    # Base (dos triangulos)
    _add_triangulo(st, p1, p3, p2, Vector3.DOWN)
    _add_triangulo(st, p1, p4, p3, Vector3.DOWN)
    
    return st.commit()
\end{lstlisting}

\subsection{Cambiar el material (colores y texturas)}
De forma análoga a la geometría, los materiales también pueden ser creados y modificados mediante código. Se modificará el script de la pirámide para asignarle un material de color rojo.
\begin{enumerate}
    \item En la función \texttt{\_ready()} del script \texttt{piramide.gd}, se instancia un nuevo \texttt{StandardMaterial3D}.
    \item Se modifica su propiedad \texttt{albedo\_color} para asignarle un color rojizo, por ejemplo, \texttt{Color(1.0, 0.1, 0.2)}.
    \item El material creado se asigna a la propiedad \texttt{material\_override} de la instancia de malla de la pirámide. Esto sobreescribe cualquier material que el objeto pudiera tener asignado en el editor.
\end{enumerate}
Este método permite un control dinámico y procedural sobre la apariencia de los objetos, abriendo la puerta a efectos visuales complejos y a la modificación de texturas en tiempo de ejecución.

\subsection{Controlar una cámara orbital por teclado y ratón}
Finalmente, se reemplazará la cámara estática por una cámara orbital interactiva. Este tipo de cámara gira alrededor de un punto de interés (el origen, en este caso), permitiendo al usuario observar la escena desde múltiples ángulos.
\begin{enumerate}
    \item Se elimina el nodo \texttt{Camera3D} anterior.
    \item Se crea un nuevo nodo \texttt{Camera3D}, se renombra a \texttt{Camara3DOrbital}, y se le asocia el script \texttt{camara\_3d\_orbital\_simple.gd}.
    \item El script gestiona los eventos de entrada del usuario (\texttt{\_input}) para actualizar los ángulos de órbita (\texttt{dxy}) y la distancia al origen (\texttt{dz}). Los eventos de teclado (flechas, +/-) y de ratón (botón derecho arrastrado, rueda) son procesados para modificar estos parámetros.
    \item La función \texttt{\_actualiza\_transf\_vista} recalcula la transformación (\texttt{transform}) de la cámara en cada cambio. Utiliza una composición de transformaciones: una traslación a lo largo del eje Z local para establecer la distancia, seguida de rotaciones alrededor de los ejes X e Y para definir la orientación orbital.
\end{enumerate}

\chapter{Carga de modelos externos y normales}
\section{Objetivos}
Esta práctica profundiza en la representación de mallas poligonales y la gestión de modelos 3D en Godot. Los objetivos específicos son:
\begin{itemize}
    \item Comprender la estructura de las mallas triangulares.
    \item Aprender a cargar y visualizar modelos 3D de formatos estándar como \texttt{glb} y \texttt{obj}.
    \item Distinguir entre los diferentes modos de sombreado que ofrece Godot y entender su impacto visual y de rendimiento.
    \item Implementar algoritmos para el cálculo de normales en vértices, un requisito indispensable para una correcta iluminación en superficies suaves.
    \item Generar mallas complejas mediante técnicas de geometría procedural, como la revolución de un perfil 2D.
\end{itemize}

\section{Requisitos previos}
Se requiere haber completado la Práctica 1 y tener configurado un proyecto base que incluya un nodo raíz, una fuente de luz y la cámara orbital implementada previamente. Es necesario disponer de los scripts \texttt{script\_raiz.gd}, \texttt{utilidades.gd} y \texttt{donut.gd}.

\section{Actividades}
\subsection{Añadir modo de visualización en alambre (wireframe)}
La visualización en modo alambre, o \textit{wireframe}, es una técnica de renderizado que muestra únicamente las aristas de los polígonos que componen una malla, en lugar de sus caras rellenas. Este modo es invaluable para la depuración de algoritmos de generación de mallas y para el análisis de la topología de un modelo 3D.

En Godot, este modo se puede activar a nivel de \textit{viewport}. Se implementará una funcionalidad que permita alternar entre el modo de renderizado estándar y el modo \textit{wireframe} al pulsar la tecla 'W'.
\begin{enumerate}
    \item Se asocia el script \texttt{script\_raiz.gd} al nodo raíz de la escena.
    \item En la función \texttt{\_init}, se habilita la generación de mallas de alambre en el servidor de renderizado con \texttt{RenderingServer.set\_debug\_generate\_wireframes(true)}.
    \item La función \texttt{\_unhandled\_key\_input} intercepta la pulsación de la tecla 'W'.
    \item Al detectar el evento, se alterna el valor de una variable booleana \texttt{dibujar\_aristas}. Dependiendo de su estado, se establece la propiedad \texttt{debug\_draw} del \textit{viewport} actual a \texttt{Viewport.DEBUG\_DRAW\_WIREFRAME} o \texttt{Viewport.DEBUG\_DRAW\_DISABLED}.
\end{enumerate}
La Figura 16 del guion de prácticas ilustra la diferencia visual entre el renderizado normal y el modo \textit{wireframe} para un modelo de toroide (donut).

\subsection{Cargar modelos 3D en formato GLB}
Godot soporta la importación de diversos formatos de modelos 3D, entre los que se encuentra el formato \texttt{glb} (GL Transmission Format Binary). Este formato es eficiente ya que empaqueta en un único fichero binario una escena completa, incluyendo mallas, materiales, texturas y animaciones.
El procedimiento de importación es el siguiente:
\begin{enumerate}
    \item \textbf{Obtención del modelo:} Se descarga un modelo en formato \texttt{glb} desde un repositorio público como Sketchfab o Poly Pizza.
    \item \textbf{Importación al proyecto:} El fichero \texttt{.glb} se copia al directorio del proyecto. Godot lo detectará automáticamente y lo mostrará en el panel del sistema de archivos. Para una mejor organización, es recomendable crear una subcarpeta \texttt{modelos\_3D} para alojar los activos importados.
    \item \textbf{Instanciación en la escena:} El modelo se arrastra desde el panel del sistema de archivos al árbol de la escena, como hijo de un nodo \texttt{Node3D} de organización (p.ej., \texttt{ObjetosP2}). Godot creará una nueva jerarquía de nodos que representa la estructura interna del fichero \texttt{glb}. Opcionalmente, se puede convertir esta instancia en una escena editable para modificar sus componentes de forma individual.
\end{enumerate}
Es posible que sea necesario ajustar la escala del nodo importado para que su tamaño sea coherente con el resto de la escena.

\subsection{Cargar modelos 3D en formato OBJ}
El formato \texttt{obj} es otro estándar de facto para modelos 3D, especialmente popular por su simplicidad (es un formato de texto). A diferencia de \texttt{glb}, un fichero \texttt{.obj} solo contiene la geometría (vértices, normales, coordenadas de textura y definición de caras). La información de materiales se suele almacenar en un fichero \texttt{.mtl} asociado, y las texturas en ficheros de imagen separados.

El proceso de carga en Godot es ligeramente diferente:
\begin{enumerate}
    \item Se descarga el modelo, que consistirá en varios ficheros (\texttt{.obj}, \texttt{.mtl}, imágenes de textura) y se organizan en una subcarpeta dentro del proyecto.
    \item Se crea un nodo \texttt{MeshInstance3D} vacío en la escena.
    \item El fichero \texttt{.obj} se arrastra desde el panel del sistema de archivos y se suelta sobre la propiedad \textit{Mesh} del nodo \texttt{MeshInstance3D} en el \textit{Inspector}. Godot cargará automáticamente la geometría y tratará de asociar los materiales y texturas definidos en el fichero \texttt{.mtl}.
\end{enumerate}

\subsection{Cálculo de normales de objetos suaves y tipos de sombreado en Godot}
Las \textbf{normales de los vértices} son vectores unitarios perpendiculares a la superficie en la posición de cada vértice, y son un atributo fundamental para los algoritmos de iluminación. Para superficies suaves (curvas), una aproximación común y efectiva es calcular la normal de un vértice como el promedio normalizado de las normales de todas las caras adyacentes a dicho vértice. Este método, conocido como el promediado de normales de caras, asume que la malla poligonal es una aproximación de una superficie subyacente continua y diferenciable.

La función \texttt{calcNormales} proporcionada en el script \texttt{utilidades.gd} implementa este algoritmo: itera sobre todos los triángulos de la malla, calcula la normal de cada cara mediante el producto vectorial de dos de sus aristas, y acumula esta normal en los tres vértices que componen el triángulo. Finalmente, normaliza la normal acumulada en cada vértice.

Asimismo, Godot, como la mayoría de los motores de renderizado en tiempo real, distingue entre dos principales técnicas de sombreado (\textit{shading}):
\begin{itemize}
    \item \textbf{Sombreado por Píxel (Pixel Shading o Fragment Shading):} El cálculo de la iluminación se realiza para cada fragmento (píxel potencial) cubierto por un triángulo. Las normales de los vértices se interpolan de forma perspectiva-correcta para cada fragmento, y esta normal interpolada se utiliza en la ecuación de iluminación. Produce resultados de alta calidad, especialmente para reflejos especulares (brillos), pero es computacionalmente más intensivo.
    \item \textbf{Sombreado por Vértice (Vertex Shading o Gouraud Shading):} La ecuación de iluminación se calcula únicamente en cada vértice de la malla. El color resultante en cada vértice se interpola linealmente a través de la superficie del triángulo para determinar el color de cada píxel interior. Es más rápido pero puede producir artefactos visuales, como la pérdida de detalle en los reflejos especulares si la malla no es suficientemente densa.
\end{itemize}
En la práctica, se creará un toroide (donut) proceduralmente, se calcularán sus normales con el algoritmo mencionado y se comparará el resultado visual de ambos tipos de sombreado modificando la propiedad \texttt{shading\_mode} del material.

\subsection{Normales de objetos con aristas reales (no suaves)}
El algoritmo de promediado de normales asume una superficie suave. Para objetos con aristas duras o reales (no suaves), como un cubo, este método produce artefactos de iluminación incorrectos, ya que suaviza visualmente las aristas que deberían ser nítidas. En un cubo, cada vértice es compartido por tres caras mutuamente perpendiculares, por lo que no existe una única normal "suave" en esa posición.

La solución consiste en \textbf{duplicar los vértices} en las aristas duras. Para un cubo, en lugar de un modelo con 8 vértices compartidos, se utiliza un modelo con 24 vértices. Cada esquina del cubo real corresponde a tres vértices en la misma posición geométrica en el modelo, pero cada uno de estos vértices pertenece a una sola cara (o a caras coplanares) y tiene una normal distinta, perpendicular a dicha cara. De esta manera, al no compartir vértices entre caras no coplanares, no se produce el promediado de normales a través de las aristas, preservando su dureza visual en el renderizado. Esta técnica es fundamental en el modelado de polígonos duros (\textit{hard-surface modeling}) para asegurar una iluminación precisa.

\subsection{Creación de mallas por revolución de un perfil (geometría procedural)}
La \textbf{geometría por revolución} es una técnica de modelado procedural que genera una malla 3D al rotar un perfil 2D alrededor de un eje. El perfil es una secuencia de puntos en un plano (por ejemplo, el plano XY) que define una sección transversal del objeto.

El algoritmo a implementar tomará como entrada un perfil 2D (un array de \texttt{Vector2}) y el número de subdivisiones angulares (copias del perfil). Por cada punto del perfil, se generarán N vértices en un círculo alrededor del eje de revolución (eje Y). Estos vértices se conectarán para formar una malla de cuadriláteros (descompuestos en dos triángulos cada uno) que constituyen la superficie de revolución. El cálculo de las normales para esta malla generada puede abordarse de dos formas:
\begin{enumerate}
    \item Aplicar el algoritmo genérico de promediado de normales sobre la malla resultante.
    \item Un método más eficiente y preciso consiste en calcular la normal para cada vértice del perfil 2D original (en su plano) y luego rotar este vector de normal junto con el propio vértice alrededor del eje de revolución para obtener las normales de todos los vértices generados.
\end{enumerate}
Esta técnica permite crear eficientemente objetos con simetría axial como vasos, botellas, peones de ajedrez o tornos.

% \chapter{Resolución práctica 2}

En esta sección se abordará el código usado para cumplir con los requisitos de la práctica 2 del curso de Informática Gráfica. Dejando de lado las nociones básicas como importar figuras en diversos formatos vamos a ir tratando el código por bloques, explicando cada uno de ellos.

Vamos a ver el código del fichero que se proporciona de \textit{utilidades.gd}.

\begin{lstlisting}[language=GDScript, caption={Código de utilidades.gd}]

extends Node  # El script extiende la clase Node de Godot

## -----------------------------------------------------------------------------
## Función que calcula las normales promedio de los vértices de una malla,
## a partir de la tabla de posiciones de vértices y la tabla de triángulos

@export var normal_length: float = 0.6  # longitud de las líneas de las normales
@export var normal_color: Color = Color(0.967, 0.83, 0.917, 1.0)  # color de las normales

func calcNormales(verts: PackedVector3Array, tris: PackedInt32Array) -> PackedVector3Array:
    # Paso 1: comprobar precondiciones
    assert(verts.size() >= 3, "CalcNormales: la malla debe tener al menos 3 vértices")
    assert(tris.size() % 3 == 0, "CalcNormales: el número de enteros en 'tris' debe ser múltiplo de 3")

    var nv: int = verts.size()  # número de vértices
    var nt: int = tris.size() / 3  # número de triángulos

    # Paso 2: inicializa normales a cero
    var normales := PackedVector3Array([])
    for i in nv:
        normales.append(Vector3.ZERO)

    # Paso 3: sumar en cada vértice las normales de sus triángulos adyacentes
    for it in nt:
        var t := Vector3i(tris[3 * it + 0], tris[3 * it + 1], tris[3 * it + 2])
        var a := verts[t[0]]
        var b := verts[t[1]]
        var c := verts[t[2]]
        var normalv := (c - a).cross(b - a).normalized()  # calcula la normal del triángulo
        for iv in 3:
            normales[t[iv]] += normalv  # suma la normal al vértice correspondiente

    # Paso 4: normalizar normales
    for iv in nv:
        normales[iv] = normales[iv].normalized()

    # Hecho
    return normales

## -----------------------------------------------------------------------------
## Función de parametrización de un toroide (donut)
## u, v: parámetros entre 0 y 1; R: radio mayor; r: radio menor

func ParamDonut(u, v, r, R: float) -> Vector3:
    var cu := cos(2.0 * PI * u)
    var su := sin(2.0 * PI * u)
    var cv := cos(2.0 * PI * v)
    var sv := sin(2.0 * PI * v)
    return Vector3((R + r * cv) * cu, (R + r * cv) * su, r * sv)

## -----------------------------------------------------------------------------
## Genera una tabla de triángulos (índices) con topología toroidal
## nu: divisiones del primer parámetro; nv: divisiones del segundo parámetro

func GenTriToroidal(nu, nv: int, indices: PackedInt32Array):
    for i in nu:
        var isig = (i + 1) % nu
        for j in nv:
            var jsig = (j + 1) % nv
            var i00 = i * nv + j
            var i01 = i * nv + jsig
            var i10 = isig * nv + j
            var i11 = isig * nv + jsig

            indices.append(i00)
            indices.append(i11)
            indices.append(i10)
            indices.append(i00)
            indices.append(i01)
            indices.append(i11)

## -----------------------------------------------------------------------------
## Función que genera un toroide (donut) con 'nu x nv' vértices
## vertices: tabla de vértices; indices: tabla de índices

func generarDonut(vertices: PackedVector3Array, indices: PackedInt32Array,
                  nu: int = 128, nv: int = 32, R: float = 1.2, r: float = 0.4):
    # Genera vértices con la geometría de un donut
    for i in nu:
        for j in nv:
            vertices.append(ParamDonut(float(i) / nu, float(j) / nv, r, R))

    # Genera los triángulos con topología toroidal
    GenTriToroidal(nu, nv, indices)

## -----------------------------------------------------------------------------
## Función que crea y devuelve un nodo MeshInstance3D que dibuja las normales
## de una malla ya existente

func crear_visualizador_de_normales(malla_instancia: MeshInstance3D) -> MeshInstance3D:
    # Comprobar que el objeto y su malla son válidos
    if not is_instance_valid(malla_instancia) or not is_instance_valid(malla_instancia.mesh):
        return null

    var malla_original: Mesh = malla_instancia.mesh
    var transform_global: Transform3D = malla_instancia.global_transform

    # Usamos MeshDataTool para leer los datos de la malla
    var mdt = MeshDataTool.new()

    # Creamos la malla para dibujar las líneas
    var immediate_mesh = ImmediateMesh.new()
    var material = StandardMaterial3D.new()
    material.shading_mode = BaseMaterial3D.SHADING_MODE_UNSHADED  # sin sombreado
    material.albedo_color = normal_color  # color de las normales

    immediate_mesh.surface_begin(Mesh.PRIMITIVE_LINES, material)

    # Recorremos cada superficie de la malla
    for i in range(malla_original.get_surface_count()):
        mdt.clear()
        # Extraemos los datos de la superficie
        if mdt.create_from_surface(malla_original, i) == OK:
            # Recorremos cada vértice para dibujar su normal
            for v_idx in range(mdt.get_vertex_count()):
                var vertice = mdt.get_vertex(v_idx)
                var normal = mdt.get_vertex_normal(v_idx)
                immediate_mesh.surface_add_vertex(vertice)  # inicio de la línea
                immediate_mesh.surface_add_vertex(vertice + normal * normal_length)  # fin de la línea

    immediate_mesh.surface_end()

    # Creamos el nodo que contendrá las líneas
    var visualizador = MeshInstance3D.new()
    visualizador.mesh = immediate_mesh
    visualizador.name = "VisualizadorNormales_" + malla_instancia.name

    # Colocamos el visualizador en la misma posición y rotación que el objeto original
    visualizador.global_transform = transform_global

    return visualizador


\end{lstlisting}

Cabe destacar que el código que se va a ver en esta sección puede cambiar respecto del que se ofrece en prácticas, ya que se ha ido mejorando y optimizando.

\section{Problema de los cuadrados}

En esta sección se verán los códigos correspondientes al cuadrado de 8 vértices, el cual tiene un problema con las normales y luego se verá el cuadrado de 24 vértices, el cual soluciona el problema de las normales. Se debe mencionar que en el código de \texttt{utilidades.gd} se añade la función \texttt{crear\_visualizador\_de\_normales} que permite visualizar las normales de cualquier malla. Además, el script inicial de la raíz esta modificado para hacer que cuando se pulse la tecla \texttt{N} se muestren o se oculten las normales de todas las mallas que haya en la escena (al pulsar W se muestran las aristas, funcionalidad que se proporciona en clase). Veremos todos estos códigos.

\begin{lstlisting}[language=GDScript, caption={Script de la raíz para alternar visualización de aristas y normales}]
extends Node3D

# --- Variables de Estado ---
var dibujar_aristas: bool = false  # Indica si se debe mostrar el modo alambre (aristas)
var dibujar_normales_activado: bool = false  # Indica si se deben mostrar las normales
var nodos_visualizadores: Array = []  # Almacena los nodos que visualizan las normales

# --- Funciones de Godot ---

# Se ejecuta al crear el nodo. Activa la generación de wireframes en el motor.
func _init():
    RenderingServer.set_debug_generate_wireframes(true)

# Gestiona la entrada de teclado no procesada por la interfaz.
func _unhandled_key_input(event: InputEvent):
    # Solo actúa cuando la tecla se suelta para evitar repeticiones.
    if event is InputEventKey and not event.pressed:
        
        # --- Tecla 'W': Alterna el modo alambre ---
        if event.keycode == KEY_W:
            dibujar_aristas = not dibujar_aristas  # Cambia el estado del modo alambre
            if dibujar_aristas:
                dibujar_normales_activado = false  # Desactiva el modo normales si se activa el alambre
            _actualizar_modos_de_vista()
            
        # --- Tecla 'N': Alterna la visualización de normales ---
        elif event.keycode == KEY_N:
            dibujar_normales_activado = not dibujar_normales_activado  # Cambia el estado del modo normales
            if dibujar_normales_activado:
                dibujar_aristas = false  # Desactiva el modo alambre si se activan las normales
            _actualizar_modos_de_vista()

# --- Funciones de Ayuda ---

# Actualiza la vista según el estado de las variables de modo.
func _actualizar_modos_de_vista():
    var viewport = get_viewport()

    # 1. Elimina los visualizadores de normales antiguos.
    for nodo in nodos_visualizadores:
        if is_instance_valid(nodo):
            nodo.queue_free()
    nodos_visualizadores.clear()

    # 2. Activa el modo correspondiente.
    if dibujar_aristas:
        viewport.debug_draw = Viewport.DEBUG_DRAW_WIREFRAME  # Activa el modo alambre
        print("Dibujar en modo aristas: activado")
    elif dibujar_normales_activado:
        viewport.debug_draw = Viewport.DEBUG_DRAW_DISABLED  # Desactiva el modo alambre
        print("Mostrando normales...")
        _buscar_y_crear_visualizadores(get_tree().root)  # Busca y crea visualizadores de normales
    else:
        viewport.debug_draw = Viewport.DEBUG_DRAW_DISABLED  # Desactiva todos los modos de depuración
        print("Modos de depuración: desactivados")

# Busca recursivamente en la escena todos los nodos de malla visibles y les crea un visualizador de normales.
func _buscar_y_crear_visualizadores(nodo_actual: Node):
    # Si el nodo es una malla visible, crea su visualizador de normales.
    if nodo_actual is MeshInstance3D and nodo_actual.is_visible_in_tree():
        var visualizador = Utilidades.crear_visualizador_de_normales(nodo_actual)
        if is_instance_valid(visualizador):
            get_tree().root.add_child(visualizador)  # Añade el visualizador a la escena
            nodos_visualizadores.append(visualizador)  # Lo guarda para poder eliminarlo después

    # Llama recursivamente a todos los hijos del nodo actual.
    for hijo in nodo_actual.get_children():
        _buscar_y_crear_visualizadores(hijo)
\end{lstlisting}

El script anterior permite alternar entre la visualización de aristas (modo alambre) y la visualización de normales en todas las mallas de la escena mediante las teclas \texttt{W} y \texttt{N}. Los comentarios explican cada bloque y línea relevante del código, facilitando su comprensión y mantenimiento. Cabe destacar que las partes de añadir iluminación, como añadir un nodo y demás se dan por hecho que el lector ya las sabe hacer, en caso contrario son muy triviales de encontrar en la documentación oficial de Godot, o bien mediante intuición al usar la plataforma.

\begin{lstlisting}[language=GDScript, caption={Cubo de 8 vértices: definición, normales y material}]
extends MeshInstance3D

func _ready() -> void:
    
    # === 1. Definición de Vértices y Triángulos (Cubo de 8 Vértices, solo Y positivas) ===
    
    var vertices := PackedVector3Array([
        # Esquina 0: (-X, +Y0, -Z)
        Vector3(-0.5, 0.0, -0.5), # 0
        # Esquina 1: (+X, +Y0, -Z)
        Vector3( 0.5, 0.0, -0.5), # 1
        # Esquina 2: (+X, +Y0, +Z)
        Vector3( 0.5, 0.0,  0.5), # 2
        # Esquina 3: (-X, +Y0, +Z)
        Vector3(-0.5, 0.0,  0.5), # 3
        # Esquina 4: (-X, +Y1, -Z)
        Vector3(-0.5, 1.0, -0.5), # 4
        # Esquina 5: (+X, +Y1, -Z)
        Vector3( 0.5, 1.0, -0.5), # 5
        # Esquina 6: (+X, +Y1, +Z)
        Vector3( 0.5, 1.0,  0.5), # 6
        # Esquina 7: (-X, +Y1, +Z)
        Vector3(-0.5, 1.0,  0.5)  # 7
    ])
    
    # Cada línea define los índices de los vértices que forman los triángulos de cada cara del cubo
    var triangulos := PackedInt32Array([
        # Cara inferior (Y baja)
        0, 3, 2,  0, 2, 1,
        # Cara superior (Y alta)
        4, 5, 6,  4, 6, 7,
        # Cara frontal (Z-)
        0, 1, 5,  0, 5, 4,
        # Cara trasera (Z+)
        3, 7, 6,  3, 6, 2,
        # Cara lateral derecha (X+)
        1, 2, 6,  1, 6, 5,
        # Cara lateral izquierda (X-)
        0, 4, 7,  0, 7, 3
    ])
    
    # 2. Cálculo de Normales Suaves
    # Se calculan las normales promedio para cada vértice usando la función de utilidades
    var normales := Utilidades.calcNormales(vertices, triangulos)
            
    # 3. Creación y asignación de la Malla
    # Se prepara el array de datos de la malla (vértices, índices y normales)
    var tablas : Array = []
    tablas.resize(Mesh.ARRAY_MAX)
    tablas[Mesh.ARRAY_VERTEX] = vertices
    tablas[Mesh.ARRAY_INDEX] = triangulos
    tablas[Mesh.ARRAY_NORMAL] = normales
    
    # Se crea la malla y se añade la superficie con los datos anteriores
    mesh = ArrayMesh.new()
    mesh.add_surface_from_arrays(Mesh.PRIMITIVE_TRIANGLES, tablas)
    
    # 4. Material (Sombreado por píxel)
    # Se crea y configura el material para el cubo
    var mat := StandardMaterial3D.new()
    mat.albedo_color = Color(0.4, 0.4, 1.0)  # Color azul claro
    mat.metallic = 0.3                       # Un poco metálico
    mat.roughness = 0.2                      # Poco rugoso
    mat.shading_mode = BaseMaterial3D.SHADING_MODE_PER_VERTEX  # Sombreado por vértice
    
    # Se asigna el material a la malla
    material_override = mat
\end{lstlisting}


\begin{lstlisting}[language=GDScript, caption={Cubo de 24 vértices: definición, normales y material}]
extends MeshInstance3D

func _ready() -> void:
    # === 1. Definición de Vértices ===
    # Se definen 24 vértices, 4 para cada una de las 6 caras del cubo.
    # Esto permite que cada cara tenga su propia normal, logrando un sombreado plano y correcto.
    var vertices := PackedVector3Array([
        # Cara frontal (Z+)
        Vector3(-0.5,  0.5,  0.5), # 0 - Arriba-Izquierda
        Vector3( 0.5,  0.5,  0.5), # 1 - Arriba-Derecha
        Vector3( 0.5, -0.5,  0.5), # 2 - Abajo-Derecha
        Vector3(-0.5, -0.5,  0.5), # 3 - Abajo-Izquierda

        # Cara trasera (Z-)
        Vector3( 0.5,  0.5, -0.5), # 4 - Arriba-Derecha
        Vector3(-0.5,  0.5, -0.5), # 5 - Arriba-Izquierda
        Vector3(-0.5, -0.5, -0.5), # 6 - Abajo-Izquierda
        Vector3( 0.5, -0.5, -0.5), # 7 - Abajo-Derecha

        # Cara derecha (X+)
        Vector3( 0.5,  0.5,  0.5), # 8 - Arriba-Frontal
        Vector3( 0.5,  0.5, -0.5), # 9 - Arriba-Trasera
        Vector3( 0.5, -0.5, -0.5), # 10 - Abajo-Trasera
        Vector3( 0.5, -0.5,  0.5), # 11 - Abajo-Frontal

        # Cara izquierda (X-)
        Vector3(-0.5,  0.5, -0.5), # 12 - Arriba-Trasera
        Vector3(-0.5,  0.5,  0.5), # 13 - Arriba-Frontal
        Vector3(-0.5, -0.5,  0.5), # 14 - Abajo-Frontal
        Vector3(-0.5, -0.5, -0.5), # 15 - Abajo-Trasera

        # Cara superior (Y+)
        Vector3(-0.5,  0.5, -0.5), # 16 - Atrás-Izquierda
        Vector3( 0.5,  0.5, -0.5), # 17 - Atrás-Derecha
        Vector3( 0.5,  0.5,  0.5), # 18 - Adelante-Derecha
        Vector3(-0.5,  0.5,  0.5), # 19 - Adelante-Izquierda

        # Cara inferior (Y-)
        Vector3(-0.5, -0.5,  0.5), # 20 - Adelante-Izquierda
        Vector3( 0.5, -0.5,  0.5), # 21 - Adelante-Derecha
        Vector3( 0.5, -0.5, -0.5), # 22 - Atrás-Derecha
        Vector3(-0.5, -0.5, -0.5)  # 23 - Atrás-Izquierda
    ])

    # === 2. Definición de Triángulos ===
    # Se definen los triángulos para cada cara usando los vértices correspondientes.
    # El orden de los vértices es horario (Clockwise, CW) para que las normales se calculen correctamente.
    var triangulos := PackedInt32Array([
        # Cara frontal (Z+)
        0, 1, 2,  0, 2, 3,
        # Cara trasera (Z-)
        4, 5, 6,  4, 6, 7,
        # Cara derecha (X+)
        8, 9, 10,  8, 10, 11,
        # Cara izquierda (X-)
        12, 13, 14,  12, 14, 15,
        # Cara superior (Y+)
        16, 17, 18,  16, 18, 19,
        # Cara inferior (Y-)
        20, 21, 22,  20, 22, 23
    ])

    # === 3. Cálculo de Normales ===
    # Se calculan las normales para cada vértice usando la función de utilidades.
    # Al tener vértices duplicados por cara, cada normal será perpendicular a su cara.
    var normales := Utilidades.calcNormales(vertices, triangulos)

    # === 4. Creación de la Malla ===
    # Se prepara el array de datos de la malla (vértices, índices y normales).
    var arrays := []
    arrays.resize(Mesh.ARRAY_MAX)
    arrays[Mesh.ARRAY_VERTEX] = vertices
    arrays[Mesh.ARRAY_INDEX] = triangulos
    arrays[Mesh.ARRAY_NORMAL] = normales

    var new_mesh := ArrayMesh.new()
    new_mesh.add_surface_from_arrays(Mesh.PRIMITIVE_TRIANGLES, arrays)
    mesh = new_mesh

    # === 5. Asignación de Material ===
    # Se crea y configura el material para el cubo.
    var mat := StandardMaterial3D.new()
    mat.albedo_color = Color(0.4, 0.4, 1.0) # Color azul claro
    mat.metallic = 0.3                      # Un poco metálico
    mat.roughness = 0.2                     # Poco rugoso
    mat.shading_mode = BaseMaterial3D.SHADING_MODE_PER_PIXEL # Sombreado por píxel

    material_override = mat
\end{lstlisting}

Se \textbf{debe de evitar} el típico error de definir los triángulos en sentido antihorario (CounterClockwise, CCW), ya que las normales se calcularán al revés y la iluminación no será correcta.


\section{Creación de mallas por revolución de un perfil}

Acorde al guión de prácticas, en esta parte se debe de definir d enuevo un fichero global conc ciertas funciones, como puede ser el cálculo de malla por revolución, entre otros. 

\begin{lstlisting}[language=GDScript, caption={Función para generar mallas por revolución de un perfil}]
# Archivo: revolucion_utils.gd
extends Node

const PI = 3.14159265359 # Se usa la constante PI

## Genera una malla indexada por revolución alrededor del eje Y.
## El perfil se asume en el plano X-Y (Vector2(x, y) -> Vector3(x, y, 0)).
func generar_malla_revolucion(
    perfil: PackedVector2Array, 
    num_copias: int, 
    vertices: PackedVector3Array, 
    triangulos: PackedInt32Array
) -> void:

    var num_puntos_perfil = perfil.size()
    if num_puntos_perfil < 2 or num_copias < 3:
        # Se requiere al menos 2 puntos en el perfil y 3 copias para la revolución
        return
        
    var angulo_paso = 2.0 * PI / float(num_copias)
    
    # 1. Generación de Vértices
    for i in range(num_copias):
        var angulo = float(i) * angulo_paso
        var cos_a = cos(angulo)
        var sin_a = sin(angulo)
        
        for j in range(num_puntos_perfil):
            var p2 = perfil[j]
            var x = p2.x
            var y = p2.y
            
            # Rotación del perfil (x, y, 0) sobre el eje Y
            var x_rot = x * cos_a
            var z_rot = x * sin_a # Rotación en el plano XZ
            
            # Coordenada Y (altura) permanece igual
            var nuevo_vertice = Vector3(x_rot, y, z_rot)
            # Se añaden vértices al array de salida
            vertices.append(nuevo_vertice) 
            
    # 2. Generación de Triángulos (Índices)
    # Se crean cuadriláteros (quads) y cada uno se divide en dos triángulos.
    for i in range(num_copias):
        # El índice 'siguiente_i' conecta el último segmento con el primero (cierre completo)
        var siguiente_i = (i + 1) % num_copias 
        
        for j in range(num_puntos_perfil - 1):
            # Índices en la capa actual (i) y la siguiente (i+1)
            var i0 = i * num_puntos_perfil + j           # Vértice A (i, j)
            var i1 = siguiente_i * num_puntos_perfil + j  # Vértice B (i+1, j)
            var i2 = siguiente_i * num_puntos_perfil + j + 1 # Vértice C (i+1, j+1)
            var i3 = i * num_puntos_perfil + j + 1           # Vértice D (i, j+1)
            
            # Triángulo 1: (A, B, D) 
            triangulos.append(i0)
            triangulos.append(i1)
            triangulos.append(i3)
            
            # Triángulo 2: (B, C, D)
            triangulos.append(i1)
            triangulos.append(i2)
            triangulos.append(i3)
\end{lstlisting}

Como ejemplos vamos a ver la creación de una esfera y la de una pieza de peón del ajedrez.

\begin{lstlisting}[language=GDScript, caption={Generación de una esfera por revolución de perfil}]
# Archivo: malla_revolucion.gd
extends MeshInstance3D

# Parámetros exportados para configurar el modelo desde el editor
@export var num_copias: int = 64                # Número de copias del perfil (divisiones horizontales)
@export var radio: float = 1.0                  # Radio de la esfera
@export var sombreado_por_pixel: bool = true    # Permite alternar el modo de sombreado

## Función que genera el perfil 2D de media circunferencia para la esfera
func generar_perfil_esfera(segmentos_verticales: int) -> PackedVector2Array:
    var perfil = PackedVector2Array()
    var R = radio
    # Recorre los segmentos verticales para crear puntos desde abajo (PI) hasta arriba (0)
    for i in range(segmentos_verticales + 1):
        var angulo = PI * float(i) / float(segmentos_verticales)
        # Calcula la posición X e Y del punto en el perfil usando funciones trigonométricas
        var x_coord = R * sin(angulo)   # Componente X del perfil
        var y_coord = R * cos(angulo)   # Componente Y del perfil
        perfil.append(Vector2(x_coord, y_coord)) # Añade el punto al perfil
    return perfil

func _ready() -> void:
    # Define el número de segmentos verticales para la esfera
    var segmentos_verticales = 32
    # Genera el perfil de media circunferencia
    var perfil_actual = generar_perfil_esfera(segmentos_verticales)
    
    # Inicializa los arrays de salida para vértices y triángulos
    var vertices   := PackedVector3Array([])
    var triangulos := PackedInt32Array([])
    
    # Genera la malla por revolución usando el perfil y el número de copias
    RevolucionUtils.generar_malla_revolucion(perfil_actual, num_copias, vertices, triangulos)
    
    # Calcula las normales promedio para cada vértice
    var normales := Utilidades.calcNormales(vertices, triangulos)
    
    # Prepara el array de datos de la malla (estructura SOA)
    var tablas : Array = []
    tablas.resize(Mesh.ARRAY_MAX)
    tablas[Mesh.ARRAY_VERTEX] = vertices      # Posiciones de los vértices
    tablas[Mesh.ARRAY_INDEX]  = triangulos    # Índices de los triángulos
    tablas[Mesh.ARRAY_NORMAL] = normales      # Normales de los vértices
    
    # Crea la malla y añade la superficie con los datos anteriores
    mesh = ArrayMesh.new()
    mesh.add_surface_from_arrays(Mesh.PRIMITIVE_TRIANGLES, tablas)
    
    # Crea y configura el material para la esfera
    var mat := StandardMaterial3D.new()
    mat.albedo_color = Color(0.2, 0.5, 1.0)   # Color azul claro
    mat.metallic = 0.3                        # Un poco metálico
    mat.roughness = 0.2                       # Poco rugoso
    
    # Configura el modo de sombreado según el parámetro exportado
    if sombreado_por_pixel:
        mat.shading_mode = BaseMaterial3D.SHADING_MODE_PER_PIXEL   # Sombreado por píxel
    else:
        mat.shading_mode = BaseMaterial3D.SHADING_MODE_PER_VERTEX # Sombreado por vértice
    
    material_override = mat
\end{lstlisting}

\begin{lstlisting}[language=GDScript, caption={Generación de un peón de ajedrez por revolución de perfil}]
# Archivo: peon_ajedrez_revolucion.gd
extends MeshInstance3D

@export var num_copias: int = 64                       # Número de copias del perfil (divisiones horizontales)
@export var segmentos_verticales_por_tramo: int = 16    # Segmentos por tramo entre puntos clave del perfil
@export var sombreado_por_pixel: bool = true            # Permite alternar el modo de sombreado

# Constante que define los puntos clave del perfil 2D del peón
const PUNTOS_PEON_CLAVE: PackedVector2Array = [
    Vector2(0.0, -1.0), Vector2(0.5, -1.0), Vector2(0.55, -0.8),
    Vector2(0.2, -0.4), Vector2(0.3, 0.1), Vector2(0.1, 0.5),
    Vector2(0.35, 0.8), Vector2(0.2, 0.95), Vector2(0.0, 1.0)
]

# Genera el perfil suavizado interpolando linealmente entre los puntos clave
func generar_perfil_peon(segmentos: int) -> PackedVector2Array:
    var perfil = PackedVector2Array()
    # Para cada tramo entre dos puntos clave, interpola 'segmentos' puntos
    for i in range(PUNTOS_PEON_CLAVE.size() - 1):
        var p_inicio = PUNTOS_PEON_CLAVE[i]
        var p_fin = PUNTOS_PEON_CLAVE[i + 1]
        for j in range(segmentos):
            perfil.append(p_inicio.lerp(p_fin, float(j) / float(segmentos)))
    perfil.append(PUNTOS_PEON_CLAVE[-1]) # Añade el último punto clave
    return perfil

func _ready() -> void:
    # Genera el perfil 2D del peón usando la función de interpolación
    var perfil_actual = generar_perfil_peon(segmentos_verticales_por_tramo)
    
    # Inicializa los arrays de salida para vértices y triángulos
    var vertices := PackedVector3Array([])
    var triangulos := PackedInt32Array([])
    
    # Genera la malla por revolución usando el perfil y el número de copias
    RevolucionUtils.generar_malla_revolucion(perfil_actual, num_copias, vertices, triangulos)
    
    # Calcula las normales promedio para cada vértice
    var normales := Utilidades.calcNormales(vertices, triangulos)
    
    # Prepara el array de datos de la malla (estructura SOA)
    var tablas: Array = []
    tablas.resize(Mesh.ARRAY_MAX)
    tablas[Mesh.ARRAY_VERTEX] = vertices      # Posiciones de los vértices
    tablas[Mesh.ARRAY_INDEX] = triangulos     # Índices de los triángulos
    tablas[Mesh.ARRAY_NORMAL] = normales      # Normales de los vértices
    
    # Crea la malla y añade la superficie con los datos anteriores
    mesh = ArrayMesh.new()
    mesh.add_surface_from_arrays(Mesh.PRIMITIVE_TRIANGLES, tablas)
    
    # Crea y configura el material para el peón
    var mat := StandardMaterial3D.new()
    mat.albedo_color = Color(1.0, 1.0, 0.8)   # Color marfil claro
    mat.metallic = 0.5                        # Más metálico
    mat.roughness = 0.4                       # Más rugoso
    
    # Configura el modo de sombreado según el parámetro exportado
    if sombreado_por_pixel:
        mat.shading_mode = BaseMaterial3D.SHADING_MODE_PER_PIXEL   # Sombreado por píxel
    else:
        mat.shading_mode = BaseMaterial3D.SHADING_MODE_PER_VERTEX # Sombreado por vértice
    
    material_override = mat
\end{lstlisting}
% \chapter{Resolución práctica 3}
\section{Introducción a la Práctica de Grafos de Escena}

\subsection{Objetivos de la Práctica 3}

La Práctica 3, titulada "Grafos de escena", tiene como propósito fundamental la aplicación de los conceptos de modelado jerárquico en el entorno de Godot Engine.

Los objetivos específicos que deben ser cubiertos por los estudiantes son:
\begin{enumerate}
    \item Aprender a diseñar e implementar \textbf{modelos jerárquicos de objetos articulados}.
    \item Aprender a \textbf{crear el grafo de escena} que formaliza la jerarquía.
    \item Comprender el funcionamiento de la \textbf{pila de transformaciones} (o composición de transformaciones).
    \item Aprender a \textbf{modificar interactivamente parámetros} del modelo.
    \item Aprender a \textbf{implementar animaciones sencillas}.
\end{enumerate}

\subsection{Marco Teórico: Modelos Jerárquicos y Transformaciones}

\subsubsection{Concepto de Grafo de Escena}

En Godot y en los motores gráficos en general, el \textbf{grafo de escena} (o jerarquía) es una estructura de datos, generalmente un árbol o un grafo dirigido acíclico, que modela las \textbf{relaciones jerárquicas} entre los objetos que componen la aplicación, tales como cámaras, luces y objetos geométricos.

\begin{definicion}[Modelado Jerárquico]
Un objeto articulado o compuesto se modela mediante un nodo no terminal en el grafo, que contiene instancias de otros nodos (hijos). Los objetos simples (mallas) actúan como nodos terminales.
\end{definicion}

El desarrollo de aplicaciones en Godot se organiza en torno a \textbf{proyectos, escenas y nodos}. Una escena es una estructura jerárquica de nodos (un árbol de escena) que representan los elementos de la aplicación.

\subsubsection{Transformaciones y Composiciones}

Cada nodo en Godot (\codeinline{Node2D} o \codeinline{Node3D}) tiene asociado un \textbf{marco afín} local, $\mathcal{N}$. Este marco $\mathcal{N}$ se define en relación con el marco de su nodo padre, $\mathcal{P}$, mediante una matriz de transformación $M_N$, llamada \textbf{transformación del nodo}. La relación se expresa como $\mathcal{P} M_N = \mathcal{N}$.

\underline{La Pila de Transformaciones}

Las coordenadas de los vértices de una malla se consideran expresadas en el marco $\mathcal{N}$ del nodo (coordenadas locales). Al visualizar un objeto, la transformación final que se aplica es la composición de las transformaciones de todos los nodos en el camino desde la raíz hasta el nodo terminal.

Para un nodo $N$, la matriz de transformación $M_N$ es la composición de varias transformaciones, y el orden es crucial. En el contexto de Godot 3D (\codeinline{Node3D}), la transformación se compone generalmente (de derecha a izquierda) como una secuencia de \textbf{Escalado, Rotación y Traslación}.

\begin{proposicion}[Transformación de Nodo 3D]
La transformación de un nodo $N$ se construye mediante la composición de transformaciones geométricas afines:
$$M_N = T \circ R \circ S$$
Donde $T$ es la Traslación (definida por \codeinline{position}), $R$ es la Rotación (definida por \codeinline{rotation} o cuaternión), y $S$ es el Escalado (definido por \codeinline{scale}).
\end{proposicion}

Es importante notar la distinción entre composición por la izquierda (actúa en el marco del padre, $M_{nuevo} = T_{padre} \cdot M_{viejo}$) y composición por la derecha (actúa en el marco local del objeto, $M_{nuevo} = M_{viejo} \cdot T_{local}$). Los métodos como \codeinline{rotate\_object\_local} componen por la derecha, interpretando los vectores de entrada en el marco local del nodo.

\subsection{Requisitos Previos y Configuración del Proyecto}

\subsubsection{Requisitos y Estructura Base}

Para comenzar la Práctica 3, es indispensable haber \textbf{completado la Práctica 2}.

El proyecto debe mantener la estructura base establecida en las prácticas anteriores, incluyendo:
\begin{itemize}
    \item Un nodo raíz de la escena principal.
    \item Un nodo para la \textbf{cámara orbital} (\codeinline{Camara3DOrbital}, con el script \codeinline{camara\_3d\_orbital\_simple.gd}).
    \item Un nodo para una \textbf{fuente de luz} (como \codeinline{DirectionalLight3D}).
    \item Un nodo para el objeto que contiene los \textbf{ejes visibles} (\codeinline{Ejes3D}).
    \item Un nodo padre de todos los objetos articulados, sugerido como \codeinline{ObjetosP3} o \codeinline{GrafoP3}.
\end{itemize}

\subsubsection{Organización del Proyecto}

Se recomienda encarecidamente una organización de archivos clara para facilitar la gestión de recursos:
\begin{itemize}
    \item Utilizar una carpeta \codeinline{modelos\_3D} para los modelos externos (\codeinline{glb}, \codeinline{obj}).
    \item Utilizar una carpeta \codeinline{scripts}. Dentro de esta, separe los scripts:
    \begin{itemize}
        \item \codeinline{scripts/nodos}: para scripts asociados a nodos específicos del grafo.
        \item \codeinline{scripts/globales}: para scripts globales (autoloads), como \codeinline{utilidades.gd}.
    \end{itemize}
\end{itemize}
Debe recordarse que los nodos y recursos deben tener \textbf{nombres descriptivos} (ej., \codeinline{PelotaTenis} en lugar de \codeinline{MeshInstance5}).

\section{Desarrollo Detallado de las Actividades}

\subsection{Actividad 1: Diseño del Grafo de Escena}

El primer paso es el \textbf{diseño formal} del modelo jerárquico.

\subsubsection{Definición del Modelo Articulado}

El modelo debe ser un objeto articulado con \textbf{al menos tres articulaciones} (grados de libertad), que se controlarán mediante giros y desplazamientos. Al menos dos de estas articulaciones deben ser \textbf{dependientes} entre sí.

\begin{ejercicio}[Diseño del Grafo de Escena Articulado]
Diseñe un objeto jerárquico. Un ejemplo sugerido es una grúa que posea tres grados de libertad (DOF):
\begin{enumerate}
    \item Ángulo de giro del brazo principal (rotación en Y).
    \item Desplazamiento del gancho a lo largo del brazo.
    \item Altura del gancho (desplazamiento vertical o extensión de un segmento).
\end{enumerate}
Este diseño se debe plasmar en un documento PDF que detalle el grafo, las transformaciones involucradas, los parámetros (grados de libertad), y las referencias a los objetos terminales (mallas). Los nodos terminales pueden provenir de:
\begin{itemize}
    \item Objetos predefinidos de Godot (\codeinline{CubeMesh}).
    \item Objetos creados en Práctica 1 (ej., la pirámide).
    \item Modelos importados (\codeinline{glb}, \codeinline{obj}) de Práctica 2.
    \item Objetos de revolución generados proceduralmente (Práctica 2).
\end{itemize}

\end{ejercicio}

\subsection{Actividad 2: Creación del Modelo en Godot}

Una vez diseñado el grafo, se procede a su implementación en Godot. Esto implica la creación de nodos (\codeinline{Node3D}) para la estructura jerárquica y nodos \codeinline{MeshInstance3D} para los elementos geométricos.

\subsubsection{Implementación de la Jerarquía}

La estructura jerárquica debe reflejar la dependencia de las transformaciones. Por ejemplo, si el *Brazo* rota, todos sus hijos (como el *Gancho*) deben rotar con él.

\begin{proposicion}[Reutilización de Mallas]
Para optimizar la memoria y el rendimiento, es crucial no duplicar mallas complejas. Los nodos \codeinline{MeshInstance3D} deben contener únicamente una referencia a la malla (\codeinline{mesh}), permitiendo que múltiples instancias compartan el mismo objeto \codeinline{ArrayMesh} o \codeinline{CubeMesh}.
\end{proposicion}

La implementación se puede realizar en el editor o mediante scripts en la función \codeinline{_ready()} del nodo raíz (\codeinline{EscenaPrincipal} o \codeinline{ObjetosP3}), creando los nodos descendientes programáticamente.

\subsection{Actividad 3: Generación de una Animación del Modelo}

La animación se logra \textbf{modificando los atributos de transformación} de los nodos que controlan los grados de libertad en cada *frame* de ejecución.

\subsubsection{Implementación de la Animación en GDScript}

La lógica de actualización continua debe residir en el método \codeinline{\_process(delta)} del script asociado al nodo que se desea animar. El parámetro \codeinline{delta} representa el tiempo transcurrido (en segundos) desde el último frame.

Para una \textbf{rotación continua} (como la del brazo de la grúa alrededor del eje Y) se puede implementar de la siguiente forma:
\begin{lstlisting}[language=GDScript, caption={Ejemplo de Animación de Rotación 3D (Brazo)}]
extends Node3D

@export var rotation_speed_deg := 10.0 # grados por segundo

func _process(delta):
    # Rotación continua en Y
    rotation.y += deg_to_rad(rotation_speed_deg * delta)
\end{lstlisting}

Para implementar animaciones sencillas, se pueden utilizar variaciones \textbf{lineales} o \textbf{oscilantes}. Un movimiento oscilante (útil para el desplazamiento del gancho o la altura) se puede lograr utilizando la función trigonométrica seno ($\sin$):

$$v = a + (b - a) \cdot \frac{1 + \sin(2\pi \cdot w \cdot t)}{2}$$

Donde $a$ y $b$ son los valores mínimo y máximo, $w$ es la frecuencia de oscilación, y $t$ es el tiempo acumulado. La actualización del estado debe hacerse en \codeinline{_process(delta)}.

\subsection{Actividad 4: Activación y Desactivación de la Animación}

Para permitir la interacción, se debe implementar la capacidad de activar o desactivar la animación de cada articulación mediante la entrada del usuario.

\subsubsection{Gestión de Eventos y Acciones de Entrada}

Se recomienda usar el sistema \codeinline{Input Map} de Godot para definir acciones, como \codeinline{activar\_cabeza} o \codeinline{activar\_brazo}, y asociarlas a teclas, como las numéricas 1 a 9.

Dentro del método \codeinline{\_process(delta)}, se verifica si se ha pulsado la acción de entrada deseada mediante \codeinline{Input.is\_action\_just\_pressed(accion)}:

\begin{lstlisting}[language=GDScript, caption={Ejemplo de Activación/Desactivación de Animación}]
extends Node3D

@export var activar := "activar\_brazo" \# Acción definida en Input Map
@export var rotation\_speed\_deg := 10.0
var activa := true 

func \_process(delta):
    \# 1. Manejo de la entrada para alternar el estado
    if Input.is\_action\_just\_pressed(activar):
            activa = !activa
    
    \# 2. Aplicar la transformación solo si está activa
    if activa:
            rotation.y += deg\_to\_rad(rotation\_speed\_deg * delta)
\end{lstlisting}

Este mecanismo de control basado en el estado (la variable \codeinline{activa}) permite la \textbf{modificación interactiva de parámetros} del modelo, cumpliendo con uno de los objetivos clave de la práctica.

\section{Entrega de la Práctica}

La entrega final debe consistir en dos elementos:

\begin{enumerate}
    \item La \textbf{carpeta del proyecto Godot} completa, comprimida en formato ZIP. El proyecto debe estar organizado según las recomendaciones de la Práctica 2 (incluyendo subcarpetas para scripts y modelos).
    \item Un \textbf{documento breve} (adicional a esta guía) que contenga:
    \begin{itemize}
        \item El \textbf{grafo de escena} diseñado en la Actividad 1.
        \item Una explicación de los \textbf{scripts de interacción} implementados para controlar los grados de libertad y la animación.
    \end{itemize}
\end{enumerate}

El diseño del grafo de escena debe ser claro, conciso y profesional, explicando cómo se han implementado las dependencias jerárquicas y cómo las transformaciones (traslación, rotación, escalado) se componen para lograr el movimiento articulado deseado.
% % Ejercicios adicionales
% \chapter{Ejercicios adicionales}    

\section{Práctica 1}

\begin{ejercicio}
    Pirámide con base en ''L''.
\end{ejercicio}

\begin{solucion}
Para resolver este ejercicio, se debe construir una malla 3D indexada que combine una base en forma de ''L'' con una pirámide de 6 caras que converge en un único vértice (ápice).

La malla debe estar compuesta por un total de 7 vértices: 6 para la base en ''L'' y 1 para el ápice. Los vértices de la base se sitúan en el plano $Y=0$, mientras que el ápice se coloca sobre la esquina interior de la ''L'', a una altura determinada.

La base se triangula utilizando el número mínimo de triángulos, en este caso 4, dividiendo la ''L'' en dos rectángulos y cada uno en dos triángulos. Los triángulos de la base se definen mediante los siguientes índices de vértices:
\begin{itemize}
    \item $0, 1, 3$
    \item $1, 2, 3$
    \item $0, 3, 5$
    \item $3, 4, 5$
\end{itemize}

Las caras laterales de la pirámide se forman conectando el ápice con cada uno de los lados de la base, resultando en 6 triángulos adicionales:
\begin{itemize}
    \item $6, 0, 1$
    \item $6, 1, 2$
    \item $6, 2, 3$
    \item $6, 3, 4$
    \item $6, 4, 5$
    \item $6, 5, 0$
\end{itemize}

En Godot, se recomienda crear un nodo \texttt{MeshInstance3D} y asignarle un script donde se definan los vértices y los índices de los triángulos. El siguiente ejemplo ilustra cómo implementar la malla en GDScript:

\begin{lstlisting}[language=GDScript]
extends MeshInstance3D

func _ready():
    var vertices = PoolVector3Array([
        Vector3(0, 0, 0), # v0
        Vector3(2, 0, 0), # v1
        Vector3(2, 0, 1), # v2
        Vector3(1, 0, 1), # v3
        Vector3(1, 0, 2), # v4
        Vector3(0, 0, 2), # v5
        Vector3(1, 2, 1)  # v6 (ápice)
    ])

    var indices = PoolIntArray([
        0, 1, 3,
        1, 2, 3,
        0, 3, 5,
        3, 4, 5,
        6, 0, 1,
        6, 1, 2,
        6, 2, 3,
        6, 3, 4,
        6, 4, 5,
        6, 5, 0
    ])

    var arrays = []
    arrays.resize(ArrayMesh.ARRAY_MAX)
    arrays[ArrayMesh.ARRAY_VERTEX] = vertices
    arrays[ArrayMesh.ARRAY_INDEX] = indices

    var mesh = ArrayMesh.new()
    mesh.add_surface_from_arrays(Mesh.PRIMITIVE_TRIANGLES, arrays)
    self.mesh = mesh
\end{lstlisting}

Al ejecutar la escena, se visualizará la pirámide con base en ''L'' y 6 caras laterales, cumpliendo con los requisitos del ejercicio.
\end{solucion}

\begin{ejercicio}
    Polígono regular en el plano $XY$.
\end{ejercicio}

\begin{solucion}
Este ejercicio consiste en construir una malla 2D en el plano $XY$ con forma de polígono regular de $n$ lados. El procedimiento es:

\begin{enumerate}
    \item \textbf{Configuración del nodo:} Crear un nodo \texttt{MeshInstance3D} y asignarle un script, por ejemplo \texttt{PoligonoRegular.gd}.
    \item \textbf{Definición de parámetros:} El número de lados $n$ es editable desde el inspector.
    \item \textbf{Generación de vértices:} Se define un vértice central y $n$ vértices exteriores, distribuidos uniformemente en círculo alrededor del centro.
    \item \textbf{Triangulación:} Se generan $n$ triángulos, cada uno formado por el centro y dos vértices exteriores consecutivos.
    \item \textbf{Material:} Se asigna un material sin sombreado, usando los colores de los vértices.
\end{enumerate}

El siguiente código en GDScript implementa la malla:

\begin{lstlisting}[language=GDScript]
extends MeshInstance3D

@export var n: int = 8

func _ready():
    if n < 3:
        n = 3
        print("El polígono debe tener al menos 3 lados. Usando n=3.")
    var new_mesh = poligono_regular(n)
    self.mesh = new_mesh
    var material = StandardMaterial3D.new()
    material.vertex_color_use_as_albedo = true
    material.shading_mode = StandardMaterial3D.SHADING_MODE_UNSHADED
    self.material_override = material

func poligono_regular(p_n: int) -> ArrayMesh:
    var vertices = PackedVector3Array()
    var colors = PackedColorArray()
    var indices = PackedInt32Array()
    var centro = Vector3(0.5, 0.5, 0)
    vertices.push_back(centro)
    colors.push_back(Color.WHITE)
    var radio = 0.5
    for i in range(p_n):
        var angulo = (float(i) / float(p_n)) * TAU
        var x = centro.x + radio * cos(angulo)
        var y = centro.y + radio * sin(angulo)
        var z = 0.0
        vertices.push_back(Vector3(x, y, z))
        colors.push_back(Color(x, y, z))
    for i in range(p_n):
        var idx_centro = 0
        var idx_v1 = i + 1
        var idx_v2 = (i + 1) % p_n + 1
        indices.push_back(idx_centro)
        indices.push_back(idx_v1)
        indices.push_back(idx_v2)
    var arrays = []
    arrays.resize(ArrayMesh.ARRAY_MAX)
    arrays[ArrayMesh.ARRAY_VERTEX] = vertices
    arrays[ArrayMesh.ARRAY_COLOR] = colors
    arrays[ArrayMesh.ARRAY_INDEX] = indices
    var mesh = ArrayMesh.new()
    mesh.add_surface_from_arrays(Mesh.PRIMITIVE_TRIANGLES, arrays)
    return mesh
\end{lstlisting}

\textbf{Explicación:} El vértice central se sitúa en $(0.5, 0.5, 0)$. Los vértices exteriores se distribuyen uniformemente en círculo. Cada triángulo conecta el centro con dos vértices consecutivos, formando así el polígono regular. El material se configura para mostrar los colores de los vértices y sin sombreado.
\end{solucion}

\begin{ejercicio}
    Estrella plana en el plano $Z=0$ alternando radios.
\end{ejercicio}

\begin{solucion}
Este ejercicio consiste en construir una malla 2D en el plano $Z=0$ con forma de estrella de $n$ puntas, alternando dos radios distintos para generar las puntas y los valles de la estrella.

El procedimiento es el siguiente:

\begin{enumerate}
    \item \textbf{Configuración del nodo:} Se crea un nodo \texttt{MeshInstance3D} y se le asigna un script, por ejemplo \texttt{EstrellaZ.gd}.
    \item \textbf{Definición de parámetros:} El número de puntas $n$ es editable. Para una estrella clásica, $n=5$.
    \item \textbf{Generación de vértices:} Se define un vértice central y $2n$ vértices exteriores, alternando entre el radio de las puntas y el de los valles. El vértice central se sitúa en $(0.5, 0.5, 0)$ y los exteriores se calculan usando ángulos equiespaciados en la circunferencia.
    \item \textbf{Triangulación:} Se generan $2n$ triángulos, cada uno formado por el centro y dos vértices exteriores consecutivos.
    \item \textbf{Material:} Se asigna un material sin sombreado y con doble cara, usando los colores de los vértices.
\end{enumerate}

A continuación se muestra el código en GDScript que implementa la malla:

\begin{lstlisting}[language=GDScript]
extends MeshInstance3D

@export var n: int = 5

func _ready():
    if n < 2:
        n = 2
        print("La estrella debe tener al menos 2 puntas. Usando n=2.")
    var new_mesh = ArrayMeshEstrellaZ(n)
    self.mesh = new_mesh
    var material = StandardMaterial3D.new()
    material.vertex_color_use_as_albedo = true
    material.shading_mode = StandardMaterial3D.SHADING_MODE_UNSHADED
    material.cull_mode = StandardMaterial3D.CULL_DISABLED
    self.material_override = material

func ArrayMeshEstrellaZ(p_n: int) -> ArrayMesh:
    var vertices = PackedVector3Array()
    var colors = PackedColorArray()
    var indices = PackedInt32Array()
    var centro = Vector3(0.5, 0.5, 0)
    vertices.push_back(centro)
    colors.push_back(Color.WHITE)
    var radio_punta = 0.5
    var radio_valle = 0.25
    var num_vertices_externos = 2 * p_n
    for i in range(num_vertices_externos):
        var radio_actual = radio_punta if i % 2 == 0 else radio_valle
        var angulo = (float(i) / float(num_vertices_externos)) * TAU
        var x = centro.x + radio_actual * cos(angulo)
        var y = centro.y + radio_actual * sin(angulo)
        var z = 0.0
        vertices.push_back(Vector3(x, y, z))
        colors.push_back(Color(x, y, z))
    for i in range(num_vertices_externos):
        var idx_centro = 0
        var idx_v1 = i + 1
        var idx_v2 = (i + 1) % num_vertices_externos + 1
        indices.push_back(idx_centro)
        indices.push_back(idx_v1)
        indices.push_back(idx_v2)
    var arrays = []
    arrays.resize(ArrayMesh.ARRAY_MAX)
    arrays[ArrayMesh.ARRAY_VERTEX] = vertices
    arrays[ArrayMesh.ARRAY_COLOR] = colors
    arrays[ArrayMesh.ARRAY_INDEX] = indices
    var mesh = ArrayMesh.new()
    mesh.add_surface_from_arrays(Mesh.PRIMITIVE_TRIANGLES, arrays)
    return mesh
\end{lstlisting}

\textbf{Explicación:} El vértice central se sitúa en el centro de la figura. Los vértices exteriores alternan entre dos radios para formar las puntas y valles de la estrella. Cada triángulo conecta el centro con dos vértices exteriores consecutivos, formando así la estrella completa. El material se configura para que la malla sea visible por ambas caras y utilice los colores de los vértices.
\end{solucion}

\section{Práctica 2}

\begin{ejercicio}
    Casa alargada en el eje $X$ con tejado a dos aguas.
\end{ejercicio}

\begin{solucion}
Se trata de construir una casa alargada, formada por un prisma rectangular sin base ni tapa, y un tejado a dos aguas. Se definen 10 vértices: 8 para las esquinas de las paredes y 2 para la arista superior del tejado. Las paredes laterales se forman con 8 triángulos (2 por cada cara), y el tejado con 6 triángulos (2 gabletes y 4 faldones). Los colores de los vértices se asignan según su posición, y se utiliza un material sin sombreado. El siguiente código en GDScript implementa la malla:

\begin{lstlisting}[language=GDScript]
extends MeshInstance3D

func _ready():
    var vertices = PackedVector3Array([
        Vector3(0.0, 0.0, 0.0), # v0
        Vector3(1.0, 0.0, 0.0), # v1
        Vector3(1.0, 0.0, 0.5), # v2
        Vector3(0.0, 0.0, 0.5), # v3
        Vector3(0.0, 0.5, 0.0), # v4
        Vector3(1.0, 0.5, 0.0), # v5
        Vector3(1.0, 0.5, 0.5), # v6
        Vector3(0.0, 0.5, 0.5), # v7
        Vector3(0.0, 1.0, 0.25), # v8
        Vector3(1.0, 1.0, 0.25)  # v9
    ])

    var colors = PackedColorArray()
    for v in vertices:
        colors.push_back(Color(v.x, v.y, v.z))

    var indices = PackedInt32Array([
        0, 1, 5,   0, 5, 4,
        2, 3, 7,   2, 7, 6,
        1, 2, 6,   1, 6, 5,
        3, 0, 4,   3, 4, 7,
        4, 7, 8,
        5, 9, 6,
        4, 5, 9,   4, 9, 8,
        7, 6, 9,   7, 9, 8
    ])

    var arrays = []
    arrays.resize(ArrayMesh.ARRAY_MAX)
    arrays[ArrayMesh.ARRAY_VERTEX] = vertices
    arrays[ArrayMesh.ARRAY_INDEX] = indices
    arrays[ArrayMesh.ARRAY_COLOR] = colors

    var new_mesh = ArrayMesh.new()
    new_mesh.add_surface_from_arrays(Mesh.PRIMITIVE_TRIANGLES, arrays)
    self.mesh = new_mesh

    var material = StandardMaterial3D.new()
    material.vertex_color_use_as_albedo = true
    material.shading_mode = StandardMaterial3D.SHADING_MODE_UNSHADED
    material.cull_mode = StandardMaterial3D.CULL_DISABLED
    self.material_override = material
\end{lstlisting}

La malla resultante representa una casa alargada con tejado a dos aguas, sin base ni tapa, y con colores interpolados según la posición de los vértices.
\end{solucion}

\begin{ejercicio}
    Pirámide con base de estrella.
\end{ejercicio}

\begin{solucion}
El objetivo es construir una pirámide cuya base es una estrella plana de $n$ puntas (como en el ejercicio anterior), y cuyas caras laterales convergen en un ápice centrado sobre la base. Se definen $2n+2$ vértices: uno central, $2n$ en el borde de la estrella (alternando radios), y uno para el ápice. Se generan $2n$ triángulos para la base y $2n$ triángulos laterales para las caras de la pirámide. El siguiente código en GDScript implementa la malla para $n=5$:

\begin{lstlisting}[language=GDScript]
extends MeshInstance3D

func _ready():
    const n: int = 5
    if n <= 1:
        push_error("n debe ser mayor que 1")
        return

    var vertices = PackedVector3Array()
    var colors = PackedColorArray()
    var indices = PackedInt32Array()

    var centro = Vector3(0.5, 0.5, 0)
    vertices.push_back(centro)
    colors.push_back(Color.WHITE)

    var radio_punta = 0.5
    var radio_valle = 0.25
    var num_vertices_externos = 2 * n

    for i in range(num_vertices_externos):
        var radio_actual = radio_punta if i % 2 == 0 else radio_valle
        var angulo = (float(i) / float(num_vertices_externos)) * TAU
        var x = centro.x + radio_actual * cos(angulo)
        var y = centro.y + radio_actual * sin(angulo)
        var z = 0.0
        vertices.push_back(Vector3(x, y, z))
        colors.push_back(Color(x, y, z))

    vertices.push_back(Vector3(0.5, 0.5, 0.5))
    colors.push_back(Color.WHITE)
    var idx_apex = vertices.size() - 1

    for i in range(num_vertices_externos):
        var idx_centro = 0
        var idx_v1 = i + 1
        var idx_v2 = (i + 1) % num_vertices_externos + 1
        indices.push_back(idx_centro)
        indices.push_back(idx_v1)
        indices.push_back(idx_v2)

    for i in range(num_vertices_externos):
        var idx_v1 = i + 1
        var idx_v2 = (i + 1) % num_vertices_externos + 1
        indices.push_back(idx_apex)
        indices.push_back(idx_v1)
        indices.push_back(idx_v2)

    var arrays = []
    arrays.resize(ArrayMesh.ARRAY_MAX)
    arrays[ArrayMesh.ARRAY_VERTEX] = vertices
    arrays[ArrayMesh.ARRAY_COLOR] = colors
    arrays[ArrayMesh.ARRAY_INDEX] = indices

    var new_mesh = ArrayMesh.new()
    new_mesh.add_surface_from_arrays(Mesh.PRIMITIVE_TRIANGLES, arrays)
    self.mesh = new_mesh

    var material = StandardMaterial3D.new()
    material.vertex_color_use_as_albedo = true
    material.shading_mode = StandardMaterial3D.SHADING_MODE_UNSHADED
    material.cull_mode = StandardMaterial3D.CULL_DISABLED
    self.material_override = material
\end{lstlisting}

La malla resultante es una pirámide con base de estrella y colores interpolados, cumpliendo con los requisitos de vértices y triángulos.
\end{solucion}

\begin{ejercicio}
    Rejilla perpendicular al eje $Y$.
\end{ejercicio}

\begin{solucion}
El objetivo es construir una malla indexada que represente una rejilla plana perpendicular al eje $Y$, es decir, situada en el plano $Y=0$ y ocupando el cuadrado $[0,1]\times[0,1]$ en los ejes $X$ y $Z$. La rejilla está formada por $m \times n$ vértices y $(m-1)\times(n-1)$ celdas, cada una compuesta por dos triángulos.

Cada vértice tiene color RGB igual a sus coordenadas $X$, $Y$, $Z$. La malla resultante tiene $m \times n$ vértices y $2(m-1)(n-1)$ triángulos.

A continuación se muestra el código en GDScript para el nodo \texttt{RejillaY}:

\begin{lstlisting}[language=GDScript]
extends MeshInstance3D

func _ready():
    var new_mesh = ArrayMeshRejilla(10, 10)
    self.mesh = new_mesh
    var material = StandardMaterial3D.new()
    material.vertex_color_use_as_albedo = true
    material.shading_mode = StandardMaterial3D.SHADING_MODE_UNSHADED
    material.cull_mode = StandardMaterial3D.CULL_DISABLED
    self.material_override = material

func ArrayMeshRejilla(m: int, n: int) -> ArrayMesh:
    var vertices = PackedVector3Array()
    var colors = PackedColorArray()
    var indices = PackedInt32Array()
    # Generar vértices y colores
    for i in range(m):
        for j in range(n):
            var x = float(i) / float(m - 1)
            var y = 0.0
            var z = float(j) / float(n - 1)
            vertices.push_back(Vector3(x, y, z))
            colors.push_back(Color(x, y, z))
    # Generar índices de triángulos
    for i in range(m - 1):
        for j in range(n - 1):
            var v0 = i * n + j
            var v1 = (i + 1) * n + j
            var v2 = i * n + (j + 1)
            var v3 = (i + 1) * n + (j + 1)
            # Triángulo 1
            indices.push_back(v0)
            indices.push_back(v2)
            indices.push_back(v3)
            # Triángulo 2
            indices.push_back(v0)
            indices.push_back(v3)
            indices.push_back(v1)
    var arrays = []
    arrays.resize(ArrayMesh.ARRAY_MAX)
    arrays[ArrayMesh.ARRAY_VERTEX] = vertices
    arrays[ArrayMesh.ARRAY_COLOR] = colors
    arrays[ArrayMesh.ARRAY_INDEX] = indices
    var mesh = ArrayMesh.new()
    mesh.add_surface_from_arrays(Mesh.PRIMITIVE_TRIANGLES, arrays)
    return mesh
\end{lstlisting}

\textbf{Explicación:} Se generan $m \times n$ vértices en el plano $Y=0$, con coordenadas $X$ y $Z$ equiespaciadas entre $0$ y $1$. Cada celda de la rejilla se triangula en dos triángulos usando los índices de los vértices. El color de cada vértice es igual a sus coordenadas. El material se configura para mostrar los colores de los vértices y sin sombreado.
\end{solucion}


\begin{ejercicio}
    Torre de planta cuadrada.
\end{ejercicio}

\begin{solucion} La tarea es construir una torre hueca de planta cuadrada y altura $n$, formada por $n$ secciones apiladas. Cada sección tiene 4 vértices en la base y 4 en la parte superior, resultando en $4(n+1)$ vértices en total. Las caras laterales se forman con triángulos que conectan los vértices de las secciones adyacentes.

\subsection*{Explicación Detallada}

El objetivo es crear una torre hueca apilando $n$ secciones de paredes cuadradas, una encima de la otra. Se generan $n+1$ "anillos" de vértices, cada uno con 4 vértices formando un cuadrado de lado 1, situados en alturas $Y=0,1,\ldots,n$. Así, hay $4(n+1)$ vértices en total.

Para las caras, cada sección conecta dos anillos consecutivos. Cada cara lateral se forma con 2 triángulos, y hay 4 caras por sección, resultando en $8n$ triángulos en total. Los índices se calculan usando aritmética modular para cerrar la figura correctamente.

\subsection*{Código en GDScript (Torre.gd)}

\begin{lstlisting}[language=GDScript]
extends MeshInstance3D

func _ready():
    const n: int = 5
    if n < 1:
        push_error("n debe ser 1 o mayor")
        return

    var vertices = PackedVector3Array()
    var indices = PackedInt32Array()

    # Generar vértices
    for i in range(n + 1):
        var y = float(i)
        vertices.push_back(Vector3(0.0, y, 0.0))
        vertices.push_back(Vector3(1.0, y, 0.0))
        vertices.push_back(Vector3(1.0, y, 1.0))
        vertices.push_back(Vector3(0.0, y, 1.0))

    # Generar triángulos
    for i in range(n):
        for j in range(4):
            var idx0 = i * 4 + j
            var idx1 = i * 4 + (j + 1) % 4
            var idx2 = (i + 1) * 4 + j
            var idx3 = (i + 1) * 4 + (j + 1) % 4

            indices.push_back(idx0)
            indices.push_back(idx1)
            indices.push_back(idx3)

            indices.push_back(idx0)
            indices.push_back(idx3)
            indices.push_back(idx2)

    var arrays = []
    arrays.resize(ArrayMesh.ARRAY_MAX)
    arrays[ArrayMesh.ARRAY_VERTEX] = vertices
    arrays[ArrayMesh.ARRAY_INDEX] = indices

    var new_mesh = ArrayMesh.new()
    new_mesh.add_surface_from_arrays(Mesh.PRIMITIVE_TRIANGLES, arrays)
    self.mesh = new_mesh

    var material = StandardMaterial3D.new()
    material.albedo = Color.WHITE
    material.shading_mode = StandardMaterial3D.SHADING_MODE_UNSHADED
    self.material_override = material
\end{lstlisting}

\textbf{Resumen:} Se construye una torre hueca de planta cuadrada y altura $n$, formada por $4(n+1)$ vértices y $8n$ triángulos, con material sin sombreado.
\end{solucion}

\section{Práctica 3}

\begin{ejercicio}
    Grafo de escena: estrella y conos (GrafoEstrellaX).
\end{ejercicio}

\begin{solucion}
Este ejercicio consiste en construir un grafo de escena jerárquico en Godot, donde el nodo raíz (\texttt{GrafoEstrellaX}) contiene una estrella plana en el plano $XZ$ y $n$ conos, uno en cada punta de la estrella. Los conos comparten malla y material, y se orientan hacia fuera desde el centro. Además, todo el grafo puede rotar sobre el eje $X$ mediante animación.

\textbf{Puntos clave:}
\begin{itemize}
    \item La estrella se genera con una función que crea la malla en el plano $XZ$.
    \item Los conos se generan con revolución de un perfil y se colocan/orientan en las puntas.
    \item Se usa un único material y malla para todos los conos.
    \item La animación rota el nodo raíz, haciendo girar todo el conjunto.
\end{itemize}

\textbf{Código GDScript:}

\begin{lstlisting}[language=GDScript]
extends Node3D

const VELOCIDAD_GIRO = 2.5 * TAU
var activar := "activar_giro_grafoEstrellaX"
var activa := true

func _ready():
    const n: int = 5
    if n <= 1:
        push_error("n debe ser > 1")
        return

    var star_mesh = ArrayMeshEstrellaZ(n)
    var star_node = MeshInstance3D.new()
    star_node.mesh = star_mesh
    var star_material = StandardMaterial3D.new()
    star_material.vertex_color_use_as_albedo = true
    star_material.shading_mode = StandardMaterial3D.SHADING_MODE_UNSHADED
    star_material.cull_mode = StandardMaterial3D.CULL_DISABLED
    star_node.material_override = star_material
    add_child(star_node)

    var cone_mesh = crear_mesh_cono()
    var cone_material = StandardMaterial3D.new()
    cone_material.albedo = Color.WHITE
    cone_material.shading_mode = StandardMaterial3D.SHADING_MODE_UNSHADED

    var centro = Vector3.ZERO
    var radio_punta = 0.5
    var num_vertices_totales_estrella = 2 * n

    for i in range(n):
        var angulo_punta = (float(i * 2) / float(num_vertices_totales_estrella)) * TAU
        var x = centro.x + radio_punta * cos(angulo_punta)
        var y = 0.0
        var z = centro.z + radio_punta * sin(angulo_punta)
        var pos_punta = Vector3(x, y, z)
        var cone_node = MeshInstance3D.new()
        cone_node.mesh = cone_mesh
        cone_node.material_override = cone_material
        cone_node.position = pos_punta
        var dir_original = Vector3.UP
        var dir_deseada = (pos_punta - centro).normalized()
        var rotation_axis = dir_original.cross(dir_deseada).normalized()
        var rotation_angle = dir_original.angle_to(dir_deseada)
        cone_node.rotate(rotation_axis, rotation_angle)
        add_child(cone_node)

func _process(delta):
    if Input.is_action_just_pressed(activar):
        activa = !activa
    if activa:
        rotate_x(VELOCIDAD_GIRO * delta)

func crear_mesh_cono() -> ArrayMesh:
    var perfil_cono = PackedVector2Array([
        Vector2(0.0, 0.15),
        Vector2(0.14, 0.0),
        Vector2(0.0, 0.0)
    ])
    var vertices = PackedVector3Array()
    var triangulos = PackedInt32Array()
    RevolucionUtils.generar_malla_revolucion(perfil_cono, 16, vertices, triangulos)
    var normales = Utilidades.calcNormales(vertices, triangulos)
    var arrays = []
    arrays.resize(ArrayMesh.ARRAY_MAX)
    arrays[Mesh.ARRAY_VERTEX] = vertices
    arrays[Mesh.ARRAY_INDEX] = triangulos
    arrays[Mesh.ARRAY_NORMAL] = normales
    var new_mesh = ArrayMesh.new()
    new_mesh.add_surface_from_arrays(Mesh.PRIMITIVE_TRIANGLES, arrays)
    return new_mesh

func ArrayMeshEstrellaZ(p_n: int) -> ArrayMesh:
    var vertices = PackedVector3Array()
    var colors = PackedColorArray()
    var indices = PackedInt32Array()
    var centro = Vector3.ZERO
    vertices.push_back(centro)
    colors.push_back(Color.WHITE)
    var radio_punta = 0.5
    var radio_valle = 0.25
    var num_vertices_externos = 2 * p_n
    for i in range(num_vertices_externos):
        var radio_actual = radio_punta if i % 2 == 0 else radio_valle
        var angulo = (float(i) / float(num_vertices_externos)) * TAU
        var x = centro.x + radio_actual * cos(angulo)
        var y = 0.0
        var z = centro.z + radio_actual * sin(angulo)
        vertices.push_back(Vector3(x, y, z))
        colors.push_back(Color(x, y, z))
    for i in range(num_vertices_externos):
        var idx_centro = 0
        var idx_v1 = i + 1
        var idx_v2 = (i + 1) % num_vertices_externos + 1
        indices.push_back(idx_centro)
        indices.push_back(idx_v1)
        indices.push_back(idx_v2)
    var arrays = []
    arrays.resize(ArrayMesh.ARRAY_MAX)
    arrays[Mesh.ARRAY_VERTEX] = vertices
    arrays[Mesh.ARRAY_COLOR] = colors
    arrays[Mesh.ARRAY_INDEX] = indices
    var new_mesh = ArrayMesh.new()
    new_mesh.add_surface_from_arrays(Mesh.PRIMITIVE_TRIANGLES, arrays)
    return new_mesh
\end{lstlisting}

\end{solucion}

\begin{ejercicio}
    Grafo de escena: cubo central y cubos pequeños (GrafoCubos).
\end{ejercicio}

\begin{solucion}
Se pide construir un grafo de escena donde un cubo central grande se forma con 6 rejillas y, en el centro de cada cara, hay un cubo pequeño alargado. Los cubos pequeños rotan alrededor del eje que pasa por su centro y el origen, usando pivotes. Todo el sistema tiene un único grado de libertad de animación.

\textbf{Puntos clave:}
\begin{itemize}
    \item El cubo central se construye con 6 mallas de rejilla, cada una orientada y posicionada en una cara.
    \item Los cubos pequeños se crean una vez y se colocan como hijos de nodos pivote, que se rotan para animar.
    \item Se usan materiales y mallas compartidos.
    \item La animación rota los pivotes de los cubos pequeños.
\end{itemize}

\textbf{Código GDScript:}

\begin{lstlisting}[language=GDScript]
extends Node3D

var angulo_giro := 0.0
const VELOCIDAD_GIRO = TAU
var activar_accion := "activar_giro_cubos"
var activa := true

var pivot_x_pos: Node3D
var pivot_x_neg: Node3D
var pivot_y_pos: Node3D
var pivot_y_neg: Node3D
var pivot_z_pos: Node3D
var pivot_z_neg: Node3D

func _ready():
    var rejilla_mesh = ArrayMeshRejilla(11, 11)
    var cubo_mesh = ArrayMeshCubo24()
    var rejilla_material = StandardMaterial3D.new()
    rejilla_material.vertex_color_use_as_albedo = true
    rejilla_material.shading_mode = StandardMaterial3D.SHADING_MODE_UNSHADED
    rejilla_material.cull_mode = StandardMaterial3D.CULL_DISABLED
    var cubo_material = StandardMaterial3D.new()
    cubo_material.albedo = Color.WHITE
    cubo_material.shading_mode = StandardMaterial3D.SHADING_MODE_UNSHADED

    var cubo_central = Node3D.new()
    add_child(cubo_central)
    var centro_rejilla = Vector3(-0.5, 0, -0.5)
    var transforms = [
        Transform3D(Basis(), Vector3(0, 0.5, 0)) * Transform3D(Basis(), centro_rejilla),
        Transform3D(Basis.from_euler(Vector3(PI, 0, 0)), Vector3(0, -0.5, 0)) * Transform3D(Basis(), centro_rejilla),
        Transform3D(Basis.from_euler(Vector3(0, 0, PI/2.0)), Vector3(0.5, 0, 0)) * Transform3D(Basis(), centro_rejilla),
        Transform3D(Basis.from_euler(Vector3(0, 0, -PI/2.0)), Vector3(-0.5, 0, 0)) * Transform3D(Basis(), centro_rejilla),
        Transform3D(Basis.from_euler(Vector3(-PI/2.0, 0, 0)), Vector3(0, 0, 0.5)) * Transform3D(Basis(), centro_rejilla),
        Transform3D(Basis.from_euler(Vector3(PI/2.0, 0, 0)), Vector3(0, 0, -0.5)) * Transform3D(Basis(), centro_rejilla)
    ]
    for i in range(6):
        var cara = MeshInstance3D.new()
        cara.mesh = rejilla_mesh
        cara.material_override = rejilla_material
        cara.transform = transforms[i]
        cubo_central.add_child(cara)

    var dist_cubo_peq = 0.7
    var escala_alargada = Vector3(0.2, 0.4, 0.2)

    pivot_y_pos = Node3D.new()
    add_child(pivot_y_pos)
    crear_cubo_pequeno(pivot_y_pos, cubo_mesh, cubo_material, Vector3(0, dist_cubo_peq, 0), escala_alargada)

    pivot_y_neg = Node3D.new()
    add_child(pivot_y_neg)
    crear_cubo_pequeno(pivot_y_neg, cubo_mesh, cubo_material, Vector3(0, -dist_cubo_peq, 0), escala_alargada)

    pivot_x_pos = Node3D.new()
    add_child(pivot_x_pos)
    crear_cubo_pequeno(pivot_x_pos, cubo_mesh, cubo_material, Vector3(dist_cubo_peq, 0, 0), escala_alargada.rotated(Vector3.FORWARD, PI/2.0))

    pivot_x_neg = Node3D.new()
    add_child(pivot_x_neg)
    crear_cubo_pequeno(pivot_x_neg, cubo_mesh, cubo_material, Vector3(-dist_cubo_peq, 0, 0), escala_alargada.rotated(Vector3.FORWARD, -PI/2.0))

    pivot_z_pos = Node3D.new()
    add_child(pivot_z_pos)
    crear_cubo_pequeno(pivot_z_pos, cubo_mesh, cubo_material, Vector3(0, 0, dist_cubo_peq), escala_alargada.rotated(Vector3.RIGHT, PI/2.0))

    pivot_z_neg = Node3D.new()
    add_child(pivot_z_neg)
    crear_cubo_pequeno(pivot_z_neg, cubo_mesh, cubo_material, Vector3(0, 0, -dist_cubo_peq), escala_alargada.rotated(Vector3.RIGHT, -PI/2.0))

func _process(delta):
    if Input.is_action_just_pressed(activar_accion):
        activa = !activa
    if not activa:
        return
    angulo_giro += VELOCIDAD_GIRO * delta
    if pivot_y_pos:
        pivot_y_pos.rotation.y = angulo_giro
        pivot_y_neg.rotation.y = angulo_giro
        pivot_x_pos.rotation.x = angulo_giro
        pivot_x_neg.rotation.x = angulo_giro
        pivot_z_pos.rotation.z = angulo_giro
        pivot_z_neg.rotation.z = angulo_giro

func crear_cubo_pequeno(pivote: Node3D, mesh: ArrayMesh, mat: StandardMaterial3D, pos: Vector3, escala: Vector3):
    var cubo = MeshInstance3D.new()
    cubo.mesh = mesh
    cubo.material_override = mat
    cubo.position = pos
    cubo.scale = escala
    pivote.add_child(cubo)

func ArrayMeshRejilla(m: int, n: int) -> ArrayMesh:
    var vertices = PackedVector3Array()
    var colors = PackedColorArray()
    var indices = PackedInt32Array()
    var normales = PackedVector3Array()
    for i in range(m):
        for j in range(n):
            var x = float(i) / float(m - 1)
            var y = 0.0
            var z = float(j) / float(n - 1)
            vertices.push_back(Vector3(x, y, z))
            colors.push_back(Color(x, y, z))
            normales.push_back(Vector3.UP)
    for i in range(m - 1):
        for j in range(n - 1):
            var v0 = i * n + j
            var v1 = (i + 1) * n + j
            var v2 = i * n + (j + 1)
            var v3 = (i + 1) * n + (j + 1)
            indices.push_back(v0)
            indices.push_back(v2)
            indices.push_back(v3)
            indices.push_back(v0)
            indices.push_back(v3)
            indices.push_back(v1)
    var arrays = []
    arrays.resize(ArrayMesh.ARRAY_MAX)
    arrays[Mesh.ARRAY_VERTEX] = vertices
    arrays[Mesh.ARRAY_COLOR] = colors
    arrays[Mesh.ARRAY_INDEX] = indices
    arrays[Mesh.ARRAY_NORMAL] = normales
    var mesh = ArrayMesh.new()
    mesh.add_surface_from_arrays(Mesh.PRIMITIVE_TRIANGLES, arrays)
    return mesh

func ArrayMeshCubo24() -> ArrayMesh:
    var vertices := PackedVector3Array([
        Vector3(-0.5,  0.5,  0.5), Vector3( 0.5,  0.5,  0.5), Vector3( 0.5, -0.5,  0.5), Vector3(-0.5, -0.5,  0.5),
        Vector3( 0.5,  0.5, -0.5), Vector3(-0.5,  0.5, -0.5), Vector3(-0.5, -0.5, -0.5), Vector3( 0.5, -0.5, -0.5),
        Vector3( 0.5,  0.5,  0.5), Vector3( 0.5,  0.5, -0.5), Vector3( 0.5, -0.5, -0.5), Vector3( 0.5, -0.5,  0.5),
        Vector3(-0.5,  0.5, -0.5), Vector3(-0.5,  0.5,  0.5), Vector3(-0.5, -0.5,  0.5), Vector3(-0.5, -0.5, -0.5),
        Vector3(-0.5,  0.5, -0.5), Vector3( 0.5,  0.5, -0.5), Vector3( 0.5,  0.5,  0.5), Vector3(-0.5,  0.5,  0.5),
        Vector3(-0.5, -0.5,  0.5), Vector3( 0.5, -0.5,  0.5), Vector3( 0.5, -0.5, -0.5), Vector3(-0.5, -0.5, -0.5)
    ])
    var triangulos := PackedInt32Array([
        0, 1, 2,  0, 2, 3,    4, 5, 6,  4, 6, 7,
        8, 9, 10, 8, 10, 11,   12, 13, 14, 12, 14, 15,
        16, 17, 18, 16, 18, 19, 20, 21, 22, 20, 22, 23
    ])
    var normales := Utilidades.calcNormales(vertices, triangulos)
    var arrays := []
    arrays.resize(ArrayMesh.ARRAY_MAX)
    arrays[Mesh.ARRAY_VERTEX] = vertices
    arrays[Mesh.ARRAY_INDEX] = triangulos
    arrays[Mesh.ARRAY_NORMAL] = normales
    var new_mesh := ArrayMesh.new()
    new_mesh.add_surface_from_arrays(Mesh.PRIMITIVE_TRIANGLES, arrays)
    return new_mesh
\end{lstlisting}

\end{solucion}

\begin{thebibliography}{99}

  \bibitem{Referencia1}
  Ismael Sallami Moreno, \textbf{Estudiante del Doble Grado en Ingeniería Informática + ADE}, Universidad de Granada, 2025.
  
  \bibitem{DiapositivasAsignatura}
  Universidad de Granada, \emph{Diapositivas de la asignatura}, Curso 2025/2026.

  % \bibitem{Referencia2}
  % Autor Apellido, \emph{Título del libro o artículo}, Editorial o Revista, Año.
  
  % \bibitem{Referencia3}
  % Nombre Autor, \emph{Título del documento}, Conferencia/URL, Año.
  
  \end{thebibliography}
    % si tienes paquetes personalizados

\end{document}
