\chapter{Preguntas Tipo}
\section*{I. Teoría de Lenguajes y Gramáticas (Tema 1)}

Aquí el foco está en la inducción matemática y la cardinalidad de conjuntos.

\subsection*{1. Demostración de corrección de una Gramática}
Esta es una pregunta clásica y fundamental. No basta con diseñar la gramática, debes probar que $G$ genera exactamente $L$.

\begin{itemize}
    \item \textbf{Pregunta tipo:} Dado el lenguaje $L$ y una gramática $G$, demuestra formalmente que $G$ genera exactamente $L$.
    \item \textbf{Lo que espero:} Tienes que demostrar la doble inclusión:
    \begin{enumerate}
        \item $L(G) \subseteq L$: ``Toda palabra generada por $G$ tiene igual número de a's y b's'' (usando inducción sobre el número de pasos de derivación).
        \item $L \subseteq L(G)$: ``Toda palabra con igual número de a's y b's puede ser generada por $G$'' (usando inducción sobre la longitud de la palabra).
    \end{enumerate}
\end{itemize}

\subsection*{2. Cardinalidad y Conjuntos No Numerables}

\begin{itemize}
    \item \textbf{Pregunta tipo:} Demuestre formalmente que el conjunto de todos los lenguajes sobre un alfabeto no vacío $\Sigma$ es \textbf{no numerable}.
    \item \textbf{Lo que espero:} Debes usar el argumento de la \textbf{diagonalización de Cantor} (reducción al absurdo).
    \item Debes suponer que existe una biyección $f$ entre los lenguajes y los números naturales.
    \item Construir un lenguaje ``trampa'' $L^*$ que difiera de cada lenguaje $L_i$ en al menos una palabra (la $i$-ésima palabra).
    \item Llegar a la contradicción de que $L^*$ no puede tener ningún número asociado.
\end{itemize}

\subsection*{3. Inducción sobre operaciones de cadenas}

\begin{itemize}
    \item \textbf{Pregunta tipo:} Demuestre por inducción que para cualquier palabra $w$, se cumple que $(w^R)^R = w$ (donde $w^R$ es la inversa) o propiedades similares de la concatenación.
\end{itemize}

\bigskip
\hrule
\bigskip

\section*{II. Autómatas Finitos: Equivalencias y Construcciones (Tema 2)}

En este bloque, las preguntas de ``demostración'' suelen ser constructivas: ``Demuestra que existe...'' se traduce en ``Construye el algoritmo que lo transforma''.

\subsection*{4. Equivalencia AFND $\to$ AFD (Construcción de Subconjuntos)}

\begin{itemize}
    \item \textbf{Pregunta tipo:} Demuestre que todo lenguaje aceptado por un Autómata Finito No Determinista (AFND) es aceptado también por un Autómata Finito Determinista (AFD).
    \item \textbf{Lo que espero:} No quiero solo el dibujo. Quiero la definición formal de la quíntupla del AFD equivalente $M'$ donde:
    \begin{itemize}
        \item Los estados $Q'$ son el conjunto potencia de los estados originales ($Q'$ = $\mathcal{P}(Q)$).
        \item Defines la transición $\delta'$ como la unión de las transiciones originales para todo $q \in Q$.
    \end{itemize}
\end{itemize}

\subsection*{5. Equivalencia Expresiones Regulares $\leftrightarrow$ Autómatas}

Aquí hay dos vías de demostración que suelen caer como ejercicios prácticos largos:

\begin{itemize}
    \item \textbf{Vía A (Análisis): De Autómata a Expresión Regular (Método $R_{i,j}^{(k)}$).}
    \begin{itemize}
        \item \textbf{Pregunta:} Dado un AFD, calcule la expresión regular del lenguaje aceptado utilizando la definición recursiva de los conjuntos $R_{i,j}^{(k)}$.
        \item \textbf{Clave:} Debes entender que $R_{i,j}^{(k)}$ son las palabras que van del estado $i$ al $j$ pasando solo por estados intermedios con índice $\leq k$. Tienes que aplicar la fórmula recursiva: 
        \[
            R_{i,j}^{(k)} = R_{i,j}^{(k-1)} \cup R_{i,k}^{(k-1)} (R_{k,k}^{(k-1)})^* R_{k,j}^{(k-1)}
        \]
    \end{itemize}
    \item \textbf{Vía B (Síntesis): De Expresión Regular a Autómata.}
    \begin{itemize}
        \item \textbf{Pregunta:} Demuestre constructivamente que para cualquier expresión regular existe un AFND-$\epsilon$ que la reconoce.
        \item \textbf{Clave:} Debes dibujar o definir los ``bloques constructivos'' (plantillas) para la unión, concatenación y clausura de Kleene (Estrella).
    \end{itemize}
\end{itemize}

\subsection*{6. Lema de Arden (Resolución de Ecuaciones)}

\begin{itemize}
    \item \textbf{Pregunta tipo:} Dado el siguiente sistema de ecuaciones de expresiones regulares derivado de un autómata, resuélvalo para encontrar la expresión del estado inicial.
    \item \textbf{Lo que espero:} Que sepas aplicar que la solución es $X = Q^*P$ (siempre que $\epsilon \notin Q$) y sepas sustituir paso a paso hasta despejar la variable del estado inicial.
\end{itemize}

\bigskip
\hrule
\bigskip

\section*{III. Gramáticas Regulares y Autómatas (Tema 2)}

\subsection*{7. Gramáticas Lineales por la Izquierda vs. Derecha}

\begin{itemize}
    \item \textbf{Pregunta tipo:} Dado un autómata finito, construya la \textbf{Gramática Lineal por la Izquierda} que genera el mismo lenguaje.
    \item \textbf{La trampa:} Es fácil sacar la lineal por la \emph{derecha} directamente del autómata ($q \to a p$). Para la lineal por la \emph{izquierda}, el proceso riguroso es:
    \begin{enumerate}
        \item Invertir el autómata (finales se vuelven iniciales, flechas al revés).
        \item Sacar la gramática lineal por la derecha de ese autómata invertido.
        \item Invertir las producciones de dicha gramática.
    \end{enumerate}
\end{itemize}

\bigskip
\hrule
\bigskip

\section*{Resumen para tu sesión de estudio}

Si quieres sacar nota de honor, asegúrate de dominar estas tres cosas ``difíciles'':

\begin{enumerate}
    \item \textbf{La inducción} para probar que $L(G) = L$.
    \item \textbf{El método recursivo $R_{i,j}^{(k)}$} (es tedioso, pero demuestra que entiendes la estructura interna de los caminos en un grafo).
    \item \textbf{La formalización de tuplas:} Cuando te pidan convertir un AFND a AFD, escribe explícitamente $M' = (Q', \Sigma, \delta', q_0', F')$. No hagas solo el dibujo de las bolitas.
\end{enumerate}

