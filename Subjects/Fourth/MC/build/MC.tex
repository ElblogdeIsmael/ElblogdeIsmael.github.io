% ========================
% estilo.latex mínimo funcional
% ========================

\documentclass[12pt]{report} % report para capítulos

% ========================
% Paquetes y comandos extra
% ========================
% ===========================
% Paquetes básicos de idioma y codificación
% ===========================
\usepackage[utf8]{inputenc}   % Codificación UTF-8
\usepackage[T1]{fontenc}      % Acentos y caracteres correctos
\usepackage[spanish]{babel}   % Traducción al español (capítulos, índices, etc.)
\usepackage{csquotes}         % Citas tipográficas correctas

% ===========================
% Tipografía
% ===========================
\usepackage{lmodern}          % Fuente Latin Modern
\usepackage{microtype}        % Mejoras tipográficas (espaciado, justificación)

% ===========================
% Márgenes y geometría
% ===========================
\usepackage{geometry}         % Control de márgenes
\geometry{a4paper, top=3cm, bottom=3cm, left=3cm, right=3cm}

% ===========================
% Matemáticas
% ===========================
\usepackage{amsmath, amssymb, amsthm} % Paquetes AMS
\usepackage{mathtools}        % Extiende amsmath
\usepackage{physics}          % Notación física y matemática (derivadas, bra-ket, etc.)
\usepackage{siunitx}          % Unidades SI (e.g. \SI{3}{m/s})
% \sisetup{locale=ES}           % Configuración para español (coma decimal, etc.)
\AtBeginDocument{\RenewCommandCopy\qty\SI} % Resolve siunitx and physics conflict


% ===========================
% Gráficos, tablas y colores
% ===========================
\usepackage{graphicx}         % Insertar imágenes
\usepackage{xcolor}           % Colores personalizados
\usepackage{tikz}             % Dibujos vectoriales
\usetikzlibrary{calc,positioning,shapes,arrows} % Librerías útiles de TikZ
\usepackage{pgfplots}         % Gráficas de funciones
\pgfplotsset{compat=1.18}
\usepackage{float}            % Control de posición de figuras/tablas
\usepackage{booktabs}         % Tablas profesionales
\usepackage{multirow}         % Celdas que ocupan varias filas
\usepackage{array}            % Más control en tablas
\usepackage{colortbl}         % Tablas con colores
\usepackage{inconsolata}


% ===========================
% Listas y enumeraciones
% ===========================
\usepackage{enumitem}         % Control de listas enumeradas y viñetas

% ===========================
% Encabezados, pies y diseño
% ===========================
\usepackage{fancyhdr}         % Encabezados y pies de página
\usepackage{titlesec}         % Personalizar títulos de capítulos/secciones
\usepackage{setspace}         % Espaciado entre líneas
\usepackage{parskip}          % Control del espacio entre párrafos

% ===========================
% Referencias, hipervínculos y citas
% ===========================
\usepackage{hyperref}         % Hipervínculos en PDF
\hypersetup{
    colorlinks = true,
    linkcolor  = red!70,
    citecolor  = red!70,
    urlcolor   = red!70,
    pdfpagelayout = SinglePage, % Asegura que el contenido se ajuste a una sola página
    pdfstartview = Fit          % Ajusta el contenido al tamaño de la página
}
\usepackage{cleveref}         % Referencias inteligentes (\cref)

% ===========================
% Código fuente
% ===========================
\usepackage{listings}         % Mostrar código con estilo
\usepackage{minted}           % (mejor opción, requiere Python y pygments)

% ===========================
% Bibliografía
% ===========================
\usepackage[backend=biber,style=apa]{biblatex} % Ejemplo: estilo APA
\addbibresource{referencias.bib}              % Archivo .bib

% ===========================
% Otros útiles
% ===========================
\usepackage{pdfpages}         % Insertar PDFs externos
\usepackage{blindtext}        % Texto de prueba
\usepackage{caption}          % Personalizar pies de figura/tabla
\usepackage{subcaption}       % Subfiguras
\usepackage{tocloft} 
\usepackage{amsthm}
\usepackage{subcaption}
\usepackage{truncate} % permite truncar texto si no cabe
\usepackage{libertinus}  % reemplaza lmodern
\usepackage{booktabs}  % para \toprule, \midrule, \bottomrule
\usepackage{array}     % para definir columnas personalizadas
\usepackage{colortbl}  % colores en tablas
\usepackage{etoolbox}
\AtBeginEnvironment{tabular}{\rowcolors{2}{gray!10}{white}\renewcommand{\arraystretch}{1.2}}

% ===========================
% Opciones de fuentes sugeridas
% ===========================
% TeX Gyre Pagella (estilo Palatino)
% \usepackage{fontspec}
% \usepackage{unicode-math}
% \setmainfont{TeX Gyre Pagella}
% \setmathfont{TeX Gyre Pagella Math}

% TeX Gyre Termes (estilo Times)
% \setmainfont{TeX Gyre Termes}
% \setmathfont{TeX Gyre Termes Math}

% Libertinus (elegante y completa)
% \setmainfont{Libertinus Serif}
% \setmathfont{Libertinus Math}

% TeX Gyre Bonum (estilo Garamond)
% \setmainfont{TeX Gyre Bonum}
% \setmathfont{TeX Gyre Bonum Math}

% Latin Modern (moderno de Computer Modern)
% \setmainfont{Latin Modern Roman}
% \setmathfont{Latin Modern Math}


% \usepackage{helvet}
% \usepackage{libertine}
% \usepackage[sfdefault]{FiraSans}

\usepackage{tcolorbox} % para cajas de colores




  % si tienes paquetes personalizados
% aquí van los comandos personalizados
% Comando para incluir imágenes
\newcommand{\incluirimagen}[3][]{%
\begin{figure}[H]
    \centering
    \includegraphics[width=\linewidth,#1]{#2}
    \caption{#3}
    \label{fig:#2}
\end{figure}
}

% comando para ejercicios con fondo
\newtheoremstyle{ejerciciostyle}
  {10pt}   % Espacio arriba
  {10pt}   % Espacio abajo
  %{\itshape} % Fuente del cuerpo
  {}
  {}       % Sangría
  {\bfseries} % Fuente del encabezado
  {}      % Puntuación tras encabezado
  { }      % Espacio tras encabezado
  {\thmname{#1} \thmnumber{#2}. \thmnote{#3}}


% % comando formal para enunciado de ejercicios
% \theoremstyle{ejerciciostyle}
% \newtheorem{ejercicio}{Ejercicio}[chapter]

\theoremstyle{ejerciciostyle}
\newtheorem{ejercicio}{Ejercicio}[section]

\renewcommand{\theejercicio}{\thechapter.\arabic{section}.\arabic{ejercicio}}


% comando formal para soluciones
\theoremstyle{remark}
\newtheorem{solucion}{Solución}[ejercicio]

\renewcommand{\thesolucion}{\thechapter.\arabic{section}.\arabic{ejercicio}}

% Comando para dos imágenes en paralelo
\newcommand{\dosimagenes}[6]{%
    \begin{figure}[h!]
        \centering
        \begin{minipage}{0.48\linewidth}
            \centering
            \includegraphics[width=\linewidth]{#1}
            \caption{#2}
            \label{#5}
        \end{minipage}\hfill
        \begin{minipage}{0.48\linewidth}
            \centering
            \includegraphics[width=\linewidth]{#3}
            \caption{#4}
            \label{#6}
        \end{minipage}
    \end{figure}
}

% \dosimagenes{media/fondo.jpg}{Descripción 1}{media/fondo.jpg}{Descripción 2}{fig:descripcion1}{fig:descripcion2}

% \ref{fig:descripcion1} es la mejor
% \ref{fig:descripcion2} es la mejor

\newcommand{\portadaimg}{\VAR{portadaimg}}

% Comando para crear una nota estilo información
% \newcommand{\nota}[2]{%
% \begin{tcolorbox}[colframe=blue!75!black, colback=blue!5!white, title=\textbf{#1}]
%     #2
% \end{tcolorbox}
% }
\newtheorem{nota}{Nota}[chapter]


% Comando para poner dos códigos en paralelo
\newcommand{\doscodigos}[4]{%
  \noindent
  \begin{minipage}{0.48\linewidth}
    \lstset{language=#1}
    \lstinputlisting{#2}
  \end{minipage}\hfill
  \begin{minipage}{0.48\linewidth}
    \lstset{language=#3}
    \lstinputlisting{#4}
  \end{minipage}
}

% Comando para poner un solo código
\newcommand{\uncodigo}[2]{%
  \begin{lstlisting}[language=#1]
#2
  \end{lstlisting}
}


% % Listas de archivos (sin guiones en los nombres de macros)
% \newcommand{\listagdfilesSesion2Mallas2D}{cargatexturas.gd, envioinmediato.gd, malla2dcontexturas.gd, mallaconcoloresdevertices.gd, mallanoindentada.gd}
% \newcommand{\listagdfilesSesion2Mallas3D}{mallaindexada3d.gd, materialconcolordeplano.gd, materialconcoloresdevertices.gd, tablas.gd}

% % Macro que recorre una lista de archivos en un subdirectorio
% \newcommand{\includegdfiles}[2]{%
%   % #1 = subdirectorio
%   % #2 = nombre de la lista de archivos
%   \foreach \filename in #2 {%
%     \includecode[gdstyle]{code/#1/\filename}{\filename}
%   }%
% }



% Comando para ejercicio resuelto
\newtheoremstyle{ejercicioresueltostyle}
    {10pt}   % Espacio arriba
    {10pt}   % Espacio abajo
    {\itshape} % Fuente del cuerpo
    {}       % Sangría
    {\bfseries} % Fuente del encabezado
    {}      % Puntuación tras encabezado
    { }      % Espacio tras encabezado
    {\thmname{#1} \thmnumber{#2}. \thmnote{#3}}

\theoremstyle{ejercicioresueltostyle}
\newtheorem{ejercicioresuelto}{Ejercicio Resuelto}[section]

\renewcommand{\theejercicioresuelto}{\thechapter.\arabic{section}.\arabic{ejercicioresuelto}}


%======================================================================== 
% PRACTICAS
%========================================================================

% Comando para definir un tema
\newcommand{\tema}[1]{%
  \section{#1}
  \addcontentsline{toc}{section}{#1}
}
\usepackage{tikz}
\usepackage{graphicx} % necesario para \resizebox
\usepackage{etoolbox}

% ======== NODOS ========
\newcommand{\nodo}[4][]{\node[state, #1] (#2) at (#3) {$#4$};}
% Uso: \nodo[initial,accepting]{q0}{0,0}{q_0}

% ======== FLECHAS ========
\newcommand{\flecha}[4][]{\draw[->, #1] (#2) -- (#3) node[midway, above] {#4};}
% Uso: \flecha{q0}{q1}{0} o \flecha[bend left]{q1}{q2}{1}

\newcommand{\flechaabajo}[4][]{\draw[->, #1] (#2) -- (#3) node[midway, below, yshift=-6pt] {#4};}
% Igual que \flecha pero con etiqueta abajo
\newcommand{\flechaarriba}[4][]{\draw[->, #1] (#2) -- (#3) node[midway, above, yshift=6pt] {#4};}
% Igual que \flecha pero con etiqueta arriba
\newcommand{\flechaderecha}[4][]{\draw[->, #1] (#2) -- (#3) node[midway, right] {#4};}
% Igual que \flecha pero con etiqueta a la derecha
\newcommand{\flechaiquierda}[4][]{\draw[->, #1] (#2) -- (#3) node[midway, left] {#4};}
% Igual que \flecha pero con etiqueta a la izquierda

\newcommand{\curva}[5][]{\draw[->, bend #1] (#2) to node[midway, #5] {#4} (#3);}
% Uso: \curva[left]{q1}{q2}{1}{below}


\newcommand{\loopa}[3]{\draw[->] (#1) edge[loop above] node {#2} (#1);}
\newcommand{\loopb}[3]{\draw[->] (#1) edge[loop below] node {#2} (#1);}
\newcommand{\loopr}[3]{\draw[->] (#1) edge[loop right] node {#2} (#1);}
\newcommand{\loopl}[3]{\draw[->] (#1) edge[loop left] node {#2} (#1);}
% Uso: \loopa{q1}{0}

% ======== ESTILOS ESPECIALES ========
\tikzset{
    error/.style={state, fill=red!20, draw=red!80!black},
    final/.style={state, accepting, fill=green!15!white, draw=green!60!black}
}
% Uso: \nodo[error]{qe}{5,0}{q_e}  o \nodo[final]{qf}{7,0}{q_f}


\newcommand{\pa}{1}      % ejemplo de valor
\newcommand{\pUno}{2}
\newcommand{\pDos}{3}
  % comandos LaTeX propios
% ===========================
% Diseño general
% ===========================
\setstretch{1.15} % interlineado
\setlength{\parskip}{0.5em} % espacio entre párrafos
\setlength{\parindent}{0pt} % sin sangría

% ===========================
% Estilo de capítulos y secciones (titlesec)
% ===========================
\titleformat{\chapter}[display]
  {\bfseries\Huge}
  {\filleft\Large\scshape Capítulo \thechapter}
  {1ex}
  {\titlerule[1pt]\vspace{1ex}\filright}
  [\vspace{1ex}\titlerule]

\titlespacing*{\chapter}{0pt}{0pt}{2em}

\titleformat{\section}
  {\Large\bfseries}
  {\thesection}{1em}{}

\titleformat{\subsection}
  {\large\bfseries}
  {\thesubsection}{1em}{}

\titleformat{\subsubsection}
  {\normalsize\bfseries\itshape}
  {\thesubsubsection}{1em}{}

% ===========================
% Encabezados y pies de página (fancyhdr)
% ===========================
\pagestyle{fancy}
\fancyhf{} % limpia
\fancyhead[L]{\small\scshape\nouppercase{\leftmark}} % sección/capítulo en mayúsculas pequeñas
\fancyhead[R]{\small\thepage}                        % número de página
%\fancyfoot[C]{\scriptsize\itshape Apuntes de la carrera} % texto fijo abajo en cursiva
% Encabezados y pies de página personalizados
% \fancyfoot[L]{\scriptsize\itshape Nombre de la asignatura} % pie de página izquierdo en cursiva
\fancyfoot[R]{\normalsize Ismael Sallami Moreno}        % pie de página derecho con el nombre del autor

% Línea bajo el encabezado
\renewcommand{\headrulewidth}{0.5pt} % línea más gruesa en el encabezado
% Línea en el pie
\renewcommand{\footrulewidth}{0.4pt} % línea fina en el pie
\renewcommand{\sectionmark}[1]{%
  \markboth{\thesection\quad #1}{}%
}

% ===========================
% Numeración de elementos
% ===========================
\numberwithin{equation}{chapter} % ecuaciones numeradas por capítulo
\numberwithin{figure}{chapter}   % figuras numeradas por capítulo
\numberwithin{table}{chapter}    % tablas numeradas por capítulo

% ===========================
% Listas y enumeraciones
% ===========================
\setlist[itemize]{label=--, left=1.5em}
\setlist[enumerate]{label=\arabic*), left=1.5em}

% ===========================
% Estilo de citas y bibliografía
% ===========================
\DefineBibliographyStrings{spanish}{%
  references = {Bibliografía},
}

% ===========================
% Entornos personalizados
% ===========================
\newtheoremstyle{cajita} % nombre del estilo
  {1em}   % espacio arriba
  {1em}   % espacio abajo
  {}      % fuente del cuerpo
  {}      % indentación
  {\bfseries} % fuente del título
  {.}     % puntuación tras título
  {0.5em} % espacio tras título
  {\thmname{#1}\thmnumber{ #2} \thmnote{(#3)}} % formato


\theoremstyle{cajita}
\newtheorem{teorema}{Teorema}[chapter]
\newtheorem{definicion}{Definición}[chapter]
\newtheorem{ejemplo}{Ejemplo}[chapter]
\newtheorem{proposicion}{Proposición}[chapter]
\newtheorem{demostracion}{Demostración}[chapter]
\newtheorem{corolario}{Corolario}[chapter]
\newtheorem{propuesta}{Propuesta}[chapter]


\newtheoremstyle{anotacionstyle} % nombre del estilo
  {1em}   % espacio arriba
  {1em}   % espacio abajo
  {}      % fuente del cuerpo (sin cursiva)
  {}      % indentación
  {\itshape} % fuente del título (Nota en cursiva)
  {.}     % puntuación tras título
  {0.5em} % espacio tras título
  {\thmname{\itshape#1}\thmnumber{ #2} \thmnote{(#3)}} % formato (solo Nota en cursiva)

\theoremstyle{anotacionstyle}
\newtheorem{anotacion}{Nota}[chapter]

% ===========================
% Configuración de lstlisting
% ===========================

% ===============================================
% ESTILO 1: MODERNO Y MINIMALISTA
% ===============================================

% Definir colores personalizados
\definecolor{codegreen}{rgb}{0,0.6,0}
\definecolor{codegray}{rgb}{0.5,0.5,0.5}
\definecolor{codepurple}{rgb}{0.58,0,0.82}
\definecolor{backcolour}{rgb}{0.95,0.95,0.92}
\definecolor{framecolor}{rgb}{0.8,0.8,0.8}

\lstset{
  backgroundcolor=\color{backcolour},   
  commentstyle=\color{codegreen},
  keywordstyle=\color{magenta},
  numberstyle=\tiny\color{codegray},
  stringstyle=\color{codepurple},
  basicstyle=\ttfamily\footnotesize,
  breakatwhitespace=false,         
  breaklines=true,                 
  captionpos=b,                    
  keepspaces=true,                 
  numbers=left,                    
  numbersep=5pt,                  
  showspaces=false,                
  showstringspaces=false,
  showtabs=false,                  
  tabsize=2,
  frame=shadowbox,
  frameround=tttt,
  rulecolor=\color{framecolor},
  rulesepcolor=\color{framecolor},
  xleftmargin=20pt,
  xrightmargin=20pt,
  aboveskip=20pt,
  belowskip=20pt,
  inputencoding=utf8,
  extendedchars=true,
  literate=
    {←}{{$\leftarrow$}}1
    {→}{{$\rightarrow$}}1
    {↑}{{$\uparrow$}}1
    {↓}{{$\downarrow$}}1
    {↔}{{$\leftrightarrow$}}1
    {⇒}{{$\Rightarrow$}}1
    {⇐}{{$\Leftarrow$}}1
    {⇔}{{$\Leftrightarrow$}}1
    {α}{{$\alpha$}}1
    {β}{{$\beta$}}1
    {γ}{{$\gamma$}}1
    {δ}{{$\delta$}}1
    {ε}{{$\epsilon$}}1
    {θ}{{$\theta$}}1
    {λ}{{$\lambda$}}1
    {μ}{{$\mu$}}1
    {π}{{$\pi$}}1
    {σ}{{$\sigma$}}1
    {φ}{{$\phi$}}1
    {ψ}{{$\psi$}}1
    {ω}{{$\omega$}}1
    {Δ}{{$\Delta$}}1
    {Θ}{{$\Theta$}}1
    {Λ}{{$\Lambda$}}1
    {Π}{{$\Pi$}}1
    {Σ}{{$\Sigma$}}1
    {Φ}{{$\Phi$}}1
    {Ψ}{{$\Psi$}}1
    {Ω}{{$\Omega$}}1
    {á}{{\'a}}1
    {é}{{\'e}}1
    {í}{{\'i}}1
    {ó}{{\'o}}1
    {ú}{{\'u}}1
    {Á}{{\'A}}1
    {É}{{\'E}}1
    {Í}{{\'I}}1
    {Ó}{{\'O}}1
    {Ú}{{\'U}}1
    {ä}{{\"a}}1
    {ë}{{\"e}}1
    {ï}{{\"i}}1
    {ö}{{\"o}}1
    {ü}{{\"u}}1
    {Ä}{{\"A}}1
    {Ë}{{\"E}}1
    {Ï}{{\"I}}1
    {Ö}{{\"O}}1
    {Ü}{{\"U}}1
    {ñ}{{\~n}}1
    {Ñ}{{\~N}}1
    {ç}{{\c{c}}}1
    {Ç}{{\c{C}}}1
    {¿}{{?`}}1
    {¡}{{!`}}1
    {à}{{\`a}}1
    {è}{{\`e}}1
    {ì}{{\`i}}1
    {ò}{{\`o}}1
    {ù}{{\`u}}1
    {À}{{\`A}}1
    {È}{{\`E}}1
    {Ì}{{\`I}}1
    {Ò}{{\`O}}1
    {Ù}{{\`U}}1
    {-}{{-}}1
    {=}{{=\allowbreak}}1  % <--- ESTA LÍNEA ES EL TRUCO PARA CORTAR LOS '===='
    % {#}{{\#}}1 
}


% ===============================================
% ESTILO 2: ELEGANTE CON BORDES REDONDEADOS
% ===============================================

% Colores para estilo elegante
\definecolor{lightblue}{rgb}{0.93,0.95,1}
\definecolor{darkblue}{rgb}{0.1,0.2,0.5}
\definecolor{mediumblue}{rgb}{0.2,0.4,0.8}
\definecolor{darkgreen}{rgb}{0,0.5,0}
\definecolor{darkred}{rgb}{0.6,0,0}

\lstdefinestyle{elegant}{
    backgroundcolor=\color{lightblue},
    commentstyle=\color{darkgreen}\itshape,
    keywordstyle=\color{darkblue}\bfseries,
    numberstyle=\tiny\color{gray},
    stringstyle=\color{darkred},
    basicstyle=\ttfamily\small,
    breakatwhitespace=false,
    breaklines=true,
    captionpos=t,
    keepspaces=true,
    numbers=left,
    numbersep=8pt,
    showspaces=false,
    showstringspaces=false,
    showtabs=false,
    tabsize=4,
    frame=single,
    frameround=tttt,
    framesep=10pt,
    xleftmargin=15pt,
    xrightmargin=15pt,
    aboveskip=15pt,
    belowskip=15pt,
    columns=flexible
}

% ===============================================
% ESTILO 3: PROFESIONAL CORPORATIVO
% ===============================================

% Colores corporativos
\definecolor{corporatebg}{rgb}{0.98,0.98,0.98}
\definecolor{corporateblue}{rgb}{0.07,0.29,0.49}
\definecolor{corporategray}{rgb}{0.4,0.4,0.4}
\definecolor{corporategreen}{rgb}{0.13,0.55,0.13}
\definecolor{corporatered}{rgb}{0.8,0.2,0.2}

\lstdefinestyle{corporate}{
    backgroundcolor=\color{corporatebg},
    commentstyle=\color{corporategreen}\slshape,
    keywordstyle=\color{corporateblue}\bfseries,
    numberstyle=\scriptsize\color{corporategray},
    stringstyle=\color{corporatered},
    basicstyle=\ttfamily\footnotesize,
    breakatwhitespace=false,
    breaklines=true,
    captionpos=b,
    keepspaces=true,
    numbers=left,
    numbersep=12pt,
    showspaces=false,
    showstringspaces=false,
    showtabs=false,
    tabsize=3,
    frame=leftline,
    framerule=3pt,
    rulecolor=\color{corporateblue},
    xleftmargin=25pt,
    aboveskip=20pt,
    belowskip=20pt,
    lineskip=1pt
}

% ===============================================
% ESTILO 4: MODERNO CON SOMBRAS
% ===============================================

% Colores modernos
\definecolor{modernbg}{rgb}{0.97,0.97,0.97}
\definecolor{moderngray}{rgb}{0.3,0.3,0.3}
\definecolor{modernpurple}{rgb}{0.5,0.2,0.8}
\definecolor{modernteal}{rgb}{0,0.5,0.5}
\definecolor{modernorange}{rgb}{0.8,0.4,0}

\lstdefinestyle{modern}{
    backgroundcolor=\color{modernbg},
    commentstyle=\color{modernteal}\itshape,
    keywordstyle=\color{modernpurple}\bfseries,
    numberstyle=\tiny\color{moderngray},
    stringstyle=\color{modernorange},
    basicstyle=\ttfamily\small,
    breakatwhitespace=false,
    breaklines=true,
    captionpos=t,
    keepspaces=true,
    numbers=left,
    numbersep=10pt,
    showspaces=false,
    showstringspaces=false,
    showtabs=false,
    tabsize=4,
    frame=tb,
    framerule=2pt,
    rulecolor=\color{modernpurple},
    xleftmargin=20pt,
    xrightmargin=20pt,
    aboveskip=25pt,
    belowskip=25pt
}

% ===============================================
% CONFIGURACIÓN PARA DIFERENTES LENGUAJES
% ===============================================

% Python
\lstdefinestyle{python}{
    language=Python,
    style=elegant,
    morekeywords={True,False,None,self,cls,def,class,import,from,as,with,yield,async,await},
    morecomment=[l]{\#},
    morestring=[b]',
    morestring=[b]"
}

% Java
\lstdefinestyle{java}{
    language=Java,
    style=corporate,
    morekeywords={var,record,sealed,permits,non-sealed}
}

% C++
\lstdefinestyle{cpp}{
    language=C++,
    style=modern,
    morekeywords={constexpr,nullptr,auto,decltype,override,final}
}

% JavaScript
\lstdefinestyle{javascript}{
    language=Java,
    style=elegant,
    morekeywords={let,const,var,function,class,extends,import,export,default,async,await,yield},
    morecomment=[l]{//},
    morecomment=[s]{/*}{*/},
    morestring=[b]',
    morestring=[b]",
    morestring=[b]`
}

% ===============================================
% EJEMPLOS DE USO
% ===============================================

% Para usar el estilo por defecto:
% \begin{lstlisting}
% código aquí
% \end{lstlisting}

% Para usar un estilo específico:
% \begin{lstlisting}[style=elegant]
% código aquí
% \end{lstlisting}

% Para incluir un archivo con estilo específico:
% \lstinputlisting[style=python]{archivo.py}

% Para código inline:
% \lstinline[style=modern]{código inline}

% ===============================================
% CONFIGURACIÓN ADICIONAL PARA TÍTULOS Y CARACTERES
% ===============================================

% Personalizar el formato de los títulos de los listados
\renewcommand\lstlistingname{Código}
\renewcommand\lstlistlistingname{Lista de Códigos}

% Configurar el formato del título con soporte para tildes
\lstset{
    %title=\lstname,
    captionpos=t,
    abovecaptionskip=10pt,
    belowcaptionskip=5pt,
    % Configuración global para caracteres especiales
    inputencoding=utf8,
    extendedchars=true
}

% ===============================================
% COMANDOS PERSONALIZADOS ÚTILES
% ===============================================

% Comando para código inline con soporte automático de tildes
\newcommand{\codeinline}[2][modern]{\lstinline[style=#1,inputencoding=utf8,extendedchars=true]{#2}}

% Comando para bloques de código con título personalizado
\newcommand{\codeblock}[3][elegant]{%
    \begin{lstlisting}[style=#1,caption={#2},label={lst:#2},inputencoding=utf8,extendedchars=true]
    #3
    \end{lstlisting}
}

% Comando para incluir archivos con configuración automática
\newcommand{\includecode}[3][python]{%
    \lstinputlisting[style=#1,caption={#3},label={lst:#3},inputencoding=utf8,extendedchars=true]{#2}
}

% ===============================================
% CONFIGURACIONES ESPECIALES PARA IDIOMAS
% ===============================================

% Configuración específica para código en español
\lstdefinestyle{español}{
    style=elegant,
    inputencoding=utf8,
    extendedchars=true,
    % Palabras clave en español para pseudocódigo
    morekeywords={función,procedimiento,inicio,fin,si,entonces,sino,mientras,para,hasta,hacer,repetir,caso,segun,verdadero,falso,entero,real,caracter,cadena,booleano,leer,escribir,imprimir}
}

% Configuración para comentarios multilíngües
\lstset{
    morecomment=[l]{//\ },
    morecomment=[l]{\#\ },
    morecomment=[s]{/*}{*/},
    morecomment=[s]{}
}

% ===============================================
% CONFIGURACIÓN PARA DIFERENTES LENGUAJES
% ===============================================

% Python
\lstdefinestyle{style1}{
    language=Python,
    style=elegant,
    morekeywords={True,False,None,self,cls,def,class,import,from,as,with,yield,async,await},
    morecomment=[l]{\#},
    morestring=[b]',
    morestring=[b]",
    % Soporte para caracteres especiales
    inputencoding=utf8,
    extendedchars=true
}

% Java
\lstdefinestyle{style2}{
    language=Java,
    style=corporate,
    morekeywords={var,record,sealed,permits,non-sealed},
    % Soporte para caracteres especiales
    inputencoding=utf8,
    extendedchars=true
}

% C++
\lstdefinestyle{style3}{
    language=C++,
    style=modern,
    morekeywords={constexpr,nullptr,auto,decltype,override,final},
    % Soporte para caracteres especiales
    inputencoding=utf8,
    extendedchars=true
}

\lstdefinelanguage{GDScript}{
  keywords={func, var, extends, class_name, if, else, for, while, return, match, in, and, or, not, break, continue, pass},
  sensitive=true,
  morecomment=[l]{\#},
  morestring=[b]",
  morestring=[b]',
}

\lstdefinestyle{gdstyle}{
  language=GDScript,
  basicstyle=\ttfamily\small,
  keywordstyle=\color{blue}\bfseries,
  commentstyle=\color{gray},
  stringstyle=\color{red!60!black},
  numbers=left,
  numberstyle=\tiny\color{gray},
  breaklines=true,
  frame=single,
  tabsize=2,
}


% ===========================
% Estilo global de tablas
% ===========================

\usepackage{booktabs}   % reglas profesionales
\usepackage{colortbl}   % color en filas
\usepackage{xcolor}     % colores
\usepackage{float}      % [H]

% Color de filas alternadas
% \rowcolors{2}{gray!10}{white}

% % Espacio vertical entre filas
% \renewcommand{\arraystretch}{1.2}

% % Cambiar el tamaño de columna por defecto
% \setlength{\tabcolsep}{8pt}

% % Redefinir tabla para que todas las tablas tengan el estilo
% \let\oldtabular\tabular
% \let\endoldtabular\endtabular
% \renewenvironment{tabular}[1]{%
%   \oldtabular{#1}%
% }{%
%   \endoldtabular
% }

% \usepackage{longtable,booktabs,xcolor}
% \rowcolors{2}{gray!10}{white}   % filas alternadas
% \renewcommand{\arraystretch}{1.2} % espacio vertical entre filas

% % Mostrar siempre el número de la tabla
% \usepackage{caption}
% \captionsetup[table]{labelformat=default, labelsep=colon, textfont=bf}


% ===========================
% Estilos para tikz y figures
% ===========================

\usepackage{caption}
\captionsetup{
    font={it},       % fuente en cursiva
    labelfont={},  % etiqueta ("Figura 1") en negrita
    textfont={it},   % texto del caption en cursiva
    justification=centering,  % centra el texto (opcional)
    font={small},    % tamaño de fuente pequeño
}

\usepackage{tikz}
\usetikzlibrary{positioning}

\tikzset{
  state/.style={
    draw,
    circle,
    minimum size=1cm,
    thick,
    fill=yellow!20
  },
  block/.style={
    rectangle,
    draw,
    fill=blue!10,
    rounded corners,
    text centered,
    minimum height=1cm,
    minimum width=2cm,
    thick
  },
  none/.style={
    draw=none,
    fill=none,
    text centered
  },
  error/.style={
    draw,
    circle,
    minimum size=1cm,
    thick,
    fill=red!30
  },
  initial text={}
}   % estilos de secciones, etc.

% ========================
% Configuración índice y listas
% ========================
\setlength{\cftbeforesecskip}{5pt}
\setlength{\headheight}{14pt}  % un poco más que 13.6pt

\renewcommand{\normalsize}{\fontsize{10}{12}\selectfont}

% Fix para listas de Pandoc
\providecommand{\tightlist}{%
  \setlength{\itemsep}{0pt}\setlength{\parskip}{0pt}}


%=======================
% fancy with parameters
%=======================
%\fancyfoot[L]{\scriptsize\itshape Modelos Computacionales}
\fancyfoot[L]{\scriptsize\itshape Modelos
Computacionales} % pie de página izquierdo en cursiva



% ========================
% Inicio del documento
% ========================
\begin{document}

%% portada.tex
\begin{titlepage}
    \newgeometry{top=2cm,bottom=2cm,left=2.5cm,right=2.5cm} % márgenes personalizados
    
    % Fondo con transparencia
    \begin{tikzpicture}[remember picture,overlay]
        \node[opacity=0.15,inner sep=0pt] at (current page.center)
            {\includegraphics[width=\paperwidth,height=\paperheight]{../../img/fondoPrueba.jpg}};
    \end{tikzpicture}

    % Contenido de la portada
    \begin{center}
        \vspace*{2cm}
        
        {\Huge \bfseries\scshape Título del Libro de Apuntes \par}
        \vspace{0.5cm}
        {\Large \itshape Subtítulo o Asignatura \par}
        \vspace{0.5cm}
        {\Large \itshape \href{https://ismael-sallami.github.io}{https://ismael-sallami.github.io} \par}


        \vfill
        
        {\LARGE Autor: \textbf{Tu Nombre Completo} \par}
        \vspace{0.3cm}
        % {\Large Universidad Ejemplo \par}
        
        \vspace{1cm}
        \includegraphics[width=0.25\textwidth]{../../img/ugr.png} % opcional: logo
        \vspace{1cm}
        
        {\large \today}
    \end{center}
    
    \restoregeometry
\end{titlepage}



%==========================
% PORTADA: ENTRADA MANUAL
%==========================

% portada.tex
\begin{titlepage}
    \newgeometry{top=2cm,bottom=2cm,left=2.5cm,right=2.5cm} % márgenes personalizados
    
    % Fondo con transparencia
    \begin{tikzpicture}[remember picture,overlay]
        \node[opacity=0.15,inner sep=0pt] at (current page.center)
            {\includegraphics[width=\paperwidth,height=\paperheight]{../../../extraFiles/img/fondo_info.jpg}};
    \end{tikzpicture}

    % Contenido de la portada
    \begin{center}
        \vspace*{2cm}
        
        {\Huge \bfseries\scshape Teoría y Práctica \par}
        \vspace{0.5cm}
        {\Large \itshape Modelos Computacionales \par}
        \vspace{0.5cm}
        {\Large \itshape \href{https://ismael-sallami.github.io}{https://ismael-sallami.github.io} \par}


        \vfill
        
        {\LARGE Autor: \textbf{Ismael Sallami Moreno} \par}
        \vspace{0.3cm}
        % {\Large Universidad de Granada \par}
        
        \vspace{1cm}
        \includegraphics[width=0.25\textwidth]{../../../extraFiles/img/ugr.png} % opcional: logo
        \vspace{1cm}
        
        {\large \today}
    \end{center}
    
    \restoregeometry
\end{titlepage}


% ===============================
% licencia.tex
% ===============================
\begin{tikzpicture}[remember picture,overlay]
\node[anchor=south west, xshift=1cm, yshift=1cm] at (current page.south west) {
\begin{minipage}{0.4\textwidth}
\begin{flushleft}
\section*{Licencia}

Este trabajo está bajo una 
\href{https://creativecommons.org/licenses/by-nc-nd/4.0/}{Licencia Creative Commons BY-NC-ND 4.0}.

\bigskip

Permisos: Se permite compartir, copiar y redistribuir el material en cualquier medio o formato.

\bigskip

Condiciones: Es necesario dar crédito adecuado, proporcionar un enlace a la licencia e indicar si se han realizado cambios. No se permite usar el material con fines comerciales ni distribuir material modificado.

\bigskip

\begin{center}
  \href{https://creativecommons.org/licenses/by-nc-nd/4.0/}{\includegraphics[width=0.35\textwidth]{../../../extraFiles/img/by-nc-nd.png}}
\end{center}
\end{flushleft}
\end{minipage}
};
\end{tikzpicture}
  % licencia
\thispagestyle{empty} % quitar número de página en la portada
\clearpage

% --- Índice ---
\tableofcontents
\thispagestyle{empty} % quitar número de página en la portada
\clearpage

% --- Contenido Markdown generado por Pandoc ---
\part{Teoría}

\hypertarget{introducciuxf3n}{%
\chapter{Introducción}\label{introducciuxf3n}}

\begin{itemize}
\tightlist
\item
  Profesor: Serafín Moral\\
\item
  Correo: smc@decsai.ugr.es\\
\item
  El profesor recomienda ir a las tutorías y avisarle antes.\\
\item
  J.E. Hopcroft, J.D. Ullman, Introduction to Automata Theory, Languages
  and Computation. Addison-Wesley (1979): un libro básico, se requiere
  ciertos conocimientos matemáticos para leerlo.\\
\item
  M. Alfonseca, J. Sancho. M. Martínez, Teoría de Autómatas y Lenguajes
  Formales. Publicaciones R.A.E.C., Textos Cátedra (1997): básico y
  fácil de entender, está en la biblioteca en español.\\
\item
  Los demás son una buena opción también.
\end{itemize}

La asignatura de Modelos de Computación se centra en el estudio de los
fundamentos teóricos de la informática, explorando conceptos como
autómatas, lenguajes formales, gramáticas y máquinas de Turing. Estos
temas proporcionan las bases para comprender las capacidades y
limitaciones de los sistemas computacionales, así como para analizar la
complejidad de los problemas y los algoritmos que los resuelven. Es una
materia esencial para quienes deseen profundizar en la teoría de la
computación y su aplicación en el diseño de sistemas eficientes y
correctos.

\hypertarget{introducciuxf3n-a-la-computaciuxf3n}{%
\chapter{Introducción a la
Computación}\label{introducciuxf3n-a-la-computaciuxf3n}}

En el pasado, había teoremas que eran verdad, pero las matemáticas no
eran capaces de demostrarlo. Turing puso solución a esto exponiendo que
esta incompletitud de las matemáticas se corregía diciendo que este tipo
de problemas eran indecidibles. En cuanto a la complejidad de los
algoritmos, como vimos en algorítmica, podemos distinguir entre p
(polinómico) y np (polinómico no determinista). Dentro de np,
encontramos los np completos y los np difíciles (problemas que no puede
resolver un ordenador convencional). Hoy día lo más importante es la
complejidad algorítmica. Nadie ha demostrado que p sea distinto de np.

\hypertarget{problema-de-la-parada}{%
\section{Problema de la parada}\label{problema-de-la-parada}}

Trata el tema de la existencia de que un programa que lea otro programa
y unos datos y nos diga si este termina o cicla indefinidamente. Un
programa es lo mismo que los datos, según Turing. En este caso, se pone
como datos del programa el mismo programa.

\begin{lstlisting}[language=Python]
If Stops(P, P) GOTO L
\end{lstlisting}

Este programa lleva a una contradicción, ya que si termina no termina y
si termina termina, cosa que no es coherente. Por ende, podemos concluir
que si un programa se llama así mismo, en este caso, llegamos a una
contradicción. Turing, con esto, llegó a la conclusión de que las
matemáticas eran incompletas, ya que este programa no existe. La
explicación es sencilla, basta con decir que es como un espejo, nunca va
a pasar STOP ya que si eso pasa el programa nunca arrancaría.

\hypertarget{definiciones}{%
\section{Definiciones}\label{definiciones}}

\begin{definicion}
\textbf{Alfabeto}: Un alfabeto es un conjunto finito cuyos elementos se denominan símbolos o letras. Si los símbolos tienen varios elementos, se representan entre $\langle \rangle$.  
\end{definicion}

\begin{definicion}
\textbf{Palabra}: Una palabra es una sucesión finita de elementos de un alfabeto $A$. Formalmente, $u = a_1 a_2 \ldots a_n$, donde $a_i \in A$ para todo $i = 1, \ldots, n$.  
\end{definicion}

\begin{definicion}
\textbf{Conjunto de Palabras}: El conjunto de todas las palabras que se pueden formar sobre un alfabeto $A$ se denota como $A^*$.  
\end{definicion}

\begin{definicion}
\textbf{Notación de Palabras}: Las palabras se denotan comúnmente como $u, v, x, y, z, \ldots$.  
\end{definicion}

\begin{definicion}
\textbf{Longitud de una Palabra}: Si $u \in A^*$, la longitud de la palabra $u$ es el número de símbolos de $A$ que contiene.  
- Notación: $\lvert u \rvert$  
- Si $u = a_1 a_2 \ldots a_n$, entonces $\lvert u \rvert = n$.  
\end{definicion}

\begin{definicion}
\textbf{Palabra Vacía}: La palabra vacía es la palabra de longitud cero.  
\begin{itemize}
    \item Notación: $\varepsilon$  
\end{itemize}
\end{definicion}

\begin{definicion}
\textbf{Conjunto de Palabras no Vacías}: El conjunto de cadenas sobre un alfabeto $A$ excluyendo la palabra vacía se denota como $A^+$.  
\end{definicion}

\hypertarget{operaciones-concatenaciuxf3n}{%
\section{Operaciones:
Concatenación}\label{operaciones-concatenaciuxf3n}}

\begin{definicion}
\textbf{Concatenación de Palabras}: Si $u, v \in A^*$, $u = a_1 \ldots a_n$, $v = b_1 \ldots b_m$, se llama concatenación de $u$ y $v$ a la cadena $u.v$ (o simplemente $uv$) dada por $a_1 \ldots a_n b_1 \ldots b_m$.
\end{definicion}

\begin{ejemplo}
Si $u = 011$, $v = 1010$, entonces $uv = 0111010$.
\end{ejemplo}

\textbf{Propiedades}\\
1. \(\lvert u.v \rvert = \lvert u \rvert + \lvert v \rvert\),
\(\forall u, v \in A^*\)\\
2. \textbf{Asociativa}: \(u.(v.w) = (u.v).w\),
\(\forall u, v, w \in A^*\)\\
3. \textbf{Elemento Neutro}: \(u.\varepsilon = \varepsilon.u = u\),
\(\forall u \in A^*\)

\textbf{Estructura de monoide}\\
La concatenación de palabras sobre un alfabeto \(A\) junto con la
palabra vacía \(\varepsilon\) forma un monoide.

\begin{definicion}
\textbf{Monoide}: Un monoide es una estructura algebraica $(M, \cdot, e)$ que consta de un conjunto $M$, una operación binaria asociativa $\cdot : M \times M \to M$, y un elemento neutro $e \in M$ tal que:  
1. **Asociatividad**: $(a \cdot b) \cdot c = a \cdot (b \cdot c)$, $\forall a, b, c \in M$.  
2. **Elemento Neutro**: $a \cdot e = e \cdot a = a$, $\forall a \in M$.  
\end{definicion}

\hypertarget{prefijos-sufijos-y-subcadenas}{%
\section{Prefijos, Sufijos y
subcadenas}\label{prefijos-sufijos-y-subcadenas}}

\begin{definicion}
\textbf{Prefijo}: Si $u \in A^*$, entonces $v$ es un prefijo de $u$ si $\exists z \in A^*$ tal que $vz = u$.  
Un prefijo $v$ de $u$ se dice propio si $v \neq \varepsilon$ y $v \neq u$.  
\end{definicion}

\begin{definicion}
\textbf{Sufijo}: Si $u \in A^*$, entonces $v$ es un sufijo de $u$ si $\exists z \in A^*$ tal que $zv = u$.  
Un sufijo $v$ de $u$ se dice propio si $v \neq \varepsilon$ y $v \neq u$.  
\end{definicion}

\begin{definicion}
\textbf{Subcadena}: Si $u \in A^*$, entonces $v$ es una subcadena de $u$ si $\exists z_1, z_2 \in A^*$ tal que $z_1vz_2 = u$.  
Una subcadena $v$ de $u$ se dice propia si $v \neq \varepsilon$ y $v \neq u$.  
\end{definicion}

\hypertarget{iteraciuxf3n-y-palabra-inversa}{%
\section{Iteración y Palabra
Inversa}\label{iteraciuxf3n-y-palabra-inversa}}

\begin{definicion}
\textbf{Iteración de una Cadena}: La iteración n-ésima de una cadena ($u^n$) se define como la concatenación de la cadena consigo misma $n$ veces.  
Si $u \in A^*$, entonces:  
\begin{itemize}
    \item $u^0 = \varepsilon$  
    \item $u^{i+1} = u^i.u$, $\forall i \geq 0$  
\end{itemize}
\end{definicion}

\begin{ejemplo}
Si $u = 010$, entonces $u^3 = 010010010$.
\end{ejemplo}

\begin{definicion}
\textbf{Palabra Inversa}: Si $u = a_1 \ldots a_n \in A^*$, entonces la palabra inversa de $u$ es la cadena $u^{-1} = a_n \ldots a_1 \in A^*$.  
\end{definicion}

\begin{ejemplo}
Si $u = 011$, entonces la palabra inversa de $u$ es $u^{-1} = 110$.
\end{ejemplo}

\hypertarget{lenguajes}{%
\section{Lenguajes}\label{lenguajes}}

\begin{definicion}
\textbf{Lenguaje}: Un lenguaje sobre un alfabeto $A$ es un subconjunto del conjunto de las cadenas sobre $A$, es decir, $L \subseteq A^*$.  
La palabra vacía siempre pertenece a un lenguaje.
\end{definicion}

\textbf{Notación}\\
- Lenguajes: \(L, M, N, \ldots\)

\begin{ejemplo}
    \begin{enumerate}
        \item $L_1 = \{a, b, \varepsilon\}$  
        \item $L_2 = \{a^i b^i \mid i = 0, 1, 2, \ldots\}$  
        \item $L_3 = \{u u^{-1} \mid u \in A^*\}$  
        \item $L_4 = \{a^{n^2} \mid n = 1, 2, 3, \ldots\}$  
    \end{enumerate}
\end{ejemplo}

\hypertarget{ejemplos-adicionales-de-lenguajes}{%
\subsection{Ejemplos Adicionales de
Lenguajes}\label{ejemplos-adicionales-de-lenguajes}}

\begin{enumerate}
\def\labelenumi{\arabic{enumi}.}
\tightlist
\item
  \(L_5 = \{a^n b^m c^k \mid n, m, k \geq 0\}\): Palabras con cualquier
  número de \(a\), \(b\), y \(c\) en ese orden.\\
\item
  \(L_6 = \{w \in \{0, 1\}^* \mid w \text{ tiene un número par de } 1\}\):
  Palabras binarias con un número par de unos.\\
\item
  \(L_7 = \{a^n b^n c^n \mid n \geq 0\}\): Palabras con el mismo número
  de \(a\), \(b\), y \(c\) en ese orden.\\
\item
  \(L_8 = \{w \in \{0, 1\}^* \mid w \text{ es un palíndromo}\}\):
  Palabras binarias que son palíndromos.\\
\item
  \(L_9 = \{a^{2^n} \mid n \geq 0\}\): Sucesiones de \(a\) cuya longitud
  es una potencia de dos.
\end{enumerate}

\hypertarget{conjuntos-numerables}{%
\subsection{Conjuntos Numerables}\label{conjuntos-numerables}}

Un conjunto se dice numerable si existe una aplicación inyectiva
(corresponde cada elemento con su imagen) de este conjunto en el
conjunto de los números naturales, o lo que es lo mismo, se le puede
asignar un número natural a cada elemento del conjunto de tal manera que
dos elementos distintos tengan números distintos.

\textbf{Ejemplos}

\begin{ejemplo}
$A^*$ es siempre numerable. Si $A = \{a_1, \ldots, a_n\}$, entonces puedo asignar un número binario distinto de 0 y de la misma longitud a cada $a_i$ de tal manera que símbolos distintos reciben números distintos, y a cada palabra $b_1 \ldots b_k$ se le asigna el número cuya representación en binario es el que se obtiene sustituyendo cada $b_i$ por su número binario.  
    - **Ejemplo**: Si $A = \{a, b\}$, podemos asignar $a = 01$, $b = 10$. Entonces, para la palabra $ab$, su representación binaria sería $0110$.  
\end{ejemplo}

\begin{ejemplo}
El conjunto de programas bien escritos en C es numerable. Esto se debe a que los programas son cadenas finitas de un alfabeto finito, y por lo tanto, se pueden enumerar de manera similar a las palabras en $A^*$.  
\end{ejemplo}

El hecho de que en el ordenador se trabaje con float, double, \ldots{}
es porque como los números reales son conjuntos no numerables, un solo
número real acabaría con toda la memoria del ordenador. Lo mismo pasa en
los lenguajes, debemos de restringirnos a los lenguajes con los que
podamos trabajar. Solo existe un conjunto numerable de programas.

\hypertarget{un-conjunto-no-numerable}{%
\section{Un Conjunto No Numerable}\label{un-conjunto-no-numerable}}

\begin{ejemplo}
El conjunto de lenguajes sobre $A^*$ (si $A$ no es vacío) nunca es numerable.  
\end{ejemplo}

Haremos la demostración por reducción al absurdo.\\
Si lo fuese, se podría asignar un número natural distinto \(f(L)\) a
cada lenguaje \(L\).

Sea \(a \in A\).\\
Definamos el lenguaje \(L\) formado por palabras de la forma \(a^i\) de
acuerdo a lo siguiente: para cada \(i\) número natural:\\
- Si este número no es de un lenguaje, entonces \(a^i \in L\).\\
- Si este número es del lenguaje \(M\) (\(i = f(M)\)):\\
- Si \(a^i \notin M\), entonces \(a^i \in L\).\\
- Si \(a^i \in M\), entonces \(a^i \notin L\).

\(L\) no puede tener ningún número asociado. Si fuese \(j = f(L)\),
entonces la pertenencia de \(a^j\) a \(L\) es contradictoria:\\
- Si \(a^j \in L\) como \(j = f(L)\), entonces \(a^j \notin L\).\\
- Si \(a^j \notin L\) y \(j = f(L)\), entonces \(a^j \in L\).

Por lo tanto, el conjunto de lenguajes sobre \(A^*\) no es numerable.

\hypertarget{operaciones-con-lenguajes-concatenaciuxf3n}{%
\section{Operaciones con Lenguajes:
Concatenación}\label{operaciones-con-lenguajes-concatenaciuxf3n}}

Dada su condición de conjuntos, además de las operaciones de unión,
intersección y complementario, los lenguajes también admiten la
operación de concatenación.

Si \(L_1, L_2\) son dos lenguajes sobre el alfabeto \(A\), la
concatenación de estos dos lenguajes se define como:

\[
L_1L_2 = \{u_1u_2 \mid u_1 \in L_1, u_2 \in L_2\}
\]

\textbf{Ejemplo}\\
Si \(L_1 = \{0^i1^i \mid i \geq 0\}\) y
\(L_2 = \{1^j0^j \mid j \geq 0\}\), entonces:

\[
L_1L_2 = \{0^i1^i1^j0^j \mid i, j \geq 0\}
\]

\hypertarget{propiedades-de-la-concatenaciuxf3n-de-lenguajes}{%
\section{Propiedades de la Concatenación de
Lenguajes}\label{propiedades-de-la-concatenaciuxf3n-de-lenguajes}}

\begin{enumerate}
\def\labelenumi{\arabic{enumi}.}
\item
  \textbf{Propiedad de Aniquilación}\\
  \[L \emptyset = \emptyset L = \emptyset\]
\item
  \textbf{Elemento Neutro}\\
  \[\{\varepsilon\}L = L\{\varepsilon\} = L\]
\item
  \textbf{Asociatividad}\\
  \[L_1(L_2L_3) = (L_1L_2)L_3\]
\end{enumerate}

\hypertarget{iteraciuxf3n-de-lenguajes-y-clausura-de-kleene}{%
\section{Iteración de Lenguajes y Clausura de
Kleene}\label{iteraciuxf3n-de-lenguajes-y-clausura-de-kleene}}

La iteración de lenguajes se define de forma recursiva:\\
- \(L^0 = \{\varepsilon\}\)\\
- \(L^{i+1} = L^iL\), \(\forall i \geq 0\)

Si \(L\) es un lenguaje sobre el alfabeto \(A\), se definen las
siguientes clausuras:\\
- \textbf{Clausura de Kleene}:\\
\[
    L^* = \bigcup_{i \geq 0} L^i
    \]

\begin{itemize}
\tightlist
\item
  \textbf{Clausura Positiva}:\\
  \[
    L^+ = \bigcup_{i \geq 1} L^i
    \]
\end{itemize}

\textbf{Ejemplo}\\
Si \(L = \{a, b\}\):\\
- \(L^0 = \{\varepsilon\}\)\\
- \(L^1 = \{a, b\}\)\\
- \(L^2 = \{aa, ab, ba, bb\}\)\\
- \(L^* = \{\varepsilon, a, b, aa, ab, ba, bb, \ldots\}\)\\
- \(L^+ = \{a, b, aa, ab, ba, bb, \ldots\}\)

\hypertarget{operaciones-con-lenguajes-propiedades-de-clausuras}{%
\section{Operaciones con Lenguajes: Propiedades de
Clausuras}\label{operaciones-con-lenguajes-propiedades-de-clausuras}}

\begin{enumerate}
\def\labelenumi{\arabic{enumi}.}
\tightlist
\item
  \textbf{Relación entre Clausura de Kleene y Clausura Positiva}

  \begin{itemize}
  \tightlist
  \item
    Si \(\varepsilon \in L\), entonces \(L^+ = L^*\).\\
  \item
    Si \(\varepsilon \notin L\), entonces
    \(L^+ = L^* \setminus \{\varepsilon\}\).
  \end{itemize}
\end{enumerate}

\begin{ejemplo}
Si $L = \{0, 01\}$:  
    \begin{itemize}
        \item $L^* =$ Conjunto de palabras sobre $\{0, 1\}$ en las que un $1$ siempre va precedido de un $0$.  
        \item $L^+ =$ Conjunto de palabras sobre $\{0, 1\}$ en las que un $1$ siempre va precedido de un $0$ y distintas de la palabra vacía.  
    \end{itemize}
\end{ejemplo}

\hypertarget{lenguaje-inverso}{%
\section{Lenguaje Inverso}\label{lenguaje-inverso}}

\begin{definicion}
\textbf{Lenguaje Inverso}: El lenguaje inverso de un lenguaje $L$ sobre un alfabeto $A$ se define como:

$$
L^{-1} = \{u \mid u^{-1} \in L\}
$$
\end{definicion}

\textbf{Ejemplo}\\
Si \(L = \{011, 101\}\), entonces:

\[
L^{-1} = \{110, 101\}
\]

\textbf{Propiedades}\\
1. \((L^{-1})^{-1} = L\)\\
2. \((L_1 \cup L_2)^{-1} = L_1^{-1} \cup L_2^{-1}\)\\
3. \((L_1L_2)^{-1} = L_2^{-1}L_1^{-1}\)\\
4. \((L^*)^{-1} = (L^{-1})^*\)

\hypertarget{cabecera-de-un-lenguaje}{%
\section{Cabecera de un Lenguaje}\label{cabecera-de-un-lenguaje}}

\begin{definicion}
\textbf{Cabecera de un Lenguaje}: La cabecera de un lenguaje $L$ sobre un alfabeto $A$ se define como:

$$
\text{CAB}(L) = \{u \mid u \in A^* \text{ y } \exists v \in A^* \text{ tal que } uv \in L\}
$$
\end{definicion}

\textbf{Ejemplo}\\
Si \(L = \{0^i1^i \mid i \geq 0\}\), entonces:

\[
\text{CAB}(L) = \{0^i1^j \mid i \geq j \geq 0\}
\]

\hypertarget{homomorfismos-entre-alfabetos}{%
\section{Homomorfismos entre
Alfabetos}\label{homomorfismos-entre-alfabetos}}

\begin{definicion}
Si $A_1$ y $A_2$ son dos alfabetos, una aplicación $h : A_1^* \to A_2^*$ se dice que es un \textbf{homomorfismo} si y solo si:

$$
h(uv) = h(u)h(v), \quad \forall u, v \in A_1^*
$$
\end{definicion}

\hypertarget{consecuencias-de-la-definiciuxf3n}{%
\subsection{Consecuencias de la
Definición}\label{consecuencias-de-la-definiciuxf3n}}

\begin{enumerate}
\def\labelenumi{\arabic{enumi}.}
\item
  \textbf{Imagen de la palabra vacía}\\
  \[
   h(\varepsilon) = \varepsilon
   \]
\item
  \textbf{Imagen de una palabra}\\
  Si \(u = a_1a_2 \ldots a_n \in A_1^*\), entonces: \[
   h(u) = h(a_1)h(a_2) \ldots h(a_n)
   \]
\end{enumerate}

\hypertarget{ejemplo-de-homomorfismo}{%
\subsection{Ejemplo de Homomorfismo}\label{ejemplo-de-homomorfismo}}

Sea \(A_1 = \{a, b\}\) y \(A_2 = \{0, 1\}\). Definimos
\(h : A_1^* \to A_2^*\) como:

\begin{itemize}
\tightlist
\item
  \(h(a) = 01\)
\item
  \(h(b) = 10\)
\end{itemize}

Entonces:

\begin{itemize}
\tightlist
\item
  \(h(\varepsilon) = \varepsilon\)
\item
  \(h(ab) = h(a)h(b) = 0110\)
\item
  \(h(aba) = h(a)h(b)h(a) = 010110\)
\end{itemize}

\hypertarget{propiedades-de-homomorfismos}{%
\subsection{Propiedades de
Homomorfismos}\label{propiedades-de-homomorfismos}}

\begin{enumerate}
\def\labelenumi{\arabic{enumi}.}
\item
  \textbf{Preservación de la Concatenación}\\
  \[
   h(u.v) = h(u)h(v), \quad \forall u, v \in A_1^*
   \]
\item
  \textbf{Homomorfismo Inverso}\\
  Si \(h : A_1^* \to A_2^*\) es un homomorfismo, entonces: \[
   h(u^{-1}) = h(u)^{-1}, \quad \forall u \in A_1^*
   \]
\item
  \textbf{Composición de Homomorfismos}\\
  Si \(h_1 : A_1^* \to A_2^*\) y \(h_2 : A_2^* \to A_3^*\) son
  homomorfismos, entonces su composición
  \(h_2 \circ h_1 : A_1^* \to A_3^*\) también es un homomorfismo.
\end{enumerate}

\textbf{Ejemplo de Composición}\\
Sea \(h_1 : A_1^* \to A_2^*\) y \(h_2 : A_2^* \to A_3^*\) definidos
como:

\begin{itemize}
\tightlist
\item
  \(h_1(a) = 01\), \(h_1(b) = 10\)
\item
  \(h_2(0) = x\), \(h_2(1) = y\)
\end{itemize}

Entonces, para \(u = ab \in A_1^*\):

\begin{itemize}
\tightlist
\item
  \(h_1(u) = 0110\)
\item
  \(h_2(h_1(u)) = h_2(0110) = xyxy\)
\end{itemize}

\hypertarget{gramuxe1tica-generativa}{%
\section{Gramática Generativa}\label{gramuxe1tica-generativa}}

\begin{definicion}
Una gramática generativa es una cuádrupla $(V, T, P, S)$ donde:

\begin{itemize}
    \item \textbf{$V$}: Es un alfabeto llamado de variables o símbolos no terminales. Sus elementos se suelen representar con letras mayúsculas.
    \item \textbf{$T$}: Es un alfabeto llamado de símbolos terminales. Sus elementos se suelen representar con letras minúsculas.
    \item \textbf{$P$}: Es un conjunto finito de pares $(\alpha, \beta)$, llamados reglas de producción, donde $\alpha, \beta \in (V \cup T)^*$ y $\alpha$ contiene al menos un símbolo de $V$.
        \begin{itemize}
            \item El par $(\alpha, \beta)$ se suele representar como $\alpha \to \beta$.
        \end{itemize}
    \item \textbf{$S$}: Es un elemento de $V$, llamado símbolo de partida.
\end{itemize}
\end{definicion}

Tiene la misma potencia que un lenguaje, por ende, podemos pensar que es
similar a un lenguaje de programación aunque pensemos que no.

\begin{ejemplo}
Sea la gramática $G = (V, T, P, S)$ definida como:  
    \begin{itemize}
        \item $V = \{S, A\}$  
        \item $T = \{a, b\}$  
        \item $P = \{S \to aA, A \to b\}$  
        \item $S = S$  
    \end{itemize}
\end{ejemplo}

Esta gramática genera el lenguaje \(L = \{ab\}\).

\hypertarget{lenguaje-generado-idea-intuitiva}{%
\section{Lenguaje Generado: Idea
Intuitiva}\label{lenguaje-generado-idea-intuitiva}}

Una gramática sirve para determinar un lenguaje. Las palabras generadas
pertenecen al conjunto de símbolos terminales \(T^*\) y se obtienen a
partir del símbolo inicial efectuando pasos de derivación. Cada paso
consiste en elegir una parte de la palabra que coincide con la parte
izquierda de una producción y sustituir esa parte por la derecha de la
misma producción.

\hypertarget{ejemplo}{%
\subsection{Ejemplo}\label{ejemplo}}

Dada la gramática:

\begin{itemize}
\tightlist
\item
  \(E \to E + E\)
\item
  \(E \to E * E\)
\item
  \(E \to (E)\)
\item
  \(E \to a\)
\item
  \(E \to b\)
\item
  \(E \to c\)
\end{itemize}

Derivación de una palabra:

\[
E \Rightarrow E * E \Rightarrow (E) * E \Rightarrow (E + E) * E \Rightarrow (a + E) * E \Rightarrow (a + b) * E \Rightarrow (a + b) * b
\]

\textbf{Palabra Generada}: \((a + b) * b\)

\hypertarget{derivaciuxf3n-y-lenguaje-generado}{%
\section{Derivación y Lenguaje
Generado}\label{derivaciuxf3n-y-lenguaje-generado}}

\hypertarget{derivaciuxf3n-en-un-paso}{%
\subsection{Derivación en un Paso}\label{derivaciuxf3n-en-un-paso}}

Dada una gramática \(G = (V, T, P, S)\) y dos palabras
\(\alpha, \beta \in (V \cup T)^*\), se dice que \(\beta\) es derivable a
partir de \(\alpha\) en un paso (\(\alpha \Rightarrow \beta\)) si y solo
si existe una producción \(\gamma \to \phi\) tal que \(\alpha\) contiene
a \(\gamma\) como subcadena y \(\beta\) se obtiene sustituyendo
\(\gamma\) por \(\phi\) en \(\alpha\).

\hypertarget{secuencia-de-derivaciuxf3n}{%
\subsection{Secuencia de Derivación}\label{secuencia-de-derivaciuxf3n}}

Se dice que \(\beta\) es derivable de \(\alpha\)
(\(\alpha \overset{*}{\Rightarrow} \beta\)) si y solo si existe una
sucesión de palabras \(\gamma_1, \ldots, \gamma_n\) (\(n \geq 1\)) tales
que:

\[
\alpha = \gamma_1 \Rightarrow \gamma_2 \Rightarrow \ldots \Rightarrow \gamma_n = \beta
\]

\hypertarget{lenguaje-generado-por-una-gramuxe1tica}{%
\subsection{Lenguaje Generado por una
Gramática}\label{lenguaje-generado-por-una-gramuxe1tica}}

El lenguaje generado por una gramática \(G = (V, T, P, S)\) es el
conjunto de cadenas formadas por símbolos terminales que son derivables
a partir del símbolo de partida \(S\). Formalmente:

\[
L(G) = \{u \in T^* \mid S \overset{*}{\Rightarrow} u\}
\]

\hypertarget{gramuxe1tica-generativa-ejemplo-y-propiedades}{%
\section{Gramática Generativa: Ejemplo y
Propiedades}\label{gramuxe1tica-generativa-ejemplo-y-propiedades}}

\hypertarget{gramuxe1tica-definida}{%
\subsection{Gramática Definida}\label{gramuxe1tica-definida}}

Sea la gramática \(G = (V, T, P, S)\) definida como:\\
- \(V = \{S, A, B\}\)\\
- \(T = \{a, b\}\)\\
-
\(P = \{S \to aB, S \to bA, A \to a, A \to aS, A \to bAA, B \to b, B \to bS, B \to aBB\}\)\\
- \(S = S\)

\hypertarget{lenguaje-generado}{%
\subsection{Lenguaje Generado}\label{lenguaje-generado}}

Esta gramática genera el lenguaje:

\[
L(G) = \{u \mid u \in \{a, b\}^+ \text{ y } N_a(u) = N_b(u)\}
\]

donde \(N_a(u)\) y \(N_b(u)\) son el número de apariciones de los
símbolos \(a\) y \(b\) en \(u\), respectivamente.

\hypertarget{interpretaciuxf3n-de-las-variables}{%
\subsection{Interpretación de las
Variables}\label{interpretaciuxf3n-de-las-variables}}

\begin{itemize}
\tightlist
\item
  \(A\): Representa palabras con una \(a\) de más.\\
\item
  \(B\): Representa palabras con una \(b\) de más.\\
\item
  \(S\): Representa palabras con igual número de \(a\) que de \(b\).
\end{itemize}

\hypertarget{propiedades-del-lenguaje-generado}{%
\subsection{Propiedades del Lenguaje
Generado}\label{propiedades-del-lenguaje-generado}}

\begin{enumerate}
\def\labelenumi{\arabic{enumi}.}
\tightlist
\item
  \textbf{Todas las palabras generadas tienen el mismo número de \(a\)
  que de \(b\).}\\
\item
  \textbf{Cualquier palabra con el mismo número de \(a\) que de \(b\)
  puede ser generada.}
\end{enumerate}

\hypertarget{demostraciuxf3n-de-la-primera-propiedad}{%
\subsection{Demostración de la Primera
Propiedad}\label{demostraciuxf3n-de-la-primera-propiedad}}

Consideremos \(N_{a,A}(\alpha)\) (número de \(a\) más número de \(A\)) y
\(N_{b,B}(\alpha)\) (número de \(b\) más número de \(B\)). Para una
derivación \(S \overset{*}{\Rightarrow} u\), tenemos:

\begin{itemize}
\tightlist
\item
  Al inicio: \(N_{a,A}(S) = N_{b,B}(S) = 0\).\\
\item
  Al aplicar cualquier regla \(\alpha_1 \to \alpha_2\), si
  \(N_{a,A}(\alpha_1) = N_{b,B}(\alpha_1)\), entonces
  \(N_{a,A}(\alpha_2) = N_{b,B}(\alpha_2)\).
\end{itemize}

Por lo tanto, al final de la derivación, \(N_{a,A}(u) = N_{b,B}(u)\).
Como \(u\) no contiene variables, \(N_a(u) = N_b(u)\), lo que demuestra
la propiedad.

\hypertarget{algoritmo-de-generaciuxf3n}{%
\subsection{Algoritmo de Generación}\label{algoritmo-de-generaciuxf3n}}

La generación de palabras se realiza por la izquierda, un símbolo a la
vez:

\begin{itemize}
\tightlist
\item
  \textbf{Para generar una \(a\):}

  \begin{itemize}
  \tightlist
  \item
    Si \(a\) es el último símbolo, aplicar \(A \to a\).\\
  \item
    Si no es el último símbolo:

    \begin{itemize}
    \tightlist
    \item
      Si la primera variable es \(S\), aplicar \(S \to aB\).\\
    \item
      Si la primera variable es \(B\), aplicar \(B \to aBB\).\\
    \item
      Si la primera variable es \(A\):

      \begin{itemize}
      \tightlist
      \item
        Si hay más variables, aplicar \(A \to a\).\\
      \item
        Si no hay más, aplicar \(A \to aS\).
      \end{itemize}
    \end{itemize}
  \end{itemize}
\item
  \textbf{Para generar una \(b\):}

  \begin{itemize}
  \tightlist
  \item
    Si \(b\) es el último símbolo, aplicar \(B \to b\).\\
  \item
    Si no es el último símbolo:

    \begin{itemize}
    \tightlist
    \item
      Si la primera variable es \(S\), aplicar \(S \to bA\).\\
    \item
      Si la primera variable es \(A\), aplicar \(A \to bAA\).\\
    \item
      Si la primera variable es \(B\):

      \begin{itemize}
      \tightlist
      \item
        Si hay más variables, aplicar \(B \to b\).\\
      \item
        Si no hay más, aplicar \(B \to bS\).
      \end{itemize}
    \end{itemize}
  \end{itemize}
\end{itemize}

\hypertarget{condiciones-de-garantuxeda}{%
\subsection{Condiciones de Garantía}\label{condiciones-de-garantuxeda}}

\begin{enumerate}
\def\labelenumi{\arabic{enumi}.}
\tightlist
\item
  Las palabras generadas tienen primero símbolos terminales y después
  variables.\\
\item
  Se genera un símbolo de la palabra en cada paso de derivación.\\
\item
  Las variables que aparecen en la palabra pueden ser:

  \begin{itemize}
  \tightlist
  \item
    Una cadena de \(A\) (si se han generado más \(b\) que \(a\)).\\
  \item
    Una cadena de \(B\) (si se han generado más \(a\) que \(b\)).\\
  \item
    Una \(S\) (si se han generado el mismo número de \(a\) y \(b\)).
  \end{itemize}
\end{enumerate}

Antes de generar el último símbolo, las variables serán:\\
- Una \(A\) si se necesita generar una \(a\).\\
- Una \(B\) si se necesita generar una \(b\).

En este caso, se aplica la primera opción para generar los símbolos, y
la palabra queda generada.

\hypertarget{gramuxe1tica-alternativa-para-el-mismo-lenguaje}{%
\section{Gramática Alternativa para el Mismo
Lenguaje}\label{gramuxe1tica-alternativa-para-el-mismo-lenguaje}}

\hypertarget{gramuxe1tica-que-incluye-la-palabra-vacuxeda}{%
\subsection{Gramática que Incluye la Palabra
Vacía}\label{gramuxe1tica-que-incluye-la-palabra-vacuxeda}}

Esta gramática genera todas las palabras con el mismo número de símbolos
\(a\) que \(b\), incluyendo la palabra vacía:

\begin{itemize}
\tightlist
\item
  \(S \to aSbS\)
\item
  \(S \to bSaS\)
\item
  \(S \to \varepsilon\)
\end{itemize}

\hypertarget{gramuxe1tica-que-excluye-la-palabra-vacuxeda}{%
\subsection{Gramática que Excluye la Palabra
Vacía}\label{gramuxe1tica-que-excluye-la-palabra-vacuxeda}}

Si no se desea incluir la palabra vacía, se puede usar la siguiente
gramática:

\begin{itemize}
\tightlist
\item
  \(S \to SS\)
\item
  \(S \to ab\)
\item
  \(S \to ba\)
\item
  \(S \to aSb\)
\item
  \(S \to bSa\)
\end{itemize}

\hypertarget{gramuxe1tica-generativa-ejemplo-adicional}{%
\section{Gramática Generativa: Ejemplo
Adicional}\label{gramuxe1tica-generativa-ejemplo-adicional}}

\hypertarget{gramuxe1tica-definida-1}{%
\subsection{Gramática Definida}\label{gramuxe1tica-definida-1}}

Sea la gramática \(G = (V, T, P, S)\) definida como:\\
- \(V = \{S, X, Y\}\)\\
- \(T = \{a, b, c\}\)\\
-
\(P = \{  S \to abc,  S \to aXbc,  Xb \to bX,  Xc \to Ybcc,  bY \to Yb,  aY \to aaX,  aY \to aa \}\)\\
- \(S = S\)

\hypertarget{lenguaje-generado-1}{%
\subsection{Lenguaje Generado}\label{lenguaje-generado-1}}

Esta gramática genera el lenguaje:

\[
L(G) = \{a^n b^n c^n \mid n \geq 1\}
\]

\hypertarget{proceso-de-derivaciuxf3n}{%
\subsection{Proceso de Derivación}\label{proceso-de-derivaciuxf3n}}

\begin{enumerate}
\def\labelenumi{\arabic{enumi}.}
\tightlist
\item
  \textbf{Caso Base}

  \begin{itemize}
  \tightlist
  \item
    \(S \to abc\): Genera la palabra \(abc\) para \(n = 1\).
  \end{itemize}
\item
  \textbf{Caso General}

  \begin{itemize}
  \tightlist
  \item
    \(S \to aXbc\): Introduce la variable \(X\) para generar palabras de
    mayor longitud.
  \end{itemize}

  A partir de \(aXbc\), el proceso es el siguiente:

  \begin{itemize}
  \tightlist
  \item
    \(aXbc \Rightarrow abXc \Rightarrow abYbcc \Rightarrow aYbbcc\)
  \end{itemize}

  En este punto, se tienen dos opciones:

  \begin{itemize}
  \item
    Aplicar \(aY \to aa\):\\
    \[
    aYbbcc \Rightarrow aabbcc
    \] Genera la palabra \(a^2b^2c^2\).
  \item
    Aplicar \(aY \to aaX\):\\
    \[
    aYbbcc \Rightarrow aaXbbcc
    \] Introduce nuevamente la variable \(X\), permitiendo repetir el
    proceso para generar palabras más largas.
  \end{itemize}
\item
  \textbf{Iteración del Proceso}

  \begin{itemize}
  \tightlist
  \item
    En cada iteración, la variable \(X\) se mueve hacia la frontera
    \(b-c\), donde se añade una \(b\) y una \(c\), y \(X\) se transforma
    en \(Y\).\\
  \item
    La variable \(Y\) se mueve hacia la frontera \(a-b\), donde se elige
    entre añadir una \(a\) o una \(aX\), permitiendo continuar el
    proceso.
  \end{itemize}
\end{enumerate}

\hypertarget{ejemplo-de-derivaciuxf3n-para-n-3}{%
\subsection{\texorpdfstring{Ejemplo de Derivación para
\(n = 3\)}{Ejemplo de Derivación para n = 3}}\label{ejemplo-de-derivaciuxf3n-para-n-3}}

\[
S \Rightarrow aXbc \Rightarrow abXc \Rightarrow abYbcc \Rightarrow aYbbcc \Rightarrow aaXbbcc \Rightarrow aabXbcc 
\] \[
\Rightarrow aabYbbccc \Rightarrow aaYbbbccc \Rightarrow aaaXbbbccc \Rightarrow aaabbbbccc
\]

\hypertarget{propiedades-del-lenguaje-generado-1}{%
\subsection{Propiedades del Lenguaje
Generado}\label{propiedades-del-lenguaje-generado-1}}

\begin{enumerate}
\def\labelenumi{\arabic{enumi}.}
\tightlist
\item
  \textbf{Todas las palabras generadas tienen la forma
  \(a^n b^n c^n\).}\\
\item
  \textbf{El proceso de derivación asegura que el número de \(a\),
  \(b\), y \(c\) es siempre igual.}\\
\item
  \textbf{El lenguaje generado es un subconjunto de \(\{a, b, c\}^*\)
  con la restricción de igualdad en las cantidades de \(a\), \(b\), y
  \(c\).}
\end{enumerate}

\chapter{Relaciones de Ejercicios}

\hypertarget{relaciuxf3n-tema-1-modelos-de-computaciuxf3n}{%
\section{Relación Tema 1: Modelos de
Computación}\label{relaciuxf3n-tema-1-modelos-de-computaciuxf3n}}

\begin{ejercicio}
Descripción de lenguajes generados por gramáticas.
\end{ejercicio}

\begin{enumerate}
\def\labelenumi{\alph{enumi})}
\item
  Describir el lenguaje generado por la siguiente gramática: \[
   S \to XYX \\
   \] \[
   X \to aX \ | \ bX \ | \ \epsilon \\
   \] \[
   Y \to bbb
   \]

  \begin{solucion}[Ejercicio 1.a]

   El lenguaje generado por la gramática está compuesto por cadenas que tienen la forma:

   \begin{enumerate}
       \item Una secuencia de cero o más a o b (generada por X).
       \item Seguido por bbb (generado por Y).
       \item Seguido nuevamente por una secuencia de cero o más a o b (generada por X).
   \end{enumerate}

   Por lo tanto, el lenguaje generado es:

   $$
   L = \{ w_1 \ bbb \ w_2 \ | \ w_1, w_2 \in \{a, b\}^* \}
   $$

   Donde $w_1$ y $w_2$ son cadenas arbitrarias (incluyendo la cadena vacía) formadas por los símbolos a y b. Otra forma es demostrándolo mediante doble inclusión (manera más matemática).

   \end{solucion}
\item
  Describir el lenguaje generado por la siguiente gramática: \[
   S \to aX 
   \] \[
   X \to aX \ | \ bX \ | \ \epsilon
   \]

  \begin{solucion}[Ejercicio 1.b]

   El lenguaje generado por la gramática está compuesto por cadenas que tienen la forma:

   \begin{enumerate}
       \item Una a inicial (generada por $S$).
       \item Seguido por una secuencia de cero o más $a$ o $b$ (generada por $X$).
   \end{enumerate}

   Por lo tanto, el lenguaje generado es:

   $$
   L = \{ a \ w \ | \ w \in \{a, b\}^* \}
   $$

   Donde $w$ es una cadena arbitraria (incluyendo la cadena vacía) formada por los símbolos a y b. Otra forma de demostrarlo es mediante doble inclusión.

   \end{solucion}
\item
  Describir el lenguaje generado por la siguiente gramática: \[
   S \to XaXaX \\
   \] \[
   X \to aX \ | \ bX \ | \ \epsilon
   \]

  \begin{solucion}[Ejercicio 1.c]

   El lenguaje generado por la gramática está compuesto por cadenas que tienen la forma:

   \begin{enumerate}
       \item Una secuencia de cero o más $a$ o $b$ (generada por $X$).
       \item Seguido por una $a$.
       \item Seguido nuevamente por una secuencia de cero o más $a$ o $b$ (generada por $X$).
       \item Seguido por otra $a$.
       \item Seguido nuevamente por una secuencia de cero o más $a$ o $b$ (generada por $X$).
   \end{enumerate}

   Por lo tanto, el lenguaje generado es:

   $$
   L = \{ w_1 \ a \ w_2 \ a \ w_3  \ | \ w_1, w_2, w_3 \in \{a, b\}^* \}
   $$

   Donde $w_1$, $w_2$ y $w_3$ son cadenas arbitrarias (incluyendo la cadena vacía) formadas por los símbolos a y b. Otra forma de demostrarlo es mediante doble inclusión.

   \end{solucion}
\item
  Describir el lenguaje generado por la siguiente gramática: \[
   S \to SS \ | \ XaXaX \ | \ \epsilon \\
   \] \[
   X \to bX \ | \ \epsilon
   \]

  \begin{solucion}[Ejercicio 1.d]

   El lenguaje generado por la gramática está compuesto por cadenas que tienen las siguientes características:

   \begin{enumerate}
       \item La gramática permite generar la cadena vacía ($\epsilon$).
       \item También permite generar cadenas de la forma $w_1 \ a \ w_2 \ a \ w_3 $, donde $w_1$, $w_2$, y $w_3$ son cadenas formadas únicamente por el símbolo $b$ (generadas por $X$).
       \item Además, permite concatenar arbitrariamente las cadenas generadas en los puntos anteriores debido a la regla $S \to SS$.
   \end{enumerate}

   Por lo tanto, el lenguaje generado es:

   $$
   L = \{ \epsilon \} \cup \{ w_1 \ a \ w_2 \ a \ w_3  \ | \ w_1, w_2, w_3 \in \{b\}^* \} \cup \{ uv \ | \ u, v \in L \}
   $$

   Donde $w_1$, $w_2$, y $w_3$ son cadenas arbitrarias (incluyendo la cadena vacía) formadas por el símbolo $b$, y $u, v$ son cadenas generadas por la gramática. Otra forma de demostrarlo es mediante doble inclusión.

   \textbf{Demostración por doble inclusión:}

   \begin{itemize}
       \item \textbf{Primera inclusión ($L \subseteq R$):}

           Sea $w \in L$. Según las reglas de la gramática, $w$ puede ser:
           \begin{itemize}
               \item La cadena vacía ($\epsilon$), que claramente pertenece a $R$.
               \item Una cadena de la forma $w_1 \ a \ w_2 \ a \ w_3 \ a$, donde $w_1, w_2, w_3 \in \{b\}^*$. Estas cadenas también pertenecen a $R$ por definición.
               \item Una concatenación de cadenas en $L$ (por la regla $S \to SS$). Si $u, v \in L$, entonces $uv \in R$ porque $R$ es cerrado bajo concatenación.
           \end{itemize}

           Por lo tanto, $w \in R$, y se cumple que $L \subseteq R$.

       \item \textbf{Segunda inclusión ($R \subseteq L$):}

           Sea $w \in R$. Según la definición de $R$, $w$ puede ser:
           \begin{itemize}
               \item La cadena vacía ($\epsilon$), que claramente puede ser generada por la gramática.
               \item Una cadena de la forma $w_1 \ a \ w_2 \ a \ w_3 \ a$, donde $w_1, w_2, w_3 \in \{b\}^*$. Estas cadenas pueden ser generadas por la regla $S \to XaXaX$ y $X \to bX \ | \ \epsilon$.
               \item Una concatenación de cadenas en $R$. Si $u, v \in R$, entonces $uv \in L$ porque la regla $S \to SS$ permite concatenar cadenas generadas por la gramática.
           \end{itemize}

           Por lo tanto, $w \in L$, y se cumple que $R \subseteq L$.
   \end{itemize}

   Dado que $L \subseteq R$ y $R \subseteq L$, se concluye que $L = R$.

   \end{solucion}
\end{enumerate}

\begin{ejercicio}
Determinar lenguajes.
\end{ejercicio}

\begin{enumerate}
\def\labelenumi{\alph{enumi})}
\item
  Dada la gramática \(G = (\{S, A\}, \{a, b\}, P, S)\) donde:\\
  \[P = \{S \to abAS, \ abA \to baab, \ S \to a, \ A \to b\}\]
  Determinar el lenguaje que genera.

  \begin{solucion}[Ejercicio 2.a]

       Cada vez que aplicamos $S \to abAS$ generamos un bloque $abA$ adicional y dejamos un $S$ al final para poder repetir la expansión. Tras $m$ aplicaciones de $S \to abAS$ obtenemos la forma $(abA)^mS$.

       Cada bloque $abA$ puede convertirse o bien en $baab$ aplicando la regla $abA \to baab$, o bien en $abb$ aplicando primero $A \to b$ (porque $abA \Rightarrow abb$). Finalmente $S \to a$. Por tanto, cada bloque se convierte en $baab$ o en $abb$ y al final queda una $a$.

       De aquí se deduce la forma general de las cadenas generadas:

       $$
       L(G) = \{ xa \ | \ x \in \{baab, abb\}^* \},
       $$

       es decir, en notación de expresiones regulares:

       $$
       L(G) = (baab \ | \ abb)^* a.
       $$

       \textbf{Prueba formal (dos sentidos)}

       \begin{enumerate}
           \item $L(G) \subseteq (baab \ | \ abb)^* a$

               Tras $m$ aplicaciones de $S \to abAS$ se tiene $(abA)^mS$ (prueba por inducción sobre $m$: base $m=0$ trivial; paso: si $S \Rightarrow (abA)^kS$ entonces aplicando $S \to abAS$ al $S$ final obtenemos $(abA)^{k+1}S$).

               Para cada uno de los $m$ factores $abA$ podemos aplicar $abA \to baab$ (obteniendo $baab$) o bien aplicar $A \to b$ (obteniendo $abb$). Por tanto, la parte antes de la última $S$ es una concatenación de $baab$ y $abb$.

               Finalmente $S \to a$. Por tanto, toda cadena derivable tiene la forma (bloques $baab$ o $abb$) seguida de $a$.

           \item $(baab \ | \ abb)^* a \subseteq L(G)$

               Sea $w = b_1b_2\cdots b_ma$ con cada $b_i \in \{baab, abb\}$.

               Expandimos $S$ $m$ veces con $S \to abAS$ para obtener $(abA)^mS$.

               Para cada $i$: si $b_i = baab$ aplicamos la regla $abA \to baab$ sobre el $i$-ésimo factor; si $b_i = abb$ aplicamos $A \to b$ en ese factor (convirtiendo $abA$ en $abb$).

               Finalmente aplicamos $S \to a$. Eso produce exactamente $w$. Por tanto, cualquier cadena del lado derecho puede derivarse.
       \end{enumerate}

   \end{solucion}
\item
  Sea la gramática \(G = (V, T, P, S)\) donde:\\
  \begin{align*}
   V &= \{\langle numero \rangle, \langle digito \rangle\} \\
   T &= \{0, 1, 2, 3, 4, 5, 6, 7, 8, 9\} \\
   S &= \langle numero \rangle \\
   \end{align*}

  \begin{itemize}
       \item $\langle numero \rangle \to \langle numero \rangle \langle digito \rangle$
       \item $\langle numero \rangle \to \langle digito \rangle$
       \item $\langle digito \rangle \to 0 \ | \ 1 \ | \ 2 \ | \ 3 \ | \ 4 \ | \ 5 \ | \ 6 \ | \ 7 \ | \ 8 \ | \ 9$
   \end{itemize}

  Determinar el lenguaje que genera.

  \begin{solucion}[Ejercicio 2.b]

   El lenguaje generado por la gramática está compuesto por cadenas que tienen las siguientes características:

   \begin{enumerate}
       \item La gramática permite generar cadenas formadas por uno o más dígitos, ya que:
           \begin{itemize}
               \item $\langle numero \rangle \to \langle numero \rangle \langle digito \rangle$ permite construir cadenas de longitud arbitraria añadiendo dígitos.
               \item $\langle numero \rangle \to \langle digito \rangle$ permite terminar la construcción con un único dígito.
           \end{itemize}
       \item Cada dígito es uno de los símbolos terminales $\{0, 1, 2, 3, 4, 5, 6, 7, 8, 9\}$, según la regla $\langle digito \rangle \to 0 \ | \ 1 \ | \ \dots \ | \ 9$.
   \end{enumerate}

   Por lo tanto, el lenguaje generado es el conjunto de todas las cadenas no vacías de dígitos, es decir:

   $$
   L = \{ w \ | \ w \in \{0, 1, 2, 3, 4, 5, 6, 7, 8, 9\}^+ \}.
   $$

   En notación de expresiones regulares, el lenguaje puede escribirse como:

   $$
   L = [0-9]^+.
   $$

   \end{solucion}
\item
  Sea la gramática \(G = (\{A, S\}, \{a, b\}, S, P)\) donde las reglas
  de producción son:\\

  \begin{itemize}
       \item $S \to aS$
       \item $S \to aA$
       \item $A \to bA$
       \item $A \to b$
   \end{itemize}

  Determinar el lenguaje generado por la gramática.

  \begin{solucion}[Ejercicio 2.c]

   El lenguaje generado por la gramática está compuesto por cadenas que tienen las siguientes características:

   \begin{enumerate}
       \item La gramática permite generar cadenas que comienzan con uno o más símbolos $a$, ya que:
           \begin{itemize}
               \item $S \to aS$ permite añadir un número arbitrario de $a$ al principio.
               \item $S \to aA$ permite terminar la secuencia de $a$ y pasar a generar $b$.
           \end{itemize}
       \item Después de la secuencia de $a$, la gramática genera uno o más símbolos $b$, ya que:
           \begin{itemize}
               \item $A \to bA$ permite añadir un número arbitrario de $b$.
               \item $A \to b$ permite terminar la secuencia de $b$.
           \end{itemize}
   \end{enumerate}

   Por lo tanto, el lenguaje generado es el conjunto de todas las cadenas que consisten en una secuencia no vacía de $a$ seguida de una secuencia no vacía de $b$, es decir:

   $$
   L = \{ a^n b^m \ | \ n \geq 1, m \geq 1 \}.
   $$

   En notación de expresiones regulares, el lenguaje puede escribirse como:

   $$
   L = a^+b^+.
   $$

   \end{solucion}
\end{enumerate}

\begin{ejercicio}
Gramáticas de tipo 2 y tipo 3. 

\begin{enumerate}[label=\alph*)]
    \item Encontrar una gramática de tipo 2 para el lenguaje de palabras en las que el número de $b$ no es tres.  
    Determinar si el lenguaje generado es de tipo 3.

    \item Encontrar una gramática de tipo 2 para el lenguaje de palabras que tienen 2 ó 3 $b$.  
    Determinar si el lenguaje generado es de tipo 3.
\end{enumerate}

\end{ejercicio}

\begin{ejercicio}
Gramáticas para lenguajes específicos.

\begin{enumerate}[label=\alph*)]
    \item Encontrar una gramática de tipo 2 para el lenguaje de palabras que no contienen la subcadena $ab$.  
    Determinar si el lenguaje generado es de tipo 3.

    \item Encontrar una gramática de tipo 2 para el lenguaje de palabras que no contienen la subcadena $baa$.  
    Determinar si el lenguaje generado es de tipo 3.
\end{enumerate}

\end{ejercicio}

\begin{ejercicio}
Lenguaje con más $a$ que $b$. Encontrar una gramática libre de contexto que genere el lenguaje sobre el alfabeto $\{a, b\}$ de las palabras que tienen más $a$ que $b$ (al menos una más).
\end{ejercicio}

\begin{ejercicio}
Gramáticas regulares o libres de contexto.

\begin{enumerate}[label=\alph*)]
    \item Encontrar, si es posible, una gramática regular (o, si no es posible, una gramática libre de contexto) que genere el lenguaje $L$ sobre el alfabeto $\{a, b\}$ tal que $u \in L$ si, y solamente si, $u$ no contiene dos símbolos $b$ consecutivos.

    \item Encontrar, si es posible, una gramática regular (o, si no es posible, una gramática libre de contexto) que genere el lenguaje $L$ sobre el alfabeto $\{a, b\}$ tal que $u \in L$ si, y solamente si, $u$ contiene dos símbolos $b$ consecutivos.
\end{enumerate}
\end{ejercicio}

\begin{ejercicio}
Propiedades de lenguajes.

\begin{enumerate}[label=\alph*)]
    \item Encontrar, si es posible, una gramática regular (o, si no es posible, una gramática libre de contexto) que genere el lenguaje $L$ sobre el alfabeto $\{a, b\}$ tal que $u \in L$ si, y solamente si, $u$ contiene un número impar de símbolos $a$.

    \item Encontrar, si es posible, una gramática regular (o, si no es posible, una gramática libre de contexto) que genere el lenguaje $L$ sobre el alfabeto $\{a, b\}$ tal que $u \in L$ si, y solamente si, $u$ no contiene el mismo número de símbolos $a$ que de símbolos $b$.
\end{enumerate}

\end{ejercicio}

\begin{ejercicio}
Gramáticas para palabras con restricciones.

\begin{enumerate}[label=\alph*)]
    \item Dado el alfabeto $A = \{a, b\}$, determinar si es posible encontrar una gramática libre de contexto que genere las palabras de longitud impar, y mayor o igual que 3, tales que la primera letra coincida con la letra central de la palabra.

    \item Dado el alfabeto $A = \{a, b\}$, determinar si es posible encontrar una gramática libre de contexto que genere las palabras de longitud par, y mayor o igual que 2, tales que las dos letras centrales coincidan.
\end{enumerate}

\end{ejercicio}

\begin{ejercicio}
Regularidad de un lenguaje.
Determinar si el lenguaje generado por la gramática  
$S \to SS$  
$S \to XXX$  
$X \to aX \ | \ Xa \ | \ b$  
es regular. Justificar la respuesta.
\end{ejercicio}

\begin{ejercicio}
Numerabilidad de un lenguaje.
Dado un lenguaje $L$ sobre un alfabeto $A$, ¿es $L^*$ siempre numerable? ¿nunca lo es? ¿o puede serlo unas veces sí y otras, no? Proporcionar ejemplos en este último caso.
\end{ejercicio}

\begin{ejercicio}
Propiedades de $L^*$. Dado un lenguaje $L$ sobre un alfabeto $A$, caracterizar cuándo $L^* = L$. Es decir, dar un conjunto de propiedades sobre $L$ de manera que $L$ cumpla esas propiedades si y sólo si $L^* = L$.
\end{ejercicio}

\begin{ejercicio}
Igualdad de homomorfismos. Dados dos homomorfismos $f : A^* \to B^*$, $g : A^* \to B^*$, se dice que son iguales si $f(x) = g(x)$, $\forall x \in A^*$. ¿Existe un procedimiento algorítmico para comprobar si dos homomorfismos son iguales?
\end{ejercicio}

\begin{ejercicio}
Lenguajes $S_i$ y $C_i$. 
Sea $L \subseteq A^*$ un lenguaje arbitrario. Sea $C_0 = L$ y definamos los lenguajes $S_i$ y $C_i$, para todo $i \geq 1$, por $S_i = C_{i-1}^+$ y $C_i = S_i^*$. 

\begin{enumerate}[label=\alph*)]
    \item ¿Es $S_1$ siempre, nunca o a veces igual a $C_2$? Justificar la respuesta.  

    \item Demostrar que $S_2 = C_3$, cualquiera que sea $L$. (Pista: Demostrar que $C_2$ es cerrado para la concatenación).
\end{enumerate}

\end{ejercicio}

\begin{ejercicio}
Numerabilidad de lenguajes finitos. Demostrar que, para todo alfabeto $A$, el conjunto de los lenguajes finitos sobre dicho alfabeto es numerable.
\end{ejercicio}

\begin{thebibliography}{99}

  \bibitem{Referencia1}
  Ismael Sallami Moreno, \textbf{Estudiante del Doble Grado en Ingeniería Informática + ADE}, Universidad de Granada, 2025.
  
  \bibitem{DiapositivasAsignatura}
  Universidad de Granada, \emph{Diapositivas de la asignatura}, Curso 2025/2026.

  % \bibitem{Referencia2}
  % Autor Apellido, \emph{Título del libro o artículo}, Editorial o Revista, Año.
  
  % \bibitem{Referencia3}
  % Nombre Autor, \emph{Título del documento}, Conferencia/URL, Año.
  
  \end{thebibliography}
  

\end{document}
