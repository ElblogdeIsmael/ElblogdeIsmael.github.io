% ========================
% estilo.latex mínimo funcional
% ========================

\documentclass[12pt]{report} % report para capítulos

% ========================
% Paquetes y comandos extra
% ========================
% ===========================
% Paquetes básicos de idioma y codificación
% ===========================
\usepackage[utf8]{inputenc}   % Codificación UTF-8
\usepackage[T1]{fontenc}      % Acentos y caracteres correctos
\usepackage[spanish]{babel}   % Traducción al español (capítulos, índices, etc.)
\usepackage{csquotes}         % Citas tipográficas correctas

% ===========================
% Tipografía
% ===========================
\usepackage{lmodern}          % Fuente Latin Modern
\usepackage{microtype}        % Mejoras tipográficas (espaciado, justificación)

% ===========================
% Márgenes y geometría
% ===========================
\usepackage{geometry}         % Control de márgenes
\geometry{a4paper, top=3cm, bottom=3cm, left=3cm, right=3cm}

% ===========================
% Matemáticas
% ===========================
\usepackage{amsmath, amssymb, amsthm} % Paquetes AMS
\usepackage{mathtools}        % Extiende amsmath
\usepackage{physics}          % Notación física y matemática (derivadas, bra-ket, etc.)
\usepackage{siunitx}          % Unidades SI (e.g. \SI{3}{m/s})
% \sisetup{locale=ES}           % Configuración para español (coma decimal, etc.)
\AtBeginDocument{\RenewCommandCopy\qty\SI} % Resolve siunitx and physics conflict


% ===========================
% Gráficos, tablas y colores
% ===========================
\usepackage{graphicx}         % Insertar imágenes
\usepackage{xcolor}           % Colores personalizados
\usepackage{tikz}             % Dibujos vectoriales
\usetikzlibrary{calc,positioning,shapes,arrows} % Librerías útiles de TikZ
\usepackage{pgfplots}         % Gráficas de funciones
\pgfplotsset{compat=1.18}
\usepackage{float}            % Control de posición de figuras/tablas
\usepackage{booktabs}         % Tablas profesionales
\usepackage{multirow}         % Celdas que ocupan varias filas
\usepackage{array}            % Más control en tablas
\usepackage{colortbl}         % Tablas con colores
\usepackage{inconsolata}


% ===========================
% Listas y enumeraciones
% ===========================
\usepackage{enumitem}         % Control de listas enumeradas y viñetas

% ===========================
% Encabezados, pies y diseño
% ===========================
\usepackage{fancyhdr}         % Encabezados y pies de página
\usepackage{titlesec}         % Personalizar títulos de capítulos/secciones
\usepackage{setspace}         % Espaciado entre líneas
\usepackage{parskip}          % Control del espacio entre párrafos

% ===========================
% Referencias, hipervínculos y citas
% ===========================
\usepackage{hyperref}         % Hipervínculos en PDF
\hypersetup{
    colorlinks = true,
    linkcolor  = red!70,
    citecolor  = red!70,
    urlcolor   = red!70,
    pdfpagelayout = SinglePage, % Asegura que el contenido se ajuste a una sola página
    pdfstartview = Fit          % Ajusta el contenido al tamaño de la página
}
\usepackage{cleveref}         % Referencias inteligentes (\cref)

% ===========================
% Código fuente
% ===========================
\usepackage{listings}         % Mostrar código con estilo
\usepackage{minted}           % (mejor opción, requiere Python y pygments)

% ===========================
% Bibliografía
% ===========================
\usepackage[backend=biber,style=apa]{biblatex} % Ejemplo: estilo APA
\addbibresource{referencias.bib}              % Archivo .bib

% ===========================
% Otros útiles
% ===========================
\usepackage{pdfpages}         % Insertar PDFs externos
\usepackage{blindtext}        % Texto de prueba
\usepackage{caption}          % Personalizar pies de figura/tabla
\usepackage{subcaption}       % Subfiguras
\usepackage{tocloft} 
\usepackage{amsthm}
\usepackage{subcaption}
\usepackage{truncate} % permite truncar texto si no cabe
\usepackage{libertinus}  % reemplaza lmodern
\usepackage{booktabs}  % para \toprule, \midrule, \bottomrule
\usepackage{array}     % para definir columnas personalizadas
\usepackage{colortbl}  % colores en tablas
\usepackage{etoolbox}
\AtBeginEnvironment{tabular}{\rowcolors{2}{gray!10}{white}\renewcommand{\arraystretch}{1.2}}

% ===========================
% Opciones de fuentes sugeridas
% ===========================
% TeX Gyre Pagella (estilo Palatino)
% \usepackage{fontspec}
% \usepackage{unicode-math}
% \setmainfont{TeX Gyre Pagella}
% \setmathfont{TeX Gyre Pagella Math}

% TeX Gyre Termes (estilo Times)
% \setmainfont{TeX Gyre Termes}
% \setmathfont{TeX Gyre Termes Math}

% Libertinus (elegante y completa)
% \setmainfont{Libertinus Serif}
% \setmathfont{Libertinus Math}

% TeX Gyre Bonum (estilo Garamond)
% \setmainfont{TeX Gyre Bonum}
% \setmathfont{TeX Gyre Bonum Math}

% Latin Modern (moderno de Computer Modern)
% \setmainfont{Latin Modern Roman}
% \setmathfont{Latin Modern Math}


% \usepackage{helvet}
% \usepackage{libertine}
% \usepackage[sfdefault]{FiraSans}

\usepackage{tcolorbox} % para cajas de colores




  % si tienes paquetes personalizados
% aquí van los comandos personalizados
% Comando para incluir imágenes
\newcommand{\incluirimagen}[3][]{%
\begin{figure}[H]
    \centering
    \includegraphics[width=\linewidth,#1]{#2}
    \caption{#3}
    \label{fig:#2}
\end{figure}
}

% comando para ejercicios con fondo
\newtheoremstyle{ejerciciostyle}
  {10pt}   % Espacio arriba
  {10pt}   % Espacio abajo
  %{\itshape} % Fuente del cuerpo
  {}
  {}       % Sangría
  {\bfseries} % Fuente del encabezado
  {}      % Puntuación tras encabezado
  { }      % Espacio tras encabezado
  {\thmname{#1} \thmnumber{#2}. \thmnote{#3}}


% % comando formal para enunciado de ejercicios
% \theoremstyle{ejerciciostyle}
% \newtheorem{ejercicio}{Ejercicio}[chapter]

\theoremstyle{ejerciciostyle}
\newtheorem{ejercicio}{Ejercicio}[section]

\renewcommand{\theejercicio}{\thechapter.\arabic{section}.\arabic{ejercicio}}


% comando formal para soluciones
\theoremstyle{remark}
\newtheorem{solucion}{Solución}[ejercicio]

\renewcommand{\thesolucion}{\thechapter.\arabic{section}.\arabic{ejercicio}}

% Comando para dos imágenes en paralelo
\newcommand{\dosimagenes}[6]{%
    \begin{figure}[h!]
        \centering
        \begin{minipage}{0.48\linewidth}
            \centering
            \includegraphics[width=\linewidth]{#1}
            \caption{#2}
            \label{#5}
        \end{minipage}\hfill
        \begin{minipage}{0.48\linewidth}
            \centering
            \includegraphics[width=\linewidth]{#3}
            \caption{#4}
            \label{#6}
        \end{minipage}
    \end{figure}
}

% \dosimagenes{media/fondo.jpg}{Descripción 1}{media/fondo.jpg}{Descripción 2}{fig:descripcion1}{fig:descripcion2}

% \ref{fig:descripcion1} es la mejor
% \ref{fig:descripcion2} es la mejor

\newcommand{\portadaimg}{\VAR{portadaimg}}

% Comando para crear una nota estilo información
% \newcommand{\nota}[2]{%
% \begin{tcolorbox}[colframe=blue!75!black, colback=blue!5!white, title=\textbf{#1}]
%     #2
% \end{tcolorbox}
% }
\newtheorem{nota}{Nota}[chapter]


% Comando para poner dos códigos en paralelo
\newcommand{\doscodigos}[4]{%
  \noindent
  \begin{minipage}{0.48\linewidth}
    \lstset{language=#1}
    \lstinputlisting{#2}
  \end{minipage}\hfill
  \begin{minipage}{0.48\linewidth}
    \lstset{language=#3}
    \lstinputlisting{#4}
  \end{minipage}
}

% Comando para poner un solo código
\newcommand{\uncodigo}[2]{%
  \begin{lstlisting}[language=#1]
#2
  \end{lstlisting}
}


% % Listas de archivos (sin guiones en los nombres de macros)
% \newcommand{\listagdfilesSesion2Mallas2D}{cargatexturas.gd, envioinmediato.gd, malla2dcontexturas.gd, mallaconcoloresdevertices.gd, mallanoindentada.gd}
% \newcommand{\listagdfilesSesion2Mallas3D}{mallaindexada3d.gd, materialconcolordeplano.gd, materialconcoloresdevertices.gd, tablas.gd}

% % Macro que recorre una lista de archivos en un subdirectorio
% \newcommand{\includegdfiles}[2]{%
%   % #1 = subdirectorio
%   % #2 = nombre de la lista de archivos
%   \foreach \filename in #2 {%
%     \includecode[gdstyle]{code/#1/\filename}{\filename}
%   }%
% }



% Comando para ejercicio resuelto
\newtheoremstyle{ejercicioresueltostyle}
    {10pt}   % Espacio arriba
    {10pt}   % Espacio abajo
    {\itshape} % Fuente del cuerpo
    {}       % Sangría
    {\bfseries} % Fuente del encabezado
    {}      % Puntuación tras encabezado
    { }      % Espacio tras encabezado
    {\thmname{#1} \thmnumber{#2}. \thmnote{#3}}

\theoremstyle{ejercicioresueltostyle}
\newtheorem{ejercicioresuelto}{Ejercicio Resuelto}[section]

\renewcommand{\theejercicioresuelto}{\thechapter.\arabic{section}.\arabic{ejercicioresuelto}}


%======================================================================== 
% PRACTICAS
%========================================================================

% Comando para definir un tema
\newcommand{\tema}[1]{%
  \section{#1}
  \addcontentsline{toc}{section}{#1}
}
\usepackage{tikz}
\usepackage{graphicx} % necesario para \resizebox
\usepackage{etoolbox}

% ======== NODOS ========
\newcommand{\nodo}[4][]{\node[state, #1] (#2) at (#3) {$#4$};}
% Uso: \nodo[initial,accepting]{q0}{0,0}{q_0}

% ======== FLECHAS ========
\newcommand{\flecha}[4][]{\draw[->, #1] (#2) -- (#3) node[midway, above] {#4};}
% Uso: \flecha{q0}{q1}{0} o \flecha[bend left]{q1}{q2}{1}

\newcommand{\flechaabajo}[4][]{\draw[->, #1] (#2) -- (#3) node[midway, below, yshift=-6pt] {#4};}
% Igual que \flecha pero con etiqueta abajo
\newcommand{\flechaarriba}[4][]{\draw[->, #1] (#2) -- (#3) node[midway, above, yshift=6pt] {#4};}
% Igual que \flecha pero con etiqueta arriba
\newcommand{\flechaderecha}[4][]{\draw[->, #1] (#2) -- (#3) node[midway, right] {#4};}
% Igual que \flecha pero con etiqueta a la derecha
\newcommand{\flechaiquierda}[4][]{\draw[->, #1] (#2) -- (#3) node[midway, left] {#4};}
% Igual que \flecha pero con etiqueta a la izquierda

\newcommand{\curva}[5][]{\draw[->, bend #1] (#2) to node[midway, #5] {#4} (#3);}
% Uso: \curva[left]{q1}{q2}{1}{below}


\newcommand{\loopa}[3]{\draw[->] (#1) edge[loop above] node {#2} (#1);}
\newcommand{\loopb}[3]{\draw[->] (#1) edge[loop below] node {#2} (#1);}
\newcommand{\loopr}[3]{\draw[->] (#1) edge[loop right] node {#2} (#1);}
\newcommand{\loopl}[3]{\draw[->] (#1) edge[loop left] node {#2} (#1);}
% Uso: \loopa{q1}{0}

% ======== ESTILOS ESPECIALES ========
\tikzset{
    error/.style={state, fill=red!20, draw=red!80!black},
    final/.style={state, accepting, fill=green!15!white, draw=green!60!black}
}
% Uso: \nodo[error]{qe}{5,0}{q_e}  o \nodo[final]{qf}{7,0}{q_f}


\newcommand{\pa}{1}      % ejemplo de valor
\newcommand{\pUno}{2}
\newcommand{\pDos}{3}
  % comandos LaTeX propios
% ===========================
% Diseño general
% ===========================
\setstretch{1.15} % interlineado
\setlength{\parskip}{0.5em} % espacio entre párrafos
\setlength{\parindent}{0pt} % sin sangría

% ===========================
% Estilo de capítulos y secciones (titlesec)
% ===========================
\titleformat{\chapter}[display]
  {\bfseries\Huge}
  {\filleft\Large\scshape Capítulo \thechapter}
  {1ex}
  {\titlerule[1pt]\vspace{1ex}\filright}
  [\vspace{1ex}\titlerule]

\titlespacing*{\chapter}{0pt}{0pt}{2em}

\titleformat{\section}
  {\Large\bfseries}
  {\thesection}{1em}{}

\titleformat{\subsection}
  {\large\bfseries}
  {\thesubsection}{1em}{}

\titleformat{\subsubsection}
  {\normalsize\bfseries\itshape}
  {\thesubsubsection}{1em}{}

% ===========================
% Encabezados y pies de página (fancyhdr)
% ===========================
\pagestyle{fancy}
\fancyhf{} % limpia
\fancyhead[L]{\small\scshape\nouppercase{\leftmark}} % sección/capítulo en mayúsculas pequeñas
\fancyhead[R]{\small\thepage}                        % número de página
%\fancyfoot[C]{\scriptsize\itshape Apuntes de la carrera} % texto fijo abajo en cursiva
% Encabezados y pies de página personalizados
% \fancyfoot[L]{\scriptsize\itshape Nombre de la asignatura} % pie de página izquierdo en cursiva
\fancyfoot[R]{\normalsize Ismael Sallami Moreno}        % pie de página derecho con el nombre del autor

% Línea bajo el encabezado
\renewcommand{\headrulewidth}{0.5pt} % línea más gruesa en el encabezado
% Línea en el pie
\renewcommand{\footrulewidth}{0.4pt} % línea fina en el pie
\renewcommand{\sectionmark}[1]{%
  \markboth{\thesection\quad #1}{}%
}

% ===========================
% Numeración de elementos
% ===========================
\numberwithin{equation}{chapter} % ecuaciones numeradas por capítulo
\numberwithin{figure}{chapter}   % figuras numeradas por capítulo
\numberwithin{table}{chapter}    % tablas numeradas por capítulo

% ===========================
% Listas y enumeraciones
% ===========================
\setlist[itemize]{label=--, left=1.5em}
\setlist[enumerate]{label=\arabic*), left=1.5em}

% ===========================
% Estilo de citas y bibliografía
% ===========================
\DefineBibliographyStrings{spanish}{%
  references = {Bibliografía},
}

% ===========================
% Entornos personalizados
% ===========================
\newtheoremstyle{cajita} % nombre del estilo
  {1em}   % espacio arriba
  {1em}   % espacio abajo
  {}      % fuente del cuerpo
  {}      % indentación
  {\bfseries} % fuente del título
  {.}     % puntuación tras título
  {0.5em} % espacio tras título
  {\thmname{#1}\thmnumber{ #2} \thmnote{(#3)}} % formato


\theoremstyle{cajita}
\newtheorem{teorema}{Teorema}[chapter]
\newtheorem{definicion}{Definición}[chapter]
\newtheorem{ejemplo}{Ejemplo}[chapter]
\newtheorem{proposicion}{Proposición}[chapter]
\newtheorem{demostracion}{Demostración}[chapter]
\newtheorem{corolario}{Corolario}[chapter]
\newtheorem{propuesta}{Propuesta}[chapter]


\newtheoremstyle{anotacionstyle} % nombre del estilo
  {1em}   % espacio arriba
  {1em}   % espacio abajo
  {}      % fuente del cuerpo (sin cursiva)
  {}      % indentación
  {\itshape} % fuente del título (Nota en cursiva)
  {.}     % puntuación tras título
  {0.5em} % espacio tras título
  {\thmname{\itshape#1}\thmnumber{ #2} \thmnote{(#3)}} % formato (solo Nota en cursiva)

\theoremstyle{anotacionstyle}
\newtheorem{anotacion}{Nota}[chapter]

% ===========================
% Configuración de lstlisting
% ===========================

% ===============================================
% ESTILO 1: MODERNO Y MINIMALISTA
% ===============================================

% Definir colores personalizados
\definecolor{codegreen}{rgb}{0,0.6,0}
\definecolor{codegray}{rgb}{0.5,0.5,0.5}
\definecolor{codepurple}{rgb}{0.58,0,0.82}
\definecolor{backcolour}{rgb}{0.95,0.95,0.92}
\definecolor{framecolor}{rgb}{0.8,0.8,0.8}

\lstset{
  backgroundcolor=\color{backcolour},   
  commentstyle=\color{codegreen},
  keywordstyle=\color{magenta},
  numberstyle=\tiny\color{codegray},
  stringstyle=\color{codepurple},
  basicstyle=\ttfamily\footnotesize,
  breakatwhitespace=false,         
  breaklines=true,                 
  captionpos=b,                    
  keepspaces=true,                 
  numbers=left,                    
  numbersep=5pt,                  
  showspaces=false,                
  showstringspaces=false,
  showtabs=false,                  
  tabsize=2,
  frame=shadowbox,
  frameround=tttt,
  rulecolor=\color{framecolor},
  rulesepcolor=\color{framecolor},
  xleftmargin=20pt,
  xrightmargin=20pt,
  aboveskip=20pt,
  belowskip=20pt,
  inputencoding=utf8,
  extendedchars=true,
  literate=
    {←}{{$\leftarrow$}}1
    {→}{{$\rightarrow$}}1
    {↑}{{$\uparrow$}}1
    {↓}{{$\downarrow$}}1
    {↔}{{$\leftrightarrow$}}1
    {⇒}{{$\Rightarrow$}}1
    {⇐}{{$\Leftarrow$}}1
    {⇔}{{$\Leftrightarrow$}}1
    {α}{{$\alpha$}}1
    {β}{{$\beta$}}1
    {γ}{{$\gamma$}}1
    {δ}{{$\delta$}}1
    {ε}{{$\epsilon$}}1
    {θ}{{$\theta$}}1
    {λ}{{$\lambda$}}1
    {μ}{{$\mu$}}1
    {π}{{$\pi$}}1
    {σ}{{$\sigma$}}1
    {φ}{{$\phi$}}1
    {ψ}{{$\psi$}}1
    {ω}{{$\omega$}}1
    {Δ}{{$\Delta$}}1
    {Θ}{{$\Theta$}}1
    {Λ}{{$\Lambda$}}1
    {Π}{{$\Pi$}}1
    {Σ}{{$\Sigma$}}1
    {Φ}{{$\Phi$}}1
    {Ψ}{{$\Psi$}}1
    {Ω}{{$\Omega$}}1
    {á}{{\'a}}1
    {é}{{\'e}}1
    {í}{{\'i}}1
    {ó}{{\'o}}1
    {ú}{{\'u}}1
    {Á}{{\'A}}1
    {É}{{\'E}}1
    {Í}{{\'I}}1
    {Ó}{{\'O}}1
    {Ú}{{\'U}}1
    {ä}{{\"a}}1
    {ë}{{\"e}}1
    {ï}{{\"i}}1
    {ö}{{\"o}}1
    {ü}{{\"u}}1
    {Ä}{{\"A}}1
    {Ë}{{\"E}}1
    {Ï}{{\"I}}1
    {Ö}{{\"O}}1
    {Ü}{{\"U}}1
    {ñ}{{\~n}}1
    {Ñ}{{\~N}}1
    {ç}{{\c{c}}}1
    {Ç}{{\c{C}}}1
    {¿}{{?`}}1
    {¡}{{!`}}1
    {à}{{\`a}}1
    {è}{{\`e}}1
    {ì}{{\`i}}1
    {ò}{{\`o}}1
    {ù}{{\`u}}1
    {À}{{\`A}}1
    {È}{{\`E}}1
    {Ì}{{\`I}}1
    {Ò}{{\`O}}1
    {Ù}{{\`U}}1
    {-}{{-}}1
    {=}{{=\allowbreak}}1  % <--- ESTA LÍNEA ES EL TRUCO PARA CORTAR LOS '===='
    % {#}{{\#}}1 
}


% ===============================================
% ESTILO 2: ELEGANTE CON BORDES REDONDEADOS
% ===============================================

% Colores para estilo elegante
\definecolor{lightblue}{rgb}{0.93,0.95,1}
\definecolor{darkblue}{rgb}{0.1,0.2,0.5}
\definecolor{mediumblue}{rgb}{0.2,0.4,0.8}
\definecolor{darkgreen}{rgb}{0,0.5,0}
\definecolor{darkred}{rgb}{0.6,0,0}

\lstdefinestyle{elegant}{
    backgroundcolor=\color{lightblue},
    commentstyle=\color{darkgreen}\itshape,
    keywordstyle=\color{darkblue}\bfseries,
    numberstyle=\tiny\color{gray},
    stringstyle=\color{darkred},
    basicstyle=\ttfamily\small,
    breakatwhitespace=false,
    breaklines=true,
    captionpos=t,
    keepspaces=true,
    numbers=left,
    numbersep=8pt,
    showspaces=false,
    showstringspaces=false,
    showtabs=false,
    tabsize=4,
    frame=single,
    frameround=tttt,
    framesep=10pt,
    xleftmargin=15pt,
    xrightmargin=15pt,
    aboveskip=15pt,
    belowskip=15pt,
    columns=flexible
}

% ===============================================
% ESTILO 3: PROFESIONAL CORPORATIVO
% ===============================================

% Colores corporativos
\definecolor{corporatebg}{rgb}{0.98,0.98,0.98}
\definecolor{corporateblue}{rgb}{0.07,0.29,0.49}
\definecolor{corporategray}{rgb}{0.4,0.4,0.4}
\definecolor{corporategreen}{rgb}{0.13,0.55,0.13}
\definecolor{corporatered}{rgb}{0.8,0.2,0.2}

\lstdefinestyle{corporate}{
    backgroundcolor=\color{corporatebg},
    commentstyle=\color{corporategreen}\slshape,
    keywordstyle=\color{corporateblue}\bfseries,
    numberstyle=\scriptsize\color{corporategray},
    stringstyle=\color{corporatered},
    basicstyle=\ttfamily\footnotesize,
    breakatwhitespace=false,
    breaklines=true,
    captionpos=b,
    keepspaces=true,
    numbers=left,
    numbersep=12pt,
    showspaces=false,
    showstringspaces=false,
    showtabs=false,
    tabsize=3,
    frame=leftline,
    framerule=3pt,
    rulecolor=\color{corporateblue},
    xleftmargin=25pt,
    aboveskip=20pt,
    belowskip=20pt,
    lineskip=1pt
}

% ===============================================
% ESTILO 4: MODERNO CON SOMBRAS
% ===============================================

% Colores modernos
\definecolor{modernbg}{rgb}{0.97,0.97,0.97}
\definecolor{moderngray}{rgb}{0.3,0.3,0.3}
\definecolor{modernpurple}{rgb}{0.5,0.2,0.8}
\definecolor{modernteal}{rgb}{0,0.5,0.5}
\definecolor{modernorange}{rgb}{0.8,0.4,0}

\lstdefinestyle{modern}{
    backgroundcolor=\color{modernbg},
    commentstyle=\color{modernteal}\itshape,
    keywordstyle=\color{modernpurple}\bfseries,
    numberstyle=\tiny\color{moderngray},
    stringstyle=\color{modernorange},
    basicstyle=\ttfamily\small,
    breakatwhitespace=false,
    breaklines=true,
    captionpos=t,
    keepspaces=true,
    numbers=left,
    numbersep=10pt,
    showspaces=false,
    showstringspaces=false,
    showtabs=false,
    tabsize=4,
    frame=tb,
    framerule=2pt,
    rulecolor=\color{modernpurple},
    xleftmargin=20pt,
    xrightmargin=20pt,
    aboveskip=25pt,
    belowskip=25pt
}

% ===============================================
% CONFIGURACIÓN PARA DIFERENTES LENGUAJES
% ===============================================

% Python
\lstdefinestyle{python}{
    language=Python,
    style=elegant,
    morekeywords={True,False,None,self,cls,def,class,import,from,as,with,yield,async,await},
    morecomment=[l]{\#},
    morestring=[b]',
    morestring=[b]"
}

% Java
\lstdefinestyle{java}{
    language=Java,
    style=corporate,
    morekeywords={var,record,sealed,permits,non-sealed}
}

% C++
\lstdefinestyle{cpp}{
    language=C++,
    style=modern,
    morekeywords={constexpr,nullptr,auto,decltype,override,final}
}

% JavaScript
\lstdefinestyle{javascript}{
    language=Java,
    style=elegant,
    morekeywords={let,const,var,function,class,extends,import,export,default,async,await,yield},
    morecomment=[l]{//},
    morecomment=[s]{/*}{*/},
    morestring=[b]',
    morestring=[b]",
    morestring=[b]`
}

% ===============================================
% EJEMPLOS DE USO
% ===============================================

% Para usar el estilo por defecto:
% \begin{lstlisting}
% código aquí
% \end{lstlisting}

% Para usar un estilo específico:
% \begin{lstlisting}[style=elegant]
% código aquí
% \end{lstlisting}

% Para incluir un archivo con estilo específico:
% \lstinputlisting[style=python]{archivo.py}

% Para código inline:
% \lstinline[style=modern]{código inline}

% ===============================================
% CONFIGURACIÓN ADICIONAL PARA TÍTULOS Y CARACTERES
% ===============================================

% Personalizar el formato de los títulos de los listados
\renewcommand\lstlistingname{Código}
\renewcommand\lstlistlistingname{Lista de Códigos}

% Configurar el formato del título con soporte para tildes
\lstset{
    %title=\lstname,
    captionpos=t,
    abovecaptionskip=10pt,
    belowcaptionskip=5pt,
    % Configuración global para caracteres especiales
    inputencoding=utf8,
    extendedchars=true
}

% ===============================================
% COMANDOS PERSONALIZADOS ÚTILES
% ===============================================

% Comando para código inline con soporte automático de tildes
\newcommand{\codeinline}[2][modern]{\lstinline[style=#1,inputencoding=utf8,extendedchars=true]{#2}}

% Comando para bloques de código con título personalizado
\newcommand{\codeblock}[3][elegant]{%
    \begin{lstlisting}[style=#1,caption={#2},label={lst:#2},inputencoding=utf8,extendedchars=true]
    #3
    \end{lstlisting}
}

% Comando para incluir archivos con configuración automática
\newcommand{\includecode}[3][python]{%
    \lstinputlisting[style=#1,caption={#3},label={lst:#3},inputencoding=utf8,extendedchars=true]{#2}
}

% ===============================================
% CONFIGURACIONES ESPECIALES PARA IDIOMAS
% ===============================================

% Configuración específica para código en español
\lstdefinestyle{español}{
    style=elegant,
    inputencoding=utf8,
    extendedchars=true,
    % Palabras clave en español para pseudocódigo
    morekeywords={función,procedimiento,inicio,fin,si,entonces,sino,mientras,para,hasta,hacer,repetir,caso,segun,verdadero,falso,entero,real,caracter,cadena,booleano,leer,escribir,imprimir}
}

% Configuración para comentarios multilíngües
\lstset{
    morecomment=[l]{//\ },
    morecomment=[l]{\#\ },
    morecomment=[s]{/*}{*/},
    morecomment=[s]{}
}

% ===============================================
% CONFIGURACIÓN PARA DIFERENTES LENGUAJES
% ===============================================

% Python
\lstdefinestyle{style1}{
    language=Python,
    style=elegant,
    morekeywords={True,False,None,self,cls,def,class,import,from,as,with,yield,async,await},
    morecomment=[l]{\#},
    morestring=[b]',
    morestring=[b]",
    % Soporte para caracteres especiales
    inputencoding=utf8,
    extendedchars=true
}

% Java
\lstdefinestyle{style2}{
    language=Java,
    style=corporate,
    morekeywords={var,record,sealed,permits,non-sealed},
    % Soporte para caracteres especiales
    inputencoding=utf8,
    extendedchars=true
}

% C++
\lstdefinestyle{style3}{
    language=C++,
    style=modern,
    morekeywords={constexpr,nullptr,auto,decltype,override,final},
    % Soporte para caracteres especiales
    inputencoding=utf8,
    extendedchars=true
}

\lstdefinelanguage{GDScript}{
  keywords={func, var, extends, class_name, if, else, for, while, return, match, in, and, or, not, break, continue, pass},
  sensitive=true,
  morecomment=[l]{\#},
  morestring=[b]",
  morestring=[b]',
}

\lstdefinestyle{gdstyle}{
  language=GDScript,
  basicstyle=\ttfamily\small,
  keywordstyle=\color{blue}\bfseries,
  commentstyle=\color{gray},
  stringstyle=\color{red!60!black},
  numbers=left,
  numberstyle=\tiny\color{gray},
  breaklines=true,
  frame=single,
  tabsize=2,
}


% ===========================
% Estilo global de tablas
% ===========================

\usepackage{booktabs}   % reglas profesionales
\usepackage{colortbl}   % color en filas
\usepackage{xcolor}     % colores
\usepackage{float}      % [H]

% Color de filas alternadas
% \rowcolors{2}{gray!10}{white}

% % Espacio vertical entre filas
% \renewcommand{\arraystretch}{1.2}

% % Cambiar el tamaño de columna por defecto
% \setlength{\tabcolsep}{8pt}

% % Redefinir tabla para que todas las tablas tengan el estilo
% \let\oldtabular\tabular
% \let\endoldtabular\endtabular
% \renewenvironment{tabular}[1]{%
%   \oldtabular{#1}%
% }{%
%   \endoldtabular
% }

% \usepackage{longtable,booktabs,xcolor}
% \rowcolors{2}{gray!10}{white}   % filas alternadas
% \renewcommand{\arraystretch}{1.2} % espacio vertical entre filas

% % Mostrar siempre el número de la tabla
% \usepackage{caption}
% \captionsetup[table]{labelformat=default, labelsep=colon, textfont=bf}


% ===========================
% Estilos para tikz y figures
% ===========================

\usepackage{caption}
\captionsetup{
    font={it},       % fuente en cursiva
    labelfont={},  % etiqueta ("Figura 1") en negrita
    textfont={it},   % texto del caption en cursiva
    justification=centering,  % centra el texto (opcional)
    font={small},    % tamaño de fuente pequeño
}

\usepackage{tikz}
\usetikzlibrary{positioning}

\tikzset{
  state/.style={
    draw,
    circle,
    minimum size=1cm,
    thick,
    fill=yellow!20
  },
  block/.style={
    rectangle,
    draw,
    fill=blue!10,
    rounded corners,
    text centered,
    minimum height=1cm,
    minimum width=2cm,
    thick
  },
  none/.style={
    draw=none,
    fill=none,
    text centered
  },
  error/.style={
    draw,
    circle,
    minimum size=1cm,
    thick,
    fill=red!30
  },
  initial text={}
}   % estilos de secciones, etc.

% ========================
% Configuración índice y listas
% ========================
\setlength{\cftbeforesecskip}{5pt}
\setlength{\headheight}{14pt}  % un poco más que 13.6pt

\renewcommand{\normalsize}{\fontsize{10}{12}\selectfont}

% Fix para listas de Pandoc
\providecommand{\tightlist}{%
  \setlength{\itemsep}{0pt}\setlength{\parskip}{0pt}}


%=======================
% fancy with parameters
%=======================
%\fancyfoot[L]{\scriptsize\itshape Dirección de Operaciones 1}
\fancyfoot[L]{\scriptsize\itshape Dirección de Operaciones
1} % pie de página izquierdo en cursiva

\setcounter{tocdepth}{1} % Muestra solo hasta subsecciones en el índice

% ========================
% Inicio del documento
% ========================
\begin{document}

%% portada.tex
\begin{titlepage}
    \newgeometry{top=2cm,bottom=2cm,left=2.5cm,right=2.5cm} % márgenes personalizados
    
    % Fondo con transparencia
    \begin{tikzpicture}[remember picture,overlay]
        \node[opacity=0.15,inner sep=0pt] at (current page.center)
            {\includegraphics[width=\paperwidth,height=\paperheight]{../../img/fondoPrueba.jpg}};
    \end{tikzpicture}

    % Contenido de la portada
    \begin{center}
        \vspace*{2cm}
        
        {\Huge \bfseries\scshape Título del Libro de Apuntes \par}
        \vspace{0.5cm}
        {\Large \itshape Subtítulo o Asignatura \par}
        \vspace{0.5cm}
        {\Large \itshape \href{https://ismael-sallami.github.io}{https://ismael-sallami.github.io} \par}


        \vfill
        
        {\LARGE Autor: \textbf{Tu Nombre Completo} \par}
        \vspace{0.3cm}
        % {\Large Universidad Ejemplo \par}
        
        \vspace{1cm}
        \includegraphics[width=0.25\textwidth]{../../img/ugr.png} % opcional: logo
        \vspace{1cm}
        
        {\large \today}
    \end{center}
    
    \restoregeometry
\end{titlepage}



%==========================
% PORTADA: ENTRADA MANUAL
%==========================

% portada.tex
\begin{titlepage}
    \newgeometry{top=2cm,bottom=2cm,left=2.5cm,right=2.5cm} % márgenes personalizados
    
    % Fondo con transparencia
    \begin{tikzpicture}[remember picture,overlay]
        % \node[opacity=0.15,inner sep=0pt] at (current page.center)
        \node[inner sep=0pt] at (current page.center)
            {\includegraphics[width=\paperwidth,height=\paperheight]{../../../extraFiles/img/fondo_ade.jpg}};
    \end{tikzpicture}

    % Contenido de la portada
    \begin{center}
        \vspace*{2cm}
        
        {\Huge \bfseries\scshape Teoría y Práctica \par}
        \vspace{0.5cm}
        {\Large \itshape Dirección de Operaciones 1 \par}
        \vspace{0.5cm}
        {\small \itshape \href{https://ismael-sallami.github.io}{https://ismael-sallami.github.io} \par}
        {\small \itshape \href{https://elblogdeismael.github.io}{https://elblogdeismael.github.io} \par}


        \vfill
        
        {\LARGE Autor: \textbf{Ismael Sallami Moreno} \par}
        \vspace{0.3cm}
        % {\Large Universidad de Granada \par}
        
        \vspace{1cm}
        \includegraphics[width=0.25\textwidth]{../../../extraFiles/img/ugr.png} % opcional: logo
        \vspace{1cm}
        
        {\large \today}
    \end{center}
    
    \restoregeometry
\end{titlepage}


% ===============================
% licencia.tex
% ===============================
\begin{tikzpicture}[remember picture,overlay]
\node[anchor=south west, xshift=1cm, yshift=1cm] at (current page.south west) {
\begin{minipage}{0.4\textwidth}
\begin{flushleft}
\section*{Licencia}

Este trabajo está bajo una 
\href{https://creativecommons.org/licenses/by-nc-nd/4.0/}{Licencia Creative Commons BY-NC-ND 4.0}.

\bigskip

Permisos: Se permite compartir, copiar y redistribuir el material en cualquier medio o formato.

\bigskip

Condiciones: Es necesario dar crédito adecuado, proporcionar un enlace a la licencia e indicar si se han realizado cambios. No se permite usar el material con fines comerciales ni distribuir material modificado.

\bigskip

\begin{center}
  \href{https://creativecommons.org/licenses/by-nc-nd/4.0/}{\includegraphics[width=0.35\textwidth]{../../../extraFiles/img/by-nc-nd.png}}
\end{center}
\end{flushleft}
\end{minipage}
};
\end{tikzpicture}
  % licencia
\thispagestyle{empty} % quitar número de página en la portada
\clearpage

% --- Índice ---
\tableofcontents
\listoffigures
\clearpage

\listoftables
\clearpage
\thispagestyle{empty} % quitar número de página en la portada
\clearpage

% --- Contenido Markdown generado por Pandoc ---
\part{Teoría}

\hypertarget{introducciuxf3n}{%
\chapter{Introducción}\label{introducciuxf3n}}

La asignatura de Dirección de Operaciones tiene como objetivo principal
proporcionar a los estudiantes los conocimientos y herramientas
necesarios para gestionar eficientemente los procesos de producción y
operaciones en una organización. A través de esta materia, se busca
desarrollar habilidades para la toma de decisiones estratégicas,
optimización de recursos y mejora continua, con el fin de alcanzar los
objetivos empresariales y adaptarse a un entorno competitivo y en
constante cambio.

Se abordarán conceptos fundamentales, metodologías y tendencias actuales
en la dirección de operaciones, con un enfoque práctico que permita
aplicar los conocimientos adquiridos en situaciones reales del ámbito
empresarial.

Aunque las clases de prácticas están divididas en dos grupos, se
recomienda venir a ambas para realizar la mayor cantidad de problemas y
además, conseguir puntuación extra.

\hypertarget{recomendaciones-bibliogruxe1ficas}{%
\section{Recomendaciones
bibliográficas}\label{recomendaciones-bibliogruxe1ficas}}

\begin{itemize}
\tightlist
\item
  ARIAS ARANDA, D. y MINGUELA RATA, B. (Coords.) (2024). Decisiones
  estratégicas de la Dirección de la producción y operaciones. Madrid:
  Ed Pirámide. DISPONIBLE EN FORMATO ELECTRÓNICO\\
\item
  HEIZER J., RENDER B. (2012). Dirección de la producción y de
  operaciones. Decisiones estratégicas. Madrid: Pearson-Prentice Hall.
  DISPONIBLE EN FORMATO ELECTRÓNICO\\
\item
  Material complementario en la Guía Docente y diapositivas.
\end{itemize}

\hypertarget{introducciuxf3n-a-la-direcciuxf3n-de-operaciones}{%
\chapter{Introducción a la dirección de
operaciones}\label{introducciuxf3n-a-la-direcciuxf3n-de-operaciones}}

\begin{definicion}[Chief Executive Officer]
El CEO, o Chief Executive Officer, es el máximo responsable de la gestión y dirección estratégica de una empresa. Su función principal es tomar decisiones clave para el éxito de la organización, establecer objetivos a largo plazo y representar a la empresa frente a accionistas, clientes y otras partes interesadas.
\end{definicion}

\begin{definicion}[Chief Operating Officer]
El COO, o Chief Operating Officer, es el encargado de supervisar las operaciones diarias de la empresa. Su responsabilidad incluye garantizar que los procesos internos sean eficientes, implementar estrategias diseñadas por el CEO y coordinar las actividades de los diferentes departamentos para alcanzar los objetivos organizacionales.
\end{definicion}

Como diferencia entre CEO y COO podemos mencionar (además de
características del COO):

\begin{itemize}
\tightlist
\item
  El CEO está por encima del COO.\\
\item
  El COO se dedica a un área específica como el marketing, dirección de
  operaciones, \ldots{}\\
\item
  En muchas empresas el cargo más importante de la empresa es el de
  CEO.\\
\item
  Reporta directamente al CEO.\\
\item
  Es el principal ejecutor de estrategias de la empresa.\\
\item
  Funciones y responsabilidades sobre el funcionamiento del corazón de
  la empresa.\\
\item
  En las empresas pequeñas lo más seguro es que no haya CEO.\\
\item
  Ver Ejemplo 1.4 del libro dirección de la producción, Arias y
  Minguela, 2024\\
\item
  COO: chief of operations office
\end{itemize}

\hypertarget{la-direcciuxf3n-de-operaciones-en-la-organizaciuxf3n}{%
\section{La dirección de operaciones en la
organización}\label{la-direcciuxf3n-de-operaciones-en-la-organizaciuxf3n}}

La producción es la creación de bienes y servicios. La Dirección de
operaciones engloba la serie de actividades relacionadas con la
producción de bienes y servicios. La Dirección de operaciones engloba la
serie de actividades relacionadas con la producción de bienes y
servicios mediante la transformación de los recursos productivos,
conocidos como inputs, en productos, conocidos como outputs. El objetivo
de esto es maximizar la productividad en la producción.

La secuencia de pasos es básicamente:

\begin{enumerate}
\def\labelenumi{\arabic{enumi}.}
\tightlist
\item
  \textbf{Inputs}: materias primas, mano de obra, \ldots{}\\
\item
  \textbf{Proceso}: procesamiento de transacciones, operaciones de
  vuelo, \ldots{}\\
\item
  \textbf{Outputs}: préstamos, transportes de pasajeros, \ldots{}
\end{enumerate}

Como empresas encontramos:

\begin{itemize}
\tightlist
\item
  Empresas \textbf{industriales}: fabrican productos o bienes
  tangibles.\\
\item
  Empresas de \textbf{servicios}: satisfacen necesidades que tienen
  valor en el mercado.
\end{itemize}

Se define a los servicios como el conjunto de actividades relacionadas
con el mantenimiento y reparación, a la hostelería, transporte,
medicina, \ldots{} Las características de un servicio son:

\begin{itemize}
\tightlist
\item
  Intangible\\
\item
  Producción y consumo simultáneo\\
\item
  Interacción con el cliente: lo que se conoce como co-producción\\
\item
  Unicidad\\
\item
  No patentables\\
\item
  Difícil medir la calidad
\end{itemize}

Hoy día, podemos apreciar que muchas empresas lo que se centran en
vender son servicios. Esto se conoce como \emph{servitización}, que se
define como la creación de valor que conduce a la oferta integrada de
bienes y servicios por parte de las organizaciones. Esto abre varias
oportunidades de negocios. Usando el ejemplo de la empresa de Apple,
podemos ver que las empresas industriales enfocadas en la producción,
rediseñan su modelo de negocio para sacar más provecho y de esta manera
prestar servicios como iCloud.

\underline{Indicadores de la servitización:}

\begin{itemize}
\tightlist
\item
  Número de servicios\\
\item
  Profundidad (intensidad): ingresos directos o indirectos obtenidos por
  servicios.\\
\item
  Tipos de servicios:

  \begin{itemize}
  \tightlist
  \item
    Básicos: centrados en la competencias de producción de empresas
    industriales.\\
  \item
    Servicios de apoyo o intermediarios: servicios más amplios basados
    en la producción existente de servicios de mantenimiento del
    cliente.\\
  \item
    Servicios avanzados: actividades ampliadas que suelen ser internas
    al cliente. No se centran en un bien físico en sí, sino en el
    rendimiento de su funcionamiento.
  \end{itemize}
\end{itemize}

\hypertarget{historia-de-la-direcciuxf3n-de-operaciones}{%
\section{Historia de la dirección de
operaciones}\label{historia-de-la-direcciuxf3n-de-operaciones}}

Comenzó Adam Smith (1776) con la ``riqueza de las naciones'' donde se
centraron en la división del trabajo, mejora de las destrezas de los
trabajadores y ahorro de tiempo. Luego llegó la revolución industrial. A
continuación, los estudios de tiempos y movimientos con Federik Taylor
(1881) consiguiendo mayor productividad, formación, \ldots{}
Desembocando finalmente en la producción en línea y el principio de
fabricación en masa con Henry Ford (1908).

La historia de la dirección de operaciones puede dividirse en varias
etapas clave que reflejan la evolución de las prácticas y enfoques en la
gestión de la producción y operaciones:

\begin{enumerate}
\def\labelenumi{\arabic{enumi}.}
\item
  \textbf{Era preindustrial}: Caracterizada por \emph{gremios}
  artesanales y \emph{economías} \emph{domésticas}. Ejemplos incluyen
  las pirámides egipcias, la muralla china y expediciones por
  continentes.
\item
  \textbf{Revolución industrial}: Introducción de la \emph{mecanización}
  en la industria, aparición de la \emph{máquina} de \emph{vapor} y gran
  protagonismo de la industria \emph{textil}.
\item
  \textbf{Primera Revolución Industrial (Frederik W. Taylor, 1881)}:
  Estudios de \emph{tiempos y movimientos} que llevaron a mayor
  productividad, formación, metodologías de trabajo y sistemas de
  incentivos.
\item
  \textbf{Segunda Revolución Industrial (Henry Ford, 1908)}:
  Implementación de la línea de ensamblaje, avances en la industria del
  petróleo y electricidad, gestión de proyectos (gráficos de Gantt,
  análisis de procesos de Taylor), muestreo estadístico y programación
  de necesidades de materiales (MRP).
\item
  \textbf{Era posindustrial}: Desarrollo de la informática e internet,
  relevancia de la gestión de la cadena de suministro y nuevas
  tecnologías que permiten flexibilidad y disminución de costes. Daniel
  Bell (1973) destacó la importancia de los servicios y su contribución
  al PIB.
\end{enumerate}

Estas etapas reflejan cómo la dirección de operaciones ha evolucionado
desde enfoques manuales y artesanales hacia sistemas altamente
automatizados y tecnológicos, adaptándose a las necesidades de cada
época.

Las revoluciones que llevamos son:

\begin{itemize}
\tightlist
\item
  Industry 1.0: mecanización, \ldots{}\\
\item
  Industry 2.0: producción en masa, \ldots{}\\
\item
  Industry 3.0: computer, \ldots{}\\
\item
  Industry 4.0: machine learning, \ldots{}\\
\item
  Industry 5.0: robots, \ldots{}\\
\item
  La siguiente será el uso masivo de IA y la colaboración de
  humano-máquina para mejorar la productividad, sin dejar de lado la
  sostenibilidad.
\end{itemize}

\hypertarget{tendencias-en-la-direcciuxf3n-de-operaciones}{%
\section{Tendencias en la dirección de
operaciones}\label{tendencias-en-la-direcciuxf3n-de-operaciones}}

Actualmente hay cierta sobrecarga de información en internet, pero el
internet y la información facilita las transacciones a nivel mundial.
Una empresa debe adaptarse a los clientes, sociedad en general,
proveedores y trabajadores para mejorar.

Hace 10 años el tema de compra era distinto, ya que las compras eran
presenciales, con dinero en efectivo, ya que se tenía miedo a este tipo
de ventas. El ecommerce español crece un 17\%, superando a la UE. El
hecho de las ventas online hace que se cierren varias empresas, como es
el caso de inditex, pero esto no quiere decir que se estén generando
menos ventas ni menos trabajo, al revés, se propone empleo en este
sector con empresas online con un mayor tamaño y mayor requerimiento de
empleados. Por otro lado, las compras online de comida no están teniendo
tanto éxito.

\begin{itemize}
\tightlist
\item
  Big Data e Industria 4.0: se usa por ejemplo, en el fútbol, para
  analizar a los jugadores, rendimiento, \ldots{} El año pasado la
  inversión en IA era de 27.000 mil millones, mientras que este año ha
  pasado a los 400.000 mil millones. ¿Cómo afectará la industria 4.0 y
  5.0 a la docencia? Suponen una amenaza ya que, por ejemplo, el hecho
  de las tutorías desaparecerán.
\item
  Producción bajo demanda: gracias a la tecnología podemos personalizar
  más los gustos de los consumidores para las distintas empresas.
  Ejemplo: hay empresas que te permiten personalizar tus propias
  zapatillas.
\item
  Colaboración en la cadena de suministro y gestión centrada en la
  cadena de suministro. Ejemplo: Alibaba que aterriza en España hace 8
  años, siendo su mercado principal China.
\item
  Enfoque de respuesta rápida. Tesla, en un principio, quería conseguir
  diferenciación de poder, pero debido a la alta competencia que ha ido
  llegando y que consiguen producir a más bajo coste, su estrategia se
  ha visto afectada. Incluye elementos de Lean Manufacturing y
  Producción justo a tiempo.
\item
  Economía circular y sostenibilidad medioambiental. La economía
  circular es un modelo económico que busca reducir al mínimo el uso de
  recursos y la generación de residuos, promoviendo un ciclo continuo de
  reutilización, reparación, reciclaje y valorización de materiales y
  productos, de esta manera se esta contribuyenfo a la sostenibilidad
  medioambiental.
\item
  Flexibilidad de las operaciones. Se debe de recalcar la
  \emph{obsolescencia}.
\item
  Empoderamiento de los usuarios. Ejemplo: la plataforma específica que
  tiene Lego sobre las ideas de los usuarios.
\end{itemize}

La estrategia de las operaciones permite obtener cierta ventaja
competitiva, partiendo de la misión y orientando al estrategia de cada
añ liderazgo de costes, diferenciación y capacidad de respuesta.

\begin{itemize}
\tightlist
\item
  Visión de una empresa: planteamiento que tiene la empresa de cara al
  futuro.
\item
  Misión de una empresa: acciones que quiere desempeñar para conseguir
  cierto número de ventas, ventas de servicio o de producto. Podemos
  distinguirlo en base al verbo (ofrecer, crear, \ldots).
\end{itemize}

\underline{Tipos de Estrategias}

\begin{itemize}
\tightlist
\item
  Para la \textbf{estrategia de diferenciación} se busca que el cliente
  aprecie un valor añadido. Puede abarcar cualquier aspecto que influya
  en el valor que le dan los consumidores:

  \begin{itemize}
  \tightlist
  \item
    Amplia gama de productos.\\
  \item
    Funcionalidades del producto o servicio relacionado con el
    producto.\\
  \item
    Sector servicios: diferenciación por experiencia, involucrar al
    cliente.
  \end{itemize}
\item
  Estrategia de \textbf{competencia en costes}: se centra en coseguir
  mejores costes que la competencia, ya sea consiguiendo mano de obra
  más barata o mediante otros medios, sin dejar de lado las expectativas
  de sus clientes, las cuales debe de cumplir. Además, no debe de bajar
  la calidad en gran medida. Ejemplo: Ikea, Primark, \ldots{}\\
\item
  Estrategia de \textbf{competencia en respuesta}: ofrecer mejor
  respuesta que la competencia, teniendo en cuenta el tiempo previsto,
  la programación fiable y una ejecución flexible. Ejemplo: Reparto los
  domingos por parte de Amazon.
\end{itemize}

Los objetivos varía notablemente dependiendo de la estrategia de
empresa:

\underline{Objetivos clásicos:}

\begin{itemize}
\tightlist
\item
  Coste o eficiencia: mano de obra, materiales, \ldots{}
\item
  Calidad

  \begin{itemize}
  \tightlist
  \item
    Externa: Satisfacer necesidades de los clientes.
  \item
    Interna: Valoración del cliente.
  \end{itemize}
\item
  Flexibilidad: modificar el producto fácilmente.
\item
  Plazo de Entrega: Rapidez de la entrega, fiabilidad, \ldots{}
\end{itemize}

\underline{Nuevos Objetivos:}

\begin{itemize}
\tightlist
\item
  Servicio

  \begin{itemize}
  \tightlist
  \item
    Postventa: problemas, opiniones, \ldots{}
  \item
    Preventa: fabricación, \ldots{}
  \end{itemize}
\item
  Innovación

  \begin{itemize}
  \tightlist
  \item
    Radical: Un gran cambio.
  \item
    Incremental: Lanzar un producto con una funcionalidad adicional.
  \end{itemize}
\item
  Ecoeficiencia: Cumplir con la legislación, prevenir legislación,
  productos ecológicos, liderazgo medioambiental que potencie la imagen
  de la empresa.
\end{itemize}

\underline{Decisiones}

\begin{itemize}
\tightlist
\item
  Decisiones de carácter estratatégico:

  \begin{itemize}
  \tightlist
  \item
    Diseño del producto y del servicio\\
  \item
    Diseño de procesos y planificación de capacidad\\
  \item
    Gestión de la calidad\\
  \item
    Localización\\
  \item
    Distribución en planta o layout\\
  \item
    Recursos humanos y diseño del trabajo\\
  \end{itemize}
\item
  Decisiones tácitas y operativas:

  \begin{itemize}
  \tightlist
  \item
    Gestión de la cadena de suministro\\
  \item
    Gestión de inventarios\\
  \item
    Planificación y programación del proyecto\\
  \item
    Mantenimiento
  \end{itemize}
\end{itemize}

\underline{Planificación de la estrategia global de la empresa}

\begin{itemize}
\tightlist
\item
  Nivel corporativo: Actuaciones enfocadas a obtener una ventaja
  competitiva.
\item
  Nivel competitivo: crear la ventaja competitiva.
\item
  Nivel funcional: actuación que se lleva a cab para deplegar los
  recursos.
\end{itemize}

\underline{Técnicas de toma de decisiones}

\begin{itemize}
\tightlist
\item
  Árboles de decisión.
\item
  Análisis Coste-Volumen-Beneficio.
\item
  Factores ponderados: importancia que el decisor le otorgue.
\item
  Método del centro de gravedad: esta técnica pretende determinar la
  localización de una instalación desde la que se distribuirán los
  productos de manera que minimice el coste total del transporte.
\item
  Equilibrado de cadenas: la problemática radica en la posibilidad de
  subdividir el flujo de trabajo lo suficiente como para que el personal
  y los equipos sean utilizados de la forma mñas ajustada posible a lo
  largo del proceso.
\end{itemize}

\incluirimagen{media/tecnicastomadecisiones.png}{Ejemplos de las diferentes técnicas de tomas de decisiones}

\hypertarget{gestiuxf3n-de-la-cadena-de-suministro}{%
\chapter{Gestión de la cadena de
suministro}\label{gestiuxf3n-de-la-cadena-de-suministro}}

\hypertarget{importancia-estratuxe9gica}{%
\section{Importancia estratégica}\label{importancia-estratuxe9gica}}

Las compras representan un porcentaje importante de los costes de
empresa. El personal, representa de la misma manera un coste elevado
debido a las diversas contribuciones que se añade a la nómina. Por otro
lado, las relaciones con los proveedores deben de estar más integradas y
ser a más largo plazo. Los proveedores como socios que contribuyen a
conseguir una ventaja competitiva.

La cadena de suministro esta compuesta por:

PROVEEDORES \(\rightarrow\) FABRICANTES Y PROVEEDORES DE SERVICIOS
\(\rightarrow\) MAYORISTAS/MINORISTAS \(\rightarrow\) CLIENTE FINAL

El objetivo es coordinar actividades de la cadena para maximizar la
ventaja competitiva y beneficios para el consumidor final.

Encontramos varios niveles de estrategia:

\begin{itemize}
\tightlist
\item
  Estrategia corporativa
\item
  Esttrategia competitiva
\item
  Estrategia funcional de operaciones
\item
  Estrategia de la cadena de suministros
\end{itemize}

\begin{table}[H]
\centering
\caption{Estrategias en la cadena de suministro}
\resizebox{\textwidth}{!}{%
\begin{tabular}{|p{4.5cm}|p{3.5cm}|p{3.5cm}|p{3.5cm}|}
\hline
\textbf{Aspecto} & \textbf{Estrategia de bajo coste} & \textbf{Estrategia de respuesta rápida} & \textbf{Estrategia de diferenciación} \\ \hline
\textbf{Principales criterios de SELECCIÓN DE PROVEEDORES} & Coste & Capacidad, Velocidad, Flexibilidad & Habilidades para el desarrollo de productos, Voluntad de compartir información, Desarrollo rápido y conjunto de productos \\ \hline
\textbf{INVENTARIO de la cadena de suministros} & Minimizar inventario para mantener bajos los costes & Utilizar stocks de reserva para asegurar un suministro rápido & Minimizar inventario para evitar la obsolescencia del producto \\ \hline
\textbf{Red de DISTRIBUCIÓN} & Transporte barato, Venta a través de distribuidores/tiendas de descuento & Transporte rápido, Prestación de un servicio excelente al cliente & Reunir y comunicar los datos de estudios de mercado, Reunir las ventas por ventas experto \\ \hline
\textbf{Características del DISEÑO DEL PRODUCTO} & Maximizar el rendimiento, Minimizar los costes & Diseño que permita bajos tiempos de preparación de los procesos productivos, Incremento rápido de la producción & Diseño por módulos para facilitar la diferenciación del producto \\ \hline
\end{tabular}%
}
\end{table}

\begin{table}[H]
\centering
\caption{Definiciones clave de la Gestión de la Cadena de Suministro (GCS)}
\begin{tabular}{|l|p{12cm}|}
\hline
\textbf{Autores} & \textbf{Concepto GCS} \\ \hline
\textbf{Jones y Riley (1985)} & La gestión del flujo total de materiales y de información desde los proveedores iniciales de materia prima hasta que el consumidor final recibe su producto o servicio. \\ \hline
\textbf{Christopher (1998)} & El conjunto de empresas que están vinculadas, a través de relaciones con otras, en los diferentes procesos y actividades que generan valor en forma de productos y servicios para el consumidor final. \\ \hline
\textbf{Ballou (2004)} & Una red constituida por todas las organizaciones y personas involucradas en el flujo de materia prima, productos elaborados, información y dinero, desde los proveedores hasta el consumidor final. \\ \hline
\textbf{Espitia y López (2005)} \\ \textbf{Arias y Minguela (2018)} & La coordinación sistemática y estratégica de las funciones de negocio dentro de una empresa en particular y a lo largo de todas aquellas empresas implicadas en la cadena, con el propósito de mejorar el rendimiento a largo plazo de cada parte y de la cadena en global. \\ \hline
\end{tabular}
\end{table}

\begin{table}[H]
\centering
\caption{Definiciones clave de la Gestión de la Cadena de Suministro (GCS)}
\begin{tabular}{|l|p{12cm}|}
\hline
\textbf{Autores} & \textbf{Concepto GCS} \\ \hline
\textbf{Jones y Riley (1985)} & La gestión del flujo total de materiales y de información desde los proveedores iniciales de materia prima hasta que el consumidor final recibe su producto o servicio. \\ \hline
\textbf{Christopher (1998)} & El conjunto de empresas que están vinculadas, a través de relaciones con otras, en los diferentes procesos y actividades que generan valor en forma de productos y servicios para el consumidor final. \\ \hline
\textbf{Ballou (2004)} & Una red constituida por todas las organizaciones y personas involucradas en el flujo de materia prima, productos elaborados, información y dinero, desde los proveedores hasta el consumidor final. \\ \hline
\textbf{Espitia y López (2005)} \\ \textbf{Arias y Minguela (2018)} & La coordinación sistemática y estratégica de las funciones de negocio dentro de una empresa en particular y a lo largo de todas aquellas empresas implicadas en la cadena, con el propósito de mejorar el rendimiento a largo plazo de cada parte y de la cadena en global. \\ \hline
\end{tabular}
\end{table}

Como vemos siempre se centra en la gestión de los flujos internos,
llevándolo a la competencia a nivel de cadenas de suministros. Se
convierte en una herramienta estratégica en el modelo de negocio
abarcando al proveedor y al cliente.

\begin{definicion}[Logística]
Parte del proceso dentro de la cadena de suministro que planifica, implementa y controla el flujo y almacenamiento eficiente y efectivo de los bienes, servicios e información relacionada desde el punto de origen hasta el de consumo satisfaciendo los requerimientos del cliente. El origen se relaciona en el mundo militar con el transporte y almacenamiento.
\end{definicion}

\begin{definicion}[Gestión de la cadena de suministros]
Gestión de la Cadena de Suministro: integración de los procesos clave desde los proveedores hasta el consumidor final para obtener productos, servicios e información que aportan valor para los consumidores y para otros grupos de interés o stakeholders (Global Supply Chain Council).
\end{definicion}

\emph{La cadena de suministro engloba los procesos logísticos y los no
logísitcos.}

\incluirimagen{media/logistica.png}{Logística dentro de la cadena de suministro}

\hypertarget{elementos-y-procesos}{%
\section{Elementos y Procesos}\label{elementos-y-procesos}}

La diferencia entre mayorista y minorista es, fundamentalmente, a quien
vende, el minorista vende al consumidor final directamente, mientras que
el mayorista no.

\emph{Los elementos clave de la GCS son los proveedores, fabricantes y
los distribuidores}.

Ejemplo de que una misma empresa produce y vende es Zara, mientras que
en el caso de aquellas son su propio proveedor, destacan las
energéticas, como es el caso de Repsol.

\underline{Diferencia entre canal de distribución y red de distribución}

La red compone todas las diferentes línea que me llevan al destino,
mientras que un canal es solo una línea de esa red como es el canal
directo.

\begin{definicion}[Omnichannel]
Buena gestión de una empresa, por ejemplo, cuando se trata de negocios onlines, estos deben de tener una buena gestión del inventario para un buen rendimiento del negocio.
\end{definicion}

La competitividad y la rentabilidad pueden verse incrementadas si las
actividaddes críticas que realiza la emresa en relación a la CGS son
realizadas de manera alineada con las de las otras empresas de la cadena
con las de las otras empresas de la cadena consiguiendo una integración
total.

\incluirimagen{media/procesos-clave.png}{Procesos clave que deben estar integrados en la GCS}

\begin{definicion}[Gestión de las relaciones con clientes]
Segmentación de los clientes para ofrecer a cada segmento los productos o servicios que demanden, manteniendo sus niveles de satisfacción y con el menor coste posible para la cadena.
\end{definicion}

\begin{definicion}[Gestión del servicio al cliente]
Puntos de contacto de la empresa con el cliente y gestionar las incidencias o reclamaciones.
\end{definicion}

\begin{definicion}[Gestión de la demanda]
Se busca alcanzar el equilibrio entre las necesidades de producción y/o solicitudes de los clientes (demanda) y las capacidades reales de producción de la cadena de suministro (capacidad productiva-oferta), flujo sin interrupciones.
\end{definicion}

\begin{definicion}[Gestión del flujo de producción]
Engloba todas las actividades para la elaboración de los productos.
\end{definicion}

\begin{definicion}[Cumplimiento de los pedidos]
Incluye todas las actividades que son necesarias para crear y gestionar una red que pueda cumplir con todas las solicitudes de los clientes en plazo y cantidad, además de minimizar los costes de los envíos.
\end{definicion}

\begin{definicion}[Gestión de las relaciones con los proveedores]
Seleccionar un grupo de proveedores con los que mantener relaciones a largo plazo. Se busca que ambas partes ganen (win-win situation).
\end{definicion}

\begin{definicion}[Desarrollo y comercialización de nuevos productos]
Se integran las aportaciones de clientes y proveedores, para reducir el tiempo necesario para introducir un nuevo producto en el mercado. Se revisan las etapas de aprovisionamiento, producción, distribución y marketing por si se tienen que reajustar.
\end{definicion}

\begin{definicion}[Devoluciones]
Incluye todas las actividades relacionadas con las devoluciones de clientes (Logística inversa).
\end{definicion}

\hypertarget{estrategias-de-gestiuxf3n-de-la-cadena-de-suministro}{%
\section{Estrategias de gestión de la cadena de
suministro}\label{estrategias-de-gestiuxf3n-de-la-cadena-de-suministro}}

El \emph{objetivo de la GCS} es satisfacer las necesidades del cliente
final, proporcionándoles el producto o servicio cuando éste lo necesita
y en las cantidades requeriadas, y todo ello a un coste competitivo.

\underline{Qué estrategia elegir atendiendo en el producto y la naturaleza de la demanda(predictibilidad)}

\begin{itemize}
\tightlist
\item
  Productos funcionales: satisfacen necesidades básicas, demanda
  estable, \ldots{}
\item
  Productos innovadores: ciclos de vida corto, precios diferenciados,
  \ldots{}
\end{itemize}

\begin{definicion}[GCS Lean]
Se centra en la eficiencia, costes logísticos bajos, nivel de inventario bajo, eliminar aquello que no aporta valor. Por ende, engloba los productos funcionales.
\end{definicion}

\begin{definicion}[GCS Ágil]
Se centra en la flexibilidad y capacidad de respuesta, alta velocidad de distribución, análisis rápido de los datos.
\end{definicion}

En el caso de Zara, este y otras muchas empresas, pueden tener cadenas
de suministros Lean o Ágil, dependiento del contexto y la situación.

\subsubsection*{Cae en el examen:}

\begin{table}[H]
\centering
\caption{Gestión de la cadena de suministro Lean y Ágil}
\begin{tabular}{|l|p{6cm}|p{6cm}|}
\hline
\textbf{Factores} & \textbf{Cadena de suministro Lean} & \textbf{Cadena de suministro Ágil} \\ \hline
\textbf{Claves} & Eficacia, productividad, economía de escala, eliminación de despilfarros & Rápida respuesta, flexibilidad, satisfacción al cliente \\ \hline
\textbf{Proveedores} & Se comparte información de carácter transaccional, pero con ciertas restricciones & Se comparte un alto volumen de información y de distinta tipología \\ \hline
\textbf{Fabricación} & Sistema de empuje (Push), manufactura focalizada, plantas especializadas & Sistemas de arrastre (Pull), manufactura flexible, sobrecapacidad \\ \hline
\textbf{Almacenamiento} & Sistemas centralizados, bajo nivel de inventario, reducida variedad & Cierta descentralización, alto nivel de inventario, alta variedad \\ \hline
\textbf{Transporte} & Pocos envíos y grandes cantidades, lentitud en el transporte por contratación en base a costes & Muchos envíos y pocas cantidades, rapidez de transporte \\ \hline
\textbf{Información} & Sistemas de captación de datos simples & Sofisticados sistemas de captación de datos y análisis de demanda \\ \hline
\end{tabular}
\end{table}

\begin{itemize}
\tightlist
\item
  Sistema de empuje Push: Se vende lo propio de la empresa.
\item
  Sistema de arrastre Pull: Se traen productos externos a la propia
  empresa.
\end{itemize}

\begin{definicion}[Fabricar o comprar]
Transferir actividades tradicionalmente e internas de la empresa a proveedores externos. Ejemplo: las empresas de smartphones lo que hacen es aportar el diseño a las empresas manufactureras para que lo diseñen.
\end{definicion}

\begin{definicion}[Subcontratar (outsourcing)]
Producir un componente o servicio dentro de la empresa o comprarlo al proveedor. Se externaliza el producto.
\end{definicion}

\begin{table}[H]
\centering
\caption{Resumen de las estrategias de suministro}
\begin{tabular}{|p{4cm}|p{12cm}|}
\hline
\textbf{Estrategia} & \textbf{Descripción} \\ \hline
\textbf{Muchos proveedores} & Cada proveedor responde a las demandas y especificaciones de una petición de oferta de la empresa. Común en productos estándar. Competencia agresiva entre proveedores. \\ \hline
\textbf{Pocos proveedores} & Relación a largo plazo con unos pocos proveedores dedicados. El proveedor obtiene economías de escala y curva de aprendizaje. El coste de cambiar de socios es alto. \\ \hline
\textbf{Integración vertical} & Producir bienes y servicios que anteriormente se compraban o adquirir a un proveedor o distribuidor. Hacia adelante: compra de clientes o distribuidores. Hacia atrás: compra de proveedores. \\ \hline
\textbf{Joint Ventures} & Colaboración formal por la que varias empresas establecen una propiedad común para lanzar nuevos productos o abrir nuevos mercados. \\ \hline
\textbf{Las redes keiretsu} & Proveedores que se convierten en parte de una coalición de empresas. Colaboración + compra a unos pocos proveedores + integración vertical. Relaciones a largo plazo, apoyo financiero, colaboración técnica, lealtad. \\ \hline
\textbf{Empresas virtuales} & Empresas (huecas o en red) que se basan en una variedad de relaciones con proveedores para proporcionar servicios bajo demanda. Están basadas en la externalización. Ej.: gestión de nóminas, contratación de personal, diseño de productos, etc. La cadena de suministro es la empresa. \\ \hline
\end{tabular}
\end{table}

\hypertarget{riesgos-en-la-cadena-de-suministro}{%
\section{Riesgos en la cadena de
suministro}\label{riesgos-en-la-cadena-de-suministro}}

\begin{itemize}
\tightlist
\item
  Efectos de desajuste entre la oferta y la demanda.
\item
  Circunstancias actuales: coste bajo de las comunicaciones y transporte
  rápido, bajo inventario e incremento de la especialización de pocos
  proveedores.
\item
  Incremento del riesgo: se compra más, se fabrica menos, mayor
  dependencia de la cadena de suministro (todo esto es respectivamente
  del punto anterior).
\end{itemize}

\begin{table}[H]
\centering
\caption{Riesgos y tácticas de mitigación en la cadena de suministro}
\begin{tabular}{|p{6cm}|p{10cm}|}
\hline
\textbf{Riesgos} & \textbf{Tácticas de mitigación} \\ \hline
Fallo del proveedor en el envío & Uso de múltiples proveedores; contratos eficaces con penalizaciones; subcontratas disponibles de reserva; planificación por anticipado. \\ \hline
Fallos en la calidad del proveedor & Cuidadosa selección del proveedor, entrenamiento, certificación y supervisión. \\ \hline
Retrasos logísticos o daños & Modalidades de transporte y almacenes múltiples/redundantes; embalaje seguro; contratos eficaces con penalizaciones. \\ \hline
Distribución & Selección cuidadosa, supervisión y contratos eficaces con penalizaciones. \\ \hline
Pérdida o deformación de la información & Bases de datos redundantes; sistemas de IT seguros; entrenamiento de los socios en la cadena de suministros en las correctas interpretaciones y usos de la información. \\ \hline
Político & Seguro de riesgo político; diversificación internacional; franquicias y concesiones. \\ \hline
Económico & Cobertura para combatir el riesgo del tipo de cambio; contratos de compra con protección ante fluctuaciones en los precios. \\ \hline
Catástrofes naturales & Seguro; suministro alternativo; diversificación internacional. \\ \hline
Robo, vandalismo y terrorismo & Seguro; protección de patente; medidas de seguridad que incluyen protección por radiofrecuencia y GPS; diversificación. \\ \hline
\end{tabular}
\end{table}

\hypertarget{uxe9tica-y-sostenibilidad}{%
\section{Ética y sostenibilidad}\label{uxe9tica-y-sostenibilidad}}

\begin{itemize}
\tightlist
\item
  Ética personal: promover la responsabilidades del empleado,
  básicamente el cumplimiento legal.
\item
  Ética dentro de la CS: Se debe de hacer las normas para sus
  proveedores. La sociedad exige comportamiento ético a lo largo de toda
  la cadena.
\item
  Relativo al medio ambiente: Evaluación del impacto medioambiental de
  principio a fin.
\end{itemize}

En cuanto a la sostenibilidad se deben de gestionar los flujos
financieros, de información y demás, para tener en cuenta preocupaciones
sociales y ambientales. Debe de abarcar la inclusión social, equidad
distributiva y el medioambiente.

Debemos de tener en cuenta la entrante o Logística directa, además de
\ldots{[}completar{]}

\begin{definicion}[Logística Inversa]
La logística inversa comprende los procesos para enviar los productos devueltos retrocediendo por la cadena de suministro para la recuperación de su valor o su eliminación: reventa, reparación, reutilización, refabricación, reciclado o eliminación.
\end{definicion}

Una cadena de suministro de bucle cerrado se refiere más al diseño
proactivo de una cadena de suministros que intenta optimizar todos los
flujos hacia delante y hacia atrás.

\begin{thebibliography}{99}

  \bibitem{Referencia1}
  Ismael Sallami Moreno, \textbf{Estudiante del Doble Grado en Ingeniería Informática + ADE}, Universidad de Granada, 2025.
  
  \bibitem{DiapositivasAsignatura}
  Universidad de Granada, \emph{Diapositivas de la asignatura}, Curso 2025/2026.

  % \bibitem{Referencia2}
  % Autor Apellido, \emph{Título del libro o artículo}, Editorial o Revista, Año.
  
  % \bibitem{Referencia3}
  % Nombre Autor, \emph{Título del documento}, Conferencia/URL, Año.
  
  \end{thebibliography}
  

\end{document}
