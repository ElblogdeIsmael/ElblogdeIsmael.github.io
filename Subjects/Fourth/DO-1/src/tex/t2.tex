\chapter{Gestión de la Cadena de Suministro}

\section{Importancia Estratégica}
La \textbf{Gestión de la Cadena de Suministro (GCS)}, o \textit{Supply Chain Management (SCM)}, se ha consolidado como una herramienta estratégica fundamental en el modelo de negocio empresarial, trascendiendo su concepción inicial como una mera operación logística. Su objetivo principal es coordinar todas las actividades dentro de la cadena, desde los proveedores iniciales hasta el consumidor final, para maximizar su ventaja competitiva y los beneficios percibidos por el cliente.

El concepto ha evolucionado desde una gestión centrada en los flujos internos de la empresa hacia un enfoque de integración con proveedores y clientes. Este cambio de paradigma implica que la competencia ya no se produce entre empresas individuales, sino a nivel de cadenas de suministro. Las compras representan un porcentaje significativo de los costes de una empresa, por lo que la gestión eficiente de las relaciones con los proveedores, considerándolos "socios" estratégicos, es clave para obtener una ventaja competitiva.

\subsection{Concepto de Cadena de Suministro}
A lo largo del tiempo, diversos autores han definido la GCS con un alcance progresivamente más amplio.
\begin{itemize}
    \item \textbf{Jones y Riley (1985):} La gestión del flujo total de materiales e información, desde los proveedores de materias primas hasta la entrega al consumidor final.
    \item \textbf{Christopher (1998):} El conjunto de empresas interrelacionadas en los procesos y actividades que generan valor en forma de productos y servicios para el cliente final.
    \item \textbf{Ballou (2004):} Una red de organizaciones y personas involucradas en el flujo de materia prima, productos, información y dinero, desde los proveedores hasta el consumidor.
    \item \textbf{Arias y Minguela (2018):} La coordinación sistemática y estratégica de las funciones de negocio, tanto dentro de una empresa como entre las empresas de la cadena, con el fin de mejorar el rendimiento a largo plazo de cada parte y de la cadena en su conjunto.
\end{itemize}

\subsection{Diferencia entre GCS y Logística}
Es importante distinguir la GCS de la logística, aunque a menudo se usen como sinónimos. La \textbf{logística} es la parte del proceso de la cadena de suministro que planifica, implementa y controla el flujo y almacenamiento de bienes, servicios e información desde el origen hasta el consumo para satisfacer los requerimientos del cliente. Su origen se remonta al ámbito militar y tradicionalmente se asocia al transporte y almacenamiento.

La GCS tiene un alcance mucho más amplio, ya que integra procesos clave que van más allá del movimiento de bienes, abarcando la gestión de la oferta y la demanda dentro y entre empresas. La logística empresarial se puede segmentar en:
\begin{itemize}
    \item \textbf{Logística de entrada (\textit{inbound logistics}):} Corresponde al proceso de aprovisionamiento.
    \item \textbf{Logística interna:} Vinculada a los movimientos de materiales dentro del proceso de producción.
    \item \textbf{Logística de salida (\textit{outbound logistics}):} Relacionada con el proceso de distribución del producto final.
\end{itemize}

\subsection{Impacto de la Estrategia Corporativa en la GCS}
La estrategia de la cadena de suministro se enmarca dentro de la jerarquía estratégica de la empresa (corporativa, competitiva y funcional). Las decisiones de la GCS deben estar alineadas con la estrategia corporativa, que puede ser de bajo coste, de respuesta rápida o de diferenciación. La Tabla siguiente detalla este impacto:

\begin{table}[H]
\centering
\caption{Impacto de la Estrategia Corporativa en las decisiones de la Cadena de Suministro.}
\begin{tabular}{p{0.2\textwidth} p{0.2\textwidth} p{0.2\textwidth} p{0.2\textwidth}}
\toprule
 & \textbf{Estrategia de bajo coste} & \textbf{Estrategia de respuesta rápida} & \textbf{Estrategia de diferenciación} \\
\midrule
\textbf{Selección de Proveedores} & Basada en el coste. & Basada en capacidad, velocidad y flexibilidad. & Basada en habilidades para el desarrollo de productos. \\
\textbf{Inventario} & Minimizar para reducir costes. & Utilizar stocks de reserva para asegurar rapidez. & Minimizar para evitar obsolescencia. \\
\textbf{Distribución} & Transporte económico, venta a través de distribuidores de descuento. & Transporte rápido, servicio al cliente excelente. & Recopilar y comunicar datos de mercado. \\
\textbf{Diseño del Producto} & Maximizar rendimiento y minimizar costes. & Diseño que permita bajos tiempos de preparación y rápido incremento de producción. & Diseño modular que facilite la diferenciación. \\
\bottomrule
\end{tabular}
\end{table}

\section{Elementos y Procesos}

\subsection{Elementos Clave de la GCS}
Toda cadena de suministro está compuesta, en general, por tres elementos o eslabones fundamentales:
\begin{itemize}
    \item \textbf{Proveedores:} Se organizan en distintos niveles. El proveedor de primer nivel suministra directamente al fabricante, el de segundo nivel al de primer nivel, y así sucesivamente.
    \item \textbf{Fabricantes:} Transforman los materiales y componentes (\textit{inputs}) en productos acabados. Pueden operar en una o varias fábricas, lo que afecta a la complejidad y coordinación de la cadena.
    \item \textbf{Distribuidores:} Incluyen mayoristas (venden a otras empresas) y minoristas (venden al cliente final).
\end{itemize}
Es importante destacar que no todos los eslabones deben estar formados por empresas diferentes; existen modelos de negocio con alta integración vertical. Además, una misma empresa, ya sea proveedora o distribuidora, puede formar parte de múltiples cadenas de suministro. Estos conceptos son aplicables tanto a empresas de productos como de servicios.

\subsection{Canales de Distribución}
El canal de distribución es el camino que sigue un producto desde el fabricante hasta el consumidor. Puede ser:
\begin{itemize}
    \item \textbf{Canal Directo:} El fabricante vende directamente al consumidor, sin intermediarios. El uso de Internet como canal directo ha crecido significativamente.
    \item \textbf{Canales Indirectos:} Involucran a uno o más intermediarios (mayoristas, minoristas). Un canal con mayoristas necesariamente debe incluir también a un minorista para llegar al consumidor final.
\end{itemize}

\subsection{Procesos Clave en la GCS}
Para que una cadena de suministro sea competitiva, es crucial la integración de sus procesos de negocio clave. Según el marco de Cooper et al. (1997), estos procesos son:
\begin{enumerate}
    \item \textbf{Gestión de las relaciones con clientes:} Define cómo se desarrollarán y mantendrán las relaciones con los clientes, segmentándolos según sus necesidades para ofrecerles los productos adecuados y mantener su satisfacción al menor coste posible.
    \item \textbf{Gestión del servicio al cliente:} Establece los puntos de contacto con el cliente y gestiona incidencias, buscando resolverlas antes de que afecten al usuario final.
    \item \textbf{Gestión de la demanda:} Busca equilibrar las necesidades del cliente con la capacidad productiva de la cadena para asegurar un flujo ininterrumpido.
    \item \textbf{Gestión del flujo de producción:} Abarca todas las actividades de fabricación, incorporando la flexibilidad necesaria para servir a los clientes.
    \item \textbf{Cumplimiento de los pedidos:} Incluye las actividades necesarias para crear una red que cumpla con las solicitudes de los clientes en plazo y cantidad, minimizando los costes de envío.
    \item \textbf{Gestión de las relaciones con los proveedores:} Proceso análogo a la gestión de clientes, pero enfocado en seleccionar un grupo de proveedores clave para establecer relaciones a largo plazo, buscando un beneficio mutuo (\textit{win-win situation}).
    \item \textbf{Desarrollo y comercialización de nuevos productos:} Integra aportaciones de clientes y proveedores para reducir el tiempo de introducción de un nuevo producto en el mercado.
    \item \textbf{Devoluciones (Logística inversa):} Gestiona todas las actividades relacionadas con el retorno de productos por parte de los clientes.
\end{enumerate}
La integración de estos procesos, tanto a nivel intraorganizacional como interorganizacional, es determinante para generar y mantener ventajas competitivas.

\section{Estrategias de Gestión de la Cadena de Suministro}
No existe una estrategia única de GCS, ya que esta debe adaptarse a la naturaleza del producto y a la predictibilidad de su demanda. Centrándose en estos factores, se pueden identificar dos enfoques principales.
\begin{itemize}
    \item \textbf{Productos funcionales:} Satisfacen necesidades básicas, con demanda estable y predecible, márgenes reducidos y baja variedad.
    \item \textbf{Productos innovadores:} Tienen un ciclo de vida corto, gran variedad, márgenes altos y una demanda difícil de predecir.
\end{itemize}

\subsection{Estrategias Lean y Ágil}
Basándose en la naturaleza de la demanda y en el objetivo de la cadena, surgen dos estrategias fundamentales:
\begin{itemize}
    \item \textbf{GCS Lean (Eficiencia):} Adecuada para productos funcionales, se enfoca en la eficiencia, la productividad y la eliminación de despilfarros para lograr bajos costes logísticos y de inventario. Utiliza sistemas de fabricación de empuje (\textit{push}).
    \item \textbf{GCS Ágil (Respuesta rápida):} Orientada a productos innovadores, prioriza la flexibilidad y la capacidad de respuesta, con una alta velocidad de distribución y selección de proveedores basada en su rapidez. Emplea sistemas de fabricación de arrastre (\textit{pull}).
\end{itemize}
Muchas empresas se ven presionadas a combinar eficiencia y rapidez, lo que ha dado lugar a estrategias híbridas como la \textit{ejecución diferida (postponement)} o el \textit{reaprovisionamiento continuo}.

\subsection{Seis Estrategias de Suministro}
Además de los enfoques lean y ágil, existen seis estrategias de suministro que una empresa puede adoptar para configurar sus relaciones externas:
\begin{enumerate}
    \item \textbf{Muchos proveedores:} Estrategia basada en la competencia agresiva entre proveedores, común para productos estándar (\textit{commodity}). Se selecciona la oferta más barata para cada petición.
    \item \textbf{Pocos proveedores:} Busca establecer relaciones a largo plazo con un número reducido de proveedores, lo que les permite alcanzar economías de escala. El coste de cambiar de proveedor es alto.
    \item \textbf{Integración vertical:} Consiste en producir internamente bienes que antes se compraban o adquirir un proveedor (integración hacia atrás) o un distribuidor (integración hacia adelante).
    \item \textbf{Joint Ventures (empresas conjuntas):} Colaboración formal en la que varias empresas establecen una propiedad común para desarrollar nuevos productos o mercados.
    \item \textbf{Redes Keiretsu:} Coalición de empresas en la que los proveedores se integran profundamente, combinando colaboración, compra a pocos proveedores e integración vertical. Se basa en relaciones a largo plazo y apoyo mutuo.
    \item \textbf{Empresas virtuales:} Organizaciones que dependen de una red de proveedores externos para proporcionar servicios bajo demanda. La cadena de suministro es, en esencia, la propia empresa.
\end{enumerate}

\section{Riesgos en la Cadena de Suministro}
La gestión de la cadena de suministro conlleva riesgos, definidos como la posibilidad y el efecto de un desajuste entre la oferta y la demanda. La creciente dependencia de la cadena (comprar más, fabricar menos), la especialización con pocos proveedores y los bajos inventarios incrementan el riesgo. Estos riesgos pueden ser de origen diverso: naturales, políticos, financieros o sistémicos, como los derivados de pandemias o desastres naturales con impacto global.

La gestión de estos riesgos se ha convertido en un reto estratégico, ya que trabajar con muchos proveedores aumenta la complejidad logística, mientras que hacerlo con pocos aumenta la dependencia. Para mitigar estos riesgos, las empresas pueden aplicar diversas tácticas:

\begin{itemize}
    \item \textbf{Riesgos de proveedores (fallos en envío o calidad):} Mitigados con el uso de múltiples proveedores, contratos con penalizaciones, una cuidadosa selección y supervisión, y la disponibilidad de subcontratas de reserva.
    \item \textbf{Riesgos logísticos (retrasos o daños):} Se gestionan mediante la diversificación de modos de transporte y almacenes, embalajes seguros y contratos eficaces.
    \item \textbf{Pérdida de información:} Se previene con bases de datos redundantes, sistemas de TI seguros y formación de los socios de la cadena.
    \item \textbf{Riesgos políticos y económicos:} Se mitigan con seguros, diversificación internacional, franquicias y coberturas para el riesgo del tipo de cambio.
    \item \textbf{Catástrofes, robos o terrorismo:} Se afrontan con seguros, fuentes de suministro alternativas, diversificación y medidas de seguridad (por ejemplo, GPS).
\end{itemize}
El proceso para mitigar los riesgos se desarrolla en cuatro etapas: 1) interpretación y visualización de riesgos, 2) medición y priorización, 3) toma de acciones, y 4) seguimiento y revisión continua.

\section{Ética y Sostenibilidad}
La gestión de la cadena de suministro debe regirse por principios éticos y de sostenibilidad. La ética personal y la ética dentro de la cadena exigen que las empresas establezcan normas para sus proveedores, similares a las que aplican para sí mismas, ya que la sociedad demanda un comportamiento ético a lo largo de toda la cadena.

\subsection{Sostenibilidad en la GCS}
La sostenibilidad en la GCS implica gestionar los flujos de productos, información y finanzas con un enfoque en las preocupaciones sociales y medioambientales. Los tres pilares de la sostenibilidad son:
\begin{itemize}
    \item \textbf{Pilar medioambiental:} Centrado en la criticidad de los recursos y la promoción de la economía circular.
    \item \textbf{Inclusión social y equidad distributiva:} Busca garantizar que quienes fabrican los productos compartan equitativamente los beneficios y que no se produzcan situaciones de exclusión o "esclavitud moderna". Leyes como la Ley de Transparencia en las Cadenas de Suministro de California buscan frenar estas prácticas.
\end{itemize}
Existe una creciente conciencia en el sector empresarial sobre la importancia de la sostenibilidad. Muchas empresas invierten en cadenas más sostenibles, no solo porque es lo correcto, sino porque también genera retornos financieros. Un aspecto clave es la distinción entre:
\begin{itemize}
    \item \textbf{Logística directa:} Flujo de productos hacia el cliente.
    \item \textbf{Logística inversa:} Flujo de productos desde el cliente de vuelta a la empresa, relacionado con devoluciones, reciclaje o reacondicionamiento.
\end{itemize}
Una \textbf{cadena de suministro de bucle cerrado} se refiere al diseño proactivo de una cadena que optimiza tanto los flujos hacia adelante como los flujos inversos, integrando la sostenibilidad desde el inicio.
