\chapter{Introducción a la Dirección de Operaciones}

\section{La Dirección de Operaciones en la Organización}

La \textbf{Dirección de Operaciones} se define como el conjunto de actividades que crean valor mediante la transformación de recursos productivos (entradas o \textit{inputs}) en productos (salidas u \textit{outputs}), ya sean bienes o servicios. Su objetivo fundamental es maximizar la productividad en este proceso de creación. Esta función es esencial en cualquier tipo de organización, desde empresas industriales hasta proveedores de servicios, e interactúa de manera integral con otras áreas funcionales clave como marketing y finanzas/contabilidad.

\subsection{El Proceso de Transformación}
El proceso de transformación constituye el núcleo de la dirección de operaciones, donde se convierten diversos \textit{inputs} en \textit{outputs} con valor añadido.
\begin{itemize}
    \item \textbf{Inputs (Entradas):} Incluyen recursos como materias primas, mano de obra, equipamiento, energía e información.
    \item \textbf{Proceso:} Abarca actividades específicas como el procesamiento de transacciones, operaciones de vuelo, intervenciones quirúrgicas o controles de calidad, que añaden valor a las entradas.
    \item \textbf{Outputs (Salidas):} Son los bienes y servicios finales, tales como préstamos, transporte de pasajeros, tratamientos médicos o productos tangibles como automóviles y electrodomésticos.
\end{itemize}
El estudio de esta disciplina es fundamental para comprender cómo se organizan las empresas, el funcionamiento de los directores de operaciones (COO), la producción de bienes y servicios, y la gestión de una de las áreas que genera mayores costes en una organización.

\subsection{Diferencias entre Bienes y Servicios}
Las empresas industriales fabrican productos tangibles, mientras que las empresas de servicios satisfacen necesidades a través de prestaciones intangibles. Aunque ambos comparten la necesidad de estándares de calidad y una programación orientada al cliente, existen diferencias clave.
\begin{itemize}
    \item \textbf{Intangibilidad:} Los servicios no son tangibles, a diferencia de los bienes físicos.
    \item \textbf{Producción y Consumo Simultáneo:} La producción y el consumo de un servicio ocurren al mismo tiempo, lo que impide su almacenamiento.
    \item \textbf{Interacción con el Cliente:} Los servicios implican una alta interacción, llegando a la coproducción, mientras que en la fabricación de bienes el contacto es menor.
    \item \textbf{Unicidad y Estandarización:} Los servicios son a menudo únicos y difíciles de estandarizar, mientras que los bienes suelen ser más homogéneos.
    \item \textbf{Medición de Calidad:} La calidad en los servicios es subjetiva y difícil de medir, en contraste con los bienes, donde es más objetiva.
\end{itemize}

\subsection{La Servitización: Un Modelo Híbrido}
La distinción entre productos y servicios ''puros'' es cada vez más difusa, dando lugar a modelos de negocio híbridos. La \textbf{servitización} es el proceso mediante el cual las empresas, especialmente las manufactureras, integran servicios a sus productos para crear valor, mejorar la satisfacción del cliente y generar ventajas competitivas. Este modelo permite a las empresas industriales evolucionar hacia la prestación de servicios basados en sus productos manufacturados, generando ingresos recurrentes y fortaleciendo la relación con el cliente.

Se pueden identificar tres niveles de servitización:
\begin{enumerate}
    \item \textbf{Servicios básicos:} Centrados en las competencias de producción.
    \item \textbf{Servicios de apoyo o intermedios:} Incluyen mantenimiento y soporte al cliente.
    \item \textbf{Servicios avanzados:} Se enfocan en el rendimiento o resultado del funcionamiento del producto, no en el bien físico en sí.
\end{enumerate}

\section{Historia de la Dirección de Operaciones}
La Dirección de Operaciones ha evolucionado a lo largo de distintas eras, marcadas por hitos tecnológicos y conceptuales.
\begin{itemize}
    \item \textbf{Era Preindustrial:} Caracterizada por la producción artesanal en gremios y economías domésticas.
    \item \textbf{Revolución Industrial (Adam Smith, 1776):} Con la aparición de la máquina de vapor, la mecanización de la industria (especialmente la textil) y la introducción de conceptos como la división del trabajo y la especialización, se sentaron las bases para la producción moderna.
    \item \textbf{Segunda Revolución Industrial (principios del siglo XX):}
    \begin{itemize}
        \item \textbf{Frederik W. Taylor (1881):} Introdujo los estudios de tiempos y movimientos, promoviendo la formación, la estandarización de métodos de trabajo y los sistemas de incentivos para aumentar la productividad.
        \item \textbf{Henry Ford (1908):} Implementó la línea de ensamblaje móvil, dando origen a la fabricación en masa. Durante esta era surgieron también técnicas como los gráficos de Gantt, la teoría de colas de Erlang, el muestreo estadístico para control de calidad y los sistemas de gestión de inventarios.
    \end{itemize}
    \item \textbf{Era Posindustrial (Daniel Bell, 1973):} Se destaca la creciente importancia del sector servicios y su contribución al PIB. El desarrollo de la informática e Internet revolucionó la comunicación y las relaciones comerciales, otorgando una gran relevancia a la gestión de la cadena de suministro.
\end{itemize}

\section{Tendencias en la Dirección de Operaciones}
La Dirección de Operaciones contemporánea se enfrenta a un entorno dinámico, marcado por las siguientes tendencias:
\begin{itemize}
    \item \textbf{Big Data e Industria 4.0:} La digitalización y el análisis de grandes volúmenes de datos (\textit{Big Data}) están transformando la economía global, permitiendo una interconexión en tiempo real entre individuos, empresas y sociedades. El Internet de las Cosas (\textit{IoT}) permite recopilar, analizar y distribuir datos para optimizar procesos en la cadena de suministro. La analítica de \textit{Big Data} (BDA) se utiliza en la toma de decisiones de marketing, logística, producción y aprovisionamiento, posibilitando simulaciones con carácter predictivo y preventivo.
    \item \textbf{Producción bajo demanda y Enfoque de respuesta rápida:} Las nuevas tecnologías permiten una mayor flexibilidad y reducción de costes, facilitando la adaptación a las necesidades cambiantes de los clientes.
    \item \textbf{Colaboración y gestión centrada en la cadena de suministro:} La competencia ya no es entre empresas, sino entre cadenas de suministro, lo que impulsa una mayor integración y colaboración.
    \item \textbf{Flexibilidad de las operaciones y Empoderamiento de los usuarios:} La capacidad de adaptación a las variaciones del mercado y la participación activa de los clientes en el diseño y producción son clave.
    \item \textbf{Economía circular y sostenibilidad medioambiental:} Se busca un replanteamiento del funcionamiento de la cadena de suministro para producir y consumir de manera más sostenible, reduciendo residuos y el agotamiento de recursos.
\end{itemize}

\section{La Estrategia de Operaciones}
La estrategia de operaciones se enmarca dentro de la estrategia global de la empresa y se articula en tres niveles jerárquicos: corporativo, competitivo y funcional. Su propósito es desplegar los recursos de un departamento específico para obtener una ventaja competitiva sostenible y alcanzar los objetivos definidos en la misión de la organización.

\subsection{Ventaja Competitiva a través de las Operaciones}
Para lograr una ventaja competitiva, la dirección de operaciones se enfoca en tres áreas clave:
\begin{itemize}
    \item \textbf{Diferenciación:} Consiste en ofrecer un valor añadido que el cliente perciba como único. Puede manifestarse a través de una amplia gama de productos, funcionalidades específicas o, en el sector servicios, mediante la creación de una ''experiencia'' que involucre al cliente.
    \item \textbf{Liderazgo en Costes:} Busca lograr el máximo valor para el consumidor mediante la reducción de costes, sin sacrificar la calidad o el valor percibido.
    \item \textbf{Capacidad de Respuesta:} Se refiere al desarrollo y entrega del producto en el tiempo previsto, con una programación fiable y una ejecución flexible. Sus componentes son:
    \begin{itemize}
        \item \textbf{Flexibilidad:} Adaptación a los cambios del mercado en diseño y volúmenes.
        \item \textbf{Fiabilidad:} Cumplimiento garantizado de los plazos de entrega.
        \item \textbf{Rapidez:} Acortamiento de los tiempos de diseño, producción y suministro.
    \end{itemize}
\end{itemize}

\subsection{Objetivos de Operaciones}
La estrategia de operaciones se traduce en un conjunto de objetivos medibles que guían las decisiones. Estos varían según la estrategia de la empresa.
\begin{itemize}
    \item \textbf{Objetivos Clásicos:}
    \begin{itemize}
        \item \textbf{Coste o Eficiencia:} Optimización de los costes de mano de obra, materiales y otros recursos directos e indirectos.
        \item \textbf{Calidad:} Tanto interna (ausencia de defectos) como externa (satisfacción de los requisitos del cliente).
        \item \textbf{Plazo de Entrega:} Incluye la rapidez y la fiabilidad en la entrega.
        \item \textbf{Flexibilidad:} Capacidad de modificar o introducir nuevos productos (en producto) y de ajustar los volúmenes de producción (en volumen).
    \end{itemize}
    \item \textbf{Nuevos Objetivos:}
    \begin{itemize}
        \item \textbf{Servicio:} Prestaciones preventa y posventa.
        \item \textbf{Innovación:} Tanto radical (disruptiva) como incremental (mejora continua).
        \item \textbf{Ecoeficiencia y Sostenibilidad:} Cumplimiento de la legislación ambiental, prevención de incidentes, fabricación de productos ecológicos y liderazgo medioambiental para potenciar la imagen de la empresa.
    \end{itemize}
\end{itemize}

\subsection{Las 10 Decisiones de Operaciones}
La estrategia de operaciones se implementa a través de diez decisiones clave, que se clasifican en estratégicas (a largo plazo) y tácticas/operativas (a corto plazo).
\begin{itemize}
    \item \textbf{Decisiones de carácter estratégico:}
    \begin{itemize}
        \item Diseño del producto y del servicio.
        \item Gestión de la calidad.
        \item Diseño de procesos y planificación de capacidad.
        \item Localización.
        \item Distribución en planta o \textit{layout}.
    \end{itemize}
    \item \textbf{Decisiones tácticas y operativas:}
    \begin{itemize}
        \item Recursos humanos y diseño del trabajo.
        \item Gestión de la cadena de suministro.
        \item Gestión de inventarios.
        \item Planificación y programación.
        \item Mantenimiento.
    \end{itemize}
\end{itemize}

\section{Técnicas de Toma de Decisiones}
Los directores de operaciones emplean diversas técnicas cuantitativas y cualitativas para fundamentar sus decisiones. A continuación, se describen algunas de las más relevantes.
\begin{itemize}
    \item \textbf{Árboles de Decisión:} Es un esquema gráfico que representa secuencias de decisiones y los posibles eventos o consecuencias que pueden afectar a cada una de ellas, facilitando el análisis de problemas complejos con incertidumbre.
    \item \textbf{Análisis Coste-Volumen-Beneficio:} Relaciona los costes, ingresos y beneficios con el volumen de producción, permitiendo evaluar la rentabilidad de diferentes escenarios operativos.
    \item \textbf{Factores Ponderados:} Este método permite evaluar alternativas considerando múltiples factores, tanto cualitativos como cuantitativos. A cada factor se le asigna una ponderación según la importancia que le otorgue el decisor.
    \item \textbf{Método del Centro de Gravedad:} Técnica específica de la Dirección de Operaciones, utilizada en decisiones de localización. Su objetivo es determinar la ubicación óptima de una instalación para minimizar los costes totales de transporte y distribución.
    \item \textbf{Equilibrado de Cadenas (o de Líneas):} Aplicado en el diseño de la distribución en planta (\textit{layout}), busca subdividir el flujo de trabajo de manera que el personal y los equipos se utilicen de la forma más eficiente y ajustada posible a lo largo de todo el proceso productivo.
\end{itemize}
