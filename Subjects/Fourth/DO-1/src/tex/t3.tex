\chapter{Diseño de Bienes y Servicios}

% \textbf{Fuentes principales: CAPÍTULO 3 DE ARIAS Y MINGUELA, 2024; CAPÍTULO 5 DE HEIZER Y RENDER, 2015}

\section{Definición de desarrollo de nuevos productos}

El \textbf{concepto de producto} se define como \textit{“algo que se ofrece a un mercado con la finalidad de que se le preste atención, sea adquirido, usado o consumido, con el objeto de satisfacer un deseo o necesidad”}. Un producto engloba un conjunto de atributos, tanto tangibles como intangibles (envase, precio, marca, servicios, etc.), que los compradores perciben como capaces de satisfacer sus necesidades.

Las actividades de producción de bienes y servicios se denominan \textbf{operaciones}, y su gestión, \textbf{Dirección de Operaciones}. Esta disciplina busca crear valor transformando recursos (inputs) en productos (outputs). Las diferencias entre bienes (manufacturas) y servicios son clave para entender su diseño y gestión:
\begin{itemize}
    \item \textbf{Bienes}: Son productos físicos y duraderos, que pueden ser inventariados. Generalmente, implican poco contacto con el cliente, tiempos de respuesta largos y su calidad es más fácil de medir objetivamente.
    \item \textbf{Servicios}: Son intangibles y perecederos, no se pueden inventariar. Requieren un alto contacto con el cliente, tiempos de respuesta cortos y su calidad es más subjetiva y, por tanto, más difícil de medir.
\end{itemize}

\begin{table}[H]
\centering
\caption{Las diferencias entre bienes y servicios influyen en cómo se aplican las 10 decisiones de operaciones}
\begin{tabular}{|p{3cm}|p{4cm}|p{4cm}|}
\hline
\textbf{Decisiones de operaciones} & \textbf{Bienes} & \textbf{Servicios} \\ \hline
Diseño de bienes y servicios & Normalmente el producto es tangible. & El producto no es tangible. Una nueva gama de atributos del producto: una sonrisa. \\ \hline
Gestión de la calidad & Muchas normas de calidad objetivas. & Muchas normas de calidad subjetivas: un color bonito. \\ \hline
Diseño del proceso y de la capacidad & El cliente no está implicado en la mayor parte del proceso. & El cliente puede estar implicado directamente en el proceso: un corte de pelo. La capacidad debe adecuarse a la demanda para evitar pérdida de ventas: los clientes normalmente evitan esperar. \\ \hline
Selección de localización & Puede ser necesario estar cerca de las materias primas o de la mano de obra. & Puede ser necesario estar cerca del cliente: alquiler de coches. \\ \hline
Diseño del layout & El layout puede mejorar la eficiencia. & Puede mejorar el producto y la producción. Ej. layout de un restaurante elegante. \\ \hline
Recursos humanos y diseño del puesto de trabajo & Mano de obra centrada en habilidades técnicas. Los estándares de trabajo pueden ser constantes. Posible sistema salarial basado en la producción. & La mano de obra directa necesita normalmente poder relacionarse con el cliente: cajero de un banco. Los estándares de trabajo varían según las exigencias del cliente: procesos legales. \\ \hline
Gestión de la cadena de suministros & Las relaciones en la cadena de suministros son vitales para el producto final. & Las relaciones de la cadena de suministros son importantes pero no son vitales. \\ \hline
Inventario & Las materias primas, los productos semiacabados y los acabados pueden almacenarse. & La mayor parte de los servicios no puede almacenarse, por lo que hay que encontrar otras formas de acomodarse a los cambios de la demanda. \\ \hline
Programación & La posibilidad de almacenar puede permitir nivelar la tasa de producción. & Ocupada en satisfacer los plazos inmediatos del cliente utilizando los recursos humanos. \\ \hline
Mantenimiento & El mantenimiento es habitualmente preventivo, y se da en el lugar de producción. & El mantenimiento es normalmente una “reparación”, que se realiza en el lugar donde está el cliente. \\ \hline
\end{tabular}
\end{table}

El \textbf{Desarrollo de Nuevos Productos (DNP)} es la secuencia de decisiones que conduce a la creación de un nuevo bien o servicio. Este proceso se puede clasificar según:
\begin{itemize}
    \item \textbf{El tipo de innovación}: Puede ser \textbf{radical}, si crea algo completamente nuevo, o \textbf{incremental}, si consiste en mejoras sobre productos ya existentes.
    \item \textbf{La complejidad}: Depende de la cantidad de variables y de la sofisticación del conocimiento requerido.
\end{itemize}

El DNP es una \textbf{decisión transversal} que requiere la coordinación de múltiples áreas funcionales como Marketing, I+D, Producción, Compras, Finanzas y Dirección General. La colaboración entre departamentos se puede gestionar mediante dos enfoques:
\begin{itemize}
    \item \textbf{Enfoque secuencial}: Cada departamento completa su fase antes de pasarla al siguiente. Es un proceso más lento y menos flexible.
    \item \textbf{Enfoque concurrente o simultáneo}: Los departamentos trabajan de forma paralela y coordinada, lo cual acelera el desarrollo del producto.
\end{itemize}

\begin{table}[H]
\centering
\caption{Comparativa de ventajas e inconvenientes de los enfoques secuencial y concurrente}
\begin{tabular}{|p{4cm}|p{5cm}|p{5cm}|}
\hline
\textbf{Aspecto} & \textbf{Enfoque Secuencial} & \textbf{Enfoque Concurrente} \\ \hline
\textbf{Ventajas} & 
\begin{itemize}
    \item Proceso estructurado y fácil de gestionar.
    \item Menor riesgo de conflictos entre departamentos.
    \item Claridad en la asignación de responsabilidades.
\end{itemize} & 
\begin{itemize}
    \item Reducción del tiempo total de desarrollo.
    \item Mayor flexibilidad y capacidad de respuesta.
    \item Fomenta la colaboración y la innovación.
\end{itemize} \\ \hline
\textbf{Inconvenientes} & 
\begin{itemize}
    \item Proceso más lento debido a la naturaleza secuencial.
    \item Menor flexibilidad ante cambios en el entorno.
    \item Posible falta de integración entre departamentos.
\end{itemize} & 
\begin{itemize}
    \item Mayor complejidad en la gestión del proyecto.
    \item Riesgo de conflictos entre departamentos.
    \item Requiere una comunicación y coordinación más intensiva.
\end{itemize} \\ \hline
\end{tabular}
\end{table}

Ventajas: mayor información, información disponible con antelación y previsión de posibles problemas.
Desventajas: mayor complejidad organizativa, mayores tiempos de toma de decisiones y mayor dificultad en la toma de decisiones. [COMPLETAR]



\section{Ciclo de vida de los productos y servicios}

El \textbf{Ciclo de Vida del Producto (CVP)} describe las distintas etapas por las que pasa un producto desde su lanzamiento al mercado hasta su desaparición. Este modelo pone en relación el tiempo (eje X) y el volumen de ventas (eje Y). Cada etapa presenta desafíos y oportunidades que exigen reajustar las estrategias de operaciones, marketing y finanzas. La duración del CVP varía según la naturaleza del producto, pero no debe confundirse con la vida útil del mismo para el consumidor.

Las cuatro fases del ciclo de vida son:
\begin{enumerate}
    \item \textbf{Introducción}: Las ventas crecen lentamente mientras el producto se da a conocer. Se caracteriza por:
    \begin{itemize}
        \item Fuertes desembolsos en I+D y modificaciones de procesos, generando pérdidas y un flujo de caja (cash-flow) negativo.
        \item Métodos de producción flexibles y poco eficientes, baja gama de producto y poca competencia.
        \item Los esfuerzos de diseño y desarrollo del producto son críticos.
        \item En esta fase se encuentran los ''early adopters'', clientes que buscan innovación y están dispuestos a pagar un precio premium.
        \item Ejemplos: coches voladores, coches autónomos, realidad virtual.
    \end{itemize}

    \item \textbf{Crecimiento}: La demanda aumenta rápidamente. En esta fase:
    \begin{itemize}
        \item El diseño del producto comienza a estabilizarse y se realizan inversiones para aumentar la capacidad productiva.
        \item El beneficio y el flujo de caja se vuelven positivos.
        \item La previsión de la demanda se vuelve crítica para la gestión de la capacidad.
    \end{itemize}

    \item \textbf{Madurez}: El mercado se satura y el volumen de ventas se estabiliza. Es la fase de mayor rentabilidad. Se caracteriza por:
    \begin{itemize}
        \item Estandarización del producto y del proceso, buscando economías de escala para reducir costes.
        \item La competencia es intensa, a menudo basada en costes.
        \item Es el momento ideal para iniciar el desarrollo de nuevos productos que sustituyan a los actuales.
    \end{itemize}

    \item \textbf{Declive}: Las ventas y los beneficios disminuyen paulatinamente. La empresa debe decidir si:
    \begin{itemize}
        \item Abandona el producto, liquidando existencias.
        \item Reubica la inversión en productos con mayor potencial.
        \item Intenta "reinventar" el producto.
        \item Desde operaciones, se debe reducir la capacidad y minimizar costes, eliminando productos sin margen aceptable.
    \end{itemize}
\end{enumerate}

\section{Etapas en el desarrollo de nuevos productos}

El proceso de DNP se puede estructurar en una secuencia de decisiones clave, que abarca desde la concepción de la idea hasta su llegada al mercado. Las etapas fundamentales son:
\begin{enumerate}
    \item \textbf{Identificación de la oportunidad de negocio (Generación y selección de ideas)}: Las ideas para nuevos productos pueden surgir de diversas fuentes:
    \begin{itemize}
        \item \textit{Consumidores (Tirón de la Demanda)}: Comprender las necesidades y deseos del cliente es un punto de partida fundamental.
        \item \textit{I+D (Empuje Tecnológico)}: Los avances tecnológicos hacen posibles nuevos productos.
        \item \textit{Competidores}: A través del benchmarking se pueden obtener ideas para mejorar o diferenciar la oferta.
        \item \textit{Empleados e Innovación Abierta}.
    \end{itemize}
    Una vez generadas, las ideas se someten a un filtro de viabilidad comercial (marketing), técnica (operaciones) y financiera (finanzas).

    \item \textbf{Diseño (Preliminar y Final)}: Esta fase transforma la idea en un concepto tangible. Las decisiones de diseño abarcan la función, costes, calidad, impacto medioambiental y métodos de producción. Para ello se utilizan diferentes \textbf{elementos y herramientas de diseño}:
    \begin{itemize}
        \item \textbf{Diseño Robusto}: Busca que pequeñas variaciones en la producción no afecten negativamente al producto final.
        \item \textbf{Diseño Modular}: Subdivide el producto en módulos intercambiables, lo que facilita la variedad y la reparación.
        \item \textbf{Diseño Asistido por Ordenador (CAD)} y \textbf{Fabricación Asistida por Ordenador (CAM)}: Programas informáticos que agilizan el diseño, la preparación de la documentación de ingeniería y el control de los equipos de producción, reduciendo costes y tiempos.
        \item \textbf{Despliegue de la Función de Calidad (QFD)}: Es una herramienta, cuya representación gráfica es la "casa de la calidad", que permite traducir los deseos del cliente en características técnicas del producto.
        \item \textbf{Ingeniería de Valor y Análisis de Valor}: Se centran en la mejora del diseño y las especificaciones para reducir costes sin sacrificar funcionalidad, antes de la producción (ingeniería) o durante ella (análisis).
    \end{itemize}

    \item \textbf{Construcción y evaluación de prototipos}: Se crean modelos o versiones iniciales del producto (maquetas, plantas piloto) para realizar evaluaciones técnicas y de mercado (lanzamiento en zonas piloto, paneles de consumidores). Esto permite probar el producto antes de comprometer recursos a gran escala.

    \item \textbf{Producción}: En esta fase, se toman las decisiones relativas al diseño del proceso y la planificación de la capacidad.

    \item \textbf{Comercialización}: Es la introducción del producto en el mercado, una función gestionada principalmente por el área de marketing.
\end{enumerate}


\section{Estrategias en el desarrollo de nuevos productos}

La creciente sofisticación tecnológica y la reducción de los ciclos de vida de los productos obligan a las empresas a \textbf{acelerar su proceso de desarrollo}. La competencia basada en el tiempo, que busca rapidez en el diseño, producción y entrega, se ha convertido en una ventaja competitiva clave.

Las estrategias de DNP se pueden clasificar en un continuo que va desde el desarrollo interno hasta el externo:

\begin{itemize}
    \item \textbf{Estrategias de desarrollo interno}:
    \begin{itemize}
        \item \textit{Mejoras de productos existentes}: Cambios incrementales en productos actuales.
        \item \textit{Migraciones de productos existentes}: Se aprovechan las plataformas de productos actuales para crear nuevas versiones, acelerando el desarrollo y reduciendo costes y riesgos.
        \item \textit{Nuevos productos desarrollados internamente}: Es la opción más lenta y arriesgada, pero ofrece un control total.
    \end{itemize}

    \item \textbf{Estrategias de desarrollo externo}: Buscan adquirir tecnología o experiencia fuera de la empresa para acelerar el proceso.
    \begin{itemize}
        \item \textit{Adquisición de tecnología}: Comprar una empresa que ya ha desarrollado la tecnología deseada.
        \item \textit{Empresas conjuntas (Joint Ventures)}: Dos o más empresas establecen una propiedad común para lanzar un nuevo producto. El riesgo y el coste se comparten.
        \item \textit{Alianzas}: Acuerdos de cooperación donde las empresas permanecen independientes pero persiguen objetivos comunes. Son adecuadas cuando las tecnologías son incipientes y los riesgos elevados.
    \end{itemize}
\end{itemize}


\section{Técnicas de resolución de ejercicios para la toma de decisiones sobre diseño de bienes y servicios}

Las decisiones sobre el diseño de productos a menudo se toman en condiciones de \textbf{riesgo} o \textbf{incertidumbre}. Para abordar estas situaciones, se utilizan técnicas cuantitativas que ayudan a estructurar el problema y a evaluar las alternativas.

\begin{itemize}
    \item \textbf{Matriz de Decisión}: Es una herramienta que permite analizar un problema con una decisión única. Se estructura con:
    \begin{itemize}
        \item \textbf{Estrategias o alternativas}: Las diferentes opciones que el decisor puede elegir.
        \item \textbf{Estados de la naturaleza}: Sucesos futuros que no están bajo el control del decisor y para los cuales se pueden conocer (o no) sus probabilidades de ocurrencia.
        \item \textbf{Resultados o desenlaces}: Las consecuencias (beneficios, costes) de cada combinación de estrategia y estado de la naturaleza.
    \end{itemize}
    En condiciones de riesgo, se conoce la probabilidad de cada estado de la naturaleza. El criterio de decisión más común es el \textbf{Valor Monetario Esperado (VME)}, que se calcula para cada alternativa sumando los resultados ponderados por sus probabilidades. Se elige la alternativa con el mayor VME.

    \item \textbf{Árboles de Decisión}: Se utilizan cuando el decisor se enfrenta a una \textbf{secuencia de decisiones} dependientes entre sí. Un árbol de decisión es un esquema gráfico que representa:
    \begin{itemize}
        \item \textbf{Nudos decisionales} (cuadrados): Puntos donde se elige entre varias alternativas (ramas decisionales).
        \item \textbf{Nudos aleatorios} (círculos): Puntos donde ocurren los estados de la naturaleza (ramas aleatorias), cada uno con su probabilidad asociada.
        \item \textbf{Resultados esperados}: Se sitúan al final de cada secuencia de ramas.
    \end{itemize}
    La resolución del árbol se realiza "hacia atrás", desde la derecha hacia la izquierda, calculando el VME en cada nudo aleatorio y eligiendo la rama con el mejor resultado en cada nudo decisional.
\end{itemize}
Ambas técnicas, aunque son herramientas estratégicas y tácticas generales, son de aplicación directa en las decisiones de diseño de productos y servicios.
