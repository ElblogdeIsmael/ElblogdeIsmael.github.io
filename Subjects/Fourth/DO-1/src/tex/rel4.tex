\newpage
\section{Análisis de CVB y Cálculo de Productividad}
\begin{ejercicio}
El \textbf{Ejercicio 9} solicita calcular el \textbf{punto muerto en unidades de tiempo} (Umbral de Rentabilidad en unidades de tiempo), basándose en la información de costes, ingresos y la distribución de ventas esperada para el próximo año.

El análisis Coste-Volumen-Beneficio (CVB) define el punto muerto ($X_0$) como el volumen de ventas en unidades físicas que hace que el beneficio económico sea igual a cero (ingresos totales = costes totales).

Para resolver este ejercicio, se deben seguir los siguientes pasos:

\underline{1. Datos del Ejercicio}

\begin{table}[h!]
\centering
\begin{tabular}{@{}lll@{}}
\toprule
\textbf{Concepto} & \textbf{Símbolo} & \textbf{Valor} \\ \midrule
Cifra de Negocio prevista (Ingresos Totales) & $IT$ & 540.000 € \\
Precio unitario de venta & $p$ & 60 €/unidad \\
Costes Fijos de explotación & $CF$ & 150.000 € \\
Costes Variables unitarios & $cv$ & 35 €/unidad \\ \bottomrule
\end{tabular}
\end{table}

\begin{table}[h!]
\centering
\begin{tabular}{@{}ll@{}}
\toprule
\textbf{Distribución de Ventas Anuales} & \textbf{Porcentaje} \\ \midrule
Primer trimestre (T1) & 30\% \\
Segundo trimestre (T2) & 25\% \\
Tercer trimestre (T3) & 40\% \\
Cuarto trimestre (T4) & 5\% \\ \bottomrule
\end{tabular}
\end{table}

\underline{2. Cálculo del Margen Bruto Unitario ($m$)}

El margen bruto unitario ($m$) es el excedente que deja cada unidad vendida para cubrir los costes fijos de explotación.

\[
m = p - cv
\]
\[
m = 60 \, \text{€/unidad} - 35 \, \text{€/unidad} = \mathbf{25 \, \text{€/unidad}}
\]

\underline{3. Cálculo del Punto Muerto en Unidades Físicas ($X_0$)}

$X_0$ es el volumen de ventas que iguala el margen bruto total ($m \cdot X_0$) a los costes fijos ($CF$).

\[
X_0 = \frac{CF}{p - cv}
\]
\[
X_0 = \frac{150.000 \, \text{€}}{25 \, \text{€/unidad}} = \mathbf{6.000 \, \text{unidades}}
\]

La empresa debe vender 6.000 unidades para alcanzar el umbral de rentabilidad.

\underline{4. Cálculo del Volumen Total de Ventas Previstas ($X$)}

Se calcula el número total de unidades ($X$) que la empresa espera vender durante el próximo año a partir de la cifra de negocio esperada y el precio unitario:

\[
X = \frac{IT}{p}
\]
\[
X = \frac{540.000 \, \text{€}}{60 \, \text{€/unidad}} = \mathbf{9.000 \, \text{unidades}}
\]

\underline{5. Cálculo del Punto Muerto en Unidades de Tiempo ($PM_t$)}

El punto muerto en unidades de tiempo indica el momento dentro del periodo de referencia (un año, en este caso) en el que se cubren todos los costes fijos y variables, y a partir del cual el beneficio empieza a ser positivo.

Primero, se determina qué porcentaje del total de ventas anuales representan las 6.000 unidades del punto muerto:

\[
\text{Porcentaje de Ventas Necesarias} = \frac{X_0}{X} = \frac{6.000}{9.000} \approx \mathbf{0,6667} \, \text{o} \, \mathbf{66,67\%}
\]

Ahora, se utiliza la distribución trimestral de ventas para determinar cuándo se alcanza este 66,67\% de las ventas anuales:

\begin{table}[h!]
\centering
\begin{tabular}{@{}lll@{}}
\toprule
\textbf{Trimestre} & \textbf{Porcentaje de Ventas} & \textbf{Porcentaje Acumulado} \\ \midrule
Primer Trimestre (T1) & 30\% & 30\% \\
Segundo Trimestre (T2) & 25\% & 55\% \\
Tercer Trimestre (T3) & 40\% & 95\% \\ \bottomrule
\end{tabular}
\end{table}

Al finalizar el Segundo Trimestre (T2), la empresa ha cubierto el 55\% de sus ventas anuales. Dado que necesita cubrir el 66,67\%, el punto muerto se alcanzará durante el \textbf{Tercer Trimestre (T3)}.

\textbf{Ventas necesarias en el T3:}
\[
66,67\% - 55\% = 11,67\% \, \text{de las ventas anuales}
\]

El Tercer Trimestre (T3) representa el 40\% de las ventas totales anuales. Suponiendo que las ventas dentro del trimestre son uniformes, se calcula la fracción de tiempo del T3 necesaria para alcanzar el punto muerto:

\[
\text{Fracción del T3} = \frac{\text{Porcentaje de ventas necesario en T3}}{\text{Porcentaje total de ventas en T3}} = \frac{11,67\%}{40\%} \approx 0,29175
\]

El año consta de 4 trimestres. El punto muerto se alcanza tras \textbf{dos trimestres completos} (T1 y T2) más una fracción del Tercer Trimestre:

\[
\text{PM}_t = 2 \, \text{trimestres} + 0,29175 \times 1 \, \text{trimestre} \approx \mathbf{2,29175 \, \text{trimestres}}
\]

\textbf{Expresión en meses:} Un trimestre representa 3 meses.
\[
\text{Tiempo en T3} = 0,29175 \times 3 \, \text{meses} \approx 0,875 \, \text{meses}
\]

\[
\text{PM}_t = 3 \, \text{meses (T1)} + 3 \, \text{meses (T2)} + 0,875 \, \text{meses (T3)} = \mathbf{6,875 \, \text{meses}}
\]

La empresa alcanzará el punto muerto al final del \textbf{sexto mes} y durante la segunda mitad del \textbf{séptimo mes} (dentro del Tercer Trimestre).

\end{ejercicio}

\begin{ejercicio}
Este ejercicio corresponde al Tema 4, \textbf{Análisis Coste-Volumen-Beneficio (CVB)}, que se centra en la perspectiva económica, asumiendo que lo que se produce se vende (supuesto en el análisis CVB). El objetivo es calcular el \textbf{Beneficio Económico (BE)}.

\underline{1. Datos del Ejercicio 10}

\begin{table}[h!]
\centering
\begin{tabular}{@{}lll@{}}
\toprule
\textbf{Concepto} & \textbf{Símbolo} & \textbf{Valor} \\ \midrule
Unidades vendidas por encima del Punto Muerto & $X - X_0$ & 3.200 unidades \\
Precio unitario de venta & $p$ & 80 €/unidad \\
Coste variable unitario & $cv$ & 35 €/unidad \\ \bottomrule
\end{tabular}
\end{table}

El \textbf{Beneficio Económico (BE)} (o beneficio de explotación) es la renta generada por los activos de la empresa, al margen de su estructura financiera.

La fórmula general para el Beneficio Económico es:
\[
BE = \text{Ingresos totales} - \text{Costes totales de la producción}
\]
\[
BE = p \cdot X - (CF + cv \cdot X)
\]
\[
BE = X(p - cv) - CF
\]

Donde $X$ es el volumen de unidades producidas y vendidas, y $CF$ son los costes fijos de explotación.

\underline{2. Relación con el Punto Muerto}

El \textbf{Punto Muerto o Umbral de Rentabilidad} ($X_0$) es el volumen de ventas en unidades físicas donde el beneficio económico es igual a cero. En este punto, el Margen Bruto Total ($M$) es igual a los Costes Fijos ($CF$): $M = m \cdot X_0 = CF$.

El \textbf{Margen Bruto Unitario ($m$)} es la contribución de cada unidad vendida a la cobertura de los costes fijos:
\[
m = p - cv
\]

\underline{3. Cálculo del Margen Bruto Unitario ($m$)}

Primero, calculamos el margen bruto unitario ($m$) de la empresa:
\[
m = 80 \, \text{€/unidad} - 35 \, \text{€/unidad} = \mathbf{45 \, \text{€/unidad}}
\]

Este valor, $m=45$ €/unidad, indica el excedente que deja cada unidad vendida para cubrir los costes fijos de explotación.

\underline{4. Cálculo del Beneficio Económico (BE)}

Sabemos que la empresa vendió 3.200 unidades \textbf{por encima} de su punto muerto ($X_0$). Esto significa que:
\[
X = X_0 + 3.200
\]

Sustituyendo $X$ en la fórmula del Beneficio Económico:
\[
BE = X(p - cv) - CF
\]
\[
BE = (X_0 + 3.200) \cdot m - CF
\]
\[
BE = X_0 \cdot m + 3.200 \cdot m - CF
\]

Dado que en el punto muerto se cumple que $X_0 \cdot m = CF$, podemos simplificar la expresión:
\[
BE = CF + 3.200 \cdot m - CF
\]
\[
BE = 3.200 \cdot m
\]

El beneficio económico es igual al margen que generan las unidades vendidas por encima del umbral de rentabilidad.

Sustituyendo el valor de $m$:
\[
BE = 3.200 \, \text{unidades} \times 45 \, \text{€/unidad} = \mathbf{144.000 \, \text{euros}}
\]

El \textbf{Beneficio Económico} de la empresa el pasado año fue de \textbf{144.000 euros}.
\end{ejercicio}

\begin{ejercicio}
A pesar de no hacer mejoras en las instalaciones, la empresa A ha experimentado una expansión en este último año por incremento de su plantilla. Así, ha pasado de 12 empleados del año pasado a los 18 del actual, si bien dos de estos últimos solo a media jornada (jornada laboral 8 horas/día). La producción ha pasado de las 200 unidades/día a 260 unidades/día, experimentando el consumo de energía un incremento al pasar de los 600 kWh al día en el año pasado a los 730 kWh en el último año. Se desea conocer la evolución de la productividad para los dos factores productivos señalados mediante el sistema de productividad de un solo factor productivo (productividad monofactorial).
\end{ejercicio}

\begin{solucion} Para la resolución del ejercicio, se siguen los siguientes pasos:

\begin{enumerate}
\item \textbf{Determinación de los Inputs de Mano de Obra (Horas/día)} 
Dado que la productividad del trabajo se mide comúnmente en horas trabajadas, se debe calcular la cantidad total de horas de mano de obra empleadas en cada período.
\begin{align*} 
\text{Horas Mano de Obra } (t_0) &= 12 \text{ empleados} \times 8 \text{ horas/día} \\ 
\mathbf{L_{t_0}} &= \mathbf{96 \text{ horas/día}} \\ 
\text{Horas Mano de Obra } (t_1) &= (16 \text{ empleados} \times 8 \text{ horas/día}) + (2 \text{ empleados} \times 4 \text{ horas/día}) \\ 
\mathbf{L_{t_1}} &= 128 + 8 = \mathbf{136 \text{ horas/día}} 
\end{align*}

\item \textbf{Cálculo de la Productividad Monofactorial}
\begin{enumerate}
\item \textbf{Productividad de la Mano de Obra}
\begin{align*} 
\text{Productividad}_{L, t_0} &= \frac{200 \text{ unidades/día}}{96 \text{ horas/día}} \\ 
\mathbf{P_{L, t_0}} &\approx \mathbf{2.0833 \text{ unidades por hora}} \\ 
\text{Productividad}_{L, t_1} &= \frac{260 \text{ unidades/día}}{136 \text{ horas/día}} \\ 
\mathbf{P_{L, t_1}} &\approx \mathbf{1.9118 \text{ unidades por hora}} 
\end{align*}

\item \textbf{Productividad de la Energía}
\begin{align*} 
\text{Productividad}_{E, t_0} &= \frac{200 \text{ unidades/día}}{600 \text{ kWh/día}} \\ 
\mathbf{P_{E, t_0}} &\approx \mathbf{0.3333 \text{ unidades por kWh}} \\ 
\text{Productividad}_{E, t_1} &= \frac{260 \text{ unidades/día}}{730 \text{ kWh/día}} \\ 
\mathbf{P_{E, t_1}} &\approx \mathbf{0.3562 \text{ unidades por kWh}} 
\end{align*}
\end{enumerate}

\item \textbf{Evolución de la Productividad (Cambio Porcentual)} 
La evolución (o cambio porcentual) se calcula como: 
\[
\frac{\text{Productividad Actual} - \text{Productividad Anterior}}{\text{Productividad Anterior}}.
\]

\begin{enumerate}
\item \textbf{Evolución de la Productividad de la Mano de Obra}
\begin{align*} 
\text{Evolución}_{L} &= \frac{P_{L, t_1} - P_{L, t_0}}{P_{L, t_0}} \\ 
\text{Evolución}_{L} &= \frac{1.9118 - 2.0833}{2.0833} \approx -0.0823 
\end{align*} 
La productividad de la mano de obra ha \textbf{disminuido aproximadamente un 8.23\%}.

\item \textbf{Evolución de la Productividad de la Energía}
\begin{align*} 
\text{Evolución}_{E} &= \frac{P_{E, t_1} - P_{E, t_0}}{P_{E, t_0}} \\ 
\text{Evolución}_{E} &= \frac{0.3562 - 0.3333}{0.3333} \approx 0.0687 
\end{align*} 
La productividad de la energía ha \textbf{aumentado aproximadamente un 6.87\%}.
\end{enumerate}

\item \textbf{Resumen de la Evolución}
La evolución de la productividad de los factores es la siguiente:
\begin{itemize} 
\item \textbf{Mano de Obra (Trabajo):} Disminución del 8.23\%.
\item \textbf{Energía:} Aumento del 6.87\%.
\end{itemize}
\end{enumerate}
\end{solucion} 

\begin{ejercicio}
Una empresa trabaja con unos costes fijos de explotación de 4.000 euros, unos costes fijos financieros de 2.000 euros y un coste variable unitario de 25 euros. Se desea saber:

\begin{itemize}
    \item[a)] El volumen de ventas para que el beneficio económico sea nulo, si se aplica a sus productos un precio unitario de venta de 40 euros.
    \item[b)] La cifra de negocio (importe de las ventas en euros) que hace que el beneficio económico sea cero (dado el precio de venta indicado en el apartado anterior).
\end{itemize}

El problema se resuelve aplicando la técnica de \textbf{Análisis Coste-Volumen-Beneficio (CVB)}, cuyo objetivo es relacionar costes, ingresos y beneficios con el volumen de producción.

En este análisis, el \textbf{beneficio económico} (BE) se define como la renta generada por los activos de la empresa, al margen de su estructura financiera, durante un período de tiempo de referencia. Por lo tanto, para calcular el punto muerto o umbral de rentabilidad ($X_0$), solo se consideran los costes fijos de explotación, excluyendo los costes fijos financieros.

\textbf{Datos de partida:}
\begin{itemize}
    \item Coste Fijo de Explotación ($F$) = 4.000 euros.
    \item Precio Unitario de Venta ($p$) = 40 euros.
    \item Coste Variable Unitario ($v$) = 25 euros.
    \item Margen Bruto Unitario ($m$) = $p - v$.
\end{itemize}

\[
m = 40 \, \text{euros/ud} - 25 \, \text{euros/ud} = 15 \, \text{euros/ud}
\]

\textbf{a) El volumen de ventas para que el beneficio económico sea nulo}

El volumen de ventas para que el beneficio económico sea nulo es el \textbf{punto muerto} o \textbf{umbral de rentabilidad en unidades físicas} ($X_0$). El beneficio económico ($BE$) es nulo cuando los ingresos totales son iguales a los costes totales ($BE = X(p - v) - F = 0$).

La fórmula para calcular el umbral de rentabilidad en unidades físicas ($X_0$) es:
\[
X_0 = \frac{F}{p - v} \quad \text{o} \quad X_0 = \frac{F}{m}
\]

Sustituyendo los valores:
\[
X_0 = \frac{4.000 \, \text{euros}}{40 \, \text{euros/ud} - 25 \, \text{euros/ud}} = \frac{4.000}{15} \, \text{unidades}
\]
\[
X_0 = 266,67 \, \text{unidades}
\]

El volumen de ventas necesario para que el beneficio económico sea nulo es de \textbf{266,67 unidades}.

\textbf{b) La cifra de negocio (importe de las ventas en euros) que hace que el beneficio económico sea cero}

La cifra de negocio que hace que el beneficio económico sea cero es el \textbf{punto muerto en unidades monetarias} ($PE\$$, o $X_0$ en u.m.).

Se calcula multiplicando el volumen de ventas en el punto muerto ($X_0$) por el precio unitario de venta ($p$):
\[
PE\$ = X_0 \cdot p
\]

Sustituyendo los valores (utilizando el valor exacto $4000/15$ para $X_0$):
\[
PE\$ = \left( \frac{4.000}{15} \, \text{unidades} \right) \cdot 40 \, \text{euros/ud}
\]
\[
PE\$ = \frac{160.000}{15} \, \text{euros}
\]
\[
PE\$ = 10.666,67 \, \text{euros}
\]

La cifra de negocio que hace que el beneficio económico sea cero es de \textbf{10.666,67 euros}.
\end{ejercicio}

\begin{ejercicio}
Este ejercicio se enmarca dentro del \textbf{Análisis Coste-Volumen-Beneficio (CVB)}, que relaciona los costes, el volumen de producción y el beneficio, asumiendo que lo que se produce se vende.

El punto clave es la relación entre el \textbf{Margen Bruto Total} ($M$) y los \textbf{Costes Fijos de Explotación} ($F$). El \textbf{punto muerto} ($X_0$) se alcanza cuando el Margen Bruto Total de ese volumen de ventas cubre exactamente los costes fijos de explotación.

\textbf{Datos proporcionados:}
\begin{itemize}
    \item Volumen de ventas real ($X$): $7.000$ unidades.
    \item Margen Bruto Total real ($M$): $140.000$ euros.
    \item Margen Bruto Total para el punto muerto ($M_0$): $50.000$ euros.
\end{itemize}

\underline{Determinación de Costes Fijos y Margen Unitario}

\begin{enumerate}
    \item \textbf{Costes Fijos de Explotación ($F$):}
    \[
    F = M_0 = 50.000 \, \text{euros}
    \]

    \item \textbf{Margen Bruto Unitario ($m$):}
    \[
    m = \frac{M}{X} = \frac{140.000 \, \text{euros}}{7.000 \, \text{unidades}} = 20 \, \text{euros/unidad}
    \]
\end{enumerate}

\underline{a) El beneficio económico que ha obtenido}

El \textbf{Beneficio Económico (BE)} se calcula como:
\[
BE = M - F
\]
Sustituyendo los valores:
\[
BE = 140.000 \, \text{euros} - 50.000 \, \text{euros} = 90.000 \, \text{euros}
\]

\textbf{El beneficio económico obtenido por la empresa en el último año es de 90.000 euros.}

\underline{b) El punto muerto en unidades físicas}

El \textbf{Punto Muerto} o \textbf{Umbral de Rentabilidad en unidades físicas} ($X_0$) se calcula como:
\[
X_0 = \frac{F}{m}
\]
Sustituyendo los valores:
\[
X_0 = \frac{50.000 \, \text{euros}}{20 \, \text{euros/unidad}} = 2.500 \, \text{unidades}
\]

\textbf{El punto muerto en unidades físicas es de 2.500 unidades.}

\underline{c) El punto muerto en unidades de tiempo}

El \textbf{Punto Muerto en unidades de tiempo} se calcula como:
\[
T_0 = \frac{X_0}{X} \times \text{Período de referencia}
\]
Asumiendo ventas uniformes a lo largo del año (12 meses):
\[
T_0 = \frac{2.500 \, \text{unidades}}{7.000 \, \text{unidades}} \times 12 \, \text{meses}
\]
\[
T_0 \approx 0,35714 \times 12 \, \text{meses} \approx 4,2857 \, \text{meses}
\]

Expresado en meses y días:
\begin{itemize}
    \item 4 meses.
    \item $0,2857 \times 30 \, \text{días} \approx 8,57 \, \text{días}$.
\end{itemize}

\textbf{El punto muerto en unidades de tiempo se alcanza a los 4 meses y, aproximadamente, 8,57 días.}
\end{ejercicio}





