\section*{Tema 4: Productividad y Análisis Coste-Volumen-Beneficio (CVB)}
\label{sec:formulario}

\subsection*{A. Productividad}

La productividad es una medida de la eficiencia de un proceso productivo, relacionando los resultados obtenidos (Output) con las entradas necesarias (Input) para generarlos.

\subsubsection*{Productividad General}
\begin{align*}
\textbf{Productividad} &= \frac{\text{Cantidad producida (OUTPUT)}}{\text{Factores productivos (INPUT)}} \quad
\end{align*}

\subsubsection*{Productividad Monofactorial (De un Solo Factor)}
La productividad monofactorial utiliza unidades físicas para el Output y para un factor productivo específico, midiendo el aprovechamiento real de un solo recurso.

\begin{align*}
\textbf{Productividad Laboral} &= \frac{\text{Output}}{\text{Horas de trabajo}} \quad \\
\textbf{Productividad Material} &= \frac{\text{Output}}{\text{Kg de m.p.}} \quad
\end{align*}

\subsubsection*{Productividad Multifactorial (Eficiencia Económica)}
Mide la productividad considerando múltiples factores de entrada. Para ello, se utiliza una unidad de medida común (unidades monetarias, u.m.) para considerar el coste económico de los factores y la evolución de los precios.

\begin{align*}
\textbf{P. Multifactorial (u.m.)} &= \frac{\text{Output (en u.m.)}}{\text{Trabajo + Material + Energía + Capital + Varios (en u.m.)}} \quad
\end{align*}

\subsubsection*{Tasa de Variación de la Productividad}
Se utiliza para analizar la evolución de la productividad al comparar dos períodos de tiempo ($t_1$ y $t_2$).

\begin{align*}
\textbf{Tasa variación productividad} &= \frac{\text{Productividad}_{t_2} - \text{Productividad}_{t_1}}{\text{Productividad}_{t_1}} \times 100 \quad
\end{align*}

\subsection*{B. Análisis Coste-Volumen-Beneficio (CVB)}

El Análisis CVB se centra en las relaciones entre el coste, el volumen de producción y el beneficio, asumiendo que lo que se produce se vende (perspectiva económica).

\subsubsection*{Componentes Clave del Modelo}
\begin{itemize}[leftmargin=2em]
    \item $X$ (ó $Q$): Volumen de unidades producidas y vendidas.
    \item $p$: Precio de venta por unidad de producto.
    \item $cv$: Coste variable unitario (depende de la producción).
    \item $CF$: Costes fijos de la explotación (independientes del volumen de producción).
    \item $IT$: Ingresos Totales ($IT = p \cdot X$).
    \item $CT$: Costes Totales ($CT = CF + cv \cdot X$).
\end{itemize}

\subsubsection*{Beneficio Económico (BE)}
El Beneficio Económico (BE) es la renta generada por los activos de la empresa al margen de su estructura financiera.

\begin{align*}
\textbf{BE} &= \text{Ingresos totales} - \text{Costes totales de la producción} \quad \\
\textbf{BE} &= IT - CT \quad \\
\textbf{BE} &= p \cdot X - (CF + cv \cdot X) \quad \\
\textbf{BE} &= X(p - cv) - CF \quad
\end{align*}

\subsubsection*{Margen Bruto Unitario ($m$) y Total ($M$)}
El Margen Bruto Unitario ($m$) indica el beneficio obtenido por unidad vendida, sin considerar los costes fijos (el excedente para cubrirlos).

\begin{align*}
m \text{ (Unitario)} &= p - cv \quad \\
M \text{ (Total)} &= m \cdot X \quad
\end{align*}

\subsubsection*{Punto Muerto o Umbral de Rentabilidad ($X_0$)}
Es el volumen de ventas en unidades físicas ($X_0$) que hace el beneficio económico igual a cero ($BE=0$), cubriendo exactamente los costes fijos ($CF = m \cdot X_0$).

\begin{align*}
X_0 \text{ (u.f.)} &= \frac{CF}{p - cv} \quad \\
X_0 \text{ (u.f.)} &= \frac{CF}{m} \quad
\end{align*}
\textbf{Condición de existencia:} El margen bruto unitario ($p - cv$) debe ser positivo, es decir, el precio de venta debe ser superior al coste variable unitario ($p > cv$).

\subsubsection*{Punto Muerto en Unidades Monetarias}
Es la cifra de ventas, expresada en unidades monetarias, que permite cubrir los costes fijos.

\begin{align*}
\textbf{PM}_{\textbf{u.m.}} &= X_0 \cdot p \quad
\end{align*}

\subsubsection*{Punto Muerto en Unidades de Tiempo}
Indica el momento dentro del período de referencia en el que la empresa empieza a tener beneficio económico. Para su cálculo, se necesita conocer $X_0$ (u.f.), las ventas previstas para el período ($X_{\text{Total}}$) y la distribución de estas ventas a lo largo del tiempo.

