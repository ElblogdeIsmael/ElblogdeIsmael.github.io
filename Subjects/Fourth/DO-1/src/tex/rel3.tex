
\section{Árboles de decisión}
\begin{ejercicio}
Silicon Inc., un fabricante de semiconductores, está investigando la posibilidad de
fabricar y comercializar un procesador. Este proyecto requerirá la compra de un
sofisticado sistema de CAD, o la contratación y la formación de varios ingenieros más. El
mercado para el producto puede ser favorable o desfavorable. Silicon Inc., por supuesto,
tiene la opción de no desarrollar el producto.
Con una acogida favorable del mercado, las ventas serían de 25.000 procesadores
vendidos a 100 dólares la unidad, y si la acogida del mercado no fuese favorable, las
ventas serían de tan sólo 8.000 procesadores vendidos a 100 dólares cada uno. El coste
del equipo CAD es de 500.000 dólares, pero el de contratar y preparar a tres nuevos
ingenieros es de tan sólo 375.000 dólares. Sin embargo, el coste de fabricación iría desde
los 50 dólares la unidad (cuando se fabrica sin el CAD) a 40 dólares (cuando se fabrica
con el CAD). La probabilidad de una acogida favorable del nuevo microprocesador es del
40\%, mientras que la probabilidad de una acogida mala es del 60\%.
Aplicando la técnica del árbol de decisión, ¿qué alternativa elegiría en base al
beneficio y cuáles serían los resultados?
\end{ejercicio}

\begin{solucion} Nos quedaría el siguiente grafo de decisiones de la figura \ref{fig:rel2_ej1}.

\begin{figure}
    \centering
    \resizebox{\textwidth}{!}{
    \begin{tikzpicture}[
        >={Stealth[round]},
        node distance=3cm,
        auto,
        shorten >=1pt,
        font=\sffamily\small
    ]

    % Estilos
    % \tikzstyle{state} = [draw, circle, minimum size=1.5cm, thick, fill=yellow!20]
    % \tikzstyle{block} = [rectangle, draw, fill=blue!10, rounded corners, text centered, minimum height=1cm, minimum width=2cm, thick]
    % \tikzstyle{none} = [draw=none, fill=none, text centered]

    % Nodos principales
    \node[block] (1) at (-5,0) {1};
    \node[block] (2) at (0,3) {2};
    \node[none] (3) at (0,-3) {$\triangle_5$};

    % Nodos intermedios
    \node[state, above right=2.2cm and 2.5cm of 2] (4) {3};
    \node[state, below right=2.2cm and 2.5cm of 2] (5) {4};

    % Nodos finales (triángulos)
    \node[none, above right=1.8cm and 2.5cm of 4] (4a) {$\triangle_1$};
    \node[none, below right=1.8cm and 2.5cm of 4] (5a) {$\triangle_2$};
    \node[none, above right=1.8cm and 2.5cm of 5] (4b) {$\triangle_3$};
    \node[none, below right=1.8cm and 2.5cm of 5] (5b) {$\triangle_4$};

    % Relaciones principales
    \draw[->, thick] (1) -- node[above, yshift=4pt] {Fabricar} (2);
    \draw[->, thick] (1) -- node[below, yshift=-4pt] {No fabricar} (3);

    % Relaciones secundarias (curvas suaves)
    \draw[->, thick, bend left=10] (2) to node[above,yshift=4pt] {CAA} (4);
    \draw[->, thick, bend right=10] (2) to node[below, yshift=-4pt] {Formación} (5);

    % Conexiones hacia triángulos
    \draw[->,dashed, thick, bend left=8] (4) to (4a);
    \draw[->,dashed, thick, bend right=8] (4) to (5a);
    \draw[->,dashed, thick, bend left=8] (5) to (4b);
    \draw[->,dashed, thick, bend right=8] (5) to (5b);

    \end{tikzpicture}
    }
    \captionof{figure}{Diagrama de Transición del AFD del Ejercicio 1.}
    \label{fig:rel2_ej1}
\end{figure}

\begin{align*}
    \triangle_1 &= 25000 \cdot (100 - 40) - 500000 = 1000000 \\
    \triangle_2 &= 8000 \cdot (100 - 40) - 500000 = -25000 \\
    \triangle_3 &= 25000 \cdot (100 - 50) - 375000 = 875000 \\
    \triangle_4 &= 8000 \cdot (100 - 50) - 375000 = 25000 \\
    \triangle_5 &= 0 \\
\end{align*}

Cálculo de los valores monetarios esperados de derecha a izquierda:
\begin{align*}
    VME_3 = & 0.4 \cdot 1000000 - 20000 \cdot 0.6  = 3880000 \\
    VME_4 = 365000 \\
    VME_2 = 388000 \\
    VME_1 = 388000 \\
\end{align*}

Podemos concluir con que la mejor solución es comprar el CADA y el VME sería de 388000 dólares.



\end{solucion}


\begin{ejercicio}
Una determinada empresa se ve en la necesidad de decidir lo antes posible, sin ninguna
otra alternativa posible, el lanzamiento al mercado de tres posibles productos a los que
ha denominado A, B y C. Los resultados que caben esperar están asociados a distintos
estados de la naturaleza, así como a la toma de otra serie de decisiones.

Si decidimos lanzar el producto A, el beneficio previsto es de 9.500 euros en el supuesto
que sea objeto de una buena aceptación, lo que se espera que ocurra con una
probabilidad del 60\%. Si el nivel de aceptación es bajo, el beneficio ascenderá tan solo a
4.500 euros.

El lanzamiento del producto B plantea mayores dificultades. Por una parte, la viabilidad
del proyecto depende de que se produzca o no un cambio legislativo en los próximos
meses que considere saludable el material utilizado en su fabricación. Existe una
probabilidad del 70\% de que este cambio tenga lugar, en cuyo caso, se haría una
campaña publicitaria que, de tener éxito, permitiría obtener un beneficio de 16.000
euros. Si la campaña resultara un fracaso, el beneficio ascendería a la mitad del caso
anterior. Si no se produce el cambio legislativo esperado entonces la empresa debe
decidir si abandonar el proyecto, modificar rápidamente la composición de materiales
(con lo que se encarecería el coste de producción), o continuar vendiendo su producto
actual. Si se abandona el proyecto los resultados van de un beneficio de 6.000 euros si
no se tiene que devolver la subvención recibida, lo que puede ocurrir con una
probabilidad del 30\%, o a unas pérdidas de 800 euros si se tiene que devolver la
subvención. El cambio de materiales, a pesar de encarecer el producto, nos llevaría a
unos beneficios de 2.538,46 euros si la competencia no consigue lanzar al mercado su
nuevo producto, lo que se espera que ocurra con una probabilidad del 65\%. Si la
competencia lanza su producto, el beneficio solo alcanzaría la cifra de 1.000 euros. Si la
empresa decide seguir con su producto actual, el beneficio esperado será de 1.900
euros, siempre que no aparezca en el mercado ninguna versión mejorada, lo que se cree
que ocurrirá con una probabilidad del 80\%. Si apareciera la versión mejorada, los
beneficios se situarían en 750 euros.

Por último, si se opta por fabricar el producto C, su éxito solo depende de la liberación
del mercado de este producto, es decir, la desaparición de aranceles a la importación de
productos sustitutivos. Si esto ocurre, el beneficio será de 6.000 euros, mientras que si
no ocurre será de 10.000 euros.

A la vista de esta información, ¿qué decisión o secuencia de decisiones debe tomar la
empresa en función del criterio del beneficio esperado aplicando la técnica de los
árboles de decisión?
\end{ejercicio}

\begin{solucion} Debemos de tener en cuenta que en los nodos cuadrados se coge el mayor valor monetario esperado y en los nodos circulares se hace la media ponderada por las probabilidades. Los nodos circulares representan aquellas situaciones en las que se debe de elegir entre varias opciones y los nodos cuadrados representan aquellas situaciones en las que no se puede elegir.

Nos quedaría el siguiente grafo de decisiones de la figura \ref{fig:rel2_ej2}.


\begin{figure}
    \centering
    \resizebox{\textwidth}{!}{
    \begin{tikzpicture}[
        >={Stealth[round]},
        node distance=3.5cm,
        auto,
        shorten >=1pt,
        font=\sffamily\small
    ]

    % --- Nodos principales ---
    \node[block] (1) {(1)};
    \node[state, above right=7cm and 7cm of 1] (2) {(2)};
    \node[state, right=7cm of 1] (3) {(3)};
    \node[state, below right=7cm and 7cm of 1] (4) {(4)};

    % --- Relaciones principales ---
    \draw[->, thick, bend left=10] (1) to node[above] {Fabricar A(2)} (2);
    \draw[->, thick] (1) -- node[above] {Fabricar B(3)} (3);
    \draw[->, thick, bend right=10] (1) to node[below] {Fabricar C(4)} (4);

    % --- Nodos secundarios para A (más separados vertical y horizontalmente) ---
    \node[none, above right=2cm and 2cm of 2] (2a) { $\triangle_1 = 9500$};
    \node[none, below right=2cm and 2cm of 2] (2b) { $\triangle_2 = 4500$};

    % --- Relaciones secundarias para A ---
    \draw[->, dashed, bend left=5] (2) to node[above] {60\% Alta aceptación} (2a);
    \draw[->, dashed, bend right=5] (2) to node[below] {Baja aceptación 40\%} (2b);

    % --- Nodos secundarios para B (más separados vertical y horizontalmente) ---
    \node[block, above right=2cm and 2cm of 3] (3a) {(5)};
    \node[block, below right=2cm and 5cm of 3] (3b) {(6)};

    % --- Relaciones secundarias para B ---
    \draw[->, dashed, bend left=5] (3) to node[above] {Cambio legislativo 70\%} (3a);
    \draw[->, dashed, bend right=5] (3) to node[below] {No cambio legislativo 30\%} (3b);

    % --- Nodos terciarios para 3a ---
    \node[none, above right=2cm and 2cm of 3a] (3a1) {$\triangle_3 = 16000$};
    \node[none, below right=2cm and 2cm of 3a] (3a2) {$\triangle_4 = 8000$};

    % --- Relaciones terciarias para 3a ---
    \draw[->, dashed, bend left=5] (3a) to node[above] {Éxito 50\%} (3a1);
    \draw[->, dashed, bend right=5] (3a) to node[below] {Fracaso 50\%} (3a2);

    % --- Nodos terciarios para 3b ---
    \node[state, above right=7cm and 7cm of 3b] (3b1) {(7)};
    \node[state, right=8cm of 3b] (3b2) {(8)};
    \node[state, below right=7cm and 7cm of 3b] (3b3) {(9)};

    % --- Relaciones terciarias para 3b ---
    \draw[->, dashed, bend left=5] (3b) to node[above] {Abandona 33\%} (3b1);
    \draw[->, dashed] (3b) -- node[above] {Modifica 33\%} (3b2);
    \draw[->, dashed, bend right=5] (3b) to node[below] {Continúa 33\%} (3b3);

    % --- Nodos cuaternarios para 3b1 (Abandona) ---
    \node[none, above right=2cm and 2cm of 3b1] (3b1a) {$\triangle_5 = 6000$};
    \node[none, below right=2cm and 2cm of 3b1] (3b1b) {$\triangle_6 = -800$};

    % --- Relaciones cuaternarias para 3b1 ---
    \draw[->, dashed, bend left=5] (3b1) to node[above] {No devolver 30\%} (3b1a);
    \draw[->, dashed, bend right=5] (3b1) to node[below] {Devolver 70\%} (3b1b);

    % --- Nodos cuaternarios para 3b2 (Modifica) ---
    \node[none, above right=2cm and 2cm of 3b2] (3b2a) {$\triangle_7 = 2538.46$};
    \node[none, below right=2cm and 2cm of 3b2] (3b2b) {$\triangle_8 = 1000$};

    % --- Relaciones cuaternarias para 3b2 ---
    \draw[->, dashed, bend left=5] (3b2) to node[above] {No competencia 65\%} (3b2a);
    \draw[->, dashed, bend right=5] (3b2) to node[below] {Competencia  35\%} (3b2b);

    % --- Nodos cuaternarios para 3b3 (Continúa) ---
    \node[none, above right=2cm and 2cm of 3b3] (3b3a) {$\triangle_9 = 1900$};
    \node[none, below right=2cm and 2cm of 3b3] (3b3b) {$\triangle_{10} = 1000$};

    % --- Relaciones cuaternarias para 3b3 ---
    \draw[->, dashed, bend left=5] (3b3) to node[above] {No versión mejorada  80\%} (3b3a);
    \draw[->, dashed, bend right=5] (3b3) to node[below] {Versión mejorada 20\%} (3b3b);

    % --- Nodos secundarios para C ---
    \node[none, above right=2cm and 2cm of 4] (4a) {$\triangle_{11} = 6000$};
    \node[none, below right=2cm and 2cm of 4] (4b) {$\triangle_{12} = 10000$};

    % --- Relaciones secundarias para C ---
    \draw[->, dashed, bend left=5] (4) to node[above] {Liberar aranceles 50\%} (4a);
    \draw[->, dashed, bend right=5] (4) to node[below] {No liberar aranceles 50\%} (4b);



    \end{tikzpicture}
    }
    \caption{Diagrama de Transición del AFD del Ejercicio 2.}
    \label{fig:rel2_ej2}
\end{figure}

Los valores monetarios esperados de los nodos terminales son:
\begin{align*}
    VME_2 = & 0.6 \cdot 9500 + 0.4 \cdot 4500 = 9000  \\
    % VME_3 = & 9000 \\
    % VME_4 = & 8000 \\
    VME_7 = & 0.3 \cdot 6000 + 0.7 \cdot (-800) = 1240 \\ 
    VME_8 = & 0.65 \cdot 2538.46 + 0.35 \cdot 1000 = 1999.99 \approx 2000 \\
    VME_6 = & 2000 (\text{ Cogemos el de mayor valor}) \\
    \cdots \\
\end{align*}

No se ha calculado todos los valores monetarios esperados de los nodos debido a que se presupone que con estos ejemplos, los demás son triviales. 
Por ende, podemos concluir que la solución es que la mejor opción es fabricar el producto B, puede que haya cambio o no, la campaña puede tener éxito o no, ...
Podemos concluir que la mejor opción es fabricar el producto B y el VME sería de 9000 euros.

\end{solucion}


\begin{ejercicio}
Inverter, S.L. es una empresa dedicada a la fabricación de material eléctrico y de
domótica. Actualmente cuenta con dos plantas de producción con una superficie
construida superior a los 12.000 m2, dotadas de las más modernas tecnologías. El año
pasado, gracias al repunte de las ventas, consiguieron un beneficio extraordinario de 1
millón de euros que quieren destinar a una inversión en un incremento de la capacidad
instalada, con el objetivo de diversificar su oferta y dar servicio a una mayor cuota de
mercado.

Después de analizar las distintas alternativas, se plantean dos posibilidades. En primer
lugar, la dirección de Inverter estima que es factible abrir una nueva planta para la
producción de nuevos productos de domótica. Si se destina a sistemas de alarmas, existe
un 30\% de probabilidad que se revalorice ese mercado, y así las ventas permitirían
obtener un resultado igual al capital invertido más un 40\%. No obstante, si no se
revaloriza el mercado, el resultado supondría una disminución del 40\% sobre el
montante invertido. Asimismo, podría optar por sistemas de hilo musical. En este caso,
si la demanda es alta (30\% de probabilidad), el resultado obtenido permitiría recuperar
la inversión y obtener unos beneficios del 20\%. Pero si ésta es baja, el resultado se
reduciría a recuperar un 80\% de la inversión.

Por otro lado, la nueva capacidad podría destinarse a la producción de nuevos productos
de material eléctrico, aunque en este caso deberán realizar la inversión en colaboración
con otra empresa para minimizar los riesgos. La dirección ha estado informándose sobre
distintas empresas candidatas y, finalmente, deberá decidir entre hacer una alianza
estratégica con un socio extranjero, o bien subcontratar a una empresa nacional. Si opta
por la alianza, el resultado podría suponer un 20\% adicional a lo invertido si la demanda
se mantiene (lo cual ocurriría con un 80\% de probabilidad), mientras que, en caso
contrario, el resultado obtenido llevaría a recuperar sólo el 70\% de lo invertido. En el
caso de la unión con la empresa nacional, si la demanda se mantiene (80\% de
probabilidad), los resultados podrían suponer una revalorización del 30\% sobre la
inversión inicial, asumiendo un menor resultado (recuperar el 60\% de lo invertido) en el
caso contrario.

Represente y resuelva el problema planteado mediante un árbol de decisión,
indicando la decisión o secuencia de decisiones a tomar por Inverter, S.L.
\end{ejercicio}
\begin{solucion} El grafo de decisiones corresponde con el de la figura \ref{fig:rel2_ej3}.

\begin{figure}
    \centering
    \resizebox{\textwidth}{!}{
    \begin{tikzpicture}[
        >={Stealth[round]},
        node distance=3.5cm,
        auto,
        shorten >=1pt,
        font=\sffamily\small
    ]

    % --- Nodos principales ---
    \node[block] (1) {(1)};

    % --- Nodos secundarios ---
    \node[block, above right=7cm and 7cm of 1] (2) {(2) };
    \node[block, below right=7cm and 7cm of 1] (3) {(3) };

    % --- Relaciones principales ---
    \draw[->, thick, bend left=10] (1) to node[above] {Productos Domésticos} (2);
    \draw[->, thick, bend right=10] (1) to node[below] {Productos Eléctricos} (3);

    % --- Nodos terciarios para Productos Domésticos ---
    \node[state, above right=3cm and 3cm of 2] (2a) {(4)   };
    \node[state, below right=3cm and 3cm of 2] (2b) {(5)  };

    % --- Relaciones terciarias para Productos Domésticos ---
    \draw[->, thick, bend left=10] (2) to node[above] {Sistemas de Alarmas} (2a);
    \draw[->, thick, bend right=10] (2) to node[below] {Hilo Musical} (2b);

    % --- Nodos cuaternarios para Sistemas de Alarmas ---
    \node[none, above right=2cm and 2cm of 2a] (2a1) {$\triangle_1 = 1.4 \cdot \text{Inversión}$};
    \node[none, below right=2cm and 2cm of 2a] (2a2) {$\triangle_2 = 0.6 \cdot \text{Inversión}$};

    % --- Relaciones cuaternarias para Sistemas de Alarmas ---
    \draw[->, dashed, bend left=5] (2a) to node[above] {Revaloriza 30\%} (2a1);
    \draw[->, dashed, bend right=5] (2a) to node[below] {No revaloriza 70\%} (2a2);

    % --- Nodos cuaternarios para Hilo Musical ---
    \node[none, above right=2cm and 2cm of 2b] (2b1) {$\triangle_3 = 1.2 \cdot \text{Inversión}$};
    \node[none, below right=2cm and 2cm of 2b] (2b2) {$\triangle_4 = 0.8 \cdot \text{Inversión}$};

    % --- Relaciones cuaternarias para Hilo Musical ---
    \draw[->, dashed, bend left=5] (2b) to node[above] {Demanda alta 30\%} (2b1);
    \draw[->, dashed, bend right=5] (2b) to node[below] {Demanda baja 70\%} (2b2);

    % --- Nodos terciarios para Productos Eléctricos ---
    \node[state, above right=3cm and 3cm of 3] (3a) {(6)};
    \node[state, below right=3cm and 3cm of 3] (3b) {(7)};

    % --- Relaciones terciarias para Productos Eléctricos ---
    \draw[->, thick, bend left=10] (3) to node[above] {Extranjero} (3a);
    \draw[->, thick, bend right=10] (3) to node[below] {Nacional} (3b);

    % --- Nodos cuaternarios para Extranjero ---
    \node[none, above right=2cm and 2cm of 3a] (3a1) {$\triangle_5 = 1.2 \cdot \text{Inversión}$};
    \node[none, below right=2cm and 2cm of 3a] (3a2) {$\triangle_6 = 0.7 \cdot \text{Inversión}$};

    % --- Relaciones cuaternarias para Extranjero ---
    \draw[->, dashed, bend left=5] (3a) to node[above] {Demanda alta 80\%} (3a1);
    \draw[->, dashed, bend right=5] (3a) to node[below] {Demanda baja 20\%} (3a2);

    % --- Nodos cuaternarios para Nacional ---
    \node[none, above right=2cm and 2cm of 3b] (3b1) {$\triangle_7 = 1.3 \cdot \text{Inversión}$};
    \node[none, below right=2cm and 2cm of 3b] (3b2) {$\triangle_8 = 0.6 \cdot \text{Inversión}$};

    % --- Relaciones cuaternarias para Nacional ---
    \draw[->, dashed, bend left=5] (3b) to node[above] {Demanda alta 80\%} (3b1);
    \draw[->, dashed, bend right=5] (3b) to node[below] {Demanda baja 20\%} (3b2);


    \end{tikzpicture}
    }
    \caption{Diagrama de Transición del AFD del Ejercicio 3.}
    \label{fig:rel2_ej3}
\end{figure}


\begin{align*}
    VME_4 = & 0.3 \cdot 1.4 \cdot \text{Inversión} + 0.7 \cdot 0.6 \cdot \text{Inversión} \\
    VME_5 = & 0.3 \cdot 1.2 \cdot \text{Inversión} + 0.7 \cdot 0.8 \cdot \text{Inversión} \\
    VME_2 = & \max(VME_4, VME_5) \\
    VME_6 = & 0.8 \cdot 1.2 \cdot \text{Inversión} + 0.2 \cdot 0.7 \cdot \text{Inversión} \\
    VME_7 = & 0.8 \cdot 1.3 \cdot \text{Inversión} + 0.2 \cdot 0.6 \cdot \text{Inversión} \\
    VME_3 = & \max(VME_6, VME_7) \\
    VME_1 = & \max(VME_2, VME_3) \\
\end{align*}
Tomamos el valor de inversión como 1.000.000 euros, de manera que los valores numéricos serían:
\begin{align*}
    VME_4 &= 840,000 \,€ \\
    VME_5 &= 920,000 \,€ \\
    VME_2 &= 920,000 \,€ \\
    VME_6 &= 1,100,000 \,€ \\
    VME_7 &= 1,160,000 \,€ \\
    VME_3 &= 1,160,000 \,€ \\
    VME_1 &= 1,160,000 \,€ \\
\end{align*}

Por lo tanto, la mejor opción es invertir en productos eléctricos con una VME de 1.160.000 euros.

\end{solucion}





\begin{ejercicio}
A pesar de unos excelentes beneficios de 3.900 euros, la empresa PROVASA debe elegir
su estrategia de productos para el siguiente periodo de planificación dado el dinamismo
con el que se mueve el entorno. En este sentido, cuatro son las estrategias que se están
planteando. Puede realizar internamente el diseño de un nuevo producto, o bien
asociarse con otras empresas que ya tienen desarrollados los productos que son de
interés. También cabe la posibilidad de seguir como hasta ahora, o lanzar al mercado
uno de los productos que la empresa ya diseñó hace tiempo pero que nunca ofertó al
mercado.

Si se procede al diseño interno del producto los resultados van a depender, en una
primera instancia de la decisión que se tome en cuanto a dirigirse a un mercado de alta
gama o de baja gama. En el caso de un mercado de gama alta, esto permitiría a PROVASA
obtener un beneficio de 5.000 euros si la demanda es alta y de 2.000 euros si es baja. En
el caso de dirigirse a un mercado de baja gama los beneficios esperados serían de 4.000
euros en el escenario de alta demanda, y de 2.500 euros si la demanda es baja. PROVASA
no tiene experiencia en mercados de gama baja, por tanto, desconoce el
comportamiento de la demanda. Sin embargo, estima que en el mercado de gama alta
la probabilidad de que la demanda sea elevada es del 70\%.

En el caso de asociación los resultados dependerán de que las otras empresas acepten
o no dicha colaboración. Se estima en un 60\% la probabilidad de que las otras empresas
acepten la fórmula de la asociación. Si las otras empresas aceptan, PROVASA podría
inclinarse por una joint-venture, controlando el capital en un 60\% que le permitiría
definir la estrategia a seguir y le reportaría un beneficio de 5.500 euros. En cambio, si se
decantan por una simple asociación de colaboración, la falta de claridad en las relaciones
entre las empresas haría que los resultados esperados fueran más inciertos. Por ello, al
no tener el control de la estrategia, el beneficio que se espera obtener es de 4.500 euros
si la competencia no reacciona y de 2.500 euros si se produce una reacción fuerte de la
competencia.

En el caso que las otras empresas no acepten la asociación, solo cabría la posibilidad de
seguir como hasta ahora, con un beneficio de 3.000 euros, o bien crear un equipo de
trabajo con personal de la empresa para proceder al diseño de un nuevo producto con
nuestros propios medios. En este último caso, todo depende de que el tiempo disponible
sea suficiente para que el equipo de trabajo cumpla con su objetivo a un nivel
satisfactorio. Si el tiempo disponible es suficiente, el beneficio sería de 4.250 euros,
mientras que, si el tiempo no es suficiente, las pérdidas serían de 1.500 euros.

Esta complejidad en la decisión contrasta con la “simpleza” de seguir como hasta ahora,
en cuyo caso por la falta de adaptación al mercado, el beneficio sería solo de 3.000
euros.

Inclinarse por la opción de lanzar al mercado el producto que en su día ya diseñó la
empresa supone unos beneficios de 3.750 euros en el caso de que la demanda de este
tipo de producto sea alta, o de 1.800 euros si resultara que la demanda es baja, siendo
la probabilidad de esta última circunstancia del 20\%.

Utilizando la técnica de los árboles de decisión, ¿qué decisión o secuencia de
decisiones deben tomar los directivos de PROVASA?
\end{ejercicio}


\begin{solucion} El grafo de decisiones corresponde con el de la figura \ref{fig:rel2_ej4}.


\begin{figure}
    \centering
    \resizebox{\textwidth}{!}{
    \begin{tikzpicture}[
        >={Stealth[round]},
        node distance=3.5cm,
        auto,
        shorten >=1pt,
        font=\sffamily\small
    ]

    % --- Nodos principales ---
    \node[block] (1) {(1)};

    % --- Nodos secundarios ---
    \node[state, above right=7cm and 7cm of 1] (2) {(2))};
    \node[state, right=15cm of 1] (3) {(3)};
    \node[state, below right=7cm and 7cm of 1] (4) {(4)};
    \node[state, below right=14cm and 7cm of 1] (5) {(5)};

    % --- Relaciones principales ---
    \draw[->, thick, bend left=10] (1) to node[above] {Diseño Interno} (2);
    \draw[->, thick] (1) -- node[above] {Asociarse} (3);
    \draw[->, thick, bend right=10] (1) to node[below] {Seguir Así} (4);
    \draw[->, thick, bend right=20] (1) to node[below] {Lanzar Producto} (5);

    % --- Nodos terciarios para Diseño Interno ---
    \node[state, above right=3cm and 3cm of 2] (2a) {(6)};
    \node[state, below right=1cm and 5cm of 2] (2b) {(7)};

    % --- Relaciones terciarias para Diseño Interno ---
    \draw[->, thick, bend left=10] (2) to node[above] {Alta Gama} (2a);
    \draw[->, thick, bend right=10] (2) to node[below] {Baja Gama} (2b);

    % --- Nodos cuaternarios para Alta Gama ---
    \node[none, above right=2cm and 2cm of 2a] (2a1) {$\triangle_1 = 5000$};
    \node[none, below right=2cm and 2cm of 2a] (2a2) {$\triangle_2 = 2000$};

    % --- Relaciones cuaternarias para Alta Gama ---
    \draw[->, dashed, bend left=5] (2a) to node[above] {Alta Demanda 70\%} (2a1);
    \draw[->, dashed, bend right=5] (2a) to node[below] {Baja Demanda 30\%} (2a2);

    % --- Nodos cuaternarios para Baja Gama ---
    \node[none, above right=2cm and 2cm of 2b] (2b1) {$\triangle_3 = ?$};
    \node[none, below right=2cm and 2cm of 2b] (2b2) {$\triangle_4 = ?$};

    % --- Relaciones cuaternarias para Baja Gama ---
    \draw[->, dashed, bend left=5] (2b) to node[above] {Incierto 1} (2b1);
    \draw[->, dashed, bend right=5] (2b) to node[below] {Incierto 2} (2b2);

    % --- Nodos terciarios para Asociarse ---
    \node[block, above right=3cm and 3cm of 3] (3a) {(8)};
    \node[block, below right=3cm and 3cm of 3] (3b) {(9)};

    % --- Relaciones terciarias para Asociarse ---
    \draw[->, thick, bend left=10] (3) to node[above] {Aceptan 60\%} (3a);
    \draw[->, thick, bend right=10] (3) to node[below] {No Aceptan 40\%} (3b);

    % --- Nodos cuaternarios para Aceptan ---
    \node[none, above right=2cm and 2cm of 3a] (3a1) {$\triangle_5=5500$};
    \node[state, below right=2cm and 2cm of 3a] (3a2) {(10)};

    % --- Relaciones cuaternarias para Aceptan ---
    \draw[->, thick, bend left=5] (3a) to node[above] {Joint-Venture 60\%} (3a1);
    \draw[->, thick, bend right=5] (3a) to node[below] {Simple Asociación 40\%} (3a2);

    % --- Nodos quíntuples para Competencia ---
    \node[none, above right=2cm and 2cm of 3a2] (3a2a) {$\triangle_6 = 4500$};
    \node[none, below right=2cm and 2cm of 3a2] (3a2b) {$\triangle_7 = 2500$};

    % --- Relaciones quíntuples para Competencia ---
    \draw[->, dashed, bend left=5] (3a2) to node[above] {No reacciona 50\%} (3a2a);
    \draw[->, dashed, bend right=5] (3a2) to node[below] {Reacciona 50\%} (3a2b);

    % --- Nodos quíntuples para No Aceptan ---
    \node[none, above right=2cm and 2cm of 3b] (3b1) {$\triangle_8 = 3000$};
    \node[state, below right=2cm and 2cm of 3b] (3b2) {(11)};

    % --- Relaciones quíntuples para No Aceptan ---
    \draw[->, thick, bend left=5] (3b) to node[above] {Seguir 50\%} (3b1);
    \draw[->, thick, bend right=5] (3b) to node[below] {Crear Equipo 50\%} (3b2);

    % --- Nodos séxtuples para Crear Equipo ---
    \node[none, above right=2cm and 2cm of 3b2] (3b2a) {$\triangle_9 = 4250$};
    \node[none, below right=2cm and 2cm of 3b2] (3b2b) {$\triangle_{10} = -1500$};

    % --- Relaciones séxtuples para Crear Equipo ---
    \draw[->, dashed, bend left=5] (3b2) to node[above] {Tiempo suficiente 50\%} (3b2a);
    \draw[->, dashed, bend right=5] (3b2) to node[below] {Tiempo insuficiente 50\%} (3b2b);

    \node[none, below right=2cm and 2cm of 4] (4c) {$\triangle_{13} = 3000$};
    \draw[->, dashed, bend right=5] (4) to node[below] {} (4c);

    % --- Nodos terciarios para Lanzar Producto ---
    \node[none, above right=2cm and 2cm of 5] (5a) {$\triangle_{11} = 3750$};
    \node[none, below right=2cm and 2cm of 5] (5b) {$\triangle_{12} = 1800$};

    % --- Relaciones terciarias para Lanzar Producto ---
    \draw[->, dashed, bend left=5] (5) to node[above] {Demanda alta 80\%} (5a);
    \draw[->, dashed, bend right=5] (5) to node[below] {Demanda baja 20\%} (5b);



    \end{tikzpicture}
    }
    \caption{Diagrama de Transición del AFD del Ejercicio 3.}
    \label{fig:rel2_ej4}
\end{figure}

\begin{align*}
    VME_6 = & 0.7 \cdot 5000 + 0.3 \cdot 2000 = 4100 \\
    VME_7 = & 0.5 \cdot ? + 0.5 \cdot ? = ? \\
    VME_2 = & \max(4100, ?) = 4100 \\
    VME_{10} = & 0.6 \cdot 5500 + 0.4 \cdot (0.5 \cdot 4500 + 0.5 \cdot 2500) = 4400 \\
    VME_{11} = & 0.5 \cdot 3000 + 0.5 \cdot (0.5 \cdot 4250 + 0.5 \cdot -1500) = 2125 \\
    VME_3 = & \max(4400, 2125) = 4400 \\
    VME_4 = & 3000 \\
    VME_5 = & 0.8 \cdot 3750 + 0.2 \cdot 1800 = 3150 \\
    VME_1 = & \max(4100, 4400, 3000, 3150) = 4400 \\
\end{align*}

Por lo tanto, la mejor opción es asociarse mediante una joint-venture con una VME de 4400 euros.

\end{solucion}









\begin{ejercicio}
La empresa Bergé es uno de los grupos líderes españoles y europeos en transporte,
logística y distribución de automóviles. El grupo, que tiene su origen en Bilbao en el siglo
XIX, ha crecido hasta consolidarse como empresa de exportación e importación de
vehículos en 11 países y distribuidora de automóviles de 29 marcas en 15 mercados de
Europa y Sudamérica. La empresa, dada su posición de liderazgo, debe seguir
satisfaciendo a sus accionistas, por lo que se plantea tres estrategias alternativas de
acción para mejorar su cartera de servicios, entre las que debe decidir una.

En primer lugar, la empresa plantea mejorar sus operaciones portuarias. Con base en
los estudios de mercado y datos históricos recabados, la demanda de este modo de
transporte marítimo para 2026 podría incrementarse, decrecer o estancarse. En caso de
que la demanda del transporte marítimo se incremente (la probabilidad estimada es del
55\%), el director comercial ha recibido información sobre el posible encarecimiento de
los gastos aduaneros. Si los gastos aduaneros no suben (con una probabilidad del 40\%),
las estimaciones basadas en el incremento de la demanda apuntan a un beneficio
esperado de 100 millones de euros. En caso de que finalmente los gastos aduaneros
aumenten, los resultados se verían mermados, por lo que el beneficio estimado se
situaría en los 75 millones de euros. Si la demanda del transporte marítimo decrece en
2026, lo cual puede ocurrir con un 25\% de probabilidad, la empresa tendrá que estimular
las ventas, presentándose entonces dos acciones alternativas entre las que elegir:
mejorar la gestión del almacenamiento en los puertos, en cuyo caso el beneficio se
situaría en 60 millones de euros, o invertir en una campaña de marketing, lo que llevaría
a obtener un beneficio estimado de 50 millones de euros. Si la demanda de transporte
marítimo se estanca, el equipo directivo de la empresa no tiene previstas acciones
posibles y asume un beneficio estimado de 65 millones de euros.

En segundo lugar, Bergé se plantea la opción de invertir en logística de valor añadido,
esto es, aprovechar sus operaciones de logística para que las empresas clientes mejoren
la eficiencia de sus procesos, lo que llevaría a obtener un beneficio esperado de 56
millones de euros.

Finalmente, Bergé considera una tercera opción estratégica, consistente en ampliar su
presencia internacional. Uno de los mercados en los que podría entrar es en Finlandia,
a través de la distribución de vehículos de la marca Mitsubishi. El beneficio esperado es
de 60 millones de euros, si la marca de vehículos tiene buena aceptación en el mercado.
En caso contrario, con una probabilidad de ocurrencia del 35\%, el beneficio esperado es
de 47 millones de euros. Por otra parte, Bergé plantea su introducción en México de la
mano de Audi. En esta opción, se encargaría desde la entrada de contenedores
marítimos y trailers en la planta, hasta la distribución final de los vehículos a la red de
concesionarios. El beneficio esperado es de 95 millones de euros, si se mantiene la 
estabilidad en las condiciones de suministro, con una probabilidad de ocurrencia del
45\%. En caso contrario, si las condiciones de suministro no son estables, el beneficio
esperado sería menor, concretamente de 70 millones de euros. Considerando la
información descrita, se pide:

\begin{enumerate}
    \item Represente y resuelva el problema planteado mediante un árbol de decisión.
    \item ¿Qué decisión debería tomar Bergé si atendemos al criterio del beneficio esperado?
\end{enumerate}

esperado? Exprese la solución por escrito indicando la decisión o secuencia de
decisiones a tomar.
\end{ejercicio}


\begin{solucion}
El grafo de decisiones corresponde con el de la figura \ref{fig:rel2_ej5}. Las cantidades están el millones de euros.

\begin{figure}
    \centering
    \resizebox{\textwidth}{!}{
    \begin{tikzpicture}[
        >={Stealth[round]},
        node distance=3cm,
        auto,
        shorten >=1pt,
        font=\sffamily\small
    ]

    % Nodo inicial
    \node[block] (0) {(1)};

    % Relación desde el nodo inicial al primer nodo principal
    % \draw[->, thick] (0) -- node[above] {Decisión inicial} (1);

    % Nodo principal
    \node[state, above right=7cm and 7cm of 0] (1) {(2)};
    \node[state, right=7cm of 0] (2) {(3)};
    \node[block, below right=7cm and 7cm of 0] (3) {(4)};

    % Relación desde el nodo inicial al nodo principal
    \draw[->, thick] (0) -- node[above] {Operaciones Portuarias} (1);
    \draw[->, thick, right=7cm of 0] (0) -- node[above] {Logística Valor Añadido} (2);
    \draw[->, thick, below right=7cm and 7cm of 0] (0) -- node[below] {Ampliar Presencia Internacional} (3);

    % Nodos secundarios para la demanda
    \node[state, above right=3cm and 3cm of 1] (2a) {(5)};
    \node[state, right=7cm of 1] (2b) {(6)};
    \node[block, below right=3cm and 3cm of 1] (2c) {(7)};

    % Relaciones desde el nodo 2 a los nodos secundarios
    \draw[->, thick, bend left=10] (1) to node[above] {Demanda Incrementa 55\%} (2a);
    \draw[->, thick] (1) -- node[above] {Demanda Estanca 20\%} (2b);
    \draw[->, thick, bend right=10] (1) to node[below] {Demanda Decrece 25\%} (2c);


    % Nodos terciarios para Incremento de Demanda
    \node[none, above right=2cm and 2cm of 2a] (2a1) {$\triangle_1 = 100$};
    \node[none, below right=2cm and 2cm of 2a] (2a2) {$\triangle_2 = 75$};

    % Relaciones terciarias para Incremento de Demanda
    \draw[->, dashed, bend left=5] (2a) to node[above] {No suben gastos 40\%} (2a1);
    \draw[->, dashed, bend right=5] (2a) to node[below] {Suben gastos 60\%} (2a2);

    % Nodos terciarios para Demanda Decrece
    \node[none, above right=2cm and 2cm of 2c] (2c1) {$\triangle_3 = 60$};
    \node[none, below right=2cm and 2cm of 2c] (2c2) {$\triangle_4 = 50$};

    % Relaciones terciarias para Demanda Decrece
    \draw[->, dashed, bend left=5] (2c) to node[above] {Mejorar Almacenamiento} (2c1);
    \draw[->, dashed, bend right=5] (2c) to node[below] {Invertir en Marketing} (2c2);

    % Nodo terciario para Demanda Estanca
    \node[none, right=2cm of 2b] (2b1) {$\triangle_5 = 65$};

    % Relación terciaria para Demanda Estanca
    \draw[->, dashed] (2b) -- (2b1);

    % Nodo hijo del nodo 3 (logística valor añadido)
    \node[none, right=3cm of 2] (2_1) {$\triangle_{6} = 56$};
    
    % Relación hacia el triángulo 56
    \draw[->, dashed] (2) -- (2_1);

    % Nodos secundarios para la demanda
    \node[state, above right=3cm and 3cm of 3] (3a) {(8)};
    \node[state, below right=3cm and 3cm of 3] (3b) {(9)};

    % Relaciones desde el nodo 3 a los nodos secundarios
    \draw[->, thick, bend left=10] (3) to node[above] {Finlandia } (3a);
    \draw[->, thick, bend right=10] (3) to node[below] {México } (3b);

    % Nodos terciarios para Finlandia buena aceptación o mala 
    \node[none, above right=2cm and 2cm of 3a] (3a1) {$\triangle_7 = 60$};
    \node[none, below right=2cm and 2cm of 3a] (3a2) {$\triangle_8 = 47$};

    % Relaciones terciarias para Finlandia
    \draw[->, dashed, bend left=5] (3a) to node[above] {Buena Aceptación 65\%} (3a1);
    \draw[->, dashed, bend right=5] (3a) to node[below] {Mala Aceptación 35\%} (3a2);

    % Nodos terciarios para México condiciones estables o inestables
    \node[none, above right=2cm and 2cm of 3b] (3b1) {$\triangle_9 = 95$};
    \node[none, below right=2cm and 2cm of 3b] (3b2) {$\triangle_{10} = 70$};

    % Relaciones terciarias para México
    \draw[->, dashed, bend left=5] (3b) to node[above] {Condiciones Estables 45\%} (3b1);
    \draw[->, dashed, bend right=5] (3b) to node[below] {Condiciones Inestables 55\%} (3b2);    

    \end{tikzpicture}
    }
    \captionof{figure}{Diagrama de Transición del AFD del Ejercicio 5.}
    \label{fig:rel2_ej5}
\end{figure}


\begin{align*}
    VME_{5} = & 0.4 \cdot 100 + 0.6 \cdot 75 = 85 \\
    VME_{6} = & 65 \\
    VME_{7} = & 0.5 \cdot 60 + 0.5 \cdot 50 = 55 \\
    VME_{2} = & 0.55 \cdot 85 + 0.2 \cdot 65 + 0.25 \cdot 55 = 73.5 \\
    VME_{3} = & 56 \\
    VME_{8} = & 0.65 \cdot 60 + 0.35 \cdot 47 = 55.95 \\
    VME_{9} = & 0.45 \cdot 95 + 0.55 \cdot 70 = 80.75 \\
    VME_{4} = & \max(55.95, 80.75) = 80.75 \\
    VME_{1} = & \max(73.5, 56, 80.75) = 80.75 \\
\end{align*}

Por lo tanto, la mejor opción es ampliar la presencia internacional con una VME de 80.75 millones de euros.




\end{solucion}






\begin{ejercicio}
Lesla Inc. es una empresa estadounidense que diseña, fabrica y vende automóviles
eléctricos. Con el fin de aumentar su capacidad productiva a gran escala y satisfacer la
demanda de vehículos prevista, su CEO, Nelson Dusk, está barajando dos alternativas de
construcción de gigafactorías. Deberá elegir entre dos opciones: Giga Berlín y Giga
Texas, teniendo también la opción de no construir ninguna (lo que conllevaría obtener
un beneficio nulo).
Después de analizar las alternativas correspondientes a la construcción de una nueva
gigafactoría, los resultados que cabe esperar están asociados a distintos estados de la
naturaleza, así como a la toma de otra serie de decisiones.
En primer lugar, Giga Berlín contaría con un área de 3 km2, lo que permitiría a Lesla
producir hasta 10.000 unidades semanales de su vehículo estrella, Lesla Model Y. La
demanda puede alcanzar ese nivel con un 70\% de probabilidad, lo que llevaría a obtener
un beneficio anual de 1,2 millones de dólares. Sin embargo, la demanda puede ser
menor, lo cual supondría menos ingresos y la asunción de costes por capacidad ociosa,
dando lugar a un menor beneficio, estimado en 600.000 dólares anuales.
En segundo lugar, Giga Texas, con 8,5 km2, contaría con una capacidad de producción
bastante superior a la de Giga Berlín: cerca de 22.000 unidades de vehículos a la semana.
Sin embargo, dadas sus elevadas dimensiones, su apertura podría retrasarse para el
próximo año con una probabilidad del 25\%, lo que le llevaría a incurrir en unas pérdidas
de 500.000 dólares. Si no hay retrasos y logra abrir la gigafactoría de Texas a tiempo, el
señor Dusk deberá elegir entre dos alternativas: producir solamente el Lesla Model Y o
desarrollar, además de este modelo, un nuevo producto, una camioneta eléctrica
denominada Lesla Cybertruck. 
En caso de desarrollar únicamente el Model Y, Lesla obtendría unos beneficios de 2,5
millones si la demanda es favorable, mientras que dichos beneficios se reducirían en 1
millón si la demanda resulta desfavorable. En caso de desarrollar también la camioneta,
Lesla debería repartir la capacidad de producción disponible entre ambos vehículos. Los
resultados estimados de esta estrategia, que se basa en economías de alcance para
producir ambos vehículos, dependerán de la demanda del modelo existente, Lesla
Model Y, y de la acogida en el mercado del nuevo modelo, Lesla Cybertruck. Así, si la
demanda de ambos modelos es favorable, algo que se estima que ocurra con un 35\% de
probabilidad, el beneficio estimado sería de 3 millones de dólares anuales. Sin embargo,
si la demanda del Model Y es favorable pero el Lesla Cybertruck tiene una mala acogida
en el mercado (circunstancias que se estima que ocurran con un 45\% de probabilidad),
el beneficio estimado sería de 1,3 millones de dólares, mientras que, si el Lesla
Cybertruck cuenta con una buena acogida en el mercado, pero la demanda del Model Y
no resulta favorable (circunstancias que se estiman con una probabilidad del 15\%), los
beneficios serían de 800.000 dólares únicamente. En el peor de los escenarios (ningún
modelo cuenta con demanda favorable y/o aceptación en el mercado), las pérdidas
ascenderían a 600.000 dólares.
Considerando la información descrita, ¿qué decisión o secuencia de decisiones debería
tomar Lesla si atendemos al criterio del beneficio esperado? Represente y resuelva el
problema planteado mediante un árbol de decisión, expresando la solución por
escrito.
\end{ejercicio}

\begin{solucion}
El grafo de decisiones corresponde con el de la figura \ref{fig:rel2_ej6}. Las cantidades están el millones de euros.

\begin{figure}
    \centering
    \resizebox{\textwidth}{!}{
    \begin{tikzpicture}[
        >={Stealth[round]},
        node distance=3cm,
        auto,
        shorten >=1pt,
        font=\sffamily\small
    ]

    % Nodo inicial
    \node[block] (0) {(1)};

    \node[block, above right=7cm and 7cm of 0] (1) {(2)};
    \node[none, below right=7cm and 7cm of 0] (2) {($\triangle_{10} = 0$)};

    % Relación entre el nodo 1 y el nodo 2 (Producir Factorías)
    \draw[->, thick, bend left=10] (0) to node[above] {Producir Factorías} (1);

    % Relación entre el nodo 1 y el nodo 3 (No Producir Factorías)
    \draw[->, thick, bend right=10] (0) to node[below] {No Producir Factorías} (2);

    % Nodos secundarios para Giga Berlín y Giga Texas
    \node[state, above right=3cm and 3cm of 1] (1a) {(3)};
    \node[state, below right=3cm and 3cm of 1] (1b) {(4)};

    % Relaciones desde el nodo 2 a los nodos secundarios
    \draw[->, thick, bend left=10] (1) to node[above] {Giga Berlín} (1a);
    \draw[->, thick, bend right=10] (1) to node[below] {Giga Texas} (1b);

    % Nodos terciarios para Giga Berlín
    \node[none, above right=2cm and 2cm of 1a] (1a1) {$\triangle_1 = 1.2 M$};
    \node[none, below right=2cm and 2cm of 1a] (1a2) {$\triangle_2 = 0.6 M$};

    % Relaciones terciarias para Giga Berlín
    \draw[->, dashed, bend left=5] (1a) to node[above] {Demanda Alta 70\%} (1a1);
    \draw[->, dashed, bend right=5] (1a) to node[below] {Demanda Baja 30\%} (1a2);

    % Nodos terciarios para Giga Texas
    \node[none, above right=2cm and 2cm of 1b] (1b1) {$\triangle_3 = -0.5 M$};
    \node[block, below right=2cm and 2cm of 1b] (1b2) {(5)};

    % Relaciones terciarias para Giga Texas
    \draw[->, dashed, bend left=5] (1b) to node[above] {Se retrasa 25\%} (1b1);
    \draw[->, dashed, bend right=5] (1b) to node[below] {No se retrasa 75\%} (1b2);

    % Nodos terciarios para Giga Texas (desarrollar Model Y o ambos)
    \node[state, above right=2cm and 2cm of 1b2] (1b2a) {(6)};
    \node[state, below right=2cm and 2cm of 1b2] (1b2b) {(7)};

    % Relaciones terciarias para Giga Texas
    \draw[->, thick, bend left=5] (1b2) to node[above] {Desarrollar Model Y} (1b2a);
    \draw[->, thick, bend right=5] (1b2) to node[below] {Desarrollar Ambos} (1b2b);

    % Nodos cuaternarios para desarrollar Model Y
    \node[none, above right=2cm and 2cm of 1b2a] (1b2a1) {$\triangle_4 = 2.5 M$};
    \node[none, below right=2cm and 2cm of 1b2a] (1b2a2) {$\triangle_5 = 1.5 M$};

    % Relaciones cuaternarias para desarrollar Model Y
    \draw[->, dashed, bend left=5] (1b2a) to node[above] {Demanda Favorable 50\%} (1b2a1);
    \draw[->, dashed, bend right=5] (1b2a) to node[below] {Demanda Desfavorable 50\%} (1b2a2);

    % Nodos cuaternarios para desarrollar Ambos
    \node[none, above right=2cm and 2cm of 1b2b] (1b2b1) {$\triangle_6 = 3 M$};
    \node[none, right=6cm of 1b2b] (1b2b2) {$\triangle_7 = 1.3 M$};
    \node[none, below right=2cm and 2cm of 1b2b] (1b2b3) {$\triangle_8 = 0.8 M$};
    \node[none, below right=4cm and 2cm of 1b2b] (1b2b4) {$\triangle_9 = -0.6 M$};

    % Relaciones cuaternarias para desarrollar Ambos
    \draw[->, dashed, bend left=5] (1b2b) to node[above] {Ambos Favorables 35\%} (1b2b1);
    \draw[->, dashed] (1b2b) -- ++(6,0) node[above] {Model Y Favorable, Cybertruck No 45\%} (1b2b2);
    \draw[->, dashed, bend right=5] (1b2b) to node[below] {Cybertruck Favorable, Model Y No 15\%} (1b2b3);
    \draw[->, dashed, bend right=18] (1b2b) to node[below] {Ninguno Favorable 5\%} (1b2b4);


    


    \end{tikzpicture}
    }
    \captionof{figure}{Diagrama de Transición del AFD del Ejercicio 6.}
    \label{fig:rel2_ej6}
\end{figure}

Ahora vamos a calcular los valores monetarios esperados (VME) de cada nodo:
\begin{align*}
    VME_6 = & 0.5 \cdot 2.5M + 0.5 \cdot 1.5M = 2M \\
    VME_7 = & 0.35 \cdot 3M + 0.45 \cdot 1.3M + 0.15 \cdot 0.8M + 0.05 \cdot (-0.6M) = 1.725M \\
    VME_5 = & max(2M, 1.725M) = 2M \\
    VME_3 = & 0.7 \cdot 1.2M + 0.3 \cdot 0.6M = 1.02M \\
    VME_4 = & 0.25 \cdot (-0.5M) + 0.75 \cdot 2M = 1.375M \\
    VME_2 = & max(1.02M, 1.375M) = 1.375M \\
    VME_1 = & max(1.375M, 0) = 1.375M \\
\end{align*}

De esta manera es directo ver que el camino solución el aquel que nos proporcionar un mayor VME, es decir, construir Giga Texas sin retrasos y desarrollar únicamente el Model Y, con un beneficio esperado de 1.375 millones de dólares anuales\footnote{Todo esto teniendo en cuenta que todo sale según lo planeado.}.





\end{solucion}






\begin{ejercicio}
''El futuro de Seat es Cupra''. Estas palabras de su director ejecutivo, Thomas Schäfer,
durante el Salón del Automóvil de Múnich 2023 han hecho saltar las alarmas sobre la
continuidad de la marca Seat, la rentabilidad de la planta de Martorell (Barcelona) y el
impacto en los puestos de trabajo de la fábrica catalana. Con el fin de luchar por la
viabilidad de la fábrica, que da empleo a más de 14.000 trabajadores, Seat España está
barajando varias alternativas, de tal manera que la rentabilidad de la planta quede
garantizada de cara a los próximos 10 años y sea una apuesta de futuro para la directiva
del grupo Volkswagen.
Una primera alternativa es apostar por las nuevas formas de movilidad vinculadas al
ámbito urbano. Esta decisión supondría una reconfiguración total del proceso de
producción. Si la directiva central no aprueba esta decisión, la empresa asumiría unas
pérdidas de 10.000 euros derivadas de la preparación de la propuesta. Si la directiva
aprueba esta decisión, lo que se espera que ocurra con un 90\% de probabilidad, la
empresa se plantea dos posibles cursos de acción. Por un lado, si la empresa decide 
fabricar patinetes eléctricos y hoverboards (sin manillar, se maneja con la inclinación del
cuerpo), podría alcanzar un beneficio esperado de 87 millones de euros si existe una alta
aceptación en el mercado, lo que se espera que ocurra con una probabilidad del 60\%.
En caso contrario, los beneficios esperados ascenderían a 62 millones de euros. Si, en
cambio, decide apostar por la fabricación de scooters eléctricas, podría alcanzar un
beneficio esperado de 105 millones de euros si existe un programa gubernamental de
incentivos a este tipo de vehículo eficiente, lo que se espera que ocurra con una
probabilidad del 30\%. En caso contrario, los beneficios esperados ascenderían a 58
millones de euros.
La segunda alternativa es apostar por la electrificación de los vehículos. Dos son los
modelos entre los que se plantea elegir. En primer lugar, la fábrica de Martorell puede
apostar por fabricar el modelo 100 \% eléctrico de Cupra denominado Born. Si la directiva
central apoya esta decisión, lo que se espera que ocurra con un 70\% de probabilidad,
los beneficios esperados serían de 88 millones de euros. En caso de no contar con el
apoyo de la directiva, los beneficios esperados serían de 44 millones de euros, ya que
continuarían con la fabricación de modelos de combustión e híbridos como hasta ahora.
En segundo lugar, podrían apostar por producir un Ibiza 100\% eléctrico de bajo coste y
competir con sus propios vehículos de combustión e híbridos, tal y como lo hace el
binomio Dacia-Renault. Convencer a la directiva de esta posibilidad es muy difícil, se
espera que ocurra con una probabilidad del 20\%. En caso positivo, el beneficio esperado
sería de 120 millones de euros. En caso contrario, continuarían con la fabricación de
modelos de combustión e híbridos como hasta ahora, con unos beneficios esperados de
44 millones de euros, al igual que en la decisión anterior.
Considerando la información descrita, ¿qué decisión o secuencia de decisiones debería
tomar Seat España si atendemos al criterio del beneficio esperado? Represente y
resuelva el problema planteado mediante un árbol de decisión. Represente y resuelva
el problema planteado mediante un árbol de decisión expresando la solución por
escrito.
\end{ejercicio}

\begin{solucion}
    El grafo de decisiones corresponde con el de la figura \ref{fig:rel2_ej7}. Algunas de las cantidades están el millones de euros.

\begin{figure}
    \centering
    \resizebox{\textwidth}{!}{
    \begin{tikzpicture}[
        >={Stealth[round]},
        node distance=3cm,
        auto,
        shorten >=1pt,
        font=\sffamily\small
    ]

    % Nodo inicial
    \node[block] (0) {(1)};

    % Nodos secundarios para nuevas formas de movilidad o electrificar vehículos
    \node[state, above right=3cm and 3cm of 0] (1c) {(2)};
    \node[block, below right=12cm and 3cm of 0] (1d) {(3)};

    % Relaciones desde el nodo inicial
    \draw[->, thick, bend left=10] (0) to node[above] {Nuevas formas de movilidad} (1c);
    \draw[->, thick, bend right=10] (0) to node[below] {Electrificar vehículos} (1d);

    % Nodos terciarios para la decisión de la directiva
    \node[none, above right=2cm and 2cm of 1c] (1c1) {$\triangle_{1} = -10000$};
    \node[block, below right=2cm and 2cm of 1c] (1c2) {(4)};

    % Relaciones terciarias para la decisión de la directiva
    \draw[->, dashed, bend left=5] (1c) to node[above] {No aprueba 10\%} (1c1);
    \draw[->, dashed, bend right=5] (1c) to node[below] {Aprueba 90\%} (1c2);

    % Nodos terciarios para patinetes y hoverboards o scooters eléctricas
    \node[state, above right=2cm and 2cm of 1c2] (1c2a) {(5)};
    \node[state, below right=2cm and 2cm of 1c2] (1c2b) {(6)};

    % Relaciones terciarias para patinetes y hoverboards o scooters eléctricas
    \draw[->, thick, bend left=5] (1c2) to node[above] {Patinetes y Hoverboards} (1c2a);
    \draw[->, thick, bend right=5] (1c2) to node[below] {Scooters Eléctricas} (1c2b);

    % Nodos cuaternarios para aceptación o no aceptación de patinetes y hoverboards
    \node[none, above right=2cm and 2cm of 1c2a] (1c2a1) {$\triangle_{3} = 87 M$};
    \node[none, below right=2cm and 2cm of 1c2a] (1c2a2) {$\triangle_{4} = 62 M$};

    % Relaciones cuaternarias para aceptación o no aceptación de patinetes y hoverboards
    \draw[->, dashed, bend left=5] (1c2a) to node[above] {Alta aceptación 60\%} (1c2a1);
    \draw[->, dashed, bend right=5] (1c2a) to node[below] {Baja aceptación 40\%} (1c2a2);

    % Nodos cuaternarios para programa gubernamental o no
    \node[none, above right=2cm and 2cm of 1c2b] (1c2b1) {$\triangle_{5} = 105M$};
    \node[none, below right=2cm and 2cm of 1c2b] (1c2b2) {$\triangle_{6} = 58M$};

    % Relaciones cuaternarias para programa gubernamental o no
    \draw[->, dashed, bend left=5] (1c2b) to node[above] {Programa 30\%} (1c2b1);
    \draw[->, dashed, bend right=5] (1c2b) to node[below] {No programa 70\%} (1c2b2);

    % Nodos terciarios para Cupra Born o Ibiza Eléctrico
    \node[state, above right=2cm and 2cm of 1d] (1d1) {(7)};
    \node[state, below right=2cm and 2cm of 1d] (1d2) {(8)};

    % Relaciones terciarias para Cupra Born o Ibiza Eléctrico
    \draw[->, thick, bend left=5] (1d) to node[above] {Cupra Born} (1d1);
    \draw[->, thick, bend right=5] (1d) to node[below] {Ibiza Eléctrico} (1d2);

    % Nodos cuaternarios para Cupra Born
    \node[none, above right=2cm and 2cm of 1d1] (1d1a) {$\triangle_{7} = 88M$};
    \node[none, below right=2cm and 2cm of 1d1] (1d1b) {$\triangle_{8} = 44M$};

    % Relaciones cuaternarias para Cupra Born
    \draw[->, dashed, bend left=5] (1d1) to node[above] {Directiva aprueba 70\%} (1d1a);
    \draw[->, dashed, bend right=5] (1d1) to node[below] {Directiva no aprueba 30\%} (1d1b);

    % Nodos cuaternarios para Ibiza Eléctrico
    \node[none, above right=2cm and 2cm of 1d2] (1d2a) {$\triangle_{9} = 120M$};
    \node[none, below right=2cm and 2cm of 1d2] (1d2b) {$\triangle_{10} = 44M$};

    % Relaciones cuaternarias para Ibiza Eléctrico
    \draw[->, dashed, bend left=5] (1d2) to node[above] {Directiva aprueba 20\%} (1d2a);
    \draw[->, dashed, bend right=5] (1d2) to node[below] {Directiva no aprueba 80\%} (1d2b);



    


    \end{tikzpicture}
    }
    \captionof{figure}{Diagrama de Transición del AFD del Ejercicio 7.}
    \label{fig:rel2_ej7}
\end{figure}

Ahora vamos a calcular los valores monetarios esperados (VME) de cada nodo:
\begin{align*}
    VME_5 = & 0.6 \cdot 87M + 0.4 \cdot 62M = 77M \\
    VME_6 = & 0.3 \cdot 105M + 0.7 \cdot 58M = 72.1M \\
    VME_7 = & 0.7 \cdot 88M + 0.3 \cdot 44M = 74.8M \\
    VME_8 = & 0.2 \cdot 120M + 0.8 \cdot 44M = 59.2M \\
    VME_4 = & max(77M, 72.1M) = 77M \\
    VME_3 = & max(74.8M, 59.2M) = 74.8M \\
    VME_2 = & 0.9 \cdot 77M + 0.1 \cdot (-0.01M) = 69.299M \\
    VME_1 = & max(69.299M, 74.8M) = 74.8M \\
\end{align*}

Por lo tanto, la mejor opción es electrificar vehículos apostando por fabricar el modelo 100 \% eléctrico de Cupra denominado Born, con una VME de 74.8 millones de euros.





\end{solucion}







\begin{ejercicio}
Tom tiene un negocio de venta de automóviles. Un importante comerciante de la ciudad
le ofrece la oportunidad de realizar un contrato por el que podría elegir un vehículo de
este distribuidor para tratar de venderlo. Si lo vende, entonces tendría la posibilidad de
escoger otro vehículo diferente. Existe un total de 3 vehículos distintos que Tom puede
elegir vender (ver la tabla a continuación). Igualmente, Tom tiene la posibilidad de no
aceptar la oferta del comerciante, obteniendo un beneficio nulo.
El comerciante concluye la oferta de la siguiente manera:
''Tom, no hemos tratado con usted de forma previa, así que procederemos con cautela:
si acepta este trato, primero debe tomar el Micra. Si vende el Micra, obtendrá su
comisión de venta y podrá elegir entre el Focus o el Passat, o bien finalizar el contrato.
En caso de decidir continuar con el contrato, si vende el segundo coche (obteniendo la
correspondiente comisión), puede elegir el modelo restante para venderlo, o bien
finalizar el contrato.”
\begin{table}[H]
\centering
\begin{tabular}{|c|c|c|c|}
\hline
\textbf{Modelo} & \textbf{Comisión de Tom por venta} & \textbf{Costes} & \textbf{Probabilidad de venta} \\ \hline
Micra & 900 euros & 600 euros & $\frac{3}{4}$ \\ \hline
Focus & 1500 euros & 200 euros & $\frac{2}{3}$ \\ \hline
Passat & 3000 euros & 600 euros & $\frac{1}{2}$ \\ \hline
\end{tabular}
\caption{Datos de los vehículos disponibles para Tom.}
\label{tab:vehiculos_tom}
\end{table}
la(s) venta(s) de vehículo(s) que realice Tom según proceda en cada caso, y
considerando los costes asociados a cada modelo. Incluya también un breve informe
aconsejando qué decisión debería tomar Tom en función del criterio del beneficio
esperado.
\end{ejercicio}

\begin{solucion}
    El grafo de decisiones corresponde con el de la figura \ref{fig:rel2_ej8}. Las cantidades están el millones de euros.

\begin{figure}
    \centering
    \resizebox{\textwidth}{!}{
    \begin{tikzpicture}[
        >={Stealth[round]},
        node distance=3cm,
        auto,
        shorten >=1pt,
        font=\sffamily\small
    ]

    % Nodo inicial
    \node[block] (0) {(1)};

    % Nodos secundarios para aceptar o no vender Micra
    \node[state, above right=3cm and 3cm of 0] (1a) {(2)};
    \node[none, below right=3cm and 3cm of 0] (1b) {($\triangle_{11} = 0$)};

    % Relaciones desde el nodo inicial
    \draw[->, thick, bend left=10] (0) to node[above] {Aceptar vender Micra} (1a);
    \draw[->, thick, bend right=10] (0) to node[below] {No aceptar} (1b);

    % Nodos terciarios para aceptar vender Micra
    \node[none, above right=2cm and 2cm of 1a] (1a1) {$\triangle_{1} = -0.6$};
    \node[block, below right=2cm and 2cm of 1a] (1a2) {(3)};

    % Relaciones terciarias para aceptar vender Micra
    \draw[->, dashed, bend left=5] (1a) to node[above] {No vende 25\%} (1a1);
    \draw[->, dashed, bend right=5] (1a) to node[below] {Vende 75\%} (1a2);

    % Nodos cuaternarios para decidir finalizar o continuar
    \node[none, above right=2cm and 2cm of 1a2] (1a2a) {$\triangle_{2} = 300 = 900 - 600$};
    \node[block, below right=2cm and 2cm of 1a2] (1a2b) {(4)};

    % Relaciones cuaternarias para decidir finalizar o continuar
    \draw[->, dashed, bend left=5] (1a2) to node[above] {Finalizar trato} (1a2a);
    \draw[->, dashed, bend right=5] (1a2) to node[below] {Continuar trato} (1a2b);

    % Nodos quíntuples para optar por Passat o Focus
    \node[state, above right=2cm and 2cm of 1a2b] (1a2b1) {(5)};
    \node[state, below right=6cm and 2cm of 1a2b] (1a2b2) {(6)};

    % Relaciones quíntuples para optar por Passat o Focus
    \draw[->, dashed, bend left=5] (1a2b) to node[above] {Optar por Passat} (1a2b1);
    \draw[->, dashed, bend right=5] (1a2b) to node[below] {Optar por Focus} (1a2b2);

    % Nodos séxtuples para decidir si vende el Passat o no
    \node[none, above right=2cm and 2cm of 1a2b1] (1a2b1a) {$\triangle_{3} = 900 - 600 - 600 = -300$};
    \node[block, below right=2cm and 2cm of 1a2b1] (1a2b1b) {(7)};

    % Relaciones séxtuples para decidir si vende el Passat o no
    \draw[->, dashed, bend left=5] (1a2b1) to node[above] {No vende Passat 50\%} (1a2b1a);
    \draw[->, dashed, bend right=5] (1a2b1) to node[below] {Vende Passat 50\%} (1a2b1b);

    % Nodos séxtuples para decidir si vende el Focus o no
    \node[none, above right=2cm and 2cm of 1a2b1b] (1a2b1b1) {$\triangle_{4} = 300 + (3000-600) = 2700$};
    \node[state, below right=2cm and 2cm of 1a2b1b] (1a2b1b2) {(9)};

    % Relaciones séxtuples para decidir si vende el Focus o no
    \draw[->, dashed, bend left=5] (1a2b1b) to node[above] {Fin} (1a2b1b1);
    \draw[->, dashed, bend right=5] (1a2b1b) to node[below] {Vende Focus} (1a2b1b2);

    % Nodos séxtuples para decidir si vende el Focus o no
    \node[none, above right=2cm and 2cm of 1a2b1b2] (1a2b1b2a) {$\triangle_{5} = 1300 + 2700 = 4000$};
    \node[none, below right=2cm and 2cm of 1a2b1b2] (1a2b1b2b) {$\triangle_{6} = 2500$};

    % Relaciones séxtuples para decidir si vende el Focus o no
    \draw[->, dashed, bend left=5] (1a2b1b2) to node[above] {Vende Focus 2/3} (1a2b1b2a);
    \draw[->, dashed, bend right=5] (1a2b1b2) to node[below] {No vende Focus 1/3} (1a2b1b2b);
    
    % Nodos séxtuples para decidir si vende el Focus o no desde el nodo 6
    \node[none, above right=2cm and 2cm of 1a2b2] (1a2b2a) {$\triangle_{7} = 100 = 900 - 600 - 200$};
    \node[block, below right=2cm and 2cm of 1a2b2] (1a2b2b) {(8)};

    % Relaciones séxtuples para decidir si vende el Focus o no desde el nodo 6
    \draw[->, dashed, bend left=5] (1a2b2) to node[above] {Vende Focus 33\%} (1a2b2a);
    \draw[->, dashed, bend right=5] (1a2b2) to node[below] {No vende Focus 67\%} (1a2b2b);

    % Nodos séxtuples para decidir si intenta con el Passat o no desde el nodo 8
    \node[none, above right=2cm and 2cm of 1a2b2b] (1a2b2b1) {$\triangle_{8} = 300 - 600  = -300$};
    \node[state, below right=2cm and 2cm of 1a2b2b] (1a2b2b2) {(10)};

    % Relaciones séxtuples para decidir si intenta con el Passat o no desde el nodo 8
    \draw[->, dashed, bend left=5] (1a2b2b) to node[above] {Fin} (1a2b2b1);
    \draw[->, dashed, bend right=5] (1a2b2b) to node[below] {Intentar con Passat} (1a2b2b2);

    % Nodos séptuples para decidir si vende el Passat o no desde el nodo 10
    \node[none, above right=2cm and 2cm of 1a2b2b2] (1a2b2b2a) {$\triangle_{9} = 4000$};
    \node[none, below right=2cm and 2cm of 1a2b2b2] (1a2b2b2b) {$\triangle_{10} = 1000$};

    % Relaciones séptuples para decidir si vende el Passat o no desde el nodo 10
    \draw[->, dashed, bend left=5] (1a2b2b2) to node[above] {Vende Passat 50\%} (1a2b2b2a);
    \draw[->, dashed, bend right=5] (1a2b2b2) to node[below] {No vende Passat 50\%} (1a2b2b2b);




    


    \end{tikzpicture}
    }
    \captionof{figure}{Diagrama de Transición del AFD del Ejercicio 8.}
    \label{fig:rel2_ej8}
\end{figure}


\begin{align*}
    VME_{9} = & \frac{2}{3} \cdot 4000 + \frac{1}{3} \cdot 2500 = 3500 \\
    VME_{7} = & max(-300, 3500) = 3500 \\
    VME_{5} = & 0.5 \cdot -300 + 0.5 \cdot 3500 = 1600 \\
    VME_{10} = & 0.5 \cdot 4000 + 0.5 \cdot 1000 = 2500 \\
    VME_{8} = & max(-300, 2500) = 2500 \\
    VME_{6} = & \frac{1}{3} \cdot 100 + \frac{2}{3} \cdot 2500 = 1700 \\
    VME_{4} = & max(1600, 1700) = 1700 \\
    VME_{3} = & max(300, 1700) = 1700 \\
    VME_{2} = & 0.25 \cdot -600 + 0.75 \cdot 1700 = 1125 \\
    VME_{1} = & max(1125, 0) = 1125 \\
\end{align*}

\end{solucion}
