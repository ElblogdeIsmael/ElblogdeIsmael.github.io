% ========================
% estilo.latex mínimo funcional
% ========================

\documentclass[12pt]{book} % report para capítulos

% ========================
% Paquetes y comandos extra
% ========================
% ===========================
% Paquetes básicos de idioma y codificación
% ===========================
\usepackage[utf8]{inputenc}   % Codificación UTF-8
\usepackage[T1]{fontenc}      % Acentos y caracteres correctos
\usepackage[spanish]{babel}   % Traducción al español (capítulos, índices, etc.)
\usepackage{csquotes}         % Citas tipográficas correctas

% ===========================
% Tipografía
% ===========================
\usepackage{lmodern}          % Fuente Latin Modern
\usepackage{microtype}        % Mejoras tipográficas (espaciado, justificación)

% ===========================
% Márgenes y geometría
% ===========================
\usepackage{geometry}         % Control de márgenes
\geometry{a4paper, top=3cm, bottom=3cm, left=3cm, right=3cm}

% ===========================
% Matemáticas
% ===========================
\usepackage{amsmath, amssymb, amsthm} % Paquetes AMS
\usepackage{mathtools}        % Extiende amsmath
\usepackage{physics}          % Notación física y matemática (derivadas, bra-ket, etc.)
\usepackage{siunitx}          % Unidades SI (e.g. \SI{3}{m/s})
% \sisetup{locale=ES}           % Configuración para español (coma decimal, etc.)
\AtBeginDocument{\RenewCommandCopy\qty\SI} % Resolve siunitx and physics conflict


% ===========================
% Gráficos, tablas y colores
% ===========================
\usepackage{graphicx}         % Insertar imágenes
\usepackage{xcolor}           % Colores personalizados
\usepackage{tikz}             % Dibujos vectoriales
\usetikzlibrary{calc,positioning,shapes,arrows} % Librerías útiles de TikZ
\usepackage{pgfplots}         % Gráficas de funciones
\pgfplotsset{compat=1.18}
\usepackage{float}            % Control de posición de figuras/tablas
\usepackage{booktabs}         % Tablas profesionales
\usepackage{multirow}         % Celdas que ocupan varias filas
\usepackage{array}            % Más control en tablas
\usepackage{colortbl}         % Tablas con colores
\usepackage{inconsolata}


% ===========================
% Listas y enumeraciones
% ===========================
\usepackage{enumitem}         % Control de listas enumeradas y viñetas

% ===========================
% Encabezados, pies y diseño
% ===========================
\usepackage{fancyhdr}         % Encabezados y pies de página
\usepackage{titlesec}         % Personalizar títulos de capítulos/secciones
\usepackage{setspace}         % Espaciado entre líneas
\usepackage{parskip}          % Control del espacio entre párrafos

% ===========================
% Referencias, hipervínculos y citas
% ===========================
\usepackage{hyperref}         % Hipervínculos en PDF
\hypersetup{
    colorlinks = true,
    linkcolor  = red!70,
    citecolor  = red!70,
    urlcolor   = red!70,
    pdfpagelayout = SinglePage, % Asegura que el contenido se ajuste a una sola página
    pdfstartview = Fit          % Ajusta el contenido al tamaño de la página
}
\usepackage{cleveref}         % Referencias inteligentes (\cref)

% ===========================
% Código fuente
% ===========================
\usepackage{listings}         % Mostrar código con estilo
\usepackage{minted}           % (mejor opción, requiere Python y pygments)

% ===========================
% Bibliografía
% ===========================
\usepackage[backend=biber,style=apa]{biblatex} % Ejemplo: estilo APA
\addbibresource{referencias.bib}              % Archivo .bib

% ===========================
% Otros útiles
% ===========================
\usepackage{pdfpages}         % Insertar PDFs externos
\usepackage{blindtext}        % Texto de prueba
\usepackage{caption}          % Personalizar pies de figura/tabla
\usepackage{subcaption}       % Subfiguras
\usepackage{tocloft} 
\usepackage{amsthm}
\usepackage{subcaption}
\usepackage{truncate} % permite truncar texto si no cabe
\usepackage{libertinus}  % reemplaza lmodern
\usepackage{booktabs}  % para \toprule, \midrule, \bottomrule
\usepackage{array}     % para definir columnas personalizadas
\usepackage{colortbl}  % colores en tablas
\usepackage{etoolbox}
\AtBeginEnvironment{tabular}{\rowcolors{2}{gray!10}{white}\renewcommand{\arraystretch}{1.2}}

% ===========================
% Opciones de fuentes sugeridas
% ===========================
% TeX Gyre Pagella (estilo Palatino)
% \usepackage{fontspec}
% \usepackage{unicode-math}
% \setmainfont{TeX Gyre Pagella}
% \setmathfont{TeX Gyre Pagella Math}

% TeX Gyre Termes (estilo Times)
% \setmainfont{TeX Gyre Termes}
% \setmathfont{TeX Gyre Termes Math}

% Libertinus (elegante y completa)
% \setmainfont{Libertinus Serif}
% \setmathfont{Libertinus Math}

% TeX Gyre Bonum (estilo Garamond)
% \setmainfont{TeX Gyre Bonum}
% \setmathfont{TeX Gyre Bonum Math}

% Latin Modern (moderno de Computer Modern)
% \setmainfont{Latin Modern Roman}
% \setmathfont{Latin Modern Math}


% \usepackage{helvet}
% \usepackage{libertine}
% \usepackage[sfdefault]{FiraSans}

\usepackage{tcolorbox} % para cajas de colores




  % si tienes paquetes personalizados
% aquí van los comandos personalizados
% Comando para incluir imágenes
\newcommand{\incluirimagen}[3][]{%
\begin{figure}[H]
    \centering
    \includegraphics[width=\linewidth,#1]{#2}
    \caption{#3}
    \label{fig:#2}
\end{figure}
}

% comando para ejercicios con fondo
\newtheoremstyle{ejerciciostyle}
  {10pt}   % Espacio arriba
  {10pt}   % Espacio abajo
  %{\itshape} % Fuente del cuerpo
  {}
  {}       % Sangría
  {\bfseries} % Fuente del encabezado
  {}      % Puntuación tras encabezado
  { }      % Espacio tras encabezado
  {\thmname{#1} \thmnumber{#2}. \thmnote{#3}}


% % comando formal para enunciado de ejercicios
% \theoremstyle{ejerciciostyle}
% \newtheorem{ejercicio}{Ejercicio}[chapter]

\theoremstyle{ejerciciostyle}
\newtheorem{ejercicio}{Ejercicio}[section]

\renewcommand{\theejercicio}{\thechapter.\arabic{section}.\arabic{ejercicio}}


% comando formal para soluciones
\theoremstyle{remark}
\newtheorem{solucion}{Solución}[ejercicio]

\renewcommand{\thesolucion}{\thechapter.\arabic{section}.\arabic{ejercicio}}

% Comando para dos imágenes en paralelo
\newcommand{\dosimagenes}[6]{%
    \begin{figure}[h!]
        \centering
        \begin{minipage}{0.48\linewidth}
            \centering
            \includegraphics[width=\linewidth]{#1}
            \caption{#2}
            \label{#5}
        \end{minipage}\hfill
        \begin{minipage}{0.48\linewidth}
            \centering
            \includegraphics[width=\linewidth]{#3}
            \caption{#4}
            \label{#6}
        \end{minipage}
    \end{figure}
}

% \dosimagenes{media/fondo.jpg}{Descripción 1}{media/fondo.jpg}{Descripción 2}{fig:descripcion1}{fig:descripcion2}

% \ref{fig:descripcion1} es la mejor
% \ref{fig:descripcion2} es la mejor

\newcommand{\portadaimg}{\VAR{portadaimg}}

% Comando para crear una nota estilo información
% \newcommand{\nota}[2]{%
% \begin{tcolorbox}[colframe=blue!75!black, colback=blue!5!white, title=\textbf{#1}]
%     #2
% \end{tcolorbox}
% }
\newtheorem{nota}{Nota}[chapter]


% Comando para poner dos códigos en paralelo
\newcommand{\doscodigos}[4]{%
  \noindent
  \begin{minipage}{0.48\linewidth}
    \lstset{language=#1}
    \lstinputlisting{#2}
  \end{minipage}\hfill
  \begin{minipage}{0.48\linewidth}
    \lstset{language=#3}
    \lstinputlisting{#4}
  \end{minipage}
}

% Comando para poner un solo código
\newcommand{\uncodigo}[2]{%
  \begin{lstlisting}[language=#1]
#2
  \end{lstlisting}
}


% % Listas de archivos (sin guiones en los nombres de macros)
% \newcommand{\listagdfilesSesion2Mallas2D}{cargatexturas.gd, envioinmediato.gd, malla2dcontexturas.gd, mallaconcoloresdevertices.gd, mallanoindentada.gd}
% \newcommand{\listagdfilesSesion2Mallas3D}{mallaindexada3d.gd, materialconcolordeplano.gd, materialconcoloresdevertices.gd, tablas.gd}

% % Macro que recorre una lista de archivos en un subdirectorio
% \newcommand{\includegdfiles}[2]{%
%   % #1 = subdirectorio
%   % #2 = nombre de la lista de archivos
%   \foreach \filename in #2 {%
%     \includecode[gdstyle]{code/#1/\filename}{\filename}
%   }%
% }



% Comando para ejercicio resuelto
\newtheoremstyle{ejercicioresueltostyle}
    {10pt}   % Espacio arriba
    {10pt}   % Espacio abajo
    {\itshape} % Fuente del cuerpo
    {}       % Sangría
    {\bfseries} % Fuente del encabezado
    {}      % Puntuación tras encabezado
    { }      % Espacio tras encabezado
    {\thmname{#1} \thmnumber{#2}. \thmnote{#3}}

\theoremstyle{ejercicioresueltostyle}
\newtheorem{ejercicioresuelto}{Ejercicio Resuelto}[section]

\renewcommand{\theejercicioresuelto}{\thechapter.\arabic{section}.\arabic{ejercicioresuelto}}


%======================================================================== 
% PRACTICAS
%========================================================================

% Comando para definir un tema
\newcommand{\tema}[1]{%
  \section{#1}
  \addcontentsline{toc}{section}{#1}
}
\usepackage{tikz}
\usepackage{graphicx} % necesario para \resizebox
\usepackage{etoolbox}

% ======== NODOS ========
\newcommand{\nodo}[4][]{\node[state, #1] (#2) at (#3) {$#4$};}
% Uso: \nodo[initial,accepting]{q0}{0,0}{q_0}

% ======== FLECHAS ========
\newcommand{\flecha}[4][]{\draw[->, #1] (#2) -- (#3) node[midway, above] {#4};}
% Uso: \flecha{q0}{q1}{0} o \flecha[bend left]{q1}{q2}{1}

\newcommand{\flechaabajo}[4][]{\draw[->, #1] (#2) -- (#3) node[midway, below, yshift=-6pt] {#4};}
% Igual que \flecha pero con etiqueta abajo
\newcommand{\flechaarriba}[4][]{\draw[->, #1] (#2) -- (#3) node[midway, above, yshift=6pt] {#4};}
% Igual que \flecha pero con etiqueta arriba
\newcommand{\flechaderecha}[4][]{\draw[->, #1] (#2) -- (#3) node[midway, right] {#4};}
% Igual que \flecha pero con etiqueta a la derecha
\newcommand{\flechaiquierda}[4][]{\draw[->, #1] (#2) -- (#3) node[midway, left] {#4};}
% Igual que \flecha pero con etiqueta a la izquierda

\newcommand{\curva}[5][]{\draw[->, bend #1] (#2) to node[midway, #5] {#4} (#3);}
% Uso: \curva[left]{q1}{q2}{1}{below}


\newcommand{\loopa}[3]{\draw[->] (#1) edge[loop above] node {#2} (#1);}
\newcommand{\loopb}[3]{\draw[->] (#1) edge[loop below] node {#2} (#1);}
\newcommand{\loopr}[3]{\draw[->] (#1) edge[loop right] node {#2} (#1);}
\newcommand{\loopl}[3]{\draw[->] (#1) edge[loop left] node {#2} (#1);}
% Uso: \loopa{q1}{0}

% ======== ESTILOS ESPECIALES ========
\tikzset{
    error/.style={state, fill=red!20, draw=red!80!black},
    final/.style={state, accepting, fill=green!15!white, draw=green!60!black}
}
% Uso: \nodo[error]{qe}{5,0}{q_e}  o \nodo[final]{qf}{7,0}{q_f}


\newcommand{\pa}{1}      % ejemplo de valor
\newcommand{\pUno}{2}
\newcommand{\pDos}{3}
  % comandos LaTeX propios
% ===========================
% Diseño general
% ===========================
\setstretch{1.15} % interlineado
\setlength{\parskip}{0.5em} % espacio entre párrafos
\setlength{\parindent}{0pt} % sin sangría

% ===========================
% Estilo de capítulos y secciones (titlesec)
% ===========================
\titleformat{\chapter}[display]
  {\bfseries\Huge}
  {\filleft\Large\scshape Capítulo \thechapter}
  {1ex}
  {\titlerule[1pt]\vspace{1ex}\filright}
  [\vspace{1ex}\titlerule]

\titlespacing*{\chapter}{0pt}{0pt}{2em}

\titleformat{\section}
  {\Large\bfseries}
  {\thesection}{1em}{}

\titleformat{\subsection}
  {\large\bfseries}
  {\thesubsection}{1em}{}

\titleformat{\subsubsection}
  {\normalsize\bfseries\itshape}
  {\thesubsubsection}{1em}{}

% ===========================
% Encabezados y pies de página (fancyhdr)
% ===========================
\pagestyle{fancy}
\fancyhf{} % limpia
\fancyhead[L]{\small\scshape\nouppercase{\leftmark}} % sección/capítulo en mayúsculas pequeñas
\fancyhead[R]{\small\thepage}                        % número de página
%\fancyfoot[C]{\scriptsize\itshape Apuntes de la carrera} % texto fijo abajo en cursiva
% Encabezados y pies de página personalizados
% \fancyfoot[L]{\scriptsize\itshape Nombre de la asignatura} % pie de página izquierdo en cursiva
\fancyfoot[R]{\normalsize Ismael Sallami Moreno}        % pie de página derecho con el nombre del autor

% Línea bajo el encabezado
\renewcommand{\headrulewidth}{0.5pt} % línea más gruesa en el encabezado
% Línea en el pie
\renewcommand{\footrulewidth}{0.4pt} % línea fina en el pie
\renewcommand{\sectionmark}[1]{%
  \markboth{\thesection\quad #1}{}%
}

% ===========================
% Numeración de elementos
% ===========================
\numberwithin{equation}{chapter} % ecuaciones numeradas por capítulo
\numberwithin{figure}{chapter}   % figuras numeradas por capítulo
\numberwithin{table}{chapter}    % tablas numeradas por capítulo

% ===========================
% Listas y enumeraciones
% ===========================
\setlist[itemize]{label=--, left=1.5em}
\setlist[enumerate]{label=\arabic*), left=1.5em}

% ===========================
% Estilo de citas y bibliografía
% ===========================
\DefineBibliographyStrings{spanish}{%
  references = {Bibliografía},
}

% ===========================
% Entornos personalizados
% ===========================
\newtheoremstyle{cajita} % nombre del estilo
  {1em}   % espacio arriba
  {1em}   % espacio abajo
  {}      % fuente del cuerpo
  {}      % indentación
  {\bfseries} % fuente del título
  {.}     % puntuación tras título
  {0.5em} % espacio tras título
  {\thmname{#1}\thmnumber{ #2} \thmnote{(#3)}} % formato


\theoremstyle{cajita}
\newtheorem{teorema}{Teorema}[chapter]
\newtheorem{definicion}{Definición}[chapter]
\newtheorem{ejemplo}{Ejemplo}[chapter]
\newtheorem{proposicion}{Proposición}[chapter]
\newtheorem{demostracion}{Demostración}[chapter]
\newtheorem{corolario}{Corolario}[chapter]
\newtheorem{propuesta}{Propuesta}[chapter]


\newtheoremstyle{anotacionstyle} % nombre del estilo
  {1em}   % espacio arriba
  {1em}   % espacio abajo
  {}      % fuente del cuerpo (sin cursiva)
  {}      % indentación
  {\itshape} % fuente del título (Nota en cursiva)
  {.}     % puntuación tras título
  {0.5em} % espacio tras título
  {\thmname{\itshape#1}\thmnumber{ #2} \thmnote{(#3)}} % formato (solo Nota en cursiva)

\theoremstyle{anotacionstyle}
\newtheorem{anotacion}{Nota}[chapter]

% ===========================
% Configuración de lstlisting
% ===========================

% ===============================================
% ESTILO 1: MODERNO Y MINIMALISTA
% ===============================================

% Definir colores personalizados
\definecolor{codegreen}{rgb}{0,0.6,0}
\definecolor{codegray}{rgb}{0.5,0.5,0.5}
\definecolor{codepurple}{rgb}{0.58,0,0.82}
\definecolor{backcolour}{rgb}{0.95,0.95,0.92}
\definecolor{framecolor}{rgb}{0.8,0.8,0.8}

\lstset{
  backgroundcolor=\color{backcolour},   
  commentstyle=\color{codegreen},
  keywordstyle=\color{magenta},
  numberstyle=\tiny\color{codegray},
  stringstyle=\color{codepurple},
  basicstyle=\ttfamily\footnotesize,
  breakatwhitespace=false,         
  breaklines=true,                 
  captionpos=b,                    
  keepspaces=true,                 
  numbers=left,                    
  numbersep=5pt,                  
  showspaces=false,                
  showstringspaces=false,
  showtabs=false,                  
  tabsize=2,
  frame=shadowbox,
  frameround=tttt,
  rulecolor=\color{framecolor},
  rulesepcolor=\color{framecolor},
  xleftmargin=20pt,
  xrightmargin=20pt,
  aboveskip=20pt,
  belowskip=20pt,
  inputencoding=utf8,
  extendedchars=true,
  literate=
    {←}{{$\leftarrow$}}1
    {→}{{$\rightarrow$}}1
    {↑}{{$\uparrow$}}1
    {↓}{{$\downarrow$}}1
    {↔}{{$\leftrightarrow$}}1
    {⇒}{{$\Rightarrow$}}1
    {⇐}{{$\Leftarrow$}}1
    {⇔}{{$\Leftrightarrow$}}1
    {α}{{$\alpha$}}1
    {β}{{$\beta$}}1
    {γ}{{$\gamma$}}1
    {δ}{{$\delta$}}1
    {ε}{{$\epsilon$}}1
    {θ}{{$\theta$}}1
    {λ}{{$\lambda$}}1
    {μ}{{$\mu$}}1
    {π}{{$\pi$}}1
    {σ}{{$\sigma$}}1
    {φ}{{$\phi$}}1
    {ψ}{{$\psi$}}1
    {ω}{{$\omega$}}1
    {Δ}{{$\Delta$}}1
    {Θ}{{$\Theta$}}1
    {Λ}{{$\Lambda$}}1
    {Π}{{$\Pi$}}1
    {Σ}{{$\Sigma$}}1
    {Φ}{{$\Phi$}}1
    {Ψ}{{$\Psi$}}1
    {Ω}{{$\Omega$}}1
    {á}{{\'a}}1
    {é}{{\'e}}1
    {í}{{\'i}}1
    {ó}{{\'o}}1
    {ú}{{\'u}}1
    {Á}{{\'A}}1
    {É}{{\'E}}1
    {Í}{{\'I}}1
    {Ó}{{\'O}}1
    {Ú}{{\'U}}1
    {ä}{{\"a}}1
    {ë}{{\"e}}1
    {ï}{{\"i}}1
    {ö}{{\"o}}1
    {ü}{{\"u}}1
    {Ä}{{\"A}}1
    {Ë}{{\"E}}1
    {Ï}{{\"I}}1
    {Ö}{{\"O}}1
    {Ü}{{\"U}}1
    {ñ}{{\~n}}1
    {Ñ}{{\~N}}1
    {ç}{{\c{c}}}1
    {Ç}{{\c{C}}}1
    {¿}{{?`}}1
    {¡}{{!`}}1
    {à}{{\`a}}1
    {è}{{\`e}}1
    {ì}{{\`i}}1
    {ò}{{\`o}}1
    {ù}{{\`u}}1
    {À}{{\`A}}1
    {È}{{\`E}}1
    {Ì}{{\`I}}1
    {Ò}{{\`O}}1
    {Ù}{{\`U}}1
    {-}{{-}}1
    {=}{{=\allowbreak}}1  % <--- ESTA LÍNEA ES EL TRUCO PARA CORTAR LOS '===='
    % {#}{{\#}}1 
}


% ===============================================
% ESTILO 2: ELEGANTE CON BORDES REDONDEADOS
% ===============================================

% Colores para estilo elegante
\definecolor{lightblue}{rgb}{0.93,0.95,1}
\definecolor{darkblue}{rgb}{0.1,0.2,0.5}
\definecolor{mediumblue}{rgb}{0.2,0.4,0.8}
\definecolor{darkgreen}{rgb}{0,0.5,0}
\definecolor{darkred}{rgb}{0.6,0,0}

\lstdefinestyle{elegant}{
    backgroundcolor=\color{lightblue},
    commentstyle=\color{darkgreen}\itshape,
    keywordstyle=\color{darkblue}\bfseries,
    numberstyle=\tiny\color{gray},
    stringstyle=\color{darkred},
    basicstyle=\ttfamily\small,
    breakatwhitespace=false,
    breaklines=true,
    captionpos=t,
    keepspaces=true,
    numbers=left,
    numbersep=8pt,
    showspaces=false,
    showstringspaces=false,
    showtabs=false,
    tabsize=4,
    frame=single,
    frameround=tttt,
    framesep=10pt,
    xleftmargin=15pt,
    xrightmargin=15pt,
    aboveskip=15pt,
    belowskip=15pt,
    columns=flexible
}

% ===============================================
% ESTILO 3: PROFESIONAL CORPORATIVO
% ===============================================

% Colores corporativos
\definecolor{corporatebg}{rgb}{0.98,0.98,0.98}
\definecolor{corporateblue}{rgb}{0.07,0.29,0.49}
\definecolor{corporategray}{rgb}{0.4,0.4,0.4}
\definecolor{corporategreen}{rgb}{0.13,0.55,0.13}
\definecolor{corporatered}{rgb}{0.8,0.2,0.2}

\lstdefinestyle{corporate}{
    backgroundcolor=\color{corporatebg},
    commentstyle=\color{corporategreen}\slshape,
    keywordstyle=\color{corporateblue}\bfseries,
    numberstyle=\scriptsize\color{corporategray},
    stringstyle=\color{corporatered},
    basicstyle=\ttfamily\footnotesize,
    breakatwhitespace=false,
    breaklines=true,
    captionpos=b,
    keepspaces=true,
    numbers=left,
    numbersep=12pt,
    showspaces=false,
    showstringspaces=false,
    showtabs=false,
    tabsize=3,
    frame=leftline,
    framerule=3pt,
    rulecolor=\color{corporateblue},
    xleftmargin=25pt,
    aboveskip=20pt,
    belowskip=20pt,
    lineskip=1pt
}

% ===============================================
% ESTILO 4: MODERNO CON SOMBRAS
% ===============================================

% Colores modernos
\definecolor{modernbg}{rgb}{0.97,0.97,0.97}
\definecolor{moderngray}{rgb}{0.3,0.3,0.3}
\definecolor{modernpurple}{rgb}{0.5,0.2,0.8}
\definecolor{modernteal}{rgb}{0,0.5,0.5}
\definecolor{modernorange}{rgb}{0.8,0.4,0}

\lstdefinestyle{modern}{
    backgroundcolor=\color{modernbg},
    commentstyle=\color{modernteal}\itshape,
    keywordstyle=\color{modernpurple}\bfseries,
    numberstyle=\tiny\color{moderngray},
    stringstyle=\color{modernorange},
    basicstyle=\ttfamily\small,
    breakatwhitespace=false,
    breaklines=true,
    captionpos=t,
    keepspaces=true,
    numbers=left,
    numbersep=10pt,
    showspaces=false,
    showstringspaces=false,
    showtabs=false,
    tabsize=4,
    frame=tb,
    framerule=2pt,
    rulecolor=\color{modernpurple},
    xleftmargin=20pt,
    xrightmargin=20pt,
    aboveskip=25pt,
    belowskip=25pt
}

% ===============================================
% CONFIGURACIÓN PARA DIFERENTES LENGUAJES
% ===============================================

% Python
\lstdefinestyle{python}{
    language=Python,
    style=elegant,
    morekeywords={True,False,None,self,cls,def,class,import,from,as,with,yield,async,await},
    morecomment=[l]{\#},
    morestring=[b]',
    morestring=[b]"
}

% Java
\lstdefinestyle{java}{
    language=Java,
    style=corporate,
    morekeywords={var,record,sealed,permits,non-sealed}
}

% C++
\lstdefinestyle{cpp}{
    language=C++,
    style=modern,
    morekeywords={constexpr,nullptr,auto,decltype,override,final}
}

% JavaScript
\lstdefinestyle{javascript}{
    language=Java,
    style=elegant,
    morekeywords={let,const,var,function,class,extends,import,export,default,async,await,yield},
    morecomment=[l]{//},
    morecomment=[s]{/*}{*/},
    morestring=[b]',
    morestring=[b]",
    morestring=[b]`
}

% ===============================================
% EJEMPLOS DE USO
% ===============================================

% Para usar el estilo por defecto:
% \begin{lstlisting}
% código aquí
% \end{lstlisting}

% Para usar un estilo específico:
% \begin{lstlisting}[style=elegant]
% código aquí
% \end{lstlisting}

% Para incluir un archivo con estilo específico:
% \lstinputlisting[style=python]{archivo.py}

% Para código inline:
% \lstinline[style=modern]{código inline}

% ===============================================
% CONFIGURACIÓN ADICIONAL PARA TÍTULOS Y CARACTERES
% ===============================================

% Personalizar el formato de los títulos de los listados
\renewcommand\lstlistingname{Código}
\renewcommand\lstlistlistingname{Lista de Códigos}

% Configurar el formato del título con soporte para tildes
\lstset{
    %title=\lstname,
    captionpos=t,
    abovecaptionskip=10pt,
    belowcaptionskip=5pt,
    % Configuración global para caracteres especiales
    inputencoding=utf8,
    extendedchars=true
}

% ===============================================
% COMANDOS PERSONALIZADOS ÚTILES
% ===============================================

% Comando para código inline con soporte automático de tildes
\newcommand{\codeinline}[2][modern]{\lstinline[style=#1,inputencoding=utf8,extendedchars=true]{#2}}

% Comando para bloques de código con título personalizado
\newcommand{\codeblock}[3][elegant]{%
    \begin{lstlisting}[style=#1,caption={#2},label={lst:#2},inputencoding=utf8,extendedchars=true]
    #3
    \end{lstlisting}
}

% Comando para incluir archivos con configuración automática
\newcommand{\includecode}[3][python]{%
    \lstinputlisting[style=#1,caption={#3},label={lst:#3},inputencoding=utf8,extendedchars=true]{#2}
}

% ===============================================
% CONFIGURACIONES ESPECIALES PARA IDIOMAS
% ===============================================

% Configuración específica para código en español
\lstdefinestyle{español}{
    style=elegant,
    inputencoding=utf8,
    extendedchars=true,
    % Palabras clave en español para pseudocódigo
    morekeywords={función,procedimiento,inicio,fin,si,entonces,sino,mientras,para,hasta,hacer,repetir,caso,segun,verdadero,falso,entero,real,caracter,cadena,booleano,leer,escribir,imprimir}
}

% Configuración para comentarios multilíngües
\lstset{
    morecomment=[l]{//\ },
    morecomment=[l]{\#\ },
    morecomment=[s]{/*}{*/},
    morecomment=[s]{}
}

% ===============================================
% CONFIGURACIÓN PARA DIFERENTES LENGUAJES
% ===============================================

% Python
\lstdefinestyle{style1}{
    language=Python,
    style=elegant,
    morekeywords={True,False,None,self,cls,def,class,import,from,as,with,yield,async,await},
    morecomment=[l]{\#},
    morestring=[b]',
    morestring=[b]",
    % Soporte para caracteres especiales
    inputencoding=utf8,
    extendedchars=true
}

% Java
\lstdefinestyle{style2}{
    language=Java,
    style=corporate,
    morekeywords={var,record,sealed,permits,non-sealed},
    % Soporte para caracteres especiales
    inputencoding=utf8,
    extendedchars=true
}

% C++
\lstdefinestyle{style3}{
    language=C++,
    style=modern,
    morekeywords={constexpr,nullptr,auto,decltype,override,final},
    % Soporte para caracteres especiales
    inputencoding=utf8,
    extendedchars=true
}

\lstdefinelanguage{GDScript}{
  keywords={func, var, extends, class_name, if, else, for, while, return, match, in, and, or, not, break, continue, pass},
  sensitive=true,
  morecomment=[l]{\#},
  morestring=[b]",
  morestring=[b]',
}

\lstdefinestyle{gdstyle}{
  language=GDScript,
  basicstyle=\ttfamily\small,
  keywordstyle=\color{blue}\bfseries,
  commentstyle=\color{gray},
  stringstyle=\color{red!60!black},
  numbers=left,
  numberstyle=\tiny\color{gray},
  breaklines=true,
  frame=single,
  tabsize=2,
}


% ===========================
% Estilo global de tablas
% ===========================

\usepackage{booktabs}   % reglas profesionales
\usepackage{colortbl}   % color en filas
\usepackage{xcolor}     % colores
\usepackage{float}      % [H]

% Color de filas alternadas
% \rowcolors{2}{gray!10}{white}

% % Espacio vertical entre filas
% \renewcommand{\arraystretch}{1.2}

% % Cambiar el tamaño de columna por defecto
% \setlength{\tabcolsep}{8pt}

% % Redefinir tabla para que todas las tablas tengan el estilo
% \let\oldtabular\tabular
% \let\endoldtabular\endtabular
% \renewenvironment{tabular}[1]{%
%   \oldtabular{#1}%
% }{%
%   \endoldtabular
% }

% \usepackage{longtable,booktabs,xcolor}
% \rowcolors{2}{gray!10}{white}   % filas alternadas
% \renewcommand{\arraystretch}{1.2} % espacio vertical entre filas

% % Mostrar siempre el número de la tabla
% \usepackage{caption}
% \captionsetup[table]{labelformat=default, labelsep=colon, textfont=bf}


% ===========================
% Estilos para tikz y figures
% ===========================

\usepackage{caption}
\captionsetup{
    font={it},       % fuente en cursiva
    labelfont={},  % etiqueta ("Figura 1") en negrita
    textfont={it},   % texto del caption en cursiva
    justification=centering,  % centra el texto (opcional)
    font={small},    % tamaño de fuente pequeño
}

\usepackage{tikz}
\usetikzlibrary{positioning}

\tikzset{
  state/.style={
    draw,
    circle,
    minimum size=1cm,
    thick,
    fill=yellow!20
  },
  block/.style={
    rectangle,
    draw,
    fill=blue!10,
    rounded corners,
    text centered,
    minimum height=1cm,
    minimum width=2cm,
    thick
  },
  none/.style={
    draw=none,
    fill=none,
    text centered
  },
  error/.style={
    draw,
    circle,
    minimum size=1cm,
    thick,
    fill=red!30
  },
  initial text={}
}   % estilos de secciones, etc.

% ========================
% Configuración índice y listas
% ========================
\setlength{\cftbeforesecskip}{5pt}
\setlength{\headheight}{14pt}  % un poco más que 13.6pt

\renewcommand{\normalsize}{\fontsize{10}{12}\selectfont}

% Fix para listas de Pandoc
\providecommand{\tightlist}{%
  \setlength{\itemsep}{0pt}\setlength{\parskip}{0pt}}



%===============
% ESPACIOS
%===============

% --- Compactar secciones ---
\titlespacing*{\section}{0pt}{1.2ex plus 0.5ex minus 0.2ex}{0.8ex}
\titlespacing*{\subsection}{0pt}{1ex plus 0.3ex minus 0.2ex}{0.5ex}

% --- Compactar flotantes (figuras/tablas) ---
\setlength{\textfloatsep}{8pt}
\setlength{\intextsep}{6pt}
\setlength{\floatsep}{6pt}

% --- Compactar listas ---
\setlist{nosep}

% --- Espacio entre párrafos ---
\setlength{\parskip}{4pt}



%=======================
% fancy with parameters
%=======================
%\fancyfoot[L]{\scriptsize\itshape Diseño y Desarrollo de SI}
\fancyfoot[L]{\normalsize Diseño y Desarrollo de
SI} % pie de página izquierdo con tamaño normal

\setcounter{tocdepth}{1} % Muestra solo hasta subsecciones en el índice

% ========================
% Inicio del documento
% ========================
\begin{document}

% Cambiar puntos suspensivos en el índice
\renewcommand{\cftsecleader}{\cftdotfill{\cftdotsep}}

% Ajustar formato de secciones y subsecciones en el índice
\renewcommand{\cftsecfont}{\bfseries} % Secciones en negrita
\renewcommand{\cftsecpagefont}{\bfseries} % Números de página en negrita para secciones
\renewcommand{\cftsubsecfont}{\normalfont} % Subsecciones en formato normal
\renewcommand{\cftsubsecpagefont}{\normalfont} % Números de página en formato normal para subsecciones

% Espaciado entre entradas del índice
\setlength{\cftbeforesecskip}{8pt} % Espaciado antes de secciones
\setlength{\cftbeforesubsecskip}{4pt} % Espaciado antes de subsecciones



%% portada.tex
\begin{titlepage}
    \newgeometry{top=2cm,bottom=2cm,left=2.5cm,right=2.5cm} % márgenes personalizados
    
    % Fondo con transparencia
    \begin{tikzpicture}[remember picture,overlay]
        \node[opacity=0.15,inner sep=0pt] at (current page.center)
            {\includegraphics[width=\paperwidth,height=\paperheight]{../../img/fondoPrueba.jpg}};
    \end{tikzpicture}

    % Contenido de la portada
    \begin{center}
        \vspace*{2cm}
        
        {\Huge \bfseries\scshape Título del Libro de Apuntes \par}
        \vspace{0.5cm}
        {\Large \itshape Subtítulo o Asignatura \par}
        \vspace{0.5cm}
        {\Large \itshape \href{https://ismael-sallami.github.io}{https://ismael-sallami.github.io} \par}


        \vfill
        
        {\LARGE Autor: \textbf{Tu Nombre Completo} \par}
        \vspace{0.3cm}
        % {\Large Universidad Ejemplo \par}
        
        \vspace{1cm}
        \includegraphics[width=0.25\textwidth]{../../img/ugr.png} % opcional: logo
        \vspace{1cm}
        
        {\large \today}
    \end{center}
    
    \restoregeometry
\end{titlepage}



%==========================
% PORTADA: ENTRADA MANUAL
%==========================

% portada.tex
\begin{titlepage}
    \newgeometry{top=2cm,bottom=2cm,left=2.5cm,right=2.5cm} % márgenes personalizados
    
    % Fondo con transparencia
    \begin{tikzpicture}[remember picture,overlay]
        % \node[opacity=0.15,inner sep=0pt] at (current page.center)
        \node[inner sep=0pt] at (current page.center)
            {\includegraphics[width=\paperwidth,height=\paperheight]{../../../extraFiles/img/fondo_info.jpg}};
    \end{tikzpicture}

    % Contenido de la portada
    \begin{center}
        \vspace*{2cm}
        
        {\Huge \bfseries\scshape Diseño y Desarrollo de SI \par}
        \vspace{0.5cm}
        {\Large \itshape Temario \par}
        \vspace{0.5cm}
        % {\small \itshape \href{https://ismael-sallami.github.io}{https://ismael-sallami.github.io} \par}
        % {\small \itshape \href{https://elblogdeismael.github.io}{https://elblogdeismael.github.io} \par}


        \vfill
        
        % {\LARGE Ismael Sallami Moreno \par}

        \begin{flushright}
            {Ismael Sallami Moreno \par}
            {\small \itshape \href{https://elblogdeismael.github.io}{Recursos Ingeniería Informática y Ade} \par}
        \end{flushright}
        \vspace{0.3cm}
        % {\Large Universidad de Granada \par}
        
        % \vspace{1cm}
        % \includegraphics[width=0.25\textwidth]{../../../extraFiles/img/ugr.png} % opcional: logo
        % \vspace{1cm}
        
        % {\large \today}
    \end{center}
    
    \restoregeometry
\end{titlepage}


% ===============================
% licencia.tex
% ===============================
\begin{tikzpicture}[remember picture,overlay]
\node[anchor=south west, xshift=1cm, yshift=1cm] at (current page.south west) {
\begin{minipage}{0.4\textwidth}
\begin{flushleft}
\section*{Licencia}

Este trabajo está bajo una 
\href{https://creativecommons.org/licenses/by-nc-nd/4.0/}{Licencia Creative Commons BY-NC-ND 4.0}.

\bigskip

Permisos: Se permite compartir, copiar y redistribuir el material en cualquier medio o formato.

\bigskip

Condiciones: Es necesario dar crédito adecuado, proporcionar un enlace a la licencia e indicar si se han realizado cambios. No se permite usar el material con fines comerciales ni distribuir material modificado.

\bigskip

\begin{center}
  \href{https://creativecommons.org/licenses/by-nc-nd/4.0/}{\includegraphics[width=0.35\textwidth]{../../../extraFiles/img/by-nc-nd.png}}
\end{center}
\end{flushleft}
\end{minipage}
};
\end{tikzpicture}
  % licencia
\thispagestyle{empty} % quitar número de página en la portada
\clearpage

% --- Índice ---
\tableofcontents
% \listoffigures
\clearpage

%\listoftables
%\clearpage
%\thispagestyle{empty} % quitar número de página en la portada
%\clearpage
%
% Índice de código
%\renewcommand{\lstlistlistingname}{Índice de Código}
%\lstlistoflistings
%\clearpage
%
% Índice de ecuaciones
%\renewcommand{\listtheoremname}{Índice de Ecuaciones}
%\listoftheorems[ignoreall,show={equation}]
%\clearpage

% --- Contenido Markdown generado por Pandoc ---
\part{Teoría}

\hypertarget{introducciuxf3n-a-los-sistemas-de-informaciuxf3n}{%
\chapter{Introducción a los Sistemas de
Información}\label{introducciuxf3n-a-los-sistemas-de-informaciuxf3n}}

\begin{definicion}[Definición de sistema de información]
Entendemos por SI a aquel que mediante una aplicación podemos acceder a ciertos datos tratando variables como volumen de datos, restricciones y demás. En el contexto de la asignatura, estos son aquellos que utilizan los SGBD (Sistemas Gestores de Bases de Datos), este es parte del SI, no es un SI como tal. Un SI puede usar varios o bien manejar bases de datos distribuidas. Diversas SI pueden usar una misma DB. Ejemplo de un SI es el sistema de actas de la UGR.
\end{definicion}

Las redes sociales son ejemplos de SI, es habitual que usen diversos
SGBD, muchos de ellos no basados en el modelo relacional. Por ejemplo,
Instagram, entre otras base de datos, usa PostgreSQL y cassandra
(NoSql).

\begin{definicion}[Sistemas de Gestión de Aprendizaje]
SI para administrar, distribuir y controlar las actividades de formación en cursos de educación. Como es el caso de Moodle. Es open source y puede usar SGBD (Microsoft SQL, ...). Por ejemplo, Prado esta montado sobre esta platafora educativa.
\end{definicion}

Algunos SI tienen que gestionar datos de gran complejidad, que mezclan
números con imágenes, gran volúmen de columnas, como es el caso de la
invesstigación médica de los genes, \ldots{} Por ello, suelen usar
modelos de datos más complejos.

Los Sistemas de Información (SI) son esenciales en las empresas
modernas, facilitando la gestión y análisis de datos. Suelen requerir
conocimientos avanzados de configuración y uso, más que de desarrollo.
Entre los tipos destacados se encuentran:

\begin{itemize}
\tightlist
\item
  \textbf{Sistemas de Inteligencia de Negocio (Business Intelligence):}
  Enfocados en la exploración y visualización de datos, basados en SGBD
  Multidimensionales y técnicas como OLAP para gestionar Almacenes de
  Datos.
\item
  \textbf{Ciencia de Datos y Análisis Avanzado:} Incluyen análisis
  estadístico, Minería de Datos, Big Data e Inteligencia Artificial para
  descubrir patrones y conocimiento en bases de datos.
\item
  \textbf{Sistemas de Información para Empresas:} Herramientas
  específicas que se abordarán más adelante en el tema.
\end{itemize}

Muchos de los sistemas informáticos más complejos que existen son de
Sistemas de Información.

Los Sistemas de Información (SI) se benefician de una amplia gama de
recursos tecnológicos para su desarrollo. Suelen construirse sobre
múltiples capas de abstracción y emplear herramientas basadas en
configuración, como SGAs y ERPs, destacando las fases de análisis y
diseño. El uso de tecnologías web y servicios de computación en la nube
(AWS, Google Cloud, Azure, etc.) es fundamental, junto con arquitecturas
hardware complejas, como clusters en redes sociales y cloud computing.
Además, hacen un uso intensivo de bases de datos distribuidas y
tecnologías para el procesamiento distribuido de datos.

Estos deben de cumplir aspectos legales y éticos relacionados con las
leyes de protección de datos.

\begin{itemize}
\tightlist
\item
  \textbf{En España:} Existe la \href{https://www.aepd.es/es}{Agencia
  Española de Protección de Datos}.
\item
  \textbf{En la UGR:}
  \href{https://secretariageneral.ugr.es/unidades/oficina-proteccion-datos}{Oficina
  de Protección de Datos}.
\item
  \textbf{A nivel Europeo:} El
  \href{https://edpb.europa.eu/edpb_es}{Comité Europeo de Protección de
  Datos}.
\end{itemize}

Toda base de datos con información personal debe de tener a un
responsable. La seguridad de los datos es un aspecto clave, para evitar
ciberataques (Blockchain desarrollo creciente).

\nota{Ideas principales}{

\begin{itemize}
    \item Los SI son sistemas informáticos que gestionan información almacenada en BD, para lo cual usan al menos un SGBD.
    \item En la Sociedad de la Información, estamos rodeados y usamos a diario estos sistemas en prácticamente todos los aspectos de la vida.
    \item Son fundamentales en el ámbito científico, económico y social.
    \item Pueden ser extraordinariamente complejos.
    \item Gran cantidad de recursos, plataformas y tecnologías para su desarrollo.
    \item Existe una gran demanda de perfiles profesionales relacionados con el diseño y desarrollo de SI. Esta demanda no para de crecer.
\end{itemize}

}

\textbf{Organización jerárquica de los sistemas de información
empresarial}

Los sistemas de información empresarial se organizan jerárquicamente en
función de los niveles de decisión y las necesidades de información
dentro de una organización. Estos niveles incluyen:

\begin{itemize}
\item
  \textbf{Sistemas de Procesamiento de Transacciones (TPS):} Son la base
  de la jerarquía y se encargan de gestionar las operaciones diarias de
  la empresa, como ventas, compras, inventarios, etc. Proveen datos
  estructurados y detallados que alimentan los niveles superiores.
\item
  \textbf{Sistemas de Información Gerencial (MIS):} Utilizan los datos
  recopilados por los TPS para generar informes y resúmenes que apoyan
  la toma de decisiones a nivel táctico. Ayudan a los gerentes a evaluar
  el desempeño y planificar actividades.
\item
  \textbf{Sistemas de Soporte a la Decisión (DSS):} Proveen herramientas
  analíticas y modelos para ayudar en la toma de decisiones complejas y
  no estructuradas. Se enfocan en el análisis de datos y la simulación
  de escenarios.
\item
  \textbf{Sistemas de Información Ejecutiva (EIS):} Diseñados para los
  altos directivos, ofrecen una visión general y estratégica de la
  organización. Presentan información clave de manera resumida y visual,
  facilitando la toma de decisiones estratégicas.
\item
  \textbf{Software de Gestión Empresarial (ERP):} Los ERP (Enterprise
  Resource Planning) son sistemas integrados que permiten gestionar y
  automatizar los procesos clave de una organización, como finanzas,
  recursos humanos, producción, logística y ventas. Estos sistemas
  centralizan la información en una única base de datos, facilitando la
  comunicación entre departamentos y mejorando la eficiencia operativa.
  Ejemplos destacados incluyen SAP, Oracle ERP y Microsoft Dynamics.
\end{itemize}

\incluirimagen{media/areaempresa.png}{Áreas funcionales de una empresa}

\begin{definicion}[Sistemas OLTP (Online Transaction Processing)]
Sistemas de procesamiento de transacciones en línea que se centran en la gestión de datos operativos y transaccionales. Son utilizados para registrar, almacenar y procesar transacciones en tiempo real, como ventas, compras o inventarios.
\end{definicion}

\begin{definicion}[Sistemas OLAP (Online Analytical Processing)]
Sistemas de procesamiento analítico en línea diseñados para realizar consultas complejas y análisis multidimensional de grandes volúmenes de datos. Son utilizados en aplicaciones de inteligencia de negocio para explorar y visualizar datos históricos.
\end{definicion}

\begin{definicion}[Sistemas de procesamiento de datos]
Sistemas que procesan información operativa en una empresa mediante la recopilación, manipulación, almacenamiento y preparación de datos. Incluyen actividades como clasificación, ordenación, cálculos, resúmenes y generación de informes.
\end{definicion}

\begin{definicion}[Sistemas de información contable]
Sistemas diseñados para gestionar y procesar la información financiera y contable de una organización. Facilitan el registro de transacciones económicas, la generación de informes financieros y el cumplimiento de normativas contables.
\end{definicion}

\begin{definicion}[Sistemas de información empleados en el funcionamiento cotidiano de una empresa]
Sistemas que procesan la información operativa generada en las actividades diarias de una empresa. Incluyen:
- Recopilación de datos (transacciones).
- Manipulación de datos: clasificación, ordenación, cálculos, resúmenes.
- Almacenamiento de datos (base de datos).
- Preparación de documentos (informes).
- Gestión de contenidos (CMS).
- Business Intelligence: Data Warehousing y Data Mining.
\end{definicion}

\begin{definicion}[Planificador de recursos empresariales (ERP)]
Sistema integrado de gestión que permite integrar los distintos flujos de información de la empresa (finanzas, compras, ventas, recursos humanos...) de forma modular y adaptada al cliente. “Sistema integrado de Software de Gestión Empresarial, compuesto por un conjunto de módulos funcionales, susceptibles de ser adaptados a las necesidades de cada cliente.”\footnote{Sistemas de Información. Herramientas prácticas para la gestión empresarial. 4ª Edición. (Gómez Vieites; Suárez Rey: 2011)} Es el único integrado en toda la compañía, usa base de datos centralizada y actualiza los datos en tiempo real para la compañia.
\end{definicion}

\begin{definicion}[SAP - Systemanalyse und Programmentwicklung (SAP®)]

\begin{itemize}
    \item Es un sistema de información estándar modular.
    \item Puede ser parametrizado para adaptarse a las necesidades específicas de cada compañía.
    \item Proporciona datos disponibles en tiempo real.
    \item Genera pantallas con información resumida para facilitar la toma de decisiones.
\end{itemize}

\textbf{Módulos SAP®}

\begin{itemize}
\item Financial Accounting (FI)
\item Financial Supply Chain Management (FSCM)
\item Controlling (CO)
\item Materials Management (MM)
\item Sales and Distribution (SD)
\item Logistics Execution (LE)
\item Production Planning (PP)
\item Quality Management (QM)
\item Plant Maintenance (PM)
\item Project System (PS)
\item Human Resources (HR)
\end{itemize}

\textbf{Características SAP®}

\begin{itemize}
\item Indicado para grandes volúmenes de datos (grandes compañías).
\item Coste elevado.
\item Desarrollado siguiendo estándares de calidad:
    \begin{itemize}
    \item Los ingenieros de SAP diseñan el producto para que los diversos procesos de negocio se realicen siguiendo las mejores prácticas de la industria.
    \item En ocasiones el proceso de implantación no se trata tanto de adaptar SAP a la empresa, sino de adaptar la empresa a SAP.
    \end{itemize}
\end{itemize}

\textbf{Adaptación de SAP®}

\begin{itemize}
\item Los paquetes de SAP® incluyen diversas opciones de configuración para adaptarse a los detalles de operación de cada empresa.
\item Cuando los requisitos van más allá de retocar algún parámetro, se pueden escribir nuevas funcionalidades usando el lenguaje ABAP® (Advanced Business Application Programming).
\end{itemize}
\end{definicion}

\hypertarget{desarrollo-de-sistemas-de-informaciuxf3n}{%
\chapter{Desarrollo de Sistemas de
Información}\label{desarrollo-de-sistemas-de-informaciuxf3n}}

Las tareas a realizar son las mismas que para otro Sistema Informático:
Planificacion, Análisis, Diseño, Implementación, Pruebas, Instalación,
Mantenimiento. Se usan, asimismo, las mismas herramientas UML que en
cualquier otro proceso de software a las que se van a añadir otras.

\subsubsection*{Componentes de SI}

\begin{itemize}
\tightlist
\item
  Agentes externos: son elementos externos al sistema y que interactúan
  con el mismo. Pueden ser personas o sistemas informáticos, incluyendo
  otros SI. La interacción con el sistema de estos agentes consiste en
  enviar y recibir datos y eventos indicativos de acciones a realizar
  por el SI (o bien que informan de un estado). En la fase de análisis
  se identifica a a los agentes externos y sus roles de interacción.
\item
  Datos: componen un papel central en SI. Encontramos:

  \begin{itemize}
  \tightlist
  \item
    Aspectos Estructurales.
  \item
    Restricciones que nos indican configuraciones de los datos
    almacenados. Ambos aspectos pueden recogerse en la fase de Análisis,
    gracias al Análisis Conjunto guiado por las funciones.
  \end{itemize}
\item
  Software.
\item
  Hardware.
\end{itemize}

En la fase de diseño se usan los modelos de datos conceptuales y lógicos
adecuados al SI para tratar las estructuras y las restricciones.

En la fase de Análisis se sirve de los flujos de datos para describir el
sistema desde el punto de vista de almacenamiento, procesamientos,
adquisición y publicación de datos.

Se usan diagramas de flujos de datos (DFD) o el equivalente a diagrama
de actividades de
UML\footnote{Más rico semánticamente y que permite integrar flujos de datos y de control.}.

Se deben de tener en cuenta ciertos aspectos relacionados con el ciclo
de vida en el desarrollo SI:

\begin{itemize}
\tightlist
\item
  Adquisición de datos: fuente de datos y métodos de adquisición.
\item
  Uso de datos en el SI: flujos de datos, transformación y almacenado a
  través del SI e interacción con el exterior.
\item
  Archivado de datos: cuando y como eliminar datos del sistema.
\end{itemize}

diapo 8 acabar hemos dado hasta la 11

\begin{thebibliography}{99}

  \bibitem{Referencia1}
  Ismael Sallami Moreno, \textbf{Estudiante del Doble Grado en Ingeniería Informática + ADE}, Universidad de Granada, 2025.
  
  \bibitem{DiapositivasAsignatura}
  Universidad de Granada, \emph{Diapositivas de la asignatura}, Curso 2025/2026.

  % \bibitem{Referencia2}
  % Autor Apellido, \emph{Título del libro o artículo}, Editorial o Revista, Año.
  
  % \bibitem{Referencia3}
  % Nombre Autor, \emph{Título del documento}, Conferencia/URL, Año.
  
  \end{thebibliography}
  

\end{document}
