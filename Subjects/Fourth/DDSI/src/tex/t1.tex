\chapter{Introducción y Fundamentos de los Sistemas de Información (SI)}

\section{Conceptos Fundamentales}

\subsection{Definición de Sistema de Información (SI) e Información}

Un \textbf{Sistema de Información (SI)} se define como un conjunto, ya sea automatizado o manual, que integra personas, máquinas y/o métodos organizados con el propósito de recopilar, procesar, almacenar, transmitir, visualizar, diseminar y organizar información. Dentro del ámbito de esta disciplina, nos centramos específicamente en los Sistemas de Información Informáticos, que son aquellos que emplean Sistemas Gestores de Bases de Datos (SGBD) para el almacenamiento de la información que procesan.

La \textbf{información} es un concepto central para los SI. Se define como un conjunto organizado de datos procesados que constituyen un mensaje. Este mensaje tiene la capacidad de cambiar el estado de conocimiento del sujeto o sistema que lo recibe, aportando un nuevo saber sobre hechos, sucesos o entidades.

\begin{definicion}[Sistema de Información]
Un sistema, automatizado o manual, que engloba a personas, máquinas y/o métodos organizados para la recopilación, procesamiento, almacenamiento, transmisión, visualización, diseminación y organización de información.
\end{definicion}

\subsection{Relación entre Sistemas de Información y Bases de Datos}

La relación entre los Sistemas de Información (SI) y los Sistemas Gestores de Bases de Datos (SGBD) es jerárquica y funcional: \textbf{un SGBD es un componente fundamental de un SI, pero no es un SI en sí mismo}. Los SI utilizan los SGBD como el medio principal para almacenar y gestionar la información que procesan.

Es importante destacar varias dinámicas clave en esta relación:
\begin{itemize}
    \item Un mismo SI puede utilizar múltiples SGBD, o bien un SGBD que gestione una base de datos distribuida, o incluso una combinación de ambas arquitecturas.
    \item Diversos SI pueden interactuar y compartir datos de una misma base de datos, lo cual es una de las motivaciones principales para el uso de SGBD.
\end{itemize}

\begin{ejemplo}[Sistemas de la Universidad de Granada]
Los sistemas de Matrícula y de Actas de la Universidad de Granada (UGR) son ejemplos claros de SI que operan sobre una base de datos compartida gestionada por un SGBD Oracle. A su vez, estos pueden ser considerados subsistemas del sistema de información integral de la UGR, que abarca muchas otras áreas como nóminas, gestión de títulos, etc..
\end{ejemplo}

\section{Tipos, Ejemplos y Panorama General de los Sistemas de Información}

\subsection{Ejemplos de Sistemas de Información de uso común}

Los Sistemas de Información son omnipresentes en la vida cotidiana. A continuación, se presentan algunos ejemplos representativos:
\begin{itemize}
    \item \textbf{Redes Sociales}: Plataformas como Instagram, Facebook, X (antes Twitter) o TikTok son SI complejos. Frecuentemente, utilizan diversos SGBD, muchos de los cuales no se basan en el modelo relacional, para gestionar datos de usuarios, imágenes, vídeos, etc..
    \item \textbf{Sistemas de Gestión de Aprendizaje (SGA)}: Son SI diseñados para administrar, distribuir y controlar actividades formativas. Un ejemplo prominente es Moodle, que es de código abierto y compatible con SGBD como MySQL, PostgreSQL, Oracle, entre otros. El sistema PRADO de la UGR, por ejemplo, está basado en Moodle y se integra con otras bases de datos de la universidad.
    \item \textbf{Comercio Electrónico}: Desde tiendas de un único proveedor hasta grandes \textit{marketplaces} como Amazon o eBay, todos operan sobre SI para gestionar catálogos, transacciones y clientes.
    \item \textbf{Servicios de Google}: Herramientas como el buscador, Gmail, Google Maps y otros son SI que gestionan volúmenes masivos de datos.
    \item \textbf{Proveedores de contenido}: Servicios como Netflix se basan en SI para gestionar su catálogo de contenidos y las suscripciones de los usuarios.
\end{itemize}

\subsection{El SI en la Sociedad Actual: Un Panorama Completo}

Vivimos en la "Sociedad de la Información", una era definida por la omnipresencia de los SI, que son pilares fundamentales a nivel científico, económico y social. Estos sistemas son posibles gracias a los avances en hardware, telecomunicaciones e Internet.

Algunas características clave de los SI en el panorama actual son:
\begin{itemize}
    \item \textbf{Diversidad y Complejidad}: Existe una enorme variedad de SI, tanto en los tipos de información que manejan como en sus usos. Algunos gestionan datos de gran complejidad (imágenes, texto libre, redes), por lo que suelen emplear modelos de datos más avanzados que el relacional. De hecho, muchos de los sistemas informáticos más complejos que existen son SI.
    \item \textbf{Rol en la Empresa}: En el ámbito empresarial, los SI pueden dar soporte a la gestión del negocio o, en muchos casos, los datos mismos \textit{son} el negocio (por ejemplo, Google o Axesor).
    \item \textbf{Tecnologías de Desarrollo}: Se observa un uso intensivo de tecnologías web y de computación en la nube (\textit{Cloud Computing}) a través de plataformas como Amazon Web Services, Google Cloud o Microsoft Azure. Los sistemas más complejos suelen construirse sobre múltiples capas de abstracción y arquitecturas de hardware complejas, como clústeres de ordenadores.
\end{itemize}

\section{Profundización en los Sistemas de Información Empresariales}

\subsection{Roles Gerenciales y Funcionales en la Empresa}

Para comprender la estructura de los SI empresariales, es esencial analizar la organización de una empresa, que se divide en \textbf{áreas funcionales} y \textbf{niveles gerenciales}.

Las áreas funcionales típicas incluyen:
\begin{itemize}
    \item Finanzas
    \item Recursos Humanos
    \item Producción
    \item Marketing
    \item Servicios de información.
\end{itemize}

Los niveles gerenciales se estructuran jerárquicamente:
\begin{enumerate}
    \item \textbf{Planificación Estratégica}: Ocupado por los ejecutivos de alto nivel, se centra en las decisiones a largo plazo y la dirección general de la empresa.
    \item \textbf{Control Gerencial}: Involucra a directores de producto y jefes de división, responsables de la gestión táctica y la supervisión de las operaciones.
    \item \textbf{Control Operativo}: A cargo de jefes de departamento, jefes de proyecto y supervisores, se enfoca en las tareas cotidianas y la ejecución de planes.
\end{enumerate}

\subsection{Clasificación Jerárquica de los SI Empresariales}

Los Sistemas de Información Empresariales se pueden clasificar según el nivel gerencial al que dan soporte, formando una estructura jerárquica.

\subsubsection{Sistemas de Información Ejecutiva (EIS)}
Los \textbf{Executive Information Systems (EIS)} están diseñados para los ejecutivos del nivel de planificación estratégica. Su función es ofrecer una visión global y resumida del estado de la empresa de forma visual y sencilla, a menudo a través de un "cuadro de mando" (\textit{dashboard}). Permiten realizar análisis de escenarios (\textit{what-if analysis}) para simular el impacto de decisiones estratégicas.

\subsubsection{Sistemas de Soporte a la Decisión (DSS)}
Los \textbf{Decision Support Systems (DSS)} asisten a los gestores del nivel de control gerencial en la toma de decisiones, especialmente para problemas semiestructurados. Ayudan a analizar el impacto de las decisiones y pueden llegar a proponer cursos de acción.
Un tipo particular de DSS son los \textbf{Sistemas Basados en el Conocimiento (KBS)}, como los Sistemas Expertos, que codifican el conocimiento de un experto (p. ej., mediante reglas IF-THEN) para ampliar la capacidad de resolución de problemas de una persona.

\subsubsection{Sistemas de Información Gerencial (MIS)}
Los \textbf{Management Information Systems (MIS)} proporcionan información a los gestores, normalmente en forma de informes periódicos, para ayudarles a desempeñar sus funciones. Se especializan por área funcional:
\begin{itemize}
    \item \textbf{Marketing}: Apoyan en investigaciones de mercado, análisis de ventas, gestión de productos y precios. Un ejemplo clave es el CRM (\textit{Customer Relationship Management}), un sistema para la gestión integral de las relaciones con los clientes.
    \item \textbf{Producción}: Incluyen sistemas para el control de producción, inventario, calidad y la gestión de la cadena de suministro (SCM).
    \item \textbf{Finanzas}: Gestionan la contabilidad, auditorías, presupuestos y administración de fondos.
    \item \textbf{Recursos Humanos}: Cubren la planificación de personal, reclutamiento, nóminas y formación.
\end{itemize}

\subsubsection{Sistemas de Procesamiento de Transacciones (TPS)}
Los \textbf{Transaction Processing Systems (TPS)}, también conocidos como OLTP (\textit{Online Transaction Processing}), se sitúan en el nivel de control operativo y gestionan el funcionamiento cotidiano de la empresa. Procesan la información operativa generada por las transacciones diarias, como la recopilación de datos, su manipulación (clasificación, cálculos), almacenamiento y la preparación de documentos e informes.

\subsection{Sistemas de Planificación de Recursos Empresariales (ERP)}

Un \textbf{Enterprise Resource Planning (ERP)} es un sistema integrado de gestión que unifica los distintos flujos de información de una empresa (finanzas, compras, ventas, RRHH, etc.) en una única plataforma modular.

\begin{definicion}[ERP]
Un sistema integrado de software de gestión empresarial, compuesto por un conjunto de módulos funcionales que pueden ser adaptados a las necesidades de cada cliente.
\end{definicion}

Las características fundamentales de un ERP son:
\begin{itemize}
    \item Es un \textbf{único SI integrado} para toda la compañía.
    \item Utiliza una \textbf{base de datos centralizada} para facilitar el intercambio de información entre departamentos.
    \item Proporciona información \textbf{actualizada en tiempo real} sobre todos los procesos de negocio.
\end{itemize}

\subsubsection{Ejemplos de ERP Propietarios}
Existen varios sistemas ERP líderes en el mercado:
\begin{itemize}
    \item \textbf{SAP\textregistered}: Es un sistema estándar modular que puede ser parametrizado para cada empresa. Está indicado para grandes volúmenes de datos y su implementación a menudo implica adaptar los procesos de la empresa a las "mejores prácticas" que incorpora el software. Permite personalizaciones avanzadas mediante su propio lenguaje de programación, ABAP\textregistered.
    \item \textbf{Oracle\textregistered E-Business Suite}: Es un conjunto de aplicaciones empresariales integradas (CRM, Financials, HRMS, etc.) que utilizan un SGBD Oracle como repositorio central de datos.
    \item \textbf{Microsoft\textregistered Dynamics}: Es una suite de software ERP y CRM que resulta de la unión de varios productos adquiridos por Microsoft, como Dynamics AX, GP, NAV, etc..
\end{itemize}
Además de las soluciones propietarias, también existen ERP de software libre y código abierto.

\section{El Impacto de los SI: Ciberseguridad y el Mercado Profesional}

\subsection{Importancia Social y Económica de los SI}

Los Sistemas de Información son un pilar de la sociedad actual, con un impacto profundo a nivel científico, económico y social. Prácticamente todas las empresas, desde pymes hasta grandes corporaciones, utilizan uno o varios SI para operar o pueden beneficiarse de su uso. Lo mismo ocurre en la investigación científica, que se basa en la recopilación y análisis de datos. Además, los SI han transformado radicalmente las relaciones sociales a través de las redes y la mensajería instantánea.

\subsection{Aspectos Legales, Éticos y de Seguridad}

El uso generalizado de los SI conlleva importantes responsabilidades legales y éticas, especialmente en lo que respecta a la \textbf{protección de datos personales}. Los SI deben cumplir con la legislación vigente sobre adquisición, uso, conservación y privacidad de la información. Organismos como la Agencia Española de Protección de Datos (AEPD) en España y el Comité Europeo de Protección de Datos (CEPD) a nivel europeo velan por el cumplimiento de estas normativas.

Un aspecto clave es la \textbf{seguridad de los datos}, que implica controlar el acceso y evitar el robo o la destrucción de la información. Esto ha impulsado el crecimiento del ámbito de la \textbf{ciberseguridad} y el desarrollo de tecnologías como Blockchain para garantizar la integridad de los datos. También es relevante el control de la desinformación o \textit{Fake News} en Internet.

\subsection{Desarrollo Profesional en DDSI}

El campo del Diseño y Desarrollo de Sistemas de Información (DDSI) presenta una \textbf{alta y creciente demanda de profesionales}. Algunos de los perfiles más solicitados incluyen:
\begin{itemize}
    \item Directores de Sistemas de Información (CIO)
    \item Arquitectos de Computación en la Nube
    \item Especialistas en integración de ERPs
    \item Ingenieros de Big Data y Científicos de Datos
    \item Analistas y arquitectos de seguridad de datos
    \item Desarrolladores Web y de bases de datos.
\end{itemize}
Las asignaturas de la mención de Sistemas de Información en grados como el de Ingeniería Informática buscan dar respuesta a esta demanda social y profesional.

\section{Resumen y Conclusiones del Tema}

\begin{itemize}
    \item Los \textbf{Sistemas de Información (SI)} son sistemas informáticos que gestionan información almacenada en bases de datos (BD), utilizando para ello uno o más SGBD.
    \item En la actual Sociedad de la Información, los SI son omnipresentes y fundamentales en los ámbitos científico, económico y social, y pueden alcanzar una complejidad extraordinaria.
    \item El desarrollo de SI se apoya en una gran cantidad de recursos, plataformas y tecnologías, destacando las arquitecturas web y en la nube.
    \item El sector del DDSI experimenta una \textbf{demanda constante y creciente de perfiles profesionales} cualificados para hacer frente a los retos de un mundo cada vez más digitalizado.
\end{itemize}
