%%%%%%%%%%%%%%%%%%%%%%%%%%%%%%%%%%%%%%%%%%%%%%%%%%%%%%%%%%
% PREÁMBULO
%%%%%%%%%%%%%%%%%%%%%%%%%%%%%%%%%%%%%%%%%%%%%%%%%%%%%%%%%%

\documentclass{beamer}

% --- Paquetes Esenciales ---
\usepackage[utf8]{inputenc} 
\usepackage[T1]{fontenc}
\usepackage[spanish]{babel} 
\usepackage{graphicx} 
\usepackage{amsmath} 

% --- Configuración del Tema de Beamer ---
\usetheme{Madrid} 
\usecolortheme{default} 

% --- Metadatos de la Presentación (para la portada) ---
\title{Estándares Globales y Soberanía Nacional}
\subtitle{Análisis del Caso "Country Responsibility"}
\author{Ismael Sallami Moreno}
\institute{UGR}
\date{\today}

% Fuente de letra pequeña
\setbeamerfont{footnote}{size=\tiny}

%%%%%%%%%%%%%%%%%%%%%%%%%%%%%%%%%%%%%%%%%%%%%%%%%%%%%%%%%%
% DOCUMENTO
%%%%%%%%%%%%%%%%%%%%%%%%%%%%%%%%%%%%%%%%%%%%%%%%%%%%%%%%%%

\begin{document}

% --- Diapositiva 1: Portada ---
\begin{frame}
\titlepage 
\end{frame}

% --- Diapositiva 2: El Dilema de la Responsabilidad Nacional ---
\section{Estándares y Soberanía}
\begin{frame}
\frametitle{El Dilema de la Responsabilidad Nacional}

\begin{itemize}
    \item \textbf{Soberanía vs. Interdependencia:} ¿Debe la nación exportadora definir sus propias normas laborales y ambientales?
    \item \textbf{El Riesgo del "Race to the Bottom":}
    \begin{enumerate}
        \item Países con normas laxas (\(\text{S}_L\)) atraen inversión (\textbf{FDI}).
        \item Objetivo: Aumentar la \alert{ventaja comparativa de bajos costos}.
    \end{enumerate}
    \item \textbf{La Presión Externa (EE. UU./UE):}
    \begin{itemize}
        \item Impedir la importación de bienes producidos bajo \(\text{S}_L\).
    \end{itemize}
\end{itemize}

\vfill
\centering
\alert{\textbf{Pregunta Central:}} ¿Son los estándares una herramienta de política social o una \textbf{barrera comercial disfrazada}?

\end{frame}

% --- Diapositiva 3: Estándares como Fricción Comercial ---
\section{Impacto Económico}
\begin{frame}
\frametitle{Estándares como Barreras No Arancelarias (NTB)}

\begin{block}{\textbf{Estándares Mínimos vs. Costos de Producción}}
\begin{enumerate}
    \item \textbf{Aumento de Costos Fijos:} Cumplir con estándares laborales (salarios mínimos, seguridad) o ambientales requiere \textbf{inversión y mayor costo marginal}.
    \item \textbf{Efecto Protección:} Un estándar (ej. Cero deforestación) es equivalente a un \alert{arancel implícito} para los exportadores.
    \item \textbf{Distorsión de la Ventaja Comparativa:} Los costos impuestos eliminan la ventaja que deriva de diferencias regulatorias legítimas.
\end{enumerate}
\end{block}

\vfill
% \begin{figure}
% \centering
% \includegraphics[width=0.7\textwidth]{images/img-3-16.png} % Usamos una imagen de ejemplo o conceptual
% \caption*{Las NTB reducen los flujos de comercio.}
% \end{figure}

\end{frame}

% --- Diapositiva 4: Soluciones Globales y la OMC ---
\section{Gobernanza y Cooperación Internacional}
\begin{frame}
\frametitle{Mecanismos de Responsabilidad y Cooperación}

\begin{itemize}
    \item \textbf{Marco Multilateral (OMC):} Los países grandes tienen incentivos para usar estándares como fricciones comerciales. El \alert{WTO} debe gestionar estos conflictos para evitar la guerra de aranceles.
    \item \textbf{Acuerdos Ambientales Globales:}
    \begin{itemize}
        \item Ejemplos de esfuerzos de cooperación: \textbf{Protocolo de Kioto} y \textbf{Acuerdo de París (COP21)}.
        \item Estos evitan la acción unilateral que distorsiona el libre comercio.
    \end{itemize}
    \item \textbf{Principio de Subsidio (Feenstra/Taylor):} La responsabilidad recae en el país de origen, pero los acuerdos pueden facilitar la \alert{convergencia} y reducir el riesgo de proteccionismo.
\end{itemize}

\end{frame}

% --- Diapositiva 5: Conclusión ---
\begin{frame}
\frametitle{Conclusiones: Comercio y Coherencia Regulatoria}

\begin{block}{\textbf{Síntesis}}
\begin{itemize}
    \item La \textbf{responsabilidad nacional} es esencial, pero la globalización crea conflictos de externalidades.
    \item Los \textbf{estándares externos} actúan como un costo de transacción significativo, limitando el acceso al mercado y afectando el \textbf{patrón de comercio}.
    \item La solución efectiva requiere \alert{mecanismos de cooperación} que trasciendan la soberanía sin caer en el proteccionismo unilateral.
\end{itemize}
\end{block}

\end{frame}


% --- Diapositiva 4: Preguntas ---
\begin{frame}
\centering
\Huge \textbf{¡Gracias!}
\vfill
\Large ¿Preguntas?
\end{frame}

\end{document}
