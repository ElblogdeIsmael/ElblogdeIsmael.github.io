%%%%%%%%%%%%%%%%%%%%%%%%%%%%%%%%%%%%%%%%%%%%%%%%%%%%%%%%%%

% PREÁMBULO

%%%%%%%%%%%%%%%%%%%%%%%%%%%%%%%%%%%%%%%%%%%%%%%%%%%%%%%%%%

\documentclass{beamer}

% --- Paquetes Esenciales ---

\usepackage[utf8]{inputenc} % Acentos y caracteres en español

\usepackage[T1]{fontenc}

\usepackage[spanish]{babel} % Configuración para el idioma español

\usepackage{graphicx} % Para incluir imágenes (opcional, si se usa un gráfico)

\usepackage{amsmath} % Para fórmulas económicas

% --- Configuración del Tema de Beamer ---

\usetheme{Madrid} % Un tema popular y limpio.

\usecolortheme{default} % Esquema de color

% --- Metadatos de la Presentación (para la portada) ---

\title{Caso 4.10: Alimentos Biotecnológicos en Europa}

\subtitle{Interacción entre Estándares Internos y Comercio Internacional}

\author{Ismael Sallami Moreno}

\institute{UGR}

\date{\today}

% Fuente de letra pequeña

\setbeamerfont{footnote}{size=\tiny}

%%%%%%%%%%%%%%%%%%%%%%%%%%%%%%%%%%%%%%%%%%%%%%%%%%%%%%%%%%

% DOCUMENTO

%%%%%%%%%%%%%%%%%%%%%%%%%%%%%%%%%%%%%%%%%%%%%%%%%%%%%%%%%%

\begin{document}

% --- Diapositiva 1: Portada ---

\begin{frame}

\titlepage % Muestra la portada con la información definida arriba

\end{frame}

% --- Diapositiva 2: Introducción al Conflicto (Sección 1) ---

\section{El Conflicto Comercial por los OGM}

\begin{frame}

\frametitle{El Desafío de los Alimentos Biotecnológicos en Europa}

\begin{block}{\textbf{El Choque de Estándares}}

\begin{itemize}

\item \textbf{Contexto:} Conflicto entre \alert{EE. UU.} (principal exportador agrícola) y la \alert{UE} (alta preocupación del consumidor).

\item \textbf{Restricciones:} La UE impone normativas estrictas de importación por seguridad alimentaria y preocupaciones ambientales sobre \textbf{Organismos Genéticamente Modificados (OGM)}.

\item \textbf{Debate Comercial:} ¿Son estas normativas una legítima protección sanitaria o una \textbf{barrera comercial injustificada} encubierta?

\end{itemize}

\end{block}

\end{frame}

% --- Diapositiva 3: Mecanismo de Restricción (Sección 2) ---

\section{El Etiquetado como Barrera No Arancelaria}

\begin{frame}

\frametitle{Mecanismo de Control: El Poder del Etiquetado}

\begin{itemize}

\item \textbf{Enfoque de la UE:} Requisito de \alert{etiquetado obligatorio} para los alimentos y piensos que contienen OGM.

\item \textbf{Efecto Económico:} El etiquetado permite al consumidor tomar decisiones informadas, alineando la \textbf{demanda de importación} con la \textbf{preferencia doméstica} por alimentos no OGM.

\item \textbf{Impacto:} El comercio se limita no por una prohibición, sino por la \textbf{elasticidad de la demanda} ante la información (etiqueta).

\end{itemize}

\vfill

\end{frame}

% --- Diapositiva 4: Consecuencias y Principio del WTO (Sección 3) ---

\section{Lecciones de Política Comercial}

\begin{frame}

\frametitle{Normas de la OMC y Preferencias del Consumidor}

\begin{itemize}

\item \textbf{Criterio WTO:} Los estándares (como el etiquetado) son difíciles de impugnar si se aplican de manera \alert{no discriminatoria} a bienes nacionales e importados.

\item \textbf{Conclusión Clave:} El etiquetado ejerce un poder significativo para "limitar tales importaciones si los consumidores así lo eligen".

\item \textbf{Implicación:} Las pérdidas comerciales son un costo que la economía (exportadora) paga por no alinearse con las \textbf{preferencias fundadas en valores} (salud, medio ambiente) del mercado importador.

\end{itemize}

\end{frame}

% --- Diapositiva 5: Preguntas ---

\begin{frame}

\centering

\Huge \textbf{¡Gracias por su atención!}

\vfill

\Large ¿Preguntas?

\end{frame}

\end{document}