%%%%%%%%%%%%%%%%%%%%%%%%%%%%%%%%%%%%%%%%%%%%%%%%%%%%%%%%%%
% PREÁMBULO
%%%%%%%%%%%%%%%%%%%%%%%%%%%%%%%%%%%%%%%%%%%%%%%%%%%%%%%%%%

\documentclass{beamer}

% --- Paquetes Esenciales ---
\usepackage[utf8]{inputenc} 
\usepackage[T1]{fontenc}
\usepackage[spanish]{babel} 
\usepackage{graphicx} 
\usepackage{amsmath} 
\usepackage{tikz} % Para gráficos conceptuales simples

% --- Configuración del Tema de Beamer ---
\usetheme{Madrid} 
\usecolortheme{default} 

% --- Metadatos de la Presentación (para la portada) ---
\title{Estándares Corporativos y Comercio Internacional}
\subtitle{El Impacto de las Órdenes de Walmart en sus Proveedores Chinos}
\author{Ismael Sallami Moreno}
\institute{UGR}
\date{\today}

% Fuente de letra pequeña
\setbeamerfont{footnote}{size=\tiny}

%%%%%%%%%%%%%%%%%%%%%%%%%%%%%%%%%%%%%%%%%%%%%%%%%%%%%%%%%%
% DOCUMENTO
%%%%%%%%%%%%%%%%%%%%%%%%%%%%%%%%%%%%%%%%%%%%%%%%%%%%%%%%%%

\begin{document}

% --- Diapositiva 1: Portada ---
\begin{frame}
\titlepage 
\end{frame}

% --- Diapositiva 2: Walmart y la Gobernanza Privada ---
\section{Gobernanza Privada y Cadena de Suministro}
\begin{frame}
\frametitle{Walmart: Regulador Privado Global}

\begin{itemize}
    \item \textbf{El Rol del Comprador Dominante:} Walmart (o Apple, Toyota) no solo transa, sino que \alert{impone normas de calidad, sociales o ambientales}.
    \item \textbf{Estándares Internos (Private Standards):}
    \begin{enumerate}
        \item Aseguran la \textbf{reputación de marca} y mitigan riesgos legales.
        \item Redefinen las \textbf{condiciones de entrada} para los proveedores chinos.
    \end{enumerate}
\end{itemize}

\vfill

\begin{figure}[h!]
\centering
\begin{tikzpicture}[scale=0.9]

    % Nodos
    \node[draw, thick, rectangle, minimum width=3cm, minimum height=1.2cm] (W) at (0, 0) {Walmart (EE. UU.)};
    \node[draw, thick, rectangle, minimum width=3cm, minimum height=1.2cm, fill=blue!10] (P) at (8, 0) {Proveedores (China)};

    % Flecha bienes/servicios
    \draw[->, very thick] (P.west) -- (W.east)
        node[midway, above=6pt] {Bienes/Servicios};

    % Flecha estándares
    \draw[->, dashed, thick, blue] (W.north east) to[bend left=25]
        node[midway, above=6pt] {Estándares y Costos $\uparrow$} (P.north west);

\end{tikzpicture}

\vspace{0.5em}
\caption{Gobernanza vertical en la cadena de valor.}
\end{figure}


\end{frame}

% --- Diapositiva 3: Costos de Cumplimiento y Arbitraje (LOOP) ---
\section{Estándares como Fricción Comercial}
\begin{frame}
\frametitle{Estándares: Barreras No Arancelarias}

\begin{block}{\textbf{Impacto Económico del Cumplimiento}}
\begin{enumerate}
    \item \textbf{Aumento de los Costos de Producción (\(C\)):} Inversión en nuevas tecnologías, certificaciones, inspecciones y salarios mejorados.
    \item \textbf{Fricción Comercial:} Los costos de cumplimiento actúan como \alert{costos de transacción} o barreras no arancelarias.
    \item \textbf{Desviación de la LOOP:} Estos costos impiden que la \textbf{Ley del Precio Único (LOOP)} se cumpla perfectamente.
\end{enumerate}
\end{block}

\vfill

\begin{figure}
\centering
\begin{itemize}
    \item[\(\rightarrow\)] Precio EE. UU. = \(E_{\$/\text{¥}} \cdot P_{\text{China}} + C_{\text{Compliance}} + C_{\text{Transporte}}\)
    \item[\(\rightarrow\)] Si $C_{\text{Compliance}} > 0$: \textbf{Arbitraje es limitado} (Teoría de LOOP/PPP)
\end{itemize}
\caption{Costos adicionales en la conversión de precios.}
\end{figure}

\end{frame}

% --- Diapositiva 4: Efectos en la Cadena de Valor Global ---
\section{Implicaciones para la Globalización}
\begin{frame}
\frametitle{Verticalización y Especialización de la Producción}

\begin{itemize}
    \item \textbf{Comercio en Insumos Intermedios:} Gran parte del comercio global se debe a la \alert{especialización vertical} y el \textbf{offshoring}.
    \item \textbf{Reconfiguración de la Cadena:} El alza de estándares y costos puede llevar a la \textbf{re-localización ("reshoring" o "quicksourcing")}.
    \begin{itemize}
        \item Proximidad Geográfica $\rightarrow$ Más importante que salarios bajos absolutos (Ej. México vs. China).
    \end{itemize}
    \item \textbf{Consistencia Regulatoria:} Los estándares de Walmart fuerzan una \alert{convergencia regulatoria de facto} entre los proveedores, mejorando potencialmente la gobernanza local.
\end{itemize}

\end{frame}

% --- Diapositiva 5: Conclusiones ---
\begin{frame}
\frametitle{Conclusiones: Estándares y Flujos de Comercio}

\begin{block}{\textbf{Síntesis Académica}}
\begin{itemize}
    \item Estándares corporativos $\neq$ Política de comercio tradicional.
    \item Funcionan como \textbf{costos fijos} o \textbf{fricciones comerciales} en el marco de la Economía Internacional.
    \item Afectan la \alert{rentabilidad de la especialización vertical}, influyendo en el \textbf{patrón de comercio} futuro.
\end{itemize}
\end{block}
\end{frame}

% --- Diapositiva 4: Preguntas ---
\begin{frame}
\centering
\Huge \textbf{¡Gracias!}
\vfill
\Large ¿Preguntas?
\end{frame}

\end{document}
