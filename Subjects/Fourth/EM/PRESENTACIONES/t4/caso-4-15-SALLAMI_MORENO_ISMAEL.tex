%%%%%%%%%%%%%%%%%%%%%%%%%%%%%%%%%%%%%%%%%%%%%%%%%%%%%%%%%%
% PREÁMBULO
%%%%%%%%%%%%%%%%%%%%%%%%%%%%%%%%%%%%%%%%%%%%%%%%%%%%%%%%%%

\documentclass{beamer}

% --- Paquetes Esenciales ---

\usepackage[utf8]{inputenc} % Acentos y caracteres en español
\usepackage[T1]{fontenc}
\usepackage[spanish]{babel} % Configuración para el idioma español
\usepackage{graphicx} % Para incluir imágenes
\usepackage{amsmath} % Para fórmulas económicas
\usepackage{bookmark} % Soluciona el warning de rerunfilecheck
\usepackage{lmodern} % Soluciona el problema de las formas de fuente

% --- Configuración del Tema de Beamer ---

\usetheme{Madrid} % Un tema popular y limpio.
\usecolortheme{default} % Esquema de color

% --- Metadatos de la Presentación (para la portada) ---

\title{Comercio de Paneles Solares y Política Ambiental}
\subtitle{Análisis de Aranceles (Caso 4.15)}
\author{Ismael Sallami Moreno}
\institute{UGR}
\date{\today}

% Fuente de letra pequeña

\setbeamerfont{footnote}{size=\tiny}

%%%%%%%%%%%%%%%%%%%%%%%%%%%%%%%%%%%%%%%%%%%%%%%%%%%%%%%%%%
% DOCUMENTO
%%%%%%%%%%%%%%%%%%%%%%%%%%%%%%%%%%%%%%%%%%%%%%%%%%%%%%%%%%

\begin{document}

% --- Diapositiva 1: Portada ---

\begin{frame}
\titlepage % Muestra la portada con la información definida arriba
\end{frame}

%%%%%%%%%%%%%%%%%%%%%%%%%%%%%%%%%%%%%%%%%%%%%%%%%%%%%%%%%%

\section{Política Comercial e Instrumentos}

% --- Diapositiva 2: Instrumentos de Política Comercial ---

\begin{frame}
\frametitle{Dumping y Derechos de Salvaguarda}

\begin{block}{\textbf{El Conflicto Solar: China vs. Occidente}}
\begin{itemize}
    \item \textbf{Causa:} Precios de exportación de China muy bajos (dumping), debido a subsidios a la producción.
    \item \textbf{Instrumentos:} EE. UU. y la UE impusieron \alert{Derechos Antidumping y Compensatorios} (CVD) .
    \item \textbf{Magnitud (Ej. EE. UU.):} Tasas combinadas de entre $\mathbf{18\%}$ y $\mathbf{32\%}$ aplicadas a productores chinos.
\end{itemize}
\end{block}

\begin{block}{\textbf{El Rol del Arancel (Section 201)}}
\begin{itemize}
    \item \textbf{EE. UU. (2018):} Sustituyó las tasas por aranceles de \alert{Salvaguarda} (Sección 201), buscando proteger a la industria doméstica (Ej. SolarWorld/Tesla).
    \item \textbf{Justificación Clave:} Proporcionar protección temporal a una potencial "Industria Infantil" (Infant Industry).
\end{itemize}
\end{block}
\end{frame}

% --- Diapositiva 3: Externalidades y Bienestar ---

\begin{frame}
\frametitle{Aranceles y Ganancia Social Neta}

\begin{block}{\textbf{Comercio de Paneles: Una Externalidad Positiva}}
\begin{itemize}
    \item \textbf{Externalidad:} Los paneles solares desplazan la generación de electricidad basada en combustibles fósiles.
    \item \alert{Efecto Ambiental:} Reducción de la Externalidad Negativa de la $\mathbf{CO}_2$ (ganancia social).
    \item \textbf{El Dilema:} Los aranceles, al mantener el precio alto, \textbf{reducen la ganancia social} asociada a la energía limpia.
\end{itemize}
\end{block}

\begin{block}{\textbf{Contraste de Políticas (Post-2018)}}
\begin{itemize}
    \item \textbf{Unión Europea:} Decidió \alert{eliminar} los derechos en 2018. Razón: Permitir que los precios cayeran al precio mundial y así maximizar las ganancias sociales/ambientales.
    \item \textbf{EE. UU.:} Mantuvo los aranceles. Prioridad: Protección de los productores nacionales sobre la maximización inmediata del bienestar del consumidor y la ganancia social.
\end{itemize}
\end{block}
\end{frame}
% --- Diapositiva 4: Preguntas ---
\begin{frame}
\centering
\Huge \textbf{¡Gracias!}
\vfill
\Large Preguntas y debate.
\end{frame}

\end{document}