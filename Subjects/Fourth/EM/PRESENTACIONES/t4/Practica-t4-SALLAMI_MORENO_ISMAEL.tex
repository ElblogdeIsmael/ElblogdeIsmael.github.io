% ENUMERO DE PRÁCTICA: T4:Demuestre gráficamente que cuanto más elásticas son DN y SN, 
% mayor será el efecto creación comercio y, por tanto, mayor será la 
% ganancia de bienestar




%%%%%%%%%%%%%%%%%%%%%%%%%%%%%%%%%%%%%%%%%%%%%%%%%%%%%%%%%%
% PREÁMBULO
%%%%%%%%%%%%%%%%%%%%%%%%%%%%%%%%%%%%%%%%%%%%%%%%%%%%%%%%%%





\documentclass{beamer}

% --- Paquetes Esenciales ---
\usepackage[utf8]{inputenc} % Acentos y caracteres en español
\usepackage[T1]{fontenc}
\usepackage[spanish]{babel} % Configuración para el idioma español
\usepackage{graphicx} % Para incluir imágenes
\usepackage{amsmath} % Para fórmulas económicas
\usepackage{amssymb} % Para símbolos matemáticos
\usepackage{tikz} % Para gráficos con TikZ

% --- Configuración del Tema de Beamer ---
\usetheme{Madrid} % Un tema popular y limpio.
\usecolortheme{default} % Esquema de color

% --- Metadatos de la Presentación (para la portada) ---

\title{Elasticidad, Creación de Comercio y Bienestar}
\subtitle{Demostración Gráfica de los Efectos de la Elasticidad}

\author{Ismael Sallami Moreno}
\institute{UGR}

\date{\today}

% Fuente de letra pequeña
\setbeamerfont{footnote}{size=\tiny}

%%%%%%%%%%%%%%%%%%%%%%%%%%%%%%%%%%%%%%%%%%%%%%%%%%%%%%%%%%
% DOCUMENTO
%%%%%%%%%%%%%%%%%%%%%%%%%%%%%%%%%%%%%%%%%%%%%%%%%%%%%%%%%%

\begin{document}

% --- Diapositiva 1: Portada ---
\begin{frame}
\titlepage 
\end{frame}

% --- Diapositiva 2: Marco Teórico y Conceptos Clave ---
\begin{frame}
\frametitle{Marco Conceptual: Creación de Comercio}
\begin{block}{\textbf{Modelo de País Pequeño y Ganancias}}
\begin{itemize}
    \item \textbf{País Pequeño}: El precio mundial ($P^W$) es inalterable por las decisiones domésticas.
    \item \textbf{Ganancia de Bienestar}: Se mide por la variación del excedente (Consumidor y Productor) tras la apertura comercial.
    \item \alert{Creación de Comercio}: Es el cambio neto positivo en el excedente social ($EC+EP$) que resulta de la especialización eficiente.
    \begin{itemize}
        \item Gráficamente, corresponde a los \textbf{triángulos de eficiencia} (antiguamente la \textit{pérdida de peso muerto}, $b+d$) que se generan o se evitan al comerciar.
    \end{itemize}
\end{itemize}
\end{block}
\end{frame}

% --- Diapositiva 3: Demostración Gráfica de la Elasticidad ---
\begin{frame}
\frametitle{Demostración: Elasticidad y la Magnitud de la Ganancia}
\begin{figure}
    \centering
    \begin{tikzpicture}[scale=1.0]

        % Eje Y (Precio)
        \draw[->] (0,0) -- (0,7) node[above] {$P$};
        % Eje X (Cantidad)
        \draw[->] (0,0) -- (10,0) node[right] {$Q$};
        
        % Precio Mundial (P_W) - Bajo
        \draw[dashed, blue] (0, 2) -- (9, 2) node[right, blue] {$P^W$};
        
        % Precio Arancelario (P_T) - Alto
        \draw[dashed, red] (0, 5) -- (9, 5) node[right, red] {$P^T$};
        
        % Curva Oferta Inelástica (S_I)
        \draw[thick, orange] (1, 1) -- (4, 6) node[right, orange] {$S_I$};
        % Curva Demanda Inelástica (D_I)
        \draw[thick, orange] (8, 6) -- (5, 1) node[left, orange] {$D_I$};
        
        % Curva Oferta Elástica (S_E)
        \draw[thick, green] (0.5, 3) -- (9, 4) node[right, green!80!black] {$S_E$};
        % Curva Demanda Elástica (D_E)
        \draw[thick, green] (9, 4.5) -- (1, 3.5) node[left, green!80!black] {$D_E$};
        
        % Puntos para Curvas Inelásticas (I)
        \coordinate (SI_T) at (2.5, 5);
        \coordinate (SI_W) at (1.7, 2);
        \coordinate (DI_T) at (6.5, 5);
        \coordinate (DI_W) at (7.3, 2);

        % Puntos para Curvas Elásticas (E)
        \coordinate (SE_T) at (7.5, 5);
        \coordinate (SE_W) at (4.5, 2);
        \coordinate (DE_T) at (2.5, 5);
        \coordinate (DE_W) at (8.5, 2);

        % Triángulos (solo para el caso elástico - mayor área)
        \fill[red!30, opacity=0.7] (SE_W) -- (SE_T) -- (4.5, 5); % Triángulo b (Producción)
        \fill[blue!30, opacity=0.7] (DE_T) -- (DE_W) -- (2.5, 2); % Triángulo d (Consumo)
        
        % Leyenda de áreas (simplificada)
        \node at (6.5, 6.5) {\small Ganancia (b+d) $\propto \Delta Q$};
        \node[green!80!black] at (8.5, 6.2) {\small Curvas Elásticas (Mayor $\Delta Q$)};
        \node[orange!80!black] at (8.5, 0.7) {\small Curvas Inelásticas (Menor $\Delta Q$)};

        % Etiquetas de triángulos
        \node[red!80!black] at (6.2, 3.2) {\small $b$};
        \node[blue!80!black] at (5.2, 3.2) {\small $d$};

    \end{tikzpicture}
\end{figure}


\end{frame}

\begin{frame}
    \begin{block}{\textbf{Elasticidad y Cantidad Demandada/Ofrecida ($\Delta Q$)}}
    \begin{itemize}
        \item Curvas \textbf{más elásticas} (más planas) implican una \textbf{mayor} respuesta en $Q$ ante el cambio de precio ($P^T \to P^W$).
        \item Ganancia de Bienestar es proporcional a $1/2 \cdot \Delta P \cdot \Delta Q$.
        \item Mayor $\Delta Q$ genera \textbf{mayor área} en los triángulos de eficiencia.
    \end{itemize}
    \end{block}
\end{frame}

% --- Diapositiva 4: Conclusión e Implicaciones ---
\begin{frame}
\frametitle{Conclusión: Ajuste de Mercado y Bienestar}
\begin{block}{\textbf{El Rol de la Flexibilidad en las Ganancias}}
\begin{itemize}
    \item \textbf{Si $D_N$ es elástica}: Los consumidores se ajustan fácilmente, sustituyendo importaciones, lo que maximiza la ganancia por eficiencia de consumo (triángulo $d$).
    \item \textbf{Si $S_N$ es elástica}: Los productores se especializan rápidamente, liberando recursos para otros usos, maximizando la ganancia por eficiencia de producción (triángulo $b$).
    \item \alert{Ajuste del Mercado}: La ganancia social es mayor si los agentes económicos pueden modificar sus decisiones de consumo y producción con facilidad.
\end{itemize}
\end{block}
\end{frame}

\end{document}