\documentclass{beamer}

% --- Paquetes Esenciales ---

\usepackage[utf8]{inputenc} % Acentos y caracteres en español

\usepackage[T1]{fontenc}

\usepackage[spanish]{babel} % Configuración para el idioma español

\usepackage{graphicx} % Para incluir imágenes

\usepackage{amsmath} % Para fórmulas económicas
\usepackage{amssymb}

% --- Configuración del Tema de Beamer ---

\usetheme{Madrid} % Un tema popular y limpio.
\usecolortheme{default} % Esquema de color

% --- Metadatos de la Presentación (para la portada) ---

\title{The Effect of FDI on Rentals and Wages in 
Singapore}

\subtitle{Análisis Corto y Largo Plazo de la Inversión Extranjera Directa}

\author{Ismael Sallami Moreno}

\institute{UGR}

\date{\today}

% Fuente de letra pequeña

\setbeamerfont{footnote}{size=\tiny}

%%%%%%%%%%%%%%%%%%%%%%%%%%%%%%%%%%%%%%%%%%%%%%%%%%%%%%%%%%

% DOCUMENTO

%%%%%%%%%%%%%%%%%%%%%%%%%%%%%%%%%%%%%%%%%%%%%%%%%%%%%%%%%%

\begin{document}

% --- Diapositiva 1: Portada ---
\begin{frame}
\titlepage 
\end{frame}

%%%%%%%%%%%%%%%%%%%%%%%%%%%%%%%%%%%%%%%%%%%%%%%%%%%%%%%%%%

\section{Efectos de la IED en Singapur}

% --- Diapositiva 2: Corto Plazo (Modelo de Factores Específicos) ---
\begin{frame}
\frametitle{Corto Plazo: Modelo de Factores Específicos}
\begin{block}{\textbf{Impacto Inmediato de la Inversión Masiva de Capital}}
\begin{itemize}
    \item \textbf{Contexto:} Singapur atrae $\mathbf{IED}$ masiva (4º mayor stock mundial en 2005), especialmente en el sector electrónico.
    \item \textbf{Mecanismo Clave:} Razón Capital-Trabajo $\mathbf{(K/L)}$ crece $\mathbf{\approx 5\%}$ anual (1970–1990).
\end{itemize}
\end{block}

\begin{alertblock}{\textbf{Resultados Empíricos (1970–1990)}}
\begin{itemize}
    \item \textbf{Renta del Capital ($\mathbf{R_K}$):} $\mathbf{\downarrow -3.4\%}$ anual.\\ (\textit{Rendimientos decrecientes del factor específico}).
    \item \textbf{Salario Real ($\mathbf{W}$):} $\mathbf{\uparrow +1.6\%}$ anual.\\ (\textit{Mayor $\mathbf{K/L}$ aumenta la $\mathbf{PML}$}).
\end{itemize}
\end{alertblock}
\vfill
\end{frame}

% --- Diapositiva 3: Largo Plazo (Modelo H-O y Productividad) ---
\begin{frame}
\frametitle{Largo Plazo: Ajuste y Productividad (1970–1990)}
\begin{block}{\textbf{Teoría vs. Realidad: El Factor Productividad}}
\begin{itemize}
    \item \textbf{Predicción H-O/Factor Price Insensitivity:} El ajuste total de la industria debe mantener $\mathbf{W}$ y $\mathbf{R_K}$ constantes (Teorema de Rybczynski).
    \item \textbf{Renta del Capital (Observada):} Mantenida estable o con ligera caída ($\mathbf{\approx \pm 0.0\%}$ anual).
    \item \textbf{Salario Real (Observado):} $\mathbf{\uparrow +2.7\%}$ a $\mathbf{+3.6\%}$ anual.
\end{itemize}
\end{block}
\begin{exampleblock}{Implicación Económica}
El fuerte crecimiento salarial, sin el colapso de $\mathbf{R_K}$, indica que el efecto de la $\mathbf{IED}$ fue superado por un $\mathbf{\text{Crecimiento Significativo de la Productividad Total de los Factores (PTF)}}$.
\end{exampleblock}
\end{frame}

% --- Diapositiva 4: Conclusiones Clave ---
\begin{frame}
\frametitle{Lecciones Clave de Singapur}
\begin{itemize}
    \item \textbf{Beneficio del Factor Escaso (Trabajo):} La IED aumenta consistentemente la productividad marginal del trabajo, impulsando $\mathbf{W}$ al alza en el corto y largo plazo.
    \item \textbf{Absorción de Capital:} El crecimiento masivo de $\mathbf{K}$ (IED) se absorbe, ya sea de forma inmediata vía rendimientos decrecientes, o a largo plazo vía expansión del sector intensivo en $\mathbf{K}$.
    \item \textbf{Crecimiento Sostenido:} La $\mathbf{Productividad}$ es la clave para evitar que el retorno del capital se desplome y para asegurar el aumento salarial, contrarrestando el efecto de rendimientos decrecientes.
\end{itemize}
\vfill

\end{frame}



% --- Diapositiva 5: Preguntas ---
\begin{frame}
\centering
\Huge \textbf{¡Gracias por su atención!}
\vfill
\Large ¿Preguntas?
\end{frame}

\end{document}
