%%%%%%%%%%%%%%%%%%%%%%%%%%%%%%%%%%%%%%%%%%%%%%%%%%%%%%%%%%
% PREÁMBULO
%%%%%%%%%%%%%%%%%%%%%%%%%%%%%%%%%%%%%%%%%%%%%%%%%%%%%%%%%%

\documentclass{beamer}

% --- Paquetes Esenciales ---
\usepackage[utf8]{inputenc} % Acentos y caracteres en español
\usepackage[T1]{fontenc}
\usepackage[spanish]{babel} % Configuración para el idioma español
\usepackage{graphicx} % Para incluir imágenes

% --- Configuración del Tema de Beamer ---
\usetheme{Madrid} % Tema popular y limpio
\usecolortheme{default} % Esquema de color

% --- Metadatos de la Presentación (para la portada) ---
\title{The Effects of Migration on Wages}
\subtitle{Análisis de Corto y Largo Plazo (Feenstra \& Taylor, Cap. 5)}
\author{Ismael Sallam Moreno}
\institute{UGR}
\date{Feenstra, R. C. \& Taylor, A. M. (2020/2021)}

%%%%%%%%%%%%%%%%%%%%%%%%%%%%%%%%%%%%%%%%%%%%%%%%%%%%%%%%%%
% DOCUMENTO
%%%%%%%%%%%%%%%%%%%%%%%%%%%%%%%%%%%%%%%%%%%%%%%%%%%%%%%%%%

\begin{document}

% --- Diapositiva 1: Portada ---
\begin{frame}
\titlepage % Muestra la portada con la información definida arriba
\end{frame}

% --- Diapositiva 2: Índice (Generación opcional para estructura) ---
\begin{frame}
\frametitle{Contenido}
\tableofcontents % Muestra las secciones como un índice
\end{frame}

%%%%%%%%%%%%%%%%%%%%%%%%%%%%%%%%%%%%%%%%%%%%%%%%%%%%%%%%%%
\section{Efectos a Corto Plazo: Capital Fijo}
%%%%%%%%%%%%%%%%%%%%%%%%%%%%%%%%%%%%%%%%%%%%%%%%%%%%%%%%%%

% --- Diapositiva 3: Corto Plazo ---
\begin{frame}
\frametitle{I. El Corto Plazo: Modelo de Factores Específicos}

\begin{itemize}
    \item \textbf{Asunción:} Capital (\textit{K}) y Tierra (\textit{T}) son fijos (Modelo de Factores Específicos).
    \item \textbf{Resultado General:} La inmigración presiona a la baja los salarios.
    \item \textbf{Impacto Agregado:} Reducción salarial promedio de \alert{\textbf{$-3.0\%$}} (1990-2006).
    \item \textbf{Segmentos Vulnerables:} Mayor caída en los salarios de trabajadores de:
        \begin{itemize}
            \item Baja calificación (< 12 años de educación).
            \item Alta calificación (posgrado), por competencia directa.
        \end{itemize}
\end{itemize}

\end{frame}

%%%%%%%%%%%%%%%%%%%%%%%%%%%%%%%%%%%%%%%%%%%%%%%%%%%%%%%%%%
\section{Efectos a Largo Plazo: Ajuste del Capital}
%%%%%%%%%%%%%%%%%%%%%%%%%%%%%%%%%%%%%%%%%%%%%%%%%%%%%%%%%%

% --- Diapositiva 4: Largo Plazo ---
\begin{frame}
\frametitle{II. El Largo Plazo: Ajuste de Capital (HO Modificado)}

\begin{block}{Ajuste de Capital (Long-Run)}
    \begin{itemize}
        \item Se permite que el Capital (\textit{K}) crezca para mantener el rendimiento constante.
        \item \textbf{Resultado Neto:} El efecto salarial promedio se neutraliza: \alert{\textbf{$+0.1\%$}}.
        \item \textbf{Implicación:} La economía (vía inversión) absorbe a los nuevos trabajadores.
    \end{itemize}
\end{block}

\end{frame}

%%%%%%%%%%%%%%%%%%%%%%%%%%%%%%%%%%%%%%%%%%%%%%%%%%%%%%%%%%
\section{Conclusiones Clave y Sustitución Imperfecta}
%%%%%%%%%%%%%%%%%%%%%%%%%%%%%%%%%%%%%%%%%%%%%%%%%%%%%%%%%%

% --- Diapositiva 5: Conclusión y Sustitución Imperfecta ---
\begin{frame}
\frametitle{III. Conclusiones: La Paradoja de la Sustitución}

\begin{columns}[T]
\begin{column}{0.5\textwidth}
\begin{block}{Sustitución Imperfecta}
    \begin{itemize}
        \item Los trabajadores nativos y extranjeros son sustitutos \textbf{imperfectos}.
        \item Realizan tareas complementarias.
    \end{itemize}
\end{block}
\end{column}

\begin{column}{0.5\textwidth}
\begin{itemize}
    \item \textbf{Nativos:} Aumento salarial promedio de \textbf{$+0.6\%$}.
    \item \textbf{Inmigrantes Previos:} Caída salarial de \alert{\textbf{$-6.4\%$}} (mayor competencia).
\end{itemize}
\vfill
\centering
\textbf{Política Clave: La inversión de capital mitiga la presión salarial.}
\end{column}
\end{columns}

\end{frame}

% --- Diapositiva 6: Preguntas ---
\begin{frame}
\centering
\Huge \textbf{¡Gracias por su atención!}
\vfill
\Large ¿Preguntas?
\end{frame}

\end{document}


