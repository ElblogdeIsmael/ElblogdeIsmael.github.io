
\documentclass{beamer}

% --- Paquetes Esenciales ---

\usepackage[utf8]{inputenc} % Acentos y caracteres en español
\usepackage[T1]{fontenc}
\usepackage[spanish]{babel} % Configuración para el idioma español
\usepackage{graphicx} % Para incluir imágenes (aunque no se usen en este caso)
\usepackage{amsmath} % Para fórmulas económicas
\usepackage{booktabs} % Para tablas profesionales

% --- Configuración del Tema de Beamer ---

\usetheme{Madrid} % Un tema popular y limpio.
\usecolortheme{default} % Esquema de color

% --- Metadatos de la Presentación (para la portada) ---

\title{Impacto de la Inmigración en los Salarios de EE. UU.}

\subtitle{Análisis en el Corto y Largo Plazo: 1990-2006}

\author{Profesor de Economía Internacional}

\institute{Universidad de Harvard}

\date{Seminario Avanzado de Comercio Internacional}

% Fuente de letra pequeña
\setbeamerfont{footnote}{size=\tiny}

%%%%%%%%%%%%%%%%%%%%%%%%%%%%%%%%%%%%%%%%%%%%%%%%%%%%%%%%%%
% DOCUMENTO
%%%%%%%%%%%%%%%%%%%%%%%%%%%%%%%%%%%%%%%%%%%%%%%%%%%%%%%%%%

\begin{document}

% --- Diapositiva 1: Portada ---

\begin{frame}
\titlepage
\end{frame}

% --- Diapositiva 2: Corto Plazo: Modelo de Factores Específicos ---

\begin{frame}
\frametitle{1. El Impacto a Corto Plazo (SFM)}

\begin{block}{\textbf{Modelo de Factores Específicos (SFM)}}
\begin{itemize}
    \item \textbf{Premisa Clave:} Capital y Tierra son \alert{Fijos}.
    \item \textbf{Impacto Salarial Promedio (1990-2006):} $\mathbf{-3.0\%}$
    \item \textbf{Efecto Distributivo:} Mayor pérdida en extremos de educación.
\end{itemize}
\end{block}

\vfill

\begin{figure}
\centering
\caption{Pérdida Salarial Estimada (SFM, 1990-2006)}
% \includegraphics[width=0.85\textwidth]{images/placeholder_chart_sfm.png}
\caption*{*La estimación inicial del SFM fija el capital, alimentando la retórica de competencia laboral.}
\end{figure}

\end{frame}

% --- Diapositiva 3: Largo Plazo: Ajuste de Capital (Modelo HO Ampliado) ---

\begin{frame}
\frametitle{2. Ajuste a Largo Plazo y el Capital}

\begin{block}{\textbf{Ajuste del Capital y Equilibrio HO}}
\begin{itemize}
    \item \textbf{Ajuste Largo Plazo:} El Capital ($K$) es \alert{Móvil} y se ajusta (Rental constante).
    \item \textbf{Nuevo Impacto Salarial Promedio:} $\mathbf{+0.1\%}$ (Efecto casi nulo).
    \item \textbf{Mecanismo:} \textbf{Teorema de Rybczynski}.
    \begin{itemize}
        \item La economía absorbe la mano de obra expandiendo la producción del bien intensivo en ese factor.
    \end{itemize}
\end{itemize}
\end{block}

\vfill

\textbf{Ejemplo Relevante:} Expansión industrial en Miami post-Mariel (1980) para absorber mano de obra.

\end{frame}

% --- Diapositiva 4: Evidencia Empírica Avanzada: Sustitutos Imperfectos ---

\begin{frame}
\frametitle{3. La Competición Real: Sustitutos Imperfectos}

\begin{block}{\textbf{Reclasificación de Factores Laborales}}
\begin{itemize}
    \item \textbf{Nativos} y \textbf{Extranjeros} son \alert{Complementarios} (Sustitutos Imperfectos).
    \item \textbf{Ganancia Salarial Neta (Nativos):} $\mathbf{+0.6\%}$ (Promedio total).
    \item \textbf{Mayor Pérdida Salarial:} Inmigrantes ya establecidos ($\mathbf{-6.4\%}$).
\end{itemize}
\end{block}

\vfill

\textbf{Conclusión Clave:} La mayor competencia se da \textit{dentro} de la población inmigrante.

\end{frame}

% --- Diapositiva 5: Conclusión y Preguntas ---

\begin{frame}
\centering
\Huge \textbf{¡Gracias por su atención!}
\vfill
\Large ¿Preguntas?
\end{frame}

\end{document}
