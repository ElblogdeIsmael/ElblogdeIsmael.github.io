\documentclass{beamer}

% --- Paquetes Esenciales ---

\usepackage[utf8]{inputenc} % Acentos y caracteres en español
\usepackage[T1]{fontenc}
\usepackage[spanish]{babel} % Configuración para el idioma español
\usepackage{graphicx} % Para incluir imágenes
\usepackage{amsmath} % Para fórmulas económicas
\usepackage{xcolor} % Para colores

% --- Configuración del Tema de Beamer ---

\usetheme{Madrid} % Un tema popular y limpio.
\usecolortheme{default} % Esquema de color

% --- Metadatos de la Presentación (para la portada) ---

\title{La Economía Política de la Migración}

\subtitle{Inmigración en Estados Unidos: Efectos de Corto y Largo Plazo}

\author{Profesor de Economía Internacional}

\institute{Harvard University}

\date{\today}

% Fuente de letra pequeña
\setbeamerfont{footnote}{size=\tiny}

%%%%%%%%%%%%%%%%%%%%%%%%%%%%%%%%%%%%%%%%%%%%%%%%%%%%%%%%%%
% DOCUMENTO
%%%%%%%%%%%%%%%%%%%%%%%%%%%%%%%%%%%%%%%%%%%%%%%%%%%%%%%%%%

\begin{document}

% --- Diapositiva 1: Portada ---
\begin{frame}
\titlepage
\end{frame}

% --- Diapositiva 2: Índice (se genera automáticamente) ---
\begin{frame}
\frametitle{Contenido de la Sesión}
\tableofcontents % Muestra las secciones como un índice
\end{frame}

%%%%%%%%%%%%%%%%%%%%%%%%%%%%%%%%%%%%%%%%%%%%%%%%%%%%%%%%%%
\section{Inmigración y Competencia Laboral}
%%%%%%%%%%%%%%%%%%%%%%%%%%%%%%%%%%%%%%%%%%%%%%%%%%%%%%%%%%

% --- Diapositiva 3: Perfil Educativo ---
\begin{frame}
\frametitle{Perfil Educativo y Competencia (2017)}
\begin{block}{\textbf{Concentración en Extremos Educativos}}
\begin{itemize}
\item \textbf{13.7\%} de la población de EE. UU. es nacida en el extranjero.
\item \alert{Trabajadores No Calificados:} Cerca del $\mathbf{40\%}$ de los que tienen $\mathbf{< 12}$ años de educación son extranjeros.
\item \alert{Trabajadores Altamente Calificados:} Más del $\mathbf{30\%}$ de los trabajadores con Doctorados son extranjeros.
\end{itemize}
\vfill
\end{block}

\begin{alertblock}{\textbf{Implicación Económica}}
La competencia directa se focaliza en los nichos de muy alta y muy baja calificación laboral, y menos en los niveles medios de educación.
\end{alertblock}
\end{frame}

% --- Diapositiva 4: Ganadores y Perdedores (Corto Plazo) ---
\begin{frame}
\frametitle{Impacto a Corto Plazo: Modelo de Factores Específicos (SFM)}
\begin{block}{\textbf{Asimetría en la Distribución de Renta}}
\begin{itemize}
\item \textbf{Ganadores:} Factores específicos no laborales (Capital y Tierra).
\begin{itemize}
\item Ej. \textbf{Agricultores} (Visa H-2A) y Dueños de empresas \textbf{High-Tech} (Visa H-1B).
\item Se benefician de la \alert{reducción del coste de la mano de obra}.
\end{itemize}
\item \textbf{Perdedores:} Mano de obra nativa no calificada.
\begin{itemize}
\item Riesgo: \textbf{Depresión salarial} por alta competencia en el segmento de baja educación.
\end{itemize}
\end{itemize}
\end{block}
\end{frame}

%%%%%%%%%%%%%%%%%%%%%%%%%%%%%%%%%%%%%%%%%%%%%%%%%%%%%%%%%%
\section{Ajuste a Largo Plazo y Economía Política}
%%%%%%%%%%%%%%%%%%%%%%%%%%%%%%%%%%%%%%%%%%%%%%%%%%%%%%%%%%

% --- Diapositiva 5: Evidencia Empírica y Modelos de Largo Plazo ---
\begin{frame}
\frametitle{Ajuste a Largo Plazo: Evidencia y Mecanismos}
\begin{block}{\textbf{Impacto Salarial Promedio}}
\begin{itemize}
\item \textbf{Estimación Inicial (Corto Plazo):} Caída salarial promedio total de $\mathbf{-3.0\%}$ (1990–2006).
\item \textbf{Ajuste (Largo Plazo, HO):} Impacto salarial promedio total casi \textbf{nulo} ($\mathbf{+0.1\%}$), debido al ajuste del capital.
\end{itemize}
\end{block}

\begin{alertblock}{\textbf{Teorema de Rybczynski y Sustitución}}
\begin{itemize}
\item \textbf{Mecanismo Clave:} La economía se ajusta expandiendo la producción del bien intensivo en el factor abundante (trabajo).
\item \textbf{Realidad del Mercado:} Inmigrantes como \textbf{sustitutos imperfectos} de los nativos.
\item \textbf{Efecto Salarial Real:} El salario nativo puede \alert{aumentar} ($\mathbf{+0.6\%}$). La \textbf{mayor pérdida} se concentra en inmigrantes ya establecidos ($\mathbf{-6.4\%}$).
\end{itemize}
\end{alertblock}
\end{frame}


% --- Diapositiva 6: Preguntas ---
\begin{frame}
\centering
\Huge \textbf{¡Gracias por su atención!}
\vfill
\Large Preguntas y Análisis
\end{frame}

\end{document}
