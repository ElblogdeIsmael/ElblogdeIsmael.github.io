%%%%%%%%%%%%%%%%%%%%%%%%%%%%%%%%%%%%%%%%%%%%%%%%%%%%%%%%%%
% PREÁMBULO
%%%%%%%%%%%%%%%%%%%%%%%%%%%%%%%%%%%%%%%%%%%%%%%%%%%%%%%%%%

\documentclass{beamer}

% --- Paquetes Esenciales ---
\usepackage[utf8]{inputenc} % Acentos y caracteres en español
\usepackage[T1]{fontenc}
\usepackage[spanish]{babel} % Configuración para el idioma español
\usepackage{graphicx} % Para incluir imágenes
\usepackage{amsmath} % Para fórmulas económicas

% --- Configuración del Tema de Beamer ---
\usetheme{Madrid} % Un tema popular y limpio.
\usecolortheme{default} % Esquema de color 

% --- Metadatos de la Presentación (para la portada) ---
\title{The Political Economy of Migration}
\subtitle{Inmigration Into 
the United States}
\author{Ismael Sallami Moreno}
\institute{UGR}
\date{\today}

% Fuente de letra pequeña
\setbeamerfont{footnote}{size=\tiny}

%%%%%%%%%%%%%%%%%%%%%%%%%%%%%%%%%%%%%%%%%%%%%%%%%%%%%%%%%%
% DOCUMENTO
%%%%%%%%%%%%%%%%%%%%%%%%%%%%%%%%%%%%%%%%%%%%%%%%%%%%%%%%%%

\begin{document}

% --- Diapositiva 1: Portada ---
\begin{frame}
\titlepage % Muestra la portada con la información definida arriba 
\end{frame}

% --- Diapositiva 2: Índice (se genera automáticamente) ---
\begin{frame}
\frametitle{Contenido de la Sesión}
\tableofcontents % Muestra las secciones como un índice 
\end{frame}

%%%%%%%%%%%%%%%%%%%%%%%%%%%%%%%%%%%%%%%%%%%%%%%%%%%%%%%%%%
\section{Inmigración en EE. UU.: Estructura y Costos a Corto Plazo}
%%%%%%%%%%%%%%%%%%%%%%%%%%%%%%%%%%%%%%%%%%%%%%%%%%%%%%%%%%

% --- Diapositiva 3: Perfil Educativo ---
\begin{frame}
\frametitle{Perfil Educativo de Inmigrantes (2017)}

\begin{block}{\textbf{Perfil Educativo de Inmigrantes (2017)}}
\begin{itemize}
    \item \textbf{13.7\%} de la población de EE. UU. es nacida en el extranjero .
    \item \alert{Competencia:} Alta concentración en extremos educativos:
    \begin{enumerate}
        \item Trabajadores con $\mathbf{< 12}$ años de educación ($\approx \mathbf{40\%}$ extranjeros) .
        \item Trabajadores con Doctorados ($\mathbf{> 30\%}$ extranjeros) .
    \end{enumerate}
\end{itemize}
\end{block}



\end{frame}

\begin{frame}
\frametitle{Distribución Geográfica de Inmigrantes}

\begin{figure}
    \centering
    \includegraphics[width=0.8\textwidth]{images/img-2-15.png} % Reemplaza con la ruta de tu imagen
    %\caption{Distribución geográfica de inmigrantes en EE. UU. (2017).}
    \label{fig:geographic_distribution}
\end{figure}

\end{frame}

% --- Diapositiva 4: Ganadores y Perdedores ---
\begin{frame}
\frametitle{Ganadores y Perdedores (Corto Plazo)}

\begin{block}{\textbf{Ganadores y Perdedores (Corto Plazo)}}
El Modelo de Factores Específicos (SFM) predice que la movilidad de mano de obra impacta los ingresos de los factores específicos:
\begin{itemize}
    \item \textbf{Ganadores (Específicos a la Exportación):} Capitalistas y Dueños de Tierra.
    \begin{itemize}
        \item \textit{Ejemplos:} Agricultores (Visa H-2A) y Dueños de empresas \textit{High-Tech} (Visa H-1B) se benefician del menor costo de la mano de obra.
    \end{itemize}
    \item \textbf{Perdedores (Mano de Obra Nativa):} Trabajadores no calificados.
    \begin{itemize}
        \item \textit{Riesgo:} Depresión salarial en nichos de alta competencia, como la mano de obra con bajo nivel educativo.
    \end{itemize}
\end{itemize}
\end{block}

\end{frame}

%%%%%%%%%%%%%%%%%%%%%%%%%%%%%%%%%%%%%%%%%%%%%%%%%%%%%%%%%%
\section{Impacto Salarial: Largo Plazo y Economía Política}
%%%%%%%%%%%%%%%%%%%%%%%%%%%%%%%%%%%%%%%%%%%%%%%%%%%%%%%%%%

% --- Diapositiva 4: Evidencia Empírica y Modelos de Largo Plazo ---
\begin{frame}
\frametitle{Evidencia Empírica: Modelos de Corto vs. Largo Plazo}

\begin{itemize}
    \item \textbf{Impacto a Corto Plazo (1990–2006, SFM):}
    \begin{itemize}
        \item Estimaciones iniciales sugieren una \alert{caída salarial promedio} para todos los trabajadores del \textbf{$-3.0\%$}.
        \item Este resultado alimenta la retórica de la competencia laboral y la oposición a la inmigración.
    \end{itemize}
    \item \textbf{Ajuste a Largo Plazo (Modelo HO Ampliado):}
    \begin{itemize}
        \item Permitiendo el ajuste del capital, el impacto salarial promedio total es casi \textbf{nulo} ($\mathbf{+0.1\%}$).
        \item \textit{Mecanismo Clave:} \textbf{Teorema de Rybczynski} - La economía absorbe la mano de obra entrante al expandir la producción del bien intensivo en ese factor (ej. industria textil en Miami post-Mariel).
    \end{itemize}
    \item \textbf{Conclusión de la Competencia (Sustitutos Imperfectos):}
    \begin{itemize}
        \item Los inmigrantes se complementan con los trabajadores nativos (sustitutos imperfectos) $\rightarrow$ Salario nativo aumenta $\mathbf{+0.6\%}$ .
        \item La \textbf{mayor pérdida salarial} se concentra en los \textbf{inmigrantes ya establecidos} ($\mathbf{-6.4\%}$).
    \end{itemize}
\end{itemize}
\vfill

\end{frame}

% --- Diapositiva 5: Preguntas ---
\begin{frame}
\centering
\Huge \textbf{¡Gracias por su atención!}
\vfill
\Large ¿Preguntas?
\end{frame}

\end{document}