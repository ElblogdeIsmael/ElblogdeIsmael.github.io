%%%%%%%%%%%%%%%%%%%%%%%%%%%%%%%%%%%%%%%%%%%%%%%%%%%%%%%%%%
% PREÁMBULO
%%%%%%%%%%%%%%%%%%%%%%%%%%%%%%%%%%%%%%%%%%%%%%%%%%%%%%%%%%

\documentclass{beamer}

% --- Paquetes Esenciales ---
\usepackage[utf8]{inputenc} 
\usepackage[T1]{fontenc}
\usepackage[spanish]{babel} 
\usepackage{graphicx} 
\usepackage{amsmath} 
\usepackage{tikz}

% --- Configuración del Tema de Beamer ---
\usetheme{Madrid} 
\usecolortheme{default} 

% --- Metadatos de la Presentación (para la portada) ---
\title{Primeros Años de Tipos Flexibles}
\subtitle{Análisis Macroeconómico del Periodo 1973-1990}
\author{Ismael Sallami Moreno}
\institute{UGR}
\date{\today}

% Fuente de letra pequeña
\setbeamerfont{footnote}{size=\tiny}

%%%%%%%%%%%%%%%%%%%%%%%%%%%%%%%%%%%%%%%%%%%%%%%%%%%%%%%%%%
% DOCUMENTO
%%%%%%%%%%%%%%%%%%%%%%%%%%%%%%%%%%%%%%%%%%%%%%%%%%%%%%%%%%

\begin{document}

% --- Diapositiva 1: Portada ---
\begin{frame}
\titlepage 
\end{frame}

% --- Diapositiva 2: Transición y Shock Inicial ---
\section{El Inicio de la Flotación (1973)}
\begin{frame}
\frametitle{El Final de Bretton Woods y el Primer Shock}

\begin{itemize}
    \item \textbf{Fin del Sistema Fijo:} El sistema de tipos de cambio fijos colapsa en \alert{marzo de 1973}.
    \item \textbf{Catalizador:} Flujos de capital especulativos incontrolables.
    \item \textbf{Shock Macroeconómico:} La \textbf{Cuadruplicación del Precio del Petróleo} (1973-74).
\end{itemize}

\vfill
\centering
\textbf{Resultado Inmediato:} Inicio de una turbulenta era de \textbf{Tipos Flexibles Intervenidos}.

\end{frame}

% --- Diapositiva 3: Estanflación y Desinflación ---
\section{La Década de la Estanflación (1973-1982)}
\begin{frame}
\frametitle{Estanflación y Respuestas de Política Monetaria}

\begin{block}{\textbf{El Fenómeno de la Estanflación}}
\begin{itemize}
    \item Combinación de \alert{alto desempleo} y \textbf{elevada inflación}.
    \item Causas: Choque de oferta (petróleo) y \textbf{expectativas inflacionarias} persistentes.
\end{itemize}
\end{block}

\vfill

\begin{itemize}
    \item \textbf{Respuesta Política:} Gobiernos adoptan políticas expansivas iniciales.
    \item \textbf{Consecuencia:} La inflación se agrava, llevando a la necesidad de políticas contractivas (post-1979) y una recesión mundial.
\end{itemize}

\end{frame}

% --- Diapositiva 4: La Gran Apreciación del Dólar (1980s) ---
\section{Desajustes Cambiarios (Misalignments)}
\begin{frame}
\frametitle{El Dólar y el Mix de Políticas de los Años 80}

\begin{itemize}
    \item \textbf{Contexto:} Políticas fiscales expansivas (déficits) y políticas monetarias contractivas (altos intereses).
    \item \textbf{Efecto en E:} Fuerte \alert{apreciación del dólar} (1981-1985).
    \item \textbf{Implicación:} Ejemplo de un persistente \textbf{Desajuste del Tipo de Cambio (Misalignment)}.
\end{itemize}

\vfill
\centering
\textbf{Lección:} Los tipos flexibles no impidieron grandes alejamientos del equilibrio externo.

\end{frame}

% --- Diapositiva 5: Lecciones Clave ---
\section{Conclusiones del Periodo}
\begin{frame}
\frametitle{Balance del Sistema Flexible (1973-1990)}

\begin{itemize}
    \item \textbf{Volatilidad:} Los tipos de cambio fueron \alert{más volátiles} de lo que los defensores esperaban.
    \item \textbf{No Aislamiento:} Las crisis económicas se transmitieron entre países (ej. recesión mundial de 1981-82).
    \item \textbf{Simetría Incompleta:} El dólar mantuvo su papel central, conservando ciertas asimetrías.
\end{itemize}

\end{frame}

% --- Diapositiva 5: Preguntas ---
\begin{frame}
\centering
\Huge \textbf{¡Gracias por su atención!}
\vfill
\Large ¿Preguntas?
\end{frame}

\end{document}