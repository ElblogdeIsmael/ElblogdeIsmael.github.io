%%%%%%%%%%%%%%%%%%%%%%%%%%%%%%%%%%%%%%%%%%%%%%%%%%%%%%%%%%
% PREÁMBULO
%%%%%%%%%%%%%%%%%%%%%%%%%%%%%%%%%%%%%%%%%%%%%%%%%%%%%%%%%%

\documentclass{beamer}

% --- Paquetes Esenciales ---
\usepackage[utf8]{inputenc} 
\usepackage[T1]{fontenc}
\usepackage[spanish]{babel} 
\usepackage{graphicx} 
\usepackage{amsmath} 
\usepackage{tikz}

% --- Configuración del Tema de Beamer ---
\usetheme{Madrid} 
\usecolortheme{default} 

% --- Metadatos de la Presentación (para la portada) ---
\title{Crisis de Tipo de Cambio}
\subtitle{El Estudio del Caso: Timeline of Events (Feenstra \& Taylor)}
\author{Ismael Sallami Moreno}
\institute{UGR}
\date{\today}

% Fuente de letra pequeña
\setbeamerfont{footnote}{size=\tiny}

%%%%%%%%%%%%%%%%%%%%%%%%%%%%%%%%%%%%%%%%%%%%%%%%%%%%%%%%%%
% DOCUMENTO
%%%%%%%%%%%%%%%%%%%%%%%%%%%%%%%%%%%%%%%%%%%%%%%%%%%%%%%%%%

\begin{document}

% --- Diapositiva 1: Portada ---
\begin{frame}
\titlepage 
\end{frame}

% --- Diapositiva 2: Contexto de la Crisis ---
\section{Vulnerabilidad y Régimen Cambiario}
\begin{frame}
\frametitle{El Contexto Pre-Crisis}

\begin{itemize}
    \item \textbf{Compromiso Fijo:} El país mantiene un tipo de cambio fijo o semifijo (el \alert{ancla nominal}).
    \item \textbf{Integración Financiera:} Libre movilidad de capitales ($\mathbf{K}$).
    \item \textbf{Fricción:} El Banco Central intenta mantener la autonomía monetaria ($i$).
\end{itemize}

\vfill
\centering
\textbf{Resultado:} El régimen es estructuralmente vulnerable ante \textbf{shocks} o \textbf{expectativas negativas}.

\end{frame}

% --- Diapositiva 3: El Conflicto del Trilema ---
\section{La Restricción de la Política: El Trilema}
\begin{frame}
\frametitle{El Trilema: El Conflicto Teórico}

\begin{block}{\textbf{Sólo Dos de Tres (Feenstra y Taylor)}}
\begin{enumerate}
    \item Libre Movilidad de Capitales.
    \item Tipo de Cambio Fijo.
    \item \alert{Política Monetaria Autónoma}.
\end{enumerate}
\end{block}

\vfill
\begin{itemize}
    \item \textbf{Implicación:} Si el país mantiene $\mathbf{K}$ libre y $E$ fijo, \textbf{pierde} la capacidad de fijar su tasa de interés ($i$).
    \item \textbf{El Cronograma (Timeline):} La secuencia de eventos muestra el momento en que este conflicto se vuelve \alert{insostenible}.
\end{itemize}

\end{frame}

% --- Diapositiva 4: Dinámica del Colapso ---
\section{La Crisis: Secuencia de Eventos Clave}
\begin{frame}
\frametitle{Dinámica de la Fuga de Capitales}

\begin{enumerate}
    \item \textbf{Desconfianza:} Las expectativas cambian, anticipando una devaluación.
    \item \textbf{Fuga de Capitales:} Inversores venden activos domésticos por divisas.
    \item \textbf{Defensa de la Paridad:} El Banco Central gasta \alert{Reservas Internacionales} para comprar su propia moneda.
    \item \textbf{Punto de No Retorno:} Agotamiento de reservas (o costo político insostenible de $i \uparrow$).
    \item \textbf{Colapso:} Devaluación o flotación forzosa del tipo de cambio.
\end{enumerate}

\vfill
\centering
\textbf{Efecto:} El tipo de cambio \alert{sobrerreacciona (Overshooting)}, aumentando la volatilidad.
\end{frame}

% --- Diapositiva 5: Lecciones Clave ---
\section{Conclusiones y Lecciones}
\begin{frame}
\frametitle{Lecciones sobre la Elección del Régimen}

\begin{block}{\textbf{Síntesis Académica}}
\begin{itemize}
    \item El ''Timeline'' demuestra la \textbf{inviabilidad a largo plazo} de regímenes fijos con alta $\mathbf{K}$.
    \item La crisis es un \alert{ajuste forzoso} a las restricciones impuestas por el Trilema.
    \item La lección principal es la necesidad de elegir un \textbf{régimen cambiario coherente} (fijo versus flotante) que se ajuste al grado de integración financiera del país.
\end{itemize}
\end{block}

\end{frame}

% --- Diapositiva 5: Preguntas ---
\begin{frame}
\centering
\Huge \textbf{¡Gracias por su atención!}
\vfill
\Large ¿Preguntas?
\end{frame}



\end{document}
