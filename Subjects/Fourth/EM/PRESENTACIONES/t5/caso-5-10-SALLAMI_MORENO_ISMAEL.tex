%%%%%%%%%%%%%%%%%%%%%%%%%%%%%%%%%%%%%%%%%%%%%%%%%%%%%%%%%%
% PREÁMBULO
%%%%%%%%%%%%%%%%%%%%%%%%%%%%%%%%%%%%%%%%%%%%%%%%%%%%%%%%%%

\documentclass{beamer}

% --- Paquetes Esenciales ---
\usepackage[utf8]{inputenc} 
\usepackage[T1]{fontenc}
\usepackage[spanish]{babel} 
\usepackage{graphicx} 
\usepackage{amsmath} 
\usepackage{tikz}

% --- Configuración del Tema de Beamer ---
\usetheme{Madrid} 
\usecolortheme{default} 

% --- Metadatos de la Presentación (para la portada) ---
\title{Anclas Nominales: Teoría y Práctica}
\subtitle{Garantizando la Estabilidad a Largo Plazo}
\author{Ismael Sallami Moreno}
\institute{UGR}
\date{\today}

% Fuente de letra pequeña
\setbeamerfont{footnote}{size=\tiny}

%%%%%%%%%%%%%%%%%%%%%%%%%%%%%%%%%%%%%%%%%%%%%%%%%%%%%%%%%%
% DOCUMENTO
%%%%%%%%%%%%%%%%%%%%%%%%%%%%%%%%%%%%%%%%%%%%%%%%%%%%%%%%%%

\begin{document}

% --- Diapositiva 1: Portada ---
\begin{frame}
\titlepage 
\end{frame}

% --- Diapositiva 2: El Concepto de Ancla Nominal ---
\section{El Ancla Nominal en el Largo Plazo}
\begin{frame}
\frametitle{Definición y Rol Macroeconómico}

\begin{itemize}
    \item \textbf{Propósito Central:} Un ancla nominal es un compromiso de política que garantiza la \alert{estabilidad del nivel de precios} en el largo plazo.
    \item \textbf{Neutralidad del Dinero:} En el L/P, los modelos macroeconómicos (como el de Paridad de Poder Adquisitivo) predicen que los cambios en la oferta de dinero afectan solo a las variables nominales.
    \item \textbf{Establecer Expectativas:} El ancla nominal es crucial para fijar las \textbf{expectativas racionales} de inflación, influenciando así las decisiones de inversión y precios.
\end{itemize}

\vfill

\begin{block}{\textbf{Marco de Precios (Largo Plazo)}}
$$
P = E_{\$/\text{¥}} \cdot P^* \cdot \frac{M^s}{M^{*s}} \quad \text{(Basado en PPP)}
$$
\centering
\text{El ancla controla la trayectoria de P y $E_{\$/\text{¥}}$}
\end{block}

\end{frame}

% --- Diapositiva 3: Tipos de Anclas en la Práctica ---
\section{Estrategias y Ejemplos de Anclas}
\begin{frame}
\frametitle{Opciones de Anclaje y Costos de Oportunidad}

\begin{itemize}
    \item \textbf{1. Ancla de Tipo de Cambio Fijo (Fixed Exchange Rate):}
    \begin{itemize}
        \item Se compromete a mantener la divisa a una tasa fija respecto a otra (ej. \textbf{Perú} en ciertos momentos).
        \item Beneficio: Importa la credibilidad del país ancla (ej. EE. UU.).
    \end{itemize}
    \item \textbf{2. Metas de Inflación (Inflation Targeting):}
    \begin{itemize}
        \item El Banco Central anuncia un rango objetivo de inflación (ej. \textbf{BCE}, \textbf{FED}).
        \item Beneficio: Prioriza el control de precios internos y mantiene la \alert{autonomía monetaria}.
    \end{itemize}
\end{itemize}

\vfill
\centering
\textbf{La elección implica un trade-off fundamental en la política.}

\end{frame}

% --- Diapositiva 4: El Trilema y el Ancla Nominal ---
\section{Restricciones de Política y el Trilema}
\begin{frame}
\frametitle{Ancla Nominal y el Trilema (Imposible Trinity)}

\begin{itemize}
    \item \textbf{El Trilema:} Un país solo puede lograr dos de tres objetivos simultáneamente:
    \begin{enumerate}
        \item Autonomía de la Política Monetaria.
        \item Tipo de Cambio Fijo.
        \item \alert{Libre Movilidad de Capitales (Integración Financiera)}.
    \end{enumerate}
    \item \textbf{Consecuencia del Ancla Fija:} Si un país elige un ancla de Tipo de Cambio Fijo junto con la integración financiera, debe \textbf{sacrificar su autonomía monetaria} (no puede fijar $i_t$).
    \item \textbf{Consecuencia del Ancla Flexible:} Un ancla de Metas de Inflación permite la autonomía, pero requiere un \alert{tipo de cambio flotante} para absorber shocks.
\end{itemize}

\end{frame}

% --- Diapositiva 5: Conclusión ---
\begin{frame}
\frametitle{Conclusiones Clave}

\begin{block}{\textbf{El Rol de la Credibilidad}}
\begin{itemize}
    \item El ancla nominal es vital para el \textbf{equilibrio a largo plazo} de los precios.
    \item La elección del ancla (fijo vs. flexible) define las \alert{restricciones de política} a corto plazo impuestas por el Trilema.
    \item La \textbf{coherencia} y la \textbf{credibilidad} del Banco Central son tan importantes como el ancla elegida.
\end{itemize}
\end{block}


\end{frame}


% --- Diapositiva 5: Preguntas ---
\begin{frame}
\centering
\Huge \textbf{¡Gracias por su atención!}
\vfill
\Large ¿Preguntas?
\end{frame}

\end{document}
