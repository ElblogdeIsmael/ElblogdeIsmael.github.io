%%%%%%%%%%%%%%%%%%%%%%%%%%%%%%%%%%%%%%%%%%%%%%%%%%%%%%%%%%
% PREÁMBULO
%%%%%%%%%%%%%%%%%%%%%%%%%%%%%%%%%%%%%%%%%%%%%%%%%%%%%%%%%%

\documentclass{beamer}

% --- Paquetes Esenciales ---
\usepackage[utf8]{inputenc} 
\usepackage[T1]{fontenc}
\usepackage[spanish]{babel} 
\usepackage{graphicx} 
\usepackage{amsmath} 
\usepackage{tikz}

% --- Configuración del Tema de Beamer ---
\usetheme{Madrid} 
\usecolortheme{default} 

% --- Metadatos de la Presentación (para la portada) ---
\title{Activos Financieros Internacionales}
\subtitle{Análisis de Atributos Clave y Flujos de Capital}
\author{Ismael Sallami Moreno}
\institute{UGR}
\date{\today}

% Fuente de letra pequeña
\setbeamerfont{footnote}{size=\tiny}

%%%%%%%%%%%%%%%%%%%%%%%%%%%%%%%%%%%%%%%%%%%%%%%%%%%%%%%%%%
% DOCUMENTO
%%%%%%%%%%%%%%%%%%%%%%%%%%%%%%%%%%%%%%%%%%%%%%%%%%%%%%%%%%

\begin{document}

% --- Diapositiva 1: Portada ---
\begin{frame}
\titlepage 
\end{frame}

% --- Diapositiva 2: Los Activos en la Contabilidad Internacional ---
\section{Definición y Flujos de Capital}
\begin{frame}
\frametitle{Activos: Vehículos de Inversión Global}

\begin{itemize}
    \item \textbf{Definición:} Un activo financiero representa un \alert{derecho a un pago futuro} por parte del emisor (gobierno o corporación).
    \item \textbf{Tipos Comunes:}
    \begin{enumerate}
        \item Depósitos Bancarios (alta liquidez).
        \item Bonos Soberanos y Corporativos (renta fija).
        \item Acciones (renta variable).
    \end{enumerate}
    \item \textbf{Importancia Macroeconómica:} Estos activos son la vía por la cual se registra la \alert{Cuenta Financiera} de la Balanza de Pagos.
\end{itemize}

\vfill
\centering
\textbf{Analogía:} Los mercados de activos son el mecanismo de transmisión de la integración financiera global.
\end{frame}

% --- Diapositiva 3: Los Tres Atributos Fundamentales ---
\section{Determinantes de la Demanda de Activos}
\begin{frame}
\frametitle{Los Tres Atributos Clave}

\begin{block}{\textbf{¿Qué buscan los Inversores?}}
\begin{enumerate}
    \item \textbf{1. Rendimiento Esperado (\(i\)):} Tasa de interés o rentabilidad anticipada del activo. Base para la \textbf{Paridad de Intereses} (UIP).
    \item \textbf{2. Riesgo:} La incertidumbre sobre el rendimiento real. Incluye riesgo de incumplimiento (\alert{default}) y \textbf{riesgo cambiario}.
    \item \textbf{3. Liquidez:} Facilidad y rapidez con la que el activo puede ser convertido en efectivo sin pérdida de valor significativa.
\end{enumerate}
\end{block}

\vfill
\centering
\textbf{Decisión del Inversor:} Maximizar el rendimiento esperado para un nivel dado de riesgo y liquidez.

\end{frame}

% --- Diapositiva 4: Análisis del Riesgo Internacional ---
\section{Riesgo y Prima de Riesgo (Risk Premium)}
\begin{frame}
\frametitle{Riesgo: La Mayor Fricción del Mercado}

\begin{itemize}
    \item \textbf{Riesgo Cambiario (\(\sigma_S\)):} La variabilidad del tipo de cambio futuro esperado ($E_{t} S_{t+1}$). Si dos activos son sustitutos imperfectos, esto exige compensación.
    \item \textbf{Riesgo de Incumplimiento (Soberano):} Probabilidad de que el emisor (ej. un país) no pague. Requiere una \alert{prima de riesgo país} ($RP$).
\end{itemize}

\vfill
\begin{figure}
\centering
\begin{itemize}
    \item[] $i_{t} = i^{*}_{t} + \text{Depreciación Esperada} + \textbf{RP}$ 
\end{itemize}
\caption{Ecuación de equilibrio con prima de riesgo. La prima de riesgo es el precio del riesgo percibido en el activo internacional.}
\end{figure}

\end{frame}

% --- Diapositiva 5: Conclusión y Síntesis ---
\begin{frame}
\frametitle{Conclusiones: Activos y Asignación de Capital}

\begin{block}{\textbf{Síntesis Académica}}
\begin{itemize}
    \item La asignación de activos a nivel global depende de un complejo \textbf{balance entre rendimiento, riesgo y liquidez}.
    \item La presencia de \alert{riesgos no cubiertos} (prima de riesgo) es lo que permite que los rendimientos entre países no se igualen instantáneamente.
    \item Los atributos de los activos explican por qué los flujos de capital no son uniformes y por qué la \textbf{integración financiera} es incompleta.
\end{itemize}
\end{block}

\end{frame}

% --- Diapositiva 5: Preguntas ---
\begin{frame}
\centering
\Huge \textbf{¡Gracias por su atención!}
\vfill
\Large ¿Preguntas?
\end{frame}

\end{document}
