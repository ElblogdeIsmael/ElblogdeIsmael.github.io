%%%%%%%%%%%%%%%%%%%%%%%%%%%%%%%%%%%%%%%%%%%%%%%%%%%%%%%%%%
% PREÁMBULO
%%%%%%%%%%%%%%%%%%%%%%%%%%%%%%%%%%%%%%%%%%%%%%%%%%%%%%%%%%

\documentclass{beamer}

% --- Paquetes Esenciales ---
\usepackage[utf8]{inputenc} 
\usepackage[T1]{fontenc}
\usepackage[spanish]{babel} 
\usepackage{graphicx} 
\usepackage{amsmath} 

% --- Configuración del Tema de Beamer ---
\usetheme{Madrid} 
\usecolortheme{default} 

% --- Metadatos de la Presentación (para la portada) ---
\title{Régimen Cambiario y Flujos Comerciales}
\subtitle{¿Promueven el Comercio los Tipos de Cambio Fijos?}
\author{Ismael Sallami Moreno}
\institute{UGR}
\date{\today}

% Fuente de letra pequeña
\setbeamerfont{footnote}{size=\tiny}

%%%%%%%%%%%%%%%%%%%%%%%%%%%%%%%%%%%%%%%%%%%%%%%%%%%%%%%%%%
% DOCUMENTO
%%%%%%%%%%%%%%%%%%%%%%%%%%%%%%%%%%%%%%%%%%%%%%%%%%%%%%%%%%

\begin{document}

% --- Diapositiva 1: Portada ---
\begin{frame}
\titlepage 
\end{frame}

% --- Diapositiva 2: Fundamento Económico del Tipo Fijo ---
\section{Volatilidad como Fricción Comercial}
\begin{frame}
\frametitle{Tipo de Cambio Fijo: Reducción de Costos}

\begin{itemize}
    \item \textbf{Incertidumbre como Costo:} La volatilidad cambiaria actúa como un \alert{costo de transacción} para los exportadores/importadores.
    \item \textbf{Riesgo de Contrato:} Afecta la rentabilidad esperada de los contratos a largo plazo (e.g., \textbf{Airbus} o \textbf{Boeing} que venden a futuro).
    \item \textbf{La Hipótesis Central:} Al fijar el tipo de cambio ($E_{t} = \bar{E}$), se elimina esta incertidumbre, reduciendo los costos de cobertura (hedging) y fomentando la actividad comercial.
\end{itemize}

\vfill
\centering
\textbf{Reducción de Riesgo} $\rightarrow$ \textbf{Menores Costos} $\rightarrow$ \textbf{Mayor Comercio}

\end{frame}

% --- Diapositiva 3: Evidencia y Modelos de Comercio ---
\section{Evidencia Empírica}
\begin{frame}
\frametitle{Efectos Cuantificables del Tipo Fijo}

\begin{block}{\textbf{El Tipo Fijo en el Modelo de Gravedad}}
\begin{itemize}
    \item La fijación cambiaria reduce las \textbf{barreras fronterizas ("Border Effects")}, actuando como una \alert{reducción de la distancia} efectiva entre socios.
    \item \textbf{Especialización Intra-Industrial:} Se espera que el tipo fijo impulse el comercio de variedades de productos diferenciados (Intra-Industry Trade).
    \item \textbf{Estimaciones Empíricas (Ejemplo Clásico):} Se sugiere que los tipos fijos pueden aumentar el comercio entre un \textbf{5\% y 20\%} comparado con los regímenes flotantes.
\end{itemize}
\end{block}

\vfill
\centering
\textbf{Dilema:} El tipo fijo aumenta el comercio, pero sacrifica \textbf{autonomía monetaria} (Trilema).

\end{frame}

% --- Diapositiva 4: Conclusiones ---
\begin{frame}
\frametitle{Conclusiones: El Trade-off en la Elección del Régimen}

\begin{itemize}
    \item \textbf{Ganancia Neta:} Los tipos de cambio fijos ofrecen una \alert{ganancia de eficiencia} al reducir el riesgo cambiario.
    \item \textbf{Fricciones vs. Flexibilidad:} Si bien la Macroeconomía Internacional aboga por la flexibilidad para absorber shocks, el Comercio Internacional resalta los beneficios de la estabilidad.
    \item \textbf{Implicación:} La elección del régimen ($E$ fijo vs. $E$ flotante) es un trade-off fundamental entre \textbf{facilitar el comercio} y \textbf{mantener la capacidad de respuesta a shocks internos}.
\end{itemize}


\end{frame}

% --- Diapositiva 5: Preguntas ---
\begin{frame}
\centering
\Huge \textbf{¡Gracias por su atención!}
\vfill
\Large ¿Preguntas?
\end{frame}

\end{document}
