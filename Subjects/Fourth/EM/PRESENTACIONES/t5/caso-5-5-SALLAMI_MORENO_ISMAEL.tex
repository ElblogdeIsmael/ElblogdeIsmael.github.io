%%%%%%%%%%%%%%%%%%%%%%%%%%%%%%%%%%%%%%%%%%%%%%%%%%%%%%%%%%
% PREÁMBULO
%%%%%%%%%%%%%%%%%%%%%%%%%%%%%%%%%%%%%%%%%%%%%%%%%%%%%%%%%%

\documentclass{beamer}

% --- Paquetes Esenciales ---
\usepackage[utf8]{inputenc} 
\usepackage[T1]{fontenc}
\usepackage[spanish]{babel} 
\usepackage{graphicx} 
\usepackage{amsmath} 
\usepackage{tikz}

% --- Configuración del Tema de Beamer ---
\usetheme{Madrid} 
\usecolortheme{default} 

% --- Metadatos de la Presentación (para la portada) ---
\title{Condición de Arbitraje No Cubierto de Tasas de Interés (UIP)}
\subtitle{Evidencia Empírica y Desafíos en Macroeconomía Internacional}
\author{Ismael Sallami Moreno}
\institute{UGR}
\date{\today}

% Fuente de letra pequeña
\setbeamerfont{footnote}{size=\tiny}

%%%%%%%%%%%%%%%%%%%%%%%%%%%%%%%%%%%%%%%%%%%%%%%%%%%%%%%%%%
% DOCUMENTO
%%%%%%%%%%%%%%%%%%%%%%%%%%%%%%%%%%%%%%%%%%%%%%%%%%%%%%%%%%

\begin{document}

% --- Diapositiva 1: Portada ---
\begin{frame}
\titlepage 
\end{frame}

% --- Diapositiva 2: Concepto Fundamental (UIP) ---
\section{Teoría de la Condición de Arbitraje No Cubierto de Tasas de Interés}
\begin{frame}
\frametitle{UIP: El Principio de Arbitraje Racional}

\begin{itemize}
    \item \textbf{Definición:} En un equilibrio sin arbitraje, el \alert{rendimiento esperado} de los depósitos en dos divisas diferentes debe ser idéntico cuando se mide en una moneda común.
    \item \textbf{Mecanismo Clave:} Los inversores son indiferentes entre mantener activos nacionales o extranjeros, siempre y cuando el riesgo sea despreciable.
\end{itemize}

\vfill

\begin{block}{\textbf{La Ecuación de la UIP}}
$$
i_t = i^*_t + E_{t} \left( \frac{S_{t+1}-S_{t}}{S_{t}} \right)
$$
\centering
\text{Tasa Doméstica = Tasa Extranjera + Depreciación Esperada}
\end{block}

\end{frame}

% --- Diapositiva 3: El Desafío Empírico del Caso 5.5 ---
\section{Evidencia y Desviaciones de la UIP}
\begin{frame}
\frametitle{Caso 5.5: ¿Se Sostiene la UIP en la Realidad?}

\begin{itemize}
    \item \textbf{La Falla Empírica:} La evidencia sugiere que la UIP \alert{no se cumple estrictamente} en el corto y medio plazo. Las correlaciones esperadas a menudo son débiles o tienen el signo incorrecto (el ''UIP Puzzle'').
    \item \textbf{Causas de la Desviación:}
    \begin{enumerate}
        \item \textbf{Prima de Riesgo (Risk Premium):} Los bonos extranjeros pueden ser percibidos como más riesgosos que los nacionales, exigiendo un rendimiento extra.
        \item \textbf{Expectativas Irracionales:} El tipo de cambio futuro esperado ($E_{t} S_{t+1}$) no siempre se forma racionalmente.
    \end{enumerate}
    \item \textbf{Ejemplo Clásico:} El fenómeno del \textbf{Carry Trade}, que explota estas desviaciones al tomar préstamos en divisas con baja tasa de interés ($i^*_t$) para invertir en divisas con alta tasa ($i_t$), asumiendo que el tipo de cambio no se moverá lo suficiente para compensar el diferencial.
\end{itemize}

\end{frame}

% --- Diapositiva 4: Implicaciones Macroeconómicas ---
\section{UIP y Estabilidad del Tipo de Cambio}
\begin{frame}
\frametitle{Consecuencias de las Fricciones de la UIP}

\begin{itemize}
    \item \textbf{Control de Política Monetaria:} La UIP es esencial para modelar cómo los bancos centrales utilizan la tasa de interés ($i_t$) para influir en el tipo de cambio ($S_t$) y la inversión.
    \item \textbf{Fricciones del Mercado:} Las primas de riesgo actúan como \alert{costos de transacción} que limitan el arbitraje inmediato y permiten la persistencia de diferenciales de rendimiento.
    \item \textbf{Contexto Global:} La volatilidad de los tipos de cambio observada es a menudo mayor de lo que predice la UIP.
\end{itemize}

\vfill
\centering
\textbf{Analogía:} El mercado de divisas no es una autopista perfectamente lisa (UIP), sino una carretera con peajes y curvas (prima de riesgo), donde la velocidad real de ajuste (el arbitraje) es siempre menor de lo que la teoría ideal predice.
\end{frame}

% --- Diapositiva 5: Preguntas ---
\begin{frame}
\centering
\Huge \textbf{¡Gracias por su atención!}
\vfill
\Large ¿Preguntas?
\end{frame}

\end{document}
