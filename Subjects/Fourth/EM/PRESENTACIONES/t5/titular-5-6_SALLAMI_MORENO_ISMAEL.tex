%%%%%%%%%%%%%%%%%%%%%%%%%%%%%%%%%%%%%%%%%%%%%%%%%%%%%%%%%%
% PREÁMBULO
%%%%%%%%%%%%%%%%%%%%%%%%%%%%%%%%%%%%%%%%%%%%%%%%%%%%%%%%%%

\documentclass{beamer}

% --- Paquetes Esenciales ---
\usepackage[utf8]{inputenc} 
\usepackage[T1]{fontenc}
\usepackage[spanish]{babel} 
\usepackage{graphicx} 
\usepackage{amsmath} 
\usepackage{tikz}

% --- Configuración del Tema de Beamer ---
\usetheme{Madrid} 
\usecolortheme{default} 

% --- Metadatos de la Presentación (para la portada) ---
\title{Overshooting en la Práctica}
\subtitle{Dinámica Corto y Largo Plazo del Tipo de Cambio}
\author{Ismael Sallami Moreno}
\institute{UGR}
\date{\today}

% Fuente de letra pequeña
\setbeamerfont{footnote}{size=\tiny}

%%%%%%%%%%%%%%%%%%%%%%%%%%%%%%%%%%%%%%%%%%%%%%%%%%%%%%%%%%
% DOCUMENTO
%%%%%%%%%%%%%%%%%%%%%%%%%%%%%%%%%%%%%%%%%%%%%%%%%%%%%%%%%%

\begin{document}

% --- Diapositiva 1: Portada ---
\begin{frame}
\titlepage 
\end{frame}

% --- Diapositiva 2: Concepto Fundamental ---
\section{El Fenómeno del Overshooting}
\begin{frame}
\frametitle{Overshooting: Sobreadaptación del Tipo de Cambio}

\begin{itemize}
    \item \textbf{Definición:} Respuesta \alert{inmediata y desproporcionada} del tipo de cambio (S) ante un shock monetario (e.g., aumento de la oferta monetaria, M).
    \item \textbf{Causa Principal:} La diferencia en la velocidad de ajuste de los mercados:
    \begin{enumerate}
        \item \textbf{Precios de Bienes (P):} Rígidos en el corto plazo.
        \item \textbf{Tipos de Cambio (S):} Perfectamente flexibles.
    \end{enumerate}
\end{itemize}

\vfill
\centering
\textbf{El Overshooting es una manifestación de la rigidez nominal en la economía abierta.}

\end{frame}

% --- Diapositiva 3: Mecanismos de Ajuste (SR vs LR) ---
\section{La Dinámica Corto Plazo}
\begin{frame}
\frametitle{UIP y Rigidez Nominal: La Lógica del Exceso}

\begin{block}{\textbf{Secuencia de un Shock Monetario (SR)}}
\begin{enumerate}
    \item \textbf{Shock (\(\Delta M \uparrow\)):} El aumento de M reduce la tasa de interés nominal interna (\(i \downarrow\)).
    \item \textbf{UIP:} Para que se cumpla la Condición de Arbitraje No Cubierto de Tasas de Interés (UIP), la divisa debe \alert{esperar} una apreciación futura.
    \item \textbf{El Sobreajuste:} Dado que el tipo de cambio de equilibrio a largo plazo (\(S_{LR}\)) cambia solo proporcionalmente a \(\Delta M\), el único modo de generar la expectativa de apreciación es que el tipo de cambio se deprecie \textbf{más de lo necesario} en el corto plazo (\(S_{SR} >> S_{LR}\)).
\end{enumerate}
\end{block}

\vfill
\centering
\textbf{LR:} \(S_{LR}\) (Tipo de cambio) aumenta proporcionalmente con \(M\). 

\textbf{SR:} \(S_{SR}\) debe sobrepasar a \(S_{LR}\) para que la trayectoria de convergencia cumpla con UIP.

\end{frame}

% --- Diapositiva 4: Impacto en la Práctica y Consecuencias ---
\section{Overshooting en la Práctica}
\begin{frame}
\frametitle{Volatilidad y Consecuencias Políticas}

\begin{itemize}
    \item \textbf{Evidencia:} Los tipos de cambio muestran \alert{mucha más volatilidad} de lo que predice la Paridad de Poder Adquisitivo (PPP) o la simple Teoría de Neutralidad Monetaria a corto plazo.
    \item \textbf{Impacto del Shock:} Un pequeño cambio en la política monetaria o en las expectativas puede traducirse en una \textbf{gran fluctuación} del tipo de cambio.
    \item \textbf{Consecuencias Macroeconómicas:}
    \begin{enumerate}
        \item Aumenta la incertidumbre para exportadores e inversores.
        \item Puede generar crisis de liquidez o inestabilidad comercial.
    \end{enumerate}
\end{itemize}

\vfill
\textbf{Metáfora:} El tipo de cambio es como un velocímetro hipersensible; cualquier cambio en el motor (política monetaria) hace que la aguja se dispare momentáneamente antes de asentarse en la nueva velocidad.

\end{frame}

% --- Diapositiva 5: Preguntas ---
\begin{frame}
\centering
\Huge \textbf{¡Gracias por su atención!}
\vfill
\Large ¿Preguntas?
\end{frame}

\end{document}
