%%%%%%%%%%%%%%%%%%%%%%%%%%%%%%%%%%%%%%%%%%%%%%%%%%%%%%%%%%

% PREÁMBULO

%%%%%%%%%%%%%%%%%%%%%%%%%%%%%%%%%%%%%%%%%%%%%%%%%%%%%%%%%%

\documentclass{beamer}

% --- Paquetes Esenciales ---

\usepackage[utf8]{inputenc} % Acentos y caracteres en español

\usepackage[T1]{fontenc}

\usepackage[spanish]{babel} % Configuración para el idioma español

\usepackage{graphicx} % Para incluir imágenes

\usepackage{amsmath} % Para fórmulas económicas

% --- Configuración del Tema de Beamer ---

\usetheme{Madrid} % Un tema popular y limpio.

\title[Fin de la Guerra Comercial]{La Guerra Comercial EE. UU.-China: ¿Un Final o una Tregua Inestable?}
\author{Ismael Sallami Moreno}
\institute{UGR}
\date{\today}

\begin{document}

% --- Diapositiva 1: Título ---
\begin{frame}
\titlepage
\end{frame}

% --- Diapositiva 2: Marco Teórico y el Dilema del Prisionero ---

\begin{frame}
\frametitle{El Juego de Aranceles: El Dilema del Prisionero}

\begin{block}{\textbf{Teoría: El Equilibrio de Nash en la Guerra Comercial}}
\begin{itemize}
    \item Países grandes (EE. UU., China) tienen incentivos individuales para imponer aranceles (ganancia por \alert{Términos de Intercambio}).
    \item Cuando \textbf{ambos} aplican aranceles óptimos, el resultado es un equilibrio de Nash subóptimo.
    \item Esto genera una \alert{Pérdida Neta de Bienestar} para ambas naciones y para el mundo.
\end{itemize}
\end{block}

\vfill

\begin{block}{\textbf{Consecuencia}}
\begin{itemize}
    \item La inestabilidad de la relación sin un arbitraje multilateral (OMC) conduce a \alert{aranceles mutuos persistentes}.
\end{itemize}
\end{block}

\end{frame}

% --- Diapositiva 3: La Resolución Temporal y el Comercio Gestionado ---

\begin{frame}

\frametitle{El Acuerdo de "Fase Uno" (Enero 2020)}

\begin{block}{\textbf{Componentes Clave de la Tregua}}
\begin{itemize}
    \item Objetivo: Poner fin al conflicto arancelario de 2018-2019.
    \item Mecanismo Central: \alert{Comercio Gestionado} (\textit{Managed Trade}).
    \item Compromiso de China: Aumentar las compras de EE. UU. en $\mathbf{200}\ \text{mil millones}$ para 2021 (sobre la base de 2017).
\end{itemize}
\end{block}

\vfill

\begin{block}{\textbf{El Problema Pendiente}}
\begin{itemize}
    \item El acuerdo mantuvo \alert{altos aranceles} en vigor en muchas categorías de bienes importados.
\end{itemize}
\end{block}

\end{frame}

% --- Diapositiva 4: Conclusión y la Lección de la Economía Política ---

\begin{frame}

\frametitle{Conclusión: La Importancia del Multilateralismo}

\begin{block}{\textbf{Evaluación de la Fase Uno}}
\begin{enumerate}
    \item \textbf{Efecto Limitado:} El acuerdo no eliminó las distorsiones ni los aranceles más altos ya impuestos.
    \item \textbf{Crítica Económica:} El comercio gestionado es visto como ineficiente; los economistas dudan que pueda forzar patrones de compra.
    \item \textbf{Lección Principal:} La situación demuestra que las naciones, al actuar unilateralmente (el dilema del prisionero), llegan a un \alert{resultado subóptimo} que solo puede ser resuelto mediante la \textbf{cooperación multilateral} (OMC).
\end{enumerate}
\end{block}

\end{frame}

\end{document}