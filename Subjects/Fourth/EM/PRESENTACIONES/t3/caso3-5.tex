%%%%%%%%%%%%%%%%%%%%%%%%%%%%%%%%%%%%%%%%%%%%%%%%%%%%%%%%%%

% PREÁMBULO

%%%%%%%%%%%%%%%%%%%%%%%%%%%%%%%%%%%%%%%%%%%%%%%%%%%%%%%%%%

\documentclass{beamer}

% --- Paquetes Esenciales ---

\usepackage[utf8]{inputenc} % Acentos y caracteres en español

\usepackage[T1]{fontenc}

\usepackage[spanish]{babel} % Configuración para el idioma español

\usepackage{graphicx} % Para incluir imágenes

\usepackage{amsmath} % Para fórmulas económicas

% --- Configuración del Tema de Beamer ---

\usetheme{Madrid} % Un tema popular y limpio.

\title[Tarifas Trump]{Aranceles bajo la Administración Trump: Análisis de Economía Política}
\author{Ismael Sallami Moreno}
\institute{UGR}
\date{\today}

\begin{document}

% --- Diapositiva 1: Título ---
\begin{frame}
\titlepage
\end{frame}

% --- Diapositiva 2: Marco Político y Herramientas ---

\begin{frame}
\frametitle{Aranceles de EE. UU. (2018–2019): Marco y Justificación}

\begin{block}{\textbf{Instrumentos de Política Comercial} (Ejemplos Clave)}
\begin{itemize}
    \item \textbf{Sección 201} (Salvaguarda): Aranceles sobre lavadoras y paneles solares.
    \item \textbf{Sección 232} (Seguridad Nacional): Aranceles sobre acero (25\%) y aluminio (10\%) .
    \item \textbf{Sección 301} (Propiedad Intelectual): Aranceles masivos a importaciones chinas.
\end{itemize}
\end{block}

\vfill

\begin{block}{\textbf{El Factor Político: Negociación}}
\begin{itemize}
    \item El objetivo principal era la \alert{negociación} (ej. reducción de barreras chinas, compras agrícolas, IP).
\end{itemize}
\end{block}

\end{frame}

% --- Diapositiva 3: Impacto de Bienestar (Modelo de País Grande) ---

\begin{frame}

\frametitle{Análisis de Bienestar: Impacto en EE. UU. (2018)}

\begin{block}{\textbf{Efectos Cuantificados} (Modelo de País Grande)}

\begin{itemize}
    \item \textbf{Pérdida al Consumidor} (Costos por precios más altos): $\mathbf{68.8}$ mil millones.
    \item \textbf{Ingreso Arancelario} (Recaudación gubernamental): $\mathbf{39.4}$ mil millones.
    \item \textbf{Ganancia por Términos de Intercambio (TI):} $\mathbf{21.6}$ mil millones.
\end{itemize}
\end{block}

\vfill

\begin{block}{\textbf{Resultado Neto y Costo por Hogar}}
\begin{itemize}
    \item \alert{Pérdida Neta Total} (EE. UU.): $\mathbf{7.8}$ mil millones.
    \item \textbf{Costo Promedio por Hogar}: $\mathbf{61}$ anuales.
\end{itemize}
\end{block}

\end{frame}

% --- Diapositiva 4: Retaliación y Conclusión ---

\begin{frame}

\frametitle{Conclusiones: Aranceles y Guerra Comercial}

\begin{block}{\textbf{Consecuencias de la Estrategia Arancelaria}}
\begin{enumerate}
    \item \textbf{Retaliación:} Países extranjeros (ej. China) respondieron con aranceles a exportaciones de EE. UU. (ej. productos agrícolas).
    \item \textbf{Objetivo Político:} Uso de aranceles como "táctica de negociación" (\textit{bargaining tactic}).
    \item \textbf{Balance Económico:} A pesar de las ganancias en los T.I. (debido a la caída del precio del exportador extranjero), el impacto neto para EE. UU. fue una \alert{pérdida de bienestar}.
\end{enumerate}
\end{block}

\vfill

\end{frame}

\begin{frame}
\centering
\Huge \textbf{¡Gracias por su atención!}
\vfill
\Large ¿Preguntas?
\end{frame}

\end{document}