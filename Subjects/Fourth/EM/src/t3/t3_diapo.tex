\chapter{La política comercial internacional e instituciones}

\section{Introducción}

La disciplina de la economía internacional establece que el \textbf{libre comercio} genera beneficios sustanciales para las naciones participantes, derivados principalmente de las ventajas comparativas y las economías de escala . No obstante, la \textbf{política comercial} se manifiesta a través de diversas restricciones impuestas por los gobiernos, las cuales configuran las barreras al comercio internacional . El estudio de estas restricciones —principalmente los aranceles y las barreras no arancelarias—, así como de las instituciones multilaterales que buscan regularlas (como el GATT y la OMC), constituye el núcleo de la política comercial contemporánea .

\section{Beneficios del libre comercio}
El análisis del bienestar económico derivado del comercio se fundamenta en la teoría de la oferta y la demanda, cuantificando los efectos en el \textbf{excedente del consumidor} (EC) y el \textbf{excedente del productor} (EP) .

\subsection{Excedente del consumidor y productor}
El \textbf{Excedente del Consumidor (EC)} representa la utilidad o satisfacción que obtienen los consumidores por encima del precio efectivamente pagado por la mercancía . Gráficamente, corresponde al área bajo la curva de demanda y por encima del precio de equilibrio. Por su parte, el \textbf{Excedente del Productor (EP)} se define como el rendimiento que perciben los factores fijos de producción en esa industria, siendo equivalente a los ingresos totales de venta menos los costes variables de producción .

\subsection{El bienestar con libre comercio}
En una economía cerrada, el bienestar total es la suma del EC y el EP, alcanzado en el precio de autarquía ($P_A$) . Para un \textbf{país pequeño} (un \textit{price taker} que acepta el precio mundial $P_M$), la apertura al libre comercio altera significativamente la distribución del bienestar. Si el precio mundial ($P_M$) es inferior al precio de autarquía ($P_A$), el precio interno cae a $P_M$. Esta disminución incrementa el EC (equivalente a la suma de las áreas B y D) y reduce el EP (equivalente al área B). El efecto neto sobre el bienestar de la nación es positivo e igual al área D, lo que representa la \textbf{ganancia neta por eficiencia} para el país pequeño que importa, demostrando los beneficios estáticos del comercio .

\subsection{La demanda de importaciones}
La \textbf{curva de demanda de importaciones ($M$)} es una función inversa del precio y se deriva de la diferencia horizontal entre la cantidad demandada internamente ($Q_D$) y la cantidad ofertada internamente ($Q_S$) para un bien a un precio determinado: $M = Q_D - Q_S$ . En el precio de autarquía ($P_A$), la demanda de importaciones es nula ($M=0$). A medida que el precio desciende por debajo de $P_A$, la demanda interna supera a la oferta nacional, generando una demanda positiva de importaciones que culmina en el equilibrio internacional .

\section{Definición y clasificación de los aranceles}
Un \textbf{arancel} es un impuesto o gravamen aduanero que se aplica a una mercancía al cruzar la frontera nacional, siendo clasificado como \textit{arancel a la importación} o \textit{arancel a la exportación} .

Los principales tipos de aranceles a la importación son:
\begin{enumerate}
    \item \textbf{Ad valorem:} Se expresa como un porcentaje fijo sobre el valor de la mercancía importada (ejemplo: un arancel del 10\% sobre el valor de los automóviles importados de EE. UU. a la UE) .
    \item \textbf{Específico:} Es una cantidad fija por unidad física intercambiada (ejemplo: 1.000 euros por vehículo) .
    \item \textbf{Mixto:} Combina las características del arancel \textit{ad valorem} y el específico .
\end{enumerate}

\section{Los aranceles a la importación en nuestro país pequeño}
El análisis en un \textbf{país pequeño} (Nación) se basa en la premisa de que esta nación es incapaz de influir en el precio internacional ($P_M$), por lo que la oferta de exportaciones extranjeras ($X_E$) es infinitamente elástica al precio mundial .

\subsection{El libre comercio en un país pequeño}
Bajo el \textbf{libre comercio}, el precio interno es igual al precio mundial ($P_M$), y el país importa la cantidad deseada al enfrentarse a una curva de oferta de exportaciones extranjeras horizontal .

\subsection{El efecto del arancel en nuestro un país pequeño}
La imposición de un arancel ($t$) por un país pequeño incrementa el precio interno a $P_D = P_M + t$ . Este aumento genera cuatro efectos clave en el mercado nacional :
\begin{enumerate}
    \item \textbf{Efecto Consumo:} Disminuye el EC en $-(A+B+C+D)$.
    \item \textbf{Efecto Producción:} Aumenta el EP en $+A$, incentivando la producción nacional.
    \item \textbf{Efecto Ingreso Gubernamental:} Aumenta la recaudación arancelaria en $+C$.
    \item \textbf{Efecto Neto sobre Bienestar:} La suma de los efectos es $-(B+D)$. El término $B$ representa la \textbf{pérdida de eficiencia de consumo} (o distorsión del consumo), y $D$ representa la \textbf{pérdida de eficiencia de producción} (o distorsión de la producción, ya que el coste marginal interno es mayor que el precio mundial) .
\end{enumerate}
En síntesis, un arancel en un país pequeño siempre resulta en una \textbf{pérdida neta de bienestar} .

\subsection{El efecto del arancel sobre la economía mundial}
Dado que un país pequeño no afecta los precios internacionales, el arancel no genera cambios en los términos de intercambio. La pérdida neta de bienestar para el país pequeño, $-(B+D)$, se traduce en la misma pérdida neta para la economía mundial. Esto se debe a que la ineficiencia de producción (B) y la ineficiencia de consumo (D) en la Nación no se compensan con ninguna ganancia en el resto del mundo .

Un ejemplo práctico de estas políticas es la aplicación de aranceles de salvaguardia (Sección 201 de la Ley de Comercio de 1974) por parte de EE. UU. La Comisión de Comercio Internacional (ITC) puede autorizar aranceles temporales si el aumento súbito de importaciones causa daño grave a una industria nacional . Por ejemplo, el arancel de hasta el 30\% sobre el acero aplicado por la administración Bush en 2002 se justificó bajo la Sección 201 , aunque resultó en una pérdida neta anual estimada para EE. UU. de \$185 millones (4.5\% del valor de la importación) .

\section{Los aranceles a la importación en nuestro país grande}
Un \textbf{país grande} es aquel que, debido al volumen de sus transacciones, tiene la capacidad de \textbf{modificar el precio internacional} de la mercancía al imponer una política comercial .

\subsection{La oferta de exportaciones del país extranjero}
Para el país grande (Nación), la curva de oferta de exportaciones del país extranjero ($X_E$) ya no es horizontal, sino que presenta una pendiente positiva . A medida que el precio internacional ($P_M$) aumenta, el país extranjero incrementa su exceso de oferta ($Q^S_{E} - Q^D_{E}$), resultando en una mayor cantidad ofrecida para exportación .

\subsection{El arancel y los precios en nuestro país y el extranjero}
Cuando la Nación impone un arancel ($t$), los efectos se transmiten de forma asimétrica. El precio interno de la Nación ($P_D$) sube, pero en una cuantía menor al arancel (a $P_D = P_{M}^{t+}$). Simultáneamente, el precio que reciben los exportadores extranjeros ($P_E$) disminuye (a $P_{M}^{t-}$), ya que el país grande reduce su demanda y obliga a los vendedores a bajar precios para mantener el volumen de ventas . La diferencia entre el precio interno y el precio externo es el arancel: $t = P_{M}^{t+} - P_{M}^{t-}$ .
Históricamente, se asumía que EE. UU. era un país pequeño en el comercio de acero, pero hoy se le considera un \textbf{país grande} en ese sector, capaz de modificar los precios internacionales. Cuando EE. UU. impone un arancel al acero, los países exportadores están incentivados a reducir sus precios para no perder a EE. UU. como cliente .

\subsection{El efecto del arancel en nuestro país grande}
El efecto neto sobre el bienestar de la Nación (país grande) es $E - (B+D)$, donde:
\begin{enumerate}
    \item $E$ es la \textbf{ganancia por los términos de intercambio} (la Nación paga un precio más bajo $P_{M}^{t-}$ a los exportadores extranjeros). Esta es una transferencia de bienestar del extranjero a la Nación .
    \item $(B+D)$ es la \textbf{pérdida de eficiencia} (o pérdida irrecuperable de peso muerto, \textit{deadweight loss}), derivada de la distorsión en la producción ($B$) y el consumo ($D$) .
\end{enumerate}
El bienestar nacional aumenta solo si la ganancia por la mejora en los términos de intercambio ($E$) supera la pérdida de eficiencia ($B+D$) . La administración Trump, al usar aranceles (Secciones 232 y 301) , generó una ganancia de 21.6 mil millones por términos de intercambio, pero una pérdida al consumidor de 68.8 mil millones. El \textbf{resultado neto agregado fue una pérdida de bienestar} de 7.8 mil millones de dólares para EE. UU., indicando que la pérdida de eficiencia fue mayor que la ganancia en los términos de intercambio .

\subsection{El efecto del arancel sobre la economía mundial}
Mientras que el país grande puede ganar bienestar bajo ciertas condiciones, el país extranjero siempre pierde, sufriendo una reducción de bienestar de $-(E+F)$, donde $E$ es la pérdida por los términos de intercambio y $F$ es la pérdida de eficiencia de sus productores (efecto empobrecimiento) .
El efecto neto sobre el bienestar mundial siempre es negativo: $-(B+D+F)$. Esto confirma que los aranceles, incluso los impuestos por países grandes, son globalmente ineficientes al introducir distorsiones que superan las ganancias unilaterales . Esta dinámica se asemeja a un \textbf{Equilibrio de Nash} subóptimo en la teoría de juegos, donde ambos países tienen incentivos para imponer aranceles (ganancia por $E$), pero si ambos lo hacen, terminan con una pérdida neta para el mundo .

\subsection{El arancel óptimo: El caso de un país grande}
El \textbf{arancel óptimo} ($t^O$) es el tipo arancelario que maximiza el aumento neto de bienestar para el país importador grande . El arancel óptimo es inversamente proporcional a la \textbf{elasticidad de la oferta de exportación} del país extranjero .
\begin{itemize}
    \item Si la elasticidad de oferta de exportación es \textbf{baja}, el país grande puede imponer un arancel alto y aún obtener ganancias (ejemplo: arancel óptimo del 370\% para acero aleado, según un estudio de elasticidad) .
    \item Si la elasticidad es \textbf{alta}, el arancel óptimo es cercano a cero (ejemplo: arancel óptimo casi 0\% para acero no aleado, elasticidad 750) .
\end{itemize}

\section{Otros instrumentos de política comercial}
Además de los aranceles, los países emplean instrumentos como las cuotas de importación y los subsidios a la exportación para restringir el comercio .

\subsection{Las cuotas de importación}
Una \textbf{cuota de importación} es una limitación cuantitativa directa sobre el número de unidades de un bien que se puede importar .

\subsubsection{Las cuotas a la importación en un país pequeño}
Una cuota de importación es similar, en sus efectos de consumo y producción, a un arancel \textit{ad valorem} que eleva el precio interno de $P_M$ a $P_C$ . Los efectos sobre el EC y el EP son idénticos a los del arancel . La diferencia crucial radica en el área de recaudación ($C$), que en el caso de la cuota se convierte en \textbf{rentas de la cuota} . El efecto neto sobre el bienestar del país pequeño es $-(B+D) + C$ (si las rentas permanecen en el país) .

\subsubsection{Mecanismos de asignación de las cuotas a la importación}
La asignación de las rentas de la cuota (área $C$) determina el impacto final en el bienestar nacional:
\begin{enumerate}
    \item \textbf{Licencias de Cuota:} Si el gobierno otorga licencias a empresas o individuos nacionales, la Nación conserva la renta $C$, y la pérdida neta es $-(B+D)$ .
    \item \textbf{Búsqueda de Rentas (\textit{Rent-seeking}):} Si las empresas gastan recursos equivalentes al valor de la renta ($C$) para presionar o sobornar y obtener las licencias, esta renta se disipa. La pérdida neta de bienestar para la Nación es $-(B+C+D)$ .
    \item \textbf{Subasta de Cuotas:} Si el gobierno subasta las licencias, recauda el valor de $C$. La pérdida neta es $-(B+D)$ .
    \item \textbf{Restricciones Voluntarias de la Exportación (VER):} Bajo esta modalidad, es el gobierno del país exportador el que limita las exportaciones. Las rentas de la cuota ($C$) son percibidas por las empresas del país exportador (Foreign). La Nación importadora pierde el área $C$, por lo que el efecto neto es $-(B+C+D)$ .
\end{enumerate}
Un caso histórico de cuotas fue el \textbf{Acuerdo Multifibras (MFA)} que permitió a países importadores ricos (EE. UU., UE) limitar las importaciones de textiles de países en desarrollo . Tras su expiración en 2005, las exportaciones chinas aumentaron drásticamente (ej. $+40\%$ a EE. UU.), y los precios de los productos restringidos cayeron un 38\% . El costo estimado de bienestar del MFA para EE. UU. era de \$11.4 mil millones anuales .

\subsection{Subsidios a la exportación}
Un \textbf{subsidio a la exportación} es un pago del gobierno a las empresas por unidad de bien exportada, pudiendo ser \textit{ad valorem} o \textit{específico} .

\subsubsection{Definición y clasificación de los subsidios a la exportación}
Se clasifican de forma análoga a los aranceles: \textit{ad valorem} (porcentaje del valor exportado) o \textit{específico} (cantidad fija por unidad física exportada) .

\subsubsection{Los subsidios a la exportación en un país pequeño.}
Para un país pequeño, el subsidio ($s$) aumenta el precio interno a $P_M + s$, distorsionando el mercado: aumenta la producción ($Q^S$) y reduce el consumo ($Q^D$) . Los efectos de bienestar son: pérdida de EC (B), ganancia de EP ($A+B+C$), y costo gubernamental ($-s \times X_2$, equivalente a $B+C+D$) . El \textbf{efecto neto sobre el bienestar nacional siempre es una pérdida} de $-(B+D)$, que representa la pérdida de eficiencia de consumo ($B$) y la pérdida de eficiencia de producción ($D$) .

\subsubsection{Los subsidios a la exportación en un país grande.}
En un país grande, el subsidio a la exportación incrementa la oferta mundial del bien, lo que provoca una \textbf{caída del precio internacional} ($P_E < P_M$) . El efecto neto sobre el bienestar de la Nación es $-(B+D+E)$, donde $E$ es la \textbf{pérdida por los términos de intercambio} (el país exporta $X_2$ a un precio internacional inferior) . El efecto global es una pérdida mundial de $-(B+D+F)$ .

\section{Resumen de los efectos de las barreras al libre comercio}
Las barreras comerciales tienen efectos consistentes en la economía doméstica :
\begin{itemize}
    \item \textbf{Excedente del productor:} Aumenta siempre (en aranceles, subsidios y cuotas), ya que sube el precio interno o se incentiva directamente la producción.
    \item \textbf{Excedente del consumidor:} Disminuye siempre, debido al aumento del precio interno.
    \item \textbf{Bienestar Nacional:} Es indeterminado para aranceles y cuotas en un país grande (dependiendo de la ganancia en T.I. vs. la pérdida de eficiencia), pero \textbf{siempre disminuye} en el caso de un país pequeño y en el caso de subsidios a la exportación (país grande o pequeño) .
\end{itemize}

\section{Otras barreras no arancelarias}
Las \textbf{barreras no arancelarias (BNA)} han proliferado desde la II Guerra Mundial como una forma de restricción comercial alternativa a los aranceles . Incluyen:
\begin{itemize}
    \item \textbf{Regulaciones Técnicas y Sanitarias:} Imponen condiciones de seguridad (ej. automóviles, equipos electrónicos) o sanidad (ej. alimentos) que obstaculizan la importación .
    \item \textbf{Campañas Gubernamentales:} Programas que incitan el consumo de productos nacionales (ej. Ley \textit{“buy to US”}) o impuestos fronterizos .
    \item \textbf{Cárteles Internacionales:} Organizaciones de proveedores (ej. la OPEP) que buscan restringir la producción y exportación para maximizar beneficios .
    \item \textbf{Dumping:} Exportación de mercancías a precios inferiores al costo de producción o al precio en el mercado nacional .
    \item \textbf{Compras Estatales:} Preferencia de gobiernos por productos nacionales, aunque sean más caros .
\end{itemize}

\section{Libre comercio versus proteccionismo}
El debate entre libre comercio y proteccionismo se analiza desde tres perspectivas principales: teórica, de beneficios adicionales y de economía política .

\subsection{Argumentos a favor del libre comercio}

\subsubsection{Desde la perspectiva teórica}
La defensa central del libre comercio es la \textbf{eliminación de las pérdidas de eficiencia} ($B+D$) causadas por los aranceles en el país pequeño . Además, el libre comercio maximiza el bienestar mundial al evitar la pérdida global $-(B+D+F)$ generada por los aranceles de países grandes .

\subsubsection{Desde la perspectiva de los beneficios adicionales}
El libre comercio produce ganancias dinámicas que van más allá de los efectos estáticos de excedentes: genera \textbf{economías de escala} (aumentando la eficiencia productiva) y fomenta la competencia, lo que se traduce en una mayor variedad de productos para los consumidores .

\subsubsection{Desde la perspectiva de la economía política}
El libre comercio se justifica como una política que contrarresta las presiones de \textbf{grupos de interés} concentrados (productores nacionales) que buscan rentas o subsidios a expensas de los costes distribuidos entre millones de consumidores .

\subsection{Argumentos en contra del libre comercio}

\subsubsection{La imposición de un arancel en el caso de un país grande}
El argumento del proteccionismo más sólido reside en la capacidad del país grande para mejorar sus términos de intercambio. Un arancel es beneficioso para el país si $E > (B+D)$ . Sin embargo, esta ganancia unilateral se obtiene a costa de empobrecer al país extranjero.

\subsubsection{Los fallos del mercado y el beneficio marginal social en un país pequeño}
El proteccionismo se justifica si existen \textbf{fallos de mercado} que resultan en un Beneficio Marginal Social ($B_{Marg}^S$) positivo. Por ejemplo, si la producción genera un beneficio social (como el aumento de ingresos impositivos) que no está internalizado en el precio, un arancel podría ser el instrumento de segundo mejor para incrementar la producción nacional ($Q_{S2} > Q_{S1}$) y capturar el Beneficio Social, aunque a costa de las pérdidas de eficiencia $B$ y $D$ .

\subsubsection{El caso de la industria naciente}
La protección temporal a una \textbf{industria naciente} se justifica en el supuesto de que esta no puede competir inicialmente con industrias extranjeras maduras. El objetivo es permitir que la industria doméstica alcance la escala suficiente y las ventajas comparativas necesarias para ser competitiva internacionalmente y eliminar la protección posteriormente .

\subsubsection{La argumentación falaz y la retórica del proteccionismo}
Si bien hay argumentos válidos para el proteccionismo (país grande y fallos de mercado), gran parte de la retórica proteccionista se basa en argumentos falaces que ocultan los costos netos para la sociedad, o magnifican los beneficios localizados de la protección.

\section{Los acuerdos comerciales y la política comercial}
El desmantelamiento de las barreras arancelarias desde los años 30 del siglo XX fue posible gracias a la \textbf{negociación internacional} y los acuerdos comerciales .

\subsection{La lógica de los acuerdos comerciales multilaterales}
La necesidad de acuerdos multilaterales se explica por la \textbf{teoría de juegos} y el \textbf{Dilema del Prisionero} .
\begin{itemize}
    \item \textbf{Estrategia dominante (Equilibrio Nash):} En ausencia de un acuerdo, la imposición de proteccionismo (PC) es la mejor opción unilateral para cada país grande (ya que $E > (B+D)$), lo que conduce a un equilibrio Nash subóptimo donde ambas naciones aplican aranceles mutuos y sufren una pérdida neta de bienestar global.
    \item \textbf{Estrategia Cooperativa:} La cooperación internacional, lograda a través de acuerdos, permite a los países alcanzar una solución mutuamente beneficiosa (LC/LC), superando el equilibrio de Nash .
\end{itemize}
La tregua de la "Fase Uno" del conflicto arancelario entre EE. UU. y China, que involucró \textbf{comercio gestionado} (\textit{Managed Trade}), es un ejemplo de intento de cooperación que, sin eliminar completamente los aranceles, resultó en un acuerdo que los economistas critican como ineficiente y que mantiene distorsiones .

\subsection{Una breve historia de los acuerdos comerciales: del GATT a la OMC}
Tras la II Guerra Mundial, la Conferencia de Bretton Woods configuró el sistema de libre comercio internacional que dio lugar al \textbf{Acuerdo General de Tarifas y Comercio (GATT)} . El GATT operó a través de "rondas" de negociación hasta la \textbf{Ronda de Uruguay}, donde se decidió crear la \textbf{Organización Mundial del Comercio (OMC)} .

\subsection{Disposiciones claves del GATT relacionadas con la política comercial internacional}
Las cláusulas fundamentales del GATT, que regulan la política comercial internacional, incluyen :
\begin{itemize}
    \item \textbf{Cláusula de la Nación Más Favorecida (Art. I):} Requiere que los aranceles otorgados a un socio comercial se extiendan a todos los demás miembros de la OMC .
    \item \textbf{Cláusula Antidumping (Art. VI):} Permite imponer aranceles para contrarrestar el \textit{dumping} .
    \item \textbf{Cláusula de Prohibición de Cuotas (Art. XI):} Prohíbe el uso de cuotas a la importación.
    \item \textbf{Disposición de Salvaguarda o Cláusula de Escape (Art. XIX):} Permite restricciones temporales a la importación (\textbf{Sección 201}) si un aumento de las importaciones amenaza con causar daño grave a la industria nacional .
    \item \textbf{Excepción para Acuerdos Regionales (Art. XXIV):} Autoriza la formación de Áreas de Libre Comercio y Uniones Aduaneras, siempre que no eleven barreras al comercio con terceros .
\end{itemize}

\subsection{La OMC: objetivos, funciones y diferencias básicas con el GATT}
Los objetivos de la OMC son esencialmente los mismos que los del GATT . Las \textbf{funciones} incluyen facilitar la aplicación de acuerdos, servir de foro de negociación y operar la Cláusula de Entendimiento en la Solución de Diferencias (ESD) .
Las \textbf{diferencias básicas} son :
\begin{itemize}
    \item El GATT fue un acuerdo provisional; la OMC es una organización internacional \textbf{permanente} y de pleno derecho.
    \item El GATT se centró en la \textbf{coordinación} de políticas; la OMC permite a los miembros fijar el modo de funcionamiento.
    \item La OMC destaca por su \textbf{mecanismo de resolución de conflictos} vinculante.
    \item La OMC amplió su cobertura del comercio de mercancías (GATT) a \textbf{servicios} y \textbf{propiedad intelectual}.
\end{itemize}

\section{Las políticas comerciales en los países en vías de desarrollo}
Los países en desarrollo a menudo enfrentan una alta dependencia de las naciones avanzadas, y sus exportaciones se concentran históricamente en \textbf{productos primarios} .

\subsection{La industrialización mediante sustitución de importaciones}
La \textbf{Industrialización por Sustitución de Importaciones (ISI)} tuvo como objetivo fomentar la creación de industrias nacionales para el mercado interno, justificando la imposición de altas barreras comerciales .
\begin{itemize}
    \item \textbf{Ventajas:} Mínimos riesgos, generación de empleo y facilidad para implementar la protección .
    \item \textbf{Inconvenientes:} Falta de incentivos para la eficiencia, incapacidad para aprovechar las economías de escala y discriminación implícita contra la industria exportadora . Los fabricantes nacionales tendieron a resistirse a la futura disminución de barreras .
\end{itemize}

\subsection{La liberalización comercial de los años ochenta}
A mediados de los 80, muchos países en desarrollo reorientaron sus políticas comerciales, disminuyendo aranceles y eliminando cuotas . Esto resultó en un aumento del volumen comercial y un cambio en el patrón de comercio hacia la \textbf{exportación de bienes manufacturados} . La liberalización tuvo consecuencias asimétricas: mientras que países como Brasil experimentaron una disminución de salarios reales para trabajadores poco cualificados, India y Chile registraron un fuerte crecimiento económico .

Este período también vio un masivo crecimiento de las \textbf{reservas internacionales} en países en desarrollo (China, India, Rusia, Brasil) desde fines de los 90 . La acumulación de reservas fue un \textbf{nuevo motivo de precaución} tras la crisis asiática (1997–1998), utilizándolas como \textbf{seguro} para defender la moneda ante ataques especulativos y protegerse de interrupciones repentinas en la entrada de capitales .

\subsection{La industrialización fundamentada en la exportación: las economías asiáticas}
Los países del Este de Asia se distinguieron por un crecimiento económico notable y \textbf{estable} entre los años sesenta y noventa, manteniendo una distribución de la renta más equitativa . Factores clave de este éxito, catalogado por el Banco Mundial como \textit{Economías Asiáticas de Alto Crecimiento} (HPAE) , incluyeron :
\begin{itemize}
    \item \textbf{Altas tasas de ahorro e inversión:} En 1990, ahorraban en promedio el 34\% de su PIB, lo que financió inversiones productivas sin dependencia externa excesiva .
    \item \textbf{Dotación de capital humano:} La inversión fundamental en educación básica y secundaria creó una población más capacitada y productiva (ej., Corea del Sur, Singapur, Hong Kong) .
    \item \textbf{Apertura y Estabilidad Macroeconómica:} Mantuvieron baja inflación, crecimiento sostenido y una fuerte apertura al comercio internacional, atrayendo \textbf{Inversión Extranjera Directa (IED)} e importaciones de tecnología . China, por ejemplo, usó la acumulación de reservas (que alcanzó el 50\% de su PIB hacia 2010) tanto como seguro financiero como herramienta para mantener su competitividad exportadora al evitar la apreciación del yuan .
\end{itemize}

