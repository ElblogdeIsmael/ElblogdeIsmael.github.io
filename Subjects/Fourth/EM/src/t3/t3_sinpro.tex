% =========================================================================
% INICIO DEL CAPÍTULO 3
% =========================================================================

\chapter{La política comercial internacional e instituciones}
\label{ch:tema3}


\section*{Objetivos de Aprendizaje del Tema 3}
\addcontentsline{toc}{section}{Objetivos de Aprendizaje del Tema 3}

A partir del estudio de este tema, el alumno podrá:

\begin{enumerate}
    \item Evaluar los costes y los beneficios de los aranceles, sus efectos sobre el bienestar y los ganadores y perdedores de las políticas arancelarias.
    \item Analizar qué son los subsidios a la agricultura y a la exportación y explicar cómo afectan al comercio.
    \item Comprender cómo han promovido el comercio mundial los acuerdos y las negociaciones internacionales.
    \item Explicar los argumentos en favor del libre comercio que van más allá de las ganancias convencionales derivadas del comercio.
    \item Valorar la teoría y la evidencia empírica que sustentan los enfoques de "economía política" de la política comercial.
    \item Entender la función de los aranceles y las barreras no arancelarias (BNC) y la diferencia entre ellas, como instrumentos proteccionistas.
\end{enumerate}

\section{Introducción}

El estudio de la política comercial responde a la pregunta de cómo debería ser la política comercial de una nación. Las políticas gubernamentales que afectan al comercio internacional se analizan a menudo desde la perspectiva del \textbf{equilibrio parcial}. Sin embargo, la comprensión de las políticas comerciales también requiere analizar los efectos del comercio no solo sobre el país en su conjunto, sino sobre la \textbf{distribución de la renta} dentro del mismo, ya que el comercio genera ganadores y perdedores.

\section{Beneficios del libre comercio}

El libre comercio ofrece ganancias que van más allá de las convencionales derivadas de la especialización estática. El libre comercio proporciona a los empresarios incentivos para buscar nuevas vías para exportar o competir con las importaciones, fomentando el \textbf{aprendizaje y la innovación} más que un sistema de comercio "administrado". Además, el libre comercio permite la especialización conforme al principio de la ventaja comparativa, lo que resulta mutuamente beneficioso para las naciones que participan.

\subsection{Excedente del consumidor y productor}

Para evaluar los efectos de la política comercial se utilizan las herramientas del excedente del consumidor y del productor:

\subsubsection{Excedente del Consumidor}
El \textbf{excedente del consumidor (EC)} se define como la diferencia entre lo que los consumidores están dispuestos a pagar por un bien y lo que realmente pagan. Se calcula restando el precio por la cantidad demandada del área debajo de la curva de demanda hasta esa cantidad. Si el precio sube de $P_1$ a $P_2$, el excedente del consumidor se reduce.

\subsubsection{Excedente del Productor}
El \textbf{excedente del productor (EP)} se define como la renta extra que obtienen los productores nacionales en el mercado. En un gráfico de oferta y demanda, corresponde al área por encima de la curva de oferta.

\subsection{El bienestar con libre comercio}

En condiciones de libre comercio, la producción y el consumo de un país están determinados por los precios internacionales. Los economistas han desarrollado modelos analíticos para determinar los efectos de las políticas gubernamentales que afectan al comercio y realizar un \textbf{análisis coste-beneficio}.

\subsection{La demanda de importaciones}

La demanda de importaciones de un país es la cantidad de un bien que los residentes de un país desean comprar que excede la cantidad que los productores nacionales están dispuestos a ofrecer a un precio dado. En un mercado libre, el precio internacional equilibra la demanda de importaciones de un país con la oferta de exportaciones del otro.

\section{Definición y clasificación de los aranceles}

Un arancel es un impuesto que grava las importaciones. Los aranceles pueden clasificarse en varios tipos:
\begin{enumerate}
    \item \textbf{Arancel Específico:} Un cargo fijo por unidad física del bien importado. Por ejemplo, \$10 por tonelada de acero.
    \item \textbf{Arancel Ad Valorem:} Un impuesto calculado como un porcentaje del valor del bien importado. Por ejemplo, 10\% sobre el valor de un automóvil.
    \item \textbf{Arancel Compuesto:} Una combinación del arancel específico y el \textit{ad valorem}.
\end{enumerate}

\section{Los aranceles a la importación en nuestro país pequeño}

Un \textbf{país pequeño} es aquel que, debido a su limitado poder de mercado, es incapaz de influir en el precio internacional de un bien.

\subsection{El libre comercio en un país pequeño}

En libre comercio, el precio interno de un bien se iguala al precio mundial ($P_W$). En este punto, el excedente del consumidor y del productor reflejan el máximo bienestar posible bajo ese precio.

\subsection{El efecto del arancel en nuestro un país pequeño}

Si un país pequeño impone un arancel ($t$), el precio interno sube exactamente en la cuantía del arancel, de $P_W$ a $P_W + t$. Este arancel tiene los siguientes efectos sobre el bienestar:
\begin{enumerate}
    \item El precio interno aumenta, reduciendo el \textbf{excedente del consumidor}.
    \item El precio interno más alto aumenta el \textbf{excedente del productor} nacional.
    \item El gobierno recauda ingresos (producto del arancel multiplicado por el volumen de importaciones).
    \item Se generan \textbf{pérdidas irrecuperables de eficiencia} (costes del bienestar), que se dividen en el efecto distorsión de la producción (pérdida de eficiencia por producir a un coste más alto) y el efecto distorsión del consumo (pérdida de eficiencia por consumir menos).
\end{enumerate}

\section{El arancel en un país grande}

Un \textbf{país grande} puede influir en el precio internacional de un bien. Cuando un país grande aplica un arancel, el precio mundial neto del arancel pagado a los exportadores extranjeros \textbf{disminuye}.

\subsection{El efecto en el bienestar de un arancel en un país grande}

Para un país grande, un arancel genera los mismos costes de distorsión del consumo y la producción que en un país pequeño, pero también puede generar un \textbf{beneficio} al obtener una mejora en sus \textbf{términos de intercambio} (el precio de sus exportaciones dividido por el precio de sus importaciones). Este beneficio surge porque el arancel reduce el precio que el país paga a los extranjeros por las importaciones. Si la ganancia por la mejora en los términos de intercambio supera las pérdidas por las distorsiones, el bienestar nacional \textbf{aumenta}, resultando en un \textbf{arancel óptimo}.

\section{Tasa de protección efectiva}

La \textbf{tasa de protección efectiva (TPE)} es una medida crucial para entender el impacto real del proteccionismo. La TPE se define como el cambio porcentual en el valor añadido nacional por unidad de producto, que resulta de la estructura arancelaria, incluyendo los aranceles sobre los bienes finales y los aranceles sobre los \textbf{insumos intermedios}.

La TPE es relevante porque las empresas a menudo importan insumos utilizados en su producción (como chips de memoria o microprocesadores). El arancel nominal (el impuesto sobre el producto final) puede ser engañoso si se ignora el arancel sobre los componentes utilizados para fabricar el producto final.

\section{Barreras no arancelarias al comercio (BNC)}

Las \textbf{barreras no arancelarias (BNC)} han ganado importancia en las últimas dos décadas como herramientas proteccionistas, ya que restringen el comercio sin utilizar impuestos directos.

\subsection{Tipos de BNC}

Ejemplos de BNC incluyen cuotas de importación, restricciones voluntarias a la exportación (RVE), políticas de adquisición gubernamental, regulaciones sociales y restricciones al transporte marítimo.

\subsection{Cuotas de importación}

Una \textbf{cuota de importación} es una restricción directa a la cantidad de un bien que se permite importar. A diferencia de los aranceles, que permiten la entrada de un bien pagando un impuesto, la cuota limita físicamente la cantidad.

Una diferencia clave entre cuotas y aranceles radica en quién se apropia de las \textbf{rentas} generadas por el aumento del precio interno. Mientras que un arancel genera ingresos para el gobierno, una cuota, si las licencias de importación son asignadas bajo demanda, genera \textbf{rentas de cuota} que pueden ser capturadas por los importadores nacionales o los exportadores extranjeros.

\section{Argumentos a favor de la intervención}

La justificación para que los gobiernos interfieran en el comercio se basa a menudo en desafiar los supuestos que subyacen a las razones que propugnan el libre comercio, buscando objetivos que van más allá del mero análisis coste-beneficio.

\begin{enumerate}
    \item \textbf{Arancel óptimo:} Un país grande puede maximizar el bienestar nacional a través de un arancel que mejore sus términos de intercambio.
    \item \textbf{Argumento de las industrias incipientes:} Los países en desarrollo o incluso los avanzados pueden justificar la protección temporal de industrias "nacientes" que tienen potencial de crecimiento a largo plazo y que de otro modo sufrirían pérdidas iniciales.
    \item \textbf{Externalidades y fallos de mercado:} La intervención puede justificarse cuando hay externalidades (por ejemplo, en investigación y desarrollo), donde los beneficios de la producción de conocimiento no son totalmente apropiados por la empresa, requiriendo un subsidio para incentivar la inversión.
    \item \textbf{Argumentos no económicos:} Incluyen la seguridad nacional o la soberanía alimentaria.
\end{enumerate}

\section{La economía política de la política comercial}

Los gobiernos a menudo aplican políticas que, según el análisis coste-beneficio, producen más costes que beneficios, ya que reflejan \textbf{objetivos que van más allá} de las meras medidas económicas.

\subsection{Argumentos a favor del libre comercio}

Los argumentos en favor del libre comercio son sólidos:
\begin{enumerate}
    \item \textbf{Ganancias de eficiencia:} Permite a los países especializarse de acuerdo con su ventaja comparativa, aumentando la eficiencia global.
    \item \textbf{Economías de escala dinámicas:} Las empresas que exportan y compiten con importaciones tienen más oportunidades de \textbf{aprendizaje e innovación}, lo que fomenta el crecimiento a largo plazo.
    \item \textbf{Evitar la búsqueda de rentas (rent-seeking):} El libre comercio elimina los incentivos para que las empresas y los grupos de interés gasten recursos en influir en las políticas comerciales, recursos que de otro modo podrían utilizarse de forma productiva.
    \item \textbf{Evitar guerras comerciales:} El libre comercio evita las represalias comerciales y las guerras arancelarias que resultan destructivas para el bienestar global.
\end{enumerate}

\subsection{Argumentos en contra del libre comercio}

Los argumentos en contra del libre comercio suelen ser de naturaleza distributiva, destacando que el comercio \textbf{genera ganadores y perdedores} dentro de la nación:
\begin{enumerate}
    \item \textbf{Distribución de la renta:} Las restricciones al comercio pueden beneficiar a los factores que son relativamente escasos en la nación (Teorema de Stolper-Samuelson), lo que explica por qué ciertos grupos, como los trabajadores no especializados en países desarrollados, pueden oponerse a un mayor comercio con naciones de salarios bajos.
    \item \textbf{Competencia desleal:} Los argumentos contra el libre comercio a menudo se centran en el \textbf{dumping} o los subsidios extranjeros, que se perciben como prácticas de comercio desleales que justifican una respuesta proteccionista.
\end{enumerate}

\section{Acuerdos internacionales de comercio}

Los acuerdos internacionales, como el GATT y la OMC, han impulsado el comercio mundial mediante el establecimiento de reglas y procesos de negociación.

\subsection{Preferencias comerciales}

Las preferencias comerciales se refieren a acuerdos comerciales regionales (como zonas de libre comercio o uniones aduaneras) que son \textbf{discriminatorios} porque aplican aranceles bajos o nulos solo a los miembros, no a todas las naciones comerciales (en contraste con la cláusula de nación más favorecida).

\subsection{Una breve historia de los acuerdos comerciales: del GATT a la OMC}

El \textbf{Acuerdo General sobre Aranceles Aduaneros y Comercio (GATT)} fue el marco principal para las negociaciones comerciales internacionales durante medio siglo. El GATT buscó la reducción progresiva de las barreras al comercio y, junto con su sucesor, la OMC, estableció un conjunto de reglas para la conducta comercial.

\subsection{Disposiciones claves del GATT relacionadas con la política comercial}

Una disposición clave del sistema GATT/OMC es el proceso de \textbf{vinculación} (\textit{binding}) arancelaria. Cuando un tipo arancelario es vinculado, el país que lo impone se compromete a no elevarlo en el futuro. En la práctica, esto ha sido muy eficaz para evitar retrocesos en la liberalización.

\subsection{La OMC: objetivos, funciones y diferencias básicas con el GATT}

La \textbf{Organización Mundial del Comercio (OMC)} fue establecida en 1995, sucediendo al GATT.
\begin{enumerate}
    \item \textbf{Objetivos y funciones:} La OMC proporciona un foro para las negociaciones comerciales y, crucialmente, cuenta con un sistema de \textbf{solución de disputas} que da dientes legales a los acuerdos.
    \item \textbf{Diferencias con el GATT:} A diferencia del GATT, que era un acuerdo provisional, la OMC es una \textbf{organización formal} con personalidad jurídica. Además, la OMC amplió la cobertura del comercio a áreas no cubiertas por el GATT, incluyendo el Acuerdo sobre las Cuestiones Comerciales de la Propiedad Intelectual (\textit{TRIPS}) y el comercio de servicios (\textit{GATS}).
\end{enumerate}

\section{Las políticas comerciales en los países en vías de desarrollo}

Los países en desarrollo, a menudo definidos por niveles bajos de PIB per cápita, enfrentan desafíos únicos en la formulación de políticas comerciales.

\subsection{La industrialización mediante sustitución de importaciones}

Históricamente, muchos países en desarrollo adoptaron la estrategia de \textbf{industrialización mediante sustitución de importaciones (ISI)}. Este enfoque buscaba reemplazar los bienes importados con producción nacional, a menudo requiriendo \textbf{controles directos sobre el comercio} y los pagos para salvaguardar el nivel de empleo. Sin embargo, esta estrategia fue criticada por su ineficiencia y por llevar a un crecimiento sacrificado a largo plazo.

\subsection{La liberalización comercial de los años ochenta}

Durante los años ochenta, muchos países en desarrollo adoptaron reformas orientadas al mercado, a menudo influenciadas por organismos internacionales. Esta \textbf{liberalización comercial} supuso un cambio en el orden en que se aplicaban las medidas, buscando un proceso de reforma económica que asegurara que el sistema financiero nacional fuera lo suficientemente fuerte antes de abrir la cuenta de capital.

\subsection{La industrialización fundamentada en la exportación: las economías asiáticas}

Una estrategia alternativa a la ISI es la \textbf{industrialización fundamentada en la exportación} (IE). Este enfoque, adoptado por muchas economías asiáticas, se centra en la apertura del mercado y la promoción de las exportaciones, lo que les permitió descubrir oportunidades inesperadas y lograr un crecimiento significativo.
