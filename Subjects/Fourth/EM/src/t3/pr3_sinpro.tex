
\chapter{La Política Comercial Internacional e Instituciones}
\section{Análisis de la Elasticidad de la Demanda en la Imposición de Aranceles}

\begin{ejercicio}
\textbf{La política comercial internacional e instituciones}

Analice el efecto del arancel, tanto en un país pequeño, como grande, cuando la curva de demanda del bien X en el país Nación (N) es más elástica que la representada en clase. Se le recomienda que represente una curva de demanda del bien X casi totalmente elástica.
\end{ejercicio}

\begin{solucion}[Práctica 3]

\end{solucion}

\subsection{Introducción Teórica: El Arancel y el Bienestar Nacional}

Un \textbf{arancel} es un impuesto o derecho de aduana que grava un producto cuando cruza las fronteras de una nación, generalmente aplicado a las importaciones. El objetivo principal de un arancel puede ser recaudatorio o proteccionista, buscando resguardar a los productores nacionales de la competencia extranjera.

El análisis de los efectos de los aranceles sobre el \textbf{bienestar} se realiza cuantificando las ganancias y pérdidas mediante el excedente del consumidor (EC) y el excedente del productor (EP). El efecto neto de un arancel en el bienestar nacional se descompone en:

\begin{enumerate}
    \item Pérdidas de eficiencia (peso muerto), que incluyen la distorsión de la producción (área $b$, por producir internamente de forma ineficiente) y la distorsión del consumo (área $d$, por consumir menos a un precio más alto).
    \item Un efecto de transferencia (excedente del consumidor transferido a productores y al gobierno).
    \item La ganancia de la \textbf{relación de intercambio} (área $e$), que solo ocurre si el país importador es lo suficientemente grande como para reducir el precio mundial de exportación.
\end{enumerate}

\subsubsection*{Distinción entre Nación Pequeña y Nación Grande}

\begin{itemize}
    \item \textbf{Nación Pequeña:} Es una tomadora de precios, incapaz de influir en el precio internacional ($P_M$). El arancel ($t$) se traslada totalmente al consumidor nacional ($P_T = P_M + t$). Dado que la ganancia de la relación de intercambio ($e$) es nula, el arancel siempre reduce el bienestar nacional en la cuantía de las pérdidas de peso muerto ($b+d$).
    \item \textbf{Nación Grande:} Es lo suficientemente importante como para que una reducción de sus importaciones disminuya el precio mundial ($P_T^* < P_M$). El bienestar nacional puede aumentar si la ganancia por la mejora de la relación de intercambio (área $e$) supera las pérdidas de peso muerto ($b+d$).
\end{itemize}

\subsection{Análisis del Efecto Arancelario en un Mercado con Demanda Altamente Elástica}

La \textbf{elasticidad precio de la demanda} mide la sensibilidad de los compradores a los cambios en el precio. Una demanda muy elástica (cercana a la horizontalidad) implica que un pequeño aumento en el precio provoca una caída desproporcionadamente grande en la cantidad demandada.

\subsubsection*{Efectos del Arancel sobre Precios y Cantidades}

En ambos modelos, la imposición de un arancel $t$ provoca un aumento del precio interno y una reducción del volumen de importaciones ($Q_T$).

\begin{enumerate}
    \item \textbf{Aumento de Precio Interno ($P_T$):} En el país pequeño, $P_T$ sube por $t$. En el país grande, $P_T$ sube por menos de $t$ (porque $P_T^*$ baja).
    \item \textbf{Reducción de Cantidad Demandada ($D_1$ a $D_2$):} Debido a la alta elasticidad, la subida del precio interno, aunque sea pequeña, provocará una \textbf{caída masiva} en la cantidad consumida.
    \item \textbf{Reducción de Importaciones:} La diferencia entre la demanda interna reducida ($D_2$) y la oferta interna aumentada ($S_2$) será el nuevo volumen de importaciones ($Q_T$), el cual será significativamente menor debido a la alta elasticidad.
\end{enumerate}

\subsubsection*{Impacto de la Elasticidad en el Bienestar de la Nación Pequeña}

En una nación pequeña, el arancel siempre reduce el bienestar en $b+d$. Cuando la demanda es altamente elástica, la pérdida de bienestar es \textbf{magnificada}:

\begin{itemize}
    \item \textbf{Pérdida por Distorsión del Consumo ($d$):} Esta pérdida es proporcional a la reducción de la cantidad demandada ($D_1$ a $D_2$). Dado que la demanda es muy elástica, la caída de $D_2$ es muy grande. Por lo tanto, el área del triángulo de pérdida $d$ (el efecto consumo) se incrementa significativamente, reflejando una gran pérdida de excedente del consumidor.
    \item \textbf{Pérdida Neta:} La alta elasticidad asegura que el coste de la política arancelaria, en términos de eficiencia económica, sea \textbf{máximo}. Esto refuerza la conclusión de que la imposición de un arancel por un país pequeño es siempre perjudicial para su bienestar nacional.
\end{itemize}

\subsubsection*{Impacto de la Elasticidad en el Bienestar de la Nación Grande}

En la nación grande, la decisión de bienestar compara la ganancia de los términos de intercambio ($e$) con las pérdidas de peso muerto ($b+d$).

\begin{itemize}
    \item \textbf{Pérdidas de Peso Muerto ($b+d$):} Al igual que en el país pequeño, las pérdidas de eficiencia se \textbf{magnifican} debido a la gran reducción del consumo ($d$).
    \item \textbf{Ganancia de los Términos de Intercambio ($e$):} Esta ganancia es el producto de la reducción del precio extranjero ($P_M - P_T^*$) multiplicado por el volumen de importaciones posterior al arancel ($Q_T$).
\end{itemize}

Aunque el arancel de un país grande sigue induciendo una caída en el precio extranjero (mejora los términos de intercambio), la alta elasticidad de la demanda provoca una \textbf{reducción drástica en el volumen de importaciones} ($Q_T$). Esto significa que la ganancia de la relación de intercambio ($e$) se aplica a una base de comercio mucho menor.

\begin{center}
Ganancia de Bienestar Neto = $e - (b + d)$
\end{center}

Dado que las pérdidas $b+d$ se maximizan por la elasticidad y la ganancia $e$ se aplica a un volumen de comercio severamente reducido, es mucho \textbf{menos probable} que el arancel resulte beneficioso para el país grande en este caso. La elasticidad de la demanda actúa como un factor que \textbf{agrava los costes} de la intervención.

\subsection{Representación Esquemática (Figura conceptual)}

\begin{figure}[h!]
    \centering
    \begin{tikzpicture}[scale=0.9]
        % Ejes
        \draw[->] (0,0) -- (8.2,0) node[right] {Cantidad, Q};
        \draw[->] (0,0) -- (0,6.2) node[above] {Precio, P};

        % Curva de Oferta (Relativamente Inelástica)
        \draw[line width=1pt] (1,1.5) -- (7,4.5) node[right, xshift=5pt] {$O_N$};

        % Curva de Demanda ALTAMENTE ELÁSTICA (casi horizontal)
        \draw[line width=1.2pt, color=blue] (0.5, 5) -- (7.5, 4.8);
        \node[blue, right, yshift=3pt] at (7.5,4.8) {$D_N$ (Muy elástica)};

        % Precios y Cantidades de Libre Comercio (Pequeña Nación)
        \draw[dashed, color=gray] (0, 3) node[left] {$P_W$} -- (5.3, 3);
        \node at (5.3, 3) [circle, fill, inner sep=1.5pt] {};
        \node at (2.9, 3) [circle, fill, inner sep=1.5pt] {};
        \node[below, yshift=-2pt] at (2.9, 0) {$S_1$};
        \node[below, yshift=-2pt] at (5.3, 0) {$D_1$};

        % Precio y Cantidades con Arancel (Nación Pequeña)
        \draw[dashed, color=gray!70] (0, 3.5) node[left] {$P_T = P_W + t$} -- (6.3, 3.5);
        \node at (4.0, 3.5) [circle, fill, inner sep=1.5pt] {};
        \node at (6.3, 3.5) [circle, fill, inner sep=1.5pt] {};
        \node[below, yshift=-2pt] at (4.0, 0) {$S_2$};
        \node[below, yshift=-2pt] at (6.3, 0) {$D_2$};

        % Áreas de pérdida y recaudación
        \draw[pattern=horizontal lines, pattern color=red!70] (5.3, 3) -- (6.3, 3) -- (6.3, 3.5) -- (5.3, 3) -- cycle;
        \draw[pattern=dots, pattern color=red!70] (2.9, 3) -- (4.0, 3) -- (4.0, 3.5) -- (2.9, 3.5) -- cycle;

        % Letras de las áreas, desplazadas para no pisar las líneas
        \node at (3.45, 3.25) [yshift=3pt] {b};
        \node at (4.65, 3.25) [yshift=3pt] {c};
        \node at (5.8, 3.25) [yshift=3pt] {d};
    \end{tikzpicture}

    \captionof{figure}{Efecto de un Arancel en una Nación Pequeña con Demanda Altamente Elástica (Esquema Conceptual)}
    \vspace{0.5em}
    % \footnotesize{\emph{Nota:} La curva de demanda domesticada ($D_N$) es casi horizontal, lo que implica que el área de pérdida por distorsión del consumo ($d$) es significativamente mayor, mientras que la producción ($S_1 \text{ a } S_2$) y la recaudación ($c$) son menores de lo que serían con una curva de demanda menos elástica.}
\end{figure}
\begin{anotacion}
La curva de demanda domesticada ($D_N$) es casi horizontal, lo que implica que el área de pérdida por distorsión del consumo ($d$) es significativamente mayor, mientras que la producción ($S_1 \text{ a } S_2$) y la recaudación ($c$) son menores de lo que serían con una curva de demanda menos elástica.
\end{anotacion}


\subsection{Conclusión}

La \textbf{elasticidad de la demanda} juega un papel crucial en la determinación de los costos de bienestar de la política arancelaria. En la práctica, una demanda doméstica altamente elástica para un bien importado (Bien X) hace que la restricción comercial sea \textbf{mucho más costosa} para la economía.

\begin{itemize}
    \item Para la \textbf{Nación Pequeña}, donde la pérdida de bienestar es siempre la regla, la alta elasticidad magnifica la pérdida por distorsión del consumo (área $d$), confirmando que el arancel impone un alto costo al bienestar nacional.
    \item Para la \textbf{Nación Grande}, la elasticidad reduce la probabilidad de que el arancel sea óptimo. Aunque el país obtiene una ganancia por la mejora de la relación de intercambio (área $e$) al reducir el precio extranjero, esta ganancia se aplica a un volumen de importaciones ($Q_T$) severamente reducido debido a la sensibilidad del consumidor. Las pérdidas de eficiencia ($b+d$) tienden a superar esta reducida ganancia.
\end{itemize}

En resumen, la elasticidad de la demanda determina la magnitud de la respuesta en la cantidad consumida. Cuando la demanda es muy elástica, cualquier arancel provoca una \textbf{gran pérdida de eficiencia social} debido a las distorsiones de consumo, lo que debilita fuertemente la justificación económica para la intervención proteccionista, incluso para una nación grande que busque un arancel óptimo.
