\chapter{El comercio internacional, la política comercial y la integración económica (casos, titulares y práctica)}

\section{El comercio internacional y la movilidad de factores productivos}

\subsection{Introducción}

El estudio del comercio y la movilidad de factores se centra en por qué los países intercambian bienes y servicios y cómo los factores de producción (trabajo y capital) se mueven internacionalmente. La macroeconomía internacional examina estos flujos, contrastando las predicciones de los modelos teóricos con la evidencia empírica global. El \textbf{Caso 2.1 (Mapa del Comercio Internacional)} destaca la distribución geográfica de los flujos comerciales: Europa posee el mayor volumen interno, facilitado por la integración regional (Unión Europea) y la proximidad. El comercio global está dominado por Europa y América, aunque Asia (con un 29\% del comercio mundial) exhibe modelos productivos diversos, desde el modelo de salarios bajos de China hasta la alta productividad intensiva en capital de Japón, y el modelo de exportación de servicios de la India. Otras regiones, como Oriente Medio, basan sus exportaciones en recursos naturales (petróleo y gas natural).

\subsection{La teoría clásica}
Los modelos clásicos se fundamentan en el factor trabajo como único insumo productivo y buscan determinar la estructura y las ganancias del comercio.

\subsubsection{El modelo de la ventaja absoluta}
El modelo de la ventaja absoluta de Adam Smith postula que las naciones se benefician mutuamente al especializarse en la producción de aquellos bienes en los que son absolutamente más eficientes (requieren menor valor trabajo). El \textbf{Caso 2.3 (Canadá y Nicaragua)} ilustra este principio: el clima continental de Canadá le otorga una ventaja en la producción de trigo, mientras que Nicaragua posee una ventaja absoluta en la producción de plátanos gracias a su clima tropical. La especialización y el intercambio subsecuente generan superávit comercial para ambos, confirmando los beneficios de la especialización basada en las condiciones naturales y productivas. La \textbf{Práctica del Tema 2} ejemplifica este cálculo: una Nación (N) con un menor valor trabajo en el bien X se especializará y exportará X, mientras que un país Extranjero (E) hará lo propio con el bien Y.

\subsubsection{El modelo de la ventaja comparativa}
La Ley de la Ventaja Comparativa de David Ricardo extiende el principio al demostrar que el comercio es mutuamente beneficioso incluso si una nación no posee ventaja absoluta en ningún bien. La especialización debe darse en el bien cuyo coste de oportunidad sea menor. El \textbf{Caso 2.2 (Protestas de los agricultores franceses)} refleja la confrontación de ventajas comparativas tras el ingreso de España en la Comunidad Económica Europea (CEE). Los agricultores franceses temían la competencia de los productos hortofrutícolas españoles, producidos a menor coste debido a las condiciones climáticas y la mayor producción por hectárea. El argumento español se basaba en el beneficio mutuo, donde España exportaría productos agrícolas (su ventaja comparativa) a cambio de productos manufacturados de alta tecnología (ventaja comparativa de la CEE). De manera análoga, el \textbf{Caso 2.4 (Rosas por San Valentín)} ilustra la reciprocidad de la ventaja comparativa: la producción de rosas a bajo coste en Colombia (debido al clima y mano de obra) se intercambia por procesadores de ordenador de alta tecnología de EE. UU., en los que este último tiene ventaja. El ingreso de divisas de EE. UU. por sus exportaciones tecnológicas a Colombia supera el gasto en la importación de flores, refutando la visión proteccionista de que las importaciones siempre perjudican la economía.

\subsubsection{Ganancias del comercio}
El \textbf{Caso 2.7 (Ganancias del Comercio)} busca cuantificar el impacto de la apertura comercial. Ejemplos históricos, como el embargo de EE. UU. (1807–1809) y la apertura forzada de Japón (1854), muestran que las ganancias del comercio son significativas, estimándose en 4-5\% del PIB, principalmente debido a la mejora en la relación de intercambio (subida de precios de exportaciones y bajada de precios de importaciones) y el consecuente aumento del bienestar.

\subsection{La dotación de factores y la teoría de Heckscher-Ohlin (H-O)}
La teoría H-O explica el patrón comercial a través de las diferencias en la dotación de factores (capital, trabajo, tierra) entre países.

\subsubsection{La teoría de H-O y la evidencia empírica}
El \textbf{Teorema H-O} establece que un país exportará el bien que utiliza intensivamente su factor de producción abundante e importará el bien intensivo en su factor escaso. Sin embargo, la \textbf{Paradoja de Leontief} (discutida en el \textbf{Caso 2.12}) desafió esta teoría, al encontrar que EE. UU., el país más intensivo en capital, exportaba bienes más intensivos en trabajo e importaba bienes más intensivos en capital.

Esta discrepancia se explica en parte por el \textbf{Caso 2.8 (Intensidades Factoriales)}, que demuestra que las intensidades factoriales no son idénticas entre países. La tecnología difiere: EE. UU. puede producir automóviles con alta automatización (intensivo en capital), mientras que India produce el mismo automóvil con mucha más mano de obra (intensivo en trabajo). Si las tecnologías difieren o la calidad del factor (trabajo o capital) varía, se rompe el supuesto de la teoría H-O, aunque el principio subyacente de la especialización basada en la abundancia factorial sigue siendo relevante.

El \textbf{Caso 2.9 (Evolución Capital China)} ofrece una perspectiva dinámica a la teoría H-O. China pasó de ser escasa a \textbf{abundante en capital físico} desde 2010, y más recientemente (2017), se volvió \textbf{abundante en científicos de I+D efectivos} (ajustados por productividad). Esta rápida evolución en su dotación de factores predice un cambio en su ventaja comparativa, trasladándose de la producción intensiva en mano de obra barata a la competencia directa con países desarrollados en bienes intensivos en capital y tecnología.

\subsubsection{La igualación del precio de los factores y la remuneración}
El \textbf{Teorema de Igualación de Precios de los Factores} (H-O-S) predice que el comercio internacional, al especializar la producción, igualará las remuneraciones (salarios y renta del capital) entre países, actuando como sustituto de la movilidad de factores. El \textbf{Teorema de Stolper-Samuelson} se enfoca en la distribución de la renta: el comercio beneficia al factor productivo abundante y perjudica al escaso.

El \textbf{Caso 2.13 (Comercio, Pobreza y Desigualdad)} evalúa estas predicciones. Mientras que el comercio contribuyó significativamente a reducir la \textbf{pobreza global} (sacando a millones de chinos de la pobreza extrema), ha exacerbado la \textbf{desigualdad interna} tanto en países desarrollados (EE. UU.) como en desarrollo (China), ya que los trabajadores cualificados ganan más que los poco cualificados, un resultado que contradice la predicción simple de H-O. Además, el \textbf{Caso 2.10 (Opiniones sobre Libre Comercio)} demuestra que la postura a favor o en contra del libre comercio está directamente determinada por el interés económico personal del individuo (sector de empleo, nivel de cualificación y educación, y localización territorial), reflejando la dicotomía de ganadores y perdedores predicha por los modelos factoriales.

\subsection{La movilidad del trabajo y los flujos financieros y de capital}

\subsubsection{La movilidad internacional del factor trabajo}
La movilidad internacional del trabajo genera ganancias de bienestar global, ya que el trabajo se mueve de países con menor Producto Marginal del Trabajo (PMgL) a países con mayor PMgL, donde es más productivo. En el corto plazo (Modelo de Factores Específicos), la inmigración suele presionar a la baja los salarios de los trabajadores nativos poco cualificados, beneficiando a los dueños del factor específico (capitalistas o terratenientes) al reducirse el coste laboral.

El \textbf{Caso 2.19 (Efectos Migración en Salarios)} cuantifica este efecto: a corto plazo (capital fijo), las estimaciones sugerían una caída salarial promedio de $-3.0\%$. Sin embargo, en el largo plazo (Modelo H-O modificado, con capital móvil), el efecto salarial promedio se neutraliza $(+0.1\%)$, ya que la economía se ajusta mediante la expansión del sector intensivo en ese factor (Teorema de Rybczynski).

El \textbf{Caso 2.15 (Éxodo de Mariel)} evaluó el impacto de la llegada de cubanos (trabajo no cualificado) a Miami. La predicción del \textbf{Teorema de Rybczynski} (expansión del sector intensivo en el factor abundante) se cumplió solo parcialmente, observándose una adopción tecnológica más lenta y una estabilidad salarial.

El \textbf{Caso 2.21 (Ganancias Inmigración)} muestra que, si bien hay efectos redistributivos importantes, la inmigración genera ganancias netas para el país receptor. Estas ganancias son mayores (hasta 4\% del PIB) si la migración incluye trabajadores STEM altamente cualificados, que aumentan la productividad e innovación, como argumentan algunos modelos. Esto se relaciona con el \textbf{Caso 2.16 (Inmigración Ilegal EE. UU.)}, donde los empleadores se benefician de las visas especializadas (H-1B para cualificados, H-2A para agrícolas) a través de menores costes laborales.

Finalmente, el \textbf{Caso 2.20 (Remesas)} aborda la pérdida para los países de origen: aunque las remesas son un ingreso significativo, no compensan necesariamente la pérdida de mano de obra cualificada (fuga de cerebros), lo que ha llevado a proponer un impuesto sobre los ingresos de los emigrantes calificados para compensar parcialmente a las naciones de origen.

\subsubsection{La inversión extranjera y la empresa multinacional}
La Inversión Extranjera Directa (IED) son flujos internacionales de capital que permiten a las empresas multinacionales establecer filiales, buscando beneficios y transferencia de tecnología. El \textbf{Caso 2.23 (Ganancias IED)} muestra que la IED genera beneficios claros para el país receptor, incluyendo salarios más altos (un 7\% más en promedio, aunque con variaciones, como el pago menor de empresas chinas) y un efecto multiplicador en el empleo local (0.4 empleos locales creados por cada empleo extranjero). El caso de Chile, que se convirtió en exportador de vino de clase mundial tras la llegada de IED, ilustra la transferencia tecnológica y de gestión.

El \textbf{Caso 2.22 (IED y Salarios en Singapur)} analizó los efectos de la IED masiva: a corto plazo, la entrada de capital aumenta el Salario Real (W) y reduce la Renta del Capital (RK). A largo plazo, sin embargo, el crecimiento salarial sostenido y la estabilidad de la RK se explican por el significativo \textbf{Crecimiento de la Productividad Total de los Factores (PTF)}, que supera los rendimientos decrecientes. Esto contrasta con la tesis del \textbf{Titular 2.2 (Mito del Milagro Asiático)}, que advertía que el crecimiento asiático basado solo en la acumulación de factores (sin aumento de productividad) estaba sujeto a la limitación de los rendimientos decrecientes.

\section{La política comercial internacional e instituciones}

\subsection{Beneficios del libre comercio}
El libre comercio maximiza el bienestar, medido por el Excedente del Consumidor (EC) y el Excedente del Productor (EP), generando una ganancia neta de eficiencia estática al precio mundial.

\subsection{Los aranceles a la importación en nuestro país pequeño}
La imposición de un arancel por un país pequeño (tomador de precios) aumenta el precio interno en la misma cuantía. El \textbf{Titular 3.1} describe las \textbf{disposiciones legales de EE. UU.} (Sección 201) que permiten imponer aranceles de \textbf{salvaguardia} si el aumento repentino de importaciones amenaza con causar daño grave a una industria. El \textbf{Caso 3.1 (Aranceles al Acero y Neumáticos EE. UU.)} analiza la aplicación de estos aranceles, cuantificando la pérdida de eficiencia (\textit{deadweight loss}) que genera esta medida bajo el modelo de país pequeño.

\subsection{Los aranceles a la importación en nuestro país grande}
Un país grande tiene poder de mercado para reducir el precio que pagan los exportadores extranjeros, logrando una \textbf{ganancia por términos de intercambio}. El \textbf{Caso 3.4 (Aranceles al Acero EE. UU. de nuevo)} señala que EE. UU. es ahora considerado un país grande en el comercio de acero. Este caso utiliza el cálculo del \textbf{arancel óptimo}, el cual maximiza el bienestar de la Nación. Si la elasticidad de la oferta de exportación extranjera es baja (inelasticidad), el arancel óptimo puede ser muy alto; si es alta, el arancel óptimo es cercano a cero.

Los \textbf{Casos 3.2 y 3.5 (Aranceles Trump)} ejemplifican esta política usando las Secciones 201, 232 (seguridad nacional) y 301 (propiedad intelectual) para imponer aranceles masivos (ej. al acero y a China). Aunque hubo una ganancia por los términos de intercambio, la \textbf{pérdida al consumidor} (por precios más altos) superó esta ganancia, resultando en una pérdida neta de bienestar para EE. UU. de 7.8 mil millones de dólares. La \textbf{Práctica del Tema 3} subraya que si la curva de demanda es muy elástica, el arancel maximiza la pérdida del consumidor (si no hay efecto país grande), aunque en un país grande, la recaudación arancelaria es parcialmente asumida por el extranjero.

\subsection{Otros instrumentos de política comercial}
\subsubsection{Las cuotas de importación}
Una cuota de importación restringe la cantidad física importada, replicando los efectos de un arancel en precios y distorsiones, pero genera \textbf{rentas de la cuota}. La asignación de estas rentas es crucial. El \textbf{Caso 3.6 (China y el Acuerdo Multifibras - MFA)} ejemplifica el uso de cuotas por países desarrollados para restringir importaciones textiles de países en desarrollo. La expiración del MFA en 2005 confirmó los efectos de las cuotas: las exportaciones chinas se dispararon y los precios de los productos restringidos cayeron bruscamente, eliminando las rentas y mejorando el excedente del consumidor en los países importadores.

\subsection{Libre comercio versus proteccionismo}
El debate es tanto económico como político. El \textbf{Caso 3.3 (Economía Política de los Aranceles)} explica la persistencia del proteccionismo: aunque los costos para los consumidores son dispersos y pequeños individualmente, los beneficios para los productores son concentrados, dándoles un alto poder político para presionar por la imposición de aranceles.

\subsection{Los acuerdos comerciales y la política comercial}
El \textbf{Titular 3.2 (Guerra Comercial EE. UU.-China)} contextualiza la política comercial actual en el marco de la \textbf{teoría de juegos} (Dilema del Prisionero). La imposición mutua de aranceles por países grandes conduce a un Equilibrio de Nash subóptimo, donde ambos pierden bienestar. El acuerdo de "Fase Uno" fue una \textbf{tregua inestable} basada en el "Comercio Gestionado" (\textit{Managed Trade}), que no resolvió las distorsiones fundamentales, subrayando la necesidad de la cooperación multilateral.

\subsection{Las políticas comerciales en los países en vías de desarrollo}
La \textbf{Industrialización basada en la Exportación} (en contraste con la Sustitución de Importaciones) es clave para el éxito. El \textbf{Caso 3.8 (Éxito de Asia)} identifica el secreto de las Economías Asiáticas de Alto Crecimiento (HPAE): \textbf{altas tasas de ahorro}, fuerte \textbf{inversión en educación} (capital humano) y \textbf{estabilidad macroeconómica} con apertura comercial. Complementariamente, el \textbf{Caso 3.7 (Reservas Internacionales PVD)} explica la masiva acumulación de reservas por países en desarrollo (ej. China) tras crisis financieras, utilizándolas como \textbf{seguro de liquidez} contra ataques especulativos y como herramienta \textbf{mercantilista} para evitar la apreciación de la moneda y mantener la competitividad exportadora.

\section{La integración económica internacional}

\subsection{Tipos de integración económica}
La integración progresa desde el \textbf{Acuerdo Comercial Preferencial} hasta la \textbf{Unión Económica y Monetaria} (ej. la Eurozona), implicando grados crecientes de eliminación de barreras comerciales y de movilidad de factores, y una unificación de políticas económicas.

\subsection{Efectos estáticos de la unión aduanera}
Los efectos estáticos primarios son la creación y la desviación de comercio. El \textbf{Caso 4.1 (Ejemplo Numérico)} y el \textbf{Titular 4.1 (Canadá)} analizan estos efectos. La \textbf{creación de comercio} (sustitución de producción doméstica ineficiente por importaciones baratas del socio) siempre aumenta el bienestar. La \textbf{desviación de comercio} (sustitución de importaciones del proveedor mundial más eficiente por importaciones del socio, que solo es más barato por la eliminación del arancel interno) implica una pérdida de bienestar. El efecto neto de una Unión Aduanera es positivo si la creación de comercio supera la desviación.

\subsection{Efectos dinámicos de la unión aduanera}
Los efectos dinámicos son a largo plazo e incluyen el \textbf{efecto procompetitivo} (sustitución de monopolios nacionales por mercados más competitivos), las \textbf{economías de escala} (reducción del coste unitario al aumentar el tamaño del mercado) y el \textbf{estímulo a la inversión} (nacional y IED para evitar el Arancel Externo Común).

\subsection{Multilateralismo y regionalismo}
La teoría evalúa el regionalismo como \textbf{escalón} (optimista) o \textbf{obstáculo} (pesimista) al libre comercio multilateral. El \textbf{Titular 4.2 (OMC Debilitada)} sugiere que la OMC está fallando en su papel multilateral, llevando a los países a buscar el regionalismo. Este comportamiento es predicho por el Dilema del Prisionero, donde el regionalismo es a menudo el Equilibrio de Nash subóptimo.

\subsection{Los efectos de los acuerdos internacionales: cuestiones laborales y medioambientales}

\subsubsection{Las cuestiones laborales}
El debate laboral se centra en quién debe asegurar los estándares:
\begin{enumerate}
    \item \textbf{Responsabilidad del Consumidor (Caso 4.3):} Usar el poder de compra para forzar estándares.
    \item \textbf{Responsabilidad Corporativa (Caso 4.4 y Titular 4.3):} Grandes minoristas (ej. Wall-Mart) imponen estándares a sus proveedores globales.
    \item \textbf{Responsabilidad Pública (Caso 4.5 y Titulares 4.4/4.5):} Gobiernos vinculan el acceso preferencial a sus mercados con el cumplimiento de estándares de seguridad (ej. suspensión del estatus comercial de Bangladés por EE. UU.).
    \item \textbf{Salarios Dignos (Caso 4.6):} Debate sobre si el comercio debe garantizar un salario que cubra las necesidades básicas, afectando la ventaja comparativa.
\end{enumerate}

\subsubsection{Las cuestiones medioambientales: en el GATT y la OMC}
La OMC media en conflictos donde la regulación ambiental se usa como barrera. Los \textbf{Casos 4.7 (Atún-Delfín) y 4.8 (Camarón-Tortuga)} establecieron límites a la capacidad de un país de imponer sus estándares de producción y proceso (\textit{PPMs}) a otros, aunque el caso Camarón-Tortuga suavizó ligeramente esta postura si hay esfuerzos multilaterales. Los \textbf{Casos 4.9 (Gasolina) y 4.10 (Alimentos Transgénicos)} reafirman que las regulaciones deben aplicarse sin discriminación (trato nacional) y que el Principio de Precaución (UE) es un foco de disputa.

\subsubsection{Perjuicios o beneficios del comercio sobre el medioambiente}
Las políticas proteccionistas pueden generar consecuencias ambientales negativas: la \textbf{Cuota de Azúcar de EE. UU. (Caso 4.11)} promueve una producción doméstica menos eficiente. La \textbf{Restricción Voluntaria de Exportación Automovilística (Caso 4.12)} llevó a Japón a exportar vehículos más grandes y de mayor valor para maximizar ingresos dentro de la cuota, posiblemente aumentando las emisiones.

\subsubsection{La tragedia de la propiedad común}
El comercio, al aumentar la demanda, exacerba la sobreexplotación de recursos sin derechos de propiedad claros (\textbf{Tragedia de la Propiedad Común}). Los \textbf{Casos 4.13 (Pesca)} y \textbf{4.14 (Búfalo)} son ejemplos clásicos de agotamiento de recursos biológicos. Los \textbf{Casos 4.15 (Paneles Solares) y 4.16 (Minerales Tierras Raras)} vinculan la demanda de tecnología verde con las externalidades negativas (contaminación) generadas por la minería global.

\subsubsection{Acuerdos internacionales sobre la contaminación}
El \textbf{Caso 4.17 (Kyoto, París)} analiza los esfuerzos multilaterales para internalizar el coste de la contaminación (ej. Protocolo de Kyoto, Acuerdo de París). El \textbf{Titular 4.6 (Ley de California)} refleja la tendencia de imponer \textbf{aranceles de carbono} a las importaciones para combatir las emisiones.

\subsection{Procesos de integración}
El \textbf{Caso 4.2 (Brexit)} ilustra cómo la salida de un bloque de integración (UE) requiere negociaciones complejas, destacando la importancia crítica de las \textbf{reglas de origen} para mantener la condición de Área de Libre Comercio. Otros bloques, como el \textbf{TLCAN} (ahora T-MEC/USMCA), el fallido \textbf{ALCA} y la \textbf{APEC}, buscan la liberalización comercial y la cooperación económica regional.
