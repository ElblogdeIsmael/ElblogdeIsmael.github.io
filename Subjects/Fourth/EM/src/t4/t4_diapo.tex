\chapter{La integración económica internacional}

\section{Introducción}

La \textbf{integración económica internacional} se define como el proceso deliberado y progresivo mediante el cual dos o más países (Naciones) coordinan o unifican sus políticas económicas, reduciendo o eliminando las barreras al comercio y al movimiento de factores . Este proceso, si bien busca la eficiencia interna entre los miembros, genera inherentemente un trato \textbf{discriminatorio} frente a terceros países (Resto del Mundo, RM) . El estudio de la integración se centra en la evaluación del bienestar, analizando los \textbf{efectos estáticos} (creación y desviación de comercio) y los \textbf{efectos dinámicos} (competencia y economías de escala) resultantes de la formación de bloques económicos.

\section{Tipos de integración económica}

La integración económica se conceptualiza como un espectro jerárquico que abarca desde acuerdos arancelarios mínimos hasta la unificación total de políticas macroeconómicas, lo que implica una cesión progresiva de soberanía nacional:
\begin{enumerate}
    \item \textbf{Acuerdo Comercial Preferencial (ACP):} Es la forma más laxa, donde dos o más países reducen mutuamente sus gravámenes de importación sobre bienes y servicios específicos, manteniendo cada uno sus aranceles frente a terceros países y restricciones al movimiento de capital. Un ejemplo histórico de esta categoría es la \textit{Commonwealth} .
    \item \textbf{Área de Libre Comercio (ALC):} Los países miembros suprimen todas las barreras comerciales entre ellos, pero cada nación mantiene su \textbf{autonomía arancelaria} frente a terceros. El movimiento de capital puede seguir limitado. Ejemplos paradigmáticos incluyen el \textit{European Free Trade Area (EFTA)} y el \textit{Tratado de Libre Comercio de América del Norte (TLCAN)} .
    \item \textbf{Unión Aduanera (UA):} Los países miembros eliminan las barreras internas al comercio y, crucialmente, imponen un \textbf{Arancel Externo Común (AEC)} frente a terceros países. Esto elimina la necesidad de \textit{reglas de origen} internas, aunque puede mantener restricciones al movimiento de capitales. La Comunidad Económica Europea en 1968 sirve como referencia histórica. Dentro de esta categoría se pueden encontrar Zonas Fiscales Libres o Zonas Económicas Libres, diseñadas para atraer inversión extranjera directa (IED) mediante la exención de gravámenes a los factores productivos .
    \item \textbf{Mercado Común (MC):} Constituye una unión aduanera que, además, garantiza la \textbf{libre circulación de los factores productivos} (capital y trabajo). El \textit{Acta Única Europea de 1993} marcó la creación del Mercado Común de Europa .
    \item \textbf{Unión Económica y Monetaria (UEM):} Representa el grado máximo de integración. Sobre la base de un mercado común, los países unifican sus políticas \textbf{fiscales, monetarias y sociales}, adoptando una \textbf{moneda única}. La culminación de la Unión Europea a partir de enero de 2002 con la implantación del euro ejemplifica este estadio .
\end{enumerate}

\section{Efectos estáticos de la unión aduanera}

La evaluación de la eficiencia de una UA se fundamenta en el análisis de los efectos estáticos sobre el bienestar social, siendo los principales la creación y la desviación de comercio, tal como los desarrolló la teoría económica de la integración.

\subsection{El efecto creación de comercio}

El \textbf{efecto creación de comercio} ocurre cuando la formación de la UA resulta en la sustitución de la producción ineficiente y más costosa de un país miembro por importaciones más baratas y eficientes de otro país miembro .
Consideremos una Nación ($N$) que es un país pequeño (tomador de precios a nivel mundial) y que se integra en una UA con el país $E$. Antes de la UA, $N$ aplicaba un arancel específico $t$ a todas las importaciones, elevando el precio interno a $p = p_{N} + t$ . Tras la integración, $N$ elimina el arancel sobre $E$, y el nuevo precio interno de la UA ($p_{UA}$) se establece, siendo $p_{UA} < p_{N} + t$.
El análisis del excedente demuestra que la creación de comercio contribuye a \textbf{aumentar el bienestar} total de la Nación . Esto se debe a que la reducción del precio de $p_{N} + t$ a $p_{UA}$ genera:
\begin{itemize}
    \item \textbf{Ganancia en Excedente del Consumidor (EC):} El área $(A+B+C+D)$ se suma al EC.
    \item \textbf{Pérdida en Excedente del Productor (EP):} El área $A$ se pierde, lo que refleja el \textbf{efecto producción} (sustitución de producción nacional costosa por importaciones).
    \item \textbf{Pérdida de Recaudación Arancelaria:} El área $C$ se pierde por el gobierno al eliminarse el arancel sobre $E$.
\end{itemize}
El efecto neto sobre el bienestar es $-(A) + (A + B + C + D) - (C) = + (B + D)$ . Las áreas $B$ y $D$ representan las \textbf{ganancias netas de eficiencia}, derivadas de la corrección de la distorsión en el consumo ($D$) y la producción ($B$).

\subsection{El efecto desviación de comercio}

El \textbf{efecto desviación de comercio} ocurre cuando un país miembro cambia sus importaciones de un país del Resto del Mundo (RM), que era el proveedor de coste más bajo, a un país de la UA, que ahora es el proveedor de coste más bajo dentro del bloque debido a la eliminación arancelaria. Esto sucede si el precio mundial ($p_{RM}$) es menor que el precio dentro de la UA ($p_{UA}$), pero el arancel previo ($t$) hacía que solo se importara de $E$ (el socio de la UA) .

La desviación de comercio implica una \textbf{pérdida de bienestar} para el país, ya que la nación consume un bien a un precio que, si bien es menor que el precio arancelado anterior, es superior al coste de producción mundial.
El efecto neto sobre el bienestar total de la Nación es $(B + D) - E$, donde $B$ y $D$ son las ganancias de eficiencia (creación de comercio) y $E$ es la \textbf{pérdida de bienestar} derivada de la desviación de comercio . Si la pérdida $E$ supera la suma de las ganancias de eficiencia $(B+D)$, la UA reduce el bienestar nacional.

Este equilibrio entre creación y desviación es crucial para juzgar la conveniencia de una UA. Por ejemplo, el análisis de la \textit{Trade Creation and Diversion for Canada} (Feenstra y Taylor, 2021) aplica estos modelos para cuantificar los efectos reales de acuerdos como el TLCAN sobre la economía canadiense . De manera ilustrativa, un \textit{Numerical Example of Trade Creation and Diversion} (Feenstra y Taylor, 2021) detalla cómo calcular estas áreas de excedente .

\subsection{Otros efectos estáticos}

Además de los efectos primarios de creación y desviación de comercio, la teoría de la integración económica identifica otros efectos estáticos:
\begin{itemize}
    \item \textbf{Efecto Ahorro Administrativo:} Derivado de la eliminación de los controles aduaneros internos entre los países miembros, lo que libera recursos burocráticos y reduce costos de transacción.
    \item \textbf{Efecto Mejora de la Relación de Intercambio (RDI) de la UA:} Si la UA actúa como un \textbf{país grande} en el mercado global, la disminución de las importaciones del RM (debido a la desviación de comercio) y el aumento de la capacidad de exportación al RM (debido a la creación de comercio) pueden colectivamente mejorar sus términos de intercambio .
    \item \textbf{Efecto Mayor Capacidad Negociadora:} La UA, al concentrar un mercado significativo, adquiere mayor \textbf{poder de negociación} para alcanzar acuerdos comerciales favorables con el RM .
\end{itemize}

\section{Efectos dinámicos de la unión aduanera}

Los efectos dinámicos, que se manifiestan a largo plazo, son a menudo más significativos que los estáticos, ya que influyen en el potencial de crecimiento económico sostenido de la UA .

\subsection{El efecto procompetitivo}

La integración favorece la sustitución de estructuras de mercado nacionales caracterizadas por monopolios u oligopolios por un mercado más amplio y \textbf{más competitivo}. Esto reduce el poder de mercado de las empresas y obliga a una bajada de precios y un aumento de la producción .
Un ejemplo histórico de este \textbf{efecto procompetitivo} es la transición del sector de telecomunicaciones, donde empresas como Telefónica en España gozaban de monopolios nacionales antes del \textit{Mercado Único} .
El efecto neto de bienestar derivado de una mayor competencia se calcula como la suma de la ganancia en el Excedente del Consumidor ($A+D$) y la variación neta en el beneficio empresarial $(B-A)$, resultando en una ganancia positiva de $(D+B)$ .

\subsection{El efecto economías de escala}

La ampliación del tamaño del mercado facilitada por la UA permite a las empresas explotar \textbf{economías de escala}, lo que reduce el coste medio o unitario de producción ($c$) hasta alcanzar el \textbf{Punto de Mínima Escala Eficiente (EME)} .
La ganancia potencial de la integración es directamente proporcional a la diferencia entre el nivel de producción actual ($Q$) y la EME. Cuanto mayor sea esta diferencia, y cuanto mayor sea el coste unitario inicial ($c$), mayor será la ganancia potencial al integrarse, ya que se reduce la ineficiencia productiva .

\subsection{El efecto estímulo de la inversión}

La integración crea un entorno propicio para el aumento de la inversión por varias razones:
\begin{itemize}
    \item Los empresarios internos se motivan a invertir para aprovechar el mercado ampliado y para hacer frente a la competencia creciente .
    \item La formación de una UA con un AEC a menudo incentiva la \textbf{Inversión Extranjera Directa (IED)} por parte de empresas del RM, buscando evitar las barreras comerciales discriminatorias impuestas por el bloque. Este fenómeno se conoce como \textit{tariff-jumping FDI} .
\end{itemize}

\subsection{Otros efectos dinámicos}

Otros efectos dinámicos incluyen:
\begin{itemize}
    \item \textbf{Cambio Tecnológico:} La intensificación de la competencia estimula la investigación, el desarrollo de nuevos productos y la adopción de métodos de producción más eficientes, creando un clima propicio para la innovación .
    \item \textbf{Eliminación de la Discriminación de Precios:} En un mercado integrado, los oferentes tienden a vender el producto al mismo precio en toda la UA (ajustado por costes de transporte), eliminando las prácticas de discriminación de precios que eran viables en los mercados nacionales segmentados .
\end{itemize}

\section{Multilateralismo y regionalismo}

Históricamente, el ideal económico propugna un \textbf{acuerdo mundial de carácter multilateral y no discriminatorio}, en línea con los objetivos de la OMC . Sin embargo, la persistencia de bloques regionales plantea el debate entre multilateralismo (integración global) y regionalismo (integración por bloques).

Existen dos conjeturas principales sobre la dinámica entre ambos:
\begin{enumerate}
    \item \textbf{La Conjetura del Escalón (Visión Optimista):} Los bloques regionales sirven como un \textbf{paso intermedio} o escalón necesario para alcanzar un sistema multilateral de libre comercio abierto. Se consideran complementarios al sistema global.
    \item \textbf{La Conjetura del Obstáculo (Visión Pesimista):} Los bloques regionales \textbf{dificultan} la consecución del libre comercio multilateral, ya que los países pueden conformarse con los resultados regionales, generando resultados inferiores a los que se alcanzarían en un escenario global de libre comercio .
\end{enumerate}

Este conflicto puede modelarse como un juego de estrategias (Dilema del Prisionero) donde los países $N$ y $E$ eligen entre Integración Regional ($R$) o Acuerdo Mundial Multilateral ($M$) . La matriz de pagos resultante a menudo revela:
\begin{itemize}
    \item \textbf{Estrategia Dominante (Equilibrio de Nash):} La opción $R$ (regionalismo) puede ser la mejor opción unilateral para cada país, incluso si la solución resultante ($R/R$) es subóptima para el bienestar de ambos en comparación con el resultado de cooperación.
    \item \textbf{Estrategia Cooperativa:} La solución $M/M$ (libre comercio multilateral) es el resultado más deseable, pero requiere un acuerdo vinculante para evitar la tentación de la desviación unilateral .
\end{itemize}
En la práctica, la \textit{Organización Mundial del Comercio (OMC)} enfrenta dificultades para prevenir que los acuerdos regionales socaven el sistema multilateral, lo que sugiere que la organización se está debilitando, como se discute en el titular \textit{The World Trade Organization Is Faltering. The US Can’t Fix It Alone} (Feenstra y Taylor, 2021) .

\section{Los efectos de los acuerdos internacionales: cuestiones laborales y medioambientales}

Los acuerdos comerciales modernos (como los negociados en las Rondas del GATT/OMC) se extienden más allá de los aranceles para abordar cuestiones transversales como los derechos laborales y la sostenibilidad ambiental.

\subsection{Las cuestiones laborales}

El comercio internacional plantea desafíos éticos y de gobernanza respecto a las condiciones laborales, especialmente en países en desarrollo. La respuesta ante estas cuestiones requiere una acción coordinada entre diversos actores:

\subsubsection{La responsabilidad de los consumidores}
Los consumidores tienen un papel en la promoción de estándares laborales mediante sus decisiones de compra, aunque la implementación de esta \textit{Consumer Responsibility} se enfrenta a retos de información y coordinación, como se detalla en el caso de Feenstra y Taylor (2017) .

\subsubsection{La responsabilidad corporativa}
Las grandes empresas transnacionales son responsables de sus cadenas de suministro globales (\textit{Corporate Responsibility}). Ejemplos como el caso \textit{Wall-Mart Orders Chinese Suppliers to Lift Standards} (Feenstra y Taylor, 2017) ilustran cómo las grandes minoristas pueden imponer estándares laborales y de producción más altos a sus proveedores, incluso en ausencia de regulación gubernamental estricta . Las corporaciones, como Nike, han llegado a desvelar su cadena de suministro como parte de una estrategia de transparencia .

\subsubsection{La responsabilidad de los poderes públicos}
Los gobiernos nacionales tienen la responsabilidad de asegurar condiciones laborales mínimas (\textit{Country Responsibility}). Un caso notable de la intervención gubernamental fue la suspensión del estatus comercial preferencial de Bangladés por parte de EE. UU. tras el colapso de una fábrica, vinculando explícitamente el acceso a mercados con el cumplimiento de estándares de seguridad laboral (véase \textit{American Tariffs, Bangladeshi Deaths} y \textit{U.S. Suspends Bangladesh’s Preferential Trade Status}, Feenstra y Taylor, 2017) .

\subsubsection{Salarios dignos}
El debate sobre los \textbf{salarios dignos} (\textit{Living Wage}) se centra en si el comercio debería garantizar que los trabajadores en los países en desarrollo reciban una remuneración que cubra sus necesidades básicas, un tema con implicaciones directas en la estructura de costos y la ventaja comparativa de esas naciones (Feenstra y Taylor, 2017) .

\subsection{Las cuestiones medioambientales: en el GATT y la OMC}

El GATT y la OMC se han enfrentado al desafío de conciliar las regulaciones ambientales nacionales con los principios de libre comercio, especialmente cuando las medidas nacionales se perciben como barreras comerciales disfrazadas.

\subsubsection{Los casos del atún y del delfín}
El caso \textit{Tuna-Dolphin} (Feenstra y Taylor, 2021) fue uno de los primeros en evidenciar el conflicto. EE. UU. prohibió la importación de atún pescado con métodos que dañaban a los delfines. El panel del GATT determinó que EE. UU. no podía imponer sus estándares de proceso y producción (PPMs) a otros países, pues esto infringía el Artículo III (trato nacional), a menos que el producto en sí mismo fuera diferente, sentando un precedente complejo .

\subsubsection{Los asuntos del camarón y la tortuga}
En el caso \textit{Shrimp-Turtle} (Feenstra y Taylor, 2021), EE. UU. prohibió las importaciones de camarón que no estuvieran pescados con dispositivos de exclusión de tortugas (TEDs). Si bien inicialmente fue rechazado, la OMC revisó su postura, permitiendo la medida bajo el Artículo XX (Excepciones Generales) del GATT, siempre que se aplicara de manera no discriminatoria y con esfuerzos de negociación para acuerdos multilaterales, marcando una ligera apertura a considerar preocupaciones ambientales en los PPMs .

\subsubsection{Gasolina procedente de Venezuela y Brasil}
El caso \textit{Gasoline from Venezuela and Brazil} (Feenstra y Taylor, 2021) se centró en las regulaciones de EE. UU. sobre la calidad de la gasolina. La OMC encontró que estas regulaciones discriminaban contra las importaciones de Venezuela y Brasil, violando el Artículo III. El caso subrayó que, si bien la protección ambiental es legítima, las regulaciones deben aplicarse de manera equitativa a productos importados y domésticos .

\subsubsection{Alimentos transgénicos en Europa}
El caso \textit{Biotech Food in Europe} (Feenstra y Taylor, 2021) ilustró el conflicto sobre el Principio de Precaución, donde la Unión Europea impuso restricciones a la importación de alimentos transgénicos. EE. UU. y otros argumentaron que esto carecía de base científica, mientras que la UE priorizaba la salud pública y la precaución. Estos casos ilustran el desafío de armonizar los estándares de seguridad alimentaria en el comercio global .

\subsection{Perjuicios o beneficios del comercio sobre el medioambiente}

La relación entre comercio y medioambiente no es unidimensional, mostrando tanto perjuicios como beneficios.

\subsubsection{Cuota estadounidense sobre el azúcar}
Las \textit{U.S. Trade Restrictions in Sugar and Ethanol} (Feenstra y Taylor, 2017) demuestran cómo las políticas proteccionistas (como las cuotas sobre el azúcar) pueden tener efectos negativos colaterales en el medioambiente. Al distorsionar el mercado, estas políticas pueden incentivar una producción doméstica menos eficiente o con mayores externalidades negativas .

\subsubsection{La restricción voluntaria de las exportaciones en la automoción estadounidense}
La \textit{U.S. Automobile VER} (Feenstra y Taylor, 2017) impuesta a Japón en los años 80 tuvo la consecuencia inesperada de impulsar a los exportadores japoneses a enviar modelos de mayor valor y tamaño para maximizar sus ingresos dentro de la cuota, lo que en algunos casos aumentó el consumo de combustible y las emisiones, empeorando indirectamente el impacto ambiental .

\subsection{La tragedia de la propiedad común}

La \textbf{tragedia de la propiedad común} ocurre cuando los recursos no tienen derechos de propiedad claros, lo que lleva a su sobreexplotación por parte de los agentes económicos. El comercio internacional puede exacerbar este problema al aumentar la demanda y, por lo tanto, el incentivo para la sobreexplotación.

\subsubsection{El comercio de la pesca}
El \textit{Trade in Fish} (Feenstra y Taylor, 2021) es el ejemplo clásico. El acceso abierto a los caladeros globales (propiedad común) y la alta demanda internacional generan una sobrepesca que amenaza la sostenibilidad de las especies. Sin regulaciones de acceso o cuotas, el comercio acelera el agotamiento de los recursos pesqueros .

\subsubsection{El comercio del búfalo}
El \textit{Trade in Buffalo} (Feenstra y Taylor, 2021) ilustra cómo la caza indiscriminada impulsada por la demanda de mercado puede llevar a la extinción o casi extinción de especies, un resultado directo de la propiedad común y la falta de incentivos para la conservación .

\subsubsection{Comercio de paneles solares y algunos minerales}
El \textit{Trade in Solar Panels} (Feenstra y Taylor, 2021) y \textit{Trade in Rare Earth Minerals} (Feenstra y Taylor, 2017) abordan el problema en el contexto de la energía verde y las tecnologías avanzadas. La minería de minerales de tierras raras, esencial para paneles solares y electrónica de alta tecnología, a menudo genera graves externalidades negativas (contaminación) debido a la búsqueda de bajos costes de producción para satisfacer la demanda global .

\subsection{Acuerdos internacionales sobre la contaminación}

\subsubsection{La lógica de la regulación contra la contaminación}
La solución económica a las externalidades negativas (como la contaminación) es la \textbf{internalización de costos}, generalmente a través de impuestos (como un impuesto al carbono o un impuesto \textit{Pigouviano}) o mediante la regulación directa. Cuando la contaminación es transfronteriza, se requieren acuerdos internacionales para evitar la \textit{carrera hacia el fondo} (competencia regulatoria a la baja) . La \textit{California Law Aims to Tackle Imported Emissions} (Feenstra y Taylor, 2021) ejemplifica el intento de imponer estándares a las importaciones, a menudo a través de un \textbf{arancel de carbono} (\textit{carbon tariff}) .

\subsubsection{El protocolo de Kyoto.}
El \textit{Kyoto Protocol}, el \textit{Paris Agreement} y el \textit{Green Deal} (Feenstra y Taylor, 2021) representan los principales intentos multilaterales para abordar el cambio climático. El Protocolo de Kyoto (1997) estableció objetivos vinculantes de reducción de emisiones, aunque la exclusión de grandes emisores como China y EE. UU. limitó su efectividad. El Acuerdo de París (2015) adoptó un enfoque más flexible basado en contribuciones determinadas a nivel nacional. La implementación de tales protocolos subraya la tensión entre la mitigación climática y el mantenimiento de la competitividad económica .

\section{La Unión Europea (UE)}

La \textbf{Unión Europea} es el proceso de integración económica más ambicioso y avanzado. Sus orígenes se remontan a 1957 con el \textbf{Tratado de Roma}, que estableció la Comunidad Económica Europea (CEE) y buscó coordinar la política monetaria y suprimir barreras comerciales entre los seis miembros fundadores (Bélgica, Francia, Italia, Luxemburgo, Holanda y Alemania Occidental) .

Hitos clave incluyen:
\begin{itemize}
    \item \textbf{1979:} Creación del Sistema Monetario Europeo (SME) con el objetivo de reducir la inestabilidad monetaria y controlar la inflación .
    \item \textbf{1986:} Aprobación del Acta Única Europea (AUE), que creó el Mercado Interior Unificado (MIU), eliminando restricciones al comercio y permitiendo la libre circulación de factores .
    \item \textbf{1992:} Firma del Tratado de la Unión Europea (TUE) en Maastricht, que impulsó definitivamente la \textbf{Unión Económica y Monetaria (UEM)} .
\end{itemize}

Para acceder a la tercera fase de la UEM (la moneda única), los países debían cumplir los \textbf{criterios de convergencia} nominal establecidos en Maastricht :
\begin{enumerate}
    \item \textbf{Estabilidad de Precios:} Inflación no superior en un 1.5\% a la media de los tres países con menor inflación.
    \item \textbf{Sostenibilidad de las Finanzas Públicas:} Déficit público no superior al 3\% del PIB y deuda pública no superior al 60\% del PIB.
    \item \textbf{Estabilidad Cambiaria:} Mantener la divisa dentro de las bandas de fluctuación del 2.25\% del Mecanismo de Tipos de Cambio (MTC, o ERM II para los nuevos miembros) durante al menos dos años (aunque esta banda se amplió tácitamente al $\pm 15\%$ en 1993) .
    \item \textbf{Convergencia en Tipos de Interés a Largo Plazo:} El tipo medio no podía exceder en más de un 2\% a la media de los tipos a largo plazo de los tres países con menor inflación .
\end{enumerate}

\section{El Tratado de Libre Comercio de América del Norte (TLCAN)}

El \textbf{TLCAN} (NAFTA), que entró en vigor el 1 de enero de 1994, integró a \textbf{EE. UU., Canadá y México}, estableciendo una zona de libre comercio con el objetivo de crear un mercado amplio y seguro de bienes y servicios, garantizar derechos de propiedad intelectual, liberalizar la inversión y desarrollar protocolos laborales y ambientales .
El TLCAN tiene sus raíces en el acuerdo de libre comercio previo entre Canadá y EE. UU. (CUSFTA). El acuerdo fue objeto de renegociación bajo la administración Trump, resultando en el \textbf{Acuerdo Estados Unidos-México-Canadá (T-MEC o USMCA)} , que introdujo cambios significativos, especialmente en las reglas de origen para el sector automotriz .

\section{El Área de Libre Comercio de las Américas (ALCA)}

El \textbf{ALCA} (FTAA) fue un proyecto que surgió de la Conferencia de Miami en 1994, con la meta de eliminar progresivamente las barreras al comercio y la inversión extranjera en todo el continente americano . A pesar de un calendario inicial ambicioso, el proyecto se paralizó debido a los cambios en la política estadounidense (cuando Bush sucedió a Clinton) y a la resistencia de bloques regionales sudamericanos, en particular \textbf{Brasil y Argentina}, que priorizaron el proyecto \textbf{MERCOSUR} .

\section{La Cooperación Económica Asia-Pacífico (APEC)}

La \textbf{APEC} fue fundada en noviembre de 1989 y agrupa a economías clave del Pacífico, incluyendo EE. UU., China, Japón, Corea del Sur y Australia . El principal desafío de la APEC, formalizado en el Plan de Acción de Manila de 1996, fue la plena \textbf{eliminación de los aranceles} antes de 2010 para los países industrializados y antes de 2020 para el resto, además de abordar temas como propiedad intelectual y competencia .

\section{Las Economías en Transición}

Tras la Segunda Guerra Mundial, la URSS promovió el \textbf{Consejo de Asistencia Mutua Económica (COMECON)} y el Pacto de Varsovia para los países de Europa del Este . El COMECON se caracterizó por una falta de precios de mercado y la hegemonía de la URSS, resultando en un subdesarrollo generalizado .

A partir de 1989, muchos de estos países iniciaron la \textbf{transición hacia una economía de mercado} y la instauración del orden democrático. Estos países en transición se dividieron en dos grupos :
\begin{enumerate}
    \item Los que adoptaron un \textbf{sistema capitalista} (ej., República Checa, Lituania, Estonia, Hungría).
    \item Los que \textbf{no completaron las reformas} necesarias (ej., Rusia, Polonia, Ucrania, Bielorrusia) .
\end{enumerate}
Varios de los países del primer grupo (Estonia, Letonia, Lituania, Polonia, República Checa, Eslovaquia, Eslovenia, Hungría, Chipre y Malta) se formalizaron como miembros de la Unión Europea en el Tratado de Atenas de 2004, integrándolos plenamente en el proceso de desarrollo económico y social occidental .

