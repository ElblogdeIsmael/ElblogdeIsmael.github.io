% =========================================================================
% INICIO DEL CAPÍTULO 2
% (El preámbulo y la configuración de estilos se asumen cargados previamente
% de acuerdo con las fuentes)
% =========================================================================

\chapter{El comercio internacional y el movimiento de los factores productivos}
\label{ch:tema2}

\vspace{1em}


\section*{Objetivos de Aprendizaje del Tema 2}
\addcontentsline{toc}{section}{Objetivos de Aprendizaje del Tema 2}

A partir del estudio de este tema, el alumno podrá:

\begin{enumerate}
    \item Entender por qué hay comercio internacional de bienes y servicios.
    \item Saber por qué hay movilidad internacional de los factores trabajo y capital.
    \item Comprender los planteamientos clásicos del comercio internacional: teorías de la ventaja absoluta y ventaja comparativa.
    \item Analizar los efectos del comercio internacional en términos de bienestar.
    \item Entender la importancia de la teoría estándar del comercio.
\end{enumerate}

\section{Introducción}

El estudio de la economía internacional se divide en dos grandes campos: el comercio internacional (transacciones reales) y las finanzas internacionales (transacciones monetarias y de activos). El comercio internacional de bienes y servicios, junto con el movimiento de factores productivos (trabajo y capital), constituyen la base del análisis del comercio real.

Los países participan en el comercio internacional y obtienen ganancias del mismo por dos razones fundamentales: las diferencias entre países (en recursos o tecnología) y la existencia de economías de escala. Estas razones dan lugar al patrón de comercio observado en el mundo.

\section{La teoría clásica}

Los modelos clásicos de comercio internacional se basan en el principio fundamental de que el comercio surge de las diferencias entre países. Históricamente, estos modelos se centraron en la productividad del trabajo como principal fuente de ventaja comparativa.

\subsection{Supuestos}

Los modelos clásicos, como el ricardiano, se basan en supuestos simplificadores, aunque poderosos, para ilustrar el principio de la ventaja comparativa. El modelo de David Ricardo opera bajo el supuesto de que el trabajo es el \textbf{único factor de producción} en una economía. La tecnología se resume en los \textbf{requerimientos unitarios de trabajo} ($a_{L i}$), que es el número de horas de trabajo necesarias para producir una unidad de un bien $i$, siendo la inversa de la productividad.

\subsection{El modelo de la ventaja absoluta}

El modelo de la ventaja absoluta (asociado a Adam Smith) establece que un país debe especializarse en la producción de aquellos bienes en los que es \textbf{más eficiente}. Un país tiene ventaja absoluta en un bien si utiliza menos horas de trabajo ($L$) para producir una unidad de ese bien, es decir, si el precio unitario o valor trabajo ($v$) es menor.

Si un país es más productivo que otro en la producción de un bien (ventaja absoluta), se infiere que ese país debería especializarse en la producción y exportación de ese bien. No obstante, la evidencia empírica sugiere que el patrón de comercio no se determina únicamente por la ventaja absoluta, sino por las ventajas relativas o comparativas.

\section{El modelo de la ventaja comparativa de David Ricardo}

El concepto de \textbf{ventaja comparativa} (introducido por David Ricardo) es central para entender el patrón de comercio.

\subsection{Concepto de ventaja comparativa}

La ventaja comparativa surge de las diferencias en el \textbf{coste de oportunidad} de la producción entre países. Un país tiene ventaja comparativa en la producción de un bien si el coste de oportunidad de producir ese bien es \textbf{inferior} en ese país que en el extranjero.

En el modelo ricardiano, el coste de oportunidad se determina por la ratio de requerimientos unitarios de trabajo entre los dos bienes. El país se especializará en el bien para el que su productividad laboral relativa (o su coste de oportunidad relativo) es mayor.

\subsection{Ganancias del comercio}

El comercio internacional permite una \textbf{reordenación mutuamente beneficiosa de la producción mundial}. Si cada país se especializa de acuerdo con su ventaja comparativa, la producción mundial total aumenta. Esto incrementa el tamaño de la "tarta económica mundial", lo que, en principio, hace posible elevar el nivel de vida de todos los países.

Aunque el modelo ricardiano simple predice un grado de especialización extremo que no se observa en la realidad y no considera la distribución de la renta, las ganancias agregadas del comercio se mantienen como un principio fundamental, incluso en enfoques más generales (como el Modelo Estándar de Comercio).

\section{Extensiones al modelo clásico y nuevos modelos}

Los modelos clásicos iniciales fueron ampliados para incorporar la distribución de la renta y otros determinantes del comercio, como las diferencias de recursos y las economías de escala.

\subsection{Modelo de Factores Específicos}

El modelo de factores específicos aborda la \textbf{distribución de la renta} a corto plazo.

\subsubsection{Supuestos y distribución de la renta a corto plazo}
Este modelo asume una economía con tres factores: trabajo ($L$), capital ($K$) y tierra ($T$). El trabajo es un factor \textbf{móvil} que puede utilizarse en ambos sectores, mientras que el capital y la tierra son \textbf{factores específicos} ligados permanentemente a un solo sector (por ejemplo, manufacturas y alimentos).

Dado que el trabajo es móvil, este modelo es crucial para analizar por qué el comercio \textbf{genera ganadores y perdedores} a corto plazo, ya que los factores específicos pueden verse afectados de manera diferente por los cambios en los precios relativos causados por el comercio.

\subsection{Modelo de Proporciones Factoriales (Heckscher-Ohlin)}

El modelo de proporciones factoriales (H-O) se centra en las diferencias de \textbf{dotaciones de recursos} como causa fundamental del comercio.

\subsubsection{Ventaja comparativa por abundancia de factores}
Este modelo asume movilidad de todos los factores a \textbf{largo plazo} (trabajo y capital) entre sectores. La ventaja comparativa se ve afectada por la interacción entre la \textbf{abundancia relativa de factores} en una nación y la \textbf{intensidad relativa} con la que diferentes factores son utilizados en la producción de distintos bienes.

El \textbf{Teorema de Heckscher-Ohlin} predice que un país abundante en un factor exportará el bien cuya producción es intensiva en ese factor. Por ejemplo, un país abundante en capital exportará bienes intensivos en capital.

\subsection{Modelos basados en Economías de Escala y Rendimientos Crecientes}

Estos modelos explican el comercio que ocurre incluso entre naciones aparentemente similares, y donde la ventaja comparativa por diferencias no es la fuerza dominante.

\subsubsection{Economías de escala y comercio intraindustrial}
Las \textbf{economías de escala} (o rendimientos crecientes) hacen ventajoso que cada país se especialice en un rango limitado de bienes y servicios, permitiendo una producción más eficiente a mayor escala.

Cuando las economías de escala son \textbf{internas} (a nivel de empresa), conducen a estructuras de mercado de competencia imperfecta (monopolística u oligopolio). Esto, combinado con la diferenciación de productos, genera \textbf{comercio intraindustrial}: el intercambio mutuo de bienes similares (e.g., diferentes modelos de automóviles entre países similares). El comercio intraindustrial ofrece ganancias adicionales de bienestar a los consumidores a través de una \textbf{mayor variedad de productos} y \textbf{precios más bajos}.

\section{La movilidad internacional de los factores productivos}

Además del comercio de bienes y servicios, el movimiento de los factores productivos (trabajo y capital) a través de las fronteras nacionales es un aspecto clave de la economía internacional.

\subsection{Movilidad internacional del trabajo}

La \textbf{movilidad internacional del trabajo} (migración) amplía la población activa en el país receptor. Un aumento en la fuerza de trabajo, si los demás factores permanecen invariantes, puede esperarse que reduzca los salarios.

En el contexto de modelos como el de factores específicos, la movilidad del trabajo y la inmigración tienen efectos significativos en la distribución de la renta dentro del país (generando preocupaciones por la convergencia salarial o la desigualdad).

\subsection{La movilidad internacional del capital y la inversión}

El movimiento internacional de capital se manifiesta a través de los \textbf{préstamos y endeudamientos internacionales} y la \textbf{Inversión Extranjera Directa (IED)}.

\subsubsection{Comercio intertemporal}
Los préstamos y endeudamientos internacionales pueden entenderse como una forma de \textbf{comercio intertemporal}. Los países intercambian consumo presente por consumo futuro, donde el precio relativo de este intercambio es igual a $1$ más el tipo de interés real ($1+r$).

\subsubsection{La inversión extranjera y la empresa multinacional}
La IED se produce cuando una empresa de un país invierte en instalaciones productivas en otro. La IED puede ser clasificada como:

\begin{itemize}
    \item \textbf{IED Horizontal:} Inversión realizada para acceder a un mercado extranjero, evitando costes de comercio ($t$) mediante la construcción de filiales productivas en el extranjero (multinacionales).
    \item \textbf{IED Vertical:} Inversión motivada por las diferencias de costes de producción entre países, trasladando partes del proceso productivo (bienes intermedios) a países donde son más eficientes (siguiendo el principio de la ventaja comparativa).
\end{itemize}

La empresa multinacional (EMN) que realiza IED vertical aprovecha las diferencias de costes (abundancia de factores) para dividir su producción, concentrando las actividades intensivas en capital humano (I+D) en países donde este es abundante y las actividades intensivas en trabajo no cualificado (ensamblaje) en países donde este factor es relativamente abundante.

\section{Formulación Matemática de las Teorías del Comercio}
\addcontentsline{toc}{section}{Formulación Matemática de las Teorías del Comercio}

\subsection{Modelo Clásico de la Ventaja Comparativa (Modelo Ricardiano)}

El modelo ricardiano se basa en el trabajo como único factor de producción. La tecnología se define por los requerimientos unitarios de trabajo ($a_{Li}$), y la restricción principal es la disponibilidad total de trabajo ($L$).

\subsubsection{Frontera de Posibilidades de Producción (FPP)}
La FPP, que es una línea recta en el modelo de un factor, está definida por el límite de recursos (trabajo):
\begin{equation}
a_{LV} Q_V + a_{LQ} Q_Q \leq L
\label{eq:fpp_ricardiana}
\end{equation}
donde $Q_V$ y $Q_Q$ son las cantidades producidas de vino y queso, y $a_{LV}$ y $a_{LQ}$ son los requerimientos unitarios de trabajo para vino y queso, respectivamente.

\subsubsection{Regla de Especialización y Ventaja Comparativa}
La ventaja comparativa se determina por las diferencias en los costes de oportunidad relativos. En un mundo con muchos bienes ($N$), la ordenación de los bienes según la ratio de requerimientos de trabajo entre el país nacional y el extranjero ($a_{Li}/a^*_{Li}$) determina el patrón de comercio. Los bienes se ordenan tal que:
\begin{equation}
\frac{a_{L1}}{a^*_{L1}} < \frac{a_{L2}}{a^*_{L2}} < \frac{a_{L3}}{a^*_{L3}} < \ldots < \frac{a_{LN}}{a^*_{LN}}
\label{eq:ventaja_comparativa}
\end{equation}
La regla de asignación de la producción establece que el bien $i$ se producirá en el país nacional si el coste de producirlo es menor allí. Si $w$ es el salario nacional y $w^*$ el extranjero:
\begin{equation}
w \cdot a_{Li} < w^* \cdot a^*_{Li}
\label{eq:coste_comparativo}
\end{equation}

\subsection{Modelo de Factores Específicos (Corto Plazo)}

Este modelo introduce factores específicos (capital $K$, tierra $T$) y un factor móvil (trabajo $L$). La restricción del trabajo es:
\begin{equation}
L_M + L_A = L
\label{eq:restriccion_trabajo_fe}
\end{equation}
donde $L_M$ y $L_A$ son el trabajo empleado en manufacturas y alimentos, respectivamente.

\subsubsection{Determinación del Salario y Equilibrio}
En una economía competitiva, el trabajo se mueve hasta que el valor de su producto marginal se iguala en ambos sectores, determinando el salario ($w$):
\begin{equation}
PMgL_M \times P_M = PMgL_A \times P_A = w
\label{eq:equilibrio_salario_fe}
\end{equation}
donde $PMgL_i$ es el producto marginal del trabajo en el sector $i$, y $P_i$ es el precio del bien $i$.

\subsubsection{Pendiente de la FPP}
La pendiente de la FPP refleja el coste de oportunidad, y en el equilibrio de producción, debe ser tangente a la línea de precios relativos:
\begin{equation}
-\frac{PMgL_A}{PMgL_M} = - \frac{P_M}{P_A}
\label{eq:pendiente_fpp_fe}
\end{equation}

\subsection{Modelo de Proporciones Factoriales (Heckscher-Ohlin)}

Este modelo asume dos factores (Trabajo $L$ y Capital $K$) y dos bienes (Tela $T$ y Alimentos $A$), con movilidad de factores a largo plazo. Los requisitos de factores por unidad de producto ($a_{ij}$) ya no son fijos, sino que dependen de la ratio salario-rentas ($w/r$).

\subsubsection{Restricciones de Dotación de Factores}
Los recursos totales ($K$ y $L$) deben ser plenamente empleados:
\begin{align}
a_{KA} Q_A + a_{KT} Q_T &= K \quad (\text{Restricción de Capital}) \label{eq:restriccion_capital_ho} \\
a_{LA} Q_A + a_{LT} Q_T &= L \quad (\text{Restricción de Trabajo}) \label{eq:restriccion_trabajo_ho}
\end{align}
donde $a_{ij}$ es la cantidad del factor $i$ utilizada para producir una unidad del bien $j$.

\subsubsection{Relación Precio-Coste}
En competencia perfecta, el precio del bien debe ser igual a su coste de producción (coste unitario), que es la suma ponderada del coste de los factores ($r$ = renta del capital, $w$ = salario):
\begin{align}
P_A &= a_{KA} r + a_{LA} w \label{eq:precio_coste_A} \\
P_T &= a_{KT} r + a_{LT} w \label{eq:precio_coste_T}
\end{align}

\subsection{Modelos de Rendimientos Crecientes (Economías de Escala)}

El modelo de Competencia Monopolística utiliza una función de costes que genera economías de escala internas.

\subsubsection{Función de Costes y Coste Medio}
Si $F$ es el coste fijo y $c$ es el coste marginal constante, el Coste Total ($C$) de la empresa para una producción $Q$ es:
\begin{equation}
C = F + c \times Q
\label{eq:coste_total_monopolio}
\end{equation}
El Coste Medio ($CM$) es, por lo tanto, decreciente, reflejando economías de escala:
\begin{equation}
CM = \frac{C}{Q} = \frac{F}{Q} + c
\label{eq:coste_medio_monopolio}
\end{equation}

\subsubsection{Ingreso Marginal y Demanda}
Una empresa se enfrenta a una curva de demanda donde el ingreso marginal ($IMg$) es inferior al precio ($P$):
\begin{equation}
IMg = P - \frac{Q}{B}
\label{eq:ingreso_marginal}
\end{equation}
donde $B$ es la pendiente de la curva de demanda.

\subsection{Movilidad Intertemporal de Factores (Comercio de Consumo Presente por Futuro)}

El precio relativo del consumo futuro en términos de consumo presente se determina por el tipo de interés real ($r$):
\begin{equation}
\text{Precio relativo del consumo futuro} = \frac{1}{1 + r}
\label{eq:precio_intertemporal}
\end{equation}
El consumo futuro devuelto es $(1+r)$ veces la cantidad prestada en el presente.
