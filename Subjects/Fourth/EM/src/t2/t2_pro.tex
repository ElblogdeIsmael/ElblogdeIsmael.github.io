\chapter{El Comercio Internacional y el Movimiento de los Factores Productivos}

La notación de los títulos corresponden con la enumeración del índice de la asignatura que entrega el profesorado.

\section{Introducción}
La disciplina de la Economía Internacional, que se remonta a los primeros modelos económicos desarrollados antes de \textit{La riqueza de las naciones} de Adam Smith, aborda cuestiones fundamentales sobre las interacciones económicas entre naciones soberanas. El estudio del comercio mundial comienza con una visión general de quién comercia con quién y qué se intercambia.

\subsection{Visión General del Comercio Mundial: El Caso del Mapa del Comercio}
\textit{Caso 2.1. Map of World Trade} (Feenstra, R. C y Taylor, A. M. 2021. International Trade (5th edition). Page 7-10) introduce la observación de los patrones comerciales reales, que son esencialmente un reflejo de las fuerzas que determinan la actividad económica global.

\subsubsection{El Modelo de Gravedad y la Influencia del Tamaño Económico}
La relación entre el valor del comercio entre dos países cualesquiera y su tamaño económico se describe mediante el \textbf{modelo de gravedad}.

\begin{definicion}
El modelo de gravedad relaciona el comercio entre dos países cualesquiera con el \textbf{tamaño de sus economías}.
\end{definicion}

Este modelo predice que el volumen de comercio entre dos naciones aumenta cuanto \textbf{mayor es el tamaño de cualquiera de las economías implicadas}. Las economías de gran tamaño tienden a gastar mucho en importaciones porque tienen altos ingresos y, simultáneamente, atraen grandes proporciones del gasto de otros países dado que producen una \textbf{amplia gama de productos}.

Una formulación general para este modelo sugiere que el volumen de comercio ($T_{ij}$) entre el país $i$ y el país $j$ es proporcional al producto de sus Productos Interiores Brutos ($Y_i$ y $Y_j$), e inversamente proporcional a la distancia ($D_{ij}$) entre ellos.

\begin{ejemplo}
El modelo de gravedad postula que un país con un PIB grande, como Estados Unidos (EE. UU.), tendrá una fuerte dependencia de sus principales socios comerciales. El comercio total entre dos economías cualesquiera aumenta si el PIB de las naciones es mayor.
\end{ejemplo}

\subsubsection{Los Obstáculos Físicos e Institucionales al Comercio}
A pesar de la prominencia del tamaño económico, existen factores que limitan de manera significativa los flujos comerciales internacionales, y que son analizados por el modelo de gravedad.

\begin{enumerate}[label=(\roman*)]
    \item \textbf{Distancia}: La distancia física entre países es un \textbf{obstáculo significativo al comercio}, incluso en la economía mundial moderna. La distancia representa costes de transporte y, por extensión, costes de comunicación y contacto.
    \item \textbf{Fronteras y Barreras al Comercio}: La existencia de \textbf{barreras comerciales} y fronteras nacionales reduce el volumen de intercambio.
    \item \textbf{Acuerdos Comerciales Regionales}: Los acuerdos como el Tratado de Libre Comercio de América del Norte (TLCAN/NAFTA) garantizan que la mayoría de los bienes intercambiados entre sus socios no estén sujetos a aranceles o barreras, intensificando el comercio. El modelo de gravedad es una herramienta utilizada para \textbf{valorar el efecto} de estos acuerdos, midiendo si generan un comercio significativamente mayor de lo que se esperaría por el tamaño económico y la distancia.
    \item \textbf{El Efecto Frontera}: La investigación empírica ha demostrado que las \textbf{fronteras nacionales} desincentivan el comercio de forma sustancial. Por ejemplo, la frontera entre EE. UU. y Canadá, una de las más abiertas del mundo, reduce el comercio de forma equivalente a la que existiría entre países separados por una \textbf{distancia de 2,200 a 3,700 kilómetros}. La persistencia de este efecto frontera se debe, en parte, a la existencia de distintas divisas y regulaciones, incluso cuando las barreras formales se eliminan.
\end{enumerate}

\subsubsection{Composición Actual y Futura del Comercio Global}
La composición de los bienes y servicios intercambiados a nivel mundial ha evolucionado, aunque históricamente ha estado dominada por ciertas categorías.

\begin{definicion}
La composición del comercio mundial, para el mundo en su conjunto, se centra principalmente en el intercambio de \textbf{bienes manufacturados}, como automóviles, ordenadores y ropa.
\end{definicion}

Otras categorías importantes son los \textbf{productos minerales} (como el petróleo, que es el principal elemento en la categoría moderna de minerales) y los \textbf{productos agrícolas} (como el trigo o el algodón).

Un cambio significativo en los patrones comerciales implica el aumento del \textbf{comercio de servicios}, cuya importancia probablemente seguirá creciendo.

\begin{proposicion}
La distinción clave en el futuro del comercio internacional \textbf{no será entre cosas que se pueden meter en una caja y cosas que no}. Más bien, será entre servicios que se pueden proveer electrónicamente a grandes distancias con poca o ninguna pérdida de calidad, y aquellos que no.
\end{proposicion}

Esto sugiere que el \textbf{dominio actual de las manufacturas en el comercio mundial podría ser temporal}, dando paso a los servicios suministrados de forma electrónica como el elemento más importante del comercio global a largo plazo. El 40\% del empleo en actividades comerciables en EE. UU. ya incluye más servicios que manufacturas.

\begin{ejemplo}
Mientras que un trabajador que repone estantes en un supermercado debe estar en el local (servicio no comerciable), un contable o un radiólogo que interpreta una radiografía pueden estar ubicados en otro país y prestar sus servicios a través de Internet (servicios comerciables a distancia).
\end{ejemplo}


\section{La Teoría Clásica}
La teoría clásica del comercio internacional se fundamenta en los trabajos de los economistas Adam Smith y David Ricardo, quienes postularon los principios de la ventaja absoluta y la ventaja comparativa, respectivamente. Estos modelos buscan responder preguntas fundamentales sobre la base y la dirección del comercio, los términos de intercambio y las ganancias de producción y consumo que resultan de la interacción económica entre naciones.

\subsection{Supuestos (2.2.A.)}
El desarrollo de la teoría moderna del comercio, que incluye el modelo ricardiano, se construye sobre ciertos supuestos fundamentales, especialmente la Teoría del Valor-Trabajo.

\begin{enumerate}
    \item \textbf{Factor Productivo Único}: Se asume que el \textbf{trabajo} es el único factor de producción relevante.
    \item \textbf{Movilidad del Factor Trabajo}: El trabajo es perfectamente móvil entre distintas industrias dentro de una nación.
    \item \textbf{Inmovilidad Internacional}: El trabajo es, sin embargo, totalmente inmóvil a nivel internacional, es decir, no existe migración entre países.
    \item \textbf{Costos Constantes}: El modelo ricardiano inicial asume que la producción se lleva a cabo bajo condiciones de \textbf{costos de oportunidad constantes}.
    \item \textbf{Tecnología y Productividad}: Las diferencias en la productividad del trabajo entre naciones constituyen la única fuente de comercio. La tecnología de la economía se resume en los requerimientos unitarios de trabajo (el número de horas de trabajo necesarias para producir una unidad de un bien).
\end{enumerate}

\subsection{El Modelo de la Ventaja Absoluta (2.2.B.)}
El principio de la \textbf{ventaja absoluta}, propuesto por Adam Smith, fue la primera explicación de por qué comercian las naciones.

\begin{definicion}
En un mundo con dos naciones y dos productos, la especialización internacional y el comercio serán mutuamente benéficos cuando una nación tenga una \textbf{ventaja de costos absoluta} en un producto y la otra nación tenga una ventaja de costos absoluta en el otro producto.
\end{definicion}

La perspectiva de Smith era \textbf{dinámica}, contrastando con el punto de vista estático mercantilista. Él aseveraba que la riqueza mundial no es fija; el comercio fomenta la especialización y la división del trabajo, lo que aumenta la productividad y, por ende, la producción mundial y la riqueza, beneficiando a ambos socios comerciales. La nación debe especializarse y exportar aquel producto que puede producir de manera más eficiente (con menos costo/trabajo).

\subsubsection{Caso 2.2. Protestas de los agricultores franceses ante el ingreso de España en la Comunidad Económica Europea (CEE)}
El ingreso de España en la CEE estuvo marcado por la \textbf{negativa del gobierno francés} a que los productos hortofrutícolas españoles se vendieran libres de aranceles en el resto de la CEE.

La oposición se debía a que los productos españoles presentaban ventajas: se producían con un costo menor, ya que las condiciones climáticas (temperatura) en el sur y sureste de España permitían obtener una \textbf{mayor producción por hectárea} y a un costo de producción muy inferior al de Francia. Esta situación reflejaba una ventaja potencial o real para España.

La reacción violenta de los agricultores del sur de Francia, que volcaban o incendiaban camiones españoles cargados con estos productos, buscaba oponerse al ingreso de España de pleno derecho en la CEE, con la impasividad de las fuerzas del orden público. A pesar de esto, el gobierno de España argumentaba que Francia y otros países se beneficiarían del intercambio.

\subsubsection{Titular 2.1. Jornada de Protesta de los Agricultores del Mercado Común}
Las protestas de los agricultores se extendieron por Europa, reflejando la tensión generada por la competencia internacional y los bajos precios.

\begin{ejemplo}
En una jornada de protesta en 1974, millares de campesinos alemanes, belgas y franceses invadieron carreteras y ciudades con sus tractores en señal de oposición a los bajos precios de sus productos. Los manifestantes exigían "justicia para la agricultura" y "precios justos". Los agricultores franceses, en particular, solicitaban una subida del 8\% de los precios agrícolas. Estas acciones ilustran las dificultades y la oposición política que surgen en sectores específicos (como el agropecuario) ante la competencia y la globalización.
\end{ejemplo}

\subsubsection{Caso 2.4. Rosas por San Valentín}
Este caso ilustra el concepto de ventaja comparativa aplicado al comercio entre países ricos y pobres. La creciente proporción de rosas de invierno en Estados Unidos importadas de América del Sur, especialmente Colombia, generó críticas proteccionistas.

Sin embargo, desde la perspectiva de la ventaja comparativa, el comercio beneficia a ambos países si Colombia produce rosas y Estados Unidos fabrica ordenadores para el mercado colombiano. Negar a los trabajadores centroamericanos la oportunidad de exportar y comerciar equivaldría a condenarlos a una pobreza aún mayor.

\subsection{El Modelo de la Ventaja Comparativa (2.2.C.)}
El principio de la \textbf{ventaja comparativa}, desarrollado por David Ricardo, supera la limitación de la ventaja absoluta al explicar por qué el comercio es mutuamente beneficioso incluso si una nación tiene una desventaja absoluta en la producción de todos los bienes.

\begin{definicion}
El \textbf{principio de la ventaja comparativa} es la capacidad de una nación para fabricar un producto o servicio a un \textbf{costo de oportunidad menor} de lo que otros pueden producirlo.
\end{definicion}

El comercio se basa en las diferencias en los \textbf{costos relativos} de los países. Cada nación se especializa y exporta el producto en el que posee una ventaja comparativa (donde su desventaja absoluta es menor, o su ventaja absoluta es mayor).

\begin{ejemplo}
Considere un escenario donde Estados Unidos tiene una mayor productividad que el Reino Unido en la producción de vino y tela (ventaja absoluta en ambos). Si el Reino Unido tiene una desventaja absoluta menor en el vino que en la tela, y Estados Unidos una mayor ventaja en la tela, entonces el Reino Unido se especializará en vino (ventaja comparativa en vino) y Estados Unidos en tela (ventaja comparativa en tela). Ambos ganan del comercio.
\end{ejemplo}

\subsubsection{Caso 2.5. Comparative Advantage in Apparel, Textiles, and Wheat: United States and China}
La ventaja comparativa puede aplicarse a un mundo con más de dos productos. En el modelo ricardiano de muchos bienes, el patrón de especialización y, consecuentemente, el patrón de comercio, están determinados por el \textbf{salario relativo} de cada país.

El modelo de dotación de factores (Heckscher-Ohlin), que amplía la teoría ricardiana, es consistente con la idea de que la ventaja comparativa se basa en la dotación de recursos. Estados Unidos, que es abundante en capital humano (fuerza de trabajo bien educada y hábil) y capital físico, tiende a exportar productos intensivos en tecnología (como aviones y software). China, en contraste, se especializa en productos intensivos en mano de obra (como textiles).

La intersección entre la curva de demanda relativa de trabajo ($DR$) y la curva de oferta relativa ($OR$) determina el salario relativo de equilibrio, el cual, a su vez, define el patrón de especialización. Los cambios en el patrón de especialización se observan en las "zonas llanas" de la curva de oferta relativa.

\subsection{La Ventaja Comparativa y el Coste de Oportunidad (2.2.D.)}
El concepto clave que subyace a la ventaja comparativa es el costo de oportunidad.

\subsubsection{La Frontera de Posibilidades de Producción y el Costo de Oportunidad}
En el caso de \textbf{costos de oportunidad constantes} (como se asume en el modelo ricardiano), la Frontera de Posibilidades de Producción (FPP) se representa como una línea recta.

\begin{definicion}
La \textbf{Tasa Marginal de Transformación (TMT)} es la pendiente de la FPP y mide la cantidad de un producto que una nación debe sacrificar para obtener una unidad adicional del otro producto. Esto ilustra el concepto de costo comparativo.
\end{definicion}

La diferencia en las pendientes de las FPP de dos países indica la diferencia en sus costos relativos, proporcionando la base para una especialización y comercio mutuamente favorables.

\subsubsection{Caso 2.6. Ilustración con el Ejemplo de los Epígrafes 2.2.B y 2.2.C la Ley de los Costes Comparativos}
Aplicando la ley de los costos comparativos, se determina el patrón de comercio:
\begin{enumerate}
    \item Si, por ejemplo, el costo relativo de producir un automóvil adicional es de 0.5 fanegas de trigo en Estados Unidos, pero 2 fanegas de trigo en Canadá (TMT de EE. UU. = 0.5, TMT de Canadá = 2.0), Estados Unidos tiene la ventaja comparativa en la producción de automóviles.
    \item En consecuencia, Estados Unidos se especializa en automóviles y los exporta, mientras que Canadá se especializa en trigo y lo exporta, ya que tiene un costo relativo menor en trigo (1/2 = 0.5 unidades de automóvil por fanega de trigo en Canadá, frente a 1/0.5 = 2 unidades de automóvil por fanega de trigo en EE. UU.).
\end{enumerate}

\subsection{Especialización y Precio de Equilibrio (2.2.E.)}
El comercio internacional lleva a la especialización de las naciones en los bienes donde poseen una ventaja comparativa.

\subsubsection{Especialización y Ganancias del Comercio}
Bajo costos de oportunidad constantes, el principio de la ventaja comparativa implica que la nación alcanzará una \textbf{especialización completa} en el producto de su ventaja comparativa.

El resultado del comercio internacional es un aumento en la producción mundial total y la posibilidad de que las naciones disfruten de \textbf{ganancias de producción} (incremento de la producción debido a la especialización) y \textbf{ganancias de consumo} (posibilidad de consumir fuera de su FPP).

\subsubsection{Términos de Intercambio de Equilibrio}
El \textbf{término de intercambio} es la razón a la que se intercambian los productos en el mercado mundial. El comercio permite a las naciones consumir a lo largo de una \textbf{línea de posibilidades de comercio} que se encuentra fuera de su FPP.

El \textbf{triángulo del comercio} de una nación denota sus exportaciones, importaciones y los términos de intercambio (la pendiente de la línea de posibilidades de comercio).

La \textbf{distribución de las ganancias del comercio} entre las naciones depende de los términos de intercambio de equilibrio. Estos términos de intercambio se ubicarán dentro de los límites definidos por los costos relativos (TMTs) de autarquía de los países.

\begin{propuesta}
La \textbf{Teoría de la Demanda Recíproca} de John Stuart Mill aborda la limitación del modelo ricardiano, que solo dependía de la oferta y no podía determinar con precisión los términos reales de intercambio. Mill aseveró que, dentro de los límites externos determinados por los costos de producción, los términos de intercambio reales se determinan por la \textbf{fuerza relativa de la demanda} de cada país por el producto del otro país.
\end{propuesta}


\section{La Teoría Estándar del Comercio Internacional}
El \textbf{Modelo Estándar del Comercio} constituye un marco analítico general que unifica los modelos comerciales previos, como el Ricardiano (Capítulo 3) y el de Factores Específicos (Capítulo 4), considerándolos casos especiales. Este modelo se basa en cuatro relaciones clave: la Frontera de Posibilidades de Producción (FPP) y la oferta relativa; la relación entre precios relativos y demanda relativa; la determinación del equilibrio mundial; y el efecto de la relación de intercambio sobre el bienestar nacional.

\subsection{Frontera de Posibilidades de Producción y Oferta con Costes de Oportunidad Crecientes (2.3.A.)}

A diferencia del modelo ricardiano que asume rendimientos constantes a escala y una FPP lineal (dado un único factor de producción), la Teoría Estándar del Comercio incorpora condiciones de \textbf{costos de oportunidad crecientes}.

\subsubsection{La Frontera Cóncava y la Tasa Marginal de Transformación (TMT)}
Los costos de oportunidad crecientes dan lugar a una Frontera de Posibilidades de Producción (FPP) que es \textbf{cóncava} (o inclinada hacia afuera desde el origen del diagrama), lo cual es característico de las condiciones del mundo real para la mayoría de los productos. La concavidad refleja los rendimientos decrecientes del factor trabajo en cada sector, una diferencia fundamental con el modelo ricardiano.

\begin{definicion}
La \textbf{Tasa Marginal de Transformación (TMT)} es la pendiente absoluta de la FPP en un punto determinado y mide la cantidad de un producto que una nación debe sacrificar para obtener una unidad adicional del otro producto. Bajo costos crecientes, la TMT aumenta conforme se produce más de ese bien, es decir, la pendiente se vuelve más inclinada en valor absoluto.
\end{definicion}

\subsubsection{Precios Relativos y Oferta}
El punto en la FPP donde la economía elige producir está determinado por los precios relativos del mercado.

\begin{enumerate}
    \item \textbf{Rectas de Isovalor}: El valor total de la producción es constante a lo largo de estas rectas, cuya pendiente es igual al negativo de la razón del precio relativo de los bienes (por ejemplo, $-\text{PT}/\text{PA}$).
    \item \textbf{Decisión de Producción}: La economía se sitúa en el punto de la FPP que maximiza el valor de la producción, donde la FPP es tangente a la recta de isovalor más alta posible. En este punto, el costo de oportunidad (la pendiente de la FPP) de producir una unidad adicional de un bien es igual a su precio relativo.
    \item \textbf{Curva de Oferta Relativa (OR)}: Un aumento en el precio relativo de un bien (por ejemplo, Tela, $P_T/P_A$) hace que las rectas de isovalor se vuelvan más inclinadas. Esto incentiva a la economía a producir más de ese bien y menos del otro. Por lo tanto, la curva de oferta relativa (producción de Tela/Alimentos) tiene una \textbf{pendiente positiva}.
\end{enumerate}

\subsection{La Demanda (2.3.B.)}
Para determinar la combinación de bienes que una economía consume (demanda), es necesario considerar los gustos y preferencias de sus residentes, lo cual se representa mediante las curvas de indiferencia.

\begin{definicion}
Las \textbf{Curvas de Indiferencia} representan los gustos y preferencias nacionales, mostrando combinaciones de bienes que proporcionan un nivel constante de satisfacción o utilidad a la nación. Una curva de indiferencia más alta representa un mayor nivel de bienestar.
\end{definicion}

\subsubsection{Consumo y Efectos sobre la Demanda}
La economía consume en el punto donde la recta de isovalor más alta es tangente a la curva de indiferencia más alta posible, maximizando la utilidad de la nación sujeta a la restricción presupuestaria (dada por la producción y la relación de intercambio).

Un cambio en los precios relativos del comercio tiene dos efectos sobre la demanda:
\begin{enumerate}
    \item \textbf{Efecto Renta (Bienestar)}: Un incremento en la relación de intercambio (un precio relativo favorable) aumenta el bienestar general (renta), tendiendo a elevar el consumo de ambos bienes.
    \item \textbf{Efecto Sustitución}: El cambio en el consumo para un nivel de bienestar dado provoca que la economía consuma menos del bien que se ha vuelto relativamente más caro.
\end{enumerate}
La curva de Demanda Relativa (DR) tiene una \textbf{pendiente negativa}.

\subsection{Equilibrio Parcial en un País (2.3.C.)}
El análisis de la ventaja comparativa y los patrones de comercio requiere fundamentalmente un \textbf{análisis de equilibrio general} que tenga en cuenta las interrelaciones entre los mercados (por ejemplo, queso y vino).

El equilibrio parcial, que se centra en un solo mercado, es una herramienta analítica adecuada para estudiar los efectos de ciertas políticas comerciales (como aranceles o cuotas de importación) en un sector específico, pero no es suficiente para valorar los efectos de la ventaja comparativa en el conjunto de la economía.

En el contexto de la teoría estándar, el equilibrio interno (autarquía) de un país se determina por la intersección de su curva de Oferta Relativa (OR) y su curva de Demanda Relativa (DR). Este punto establece el precio relativo (de autarquía) que prevalecería en ausencia de comercio.

\subsection{Base del Comercio Internacional y Ganancias (2.3.D.)}

La base para el comercio internacional en el Modelo Estándar radica en la diferencia de los \textbf{precios relativos} de autarquía entre los países. Estas diferencias pueden surgir debido a distintas tecnologías (como en el modelo ricardiano) o a diferencias en la dotación de recursos (como en el modelo de Heckscher-Ohlin).

\subsubsection{Ganancias del Comercio y Bienestar}
El comercio internacional es una fuente de ganancias potenciales para todos los países. El comercio amplía el rango de elección y, por lo tanto, \textbf{mejora el bienestar} de los residentes de cada país, permitiéndoles consumir en un punto fuera de su FPP.

\begin{proposicion}
Para que el comercio sea una fuente de ganancia potencial para toda la economía, los que ganan con el comercio deben ser capaces de \textbf{compensar a aquellos que pierden} y aun así permanecer en una mejor situación que antes del comercio.
\end{proposicion}

El bienestar de un país está directamente influenciado por su \textbf{Relación de Intercambio} (o Términos de Intercambio).

\begin{definicion}
La \textbf{Relación de Intercambio} de un país es el precio de sus exportaciones dividido por el precio de sus importaciones. Si, manteniendo todo lo demás constante, la relación de intercambio de un país \textbf{aumenta (mejora)}, su bienestar se incrementa. Inversamente, si la relación de intercambio disminuye (empeora), su situación empeora.
\end{definicion}

\subsubsection{Caso 2.7. How Large Are the Gains from Trade?}
Los análisis sobre las ganancias del comercio internacional, como el referido en el Caso 2.7 (Feenstra, R. C y Taylor, A. M. 2021. International Trade 5th. Page 67-69), confirman que estas ganancias son fundamentales en el marco del comercio internacional.

\begin{nota}
El concepto de ganancias del comercio implica que el bienestar agregado de un país no puede ser nunca reducido por debajo del nivel que tendría sin comercio (autarquía). Sin embargo, el comercio tiene importantes efectos sobre la \textbf{distribución de la renta} dentro de la nación, lo que genera perdedores y ganadores, al menos a corto plazo.
\end{nota}

\subsection{La Determinación del Precio Relativo a Nivel Mundial (Equilibrio General) (2.3.E.)}

El precio de equilibrio relativo mundial (o la relación de intercambio) se determina por la interacción de la \textbf{oferta y la demanda relativas mundiales}.

\begin{enumerate}
    \item \textbf{Oferta Relativa Mundial}: Se obtiene al agregar la producción de ambos bienes (por ejemplo, Tela y Alimentos) de todos los países participantes en el comercio. Esta curva debe situarse entre las curvas de oferta relativa de cada país.
    \item \textbf{Demanda Relativa Mundial}: Es la agregación de las demandas de los bienes de todos los países.
    \item \textbf{Equilibrio}: El precio relativo de equilibrio mundial (el punto 1 en el gráfico de oferta y demanda relativas) se da en la intersección de la curva de oferta relativa mundial y la curva de demanda relativa mundial.
\end{enumerate}

Este precio relativo de equilibrio es el que define el \textbf{número de unidades} del bien de exportación que un país debe ceder para importar una cantidad determinada del otro bien, asegurando que las exportaciones deseadas de un país igualan las importaciones deseadas del otro.


\section{La Dotación de Factores y la Teoría de Heckscher-Ohlin (H-O)}
La Teoría de la Dotación de Factores, también conocida como \textbf{Teoría Heckscher-Ohlin} (H-O), aborda dos preguntas fundamentales que el modelo ricardiano dejó sin explicar: 1) ¿qué determina la ventaja comparativa? y 2) ¿qué efecto tiene el comercio internacional en las ganancias de los diversos factores de producción en las naciones? Esta perspectiva, desarrollada por los economistas suecos Eli Heckscher y Bertil Ohlin (ganador del Premio Nobel de Economía en 1977), sostiene que la ventaja comparativa de una nación está determinada por su dotación relativa de recursos. El modelo se centra en los resultados a \textbf{largo plazo}, asumiendo que todos los factores de producción son móviles entre los distintos sectores.

\subsection{Intensidad y Abundancia de Factores y la Frontera de Producción (2.4.A.)}

El modelo H-O se basa en la interacción de la \textbf{dotación de recursos} de las naciones (abundancia relativa de factores) y la \textbf{tecnología de producción} (intensidad relativa con que los factores son utilizados).

\subsubsection{Definiciones de Intensidad y Abundancia}
\begin{enumerate}
    \item \textbf{Intensidad de Uso de un Factor}: Los bienes difieren en su \textbf{intensidad de factores}. Para cualquier ratio salario-renta dado ($w/r$), la producción de un bien se considera \textbf{intensiva en un factor} (por ejemplo, trabajo) si utiliza una ratio mayor de ese factor en relación con el otro factor (capital) que la producción del otro bien. Un bien no puede ser intensivo en trabajo y capital simultáneamente.
    \item \textbf{Abundancia de Factores}: La abundancia se define en \textbf{términos de ratios} y no de cantidades absolutas. Una nación es \textbf{relativamente abundante} en un factor si la proporción de ese factor respecto a otros factores (por ejemplo, la razón capital/trabajo) es mayor que la proporción de factores en el otro país.
    \begin{ejemplo}
    Si la razón capital/trabajo de Estados Unidos es 0.5 y la de China es 0.02, Estados Unidos es el país de abundancia relativa de capital, y China el país de abundancia relativa de trabajo.
    \end{ejemplo}
\end{enumerate}

\subsubsection{La Frontera de Producción y el Efecto Rybczynski}
A diferencia del modelo ricardiano, la FPP en el modelo H-O es cóncava debido a la posibilidad de \textbf{sustituir capital por trabajo} (y viceversa).

El \textbf{efecto sesgado del incremento de los recursos} sobre las posibilidades de producción constituye una clave central. Si los precios relativos de los bienes permanecen constantes, un aumento en la oferta de un factor productivo produce una expansión sesgada de las posibilidades de producción: la producción del bien intensivo en ese factor aumenta, mientras que la producción del otro bien disminuye. Este fenómeno es conocido como el \textbf{efecto Rybczynski}.

\begin{propuesta}
Una economía tenderá a ser relativamente eficaz en la producción de bienes que son intensivos en los factores en los que el país está \textbf{relativamente mejor dotado}.
\end{propuesta}

\subsubsection{Caso 2.8. Are Factor Intensities the Same Across Countries}
La definición de intensidad de factores, esencial para el modelo H-O, debe ser consistente. Si la ganadería es intensiva en tierra comparada con el cultivo de trigo en Estados Unidos (donde la tierra es barata), la pregunta de si podemos seguir afirmando que la ganadería es intensiva en tierra en un país más densamente poblado depende de si la ratio tierra/trabajo utilizada en la ganadería sigue siendo mayor que la utilizada para el trigo, independientemente de los precios absolutos de los factores en ese país.

\subsection{Supuestos (2.4.B.)}
La Teoría de la Dotación de Factores se sustenta en una serie de supuestos clave que determinan sus resultados, especialmente la predicción de la igualación del precio de los factores:

\begin{enumerate}
    \item \textbf{Factores de Producción}: Se asumen dos factores de producción móviles (trabajo, $L$, y capital, $K$) que se emplean totalmente entre los dos sectores.
    \item \textbf{Homogeneidad de Tecnología}: Se asume que la tecnología de producción es \textbf{idéntica} entre los países, lo que implica que las funciones de producción son las mismas.
    \item \textbf{Intensidad de Factores Consistente}: La producción de productos se realiza bajo condiciones idénticas, y la intensidad de factores no se revierte.
    \item \textbf{Gustos y Preferencias Similares}: Se asume que los gustos y preferencias son aproximadamente \textbf{similares} entre los países, de modo que las curvas de indiferencia de la comunidad tienen casi la misma forma y posición.
    \item \textbf{Competencia Perfecta}: La fabricación de productos se lleva a cabo en el contexto de \textbf{competencia perfecta}.
    \item \textbf{Movilidad de Factores}: Los factores son perfectamente móviles entre industrias a largo plazo.
    \item \textbf{Igualación de Precios de Bienes}: La ausencia de costos de transporte y barreras arancelarias lleva a la \textbf{convergencia completa} de los precios de los bienes en ambos países.
    \item \textbf{Producción de Ambos Bienes}: Se asume que los dos países producen \textbf{ambos bienes}.
\end{enumerate}

\subsection{La Teoría de H-O (2.4.C.)}
La base inmediata para el comercio, según H-O, reside en la diferencia entre los precios relativos de los productos previos al comercio (precios de autarquía), determinados por las fronteras de posibilidades de producción (dadas por tecnología y dotación de recursos) y las condiciones de demanda (gustos y preferencias).

\begin{teorema}
\textbf{Teorema de Heckscher-Ohlin}: El país que es \textbf{abundante en un factor} exporta el bien cuya producción es \textbf{intensiva en ese factor}. Por el contrario, importará el producto en cuya producción utilice el factor relativamente escaso y costoso.
\end{teorema}

El comercio conduce a la \textbf{convergencia de los precios relativos} de los bienes entre países. El país abundante en trabajo (Nuestro País) tiene un precio relativo de la tela (intensiva en trabajo) menor que el país abundante en capital (Extranjero), y al comerciar, el precio relativo mundial se situará entre los precios de autarquía de ambos países.

\subsubsection{Aplicación al Comercio entre Países Desarrollados y en Desarrollo}
La esencia de la teoría de la dotación de factores se observa claramente en el comercio entre Estados Unidos y China.
\begin{itemize}
    \item \textbf{Estados Unidos}: Abundante en \textbf{capital humano} (habilidades), talento científico y capital físico. H-O predice que exportará productos intensivos en trabajo especializado y tecnología sofisticada, como aviones, software, farmacéuticos y equipo de alta tecnología.
    \item \textbf{China}: Abundante en \textbf{trabajo no especializado} (mano de obra intensiva) y escaso en talento científico. H-O predice que exportará productos que requieren una gran cantidad de trabajo no especializado, como ropa, zapatos, juguetes y ensamble final. El patrón de las exportaciones chinas se ha desplazado drásticamente a lo largo del tiempo, con un aumento en la concentración de exportaciones en sectores de alta cualificación, lo que es coherente con el cambio en sus proporciones factoriales.
\end{itemize}

\subsubsection{Caso 2.9. Actual and Effective Factor Endowments, China, 2000-2017}
Este caso aborda la evolución de la dotación de factores en China, un país que se ha convertido en una economía de alto desempeño. Si bien China es fuertemente dotada de mano de obra y se especializa en muchos productos intensivos en este factor, la investigación reciente se ha centrado en el capital humano y el talento científico. La evolución del patrón de exportaciones chinas hacia sectores de mayor intensidad de cualificaciones sugiere un incremento sustancial en la abundancia de \textbf{trabajo cualificado} a lo largo del tiempo, tal como predice el modelo H-O.

\subsection{La Igualación del Precio de los Factores y la Remuneración de los Factores (Distribución de la Renta) (2.4.D.)}

\subsubsection{Teorema de la Igualación del Precio de los Factores}
El comercio, al llevar a la convergencia de los precios relativos de los bienes, provoca una \textbf{nivelación de los precios de los factores} entre las naciones. Esto ocurre porque el comercio de bienes intensivos en un factor (por ejemplo, tela intensiva en trabajo) es un \textbf{sustituto del comercio de factores productivos} (movimiento indirecto de trabajo).

\begin{enumerate}
    \item El país abundante en trabajo (Nuestro País) exporta trabajo incorporado en sus exportaciones intensivas en trabajo, lo que hace subir el precio del trabajo ($w$) y caer el precio del capital ($r$) internamente.
    \item El país abundante en capital (Extranjero) exporta capital incorporado en sus exportaciones intensivas en capital. Esto hace subir el precio del capital y caer el precio del trabajo en ese país.
\end{enumerate}
Este proceso de redireccionamiento de la demanda lleva a que el recurso más económico de cada nación se vuelva relativamente más caro, y el caro se vuelva más barato, hasta que ocurre una nivelación.

\subsubsection{Limitaciones de la Igualación del Precio de los Factores}
En el mundo real, la \textbf{nivelación completa} de los precios de los factores no se observa. Las diferencias salariales entre países son demasiado pronunciadas para ser explicadas únicamente por diferencias en la cualificación. Las razones principales por las que la predicción no se cumple con exactitud son:
\begin{enumerate}
    \item Las \textbf{tecnologías de producción} no son idénticas en todos los países.
    \item La \textbf{ausencia de costos de comercio} no se cumple (costos de transporte, aranceles, y cuotas impiden la convergencia completa de los precios de los bienes).
    \item Los países no siempre producen el \textbf{mismo conjunto de bienes} (especialización completa), especialmente cuando las dotaciones factoriales relativas son radicalmente diferentes.
\end{enumerate}

\subsubsection{El Teorema Stolper-Samuelson y la Distribución de la Renta}
El comercio internacional ejerce un \textbf{poderoso efecto en la distribución de la renta}, incluso a largo plazo. El \textbf{Teorema Stolper-Samuelson} (SS) establece los efectos del comercio en el ingreso de los propietarios de los factores.

\begin{teorema}
De acuerdo con el SS, el factor que es \textbf{abundante y relativamente barato} en un país y que se utiliza intensivamente en las exportaciones \textbf{aumenta su precio y su ingreso real}. En contraste, el factor relativamente \textbf{escaso} utilizado intensivamente en la producción que compite con las importaciones sufre una \textbf{disminución en su ingreso real}.
\end{teorema}
Este efecto es \textbf{permanente} a largo plazo, a diferencia del modelo de factores específicos, donde los efectos sobre la distribución de la renta se consideran temporales y transitorios. El efecto de magnificación (una extensión del SS) sugiere que el cambio en el precio de un factor es mayor que el cambio en el precio del producto que utiliza ese factor de forma intensiva y relativa en su proceso de producción.

\subsubsection{Comercio y Desigualdad Salarial}
El comercio internacional agrava la \textbf{inequidad de ingresos} en países como Estados Unidos, donde el trabajo especializado es relativamente abundante, reduciendo el salario de los trabajadores no especializados en relación con los especializados.

\begin{itemize}
    \item Los propietarios de recursos relativamente \textbf{abundantes} tienden a \textbf{favorecer el libre comercio} (por ejemplo, los propietarios de capital en EE. UU. o los trabajadores no especializados en China).
    \item Los propietarios de factores relativamente \textbf{escasos} tienden a favorecer las \textbf{restricciones comerciales} (por ejemplo, los trabajadores no especializados en EE. UU.).
\end{itemize}

\subsubsection{Caso 2.10. Opinions Toward Free Trade}
Este caso se relaciona con las implicaciones políticas del Teorema Stolper-Samuelson. El modelo H-O sugiere que, aunque el libre comercio puede proporcionar \textbf{ganancias generales} para un país, inevitablemente genera \textbf{ganadores y perdedores} internos. La oposición al libre comercio proviene de los grupos que pierden ingresos (los factores escasos), lo que explica por qué los propietarios de recursos escasos respaldan las restricciones.

\subsubsection{Caso 2.12. Evidencia Empírica de la Teoría de H-O (Paradoja de Leontief)}
Los intentos iniciales de validar empíricamente la teoría H-O obtuvieron resultados mixtos o contradictorios. El primer intento empírico sólido fue realizado por Wassily Leontief en 1954.

\begin{definicion}
La \textbf{Paradoja de Leontief} se refiere al hallazgo de que las exportaciones estadounidenses (un país abundante en capital) eran \textbf{menos intensivas en capital} que los productos de competencia de importación. Estos resultados iniciales contradecían la predicción de la teoría H-O.
\end{definicion}
Investigaciones posteriores, como las de Bowen, Leamer y Sveikauskas, que ampliaron la prueba a múltiples países y factores, también encontraron que el contenido factorial del comercio apuntaba en la dirección opuesta a H-O en el 39\% de los casos.

Sin embargo, estudios más recientes han demostrado que si se relajan los supuestos irrealistas (como la tecnología común y la ausencia de costes de comercio), una versión menos restrictiva del modelo de proporciones factoriales se ajusta bastante bien al patrón de comercio observado. El consenso es que la dotación de factores explica \textbf{solo una porción} de los patrones comerciales.

\subsubsection{Caso 2.13. Does International Trade Reduce Poverty and Income Inequality?}
El debate sobre si el comercio reduce la pobreza y la desigualdad de ingresos se relaciona con los efectos distributivos de H-O. Aunque el libre comercio puede proporcionar ganancias generales para un país, sus efectos internos son complejos. En países avanzados y ricos en capital humano (como EE. UU.), el comercio con países abundantes en trabajo no especializado (como China o México) puede \textbf{agravar la inequidad de ingresos} al aumentar la renta del factor abundante (trabajo cualificado) y reducir la renta del factor escaso (trabajo no especializado). Sin embargo, la teoría económica estándar afirma que el comercio es \textbf{beneficioso para los trabajadores} del país abundante en trabajo (país en desarrollo). El crecimiento económico en países como China ha permitido que cientos de millones de personas superen la pobreza, un resultado que se relaciona, al menos parcialmente, con la especialización basada en la ventaja comparativa de su dotación de factores.

\chapter{El Comercio Internacional y el Movimiento de los Factores Productivos}

\section{La Movilidad del Trabajo y los Flujos Financieros y de Capital a Nivel Internacional (2.5.)}
Además del comercio de bienes y servicios, la \textbf{integración económica global} se mide por los movimientos de los factores de producción, incluyendo el trabajo y el capital. A lo largo de la historia, los flujos internacionales de estos factores han sido responsables de algunos de los cambios más drásticos en la economía mundial.

\subsection{La Movilidad Internacional del Trabajo (2.5.A.)}
La \textbf{migración} (movilidad internacional del trabajo) se define como un movimiento de factor productivo que, junto con el endeudamiento internacional, constituye una forma de comercio que resulta \textbf{mutuamente beneficiosa} para los países. Los trabajadores se mueven en respuesta a las diferencias salariales entre países, buscando mejores oportunidades económicas.

\subsubsection{Modelo Teórico y Efectos de la Migración}
El análisis de la migración en la teoría económica se asemeja al modelo de los factores específicos, centrándose en las consecuencias a \textbf{corto plazo} de estos flujos.

\begin{enumerate}
    \item \textbf{Causas y Convergencia Salarial}: El flujo de trabajadores ocurre en respuesta a los \textbf{diferenciales salariales} (de un país con salarios bajos a uno con salarios altos). Si la migración es sin restricciones ni costos, el proceso continuará hasta que los salarios se \textbf{igualen} en ambos países.
    \item \textbf{Efectos sobre la Producción Mundial}: La migración internacional resulta en un \textbf{aumento neto en la producción mundial}. Esto se debe a que los trabajadores son atraídos hacia áreas donde su productividad es más alta (VPMg del trabajo es mayor), lo que resulta en un uso más productivo de la mano de obra.
    \item \textbf{Efectos sobre la Distribución de la Renta}: La migración afecta significativamente la distribución del ingreso, aunque el ingreso mundial agregado aumente.
    \begin{itemize}
        \item \textit{País de Inmigración (Receptor)}: Los \textbf{dueños del capital ganan} debido a la mayor oferta de trabajo disponible. Los \textbf{trabajadores nativos pierden} porque sus salarios se reducen. El país en su conjunto se beneficia.
        \item \textit{País de Emigración (Emisor)}: Los \textbf{trabajadores restantes ganan} debido al aumento de los salarios (la mano de obra se vuelve más escasa). Los \textbf{dueños del capital pierden} debido a la escasez de mano de obra. El país en su conjunto experimenta pérdidas de ingreso.
    \end{itemize}
\end{enumerate}

\subsubsection{Caso 2.14. Inmigration to the New World y Caso 2.15. The Political Economy of Migration: Inmigration Into the United States}
La \textbf{movilidad del trabajo} no ha aumentado para Estados Unidos durante los últimos 100 años.

\begin{itemize}
    \item \textbf{Olas Históricas}: La primera ola de globalización (1870-1914) fue la etapa dorada de la movilidad laboral, donde la inmigración fue una causa principal del crecimiento demográfico en países como Estados Unidos. Durante este periodo, la política migratoria estadounidense fue muy liberal, aunque luego se impusieron restricciones.
    \item \textbf{Restricciones Históricas}: Estados Unidos promulgó la Ley de Migración de 1924, que impuso cuotas que favorecían a los migrantes del norte de Europa. En la década de 1920, se impusieron estrictas restricciones, finalizando esa primera ola y haciendo que la población nacida en el extranjero disminuyera.
    \item \textbf{Ola Reciente}: Una nueva oleada de inmigración se inició alrededor de 1970, proviniendo principalmente de \textbf{Latinoamérica y Asia}. Para 2007, aproximadamente el 12\% de la población estadounidense había nacido en el extranjero, y el 14\% de la fuerza de trabajo era extranjera.
    \item \textbf{Composición de Habilidades}: Los trabajadores nacidos en el extranjero en Estados Unidos se concentran en los grupos educativos \textbf{más altos y más bajos}. Por ejemplo, el 60\% de los trabajadores con doctorado en ciencias, tecnología, ingeniería y matemáticas (CTIM) son de origen extranjero.
    \item \textbf{Economía Política de la Migración}: Los \textbf{sindicatos} y los \textbf{trabajadores nacionales} se oponen a la migración más libre, ya que temen que reduzca sus niveles salariales y de empleo. Los \textbf{fabricantes} y dueños de capital, por otro lado, tienden a favorecer la migración como fuente de mano de obra barata.
\end{itemize}

\subsubsection{Caso 2.16. The Political Economy of Migration: Ilegal Inmigration Into the United States}
La migración ilegal representa un problema político significativo en Estados Unidos.

\begin{itemize}
    \item \textbf{Impacto Económico}: La migración ilegal (millones de migrantes, muchos de México) proporciona una \textbf{oferta barata de trabajadores} para los sectores agrícola y no calificado en Estados Unidos, especialmente en los estados del suroeste.
    \item \textbf{Distribución de Renta}: La migración ilegal tiende a \textbf{reducir el ingreso} de los trabajadores estadounidenses no calificados.
    \item \textbf{Divisas}: Para el país de emigración, como México, la migración ilegal es una fuente importante de \textbf{divisas} y actúa como un amortiguador contra el desempleo nacional.
    \item \textbf{Regulación}: La Ley de Reforma y Control de Migración de 1986 impuso multas a los empleadores que contraten migrantes ilegales.
\end{itemize}

\subsubsection{Caso 2.17. The Political Economy of Migration: Refugees at the U.S. Border; Refugees Entering into Europe and Political Economy}
La migración no se limita a factores económicos, sino que también es impulsada por \textbf{razones no económicas} como la política, la guerra y la religión. Los refugiados representan un componente de esta migración forzada que genera controversias políticas en las fronteras, como en el caso de Estados Unidos y Europa.

\subsubsection{Caso 2.18. The Effects of the Mariel Boat Lift on Industry Output in Miami y Caso 2.19. The Effects of Migration on Wages}
Los estudios empíricos buscan cuantificar los efectos reales de los flujos migratorios, especialmente en los salarios y la producción.

\begin{itemize}
    \item \textbf{Efectos Salariales a Corto Plazo}: Un estudio sobre la migración (principalmente de origen mexicano) a Estados Unidos (1980-2000) encontró que a corto plazo, la migración \textbf{redujo el salario promedio} de los trabajadores en competencia en un \textbf{3\%}. Para los trabajadores que truncaron el bachillerato, la reducción salarial fue del \textbf{8\%}.
    \item \textbf{Efectos Salariales a Largo Plazo}: A largo plazo, el capital se ajusta completamente. Aunque el salario promedio de los trabajadores en competencia no se vio afectado, el salario de aquellos que truncaron el bachillerato disminuyó aproximadamente un \textbf{5\%}. Esto confirma que, a largo plazo, el efecto negativo se concentra en los trabajadores menos capacitados de Estados Unidos. Las estimaciones del efecto en los salarios pueden variar considerablemente, desde una reducción del 8\% hasta cifras cercanas a cero.
    \item \textbf{Efecto en la Producción Agregada}: El Producto Interno Bruto (PIB) de Estados Unidos es \textbf{mayor} debido a la presencia de inmigrantes. El efecto global neto de la migración en el PIB se estima entre \textbf{\$1 mil millones y \$10 mil millones} anualmente.
\end{itemize}

\subsubsection{Caso 2.20. Inmigrants and their Remittances}
Aunque el término "remesas" no se detalla, la migración laboral proporciona una importante fuente de divisas para el país de emigración.

\subsubsection{Caso 2.21. Measuring the Gains from Inmigration}
La medición de las ganancias de la inmigración confirma que el beneficio económico es, en términos netos, reducido en relación con el PIB, pero no obstante positivo, y que su principal impacto es distributivo.

\begin{itemize}
    \item \textbf{Ganancias y Consumo}: La economía en general se beneficia porque los migrantes, al ser distintos a los nativos, pueden causar que los precios caigan, lo que \textbf{beneficia a todos los consumidores}.
    \item \textbf{Costo Fiscal Neto}: Las estimaciones del costo fiscal neto de la inmigración (impuestos pagados frente a gastos en servicios públicos) son \textbf{muy reducidas}, del orden del 0.1\% del PIB.
    \item \textbf{Distribución de la Renta}: La migración actual \textbf{redistribuye la riqueza} de los trabajadores no capacitados (cuyos salarios son reducidos) hacia los trabajadores capacitados, los dueños de empresas (que utilizan los servicios de los migrantes) y los consumidores.
    \item \textbf{Pérdida de Capital Humano}: Los países en desarrollo temen la \textbf{fuga de cerebros} (migración de personas altamente educadas), lo que puede limitar su crecimiento potencial.
    \item \textbf{Movimiento Temporal}: La migración puede ser temporal, como el caso de los \textbf{trabajadores invitados} en la Unión Europea, lo que protege a la economía receptora de los excedentes de trabajo en recesiones, pero traslada el problema a los países de emigración.
\end{itemize}


\subsection{Los Flujos Internacionales de Capital (2.5.B.)}

El movimiento internacional de capital (inversión) es, junto con el comercio de bienes y la migración laboral, uno de los principales componentes de la \textbf{integración económica global}. Estos flujos financieros internacionales constituyen una forma de comercio \textbf{mutuamente beneficiosa} que implica el \textbf{intercambio de bienes presentes por bienes futuros} (comercio intertemporal).

\subsubsection{Naturaleza y Determinantes del Flujo de Capital}
Los flujos de capital se registran en la \textbf{Cuenta de Capital y Financiera} de la balanza de pagos. Estos flujos incluyen la compra o venta de activos internacionales, tales como acciones, bonos, bienes raíces y depósitos bancarios.

\begin{enumerate}
    \item \textbf{Decisión de Inversión}: El factor más importante que impulsa los flujos de activos financieros a través de las fronteras nacionales son las \textbf{tasas de interés del mercado nacional y extranjero}. Los inversionistas comparan las tasas de recuperación de las inversiones en el extranjero con las de las inversiones en su país, buscando los rendimientos más altos (proceso conocido como arbitraje de intereses).
    \item \textbf{Otros Factores}: Además de los diferenciales de las tasas de interés, los flujos de capital están determinados por la \textbf{rentabilidad de las inversiones}, las políticas fiscales de las naciones y la \textbf{estabilidad política}.
    \item \textbf{Efecto en la Cuenta Corriente}: El flujo de capital y financiero puede iniciar cambios en la cuenta corriente. Un flujo entrante neto de capital a un país (una entrada financiera) se asocia con un déficit de cuenta corriente. Esto implica que el país está pidiendo prestado del exterior para financiar su gasto.
\end{enumerate}

\begin{nota}
En el siglo XXI se observó un fenómeno que contrasta con la teoría económica tradicional: un \textbf{exceso de ahorro global} fluía de los países en desarrollo y mercados emergentes hacia los países desarrollados, principalmente Estados Unidos, en lugar de lo contrario (que el capital fluyera a donde la tasa de rendimiento esperada fuera más alta). Este exceso de ahorro global contribuyó a mantener bajas las tasas de interés reales.
\end{nota}

\subsection{La Inversión Extranjera y la Empresa Multinacional (EMN) (2.5.C.)}

La inversión extranjera directa (IED) representa el mecanismo fundamental mediante el cual el capital es reasignado internacionalmente.

\subsubsection{Definición y Clasificación de la EMN y la IED}

\begin{definicion}
Una \textbf{Empresa Multinacional (EMN)} es una empresa que opera en muchos países, realiza actividades de investigación y desarrollo, operaciones de manufactura, minería o servicios, y es dirigida desde un centro corporativo de planeación, con una alta razón de ventas al extranjero.
\end{definicion}

\begin{definicion}
La \textbf{Inversión Extranjera Directa (IED)} ocurre cuando los residentes de un país adquieren una \textbf{participación de control} (generalmente definida como 10 por ciento o más del capital accionario) en una empresa o instalación en el extranjero.
\end{definicion}

Las EMN diversifican sus operaciones a nivel mundial de tres maneras principales:
\begin{enumerate}
    \item \textbf{Integración Vertical}: La EMN establece subsidiarias en el extranjero para fabricar \textbf{productos intermedios o insumos} para el producto terminado (integración hacia atrás o hacia adelante). Esto se realiza para beneficiarse de economías de escala y especialización internacional.
    \item \textbf{Integración Horizontal}: La empresa principal que produce un bien en el país de origen establece una subsidiaria para producir el \textbf{producto idéntico} en el país anfitrión. Esto se relaciona con la elección entre \textbf{proximidad y concentración}.
    \item \textbf{Integración Conglomerada}: Implica la diversificación en mercados no relacionados.
\end{enumerate}

\subsubsection{Motivos de la Inversión Extranjera Directa}

La IED se realiza con la \textbf{expectativa de ganancias futuras} y está activamente atraída por los países anfitriones.

\begin{itemize}
    \item \textbf{Factores de Demanda (Búsqueda de Mercados)}: Las EMN buscan nuevos mercados y fuentes de demanda, a menudo estableciendo subsidiarias para atender mercados extranjeros que no pueden ser satisfechos adecuadamente por las exportaciones directas.
    \item \textbf{Costos de los Factores y Recursos}: La IED se realiza para \textbf{reducir los costos de producción} mediante el acceso a materias primas esenciales o al aprovechamiento de condiciones climáticas favorables y \textbf{costos laborales bajos}.
    \item \textbf{Restricciones Comerciales (Rodear Barreras)}: La IED es una forma de \textbf{rodear las barreras arancelarias de importación} impuestas por el país anfitrión. Por ejemplo, los altos aranceles brasileños motivan a los productores extranjeros de automóviles a ubicar instalaciones en Brasil.
\end{itemize}

El análisis de las EMN concuerda con el principio de la ventaja comparativa: un artículo se producirá en el país de bajo costo. La diferencia es que la teoría del comercio convencional enfatiza el movimiento de la mercancía, mientras que la IED enfatiza el \textbf{movimiento internacional de los insumos de factores}.

\subsubsection{Caso 2.22. The Effect of FDI on Rentals and Wages in Singapore}
La IED se considera un factor determinante del crecimiento económico y la creación de empleos en el país anfitrión. Las empresas donde la IED es intensa tienden a tener un alto promedio de \textbf{productividad de mano de obra} y pagan \textbf{salarios más altos}.

\begin{propuesta}
El flujo de IED trae consigo \textbf{transferencia de tecnología} y habilidades gerenciales a los países en desarrollo. Esto contribuye a una mayor prosperidad en los países anfitriones que participan.
\end{propuesta}

\subsubsection{Titular 2.2. The Myth of Asia’s Miracle}
El debate sobre el "milagro asiático" (referido a las economías de Asia Oriental como Hong Kong, Corea del Sur, Singapur y Taiwán, que lograron un crecimiento fuerte y sostenido) está relacionado con la fuente de ese crecimiento. Estos países compartieron características como \textbf{altas tasas de inversión} y crecientes dotaciones de \textbf{capital humano}.

La controversia, especialmente destacada a principios de los noventa, reside en si el espectacular crecimiento se debió a una \textbf{acumulación de factores} (aumento de capital y trabajo) o a un incremento en la \textbf{productividad total de los factores}. Algunos analistas cuestionaron si la clave del éxito era la "magia" de la política industrial o simplemente la movilización masiva de recursos, sugiriendo que el crecimiento podría ser insostenible si no se basaba en mejoras de eficiencia.

\subsubsection{Caso 2.23. Measuring the Gains from Foreign Direct Investment}
Las ganancias derivadas de la IED se manifiestan en la mejora de la \textbf{asignación mundial de recursos}. Si las EMN obtienen un rendimiento más alto en inversiones en el extranjero que en su país de origen, los recursos se transfieren de usos menos productivos a otros de \textbf{mayor productividad}.

\begin{itemize}
    \item La IED es bienvenida porque genera \textbf{derramas tecnológicas} (spillovers) y mejora la administración en el país receptor.
    \item Los flujos de inversión hacia fuera (outward) permiten a las empresas del país de origen seguir siendo competitivas, \textbf{respaldando el empleo} en el país de origen e \textbf{incentivando las exportaciones} de maquinaria y bienes de capital.
    \item En términos de bienestar, la IED mejora el bienestar total tanto del país de origen como del país anfitrión, ya que la asignación global de recursos se optimiza.
\end{itemize}

\subsubsection{La Inversión Extranjera y la Empresa Multinacional: Conflictos}
A pesar de los beneficios, las EMN también son una \textbf{fuente de conflicto} que puede generar fricciones con los objetivos económicos y políticos de las naciones.

\begin{enumerate}
    \item \textbf{Empleo}: Existe controversia sobre los efectos de la IED en el empleo, especialmente cuando la inversión se utiliza para \textbf{comprar empresas locales existentes} en lugar de establecer nuevas plantas (lo que no aumenta necesariamente la capacidad de producción).
    \item \textbf{Soberanía Nacional}: Las EMN pueden ser vistas como una amenaza a la soberanía nacional si ejercen una \textbf{influencia política excesiva}.
    \item \textbf{Balanza de Pagos y Aspectos Fiscales}: Aunque la inversión inicial representa un flujo de capital saliente (negativo a corto plazo), a largo plazo, la IED fortalece la balanza de pagos del país de origen a través del aumento de las \textbf{exportaciones de equipos de capital} y el \textbf{flujo de rendimientos} de los ingresos generados por las operaciones en el extranjero (intereses y dividendos).
\end{enumerate}