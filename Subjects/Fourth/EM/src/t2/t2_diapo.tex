\chapter{El Comercio Internacional y el Movimiento de los Factores Productivos}

\section{Introducción}
El estudio del comercio internacional y la movilidad de los factores productivos se centra en dos interrogantes fundamentales: \textbf{por qué existe el intercambio de bienes y servicios} entre naciones, y \textbf{por qué ocurre la movilidad internacional del trabajo y del capital}. La respuesta a estas cuestiones implica la revisión sistemática de la teoría del comercio internacional, que abarca desde los modelos clásicos hasta las explicaciones contemporáneas basadas en la dotación de factores y las economías de escala.

El análisis del comercio mundial (Caso 2.1) revela la distribución masiva de los flujos internacionales. La \textbf{Unión Europea} se consolida como la región con el mayor volumen de comercio interno, alcanzando los 4.5 billones de dólares, facilitado por la ausencia de aranceles y la proximidad geográfica. En términos de exportaciones globales, Europa y América representan conjuntamente cerca del 49\% del total mundial, y Asia contribuye con aproximadamente el 29\%. Regiones como África (2\%) y Oriente Medio/Rusia (9\%) tienen una participación menor, basando estas últimas sus exportaciones en recursos energéticos como el petróleo y el gas natural.

\section{La teoría clásica}
La \textbf{teoría clásica} se enfoca en explicar la base del comercio (por qué comercian los países), las ganancias obtenidas, el patrón (dirección) del comercio y los precios de intercambio. Esta teoría se cimienta en una serie de simplificaciones metodológicas esenciales para el desarrollo de los modelos de ventaja absoluta y comparativa.

\subsection{Supuestos}
Los supuestos fundacionales de la teoría clásica, particularmente la \textbf{Ricardiana}, son estrictos:
\begin{enumerate}
    \item \textbf{El trabajo ($\mathbf{L}$) es el único factor de producción} empleado, y se utiliza en la misma proporción para la producción de cualquier bien.
    \item El trabajo es \textbf{homogéneo}, sin diferencias en su productividad \textit{dentro} de una nación. (La diferencia clave radica en la productividad de $L$ \textit{entre} naciones).
    \item Los \textbf{costes de oportunidad son constantes}.
\end{enumerate}

\subsection{El modelo de la ventaja absoluta}
El modelo de la \textbf{Ventaja Absoluta}, desarrollado por Adam Smith, postula que \textbf{dos naciones solo comerciarán si ambas se benefician}. La \textbf{Ley de las Ventajas Absolutas} establece que cuando una nación es más eficiente (o tiene una ventaja absoluta) en la producción de un bien, y la otra nación es más eficiente en la producción de un segundo bien, el comercio es mutuamente ventajoso. Esto promueve una asignación más eficiente de los recursos y un aumento de la producción global.

El Caso 2.3, relativo al intercambio de trigo y plátanos entre Canadá y Nicaragua, ilustra este principio. El \textbf{clima continental} de Canadá le confiere una ventaja en la producción de cereales (trigo), mientras que el \textbf{clima tropical} de Nicaragua le otorga una ventaja en la producción de plátanos. Al especializarse cada uno en el bien que produce más eficientemente debido a sus condiciones naturales, ambos países obtienen beneficios y pueden registrar superávit comercial en el intercambio mutuo.

Sin embargo, el principio puede generar \textbf{fricciones políticas y económicas} (Caso 2.2). El ingreso de España en la Comunidad Económica Europea (CEE) se encontró con la fuerte oposición de los agricultores franceses. La ventaja comparativa de España en productos hortofrutícolas se debía a que su coste de producción era significativamente menor, y la temperatura del sur de España permitía una mayor producción por hectárea. Los agricultores del sur de Francia reaccionaron violentamente, incluso incendiando camiones, con el objetivo de presionar al gobierno francés para que impidiera el ingreso de España. El argumento español, por su parte, se basaba en el beneficio mutuo: España exportaría productos agrícolas y la CEE (incluida Francia) exportaría bienes manufacturados de alta tecnología.

El Caso 2.4 (''Rosas por San Valentín'') subraya la importancia de mirar el \textbf{comercio bilateral completo}. La crítica del candidato Patrick Buchanan a la importación de rosas de bajo coste (favorecido por el clima y la mano de obra colombiana) a EE. UU. fue miope. El argumento económico recíproco es que EE. UU. exporta a Colombia productos tecnológicos (como procesadores de ordenador) en los que tiene ventaja comparativa. Los ingresos de EE. UU. por la exportación de estos procesadores a Colombia suelen \textbf{superar el gasto} en la importación de rosas, demostrando que el comercio beneficia a ambas partes.

\subsection{El modelo de la ventaja comparativa}
El modelo de la \textbf{Ventaja Comparativa}, formulado por David Ricardo, supera la limitación del modelo de Smith, demostrando que el comercio mutuamente beneficioso es posible \textbf{incluso si una nación tiene una desventaja absoluta en la producción de todos los bienes}.

La Ley de las Ventajas Comparativas establece que la nación con desventaja absoluta debe especializarse en la producción del bien en el que su \textbf{desventaja absoluta sea menor} (es decir, donde tenga la \textbf{ventaja comparativa}) para exportarlo, e importar el bien en el que su desventaja sea mayor. La ventaja comparativa se define por el diferencial relativo de productividad entre países.

El Caso 2.5 analiza la ventaja comparativa en la producción de ropa, textiles y trigo entre Estados Unidos y China. Estados Unidos posee una \textbf{ventaja absoluta} en la productividad de todos estos bienes, siendo 30 veces más productivo en trigo por trabajador que China. Sin embargo, China tiene una \textbf{ventaja comparativa} en textiles y ropa. Esta ventaja surge al comparar el \textbf{coste de oportunidad}. Por ejemplo, el coste de oportunidad en EE. UU. de obtener un dólar extra en ventas de prendas de vestir es 13,500/\$58,000 $\approx$ 0.23 fanegas de trigo sacrificadas, mientras que en textiles es 13,500/\$135,000 $\approx$ 0.10 fanegas de trigo sacrificadas. Aunque el análisis numérico completo no se presenta, el principio es claro: China se especializa en bienes de baja intensidad tecnológica, a pesar de la ventaja absoluta de EE. UU. en todos los sectores, porque su coste de oportunidad relativo es menor.

\subsection{La ventaja comparativa y el coste de oportunidad}
La decisión de especialización se determina por la \textbf{Ley del Coste Comparativo}: una nación se especializa en la producción del bien en el que tiene el \textbf{coste de oportunidad más bajo}. El coste de oportunidad de producir una unidad de un bien ($X$) se mide por la cantidad del otro bien ($Y$) a la que se debe renunciar.

La especialización de un país está condicionada por la relación entre los precios relativos de los bienes y su coste de oportunidad.

\subsection{Especialización y precio de equilibrio}
Un país se especializará en la producción de un bien ($X$) si la relación del precio de $X$ respecto a $Y$ ($\frac{p_X}{p_Y}$) es \textbf{mayor} que el coste de oportunidad de $X$ en términos de $Y$ ($\text{c}_{YX}^N$). Inversamente, se especializará en $Y$ si $\frac{p_X}{p_Y} < \text{c}_{YX}^N$.

El \textbf{precio de equilibrio} internacional se determina en una situación de equilibrio general, donde la demanda mundial total ($DM$) es igual a la oferta mundial total ($OM$).

\section{La teoría estándar del comercio internacional}
La \textbf{teoría estándar} o \textbf{neoclásica} del comercio internacional introduce una mejora crucial respecto a la teoría clásica: sustituye el supuesto de costes de oportunidad constantes por la \textbf{realidad de costes de oportunidad crecientes}. Este marco utiliza la Frontera de Posibilidades de Producción (FPP) con curvatura, lo que permite un análisis más realista de la oferta y la demanda.

\subsection{Frontera de posibilidades de producción y oferta con costes de oportunidad crecientes}
En este modelo, la \textbf{Frontera de Posibilidades de Producción (FPP)} es cóncava respecto al origen. El nivel de producción ($Q$) óptimo se deduce del proceso de maximización del valor de producción ($V$), representado por la \textbf{recta isovalor} ($V = p_X Q_X + p_Y Q_Y$). La producción se sitúa en el punto de tangencia entre la FPP y la isovalor superior, donde la pendiente de la FPP (el coste de oportunidad) es igual a la pendiente de la línea isovalor (el precio relativo).

\subsection{La demanda}
La determinación del nivel de demanda ($D$) se realiza a través de la \textbf{función de bienestar} o función de utilidad ($U$) de los consumidores. El nivel de demanda de equilibrio se alcanza mediante el proceso de maximización de la utilidad sujeta a la restricción presupuestaria (el valor de la producción, $V$). Gráficamente, esto se representa por el punto de tangencia entre la línea isovalor superior y la curva de indiferencia más alta alcanzable ($I$).

\subsection{Equilibrio parcial en un país}
El \textbf{equilibrio parcial de autarquía} para un país se alcanza en el punto donde coinciden la maximización de la producción y la maximización de la utilidad. Este es el punto de tangencia simultánea entre la FPP, la recta isovalor más alta ($V$) y la curva de indiferencia más alta ($I$).

\subsection{Base del comercio internacional y ganancias}
La \textbf{base para el comercio internacional} surge de las diferencias en los precios relativos entre los países. Si el precio relativo del bien $Y$ es mayor en el País $E$ que en el País $N$, el País $E$ tiene ventaja comparativa en $Y$ y se especializará en él, mientras que $N$ se especializará en $X$.

Las \textbf{ganancias del comercio} se manifiestan en un aumento del bienestar (utilidad) de los países participantes. Al abrirse al comercio, el país se enfrenta a un \textbf{precio relativo internacional} que le permite moverse a una recta isovalor superior (la \textit{world price line} en la terminología de Feenstra \& Taylor). Al especializarse en su bien de ventaja comparativa, la nación puede alcanzar una curva de indiferencia ($I_2$) que se encuentra más allá de su FPP original de autarquía ($I_1$), aumentando así su consumo y bienestar.

El Caso 2.7 (''How Large Are the Gains from Trade?'') cuantifica estas ganancias. Los ejemplos históricos muestran que las ganancias del comercio son \textbf{positivas y significativas}, estimadas en alrededor del \textbf{4\% al 5\% del PIB}. Por ejemplo, el embargo de EE. UU. (1807–1809) implicó un coste estimado de $\approx 5\%$ del PIB, lo que sugiere que las ganancias de comercio libre eran de esa magnitud. La apertura de Japón en 1854 resultó en ganancias estimadas de $4-5\%$ del PIB.

\subsection{La determinación del precio relativo a nivel mundial (equilibrio general)}
El \textbf{precio relativo de equilibrio} a nivel mundial se determina por la interacción de la demanda y la oferta a escala global. Gráficamente, esto se halla en el punto donde la demanda relativa mundial (determinada por la suma de las demandas relativas de los bienes $X$ y $Y$ en $N$ y $E$) se cruza con la oferta relativa mundial.

\section{La dotación de factores y la teoría de Heckscher-Ohlin (H-O)}
La \textbf{Teoría de Heckscher-Ohlin (H-O)} representa un avance fundamental, ya que profundiza en las razones que justifican las diferencias en los precios relativos (la base del comercio) y, crucialmente, \textbf{analiza los efectos del comercio en la remuneración de los factores productivos} (la distribución de la renta).

\subsection{Intensidad y abundancia de factores y la frontera de producción}
El modelo H-O se construye sobre los conceptos de intensidad y abundancia de factores:
\begin{itemize}
    \item \textbf{Intensidad Factorial:} La producción de un bien es \textbf{intensiva en capital ($K$)} si la relación $K/L$ utilizada en su producción es mayor que la relación $K/L$ utilizada en la producción del otro bien.
    \item \textbf{Abundancia de Factores (Definición 1):} Un país es \textbf{abundante en capital} si el capital disponible en relación con el trabajo total ($K/L$) es mayor que en el otro país.
    \item \textbf{Abundancia de Factores (Definición 2 - Precios):} Un país es \textbf{abundante en capital} si el precio relativo del capital ($r$) con respecto al trabajo ($w$), es decir, la razón $r/w$, es inferior al precio relativo de los factores en el otro país.
\end{itemize}
La Proposición 2.2 establece que un país con una FPP de costes crecientes se especializará en la producción del bien que requiera un uso intensivo del factor relativamente abundante en dicho país.

El Caso 2.8 cuestiona si las intensidades factoriales son las mismas en todos los países. La evidencia sugiere que \textbf{no lo son}. Un mismo bien (e.g., automóviles) puede ser producido con \textbf{diferentes combinaciones de capital y trabajo} dependiendo de la tecnología y los costes relativos del país. Por ejemplo, EE. UU. (país desarrollado) puede producir automóviles de forma intensiva en capital (robots), mientras que la India (país en desarrollo) puede producirlos de forma intensiva en trabajo (mano de obra).

\subsection{Supuestos}
Además de los supuestos neoclásicos generales, la versión canónica del modelo H-O establece:
\begin{enumerate}
    \item Hay dos naciones, dos bienes ($X$, $Y$) y dos factores productivos ($K$, $L$).
    \item \textbf{La tecnología es idéntica en ambas naciones}.
    \item \textbf{Las preferencias de los consumidores son idénticas} en ambos países.
    \item La producción exhibe \textbf{rendimientos constantes a escala} y la especialización es \textbf{incompleta}.
    \item Existe \textbf{movilidad perfecta de los factores dentro del país}, pero \textbf{inmovilidad entre países}.
    \item No hay costes de transporte, aranceles ni obstáculos al comercio.
    \item Se asume plena utilización de $K$ y $L$.
\end{enumerate}

\subsection{La teoría de H-O}
La teoría H-O se articula en torno a tres teoremas principales: H-O, de Igualación de Precios de los Factores (H-O-S) y Stolper-Samuelson.

\textbf{Teorema de Heckscher-Ohlin:} Este teorema (o Teorema de la Dotación de Factores) establece que \textbf{un país exportará el bien que requiera el uso intensivo del factor relativamente abundante} y, a su vez, importará el bien cuya producción requiera el uso intensivo del factor relativamente escaso.

El Caso 2.9 (La evolución del capital en China) proporciona evidencia empírica de la evolución dinámica de la dotación de factores. Históricamente, China era escasa en capital físico en el año 2000, pero la masiva acumulación de capital hizo que pasara a ser \textbf{abundante en capital físico en 2010}. El análisis de la dotación ''efectiva'' de científicos I+D (ajustada por productividad) muestra una escasez hasta 2013, pero en \textbf{2017, China se volvió abundante} en científicos I+D efectivos. Esta rápida transformación sugiere una evolución de su ventaja comparativa hacia bienes más intensivos en capital y tecnología, intensificando la competencia con países desarrollados en la producción de bienes de alta tecnología.

\subsection{La igualación del precio de los factores y la remuneración de los factores (distribución de la renta)}

\textbf{Teorema de Igualación de Precios de los Factores (H-O-S):} Este postulado predice que el comercio internacional de bienes \textbf{igualará la remuneración (precio) relativa y absoluta de los factores productivos} entre naciones, haciendo que el comercio actúe como un sustituto de la movilidad de factores. Este resultado idealizado se basa en los estrictos supuestos de H-O y, como se observa en la práctica, no se cumple totalmente debido a diferencias tecnológicas y la especialización incompleta. El Caso 2.12 (Evidencia Empírica) aborda esta prueba.

\textbf{Teorema de Stolper-Samuelson:} Este teorema analiza cómo el comercio afecta la distribución interna de la renta: \textbf{los propietarios del factor abundante ganan con el comercio, mientras que los propietarios del factor escaso pierden}. Por ejemplo, en un país abundante en trabajo ($L$) que exporta un bien intensivo en $L$, la demanda de $L$ aumenta, elevando el salario ($w$), y la demanda del factor escaso ($K$) cae, reduciendo la renta del capital ($r$).

El Caso 2.10 (Opiniones sobre el Libre Comercio) valida la implicación de Stolper-Samuelson sobre la existencia de ganadores y perdedores. Las opiniones sobre el libre comercio dependen del \textbf{interés económico personal}:
\begin{itemize}
    \item \textbf{A favor:} Trabajadores con alta cualificación/educación, y propietarios de empresas/tierra en zonas exportadoras.
    \item \textbf{En contra:} Trabajadores con baja cualificación/salarios bajos, y zonas afectadas por las importaciones.
\end{itemize}

El Caso 2.13 (Comercio, Pobreza y Desigualdad) muestra que, aunque el comercio ha \textbf{reducido la pobreza global} (principalmente por la apertura de China), ha \textbf{aumentado la desigualdad interna} tanto en China como en EE. UU., ya que los trabajadores cualificados ganan más que los poco cualificados. Esto contradice la predicción simple de H-O de que el factor abundante (trabajo poco cualificado en China) debería ver reducida la desigualdad interna.

El Caso 2.12 también aborda la \textbf{Paradoja de Leontief}, la primera prueba empírica de H-O. Leontief encontró que EE. UU. (país abundante en capital, $K$) exportaba bienes intensivos en trabajo ($L$) e importaba bienes intensivos en $K$. Investigaciones posteriores sugirieron que si se ajustaba el factor trabajo por su \textbf{productividad} (trabajo ''efectivo'' o cualificado), EE. UU. resultaba ser abundante en trabajo cualificado, lo que resolvería la paradoja.

\section{La movilidad del trabajo y los flujos financieros y de capital a nivel internacional}
El \textbf{movimiento internacional de factores} es otra alternativa de integración económica, aunque se rige por regulaciones legales y diferencias institucionales que lo distinguen del comercio de bienes.

\subsection{La movilidad internacional del trabajo}
La \textbf{migración} ocurre debido a las diferencias en el salario real ($\text{wr}$) entre países. En un mercado laboral competitivo, el trabajo emigra del país con salario real menor al país con salario real mayor.

El análisis macroeconómico de la migración (utilizando el concepto del producto marginal del trabajo, $\text{PMgL}$) demuestra que la migración del País $N$ al País $E$ conduce a la \textbf{convergencia de salarios} hacia un salario de equilibrio mundial ($\text{wr}^M$). Aunque la migración reduce el salario real en el país receptor ($E$) y lo aumenta en el país emisor ($N$), la \textbf{producción total mundial aumenta}, siendo la ganancia neta para el mundo el área triangular resultante del desplazamiento.

\textbf{Contexto Histórico (Caso 2.14):} La gran ola migratoria europea al Nuevo Mundo (1870–1913, con 30 millones de migrantes) buscó mejores salarios. Los salarios del Nuevo Mundo eran casi tres veces más altos en 1870. La inmigración redujo el crecimiento salarial en el Nuevo Mundo y lo aceleró en Europa, resultando en una \textbf{convergencia salarial}.

\textbf{Efectos en Salarios (Corto vs. Largo Plazo - Caso 2.19/2.21):}
\begin{itemize}
    \item \textbf{Corto Plazo (Modelo de Factores Específicos):} Asumiendo capital fijo, las estimaciones iniciales de la inmigración en EE. UU. sugieren una \textbf{caída salarial promedio de -3.0\%} para todos los trabajadores (1990–2006), afectando más a los trabajadores nativos con baja y alta cualificación.
    \item \textbf{Largo Plazo (Modelo H-O ampliado):} Al permitir el \textbf{ajuste del capital}, el efecto salarial promedio total es casi nulo ($+0.1\%$). Esto se debe a que la economía absorbe la mano de obra entrante a través de la expansión de la producción del bien intensivo en ese factor (Teorema de Rybczynski). Además, la evidencia sugiere que inmigrantes y nativos son \textbf{sustitutos imperfectos} (complementarios), lo que resulta en un ligero aumento salarial para los nativos ($+0.6\%$).
\end{itemize}
\textbf{Teorema de Rybczynski y Evidencia (Caso 2.15/2.18):} El éxodo de Mariel a Miami (1980, 125,000 refugiados cubanos, en su mayoría poco cualificados), fue un experimento natural para el \textbf{Teorema de Rybczynski}. Este teorema predice que, a precios constantes, un aumento en la oferta de un factor (trabajo no cualificado) debería expandir el sector que lo usa intensivamente (industria textil/confección) y contraer el otro sector (alta cualificación). La evidencia mostró una \textbf{confirmación parcial}: el declive de la industria de la confección en Miami se ralentizó, y las industrias de alta cualificación cayeron más rápido que en ciudades comparables. Sin embargo, los salarios de los trabajadores poco cualificados no disminuyeron, sugiriendo que la tecnología y la flexibilidad del mercado laboral compensaron el choque de oferta.

\textbf{Remesas (Caso 2.20):} Los migrantes envían una parte sustancial de sus ingresos a sus países de origen ($\approx 613$ mil millones de dólares globalmente). Para los países en desarrollo, las remesas son a menudo una fuente de ingresos mayor que la ayuda extranjera oficial. Sin embargo, la migración puede no compensar la \textbf{pérdida de mano de obra} de sus países de origen (fuga de cerebros). El Profesor Bhagwati propuso un ''impuesto sobre la fuga de cerebros'' a los emigrantes educados para compensar parcialmente a los países emisores.

\textbf{Economía Política y Refugiados (Caso 2.16/2.17):} La oposición a la inmigración se basa en el efecto redistributivo (los dueños del capital ganan, los trabajadores nativos no calificados pierden). Visas especiales como la \textbf{H-2A} (trabajadores agrícolas temporales) benefician a los dueños de la tierra, mientras que la \textbf{H-1B} busca cubrir la escasez de profesionales altamente cualificados (STEM). La crisis de refugiados en la frontera de EE. UU. (Triángulo Norte) y la entrada masiva a Europa (2015, Siria, Afganistán) son fenómenos de asilo (perseguidos) distintos a la migración económica, aunque desbordan los sistemas de asilo.

\subsection{Los flujos internacionales de capital}
Los flujos internacionales de capital se distinguen del movimiento de bienes de equipo por ser esencialmente \textbf{transacciones financieras}. Estos flujos representan un \textbf{comercio intertemporal}, donde se intercambia consumo presente ($DP$) por consumo futuro ($DF$).

El precio relativo del consumo futuro en términos de consumo presente es $\frac{1}{1+r}$, donde $r$ es el tipo de interés o rentabilidad mundial.
\begin{itemize}
    \item Un país que \textbf{ahorra} (cambia $DP$ por $DF$) presta capital al mundo, exportando consumo presente.
    \item Un país que se \textbf{endeuda} (cambia $DF$ por $DP$) pide crédito, importando consumo presente.
\end{itemize}
El patrón de flujos se determina por la \textbf{Frontera de Posibilidades de Producción (FPP) intertemporal} y las preferencias de consumo intertemporal. El equilibrio general de los flujos intertemporales se alcanza cuando las exportaciones e importaciones de consumo presente y futuro se igualan a nivel mundial.

\subsection{La inversión extranjera y la empresa multinacional}
La \textbf{Inversión Extranjera Directa (IED)} es una variante de los flujos internacionales de capital que permite a una empresa crear o ampliar una filial en otro país (empresa multinacional - MN). Las MN operan en dos o más países y cumplen una doble misión al ahorrar (prestando capital) o al endeudarse (mediante operaciones de crédito).

\textbf{Ganancias de la IED (Caso 2.23):} La IED genera beneficios claros para el país receptor. Los estudios indican que las empresas extranjeras pagan \textbf{salarios más altos} (un promedio de $+7\%$) que las empresas nacionales, incluso controlando la calidad del trabajador. Además, la IED crea empleo indirecto (cada empleo extranjero crea $\approx 0.4$ empleos adicionales en empresas locales) y facilita la \textbf{transferencia de conocimiento y tecnología} (como se vio en la industria del vino en Chile).

\textbf{Efectos en Singapur (Caso 2.22):} La IED masiva en Singapur (1970–1990) llevó a un aumento del ratio Capital-Trabajo ($K/L$). En el corto plazo, el salario real ($W$) aumentó, y la renta del capital ($R_K$) cayó (rendimientos decrecientes). Sin embargo, a largo plazo, el fuerte crecimiento salarial y la estabilidad de $R_K$ observada contradijeron la predicción simple de H-O. Esto se explica porque el efecto de la IED fue superado por un \textbf{crecimiento significativo de la Productividad Total de los Factores (PTF)}, lo cual es clave para asegurar el aumento salarial sostenido.

\textbf{El Mito del Milagro Asiático (Titular 2.2):} Paul Krugman argumentó que el espectacular crecimiento de las economías asiáticas en los años 90 podría basarse, al igual que el crecimiento soviético en los 60, en una \textbf{acumulación masiva de capital y trabajo}, y no en un aumento de la productividad o la eficiencia. El crecimiento impulsado únicamente por la acumulación enfrenta límites debido a la ley de \textbf{rendimientos decrecientes}.
