\chapter{Economía Mundial: Concepto y Método}

\section{Introducción}
La vasta extensión del conocimiento económico, impulsada por un continuo proceso de crecimiento científico, ha generado ineludiblemente la \textbf{división del trabajo de investigación}. Este fenómeno resulta en la parcelación del saber económico en múltiples \textbf{especialidades científicas}. Dichas especialidades, aunque interconectadas por principios epistemológicos comunes, a menudo adoptan metodologías diferenciadas. Dentro de este espectro de especialización, la \textbf{Economía Aplicada} emerge como un ámbito fundamental, y la \textbf{Economía Mundial} se configura como una disciplina intrínseca a ella. El objeto primordial de este estudio es contextualizar esta disciplina bajo la óptica de la \textbf{globalización} y establecer su marco conceptual y metodológico, tomando como referencia principal la teoría de la economía internacional y el análisis macroeconómico de largo plazo.

\section{La Economía Aplicada, como especialidad de la ciencia económica}

La conceptualización de la Economía Aplicada (\textit{Applied Economics}) es una empresa compleja. Si bien la distinción respecto a la economía pura se remonta a principios del siglo XIX, no es hasta la segunda mitad del siglo XX cuando esta rama adquiere su verdadera dimensión. Este auge se debe a la necesidad imperante de \textbf{asesorar a un Estado interventor} que busca la estabilidad del crecimiento y la mitigación de los efectos adversos de las crisis económicas.

Históricamente, la Economía Aplicada ha adoptado diversas acepciones:
\begin{enumerate}
    \item \textbf{Economía Práctica o Normativa (Senior y Mill):} \textbf{Senior} (1836) la definió como un conjunto de principios orientadores para las acciones económicas, cuyo objetivo prescriptivo era la maximización del bienestar social. \textbf{J. S. Mill} la concibió como una disciplina normativa que integraba los fundamentos de la economía política con elementos no económicos, procedentes de otras esferas del conocimiento, esenciales para la explicación de la realidad.
    \item \textbf{Economía Descriptiva (J. N. Keynes):} \textbf{J. N. Keynes} (1955) propuso que la economía descriptiva se centra en el estudio de normas prácticas para alcanzar fines legítimos. Distinguió entre la vertiente \textit{formal} (análisis y clasificación conceptual) y la \textit{narrativa} (estudio histórico-comparativo y cuantitativo). Más allá de la mera descripción, implica el uso de instrumental analítico para interpretar las relaciones causa-efecto del sistema económico. Esta visión posclásica otorga a la economía aplicada una finalidad netamente prescriptiva.
    \item \textbf{Visión Particularista (Neoclásicos y Lange):} Los neoclásicos como \textbf{Walras} y \textbf{Wicksell} restringieron el ámbito de la economía política aplicada al análisis sectorial de la economía real. \textbf{Lange} (1966) diferenció entre la economía descriptiva (historia, geografía, estadística económica) y la economía aplicada (industrial, agrícola, financiera), vinculando esta última a la economía pura para explicar procesos contemporáneos.
    \item \textbf{Campos Aplicados (Schumpeter y Robbins):} \textbf{Schumpeter} (1971) situó las especialidades surgidas de la división del trabajo investigador y docente, como hacienda pública o comercio exterior, dentro de los ''campos aplicados'', derivados del núcleo duro del análisis económico (teoría, estadística, historia). \textbf{Robbins} (1951) propuso una tripartición: teoría económica, historia económica y economía descriptiva.
    \item \textbf{Aportación Instrumental-Formalista (Hicks):} Esta perspectiva, actualmente en apogeo, define la economía aplicada como la disciplina científica enfocada en la \textbf{contrastación empírica} de las hipótesis formuladas por la economía pura, mediante métodos estadísticos y econométricos. \textbf{Hicks} (1950) describió el proceso de adquisición del conocimiento en cuatro etapas:
    \begin{enumerate}
        \item Teoría económica (planteamiento de interrogantes)
        \item Selección e interpretación de información (conocimiento cuantitativo)
        \item Estadística económica (corrección de datos incompletos mediante interpolación)
        \item Economía descriptiva o aplicada (síntesis de información para responder a los interrogantes)
    \end{enumerate}
\end{enumerate}

En síntesis, la Economía Aplicada es una especialidad de la ciencia económica intrínsecamente ligada a la teoría, que se distingue por su enfoque en el \textbf{estudio empírico de los hechos} que configuran el sistema económico.

\section{Economía Mundial, como disciplina de la economía aplicada}

La Economía Mundial, o \textit{World Economy}, se inscribe dentro de la estructura conceptual de la Economía Aplicada. Siguiendo la taxonomía de \textbf{Schumpeter}, la economía mundial se considera un \textbf{campo aplicado} de la ciencia económica, anclado en la \textbf{estructura económica} o economía descriptiva.

\subsection{El concepto de economía mundial}

El concepto de economía mundial (o sistema económico mundial) debe ser redefinido para reflejar la dinámica macroeconómica global y de las relaciones comerciales e internas del último medio siglo. Se define formalmente como el \textbf{conjunto de relaciones que caracterizan la estructura y el funcionamiento de una sociedad}, inmerso en un marco institucional concreto. Esto implica analizar:
\begin{enumerate}
    \item Un \textbf{colectivo de elementos, sectores e instituciones} que interactúan.
    \item La \textbf{coherencia} de dichas relaciones.
    \item Cómo estas relaciones definen la \textbf{estructura y funcionamiento} del sistema.
\end{enumerate}
En la práctica, el estudio de la economía mundial se enfoca en el largo plazo, abordando la realidad global desde la perspectiva de la \textbf{globalización}, utilizando como referencia metodológica la teoría de la economía internacional.

\subsection{Principales enfoques metodológicos de la Economía Mundial}

La metodología de la Economía Aplicada, y por extensión la de la Economía Mundial, se caracteriza por ser un compendio que integra la \textbf{lógica deductiva} del análisis teórico con el \textbf{inductivismo} derivado de la observación de regularidades cuantitativas (el economista ''aplicado'' es de carácter kantiano o sintético). La labor del economista aplicado implica razonamientos integrales que trascienden el esquema simplificado poperiano o la lógica pura de Mill.

El núcleo de consenso metodológico establece dos principios fundamentales:
\begin{itemize}
    \item Es imperativo emplear una \textbf{teoría no excesivamente abstracta}, evitando la formulación de hipótesis con un grado tal de irrealidad que dificulte la contrastación (como los conceptos de ''competencia perfecta'' o ''expectativas racionales'', que son meras metáforas de la realidad). La propuesta de \textbf{Koopmans} (1957) sobre la \textit{sucesión de modelos} --que complementa modelos generales con modelos derivados menos abstractos-- se alinea con este principio.
    \item La \textbf{utilización de las matemáticas} debe ser racional. Aunque la formalización matemática ha dotado a la economía de rigor y generalidad, su uso incoherente puede distorsionar las conclusiones teóricas. La consagración de la economía cuantitativa (Econometric Society, 1928) proporcionó el arsenal instrumental, pero su abuso es contraproducente.
\end{itemize}

\subsubsection{Fuentes precursoras}
Las raíces del pensamiento económico con una visión sistémica, precursora de la economía mundial, se encuentran en varias escuelas históricas y filosóficas:
\begin{itemize}
    \item \textbf{Visión Escolástica y Fisiológica:} Ya desde los filósofos griegos, se buscaba la organización de las relaciones sociales. Más tarde, la visión fisiológica, ejemplificada por el \textit{Tableau Économique} de Quesnay, analizaba la circulación de rentas y productos, asemejando el sistema económico a un organismo vivo.
    \item \textbf{Aportación Anatómica:} \textbf{Petty} contribuyó con el estudio cuantitativo, precursor de la estadística económica, a la comprensión de las magnitudes económicas y su relación con los diferentes órganos del sistema.
    \item \textbf{Escuela Histórica Alemana:} Esta escuela se opuso al positivismo, defendiendo el \textbf{carácter mutable} de las teorías económicas en función de las condiciones históricas de la época. Aunque utilizaban la inducción, su principal divergencia con la ortodoxia era la negación de la universalidad de los supuestos.
\end{itemize}

\subsubsection{Fuentes contemporáneas}
El desarrollo contemporáneo de los enfoques metodológicos de la Economía Mundial está marcado por corrientes que buscan explicar el funcionamiento del sistema económico de manera más integrada y dinámica.
\begin{itemize}
    \item \textbf{La Concepción Marxista:} En contraste con el individualismo metodológico de los clásicos, el marxismo se centra en las \textbf{relaciones globales de la sociedad}, entendiendo que las decisiones económicas están influenciadas por la organización social. Rechaza la tesis del orden natural y sostiene que la acción consciente puede modificar las leyes económicas, basándose en la progresiva superación dialéctica de las contradicciones.
    \item \textbf{La Nueva Escuela Alemana y los Internacionalistas:} Corrientes como la Nueva Escuela Alemana (Sombart, Eucken) continuaron el enfoque de estudiar los sistemas y estructuras económicas. La contribución de los internacionalistas, particularmente a finales del siglo XX, se centró en la \textbf{revisión del concepto de sistema económico} bajo la óptica de la globalización y la regionalización, reconociendo la incapacidad del simple esquema poperiano para capturar la complejidad de las interacciones internacionales.
\end{itemize}

\section{Contenidos de la parte de la asignatura Economía Mundial}

La asignatura de Economía Mundial está diseñada para contextualizar epistemológicamente la disciplina y cimentar la comprensión de sus fundamentos metodológicos. La lógica del programa se estructura en torno a las grandes áreas de la economía internacional, abordadas desde una perspectiva de largo plazo y globalización:
\begin{itemize}
    \item \textbf{Parte I: Comercio Internacional y Política Comercial} (Temas 2, 3, 4): Esta sección cubre los mecanismos fundamentales del comercio internacional, incluyendo el movimiento de factores productivos y la política comercial, así como el análisis de las instituciones internacionales y la integración económica regional, como la Unión Europea y el Mercosur.
    \item \textbf{Parte II: Cuestiones Monetarias y Financieras} (Temas 5, 6, 7): Esta parte se adentra en las relaciones monetarias y financieras globales, el funcionamiento del sistema monetario internacional, la integración monetaria, y el análisis del crecimiento económico en el contexto de la globalización.
\end{itemize}
El objetivo es que el estudiante conozca los fundamentos metodológicos de la economía aplicada y pueda justificar la naturaleza científica de la economía, inmerso en un proceso de especialización que refleja el crecimiento y la división del trabajo en el saber económico.
