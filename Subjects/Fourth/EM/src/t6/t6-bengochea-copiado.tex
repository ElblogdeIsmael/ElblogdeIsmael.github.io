\chapter{EL SISTEMA MONETARIO INTERNACIONAL}



\section{OBJETIVOS DEL CAPÍTULO}
\begin{enumerate}
    \item Utilizar los conceptos relacionados con tipos de cambio y paridades internacionales aprendidos en el Capítulo 7.
    \item Poner de relieve la importancia del buen funcionamiento del sistema monetario internacional para que las transacciones comerciales se desarrollen de forma ágil.
    \item Reflexionar sobre las distintas formas en que, históricamente, se ha organizado el sistema monetario internacional.
    \item Comprender los mecanismos de ajuste ante desequilibrios en las cuentas exteriores, conceptos que se deben relacionar con los estudiados en el Capítulo 7.
    \item Saber diferenciar el funcionamiento de un sistema de tipos de cambio fijos frente a uno de tipos flexibles.
    \item Estudiar la configuración del Sistema Monetario Europeo y la correspondiente a la Unión Económica y Monetaria en Europa.
\end{enumerate}

\section*{RESUMEN DE CONTENIDOS}
Las transacciones, tanto comerciales como financieras, que se realizan entre los distintos países requieren la existencia de unas normas que regulen los pagos internacionales a los que aquéllas dan lugar y que se realizan normalmente en divisas. Ésta sería la tarea encomendada al Sistema Monetario Internacional (SMI).

No obstante, a lo largo de la historia, se han instaurado diferentes formas de organizar el SMI, prevaleciendo unas u otras según se han combinado una serie de elementos. En primer lugar, la moneda internacional, que es el activo o activos que funcionan como dinero en el sistema, como el oro, el dólar, el ECU, el euro o el Derecho Especial de Giro (DEG). Un segundo elemento lo constituyen los acuerdos internacionales, que regulan el comportamiento de los agentes participantes, desde los propios sistemas bancarios nacionales (incluyendo los bancos centrales, los bancos comerciales y las instituciones bancarias internacionales) y los mercados cambiarios, hasta los mercados de capitales internacionales. En tercer lugar, el mecanismo de ajuste de la balanza de pagos. Mientras que en los sistemas de tipo de cambio flexibles el equilibrio es automático (la moneda se aprecia o deprecia para equilibrarla), en los sistemas de tipo de cambio fijo las autoridades intervendrán para situarlo en el nivel que, en teoría, mantendría equilibrada la balanza de pagos. Por último, es necesario determinar de qué manera se lleva a cabo el control del sistema, es decir, quién ostenta el liderazgo del mismo.

El sistema debe cumplir, básicamente, tres condiciones para su correcto funcionamiento: dotar de medios para realizar el ajuste con el menor coste posible; proporcionar la cantidad de reservas necesarias para dotar de liquidez a las transacciones; por último, lograr que exista confianza en el mismo.

\subsection{EL PATRÓN ORO}
Este sistema imperó durante el último cuarto del siglo XIX y hasta el comienzo de la Primera Guerra Mundial, coincidiendo con el período de dominio del imperio británico en el comercio internacional. La forma en que tradicionalmente se analiza el patrón oro fue formulada por David Hume en el siglo XVIII y recibe el nombre de modelo de los flujos de oro y los precios. Aunque esta forma de describir el sistema es relativamente simple, es el enfoque aún utilizado en la actualidad. Hume consideró un mundo en el que sólo circulaban monedas de oro y el papel de los bancos centrales era insignificante\footnote{Hay que tener en cuenta que no en todos los países funcionó el patrón oro de la misma forma. Mientras que en Inglaterra, Alemania, Francia y Estados Unidos sólo circulaba oro y, en caso de que hubiese billetes, tenían su contrapartida en oro depositada en los bancos centrales, en otros países también circulaban divisas, plata y moneda fiduciaria (para transacciones más pequeñas). De hecho, no fue inmediata la adopción del patrón oro y existieron patrones bimetálicos (oro y plata) conviviendo con el patrón oro.}. Bajo el patrón oro, cada país ligaba su moneda al oro y permitía la importación y exportación irrestricta del mismo. Por tanto, existía una paridad fija entre cada moneda y el oro, y el dinero en circulación debía condicionarse a las reservas de oro y a garantizar el mantenimiento de la paridad; los cobros y pagos internacionales se liquidaban mediante movimientos de oro y los bancos centrales estaban dispuestos a comprar y vender oro al precio establecido\footnote{No obstante, el patrón oro no hacía necesario que existiera un banco central, puesto que su actuación se reducía, básicamente, a mantener la ratio reservas de oro/dinero en circulación. Así, en Estados Unidos hubo que esperar al siglo XX para que se creara su banco central.}.

Dado que cada moneda fijaba una paridad con el oro, los tipos de cambio bilaterales con las demás monedas del sistema se obtenían indirectamente a través de estas paridades. Así, una onza de oro valía 4,252 £, mientras que en Estados Unidos su valor era 20,646 \$. Por tanto, el tipo de cambio \$/£ sería 20,646 \$ / 4,252 £, es decir, 4,856 \$/£. Sin embargo, a pesar de que se trataba de un sistema de tipos de cambio fijos, en la práctica funcionaban unas bandas determinadas por los llamados puntos del oro. Dichos puntos señalaban a partir de qué momento se producía entrada y salida de oro en el país y respondían al coste de seguro y flete de trasladar internacionalmente el oro.

Los puntos del oro funcionaban de la forma siguiente: cuando un país experimentaba un déficit prolongado de balanza de pagos, su moneda se iba depreciando al ir aumentando la oferta de su moneda para pagarlo. El mercado asumía progresivamente dicha caída en el tipo de cambio hasta que rebasase el punto en el que compensaba no cambiar una moneda por otra, sino que resultaba más interesante convertir la moneda primero en oro y luego del oro a la otra moneda para realizar los pagos, por ejemplo, de una importación. Por ello, a partir de un tipo de cambio que superase el tipo fijo más los costes de seguros y flete, los pagos se realizaban en oro. El mecanismo anterior explicaría dónde se situaba el punto de salida del oro. El razonamiento inverso sería el aplicable a la entrada de oro.

Por lo que se refiere al mecanismo de ajuste, éste era del tipo precio-flujo en especie. De esta forma, un déficit continuado, que provocaba una salida de oro, se traducía en una disminución del dinero en circulación que ocasionaba una recesión en la economía y una caída de los precios. Así se mejoraba la competitividad interna, aumentando las exportaciones y disminuyendo las importaciones, consiguiendo el reequilibrio automático de las cuentas exteriores.

A pesar de que dicho mecanismo permitió un funcionamiento estable del sistema, también es cierto que coincidió con un período de estabilidad internacional y de sincronización en el ciclo económico de los países participantes. Gran Bretaña funcionó como líder del sistema, al tiempo que existió un nivel importante de cooperación (sobre todo, entre los «países centrales» del sistema), de forma que en momentos de dificultad fluía oro hacia el banco central del país en apuros\footnote{Universidad de Granada 1000564-2601141753}.

El principal problema del patrón oro fue su sesgo deflacionista, que tenía repercusiones muy negativas sobre algunos sectores sociales, principalmente aquellos que estaban endeudados. Cuando un país sufría déficit comercial, la salida de oro permitía estabilizar el sistema pero, mientras los precios de los productos disminuían, las deudas no lo hacían en la misma medida. Cualquier deudor, pero especialmente los agricultores que pagaban una hipoteca, mientras veían disminuir sus ingresos por la venta de sus productos, estaban obligados a mantener sus pagos en la misma cuantía que antes de la deflación.

\subsection{EL PERÍODO DE ENTRE-GUERRAS}
Durante la Primera Guerra Mundial muchos países suspendieron la convertibilidad, utilizándose el oro para pagar los suministros necesarios para la guerra, por lo que se prohibió o, al menos, limitó, la exportación de oro. Estados Unidos fue el único país que no abandonó el patrón oro, ayudando a Francia y Gran Bretaña con préstamos y fijando tipos de cambio depreciados con estas monedas.

Después de la guerra hubo que estabilizar las economías y devolver los préstamos antes de restaurar la convertibilidad. Los primeros que restablecieron la convertibilidad (años 1924 y 1925) mediante reforma monetaria fueron los países que debían pagar reparaciones (Austria, Alemania, Hungría y Polonia) y que habían sufrido hiperinflaciones por el exceso de papel moneda emitido para financiar los déficit presupuestarios. El resto de países lo hizo en los meses siguientes, algunos con depreciación de su moneda, como Francia, pero otros, como Gran Bretaña y Suiza, a la paridad existente antes de la guerra. Sin embargo, la situación ya no era la misma y, en realidad, la libra estaba sobrevaluada. Para mantener la paridad, Gran Bretaña tuvo que subir los tipos de interés y sufrir enormes tasas de desempleo. La situación se hizo insostenible y, finalmente, este país abandonó el patrón oro en 1931. Estados Unidos lo haría en 1933 y hacia 1936 el sistema había desaparecido. La Gran Depresión que sufrió la economía mundial hizo que los países se centrasen más en los objetivos internos, como la lucha contra el desempleo y el crecimiento económico. Pero, además, el colapso del sistema coincidió también con el fin de la edad de oro del comercio y la libre circulación de capitales, con el retorno de políticas proteccionistas y devaluaciones competitivas de las monedas.

\subsection{EL SISTEMA DE BRETTON WOODS}
Tras el fin de la Segunda Guerra Mundial, Estados Unidos, la nueva potencia económica hegemónica, impulsó un nuevo sistema que permitiera retornar a la estabilidad conocida durante el funcionamiento del patrón-oro. El objetivo del sistema, creado en 1944 en Bretton Woods (New Hampshire), fue lograr la cooperación monetaria internacional, concediendo funciones y poderes bien definidos a la institución creada para ello, el Fondo Monetario Internacional (FMI). En definitiva, se intentaba evitar las devaluaciones competitivas y fomentar el multilateralismo comercial. La principal función del FMI era proporcionar préstamos a aquellos países que precisasen reservas en divisas. Para ser miembros de este organismo, las naciones suscribirían una cuota inicial, de la cual el 25\% se entregaba en oro y el 75\% restante en la moneda del país.

Con el sistema de Bretton Woods se buscó combinar las ventajas de los tipos de cambio fijos con las de los flexibles. Por ello se diseñó como un sistema de tipo de cambio ajustable, que proporcionara estabilidad a corto plazo pero permitiera la posibilidad de ajuste cuando la balanza de un país estuviera en un desequilibrio fundamental. No obstante, este último concepto nunca fue definido con claridad.

En lugar de vincular el valor de la moneda a una materia prima, como en el patrón oro, en Bretton Woods el ancla del sistema era el dólar, estableciéndose una paridad constante inicial respecto al oro de 35 \$ la onza, siendo los dólares mantenidos por los bancos centrales libremente convertibles en oro. Los demás países podían fijar su paridad respecto al dólar o el oro, aunque todos, a excepción de Estados Unidos, eligieron el dólar, estando obligados a intervenir en los mercados (comprando o vendiendo dólares) para mantenerla. De esta forma, las demás monedas estaban indirectamente vinculadas al oro y el dólar se convirtió en la principal moneda usada como reserva. Nuevamente, a pesar de tratarse de un sistema de tipos de cambio fijos, éstos eran ajustables, al existir unas bandas establecidas en el 1\% por encima y debajo de la paridad. Asimismo, dicha paridad podía modificarse mediante devaluación o revaluación de las monedas participantes.

En el sistema de Bretton Woods, Estados Unidos debía seguir una política monetaria independiente y anti-inflacionaria, aceptando cambiar dólares por oro a la paridad establecida, mientras que el resto de países se comprometía a mantener su moneda dentro de las bandas en torno a la paridad fijada. Durante el período 1945-1968 el sistema funcionó de forma suave, al conocer la economía mundial un período de crecimiento de la producción y del comercio.

Desde mediados de los años sesenta, sin embargo, fueron varios los factores que contribuyeron a su fracaso. En primer lugar, la existencia de unas bandas muy estrechas permitía a los especuladores tomar posiciones en contra de las monedas de países con desequilibrios importantes en su balanza de pagos, ya que siempre era posible conocer la dirección en que se iba a producir la modificación de la paridad. Si finalmente se devaluaba, los especuladores obtenían los consiguientes beneficios mientras que, en caso de no producirse la devaluación, sus pérdidas potenciales serían mínimas puesto que la banda estrecha impedía que la moneda se apreciara significativamente. En segundo lugar, las reservas del sistema no crecían lo suficientemente rápido para permitir un buen funcionamiento del mismo. Este hecho se conoció como el dilema de Triffin. Este economista observó que el oro monetario aumentaba tan sólo a razón de algo menos de un 1\% al año, obligando a los países con monedas clave (especialmente Estados Unidos) a experimentar déficit continuados de balanza de pagos para proporcionar liquidez al sistema. Al mismo tiempo, los gastos militares y los derivados de programas sociales emprendidos a mediados de los años sesenta generaron una importante expansión económica que se vio acompañada por un aumento de la inflación. El dilema consistía en que si Estados Unidos continuaba generando déficit para no disminuir la liquidez, se minaría la confianza en el sistema (dado que no habría suficiente oro para hacer frente a las emisiones de dólares); sin embargo, si Estados Unidos limitaba sus déficit, y con ello protegía al dólar de los especuladores, limitaría la liquidez internacional. La solución que se planteó fue buscar otros medios para dotar al sistema de liquidez. Así fue cómo nacieron los Derechos Especiales de Giro (DEG), un nuevo activo de reserva internacional creado en 1969 para complementar los ya existentes, convirtiéndose en la unidad de cuenta del FMI.

\subsection{LA CRISIS DE BRETTON WOODS}
Los déficit de balanza de pagos americanos (especialmente respecto a países con continuo superávit, como Alemania y Japón) durante la década de los sesenta y principios de los setenta crearon un exceso de dólares, de forma que el oro era claramente insuficiente para responder por los dólares emitidos. Al mismo tiempo, Estados Unidos no podía utilizar el tipo de cambio como instrumento para reequilibrar sus cuentas exteriores y, además, los países con superávit no aceptaron, en general, la revaluación de sus monedas. Como respuesta a esta situación, Nixon decidió en agosto de 1971 suspender el vínculo entre el dólar y el oro y eliminó la promesa de cambiar oro por dólares o convertibilidad. A partir de ese momento, se pasó de estar en un sistema llamado «cambios-oro» por uno «cambios-dólar», es decir, los dólares pasaron a ser la reserva internacional. Como consecuencia, se vivió una etapa de unos cuatro meses de fluctuaciones del dólar en el mercado de cambios, en busca de un valor de equilibrio. Finalmente se llegó al Acuerdo Smithsoniano (debido a que se firmó en el Smithsonian Institute de Washington), por el cual se fijó el precio oficial del dólar en 38 \$ por onza troy de oro, en lugar de 35; se realinearon las paridades de las monedas apreciadas, y por último, se ampliaron los márgenes de fluctuación a un 2,25\%. Sin embargo, el nuevo sistema quebró definitivamente en 1973, al desatarse presiones sobre diversas monedas que llevaron a una nueva devaluación del dólar. A partir de ese momento, se generalizó la fluctuación controlada de las monedas.

\subsection{LOS EUROMERCADOS}
Un euromercado es un mercado bancario en el que se aceptan operaciones bancarias normales pero se llevan a cabo en una moneda que no coincide con la del país en que se están realizando. Aunque Europa fue el lugar de origen de este tipo de mercados, ahora se aplica a cualquier operación que cumpla este requisito, aunque se efectúe en otro país, cobrando el prefijo «euro» el significado de «externo».

A pesar de que los euromercados nacieron en los años cincuenta, su consolidación coincidió con la crisis, durante los años setenta, del sistema de Bretton Woods. El origen de los mismos, que se sitúa en Londres, fue consecuencia de la existencia de gran cantidad de dólares fuera de Estados Unidos. Por un lado, la reconstrucción europea en el seno del Plan Marshall y, por otro, las preferencias por parte de algunos poseedores de dólares por situarlos en Londres por miedo a posibles medidas de bloqueo o confiscación. Éste fue el caso de la antigua URSS y de los países árabes tras la primera crisis del petróleo.

El abandono de Bretton Woods y las sucesivas crisis del petróleo de los años setenta favorecieron el desarrollo de los euromercados. La principal razón de todo ello estribó en la mayor flexibilidad y la ausencia de regulaciones de los mercados de dólares fuera de Estados Unidos, al no existir ninguna autoridad monetaria plenamente responsable o afectada, al menos en principio, por la actividad de estos mercados. La mayor volatilidad de los tipos de cambio, el aumento de la inflación, y en general, de la incertidumbre, así como la mayor innovación presente en los euromercados fueron factores que los impulsaron rápidamente. Durante los años ochenta y noventa su importancia y el volumen de transacciones realizadas en los mismos se multiplicó, siendo continua la aparición de nuevos productos y de agentes interesados en los mismos. Este proceso dio lugar a una institucionalización de los mercados, con la entrada de fondos de pensiones y grandes compañías de seguros, contribuyendo todo ello a la integración y globalización de los mercados financieros.

Es conveniente no confundir los euromercados con los denominados mercados off-shore, en los que se realiza intermediación financiera entre depositantes y prestamistas no residentes en el país en el que la operación está localizada. Estas operaciones se pueden realizar en la moneda del país en que se sitúan, por lo que no serían parte del euromercado.

Las euromonedas son depósitos a plazo situados en un banco fuera del país en que está denominada dicha moneda (p. ej. libras esterlinas fuera de el Reino Unido). Por último, reciben el nombre de eurobonos los bonos nominados en moneda distinta de la del país en el que dicho bono se oferta (p. ej. emitir bonos en dólares en el Reino Unido).

\subsection{EL SISTEMA MONETARIO EUROPEO}
El proceso de integración europea, comenzado en la década de los cincuenta y habiendo alcanzado un éxito considerable, especialmente en el ámbito comercial, se había beneficiado de la estabilidad cambiaria de los años sesenta. Con la quiebra de Bretton Woods, el Informe Werner estableció un plan tendente a formar una unión monetaria de manera gradual durante la década de los setenta. Sin embargo, las turbulencias de los mercados y la crisis del petróleo no hicieron posible culminar este proyecto. No obstante, el resultado de estas iniciativas fue el acuerdo cambiario conocido como Serpiente Monetaria Europea. Nacida en 1972, funcionó con escaso éxito durante los años setenta y acabó siendo un acuerdo entre el marco alemán y sus países satélites, debido a la falta de cooperación monetaria y económica en general. El acuerdo consistió en mantener un margen de fluctuación bilateral máximo entre las monedas de un 2,25\%, en lugar del 4,5\% resultante del Acuerdo Smithsoniano. Cuando se produjo la libre flotación de las monedas, los países europeos mantuvieron su compromiso, aunque en 1976, con el abandono del sistema por parte del franco francés, se puede considerar que terminó la Serpiente y se instauró una zona-marco.

A pesar del fracaso de la Serpiente, a finales de los años setenta se volvió a impulsar la creación de una zona de estabilidad cambiaria en Europa. De esta forma, impulsado por Giscard D’Estaing y Helmut Schmidt se crea el Sistema Monetario Europeo (SME) en marzo de 1979. Sus principales objetivos fueron, en primer lugar, lograr la estabilidad externa de los tipos de cambio de las monedas europeas para aislarlas de los shocks externos; en segundo lugar, la estabilidad interna de manera que se produjera la progresiva convergencia en los precios y costes, así como en las políticas económicas de los países miembros; por último, fomentar el uso privado del ecu.

El SME se organizó mediante unas reglas de intervención (la parrilla de paridades y el indicador de divergencia), una moneda de intervención (el ecu, moneda cesta que se forma como la media ponderada de doce de las monedas europeas) y, finalmente, unos mecanismos de financiación (el Fondo Europeo de Cooperación Monetaria, FECOM).

Por lo que se refiere a las reglas de intervención, fueron diseñadas para corregir el funcionamiento asimétrico de la Serpiente Europea. En concreto, la parrilla de paridades consistió en la obligación, por parte de las monedas participantes, de mantener sus tipos de cambio bilaterales dentro de unos márgenes de fluctuación (inicialmente de un 2,25\% y, tras la crisis de 1992-93, de un 15\%). Frente a la Serpiente Europea o el sistema de Bretton Woods, en los que el peso del ajuste recaía sobre la moneda depreciada, el SME se diseñó para que las dos monedas implicadas colaboraran en el ajuste. No obstante, en la práctica, el peso de la intervención recaía normalmente en los países con monedas débiles, debido a las reticencias, básicamente de Alemania, para permitir que aumentase su base monetaria y de esta forma comprometer el objetivo de control del nivel de precios. La forma en que se realizaba la intervención era comprando la moneda que se depreciaba y vendiendo la que se apreciaba, siendo la operación más costosa para el país que debía vender divisas, aunque ello le servía como mecanismo de ajuste interno. Por su parte, el indicador de divergencia se utilizó como una medida multilateral, frente a la parrilla de paridades, de carácter bilateral, de la situación de una moneda frente al resto como media.

\subsection{HACIA LA UNIÓN ECONÓMICA Y MONETARIA}
Coincidiendo con unos años de expansión económica, momentos en que ha sido tradicionalmente más sencillo avanzar en la integración europea, en 1989 se volvió a plantear la creación de una Unión Económica y Monetaria (UEM). Así se recogió en el Informe Delors y se desarrolló a lo largo de diversas cumbres europeas, plasmándose en una nueva reforma del Tratado de Roma aprobada en Maastricht a finales de 1991. En dicho tratado se optó por diseñar el proceso de formación de la unión monetaria siguiendo un enfoque gradual, recogiendo las tesis alemanas, frente a lo que se conoce como terapia de choque, que ha sido la fórmula normalmente utilizada para formar las uniones monetarias ya existentes, como la propia reunificación alemana. Por ello, se diseñaron unos criterios económicos que se deberían cumplir en un plazo determinado para poder acceder al grupo de países que formasen inicialmente la unión.

Los llamados criterios de convergencia se establecieron en términos de un grupo de variables económicas clave: la tasa de inflación y el tipo de interés no debían superar más que en un punto y medio y dos puntos, respectivamente, la media de los tres países menos inflacionistas del Sistema, el déficit público no debía ser superior al 3\% del PIB, ni el stock de deuda pública superior al 60\% del PIB (o, al menos, en proceso de reducción) y, finalmente, la moneda debía haberse mantenido en el SME sin devaluar durante los dos años previos a la fecha límite. La existencia de estos criterios embarcó a los países europeos en programas de convergencia para aproximarse a ellos, tomando medidas de carácter claramente contractivo.

Durante este proceso, la puesta en duda de la capacidad de algunos países para cumplir estos requisitos provocó la crisis sufrida por el SME durante los años 1992-1993, que obligó a ampliar las bandas de fluctuación para evitar movimientos especulativos que pusieran en peligro el proceso. No obstante, tras la decisión de auto-excluirse tomada por parte de el Reino Unido, Suecia y Dinamarca, y la imposibilidad de Grecia en aquella fecha de cumplir los criterios, los restantes países formaron la unión monetaria después de verificar que cumplían los criterios de Maastricht. El 31 de diciembre de 1998 publicaron en el Diario Oficial los tipos de cambio fijados irrevocablemente entre el euro y las monedas de los 11 Estados miembros participantes. El 1 de enero de 1999 comenzó a operar el Banco Central Europeo, iniciándose la tercera fase de la Unión Monetaria Europea. A pesar de las primeras dificultades, Grecia pudo finalmente cumplir los criterios de convergencia, formando parte de la zona euro desde enero de 2000. Finalmente, el 1 de enero de 2002 comenzó la circulación del euro y la retirada de las doce monedas nacionales.

Cabe por último destacar que los países que actualmente no se encuentran dentro de la Unión Monetaria pueden participar, como lo hace la corona danesa y lo hizo en su día el dracma griega, en el Mecanismo de Tipos de Cambio II (MTC II), heredero del SME, manteniendo una paridad fija con el euro dentro de unas bandas de fluctuación.

\subsection{EL SISTEMA MONETARIO INTERNACIONAL DESPUÉS DE BRETTON WOODS}
La solución regional adoptada por los países europeos fue una de las opciones adoptadas tras la crisis de Bretton Woods pero, por supuesto, no la única y ni siquiera la mayoritaria. Existía en los años setenta un acuerdo general sobre la bondad de los tipos de cambio flexibles, que permitían alcanzar con mayor facilidad los objetivos internos. No obstante, desde el fin del sistema las principales economías mundiales han evolucionado durante largos períodos de tiempo de forma divergente, experimentando graves perturbaciones internas y externas, así como grandes déficit fiscales y por cuenta corriente. Sin embargo, ello no ha hecho que los principales países industrializados no lleven a cabo reuniones con el fin de discutir las políticas macroeconómicas. En ocasiones las reuniones eran secretas, pero en 1985, tras un período de intensa apreciación del dólar, el Grupo de los Cinco (G-5), formado por Estados Unidos, el Reino Unido, Alemania, Japón y Francia, firmaron el Acuerdo del Plaza, anunciando que el dólar se encontraba excesivamente apreciado y que iban a emprender medidas coordinadas para corregir dicha situación. Dos años más tarde, los Acuerdos del Louvre, que incluyeron a Italia y Canadá en el ahora denominado G-7, sirvieron para anunciar que el dólar había alcanzado un valor consistente con sus variables fundamentales. A partir de ese momento, las autoridades monetarias intervendrían en los mercados sólo en contadas ocasiones, con el fin de estabilizar el sistema. Por ello, el régimen actual puede considerarse más como de flotación dirigida o flotación sucia.

Según el Informe Anual del FMI (2001), de los 181 países considerados, un 44\% de los mismos (80 países) se encontrarían en regímenes de flotación independiente (47) o flotación dirigida (33), mientras que el 56\% (101 países) se encuadrarían en sistemas de tipos de cambio más o menos fijos. Dentro de estos últimos, 25 no tendrían moneda de curso legal, 12 (zona-euro) forman una unión monetaria, 31 mantendrían tipos de cambio fijos frente a una moneda y 13 frente a una combinación de monedas, 6 se mantienen dentro de unas bandas horizontales (de ellos, la corona danesa lo hace frente al euro), 4 tienen tipos de cambios móviles y 5 se mantienen dentro de márgenes de fluctuación que se ajustan periódicamente.

Además de los regímenes convencionales de tipos de cambio fijos dentro de bandas de fluctuación más o menos amplias o los regímenes de flotación más o menos intervenida, merece la pena destacar, por su importancia relativa, ciertos ejemplos o variantes de regímenes de tipos de cambio fijos que se están utilizando ampliamente en la actualidad por parte de economías emergentes o en transición.

\begin{itemize}
    \item \textbf{Tipos de cambio móviles (crawling pegs):} este tipo de sistemas lo siguen países que vinculan su moneda con la de otra nación más estable con el fin de ganar credibilidad pero, al existir condiciones macroeconómicas muy distintas entre ambas, permiten que la paridad entre ambas vaya variando continuamente. Chile adoptó este régimen durante los primeros años de la década de los ochenta. Una variante de estos sistemas serían los denominados de bandas móviles en los que se mantienen bandas de fluctuación superiores al 1\% y el tipo de cambio se va ajustando periódicamente a tasas fijas preanunciadas. Éste sería el caso de Hungría respecto al euro. En otros, el país puede vincularse a una cesta de monedas en lugar de a una sola moneda, con el fin de reducir la volatilidad, tal y como hace en la actualidad Malta o como la República Checa en la década de los noventa.
    \item \textbf{Regímenes de caja de conversión:} este régimen se basa en el compromiso legislativo explícito de cambiar moneda nacional por una moneda extranjera especificada a un tipo de cambio fijo. Las autoridades monetarias se comprometen a emitir solamente moneda nacional respaldada por moneda extranjera, dejando poco margen de maniobra a la política monetaria nacional. Son ejemplos de este sistema el caso de Argentina (hasta 2001), Brasil (hasta 1999) y Lituania con el dólar y Estonia con el marco alemán.
\end{itemize}

Es importante observar que un gran número de países de los llamados emergentes no han adoptado regímenes «extremos» (casi fijos o flexibles\footnote{Conviene señalar que hay un grupo de países que, tras pasar por experiencias negativas dentro de regímenes intermedios, han optado por la flotación, como en el caso de Méjico o Polonia. Una alternativa atractiva, que funcionó en el caso de Chile, es adoptar regímenes de flotación con objetivos explícitos de inflación.}), sino que han optado por soluciones intermedias. En los próximos años, sus resultados económicos permitirán valorar mejor la conveniencia de la elección.

\subsection{EL DISEÑO DEL SISTEMA MONETARIO INTERNACIONAL EN EL TERCER MILENIO. EL «NO-SISTEMA»}
Las funciones que, al inicio de este tema, se decía debía desempeñar un buen sistema monetario internacional, no parece que estén presentes en la actualidad en el ámbito mundial. Se ha dejado que sean los mercados los que desempeñen parte de estas funciones, siendo necesario, para minimizar el impacto que en un mundo globalizado están teniendo las crisis financieras, que se consiga nuevamente elaborar respuestas coordinadas y reforzar las instituciones supranacionales.

La puesta en marcha de la Unión Monetaria Europea y las crisis financieras sufridas, principalmente por economías emergentes, han generado nuevamente un amplio debate sobre el futuro del sistema monetario internacional, lo que también se conoce como la nueva arquitectura financiera internacional.

De manera muy sintética, son cinco los aspectos clave de la reforma:
\begin{enumerate}
    \item La evolución del régimen cambiario de los países industriales y en desarrollo, puesto que la inestabilidad de las principales monedas (dólar, euro y yen) repercute muy negativamente en estos últimos, que tratan de decidir qué régimen cambiario puede impulsar mejor su desarrollo a largo plazo.
    \item Las medidas a adoptar para fortalecer el sistema monetario y financiero internacional, detectando los problemas «sistémicos» que han aumentado la gravedad de las recientes crisis, como la asiática.
    \item Las medidas a tomar frente a la inestabilidad de los flujos de capital, sobre todo en las economías emergentes.
    \item Evitar, dentro de lo posible, los ajustes masivos en la cuenta corriente de los países en crisis, buscando ajustes más suaves.
    \item El papel del propio FMI en el sistema monetario y financiero internacional, tanto desde el punto de vista del asesoramiento como de la financiación, aspectos muy controvertidos recientemente.
\end{enumerate}

Por otro lado, son varias las propuestas que se han formulado sobre cómo gobernar el funcionamiento del Sistema Monetario Internacional. Las tres que aquí describimos brevemente tienen en común que consideran deseable evitar oscilaciones excesivas entre los tipos de cambio de las principales monedas. No hay, sin embargo, consenso respecto a ninguna de estas propuestas.

Por un lado, la propuesta de McKinnon, formulada en los años setenta y que ha sido extendida y reformulada desde entonces, se basa en tipos de cambios fijos, que considera preferibles a los flexibles, con unas bandas del 5\% y con obligación de intervenir por parte de las autoridades monetarias. Dada la apertura de los mercados de capitales existente, dicho autor piensa que la demanda relativa de las principales monedas de uso internacional es muy volátil y que la sustitución de monedas es el elemento que crea volatilidad internacional. Esto hace que el control de la cantidad de dinero por parte de las autoridades nacionales sea inadecuado. Por ello propone, una vez determinados los tipos de cambio nominales y la tasa de crecimiento de la cantidad de dinero, que dichas autoridades intervengan con el fin de mantener fija la oferta monetaria internacional y los tipos de cambio. Esta propuesta ha sido criticada por diversos motivos, siendo el principal de ellos la duda sobre si el principal motivo de variabilidad en los tipos de cambio es la sustitución de monedas.

Por su parte, en los años ochenta, Williamson propuso la creación de zonas objetivo, que permitiesen aprovechar las ventajas tanto de los tipos de cambio fijos como de los flexibles. Para ello es necesario, en primer lugar, definir un tipo de cambio de equilibrio fundamental (FEER\footnote{FEER: Fundamental Equilibrium Exchange Rate.}), que sería compatible con un déficit o superávit por cuenta corriente igual a los flujos de capital cíclicos. Este tipo de cambio se recalcularía de forma periódica para tener en cuenta las modificaciones en sus componentes fundamentales. En segundo lugar, alrededor de este tipo de cambio se definirían unas bandas amplias (de 10\% como mínimo), que no implicarían intervención inmediata cuando se alcanzaran, pero servirían para desanimar a los especuladores. Esta propuesta se ha criticado por dos motivos fundamentales: por un lado, la dificultad práctica de calcular el tipo de cambio de equilibrio; por otro, la credibilidad de la zona objetivo: sólo será viable si los agentes creen en ella. El SME ha sido considerado un ejemplo de esta propuesta hasta la crisis sufrida en 1993.

La tercera propuesta se conoce como la «tasa de Tobin». También formulada en los años setenta, se ha venido replanteando desde entonces. La idea radica en gravar con un pequeño impuesto las transacciones de tipo de cambio, independientemente del tipo de transacción de que se trate. Con ello se intentaría desanimar los movimientos especulativos, puesto que debido a su pequeña magnitud, no sería un obstáculo para las transacciones necesarias para los intercambios comerciales o inversiones a largo plazo, pero sí para los que compran divisas con el objetivo de venderlas horas después. Tobin comenzó proponiendo que dicha tasa fuese de un 1\%, para reducirla posteriormente, considerando que debería oscilar entre el 0,25 y el 0,1\%. Esta propuesta ha sido menos debatida que las dos anteriores, siendo su mayor problema cómo lograr que todos los centros financieros la adoptaran, puesto que si esto no se consigue, los capitales se concentrarían en aquellos lugares que no la aceptaran, nuevos paraísos fiscales.