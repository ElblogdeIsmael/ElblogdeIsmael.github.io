\section*{¡Excelente comienzo!}

El texto es claro y completo, contiene información fundamental sobre la teoría monetaria clásica. No está cortado, así que podemos trabajar perfectamente con él.

Aquí tienes el análisis estructurado para asegurar ese 10 en el examen.

\subsection*{1. EXPLICACIÓN DETALLADA}

Este fragmento cubre dos grandes bloques: la definición general del \textbf{Sistema Monetario Internacional (SMI)} y el funcionamiento específico del \textbf{Patrón Oro}.

\textbf{Sobre el SMI (Conceptos generales):}

El texto define el SMI como el conjunto de reglas e instituciones que gestionan los pagos entre países. Imagínalo como el "reglamento de tráfico" del dinero mundial.

\begin{itemize}
    \item \textbf{Elementos clave:} Para que funcione, necesita una "moneda base" (como fue el oro o es el dólar), acuerdos sobre cómo comportarse (normas bancarias) y un \textbf{mecanismo de ajuste}.
    \item \textbf{Mecanismo de ajuste:} Esto es crucial. Si un país gasta más de lo que ingresa (déficit), ¿cómo se soluciona?
    \begin{itemize}
        \item \textit{Tipo de cambio flexible:} La moneda baja de precio (se deprecia) sola.
        \item \textit{Tipo de cambio fijo:} El gobierno interviene para mantener el precio.
    \end{itemize}
\end{itemize}

\textbf{Sobre el Patrón Oro (Finales s. XIX - I Guerra Mundial):}

Es un sistema de \textbf{tipos de cambio fijos}.

\begin{itemize}
    \item \textbf{La teoría (David Hume):} Se basa en el \textit{mecanismo de flujo de especie-precio}. Es un sistema automático. Si España compra mucho a Inglaterra (déficit español), España tiene que pagar en oro. El oro sale de España hacia Inglaterra.
    \item \textbf{La consecuencia:} Como en el Patrón Oro la cantidad de billetes depende del oro que tengas en la caja fuerte del banco central, si sale oro, hay menos dinero circulando.
    \item \textbf{El ajuste:} Menos dinero $\rightarrow$ la gente gasta menos $\rightarrow$ los precios bajan (deflación) $\rightarrow$ los productos españoles se vuelven baratos $\rightarrow$ España exporta más $\rightarrow$ el oro vuelve a entrar. ¡El sistema se equilibra solo!
    \item \textbf{Puntos del oro:} Aunque el cambio es fijo, había un pequeño margen de maniobra. Mover oro físico en barco cuesta dinero (seguro, transporte). Si la diferencia de valor de la moneda era muy pequeña, no merecía la pena enviar el barco con oro. Solo si la moneda caía por debajo del coste de enviar el oro (el "punto de salida"), se producía el envío físico.
    \item \textbf{El problema (Sesgo deflacionista):} Para corregir un déficit, el país tenía que sufrir una bajada de precios (deflación). Esto arruinaba a los deudores (como agricultores con hipotecas), porque sus ingresos bajaban al vender sus cosechas baratas, pero la deuda con el banco seguía siendo la misma cantidad nominal.
\end{itemize}

\subsection*{2. PUNTOS CLAVE (BULLET POINTS)}

Memoriza estos datos, son "carne de cañón" para el test:

\begin{itemize}
    \item \textbf{Objetivos del SMI:} Ajuste (corregir desequilibrios), Liquidez (reservas suficientes) y Confianza.
    \item \textbf{Cronología del Patrón Oro:} Último cuarto del siglo XIX hasta la I Guerra Mundial (aprox. 1870-1914).
    \item \textbf{Líder del sistema:} Gran Bretaña.
    \item \textbf{Teórico principal:} David Hume (siglo XVIII) con el "Modelo de los flujos de oro y los precios".
    \item \textbf{Determinación del tipo de cambio:} Indirecta. Se comparaba la paridad de cada moneda con el oro (ej. cuántos dólares vale una onza vs. cuántas libras vale una onza).
    \item \textbf{Puntos del oro:} Bandas de fluctuación del tipo de cambio determinadas por los \textbf{costes de transporte y seguro} del oro.
    \item \textbf{Ajuste automático (Precio-Flujo en especie):}
    \begin{itemize}
        \item Déficit comercial $\rightarrow$ Salida de oro.
        \item Salida de oro $\rightarrow$ Reducción de oferta monetaria $\rightarrow$ Bajada de precios (ganancia de competitividad).
    \end{itemize}
    \item \textbf{Principal fallo:} Sesgo deflacionista (perjudica a los deudores y a la actividad económica interna para salvar el equilibrio externo).
\end{itemize}

\subsection*{3. POSIBLES PREGUNTAS DE EXAMEN}

Aquí tienes tres preguntas diseñadas para poner a prueba tu comprensión, con el formato "trampa" que suelen usar:

\begin{enumerate}
    \item \textbf{Pregunta 1:} Según el modelo de David Hume sobre el Patrón Oro, ¿cuál es la consecuencia inmediata de que un país tenga un déficit continuado en su balanza de pagos?\\
    A) Una entrada masiva de oro para financiar el déficit.\\
    B) Una apreciación de su moneda local debido a la demanda de importaciones.\\
    C) Una salida de oro que provoca una disminución de la oferta monetaria y una caída de los precios internos.\\
    D) Ninguna de las anteriores es correcta.
    \item \textbf{Pregunta 2:} ¿Qué función cumplían los llamados "puntos del oro" en el sistema de tipos de cambio fijos del Patrón Oro?\\
    A) Establecían el precio máximo al que el Banco Central podía vender oro.\\
    B) Determinaban las bandas de fluctuación del tipo de cambio basadas en los costes de transporte y seguro del metal.\\
    C) Señalaban el momento exacto en el que el gobierno debía devaluar la moneda oficialmente.\\
    D) Todas las anteriores son correctas.
    \item \textbf{Pregunta 3:} Para el correcto funcionamiento de un Sistema Monetario Internacional, ¿qué condiciones básicas deben cumplirse según el texto?\\
    A) Debe existir un mecanismo de ajuste eficiente con el menor coste posible.\\
    B) Debe proporcionar suficiente liquidez para las transacciones internacionales.\\
    C) Debe generar confianza en los agentes participantes.\\
    D) Todas las anteriores son correctas.
\end{enumerate}

\textbf{Soluciones:}
\begin{itemize}
    \item \textbf{P1:} La respuesta correcta es la \textbf{C}. Es la esencia del mecanismo de ajuste de Hume: Déficit = Salida de Oro = Bajada de precios.
    \item \textbf{P2:} La respuesta correcta es la \textbf{B}. Los puntos del oro son los límites (coste de flete/seguro) que determinan cuándo es rentable mover físicamente el oro; actúan como unas "mini bandas" de fluctuación.
    \item \textbf{P3:} La respuesta correcta es la \textbf{D (Todas son correctas)}. El texto enumera explícitamente estas tres condiciones: Ajuste, Liquidez y Confianza.
\end{itemize}

\vspace{0.5cm}
¿Cómo ves este primer bloque? ¿Te ha quedado claro el concepto de los "Puntos del oro"? ¡Pásame el siguiente fragmento cuando estés listo!

\section*{¡Magnífico! Este bloque es \textbf{crucial}. Aquí es donde suelen ir a "pillarte" con las fechas y las diferencias sutiles entre sistemas. Tenemos tres fases históricas muy marcadas: el caos de entreguerras, el orden de Bretton Woods y su colapso.}

Vamos a desglosarlo para asegurar esa nota alta.

\hrulefill

\subsection*{1. EXPLICACIÓN DETALLADA}

\textbf{A. El caos de Entreguerras (1918-1939):}\\
Durante la I Guerra Mundial, los países dejaron de cambiar billetes por oro para guardar el metal para la guerra. Al acabar, intentaron volver a la normalidad, pero cometieron errores:

\begin{itemize}
    \item \textbf{El error de Gran Bretaña:} Intentó volver al Patrón Oro al \textbf{mismo precio} (paridad) que tenía antes de la guerra. Pero su economía estaba peor (inflación, deuda). Al fijar la libra tan cara (sobrevaluada), sus productos eran carísimos para el extranjero $\rightarrow$ cayeron las exportaciones $\rightarrow$ paro masivo. Tuvieron que rendirse y salir del patrón oro en 1931.
    \item \textbf{Devaluaciones competitivas:} Cuando el sistema colapsó, los países empezaron a jugar sucio. "Si yo bajo el valor de mi moneda, mis productos son más baratos y vendo más que mi vecino". Esto provocó proteccionismo y empeoró la Gran Depresión.
\end{itemize}

\textbf{B. El Sistema de Bretton Woods (1944-1971):}\\
Para evitar el caos anterior, EE. UU. (la nueva gran potencia) diseñó un sistema nuevo.

\begin{itemize}
    \item \textbf{Patrón Cambios-Oro (Gold-Exchange Standard):} Ya no es "todos contra el oro". Ahora es una pirámide:
    \begin{enumerate}
        \item \textbf{Dólar:} Es la única moneda convertible en oro (a \textbf{35\$ la onza}). EE. UU. garantiza cambiar dólares por oro a los bancos centrales.
        \item \textbf{Resto de monedas:} Fijan su precio respecto al \textbf{Dólar}.
    \end{enumerate}
    \item \textbf{Tipo de cambio AJUSTABLE:} Es un sistema de tipo fijo, pero "con truco". Si un país tenía una crisis muy grave (\textit{desequilibrio fundamental}), el FMI le daba permiso para cambiar su paridad (devaluar o revaluar).
    \item \textbf{Bandas:} Las monedas podían moverse un poquito ($\pm$1\%) arriba o abajo.
    \item \textbf{El problema de la especulación:} Como las bandas eran estrechas, los especuladores apostaban a lo seguro. Si veían una moneda débil, sabían que solo podía bajar (devaluarse). Apostaban en contra y ganaban mucho dinero con riesgo casi nulo.
\end{itemize}

\textbf{C. El Dilema de Triffin (\textit{¡Concepto Clave!}):}\\
Es la paradoja que mató a Bretton Woods. Para que el mundo creciera, hacían falta dólares (liquidez). Para que hubiera dólares, EE. UU. tenía que gastar más de lo que ingresaba (déficit).

\begin{itemize}
    \item Si EE. UU. tiene mucho déficit $\rightarrow$ hay muchos dólares por el mundo $\rightarrow$ la gente desconfía de que haya oro suficiente para respaldarlos (\textbf{Falta de Confianza}).
    \item Si EE. UU. corrige el déficit $\rightarrow$ no hay dólares para el comercio mundial (\textbf{Falta de Liquidez}).
    \item \textbf{Solución fallida:} Crearon los \textbf{DEG (Derechos Especiales de Giro)} en 1969 como una moneda "artificial" del FMI para no depender tanto del dólar, pero llegaron tarde.
\end{itemize}

\textbf{D. La Caída (1971-1973):}\\
EE. UU. gastó demasiado (Guerra de Vietnam, gasto social) y Alemania y Japón exportaban mucho. Había demasiados dólares y poco oro.

\begin{itemize}
    \item \textbf{Nixon Shock (1971):} Nixon dijo "se acabó" y rompió la convertibilidad del dólar en oro. El dólar ya no valía una cantidad fija de oro.
    \item \textbf{Acuerdo Smithsoniano:} Un intento desesperado de salvar el sistema. Devaluaron el dólar (a 38\$ la onza) y ampliaron las bandas al \textbf{2,25\%}. Fracasó en 1973.
\end{itemize}

\hrulefill

\subsection*{2. PUNTOS CLAVE (BULLET POINTS)}

\begin{itemize}
    \item \textbf{Entreguerras:} Caracterizado por la inestabilidad, el fin de la libre circulación de capitales y las \textbf{devaluaciones competitivas}.
    \item \textbf{Error británico (1925):} Volver a la paridad de preguerra provocó sobrevaluación de la libra y desempleo. Abandonó el patrón en 1931.
    \item \textbf{Bretton Woods (1944):}
    \begin{itemize}
        \item Creación del \textbf{FMI} (para dar préstamos y estabilidad).
        \item Sistema de \textbf{tipo de cambio fijo pero ajustable} (ante desequilibrios fundamentales).
        \item \textbf{Ancla:} El Dólar (convertible en oro a 35\$/onza). El resto se fija al dólar.
        \item \textbf{Bandas:} $\pm$1\%.
    \end{itemize}
    \item \textbf{Dilema de Triffin:} Conflicto entre proveer \textbf{liquidez} (necesita déficit de EE. UU.) y mantener la \textbf{confianza} (necesita superávit/estabilidad de EE. UU.).
    \item \textbf{DEG (1969):} Activo de reserva creado para complementar al dólar y al oro.
    \item \textbf{Crisis final:}
    \begin{itemize}
        \item \textbf{Agosto 1971:} Nixon suspende la convertibilidad dólar-oro.
        \item \textbf{Acuerdo Smithsoniano:} Intento de rescate. Dólar a 38\$/onza y bandas al $\pm$2,25\%.
        \item \textbf{1973:} Colapso definitivo y paso a la flotación controlada.
    \end{itemize}
\end{itemize}

\hrulefill

\subsection*{3. POSIBLES PREGUNTAS DE EXAMEN}

\textbf{Pregunta 1: En el contexto del sistema de Bretton Woods, ¿en qué consistía el conocido como "Dilema de Triffin"?}\\
A) En la imposibilidad de mantener tipos de cambio fijos y libre circulación de capitales simultáneamente.\\
B) En la contradicción de que EE. UU. debía incurrir en déficit para proveer liquidez internacional, lo cual minaba la confianza en la convertibilidad del dólar en oro.\\
C) En el conflicto entre los objetivos de inflación del FMI y las políticas de empleo de los países miembros.\\
D) Ninguna de las anteriores es correcta.

\vspace{0.2cm}

\textbf{Pregunta 2: ¿Cuál de las siguientes afirmaciones sobre el Acuerdo Smithsoniano es correcta?}\\
A) Fue el tratado que dio origen al Fondo Monetario Internacional en 1944.\\
B) Estableció la vuelta al Patrón Oro clásico tras la Segunda Guerra Mundial.\\
C) Fue un intento de salvar el sistema de Bretton Woods ampliando las bandas de fluctuación al 2,25\% y devaluando el dólar frente al oro.\\
D) Todas las anteriores son correctas.

\vspace{0.2cm}

\textbf{Pregunta 3: Durante el periodo de entreguerras, ¿cuál fue una de las principales causas de la inestabilidad económica internacional descrita en el texto?}\\
A) La adopción generalizada y exitosa del sistema de tipos de cambio flexibles desde 1918.\\
B) La política de "devaluaciones competitivas" y el retorno al proteccionismo tras el colapso del sistema.\\
C) La decisión de Gran Bretaña de devaluar su moneda inmediatamente después de la guerra para ganar competitividad.\\
D) Todas las anteriores son correctas.

\hrulefill

\subsection*{Soluciones:}

\begin{itemize}
    \item \textbf{P1:} La respuesta correcta es la \textbf{B}. Es la definición de libro del dilema: Liquidez vs. Confianza.
    \item \textbf{P2:} La respuesta correcta es la \textbf{C}. Es el "último suspiro" del sistema. Fíjate que cambia el precio del oro (\$38 en vez de \$35) y amplía la banda.
    \item \textbf{P3:} La respuesta correcta es la \textbf{B}. La A es falsa (intentaron volver al oro) y la C es falsa (Gran Bretaña hizo lo contrario, revaluó al volver a la paridad antigua, lo que la arruinó).
\end{itemize}

\vspace{0.5cm}
¡Vas muy bien! Estos temas históricos suelen ser densos, pero si entiendes la lógica de \textit{por qué} falló cada uno (Rigidez en el S.XIX $\rightarrow$ Caos en Entreguerras $\rightarrow$ Rigidez del Dólar en BW), tienes el aprobado asegurado.

¿Listo para el siguiente?

\section*{¡Perfecto! Hemos llegado a la parte moderna y a la integración europea.}

Este fragmento es denso en conceptos específicos ("Euromercado" no es lo que parece intuitivamente) y en la cronología de la UE.

Aquí tienes el análisis para dominar este bloque:

\hrulefill

\subsection*{1. EXPLICACIÓN DETALLADA}

\textbf{A. Los Euromercados (¡Cuidado con el nombre!):}\\
No pienses en "Europa". Un "Euromercado" es cualquier mercado donde se opera con una moneda \textbf{fuera de su país de origen}.

\begin{itemize}
    \item \textit{Ejemplo:} Si un banco en Londres (Reino Unido) presta dólares (EE. UU.), eso es el euromercado de dólares ("eurodólares"). Si un banco en Singapur presta yenes japoneses, eso también es un euromercado. El prefijo "euro" aquí significa "externo" u "offshore" (pero legal y regulado, no confundir con paraísos fiscales opacos).
    \item \textbf{Origen:} Nació en la Guerra Fría. La URSS tenía dólares pero le daba miedo guardarlos en bancos de EE. UU. (por si se los congelaban), así que los guardó en bancos de Londres.
    \item \textbf{Diferencia clave:}
    \begin{itemize}
        \item \textit{Euromercado:} Operas en moneda extranjera (ej. Dólares en España).
        \item \textit{Mercado doméstico:} Operas en moneda local (ej. Euros en España).
        \item \textit{Mercado Off-shore puro:} Intermedias entre dos extranjeros (ej. un banco en Panamá gestiona dinero de un alemán para prestárselo a un brasileño).
    \end{itemize}
\end{itemize}

\textbf{B. El Sistema Monetario Europeo (SME):}\\
Es el "abuelo" del Euro. Europa quería estabilidad entre sus monedas para poder comerciar sin miedo a que el precio cambiara de repente.

\begin{enumerate}
    \item \textbf{La Serpiente Monetaria (1972):} Primer intento fallido. Querían mantener las monedas europeas fluctuando muy poco entre ellas (como una serpiente moviéndose dentro de un túnel estrecho), mientras todas juntas flotaban respecto al dólar. Fracasó porque no había coordinación y Alemania era demasiado fuerte; acabó siendo una "zona marco".
    \item \textbf{El SME (1979):} El intento serio.
    \begin{itemize}
        \item \textbf{Objetivos:} Estabilidad externa (tipo de cambio) e interna (precios/inflación).
        \item \textbf{El ECU:} Una moneda "cesta" (ficticia, no existía en billetes) que era la media de todas las monedas. Servía de referencia.
        \item \textbf{Mecanismo de ajuste:} Si el Franco francés bajaba mucho respecto al Marco alemán, \textit{ambos} bancos centrales debían intervenir. En teoría era simétrico, pero en la práctica Alemania (el país fuerte) no quería imprimir marcos para ayudar a los demás porque temía la inflación. Así que el peso del ajuste caía siempre sobre el débil (Francia tenía que gastar reservas o subir tipos de interés).
    \end{itemize}
\end{enumerate}

\textbf{C. La Unión Económica y Monetaria (UEM - El Euro):}\\
Para evitar las crisis del SME, decidieron crear una moneda única.

\begin{itemize}
    \item \textbf{Tratado de Maastricht (1991):} Puso las reglas del juego. No cualquiera podía entrar al club del Euro. Había que cumplir los \textbf{Criterios de Convergencia}:
    \begin{enumerate}
        \item \textbf{Inflación:} Baja (cerca de los mejores).
        \item \textbf{Tipos de interés:} Bajos.
        \item \textbf{Déficit público:} Máximo 3\% del PIB.
        \item \textbf{Deuda pública:} Máximo 60\% del PIB (o bajando rápido).
        \item \textbf{Estabilidad cambiaria:} No haber devaluado en 2 años.
    \end{enumerate}
    \item \textbf{Cronología del Euro:}
    \begin{itemize}
        \item 1999: Nace el Euro como moneda contable (bancos/bolsa). Se fijan los tipos de cambio irrevocables.
        \item 2002: Llegan los billetes y monedas físicos a nuestros bolsillos.
    \end{itemize}
\end{itemize}

\hrulefill

\subsection*{2. PUNTOS CLAVE (BULLET POINTS)}

\begin{itemize}
    \item \textbf{Definición de Euromercado:} Mercado bancario en una moneda distinta a la del país donde se realiza la operación (ej. Dólares en Londres). Origen ligado a la URSS y los "petrodólares".
    \item \textbf{Serpiente Monetaria (1972):} Primer intento de estabilidad cambiaria europea. Fracasó y se convirtió en una "zona marco".
    \item \textbf{SME (1979):} Creado por Giscard D'Estaing y Helmut Schmidt.
    \item \textbf{ECU:} Unidad de cuenta (cesta de monedas).
    \item \textbf{Parrilla de paridades:} Bandas de fluctuación ($\pm$2,25\% general, luego $\pm$15\% tras crisis 92-93).
    \item \textbf{Asimetría:} Aunque teóricamente el ajuste era compartido, en la práctica recaía sobre la moneda débil.
    \item \textbf{Tratado de Maastricht (1991):} Establece el camino al Euro (enfoque gradual, no de choque).
    \item \textbf{Criterios de Convergencia (Maastricht):} Inflación controlada, Déficit $<$ 3\%, Deuda $<$ 60\%, Tipos de interés bajos, Estabilidad cambiaria.
    \item \textbf{Fechas Euro:}
    \begin{itemize}
        \item \textbf{1999:} Inicio de la 3ª fase (fijación de tipos, nace el BCE).
        \item \textbf{2002:} Circulación física de billetes y monedas.
    \end{itemize}
\end{itemize}

\hrulefill

\subsection*{3. POSIBLES PREGUNTAS DE EXAMEN}

\textbf{Pregunta 1: En el contexto de los mercados financieros internacionales, ¿qué se entiende por una operación en el "euromercado"?}\\
A) Cualquier operación realizada en euros entre países de la Unión Europea.\\
B) Una operación bancaria realizada en una moneda distinta a la del país donde se localiza la institución bancaria (ej. un depósito en dólares en un banco de París).\\
C) La compraventa de bonos emitidos exclusivamente por el Banco Central Europeo.\\
D) Ninguna de las anteriores es correcta.

\vspace{0.2cm}

\textbf{Pregunta 2: ¿Cuál fue una de las principales críticas al funcionamiento práctico del Sistema Monetario Europeo (SME) antes de la creación del Euro?}\\
A) La inexistencia de una unidad de cuenta común como el ECU.\\
B) La asimetría en el ajuste, donde el peso de la intervención recaía principalmente en los países con monedas débiles, a pesar de que la teoría estipulaba cooperación bilateral.\\
C) La prohibición absoluta de modificar los tipos de cambio centrales, lo que impedía cualquier tipo de ajuste ante crisis.\\
D) Todas las anteriores son correctas.

\vspace{0.2cm}

\textbf{Pregunta 3: Según los criterios de convergencia de Maastricht para acceder a la Unión Económica y Monetaria, ¿cuáles eran los límites establecidos para el déficit público y la deuda pública respectivamente?}\\
A) Déficit público no superior al 3\% del PIB y Deuda pública no superior al 60\% del PIB.\\
B) Déficit público no superior al 1\% del PIB y Deuda pública no superior al 100\% del PIB.\\
C) Déficit público 0\% (equilibrio presupuestario) y Deuda pública no superior al 50\% del PIB.\\
D) Ninguna de las anteriores, ya que Maastricht solo fijó criterios de inflación y tipos de cambio.

\hrulefill

\subsection*{Soluciones:}

\begin{itemize}
    \item \textbf{P1:} La respuesta correcta es la \textbf{B}. Recuerda: "Euro" aquí significa "externo", no la moneda euro.
    \item \textbf{P2:} La respuesta correcta es la \textbf{B}. Alemania no quería "importar inflación", así que no ayudaba tanto como la teoría decía. Los países débiles sufrían el ajuste.
    \item \textbf{P3:} La respuesta correcta es la \textbf{A}. Estos números (3\% y 60\%) son fundamentales y muy preguntables.
\end{itemize}

\vspace{0.5cm}
¡Con esto tienes cubierto todo el bloque de historia y sistemas monetarios! ¿Te queda algún fragmento más o quieres que repasemos algún concepto específico? ¡Vamos a por ese 10!


\section*{¡Vamos allá! Este es el último bloque teórico y es \textbf{denso} en conceptos modernos y propuestas de reforma. Aquí es donde los profesores suelen poner preguntas para diferenciar el notable del sobresaliente, porque incluye nombres propios (McKinnon, Williamson, Tobin) y definiciones técnicas muy precisas.}

Aquí tienes el análisis para rematar el tema:

\hrulefill

\subsection*{1. EXPLICACIÓN DETALLADA}

\textbf{A. La coordinación internacional post-Bretton Woods:}\\
Tras el colapso de 1973, el mundo pasó a tipos de cambio flexibles, pero no fue un caos total. Hubo intentos de "poner orden":

\begin{itemize}
    \item \textbf{Acuerdo del Plaza (1985):} El \textbf{G-5} (EE. UU., Reino Unido, Alemania, Japón y Francia) se reunió en el Hotel Plaza. El problema era que el \textbf{Dólar estaba demasiado fuerte} (apreciado). Acordaron intervenir para bajarlo.
    \item \textbf{Acuerdo del Louvre (1987):} Dos años después, ya con el \textbf{G-7} (se unieron Italia y Canadá), decidieron que el dólar ya había bajado suficiente y estaba en su nivel correcto. Acordaron intervenir solo para \textbf{estabilizarlo}.
    \item \textbf{Régimen actual:} Se llama \textbf{"Flotación sucia" o dirigida}. Las monedas flotan libremente (el mercado decide el precio), pero los Bancos Centrales intervienen puntualmente si la cosa se desmadra.
\end{itemize}

\textbf{B. Regímenes Cambiarios "Exóticos" (Emergentes):}\\
No todo es fijo o flexible. Hay híbridos muy usados en países en desarrollo:

\begin{itemize}
    \item \textbf{Crawling Peg (Tipos de cambio móviles):} Es un tipo de cambio fijo que \textbf{cambia poquito a poco}. Imagina que tu moneda pierde valor un 1\% cada mes de forma programada porque tienes mucha inflación. Esto da certeza a los inversores pero ajusta la realidad económica.
    \item \textbf{Caja de Conversión (Currency Board):} Es el sistema \textbf{más rígido} (usado por Argentina en los 90). Por ley, el país promete cambiar su moneda por una extranjera (ej. dólar) a un precio fijo.
    \begin{itemize}
        \item \textit{El truco:} El Banco Central \textbf{NO puede imprimir billetes} si no tiene dólares en la caja fuerte que los respalden.
        \item \textit{Consecuencia:} Ganas mucha credibilidad (se acaba la inflación), pero el gobierno pierde totalmente el control de la política monetaria (si hay crisis, no pueden "imprimir dinero" para ayudar).
    \end{itemize}
\end{itemize}

\textbf{C. Propuestas de Reforma (El "No-Sistema"):}\\
Como el sistema actual es un poco anárquico, hay tres grandes propuestas teóricas para arreglarlo:

\begin{enumerate}
    \item \textbf{Propuesta de McKinnon:} Quiere volver a tipos fijos estrictos. Dice que el problema es la \textbf{sustitución de monedas} (la gente cambia dólares por euros y crea volatilidad). Propone controlar la \textbf{oferta monetaria mundial} en conjunto.
    \item \textbf{Propuesta de Williamson (Zonas Objetivo):} Ni fijo ni flexible. Propone calcular un tipo de cambio "justo" a largo plazo (\textbf{FEER} - Tipo de Cambio de Equilibrio Fundamental) y dejar que la moneda se mueva en unas \textbf{bandas amplias} ($\pm$10\%). Si se sale de ahí, se interviene suavemente.
    \item \textbf{La Tasa Tobin:} Su idea es poner \textbf{arena en las ruedas} de la especulación. Un impuesto pequeñito (0,1\% - 0,25\%) sobre \textit{cada} cambio de moneda.
    \begin{itemize}
        \item \textit{Efecto:} A quien cambia dinero por comercio real (turismo, importación) no le afecta ese 0,1\%. Pero al especulador que compra y vende mil veces al día para ganar céntimos, ese impuesto le arruina el negocio.
        \item \textit{Problema:} Si no lo aplican TODOS los países a la vez, el dinero se va a paraísos fiscales.
    \end{itemize}
\end{enumerate}

\hrulefill

\subsection*{2. PUNTOS CLAVE (BULLET POINTS)}

\begin{itemize}
    \item \textbf{Acuerdo del Plaza (1985):} G-5. Objetivo: \textbf{Depreciar} el dólar.
    \item \textbf{Acuerdo del Louvre (1987):} G-7 (G-5 + Italia y Canadá). Objetivo: \textbf{Estabilizar} el dólar (ya estaba en equilibrio).
    \item \textbf{Régimen actual:} Predominio de la \textbf{flotación sucia} o dirigida (intervención puntual).
    \item \textbf{Crawling Peg (Tipo de cambio móvil):} Ajuste periódico y preanunciado de la paridad para compensar diferencias de inflación.
    \item \textbf{Caja de Conversión (Currency Board):}
    \begin{itemize}
        \item Respaldo total de la emisión monetaria con reservas extranjeras.
        \item Renuncia a la política monetaria nacional.
        \item Ejemplos históricos: Argentina (hasta 2001), Estonia, Lituania.
    \end{itemize}
    \item \textbf{Propuesta McKinnon:} Tipos de cambio fijos (bandas $\pm$5\%) controlando la oferta monetaria global.
    \item \textbf{Propuesta Williamson:} \textbf{Zonas Objetivo} con bandas amplias ($\pm$10\%) alrededor del \textbf{FEER} (Tipo de Cambio de Equilibrio Fundamental).
    \item \textbf{Tasa Tobin:} Impuesto sobre transacciones cambiarias para frenar la especulación a corto plazo ("echar arena en los engranajes").
\end{itemize}

\hrulefill

\subsection*{3. POSIBLES PREGUNTAS DE EXAMEN}

\textbf{Pregunta 1: ¿Cuál es la diferencia fundamental entre el Acuerdo del Plaza (1985) y el Acuerdo del Louvre (1987)?}\\
A) El Acuerdo del Plaza buscaba la apreciación del dólar, mientras que el Louvre buscaba su depreciación.\\
B) El Acuerdo del Plaza, firmado por el G-5, buscaba corregir la excesiva apreciación del dólar, mientras que el Louvre (G-7) anunció que el dólar ya había alcanzado un valor consistente y buscaba estabilizarlo.\\
C) El Acuerdo del Louvre supuso la creación del Euro, mientras que el Plaza estableció el fin del patrón oro.\\
D) Ninguna de las anteriores es correcta.

\vspace{0.2cm}

\textbf{Pregunta 2: En un régimen de Caja de Conversión (Currency Board), como el que tuvo Argentina en los 90:}\\
A) El Banco Central mantiene total libertad para emitir dinero sin respaldo para financiar el déficit público.\\
B) Existe un compromiso legislativo de cambiar la moneda nacional por una extranjera a tipo fijo, y la emisión monetaria debe estar respaldada por reservas de dicha divisa extranjera.\\
C) El tipo de cambio fluctúa libremente según la oferta y demanda del mercado sin intervención estatal.\\
D) Todas las anteriores son correctas.

\vspace{0.2cm}

\textbf{Pregunta 3: ¿En qué consiste la propuesta de reforma del Sistema Monetario Internacional conocida como "Zonas Objetivo" de Williamson?}\\
A) En establecer una moneda mundial única para eliminar los tipos de cambio.\\
B) En gravar con un impuesto las transacciones financieras para evitar la especulación (Tasa Tobin).\\
C) En definir un Tipo de Cambio de Equilibrio Fundamental (FEER) y establecer bandas de fluctuación amplias (ej. $\pm$10\%) alrededor del mismo.\\
D) Todas las anteriores son correctas.

\hrulefill

\subsection*{Soluciones:}

\begin{itemize}
    \item \textbf{P1:} La respuesta correcta es la \textbf{B}. Es vital recordar la secuencia: Plaza (bajar dólar) $\rightarrow$ Louvre (mantener dólar).
    \item \textbf{P2:} La respuesta correcta es la \textbf{B}. La clave es "compromiso legislativo" y "respaldo de reservas". Es lo opuesto a la libertad monetaria (opción A).
    \item \textbf{P3:} La respuesta correcta es la \textbf{C}. La opción B es la definición de la Tasa Tobin, no la de Williamson.
\end{itemize}

\vspace{0.5cm}
¡Hemos terminado el análisis de los textos! Con estos esquemas tienes el tema del Sistema Monetario Internacional \textbf{dominado}.

¿Quieres que repasemos algún concepto global, que hagamos un simulacro rápido de 5 preguntas mezcladas, o tienes algún otro material? ¡Estás listo para sacar nota!

\section*{Casos de Estudio: Resumen de \texttt{casosT6.pdf}}

¡Claro que sí! He analizado el documento completo \texttt{casosT6.pdf}. Este archivo es una recopilación de diapositivas que cubren varios "Casos de Estudio" fundamentales para la asignatura de Economía Internacional (probablemente del manual de Feenstra \& Taylor y Krugman).

A continuación, te explico cada uno de los casos estructurados por temas para que puedas estudiarlos fácilmente.

\subsection*{1. La Crisis de la Eurozona: Diseño vs. Realidad}

Este caso explica cómo la crisis financiera rompió las reglas originales del euro.

\begin{itemize}
    \item \textbf{El diseño original:} La Eurozona nació bajo el principio de "no rescate" (no-bailout). Se suponía que cada país era responsable exclusivo de sus propios problemas fiscales, sin reparto de cargas entre naciones.
    \item \textbf{El dilema:} Cuando estalló la crisis en la periferia (Grecia, España, etc.), la UE tuvo que elegir entre mantener las reglas (y arriesgar un colapso sistémico) o romperlas para rescatar a los países.
    \item \textbf{La solución (El Rescate):} Se eligió el rescate, no solo por solidaridad, sino para proteger a los bancos del "núcleo" (Alemania, Francia) que estaban expuestos a la deuda periférica y para evitar el pánico financiero global.
    \item \textbf{El precio político:} A cambio de los rescates, la soberanía económica se desplazó de las capitales nacionales a la "Troika" (Comisión Europea, BCE, FMI), imponiendo austeridad y reformas estructurales.
    \item \textbf{El papel de Mario Draghi:} En 2012, ante el riesgo de ruptura del euro, Draghi prometió hacer "lo que fuera necesario" (\textit{whatever it takes}). El BCE pasó de ser un árbitro técnico contra la inflación a un actor político que garantizaba la supervivencia del euro, aceptando activos de menor calidad para inyectar liquidez.
\end{itemize}

\subsection*{2. ¿Es la Eurozona un Área Monetaria Óptima (AMO)? (Caso 6.6)}

Este es uno de los casos más importantes. Compara la Eurozona con Estados Unidos usando la teoría de las Áreas Monetarias Óptimas. El diagnóstico es que \textbf{Europa NO es un área monetaria óptima}. Se basa en tres criterios:

\begin{enumerate}
    \item \textbf{Integración Comercial:}
    \begin{itemize}
        \item EE. UU.: El comercio interno es muy alto (cerca del 66\% del PIB).
        \item Eurozona: Aunque el euro impulsó el comercio, la integración es menor (alrededor del 17-18\% del PIB) y la convergencia de precios es incompleta (hay diferencias grandes en sectores como automóviles).
    \end{itemize}
    \item \textbf{Movilidad Laboral (El talón de Aquiles):}
    \begin{itemize}
        \item EE. UU.: Es muy alta; la gente se muda si no hay trabajo. Más del 40\% vive en un estado distinto al de nacimiento.
        \item Eurozona: Es bajísima debido a barreras lingüísticas y culturales. Solo el 14\% vive en otro país. Esto impide que el desempleo se ajuste moviendo trabajadores de países en crisis a países en auge.
    \end{itemize}
    \item \textbf{Federalismo Fiscal:}
    \begin{itemize}
        \item EE. UU.: Tienen un sistema de transferencias automáticas. Si un estado entra en crisis, recibe fondos federales que compensan gran parte de la caída.
        \item Eurozona: No existe un mecanismo fiscal centralizado potente para rescatar regiones automáticamente.
    \end{itemize}
\end{enumerate}

\subsection*{3. El Fin de Bretton Woods y la Gran Inflación (Caso 6.3)}

Explica por qué colapsó el sistema de tipos de cambio fijos del siglo XX.

\begin{itemize}
    \item \textbf{El problema:} Bretton Woods dependía de que EE. UU. mantuviera una inflación baja, ya que el dólar era el ancla.
    \item \textbf{La causa:} Para financiar la Guerra de Vietnam y programas sociales ("Great Society"), EE. UU. imprimió mucho dinero, aumentando su inflación.
    \item \textbf{Transmisión:} Como los demás países tenían el tipo de cambio fijo con el dólar, se vieron obligados a "importar" esa inflación comprando dólares para mantener la paridad.
    \item \textbf{El colapso (Nixon Shock):} En 1971, Nixon suspendió la convertibilidad del dólar en oro, rompiendo el sistema. En 1973, el mundo pasó a tipos de cambio flexibles.
\end{itemize}

\subsection*{4. La Recesión de "Doble Caída" y el Brexit (Caso 6.11)}

Describe la resaca de la crisis de 2008 en Europa.

\begin{itemize}
    \item \textbf{Doble Recesión:} Mientras otros se recuperaban, la Eurozona volvió a caer en recesión en 2011 debido a las políticas de austeridad prematuras.
    \item \textbf{Sin válvula de escape:} Los países del sur (como España o Grecia) tenían mucha deuda y, al estar en el euro, no podían devaluar su moneda para recuperar competitividad.
    \item \textbf{Consecuencia política:} El estancamiento económico y la imposibilidad de ajustar el tipo de cambio alimentaron el euroescepticismo, lo que culminó en el Brexit en 2016.
\end{itemize}

\subsection*{5. Economía Política del Tipo de Cambio: EE. UU. 1890s (Caso 6.2)}

Este caso histórico demuestra que elegir un sistema monetario es una decisión política, no solo técnica.

\begin{itemize}
    \item \textbf{El conflicto:} A finales del siglo XIX en EE. UU., había una lucha entre mantener el \textbf{Patrón Oro} (defendido por bancos y acreedores, generaba deflación y estabilidad) o pasar al \textbf{Patrón Plata} (defendido por agricultores y deudores, generaba inflación y aliviaba deudas).
    \item \textbf{Conclusión:} Ganó el oro, pero ilustra cómo el tipo de cambio afecta la distribución de la riqueza entre grupos sociales (acreedores vs. deudores).
\end{itemize}

\subsection*{6. El Costo de las Crisis Cambiarias (Titular 6.1)}

\begin{itemize}
    \item \textbf{Datos:} Una crisis cambiaria (depreciación brutal de la moneda) provoca una caída del PIB de entre el 3\% y el 6\% y la recuperación es muy lenta.
    \item \textbf{Mecanismo de daño:} Aumenta el costo de la deuda externa (porque ahora tu moneda vale menos), debilita los bancos y provoca fuga de capitales.
\end{itemize}

\subsection*{7. La Economía Política de las Crisis (Caso 6.1)}

¿Por qué caen los gobiernos tras una devaluación?

\begin{itemize}
    \item \textbf{Países Emergentes (ej. Argentina):} Los líderes caen por el costo económico directo (pobreza, inflación).
    \item \textbf{Países Avanzados (ej. Reino Unido):} Los líderes caen por \textbf{costo de reputación}. Incluso si la economía mejora (como en UK tras 1992), el gobierno pierde credibilidad por parecer incompetente.
\end{itemize}

\subsection*{8. El Trilema y las Crisis (Timeline)}

Se explica mediante el concepto del "Trilema" o Trinidad Imposible.

\begin{itemize}
    \item \textbf{Teoría:} Un país no puede tener al mismo tiempo: 1) Movilidad de capitales, 2) Tipo de cambio fijo, y 3) Política monetaria autónoma.
    \item \textbf{La Crisis:} Si intentas mantener el tipo fijo con capitales libres, pierdes el control de tu política monetaria. Si los inversores desconfían, atacan la moneda, agotan tus reservas y fuerzan una devaluación.
\end{itemize}

\subsection*{9. Impacto del Euro en el Comercio}

\begin{itemize}
    \item \textbf{Mito:} Al principio se pensaba que el euro triplicaría el comercio (+235\%).
    \item \textbf{Realidad:} El aumento real fue modesto, entre el 5\% y el 15\%.
    \item \textbf{Dato Clave:} Los países que \textit{no} adoptaron el euro (como Suecia o Reino Unido) también aumentaron su comercio con la Eurozona, lo que demuestra que la moneda única no es estrictamente necesaria para la integración comercial.
\end{itemize}

\vspace{0.5cm}
Espero que este resumen estructurado te sea de gran utilidad para tu examen. ¡Mucho ánimo con el estudio!