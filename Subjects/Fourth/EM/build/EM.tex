% ========================
% estilo.latex mínimo funcional
% ========================

\documentclass[12pt]{book} % report para capítulos

% ========================
% Paquetes y comandos extra
% ========================
% ===========================
% Paquetes básicos de idioma y codificación
% ===========================
\usepackage[utf8]{inputenc}   % Codificación UTF-8
\usepackage[T1]{fontenc}      % Acentos y caracteres correctos
\usepackage[spanish]{babel}   % Traducción al español (capítulos, índices, etc.)
\usepackage{csquotes}         % Citas tipográficas correctas

% ===========================
% Tipografía
% ===========================
\usepackage{lmodern}          % Fuente Latin Modern
\usepackage{microtype}        % Mejoras tipográficas (espaciado, justificación)

% ===========================
% Márgenes y geometría
% ===========================
\usepackage{geometry}         % Control de márgenes
\geometry{a4paper, top=3cm, bottom=3cm, left=3cm, right=3cm}

% ===========================
% Matemáticas
% ===========================
\usepackage{amsmath, amssymb, amsthm} % Paquetes AMS
\usepackage{mathtools}        % Extiende amsmath
\usepackage{physics}          % Notación física y matemática (derivadas, bra-ket, etc.)
\usepackage{siunitx}          % Unidades SI (e.g. \SI{3}{m/s})
% \sisetup{locale=ES}           % Configuración para español (coma decimal, etc.)
\AtBeginDocument{\RenewCommandCopy\qty\SI} % Resolve siunitx and physics conflict


% ===========================
% Gráficos, tablas y colores
% ===========================
\usepackage{graphicx}         % Insertar imágenes
\usepackage{xcolor}           % Colores personalizados
\usepackage{tikz}             % Dibujos vectoriales
\usetikzlibrary{calc,positioning,shapes,arrows} % Librerías útiles de TikZ
\usepackage{pgfplots}         % Gráficas de funciones
\pgfplotsset{compat=1.18}
\usepackage{float}            % Control de posición de figuras/tablas
\usepackage{booktabs}         % Tablas profesionales
\usepackage{multirow}         % Celdas que ocupan varias filas
\usepackage{array}            % Más control en tablas
\usepackage{colortbl}         % Tablas con colores
\usepackage{inconsolata}


% ===========================
% Listas y enumeraciones
% ===========================
\usepackage{enumitem}         % Control de listas enumeradas y viñetas

% ===========================
% Encabezados, pies y diseño
% ===========================
\usepackage{fancyhdr}         % Encabezados y pies de página
\usepackage{titlesec}         % Personalizar títulos de capítulos/secciones
\usepackage{setspace}         % Espaciado entre líneas
\usepackage{parskip}          % Control del espacio entre párrafos

% ===========================
% Referencias, hipervínculos y citas
% ===========================
\usepackage{hyperref}         % Hipervínculos en PDF
\hypersetup{
    colorlinks = true,
    linkcolor  = red!70,
    citecolor  = red!70,
    urlcolor   = red!70,
    pdfpagelayout = SinglePage, % Asegura que el contenido se ajuste a una sola página
    pdfstartview = Fit          % Ajusta el contenido al tamaño de la página
}
\usepackage{cleveref}         % Referencias inteligentes (\cref)

% ===========================
% Código fuente
% ===========================
\usepackage{listings}         % Mostrar código con estilo
\usepackage{minted}           % (mejor opción, requiere Python y pygments)

% ===========================
% Bibliografía
% ===========================
\usepackage[backend=biber,style=apa]{biblatex} % Ejemplo: estilo APA
\addbibresource{referencias.bib}              % Archivo .bib

% ===========================
% Otros útiles
% ===========================
\usepackage{pdfpages}         % Insertar PDFs externos
\usepackage{blindtext}        % Texto de prueba
\usepackage{caption}          % Personalizar pies de figura/tabla
\usepackage{subcaption}       % Subfiguras
\usepackage{tocloft} 
\usepackage{amsthm}
\usepackage{subcaption}
\usepackage{truncate} % permite truncar texto si no cabe
\usepackage{libertinus}  % reemplaza lmodern
\usepackage{booktabs}  % para \toprule, \midrule, \bottomrule
\usepackage{array}     % para definir columnas personalizadas
\usepackage{colortbl}  % colores en tablas
\usepackage{etoolbox}
\AtBeginEnvironment{tabular}{\rowcolors{2}{gray!10}{white}\renewcommand{\arraystretch}{1.2}}

% ===========================
% Opciones de fuentes sugeridas
% ===========================
% TeX Gyre Pagella (estilo Palatino)
% \usepackage{fontspec}
% \usepackage{unicode-math}
% \setmainfont{TeX Gyre Pagella}
% \setmathfont{TeX Gyre Pagella Math}

% TeX Gyre Termes (estilo Times)
% \setmainfont{TeX Gyre Termes}
% \setmathfont{TeX Gyre Termes Math}

% Libertinus (elegante y completa)
% \setmainfont{Libertinus Serif}
% \setmathfont{Libertinus Math}

% TeX Gyre Bonum (estilo Garamond)
% \setmainfont{TeX Gyre Bonum}
% \setmathfont{TeX Gyre Bonum Math}

% Latin Modern (moderno de Computer Modern)
% \setmainfont{Latin Modern Roman}
% \setmathfont{Latin Modern Math}


% \usepackage{helvet}
% \usepackage{libertine}
% \usepackage[sfdefault]{FiraSans}

\usepackage{tcolorbox} % para cajas de colores




  % si tienes paquetes personalizados
% aquí van los comandos personalizados
% Comando para incluir imágenes
\newcommand{\incluirimagen}[3][]{%
\begin{figure}[H]
    \centering
    \includegraphics[width=\linewidth,#1]{#2}
    \caption{#3}
    \label{fig:#2}
\end{figure}
}

% comando para ejercicios con fondo
\newtheoremstyle{ejerciciostyle}
  {10pt}   % Espacio arriba
  {10pt}   % Espacio abajo
  %{\itshape} % Fuente del cuerpo
  {}
  {}       % Sangría
  {\bfseries} % Fuente del encabezado
  {}      % Puntuación tras encabezado
  { }      % Espacio tras encabezado
  {\thmname{#1} \thmnumber{#2}. \thmnote{#3}}


% % comando formal para enunciado de ejercicios
% \theoremstyle{ejerciciostyle}
% \newtheorem{ejercicio}{Ejercicio}[chapter]

\theoremstyle{ejerciciostyle}
\newtheorem{ejercicio}{Ejercicio}[section]

\renewcommand{\theejercicio}{\thechapter.\arabic{section}.\arabic{ejercicio}}


% comando formal para soluciones
\theoremstyle{remark}
\newtheorem{solucion}{Solución}[ejercicio]

\renewcommand{\thesolucion}{\thechapter.\arabic{section}.\arabic{ejercicio}}

% Comando para dos imágenes en paralelo
\newcommand{\dosimagenes}[6]{%
    \begin{figure}[h!]
        \centering
        \begin{minipage}{0.48\linewidth}
            \centering
            \includegraphics[width=\linewidth]{#1}
            \caption{#2}
            \label{#5}
        \end{minipage}\hfill
        \begin{minipage}{0.48\linewidth}
            \centering
            \includegraphics[width=\linewidth]{#3}
            \caption{#4}
            \label{#6}
        \end{minipage}
    \end{figure}
}

% \dosimagenes{media/fondo.jpg}{Descripción 1}{media/fondo.jpg}{Descripción 2}{fig:descripcion1}{fig:descripcion2}

% \ref{fig:descripcion1} es la mejor
% \ref{fig:descripcion2} es la mejor

\newcommand{\portadaimg}{\VAR{portadaimg}}

% Comando para crear una nota estilo información
% \newcommand{\nota}[2]{%
% \begin{tcolorbox}[colframe=blue!75!black, colback=blue!5!white, title=\textbf{#1}]
%     #2
% \end{tcolorbox}
% }
\newtheorem{nota}{Nota}[chapter]


% Comando para poner dos códigos en paralelo
\newcommand{\doscodigos}[4]{%
  \noindent
  \begin{minipage}{0.48\linewidth}
    \lstset{language=#1}
    \lstinputlisting{#2}
  \end{minipage}\hfill
  \begin{minipage}{0.48\linewidth}
    \lstset{language=#3}
    \lstinputlisting{#4}
  \end{minipage}
}

% Comando para poner un solo código
\newcommand{\uncodigo}[2]{%
  \begin{lstlisting}[language=#1]
#2
  \end{lstlisting}
}


% % Listas de archivos (sin guiones en los nombres de macros)
% \newcommand{\listagdfilesSesion2Mallas2D}{cargatexturas.gd, envioinmediato.gd, malla2dcontexturas.gd, mallaconcoloresdevertices.gd, mallanoindentada.gd}
% \newcommand{\listagdfilesSesion2Mallas3D}{mallaindexada3d.gd, materialconcolordeplano.gd, materialconcoloresdevertices.gd, tablas.gd}

% % Macro que recorre una lista de archivos en un subdirectorio
% \newcommand{\includegdfiles}[2]{%
%   % #1 = subdirectorio
%   % #2 = nombre de la lista de archivos
%   \foreach \filename in #2 {%
%     \includecode[gdstyle]{code/#1/\filename}{\filename}
%   }%
% }



% Comando para ejercicio resuelto
\newtheoremstyle{ejercicioresueltostyle}
    {10pt}   % Espacio arriba
    {10pt}   % Espacio abajo
    {\itshape} % Fuente del cuerpo
    {}       % Sangría
    {\bfseries} % Fuente del encabezado
    {}      % Puntuación tras encabezado
    { }      % Espacio tras encabezado
    {\thmname{#1} \thmnumber{#2}. \thmnote{#3}}

\theoremstyle{ejercicioresueltostyle}
\newtheorem{ejercicioresuelto}{Ejercicio Resuelto}[section]

\renewcommand{\theejercicioresuelto}{\thechapter.\arabic{section}.\arabic{ejercicioresuelto}}


%======================================================================== 
% PRACTICAS
%========================================================================

% Comando para definir un tema
\newcommand{\tema}[1]{%
  \section{#1}
  \addcontentsline{toc}{section}{#1}
}
\usepackage{tikz}
\usepackage{graphicx} % necesario para \resizebox
\usepackage{etoolbox}

% ======== NODOS ========
\newcommand{\nodo}[4][]{\node[state, #1] (#2) at (#3) {$#4$};}
% Uso: \nodo[initial,accepting]{q0}{0,0}{q_0}

% ======== FLECHAS ========
\newcommand{\flecha}[4][]{\draw[->, #1] (#2) -- (#3) node[midway, above] {#4};}
% Uso: \flecha{q0}{q1}{0} o \flecha[bend left]{q1}{q2}{1}

\newcommand{\flechaabajo}[4][]{\draw[->, #1] (#2) -- (#3) node[midway, below, yshift=-6pt] {#4};}
% Igual que \flecha pero con etiqueta abajo
\newcommand{\flechaarriba}[4][]{\draw[->, #1] (#2) -- (#3) node[midway, above, yshift=6pt] {#4};}
% Igual que \flecha pero con etiqueta arriba
\newcommand{\flechaderecha}[4][]{\draw[->, #1] (#2) -- (#3) node[midway, right] {#4};}
% Igual que \flecha pero con etiqueta a la derecha
\newcommand{\flechaiquierda}[4][]{\draw[->, #1] (#2) -- (#3) node[midway, left] {#4};}
% Igual que \flecha pero con etiqueta a la izquierda

\newcommand{\curva}[5][]{\draw[->, bend #1] (#2) to node[midway, #5] {#4} (#3);}
% Uso: \curva[left]{q1}{q2}{1}{below}


\newcommand{\loopa}[3]{\draw[->] (#1) edge[loop above] node {#2} (#1);}
\newcommand{\loopb}[3]{\draw[->] (#1) edge[loop below] node {#2} (#1);}
\newcommand{\loopr}[3]{\draw[->] (#1) edge[loop right] node {#2} (#1);}
\newcommand{\loopl}[3]{\draw[->] (#1) edge[loop left] node {#2} (#1);}
% Uso: \loopa{q1}{0}

% ======== ESTILOS ESPECIALES ========
\tikzset{
    error/.style={state, fill=red!20, draw=red!80!black},
    final/.style={state, accepting, fill=green!15!white, draw=green!60!black}
}
% Uso: \nodo[error]{qe}{5,0}{q_e}  o \nodo[final]{qf}{7,0}{q_f}


\newcommand{\pa}{1}      % ejemplo de valor
\newcommand{\pUno}{2}
\newcommand{\pDos}{3}
  % comandos LaTeX propios
% ===========================
% Diseño general
% ===========================
\setstretch{1.15} % interlineado
\setlength{\parskip}{0.5em} % espacio entre párrafos
\setlength{\parindent}{0pt} % sin sangría

% ===========================
% Estilo de capítulos y secciones (titlesec)
% ===========================
\titleformat{\chapter}[display]
  {\bfseries\Huge}
  {\filleft\Large\scshape Capítulo \thechapter}
  {1ex}
  {\titlerule[1pt]\vspace{1ex}\filright}
  [\vspace{1ex}\titlerule]

\titlespacing*{\chapter}{0pt}{0pt}{2em}

\titleformat{\section}
  {\Large\bfseries}
  {\thesection}{1em}{}

\titleformat{\subsection}
  {\large\bfseries}
  {\thesubsection}{1em}{}

\titleformat{\subsubsection}
  {\normalsize\bfseries\itshape}
  {\thesubsubsection}{1em}{}

% ===========================
% Encabezados y pies de página (fancyhdr)
% ===========================
\pagestyle{fancy}
\fancyhf{} % limpia
\fancyhead[L]{\small\scshape\nouppercase{\leftmark}} % sección/capítulo en mayúsculas pequeñas
\fancyhead[R]{\small\thepage}                        % número de página
%\fancyfoot[C]{\scriptsize\itshape Apuntes de la carrera} % texto fijo abajo en cursiva
% Encabezados y pies de página personalizados
% \fancyfoot[L]{\scriptsize\itshape Nombre de la asignatura} % pie de página izquierdo en cursiva
\fancyfoot[R]{\normalsize Ismael Sallami Moreno}        % pie de página derecho con el nombre del autor

% Línea bajo el encabezado
\renewcommand{\headrulewidth}{0.5pt} % línea más gruesa en el encabezado
% Línea en el pie
\renewcommand{\footrulewidth}{0.4pt} % línea fina en el pie
\renewcommand{\sectionmark}[1]{%
  \markboth{\thesection\quad #1}{}%
}

% ===========================
% Numeración de elementos
% ===========================
\numberwithin{equation}{chapter} % ecuaciones numeradas por capítulo
\numberwithin{figure}{chapter}   % figuras numeradas por capítulo
\numberwithin{table}{chapter}    % tablas numeradas por capítulo

% ===========================
% Listas y enumeraciones
% ===========================
\setlist[itemize]{label=--, left=1.5em}
\setlist[enumerate]{label=\arabic*), left=1.5em}

% ===========================
% Estilo de citas y bibliografía
% ===========================
\DefineBibliographyStrings{spanish}{%
  references = {Bibliografía},
}

% ===========================
% Entornos personalizados
% ===========================
\newtheoremstyle{cajita} % nombre del estilo
  {1em}   % espacio arriba
  {1em}   % espacio abajo
  {}      % fuente del cuerpo
  {}      % indentación
  {\bfseries} % fuente del título
  {.}     % puntuación tras título
  {0.5em} % espacio tras título
  {\thmname{#1}\thmnumber{ #2} \thmnote{(#3)}} % formato


\theoremstyle{cajita}
\newtheorem{teorema}{Teorema}[chapter]
\newtheorem{definicion}{Definición}[chapter]
\newtheorem{ejemplo}{Ejemplo}[chapter]
\newtheorem{proposicion}{Proposición}[chapter]
\newtheorem{demostracion}{Demostración}[chapter]
\newtheorem{corolario}{Corolario}[chapter]
\newtheorem{propuesta}{Propuesta}[chapter]


\newtheoremstyle{anotacionstyle} % nombre del estilo
  {1em}   % espacio arriba
  {1em}   % espacio abajo
  {}      % fuente del cuerpo (sin cursiva)
  {}      % indentación
  {\itshape} % fuente del título (Nota en cursiva)
  {.}     % puntuación tras título
  {0.5em} % espacio tras título
  {\thmname{\itshape#1}\thmnumber{ #2} \thmnote{(#3)}} % formato (solo Nota en cursiva)

\theoremstyle{anotacionstyle}
\newtheorem{anotacion}{Nota}[chapter]

% ===========================
% Configuración de lstlisting
% ===========================

% ===============================================
% ESTILO 1: MODERNO Y MINIMALISTA
% ===============================================

% Definir colores personalizados
\definecolor{codegreen}{rgb}{0,0.6,0}
\definecolor{codegray}{rgb}{0.5,0.5,0.5}
\definecolor{codepurple}{rgb}{0.58,0,0.82}
\definecolor{backcolour}{rgb}{0.95,0.95,0.92}
\definecolor{framecolor}{rgb}{0.8,0.8,0.8}

\lstset{
  backgroundcolor=\color{backcolour},   
  commentstyle=\color{codegreen},
  keywordstyle=\color{magenta},
  numberstyle=\tiny\color{codegray},
  stringstyle=\color{codepurple},
  basicstyle=\ttfamily\footnotesize,
  breakatwhitespace=false,         
  breaklines=true,                 
  captionpos=b,                    
  keepspaces=true,                 
  numbers=left,                    
  numbersep=5pt,                  
  showspaces=false,                
  showstringspaces=false,
  showtabs=false,                  
  tabsize=2,
  frame=shadowbox,
  frameround=tttt,
  rulecolor=\color{framecolor},
  rulesepcolor=\color{framecolor},
  xleftmargin=20pt,
  xrightmargin=20pt,
  aboveskip=20pt,
  belowskip=20pt,
  inputencoding=utf8,
  extendedchars=true,
  literate=
    {←}{{$\leftarrow$}}1
    {→}{{$\rightarrow$}}1
    {↑}{{$\uparrow$}}1
    {↓}{{$\downarrow$}}1
    {↔}{{$\leftrightarrow$}}1
    {⇒}{{$\Rightarrow$}}1
    {⇐}{{$\Leftarrow$}}1
    {⇔}{{$\Leftrightarrow$}}1
    {α}{{$\alpha$}}1
    {β}{{$\beta$}}1
    {γ}{{$\gamma$}}1
    {δ}{{$\delta$}}1
    {ε}{{$\epsilon$}}1
    {θ}{{$\theta$}}1
    {λ}{{$\lambda$}}1
    {μ}{{$\mu$}}1
    {π}{{$\pi$}}1
    {σ}{{$\sigma$}}1
    {φ}{{$\phi$}}1
    {ψ}{{$\psi$}}1
    {ω}{{$\omega$}}1
    {Δ}{{$\Delta$}}1
    {Θ}{{$\Theta$}}1
    {Λ}{{$\Lambda$}}1
    {Π}{{$\Pi$}}1
    {Σ}{{$\Sigma$}}1
    {Φ}{{$\Phi$}}1
    {Ψ}{{$\Psi$}}1
    {Ω}{{$\Omega$}}1
    {á}{{\'a}}1
    {é}{{\'e}}1
    {í}{{\'i}}1
    {ó}{{\'o}}1
    {ú}{{\'u}}1
    {Á}{{\'A}}1
    {É}{{\'E}}1
    {Í}{{\'I}}1
    {Ó}{{\'O}}1
    {Ú}{{\'U}}1
    {ä}{{\"a}}1
    {ë}{{\"e}}1
    {ï}{{\"i}}1
    {ö}{{\"o}}1
    {ü}{{\"u}}1
    {Ä}{{\"A}}1
    {Ë}{{\"E}}1
    {Ï}{{\"I}}1
    {Ö}{{\"O}}1
    {Ü}{{\"U}}1
    {ñ}{{\~n}}1
    {Ñ}{{\~N}}1
    {ç}{{\c{c}}}1
    {Ç}{{\c{C}}}1
    {¿}{{?`}}1
    {¡}{{!`}}1
    {à}{{\`a}}1
    {è}{{\`e}}1
    {ì}{{\`i}}1
    {ò}{{\`o}}1
    {ù}{{\`u}}1
    {À}{{\`A}}1
    {È}{{\`E}}1
    {Ì}{{\`I}}1
    {Ò}{{\`O}}1
    {Ù}{{\`U}}1
    {-}{{-}}1
    {=}{{=\allowbreak}}1  % <--- ESTA LÍNEA ES EL TRUCO PARA CORTAR LOS '===='
    % {#}{{\#}}1 
}


% ===============================================
% ESTILO 2: ELEGANTE CON BORDES REDONDEADOS
% ===============================================

% Colores para estilo elegante
\definecolor{lightblue}{rgb}{0.93,0.95,1}
\definecolor{darkblue}{rgb}{0.1,0.2,0.5}
\definecolor{mediumblue}{rgb}{0.2,0.4,0.8}
\definecolor{darkgreen}{rgb}{0,0.5,0}
\definecolor{darkred}{rgb}{0.6,0,0}

\lstdefinestyle{elegant}{
    backgroundcolor=\color{lightblue},
    commentstyle=\color{darkgreen}\itshape,
    keywordstyle=\color{darkblue}\bfseries,
    numberstyle=\tiny\color{gray},
    stringstyle=\color{darkred},
    basicstyle=\ttfamily\small,
    breakatwhitespace=false,
    breaklines=true,
    captionpos=t,
    keepspaces=true,
    numbers=left,
    numbersep=8pt,
    showspaces=false,
    showstringspaces=false,
    showtabs=false,
    tabsize=4,
    frame=single,
    frameround=tttt,
    framesep=10pt,
    xleftmargin=15pt,
    xrightmargin=15pt,
    aboveskip=15pt,
    belowskip=15pt,
    columns=flexible
}

% ===============================================
% ESTILO 3: PROFESIONAL CORPORATIVO
% ===============================================

% Colores corporativos
\definecolor{corporatebg}{rgb}{0.98,0.98,0.98}
\definecolor{corporateblue}{rgb}{0.07,0.29,0.49}
\definecolor{corporategray}{rgb}{0.4,0.4,0.4}
\definecolor{corporategreen}{rgb}{0.13,0.55,0.13}
\definecolor{corporatered}{rgb}{0.8,0.2,0.2}

\lstdefinestyle{corporate}{
    backgroundcolor=\color{corporatebg},
    commentstyle=\color{corporategreen}\slshape,
    keywordstyle=\color{corporateblue}\bfseries,
    numberstyle=\scriptsize\color{corporategray},
    stringstyle=\color{corporatered},
    basicstyle=\ttfamily\footnotesize,
    breakatwhitespace=false,
    breaklines=true,
    captionpos=b,
    keepspaces=true,
    numbers=left,
    numbersep=12pt,
    showspaces=false,
    showstringspaces=false,
    showtabs=false,
    tabsize=3,
    frame=leftline,
    framerule=3pt,
    rulecolor=\color{corporateblue},
    xleftmargin=25pt,
    aboveskip=20pt,
    belowskip=20pt,
    lineskip=1pt
}

% ===============================================
% ESTILO 4: MODERNO CON SOMBRAS
% ===============================================

% Colores modernos
\definecolor{modernbg}{rgb}{0.97,0.97,0.97}
\definecolor{moderngray}{rgb}{0.3,0.3,0.3}
\definecolor{modernpurple}{rgb}{0.5,0.2,0.8}
\definecolor{modernteal}{rgb}{0,0.5,0.5}
\definecolor{modernorange}{rgb}{0.8,0.4,0}

\lstdefinestyle{modern}{
    backgroundcolor=\color{modernbg},
    commentstyle=\color{modernteal}\itshape,
    keywordstyle=\color{modernpurple}\bfseries,
    numberstyle=\tiny\color{moderngray},
    stringstyle=\color{modernorange},
    basicstyle=\ttfamily\small,
    breakatwhitespace=false,
    breaklines=true,
    captionpos=t,
    keepspaces=true,
    numbers=left,
    numbersep=10pt,
    showspaces=false,
    showstringspaces=false,
    showtabs=false,
    tabsize=4,
    frame=tb,
    framerule=2pt,
    rulecolor=\color{modernpurple},
    xleftmargin=20pt,
    xrightmargin=20pt,
    aboveskip=25pt,
    belowskip=25pt
}

% ===============================================
% CONFIGURACIÓN PARA DIFERENTES LENGUAJES
% ===============================================

% Python
\lstdefinestyle{python}{
    language=Python,
    style=elegant,
    morekeywords={True,False,None,self,cls,def,class,import,from,as,with,yield,async,await},
    morecomment=[l]{\#},
    morestring=[b]',
    morestring=[b]"
}

% Java
\lstdefinestyle{java}{
    language=Java,
    style=corporate,
    morekeywords={var,record,sealed,permits,non-sealed}
}

% C++
\lstdefinestyle{cpp}{
    language=C++,
    style=modern,
    morekeywords={constexpr,nullptr,auto,decltype,override,final}
}

% JavaScript
\lstdefinestyle{javascript}{
    language=Java,
    style=elegant,
    morekeywords={let,const,var,function,class,extends,import,export,default,async,await,yield},
    morecomment=[l]{//},
    morecomment=[s]{/*}{*/},
    morestring=[b]',
    morestring=[b]",
    morestring=[b]`
}

% ===============================================
% EJEMPLOS DE USO
% ===============================================

% Para usar el estilo por defecto:
% \begin{lstlisting}
% código aquí
% \end{lstlisting}

% Para usar un estilo específico:
% \begin{lstlisting}[style=elegant]
% código aquí
% \end{lstlisting}

% Para incluir un archivo con estilo específico:
% \lstinputlisting[style=python]{archivo.py}

% Para código inline:
% \lstinline[style=modern]{código inline}

% ===============================================
% CONFIGURACIÓN ADICIONAL PARA TÍTULOS Y CARACTERES
% ===============================================

% Personalizar el formato de los títulos de los listados
\renewcommand\lstlistingname{Código}
\renewcommand\lstlistlistingname{Lista de Códigos}

% Configurar el formato del título con soporte para tildes
\lstset{
    %title=\lstname,
    captionpos=t,
    abovecaptionskip=10pt,
    belowcaptionskip=5pt,
    % Configuración global para caracteres especiales
    inputencoding=utf8,
    extendedchars=true
}

% ===============================================
% COMANDOS PERSONALIZADOS ÚTILES
% ===============================================

% Comando para código inline con soporte automático de tildes
\newcommand{\codeinline}[2][modern]{\lstinline[style=#1,inputencoding=utf8,extendedchars=true]{#2}}

% Comando para bloques de código con título personalizado
\newcommand{\codeblock}[3][elegant]{%
    \begin{lstlisting}[style=#1,caption={#2},label={lst:#2},inputencoding=utf8,extendedchars=true]
    #3
    \end{lstlisting}
}

% Comando para incluir archivos con configuración automática
\newcommand{\includecode}[3][python]{%
    \lstinputlisting[style=#1,caption={#3},label={lst:#3},inputencoding=utf8,extendedchars=true]{#2}
}

% ===============================================
% CONFIGURACIONES ESPECIALES PARA IDIOMAS
% ===============================================

% Configuración específica para código en español
\lstdefinestyle{español}{
    style=elegant,
    inputencoding=utf8,
    extendedchars=true,
    % Palabras clave en español para pseudocódigo
    morekeywords={función,procedimiento,inicio,fin,si,entonces,sino,mientras,para,hasta,hacer,repetir,caso,segun,verdadero,falso,entero,real,caracter,cadena,booleano,leer,escribir,imprimir}
}

% Configuración para comentarios multilíngües
\lstset{
    morecomment=[l]{//\ },
    morecomment=[l]{\#\ },
    morecomment=[s]{/*}{*/},
    morecomment=[s]{}
}

% ===============================================
% CONFIGURACIÓN PARA DIFERENTES LENGUAJES
% ===============================================

% Python
\lstdefinestyle{style1}{
    language=Python,
    style=elegant,
    morekeywords={True,False,None,self,cls,def,class,import,from,as,with,yield,async,await},
    morecomment=[l]{\#},
    morestring=[b]',
    morestring=[b]",
    % Soporte para caracteres especiales
    inputencoding=utf8,
    extendedchars=true
}

% Java
\lstdefinestyle{style2}{
    language=Java,
    style=corporate,
    morekeywords={var,record,sealed,permits,non-sealed},
    % Soporte para caracteres especiales
    inputencoding=utf8,
    extendedchars=true
}

% C++
\lstdefinestyle{style3}{
    language=C++,
    style=modern,
    morekeywords={constexpr,nullptr,auto,decltype,override,final},
    % Soporte para caracteres especiales
    inputencoding=utf8,
    extendedchars=true
}

\lstdefinelanguage{GDScript}{
  keywords={func, var, extends, class_name, if, else, for, while, return, match, in, and, or, not, break, continue, pass},
  sensitive=true,
  morecomment=[l]{\#},
  morestring=[b]",
  morestring=[b]',
}

\lstdefinestyle{gdstyle}{
  language=GDScript,
  basicstyle=\ttfamily\small,
  keywordstyle=\color{blue}\bfseries,
  commentstyle=\color{gray},
  stringstyle=\color{red!60!black},
  numbers=left,
  numberstyle=\tiny\color{gray},
  breaklines=true,
  frame=single,
  tabsize=2,
}


% ===========================
% Estilo global de tablas
% ===========================

\usepackage{booktabs}   % reglas profesionales
\usepackage{colortbl}   % color en filas
\usepackage{xcolor}     % colores
\usepackage{float}      % [H]

% Color de filas alternadas
% \rowcolors{2}{gray!10}{white}

% % Espacio vertical entre filas
% \renewcommand{\arraystretch}{1.2}

% % Cambiar el tamaño de columna por defecto
% \setlength{\tabcolsep}{8pt}

% % Redefinir tabla para que todas las tablas tengan el estilo
% \let\oldtabular\tabular
% \let\endoldtabular\endtabular
% \renewenvironment{tabular}[1]{%
%   \oldtabular{#1}%
% }{%
%   \endoldtabular
% }

% \usepackage{longtable,booktabs,xcolor}
% \rowcolors{2}{gray!10}{white}   % filas alternadas
% \renewcommand{\arraystretch}{1.2} % espacio vertical entre filas

% % Mostrar siempre el número de la tabla
% \usepackage{caption}
% \captionsetup[table]{labelformat=default, labelsep=colon, textfont=bf}


% ===========================
% Estilos para tikz y figures
% ===========================

\usepackage{caption}
\captionsetup{
    font={it},       % fuente en cursiva
    labelfont={},  % etiqueta ("Figura 1") en negrita
    textfont={it},   % texto del caption en cursiva
    justification=centering,  % centra el texto (opcional)
    font={small},    % tamaño de fuente pequeño
}

\usepackage{tikz}
\usetikzlibrary{positioning}

\tikzset{
  state/.style={
    draw,
    circle,
    minimum size=1cm,
    thick,
    fill=yellow!20
  },
  block/.style={
    rectangle,
    draw,
    fill=blue!10,
    rounded corners,
    text centered,
    minimum height=1cm,
    minimum width=2cm,
    thick
  },
  none/.style={
    draw=none,
    fill=none,
    text centered
  },
  error/.style={
    draw,
    circle,
    minimum size=1cm,
    thick,
    fill=red!30
  },
  initial text={}
}   % estilos de secciones, etc.

% ========================
% Configuración índice y listas
% ========================
\setlength{\cftbeforesecskip}{5pt}
\setlength{\headheight}{14pt}  % un poco más que 13.6pt

\renewcommand{\normalsize}{\fontsize{10}{12}\selectfont}

% Fix para listas de Pandoc
\providecommand{\tightlist}{%
  \setlength{\itemsep}{0pt}\setlength{\parskip}{0pt}}



%===============
% ESPACIOS
%===============

% --- Compactar secciones ---
\titlespacing*{\section}{0pt}{1.2ex plus 0.5ex minus 0.2ex}{0.8ex}
\titlespacing*{\subsection}{0pt}{1ex plus 0.3ex minus 0.2ex}{0.5ex}

% --- Compactar flotantes (figuras/tablas) ---
\setlength{\textfloatsep}{8pt}
\setlength{\intextsep}{6pt}
\setlength{\floatsep}{6pt}

% --- Compactar listas ---
\setlist{nosep}

% --- Espacio entre párrafos ---
\setlength{\parskip}{4pt}



%=======================
% fancy with parameters
%=======================
%\fancyfoot[L]{\scriptsize\itshape Economía Mundial}
\fancyfoot[L]{\normalsize Economía
Mundial} % pie de página izquierdo con tamaño normal

\setcounter{tocdepth}{1} % Muestra solo hasta subsecciones en el índice

% ========================
% Inicio del documento
% ========================
\begin{document}

% Cambiar puntos suspensivos en el índice
\renewcommand{\cftsecleader}{\cftdotfill{\cftdotsep}}

% Ajustar formato de secciones y subsecciones en el índice
\renewcommand{\cftsecfont}{\bfseries} % Secciones en negrita
\renewcommand{\cftsecpagefont}{\bfseries} % Números de página en negrita para secciones
\renewcommand{\cftsubsecfont}{\normalfont} % Subsecciones en formato normal
\renewcommand{\cftsubsecpagefont}{\normalfont} % Números de página en formato normal para subsecciones

% Espaciado entre entradas del índice
\setlength{\cftbeforesecskip}{8pt} % Espaciado antes de secciones
\setlength{\cftbeforesubsecskip}{4pt} % Espaciado antes de subsecciones



%% portada.tex
\begin{titlepage}
    \newgeometry{top=2cm,bottom=2cm,left=2.5cm,right=2.5cm} % márgenes personalizados
    
    % Fondo con transparencia
    \begin{tikzpicture}[remember picture,overlay]
        \node[opacity=0.15,inner sep=0pt] at (current page.center)
            {\includegraphics[width=\paperwidth,height=\paperheight]{../../img/fondoPrueba.jpg}};
    \end{tikzpicture}

    % Contenido de la portada
    \begin{center}
        \vspace*{2cm}
        
        {\Huge \bfseries\scshape Título del Libro de Apuntes \par}
        \vspace{0.5cm}
        {\Large \itshape Subtítulo o Asignatura \par}
        \vspace{0.5cm}
        {\Large \itshape \href{https://ismael-sallami.github.io}{https://ismael-sallami.github.io} \par}


        \vfill
        
        {\LARGE Autor: \textbf{Tu Nombre Completo} \par}
        \vspace{0.3cm}
        % {\Large Universidad Ejemplo \par}
        
        \vspace{1cm}
        \includegraphics[width=0.25\textwidth]{../../img/ugr.png} % opcional: logo
        \vspace{1cm}
        
        {\large \today}
    \end{center}
    
    \restoregeometry
\end{titlepage}



%==========================
% PORTADA: ENTRADA MANUAL
%==========================

% portada.tex
\begin{titlepage}
    \newgeometry{top=2cm,bottom=2cm,left=2.5cm,right=2.5cm} % márgenes personalizados
    
    % Fondo con transparencia
    \begin{tikzpicture}[remember picture,overlay]
        % \node[opacity=0.15,inner sep=0pt] at (current page.center)
        \node[inner sep=0pt] at (current page.center)
            {\includegraphics[width=\paperwidth,height=\paperheight]{../../../extraFiles/img/fondo_ade.jpg}};
    \end{tikzpicture}

    % Contenido de la portada
    \begin{center}
        \vspace*{2cm}
        
        \vspace{5cm} % Añadir más espacio antes del contenido
        {\Huge \bfseries\scshape\textcolor{white}{Economía
Mundial} \par}
        \vspace{0.5cm}
        {\Large \itshape\textcolor{white}{Temario} \par}
        \vspace{0.5cm}
        % {\small \itshape \href{https://ismael-sallami.github.io}{https://ismael-sallami.github.io} \par}
        % {\small \itshape \href{https://elblogdeismael.github.io}{https://elblogdeismael.github.io} \par}


        \vfill
        
        % {\LARGE Ismael Sallami Moreno \par}

        \begin{flushright}
            {Ismael Sallami Moreno \par}
            {\small \itshape \href{https://elblogdeismael.github.io}{Recursos Ingeniería Informática y Ade} \par}
        \end{flushright}
        \vspace{0.3cm}
        % {\Large Universidad de Granada \par}
        
        % \vspace{1cm}
        % \includegraphics[width=0.25\textwidth]{../../../extraFiles/img/ugr.png} % opcional: logo
        % \vspace{1cm}
        
        % {\large \today}
    \end{center}
    
    \restoregeometry
\end{titlepage}


%==========================
% LICENCIA
%==========================

\begin{tikzpicture}[remember picture,overlay]
\node[anchor=south west, xshift=1cm, yshift=1cm] at (current page.south west) {
\begin{minipage}{0.4\textwidth}
\begin{flushleft}
\section*{Licencia}

Este trabajo está bajo una 
\href{https://creativecommons.org/licenses/by-nc-nd/4.0/}{Licencia Creative Commons BY-NC-ND 4.0}.

\bigskip

Permisos: Se permite compartir, copiar y redistribuir el material en cualquier medio o formato.

\bigskip

Condiciones: Es necesario dar crédito adecuado, proporcionar un enlace a la licencia e indicar si se han realizado cambios. No se permite usar el material con fines comerciales ni distribuir material modificado.

\bigskip

\begin{center}
  \href{https://creativecommons.org/licenses/by-nc-nd/4.0/}{\includegraphics[width=0.35\textwidth]{../../../extraFiles/img/by-nc-nd.png}}
\end{center}
\end{flushleft}
\end{minipage}
};
\end{tikzpicture}

\thispagestyle{empty}
\clearpage

%==========================
% AUTOR
%==========================

% Página del autor
\begin{center}
    \vspace*{3cm} % Añadir más espacio en la parte superior
    {\Huge Economía
Mundial}\\[2cm] % Incrementar el espacio entre líneas
    {\Large Ismael Sallami Moreno}\\[1cm] % Incrementar el espacio entre líneas
    % \includegraphics[width=0.3\textwidth]{autor.jpg} % opcional foto
    \vfill % Añadir espacio flexible para centrar verticalmente
\end{center}
\thispagestyle{empty}

% Página en blanco sin estilo
\newpage
\thispagestyle{empty}
\mbox{}

% Otra página en blanco sin estilo
\newpage
\thispagestyle{empty}
\mbox{}


%==========================
% BIOGRAFÍA
%==========================

% Breve descripción del autor
% \chapter*{Biografía}
% \addcontentsline{toc}{chapter}{Biografía} % aparece en índice
% Aquí escribes una breve descripción sobre ti, tu formación, experiencia, etc.
% \cleardoublepage


% % ===============================
% licencia.tex
% ===============================
\begin{tikzpicture}[remember picture,overlay]
\node[anchor=south west, xshift=1cm, yshift=1cm] at (current page.south west) {
\begin{minipage}{0.4\textwidth}
\begin{flushleft}
\section*{Licencia}

Este trabajo está bajo una 
\href{https://creativecommons.org/licenses/by-nc-nd/4.0/}{Licencia Creative Commons BY-NC-ND 4.0}.

\bigskip

Permisos: Se permite compartir, copiar y redistribuir el material en cualquier medio o formato.

\bigskip

Condiciones: Es necesario dar crédito adecuado, proporcionar un enlace a la licencia e indicar si se han realizado cambios. No se permite usar el material con fines comerciales ni distribuir material modificado.

\bigskip

\begin{center}
  \href{https://creativecommons.org/licenses/by-nc-nd/4.0/}{\includegraphics[width=0.35\textwidth]{../../../extraFiles/img/by-nc-nd.png}}
\end{center}
\end{flushleft}
\end{minipage}
};
\end{tikzpicture}
  % licencia
% \thispagestyle{empty} % quitar número de página en la portada
% \clearpage

% --- Índice ---
\tableofcontents
% \listoffigures
\clearpage

%\listoftables
%\clearpage
%\thispagestyle{empty} % quitar número de página en la portada
%\clearpage
%
% Índice de código
%\renewcommand{\lstlistlistingname}{Índice de Código}
%\lstlistoflistings
%\clearpage
%
% Índice de ecuaciones
%\renewcommand{\listtheoremname}{Índice de Ecuaciones}
%\listoftheorems[ignoreall,show={equation}]
%\clearpage

% --- Contenido Markdown generado por Pandoc ---
\part{Teoria}

\hypertarget{introduccion}{%
\chapter{Introduccion}\label{introduccion}}

La asignatura de Economía Mundial se centra en el estudio de los
principales aspectos económicos que configuran el entorno global. A
través de esta materia, se analizan temas como el comercio
internacional, los mercados financieros globales, las políticas
económicas de los países y los desafíos del desarrollo sostenible. El
objetivo es proporcionar una visión integral de las interacciones
económicas entre las naciones y su impacto en el bienestar global.

\hypertarget{economuxeda-mundial-muxe9todo}{%
\chapter{Economía Mundial: Método}\label{economuxeda-mundial-muxe9todo}}

\hypertarget{referencias-de-word}{%
\section{Referencias de Word}\label{referencias-de-word}}

\begin{ejercicio}

Este es un tipo de ejercicio hecho en clase sobre como insertar referencias desde Word.

\begin{enumerate}
   \item Entrar en \href{https://www.recursoscientificos.fecyt.es}{Recursos Científicos FECYT}\footnote{https://www.recursoscientificos.fecyt.es}.
   \item Acceder a “Web of Science”.
   \item Registrarse y completar los datos necesarios.
   \item Buscar la referencia deseada y exportarla a EndNote. El resto del proceso es intuitivo.
\end{enumerate}

Recomendación de libro: Milton Friedman: La libertad de elegir

\end{ejercicio}

\hypertarget{un-nuevo-concepto-de-sistema-econuxf3mico-tribuna-de-economuxeda}{%
\section{Un nuevo concepto de sistema económico (Tribuna de
economía)}\label{un-nuevo-concepto-de-sistema-econuxf3mico-tribuna-de-economuxeda}}

Es necesario redefinir el sistema económico, ya que se considera que el
sistema es una macroestructura con capacidad de autodecisión. Esta
visión
holista\footnote{La ''visión holística'' en economía es un enfoque que se opone al análisis económico tradicional, que tiende a centrarse en variables aisladas y en la maximización de la riqueza monetaria.}
contradice definiciones previas y refleja la regionalización del siglo
XX, destacando la autonomía de América, Unión Europea y Asia en sus
relaciones comerciales.

\hypertarget{quuxe9-es-el-sistema-econuxf3mico}{%
\subsection{¿Qué es el sistema
económico?}\label{quuxe9-es-el-sistema-econuxf3mico}}

Conjunto de relaciones que caracterizan la estructura y el
funcionamiento de una sociedad, dentro de un marco constitucional
concreto. Se trata de un colectivo que estan interrelacionados y que
interaccionan entre sí. Estas relaciones definen la estructura y el
comportamiento de la sociedad.

Las relaciones del sistema económico se desarrollan en un contexto
institucional que determina su funcionamiento. Según Sampedro y Martínez
Cortiña, estas relaciones institucionales condicionan las decisiones
adoptadas, siendo jurídicamente sociales según Sombart.

\hypertarget{enfoques-metodoluxf3gicos-del-sistema-econuxf3mico}{%
\subsection{Enfoques metodológicos del sistema
económico}\label{enfoques-metodoluxf3gicos-del-sistema-econuxf3mico}}

\begin{itemize}
\item
  Fuentes precursoras: hemos de reseñar aquí: las aportaciones de los
  filósofos griegos, la visión anatómica, la propuesta fisiológica y la
  contribución de la escuela histórica alemana.
\item
  La idea del sistema económico en Platón y Aristóteles: Platón, en
  \emph{La República}, introduce el concepto de sistema como una
  estructura política que mantiene unidos a los hombres en un Estado
  mediante relaciones de interdependencia. Aristóteles, siguiendo esta
  línea, clasifica las organizaciones según su estructura y
  funcionamiento, distinguiendo entre aquellas gobernadas por un solo
  líder, varias personas, o por la mayoría.
\item
  El cuerpo económico de Petty: Define que la tesis de que el
  conocimiento de la estructura exige una dirección de la realidad
  económica, como si de un cuerpo humano se tratase. Petty es el
  precursor del moderno estructuralismo. Una de las críticas es que
  reduce la estructura económica, obviando tanto la anatomía del cuerpo
  económico, como las relaciones entre los diferentes órganos.
\item
  La visión fisiológica del sistema económico: la corriente fisiológica
  reduce el estudio del sistema económico al simple análisis
  cuantitativo de los principales agregados económicos, renunciando así
  a investigar cuál es el papel que juegan los elementos sociológicos,
  históricos, \ldots{}
\item
  La visión global de la escuela histórica alemana: La escuela histórica
  alemana critica la visión fisiológica por su excesivo reduccionismo,
  al centrarse únicamente en datos cuantitativos de las variables
  macroeconómicas. En contraste, esta escuela adopta un enfoque más
  global, integrando otras disciplinas de las ciencias sociales para
  enriquecer el análisis del sistema económico. Además, es reconocida
  por ser pionera en proponer una teoría del desarrollo moderno basada
  en etapas históricas. Sombart, uno de sus últimos representantes,
  argumenta en su obra \emph{Gewerbliche} que un sistema económico se
  compone de tres elementos fundamentales: espíritu, sustancia y forma.
\item
  Fuentes modernas: Las aportaciones modernas profundizan en el análisis
  del sistema económico desde diversas perspectivas. Entre ellas
  destacan: el análisis de la nueva escuela alemana, la concepción
  marxista del sistema, la visión alternativa del neoinstitucionalismo,
  los planteamientos del sociologismo francés, la defensa neoliberal del
  libre mercado, la visión de la escuela sueca, la revisión de Snavely y
  la contribución de los economistas españoles contemporáneos. Además,
  el debate metodológico ha estado influenciado por el falsacionismo
  popperiano, la crítica positivista de Kuhn, el refinamiento de Lakatos
  y la reacción neoinstitucional en el programa de investigación del
  sistema económico.
\item
  La nueva escuela alemana: influenciada por el marxismo, el socialismo
  de cátedra y el sociologismo alemán, propone un enfoque integrador que
  combina elementos históricos, sociales y económicos para analizar el
  sistema económico. Este enfoque busca superar las limitaciones del
  reduccionismo cuantitativo y enfatiza la importancia de las
  interacciones entre los diversos componentes del sistema.
\item
  La concepción marxista del sistema económico: Influenciada por Marx,
  quien criticó el capitalismo británico del siglo XIX, esta visión
  destaca la radical división social y la configuración de la propiedad
  de los medios de producción como base del sistema económico. Según
  Marx, el capitalismo se sustenta en la acumulación y reproducción
  ampliada del capital.
\item
  La visión alternativa del neoinstitucionalismo: En los años sesenta,
  el neoinstitucionalismo, liderado por Galbraith y Myrdal, cuestionó el
  paradigma clásico y propuso un enfoque centrado en la abundancia y la
  integración internacional. Este enfoque critica la visión
  reduccionista del sistema capitalista y enfatiza la influencia de las
  instituciones en el desarrollo económico.
\item
  Los planteamientos del sociologismo francés: Este enfoque destaca la
  necesidad de integrar el análisis económico en un marco sociológico
  más amplio, considerando las relaciones sociales, económicas e
  institucionales como componentes esenciales del sistema.
\item
  La defensa neoliberal del libre mercado: Basada en las propuestas de
  la ``Public Choice'', esta visión defiende la neutralidad del Estado y
  la regulación mínima, promoviendo el libre mercado como el mecanismo
  óptimo para el funcionamiento del sistema económico.
\item
  La visión de la escuela sueca: Reconocida por su respuesta a la
  ``Public Choice'', la escuela sueca acepta los postulados del Estado
  mínimo y el rechazo al intervencionismo. Lindbeck destaca cómo las
  instituciones y mecanismos económicos influyen en la adopción de
  decisiones relacionadas con la producción, inversión y consumo en una
  economía social.
\item
  La revisión de Snavely: Snavely propone un análisis del sistema
  económico basado en tres modelos: capitalismo, socialismo y
  corporativismo. Este enfoque busca responder preguntas fundamentales
  sobre la estructura y funcionamiento del sistema, como la distribución
  de bienes y la participación en el trabajo.
\item
  La contribución de los economistas españoles contemporáneos:
  Economistas como Sampedro y Martínez Cortiña han realizado
  aportaciones significativas al estudio del sistema económico,
  destacando la importancia de las relaciones estructurales básicas y su
  impacto en la organización social. Tamames, por su parte, propone un
  enfoque integrador que combina elementos históricos y económicos para
  analizar el sistema.
\item
  El sistema económico como disciplina de la ciencia económica: Se
  define como una especialidad científica que estudia las relaciones
  económicas en un contexto institucional. Este enfoque incluye el
  análisis de la macroestructura que forma el sistema y las tipologías
  de las organizaciones económicas.
\item
  La organización del sistema económico: El sistema económico
  internacional actual refleja una transición hacia modelos más
  globalizados, influenciados por la regionalización y el papel de las
  corporaciones supranacionales. Este panorama destaca la importancia de
  revisar las características del sistema mundial y su impacto en las
  economías capitalistas.
\item
  El programa de investigación del sistema económico: Este programa
  busca analizar las principales características del sistema mundial,
  como la globalización y la interacción entre bloques económicos.
  Además, se enfoca en los desafíos del capitalismo actual, como la
  desigualdad y la sostenibilidad.
\end{itemize}

\hypertarget{ideas-principales-y-conclusiuxf3n}{%
\subsection{Ideas principales y
Conclusión}\label{ideas-principales-y-conclusiuxf3n}}

\begin{enumerate}
\def\labelenumi{\arabic{enumi}.}
\item
  \textbf{Crítica a los sistemas económicos tradicionales}

  \begin{itemize}
  \tightlist
  \item
    \textbf{Capitalismo}: se destaca su énfasis en la competencia, la
    acumulación de capital y el beneficio privado, lo cual genera
    desigualdad, ciclos de crisis y una débil consideración del
    bienestar colectivo.
  \item
    \textbf{Socialismo/Comunismo}: aunque busca la igualdad y la
    eliminación de la explotación, ha mostrado rigideces burocráticas,
    ineficiencia en la asignación de recursos y falta de incentivos
    individuales.
  \item
    Ambos modelos se presentan como \textbf{incompletos}: uno prioriza
    la libertad individual a costa de la equidad, y el otro la equidad a
    costa de la libertad.
  \end{itemize}
\item
  \textbf{La necesidad de un nuevo paradigma}

  \begin{itemize}
  \tightlist
  \item
    El autor plantea que el siglo XXI requiere un \textbf{sistema
    económico alternativo}, capaz de responder a problemas globales
    como:

    \begin{itemize}
    \tightlist
    \item
      La pobreza persistente.
    \item
      El deterioro ambiental.
    \item
      La creciente concentración de riqueza.
    \item
      La exclusión social y laboral causada por la automatización.
    \end{itemize}
  \item
    Este sistema debe \textbf{combinar eficiencia, equidad y
    sostenibilidad}, superando las limitaciones de los modelos
    anteriores.
  \end{itemize}
\item
  \textbf{Principios del nuevo concepto de sistema económico}

  \begin{enumerate}
  \def\labelenumii{\arabic{enumii}.}
  \tightlist
  \item
    \textbf{Centralidad del ser humano}: la economía no debe girar en
    torno al capital ni al Estado, sino a la dignidad y desarrollo
    integral de la persona.
  \item
    \textbf{Equilibrio entre cooperación y competencia}: en lugar de
    fomentar una rivalidad destructiva o una planificación rígida, se
    debe incentivar la colaboración sin anular la creatividad
    individual.
  \item
    \textbf{Sostenibilidad ecológica}: la explotación de recursos debe
    respetar los límites naturales y asegurar la preservación del
    planeta para futuras generaciones.
  \item
    \textbf{Justicia distributiva}: garantizar que los frutos del
    progreso se repartan de forma más equitativa, reduciendo
    desigualdades excesivas.
  \item
    \textbf{Innovación tecnológica orientada al bien común}: la
    automatización y digitalización deben ser instrumentos para mejorar
    la calidad de vida, no para excluir a las personas del sistema
    productivo.
  \end{enumerate}
\item
  \textbf{Estructura institucional y social propuesta}

  \begin{itemize}
  \tightlist
  \item
    \textbf{Economía mixta renovada}: integración de lo público, lo
    privado y lo comunitario, con reglas claras de corresponsabilidad.
  \item
    \textbf{Participación ciudadana}: democratización de la economía,
    con mayor peso de la sociedad civil en la toma de decisiones.
  \item
    \textbf{Empresas con propósito social}: impulsar modelos
    empresariales híbridos (como cooperativas, empresas sociales o de
    triple impacto) que prioricen tanto la rentabilidad como el
    bienestar social.
  \item
    \textbf{Nuevos indicadores de progreso}: superar el PIB como medida
    única, incluyendo índices de bienestar, sostenibilidad y desarrollo
    humano.
  \end{itemize}
\item
  \textbf{Dimensión ética y cultural}

  \begin{itemize}
  \tightlist
  \item
    El texto subraya que \textbf{ningún sistema económico puede
    sostenerse sin una base ética}.
  \item
    Se debe promover una \textbf{cultura de solidaridad, responsabilidad
    y corresponsabilidad global}, que integre valores humanos en la
    economía.
  \item
    La ética no es un adorno, sino el eje que garantiza equilibrio entre
    crecimiento, justicia social y preservación ambiental.
  \end{itemize}
\item
  \textbf{Perspectiva global}

  \begin{itemize}
  \tightlist
  \item
    El autor defiende que el nuevo sistema debe ser \textbf{global por
    naturaleza}, dado que:

    \begin{itemize}
    \tightlist
    \item
      Los problemas actuales (cambio climático, migraciones, crisis
      financieras) trascienden fronteras.
    \item
      Ningún país puede aislarse de la interdependencia mundial.
    \item
      Se propone avanzar hacia formas de \textbf{gobernanza económica
      mundial} más justas y participativas.
    \end{itemize}
  \end{itemize}
\item
  \textbf{Conclusión general}

  \begin{itemize}
  \tightlist
  \item
    El documento del profesor cierra con una \textbf{llamada a la acción
    intelectual y política}: no basta con criticar el capitalismo o el
    socialismo; es necesario diseñar, experimentar y poner en marcha un
    modelo alternativo.
  \item
    Este nuevo sistema económico debe \textbf{armonizar libertad y
    justicia, innovación y sostenibilidad, individuo y comunidad}.
  \end{itemize}
\end{enumerate}

\hypertarget{la-economuxeda-como-ciencia-aplicada}{%
\section{La economía como ciencia
aplicada}\label{la-economuxeda-como-ciencia-aplicada}}

En este ensayo se analizan los fundamentos epistemológicos y
metodológicos de la economía como ciencia aplicada. José Luis Sáez
Lozano argumenta que la labor del economista combina razonamientos
inductivos y deductivos, superando el esquema poperiano y otras
aproximaciones unilaterales. Se identifican tres tipos de economistas:
el racionalista, el aséptico y el aplicado, cada uno con enfoques
metodológicos distintos.

El autor destaca cómo la especialización ha fragmentado el estudio de la
economía en diversas ramas, todas conectadas por principios
metodológicos comunes pero con enfoques diferenciados. Esta división
refleja la complejidad de los fenómenos económicos y la necesidad de
metodologías específicas para abordarlos.

Finalmente, se subraya la importancia de reflexionar sobre los
fundamentos conceptuales y metodológicos de la economía aplicada,
consolidando su papel como una disciplina esencial para interpretar y
explicar la realidad económica.

\hypertarget{hacia-una-delimitaciuxf3n-de-la-economuxeda-aplicada}{%
\subsection{Hacia una delimitación de la economía
aplicada}\label{hacia-una-delimitaciuxf3n-de-la-economuxeda-aplicada}}

La economía aplicada se define como una disciplina que combina
fundamentos teóricos y análisis empírico para interpretar y explicar los
fenómenos económicos. Su distinción con la economía pura se remonta al
siglo XIX, pero adquiere relevancia en el siglo XX, cuando se vuelve
esencial para asesorar a Estados con vocación de intervención económica.

Autores como Senior, Mill y J. N. Keynes han contribuido a
conceptualizar la economía aplicada. Senior la describe como un conjunto
de principios prácticos que guían las acciones económicas, mientras que
Mill y Keynes destacan su carácter normativo y su integración de
elementos extraeconómicos. Por su parte, Hicks y Lange subrayan el uso
de métodos estadísticos y econométricos para validar hipótesis teóricas.

En síntesis, la economía aplicada es una rama de la ciencia económica
que, aunque vinculada a la teoría, se centra en el análisis empírico y
la resolución de problemas prácticos. Su metodología combina
razonamientos inductivos y deductivos, consolidando su papel como un
puente entre la teoría y la realidad económica.

\hypertarget{aspectos-metodoluxf3gicos-de-la-economuxeda-aplicada}{%
\subsection{Aspectos metodológicos de la economía
aplicada}\label{aspectos-metodoluxf3gicos-de-la-economuxeda-aplicada}}

Tal y como se expone en la introducción, la economía aplicada comparte
elementos metodológicos con la ciencia económica en general, aunque no
existe un consenso absoluto sobre la unicidad de su método científico.
Existen posturas divergentes, como el monismo metodológico, que
establece requisitos específicos para las ciencias sociales, y el
individualismo metodológico, que analiza fenómenos económicos en
términos de individuos y sus decisiones.

Se identifican tres tipos de economistas según su enfoque metodológico:

\begin{enumerate}
\def\labelenumi{\arabic{enumi}.}
\item
  \textbf{El economista racionalista}: Utiliza un razonamiento deductivo
  para formular teorías que explican el sistema económico. Contrasta
  predicciones con la realidad y corrige errores mediante acción
  colectiva. Ejemplos incluyen el modelo neoclásico de equilibrio
  general y las teorías de la economía socialista.
\item
  \textbf{El economista aséptico}: Basa su análisis en la lógica
  inductiva, identificando regularidades estadísticas en la realidad
  económica. Este enfoque incluye la escuela histórica alemana y el
  institucionalismo, que enfatizan el carácter mutable de las
  instituciones y su impacto en el comportamiento económico.
\item
  \textbf{El economista aplicado}: Combina el deductivismo y el
  inductivismo para explicar fenómenos económicos. Su labor incluye la
  construcción de modelos teóricos y su validación empírica. Este
  enfoque se consolidó con las contribuciones de Keynes, Clark, Leontief
  y otros, quienes desarrollaron herramientas como las cuentas
  nacionales y el análisis input-output.
\end{enumerate}

Finalmente, se destacan dos principios fundamentales: la necesidad de
una teoría económica no excesivamente abstracta y el uso racional de las
matemáticas. Estos principios buscan evitar distorsiones en las
conclusiones y garantizar que los modelos sean útiles para interpretar
la realidad económica. Además, se subraya la importancia de la
experimentación como herramienta complementaria en la investigación
económica aplicada.

\hypertarget{a-modo-de-epuxedlogo-el-economista-aplicado-ante-todo-economista}{%
\subsection{A modo de epílogo: El economista aplicado, ante todo,
economista}\label{a-modo-de-epuxedlogo-el-economista-aplicado-ante-todo-economista}}

Tras el crecimiento de la ciencia económica, la especialización ha
fragmentado el campo en diversas ramas conectadas por principios
epistemológicos comunes, pero con metodologías distintas. La distinción
entre economía pura y aplicada surge en el siglo XIX, consolidándose en
el siglo XX. Senior, Mill y J. N. Keynes aportaron visiones sobre la
economía aplicada, desde su carácter práctico hasta su integración de
elementos normativos y extraeconómicos.

Schumpeter identificó los campos de la economía derivados de esta
especialización, mientras que Robbins los clasificó en teoría económica,
historia económica y economía descriptiva. Actualmente, la economía
aplicada se centra en la validación empírica de hipótesis mediante
métodos estadísticos y econométricos.

Metodológicamente, se distinguen tres tipos de economistas: el
racionalista, que utiliza razonamientos deductivos; el aséptico, que
emplea lógica inductiva; y el aplicado, que combina ambos enfoques para
interpretar la realidad económica. Este último integra teoría y análisis
empírico, superando esquemas unilaterales como el poperiano.

En conclusión, la economía aplicada es esencial para conectar teoría y
realidad, evitando abstracciones excesivas y el uso indiscriminado de
matemáticas, mientras se reivindica su papel como disciplina científica
integral.

\hypertarget{ideas-principales}{%
\subsection{Ideas principales}\label{ideas-principales}}

\begin{enumerate}
\def\labelenumi{\arabic{enumi}.}
\tightlist
\item
  \textbf{Introducción}

  \begin{itemize}
  \tightlist
  \item
    La economía aplicada combina \textbf{inducción y deducción},
    superando esquemas metodológicos clásicos como el falsacionismo de
    \textbf{Popper}, el análisis causal de \textbf{Akerman} y la lógica
    inductiva de \textbf{J. S. Mill}.
  \end{itemize}
\item
  \textbf{Economía aplicada: delimitación y evolución histórica}

  \begin{itemize}
  \tightlist
  \item
    Surge en el siglo XIX como distinción de la economía pura,
    consolidándose en el XX.
  \item
    Contribuciones clave:

    \begin{itemize}
    \tightlist
    \item
      \textbf{Senior}: principios prácticos que guían la acción humana.
    \item
      \textbf{Mill}: integración de fundamentos teóricos y elementos
      extraeconómicos.
    \item
      \textbf{J. N. Keynes}: economía descriptiva y narrativa.
    \item
      \textbf{Schumpeter} y \textbf{Robbins}: clasificación de campos
      aplicados y ramas de la economía.
    \end{itemize}
  \end{itemize}
\item
  \textbf{Fundamentos metodológicos}

  \begin{itemize}
  \tightlist
  \item
    Tres tipos de economistas:

    \begin{enumerate}
    \def\labelenumii{\arabic{enumii}.}
    \tightlist
    \item
      \textbf{Racionalista}: enfoque deductivo, teorías generales.
    \item
      \textbf{Aséptico}: análisis inductivo, basado en datos empíricos.
    \item
      \textbf{Aplicado}: síntesis deductivo-inductiva, modelos validados
      empíricamente.
    \end{enumerate}
  \end{itemize}
\item
  \textbf{Núcleo de consenso metodológico}

  \begin{itemize}
  \tightlist
  \item
    Teoría no excesivamente abstracta.
  \item
    Uso prudente de matemáticas.
  \item
    Relación indisoluble entre teoría y realidad.
  \end{itemize}
\item
  \textbf{Economía aplicada y experimentación}

  \begin{itemize}
  \tightlist
  \item
    Complementa estadística y econometría con experimentación
    controlada.
  \item
    Ventajas: replicabilidad y testeo de hipótesis.
  \item
    Limitaciones: artificialidad y alejamiento de decisiones reales.
  \end{itemize}
\item
  \textbf{Epílogo}

  \begin{itemize}
  \tightlist
  \item
    La economía aplicada es esencial para conectar teoría y realidad.
  \item
    Requiere metodologías híbridas y teorías adaptadas a contextos
    concretos.
  \item
    El economista aplicado combina abstracción, observación y acción
    práctica.
  \end{itemize}
\end{enumerate}

\emph{Resumen}: La economía aplicada es el puente entre teoría y
realidad, utilizando una metodología híbrida para validar hipótesis,
orientar políticas y comprender el sistema económico. Es inseparable de
la teoría, avanzando mediante el diálogo entre abstracción, datos
empíricos y experimentación.

\begin{thebibliography}{99}

  \bibitem{Referencia1}
  Ismael Sallami Moreno, \textbf{Estudiante del Doble Grado en Ingeniería Informática + ADE}, Universidad de Granada, 2025.
  
  \bibitem{DiapositivasAsignatura}
  Universidad de Granada, \emph{Diapositivas de la asignatura}, Curso 2025/2026.

  % \bibitem{Referencia2}
  % Autor Apellido, \emph{Título del libro o artículo}, Editorial o Revista, Año.
  
  % \bibitem{Referencia3}
  % Nombre Autor, \emph{Título del documento}, Conferencia/URL, Año.
  
  \end{thebibliography}
  

\end{document}
